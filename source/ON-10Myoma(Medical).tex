\section{자궁근종:Medical treatment}
\subsection{이니시아, INISIA[Uripristal acetate]}
\myde{}{%
\begin{enumerate}\tightlist
\item 주상병
	\begin{itemize}\tightlist
	\item D25 자궁내 평활근종
​	\item D250 자궁의 점막하 평활근종
	\item D251 자궁의 벽내 평활근종
	\item D252 자궁의 장막하 평활근종
	\item D259  상세불명의 자궁의 평활근종
	\end{itemize}
\item 보조상병
	\begin{itemize}\tightlist
	\item 출혈관련
		\begin{enumerate}\tightlist
		\item N921 불규칙적 주기를 가진 과다 및 빈발월경
		\item N938  기타 명시된 이상 자궁 및 질 출혈
		\end{enumerate}	
	\item 통증관련
		\begin{enumerate}\tightlist
		\item N940 중간통
		\item R103 기타 하복부에 국한된 동통
		\item R520 급성통증
		\item N945 속발성 월경통
		\end{enumerate}	
	\item 불임
		\begin{enumerate}\tightlist
		\item N978 기타 요인에서 기원한 여성불임
		\end{enumerate}
	\end{itemize}
\end{enumerate}
청구메모>>\par
가임기 여성(18세 - 폐경전)으로 자궁근종의 크기 감소와 증상개선을 목적으로  수술전 약물 투여함\\
}%
{
\emph{투여방법}
\begin{enumerate}\tightlist
\item 2014년 6월25일 식약청 의약품 수입품목 허가사항 변경
\item 1일 1회 1정[5mg]을 3개월까지 연속 경구 투여
\item 투약은 월경시작 1주 이내에 시작
\item 3개월연속 투여는 한번더 반복 될 수 있슴
\item 첫 3개월 연속 투여 종료 후 첫번째 월경을 온전히 지내고 두번째 월경이 시작되면 일주일 이내에 재투여
\item 2cycle 이상은 안정성 확보 안됨
\end{enumerate}
\emph{청구시 주의사항}
\textcolor{red}{자궁근종 상병1 + 보조상병중 출혈관련1+ 보조상병중 통증관련1+ 청구메모 }
 기입을 해야만 \emph{삭감이 없습니다. 꼭 유념..}\par
\begin{itemize}\tightlist
\item Progesterone receptor modulator
\item Oral agents
\item Comparable efficacy : No serious AE
\item Increasingly used 
\item Major concern is endometrial effects with long term use : 1YR Tx is safe, More non-physiologic change, No sign of clinical concern, Non-physiologic change disappear at 6M
\end{itemize} 
}%
\subsection{자궁근종'이니시아'투여 '삭감'사례는?}
심평원, 4/4분기 요양급여비용 심사사례 공개 \par
자궁내장치술 실시 환자에 이니시아정 투여해도 '삭감'
자궁근종으로 인한 증상이 불명확하고, 빈혈소견이 없음에도 '이니시아정'을 투여했다면 급여가 '삭감'된다. 자궁내장치술을 실시한 환자에게 이니시아정을 투여해도 삭감된다.

건강보험심사평가원은 공개심의위원회 심의를 거쳐 4/4분기 요양급여비용 심사사례를 26일 홈페이지를 통해 공개했다.

심사사례를 보면, 53세 여성은 '상세불명의 자궁의 평활근종'으로 진단 받고, 이니시아정을 투여받았다.

그러나 심평원이 확인한 결과, 진료내역에는 '생리통 +++, 생리량 +++'만 기록돼 있으며, 자궁근종으로 인한 증상이 불명확했다. 월경과다로 인한 빈혈소견 등 근종 치료 사유가 확인되지 않았다. 복부CT에서도 R/O 자궁선근종으로 확인되므로 이니시아정은 인정하지 않았다.

이니시아정은 식품의약품안전처 허가사항 중 '가임기 성인 여성에서 중등도-중증 증상을 가진 자궁근종 환자의 수술전 치료'에 효능·효과 있다. 이런 효능으로 진료경과기록 상 '참기 힘든 생리통'과 '생리량 과다'등의 증상과 함께 중등증의 빈혈소견이 확인될 때 이니시아정을 투여한다면, 수술전 빈혈치료 목적으로 투여한 경우로 급여가 인정된다.

47세 여성은 '상세불명의 자궁의 평활근종'상병으로 자궁내장치술인 '미레나'를 삽입 했으며, 4주 후에 이니시아정을 투여받았다. 그러나 이 환자에 대한 이니시아정은 급여로 인정되지 않았다.

미레나는 월경과다증·월경곤란증·에스트로겐 대체요법시 스로게스틴의 국소 적용시 치료목적으로 자궁 내 삽입하는 약제이다. 이 환자는 미레나를 통해 생리량이 줄었음에도 불구하고 동일 목적의 약제 이니시아정을 병용 투여한 경우이므로 이니시아정은 인정하지 않았다.
\subsection{GnRH agonist}
\subsection{미레나}