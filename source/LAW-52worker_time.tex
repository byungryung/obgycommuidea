\section{직원 근무시간}
\subsection{근로시간}
\begin{mdframed}[linecolor=blue,middlelinewidth=2]
근무시간이란 직원이 사용자의 지휘・감독 하에서 근로를 제공하는 시간을 의미하며 휴게시간은 근무시간에 미포함(단, 작업을 위한 지휘.감독아래 대기시간은 포함
\end{mdframed}

\tabulinesep =_2mm^2mm
\begin {tabu} to\linewidth {|X[1,c,m]|X[3.5,l]|} \tabucline[.5pt]{-}
\rowcolor{ForestGreen!40} \centering 구분 & \centering 주요내용 \\ \tabucline[.5pt]{-}
\rowcolor{Yellow!40} 법정근로시간 & - 근기법에서 정한 기준근로시간 \newline
- 1일 8시간, 1주 40시간 (휴게시간 제외) \\ \tabucline[.5pt]{-}
\rowcolor{Yellow!40} 소정근로시간 & - 법정근로시간 내에서 노사가 근로하기로 정한 시간 \newline
- 시간급 통상임금 산정을 위한 기초) \\ \tabucline[.5pt]{-}
\rowcolor{Yellow!40} 연장근로 & - 법정근로시간을 초과하는 근로 \newline
- 당사자간 합의에 의해 1주간 12시간 한도 \newline - 1일은 통상 0~24시, 24시를 지나 이틀에 걸쳐 계속 근로하더라도 전날의 근로로 인정  \\ \tabucline[.5pt]{-}
\rowcolor{Yellow!40} 야간근로 & - 22시부터 06시 사이의 근로 \\ \tabucline[.5pt]{-}
\rowcolor{Yellow!40} 휴일근로 & - 법령, 근로계약, 취업규칙 등에서 유․무급을 불문하고 휴일로 정한 날의 근로 \newline
-  연장근로시간 한도를 계산할 때는 미 합산 \newline
- 단, 휴일근로시간이 8시간을 초과하는 경우 초과한 근로시간은 연장근로에 합산 
 \\ \tabucline[.5pt]{-}
\end{tabu}

\begin{itemize}[□]\tightlist 
\item 통상임금의 100분의  50  이상을 가산
\item 연장․야간․휴일이 중복되는 경우 가산수당을 중복하여 지급
\item 근로자대표와의 서면 합의에 의해 임금지급을 갈음한 휴가 부여 가능
(보상휴가제)
\end{itemize}
\subsection*{근로시간변경안}
\includegraphics[scale=.65]{paychange}

\subsection{연차휴가}
\tabulinesep =_2mm^2mm
\begin {tabu} to\linewidth {|X[1,l]|X[2,l]|X[2,l]|X[4,l]|X[2,l]|X[2,l]|} \tabucline[.5pt]{-}
\rowcolor{ForestGreen!40} \centering 구분 & \centering 부여조건 & 연차일수 & 가산일수 & 사용시기 & 미사용시 \\ \tabucline[.5pt]{-}
\rowcolor{Yellow!40} 연차 휴가 & 1년간 소정근로  일수를 8할 이상 근무 & 15일 & 3년 이상 계속하여 근로한 자에게는  최초 1년을 초과하는 계속근로년수  매2년마다 1일 가산(최대25일 한도) & 연차휴가 청구권  발생일로부터  1년간 & 미사용일수   만큼 수당 지급 \\ \tabucline[.5pt]{-}
\end{tabu}

\emph{계속 근로연수가 1년 미만인 직원}
\begin{itemize}[□]\tightlist 
\item 1개월간 개근시마다 1일의 연차휴가 발생
\item 1년 미만 근속기간 중 1개월 개근으로 발생한 연차휴가를 사용하지 못하고 1년 미만 근속기간 중
    퇴직하는 경우 미사용한휴가에 대해 연차유급휴가 미사용수당(이하 ‘연차수당) 지급
\item 최초 1년간의 근로에 대해 연차휴가를 부여하는 경우 1개월간 개근으로 발생한 휴가를 
    포함하여 10일 또는 15일로 하고 근로자가 휴가를 이미 사용한 경우에는 그 사용한 휴가 
    일수를 10일 또는 15일에서 뺀다.
\end{itemize}
\includegraphics[scale=.65]{Tyear}