\section{계류유산(Missed abortion)}
\myde{}{\begin{itemize}\tightlist
\item[\dsjuridical] O021 Missed abortion 
\item[\dsmedical] R4441 계류유산소파술- 12주 미만 [\myexplfn{943.34} 원]
\item[\dsmedical] R4442 계류유산소파술- 12주 미상 [\myexplfn{1281.48} 원]
\item[\dsmedical] EB511010 진단 초음파 (급여)
\item[\dsmedical] 시술시 유도(중재적) 초음파 (비급여)
\item[\dsmedical] 유착방지제 (선별급여 80/100) 
\item[\dsmedical] 포폴(Propofol) (100대100)
\item[\dsmedical] L0101	정맥마취(전신마취)
\item[\dsmedical] 영양제 (비급여)
\item[\dsmedical] KK059	정맥내유치침
\item[\dschemical] 소수술시 인정되는 Lab.
\item[\dschemical] C5915 병리조직검사[1장기당]-생검(13개이상)-내막조직검사
\end{itemize}}
{\begin{enumerate}\tightlist
\item 선별급여는 변형된 100/100 정책이다. 100/100 정책은 비급여 항목 가운데 일부를 `전액 본인이 부담하는 급여’로 지정하는 요상한 정책으로, 비급여 의료행위의 가격을 정부가 통제하는 수단이었다. 급여 필요성이 있지만 당장 급여화할 수 없는 행위를 급한 대로 100/100으로 지정했다가 추후 급여화한다는 것이 명분이었지만 실제로 그런 수순을 밟은 행위는 전무했고, 결국 유명무실해진 바 있다.\par
선별급여는 전액이 아니라 80\% 혹은 50\%만을 본인이 부담한다는 것만 다를 뿐, 기본적인 개념은 똑같다. 대상은 `의학적 필요성이 낮으나 환자 부담이 높은 고가의료, 임상근거 부족으로 비용효과 검증이 어려운 최신 의료기술’이다.\par
\item 포폴 투여시 요양급여인정 기준 : 아래와 같은 기준으로 투여시 요양급여를 인정하며, 허가사항 범위이지만 동 인정기준 이외에 투여한 경우에는 약값 전액을 환자가 부담토록 함(100대 100)
	\begin{itemize}[o]\tightlist
	\item 30분 초과 2시간 이내의 마취를 요하는 수술
	\item 뇌질환, 심장질환, 신장질환, 장기이식 시술환자, 간기능 이상환자, 간질환의 기왕력이 있는 환자에게 마취유도 및 유지목적으로 사용한 경우(단, 마취유지시 최초 10분간은 10mg/kg/hr, 추가 10분간은 8mg/kg/hr, 그 이후는 6mg/kg/hr 용량의 범위내에서 투여시)
	\item 마취유도 목적으로 150mg/15ml/Amp 1개 투여시
	\item 개심술의 마취시 구연산펜타닐을 주마취제로, 프로포폴을 보조마취제로 병용투여하는 경우에는 프로포폴제제를 4mg/kg/hr 이내로 인정
	\end{itemize}
\item 영양제 선택시 고려사항 :  1. 비급여 이면서 2. 식약처 허가사항에 수술전후라는 것으로 되어 있는 영양제로 선택
\item 정맥내주사로 확보(Keep Vein Open)시 진료수가는 어떻게 산정을 하나요? 환자 치료상 수액제 주입없이 일정기간 동안 정맥내유치침으로 정맥내주사로를 확보(Keep Vein Open)하고 하루에 수회의 약물을 투여하는 경우의 진료수가 산정방법을 다음과 같이 함.
\end{enumerate}}
\prezi{\clearpage}

\subsection{타병원서 계류유산진단후 수술하러온 경우 임신초기검사 급여가능}
\begin{itemize}\tightlist
\item 급여 가능합니다. 
\end{itemize}
\prezi{\clearpage}
\subsection{계류유산시 본인부담 경감적용기간}
\begin{commentbox}{「의료급여수가의 기준 및 일반기준」개정}
\begin{enumerate}[1)]\tightlist
\item 의료급여법 시행령(대통령령 제27730호, ‘16.12.30일 공포, ’17.1.1일 시행)
\item 의료급여법 시행규칙(보건복지부령 제459호, ‘16.12.30일 공포, ’17.1.1일 시행 )
\item 의료급여수가의 기준 및 일반기준(고시 제2016-272호, ‘17.1.1일 시행)
\item 요양비의 의료급여기준 및 방법(고시 제2016-245호, ‘17.1.1일 시행)
\item 임신\cntrdot{}출산 진료비등의 의료급여기준 및 방법(고시 제2016-246호, ‘17.1.1일 시행)
\end{enumerate}

의료급여수가의 기준 및 일반기준 중 일부를 다음과 같이 개정한다.
\begin{description}\tightlist
\item[제4조]를 삭제한다.
\item[제5조제1항제1호] 중 “긴급수술을 요하는 경우”를 “분만 및 수술을 동반하는 경우”로 한다. 
\item[제5조제1항제2호] 를 삭제하고, 제3호부터 제5호까지를 각각 제2호부터 제4호까지로 한다.
\item[제17조의8]을 다음과 같이 신설한다.
\end{description}
\emph{제17조의8(임신부, 조산아 및 저체중 출생아에 대한 의료급여)}
\begin{enumerate}[①]\tightlist
\item 영 제13조제1항 별표 1 제2호 자목의 \textcolor{red}{“임신부”란 임신이 확인된 이후 임신이 유지되는 기간에 있는 사람(유산\cntrdot{}사산으로 인한 외래진료를 받는 사람을 포함한다)을 말하며}, 차목의 “만3세까지의 조산아(早産兒) 및 저체중 출생아”에 대한 의료급여의 대상과 기간은「요양급여의 적용기준 및 방법에 관한 세부사항」에서 정하는 바에 따른다.
\end{enumerate}
\end{commentbox}
\prezi{\clearpage} 
\Que{질문)임신부 외래 진료시 본인부담 경감적용 기간은 ?} 
\Ans{답변)임신이 확인된 이후 임신이 유지되는 기간에 있는 사람과 유산 , 사산으로 인한 시술 ( 또는 수술 ) 을 위한 외래진료 당일을 포함함. 유산되고 수술후 다음날 치료시는 본인부담 경감적용 기간이 아닙니다}
\prezi{\clearpage}
\subsection{계류유산 시술시 보험가능 검사들}
\Que{계류유산 수술당일 수술전검사에 어떤 항목이 급여 적용가능한가요?}
\Ans{산전검사 항목외에 수술전 검사 와 계류유산 원인검사(갑상선 검사등)와 계류유산 합병증 검사 (DIC등)가 보험적용가능하며 60개항목 이상입니다.} 
\prezi{\clearpage}
\subsection{계류유산 재수술시 산정방법}
\Que{갑의원에서 유산수술을 하였으나 자궁내부속물의 제거를 완전히 하지 못하여 ‘을’ 의원에서 자궁소파술 시행시 수기료 산정방법}
\Ans{갑’의원에서 유산수술을 하였으나 자궁내부속물의 제거를 완전히 하지 못하여 ‘을’ 의원에서 자궁소파술 시행한 경우에 \textcolor{red}{‘갑’과‘을’ 요양기관의 시술행위료는 각각 산정}함\par
(고시 제2000-73호, 2001.11 시행)\par 
자궁소파수술 자452 R4521(53900원)}


