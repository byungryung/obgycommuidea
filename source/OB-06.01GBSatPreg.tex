\section{임산부 GBS culture}
\myde{}{%

\begin{itemize}\tightlist
\item[\dsjuridical] N760-N768 질염
\item[\dsjuridical] N730-N739 골반염
\item[\dsjuridical] N710,N711,N719 자궁염
\item[\dsjuridical] A638 기타 명시된 주로 성행위로 전파되는 질환
\item[\dsjuridical] 임신 제 2분기 이상에서 조산의 위험 증상 (조기 양막파수O42 , 조기 진통 O60 등)이 있는 경우
\item[\dschemical] \sout{B4051 미생물배양및동정검사 \myexplfn{168.28}} 
\item[\dschemical] B4133 비뇨기, 생식기검체 Urogenital Specimens \myexplfn{154.61} 원( Microorganism Culture, Identification and Disk Diffusion Sensitivity Analysis)
\end{itemize}
심사참고 “질내 세균 감염 의심되어 검사 시행함”\par
2017 년 9 월부터는 균자라든 안자라든 나-407 미생물 ( 항산균제외 ) 배양 , 동정 , 약제감수성검사 Microrganism(except AFB) Culture, Identification and Antibiotics Sensitivity Test 적용.\newline
\sout{B4061은 B4051에서 균이 검출되었을때 검사하세요. 즉 약제 감수성 검사는 같은날 하시면 삭감되시고 검사결과 나올때 청구하십시요. 진찰료 표시를 하지 않고 검사만 청구해야 합니다. [환자가 방문하지 않았으므로]}
}
{
선별검사를 산모가 원하는경우는 비급여입니다(비급여고지후 적정금액 받으시면 됩니다)
임상적으로 필요한 경우 분만전 GBS 균 검사는 GBS 검사 보험인정 기준은 없습니다.}
\prezi{\clearpage}
\subsection{GBS 치료및 advise}
\href{http://www.obgydoctor.co.kr/xe/index.php?document_srl=14331&mid=m_faq}{산부인과희망제작소 home}\par
일단 gbs  양성인경우 200명중 한명  감염 위험있고 항생제 사용하면 4000명중 1명 감염된다고 말씀하시고 신생아는 열이 나면 뇌수막염 진단위해 spinal  tapping해서 검사가 필요한데 로칼은 힘드니 대학보내서 검사해야하고 증상없으면 아무런 조치도 필요치는 않습니다.
