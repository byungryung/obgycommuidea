\section{상급병실료 개편 관련 외박수가 산정}
\Que{상급병실료 개편으로 4인실 AB357, 5인실 AB337 수가를 산정할 경우 환자가 외박을 한다면 수가 산정을 어떻게 해야 하나요?}
\Ans{2014년 9월 1일부터 상급병실제도 개편과 함께 4, 5인실 입원료가 건강보험 급여적용토록 개정되어, 기존의 ‘입원료’가 ‘기본입원료’로 변경되었고, ‘4인실 입원료, 5인실입원료’가 신설되었습니다. 신설된 4인실입원료, 5인실입원료에도 각각의 입원료 소정점수가 부여되었고, 입원료 등에 포함되어 있는 의학관리료(입원료 소정점수의 40\%), 간호관리료(소정점수의 25\%), 병원관리료(소정점수의 35\%)의 비율은 기존과 동일하므로, 외박 등 적용 시 각각의 해당 입원료 소정점수에 대한 비율을 적용하시기 바랍니다. (예: 5인실 입원 중 외박시, 5인실입원료의 35\% 적용)}

\begin{commentbox}{「건강보험 행위 급여·비급여 목록표 및 급여 상대가치점수」
제1편 제2부 제1장 기본진료료 [산정지침] 2. 입원료 등}
\begin{enumerate}[가.]\tightlist
\item 입원료 등의 소정점수에는 입원환자 의학관리료(소정점수의 40\%), 입원환자 간호관리료(소정점수의 25\%), 입원환자 병원관리료(소정점수의 35\%)가 포함되어 있으며 요양기관 종별에 따라 산정한다.
\end{enumerate}
\end{commentbox}

\begin{commentbox}{입원환자 외박 시 병원관리료 산정방법}
\begin{itemize}\tightlist
\item 입원중인 환자가 주치의의 허가를 받아 외박시 입원료는 산정할 수 있으나, 연속하여 24시간을 초과하는 경우는 입원료 중 입원환자 병원관리료만 산정함
\item 이 때, 병원관리료는 내과질환자·정신질환자·만8세 미만의 소아환자에 대한 가산, 간호인력 확보 수준에 따른 입원환자 간호관리료 차등가산 및 입원일수에 따른 체감이 적용되지 않은 상태에서 입원료 소정점수의 35\%를 산정함
 고시 제2003
\end{itemize}
\end{commentbox}

