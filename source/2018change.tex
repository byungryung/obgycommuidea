\section{2018 CHANGE}
\subsection{검체, 조직검사 개편(코드 변경)}
<보건복지부 고시 제2017 - 222호> 자료입니다\par
\url{http://obgy.org/board/read.php?bid=9&pid=1261952&} \par
\highlightY{일단 청구코드 자동변경(맵핑)하고, 신규처방코드 확인을 해야합니다!!(신규처방코드 맵핑이 안되어있는경우는 새코드작업필요)}

\subsection{「암검진실시기준 일부개정」}
\url{http://www.obgy.org/board/read.php?bid=9&pid=1315263}

\emph{주요내용}
\begin{enumerate}[다.]\tightlist
\item 건강보험 상대가치점수 개편에 따른 암검진 수가 변경(안 별표1)
	\begin{itemize}\tightlist
	\item 유방암검진시 맘모그라피 촬영 수가 청구방법 개선(일괄 4매 청구 → 편측 2매씩 분리하여 청구)
	\end{itemize}
\item 기타 암검진 결과 판정기준 및 결과통보서 등 개선(안 별지 서식)
	\begin{itemize}\tightlist
	\item 암검진 질관리 강화를 위해 검진결과 판정의사 실명제 도입(결과기록지에 판정의사의 의사면허번호 및 성명 기입)
	\item 기타, 위・대장암 2차검사시 내시경검사를 원칙으로 하도록 우선순위 조정, 결과 이상소견 및 의심증상 등 분류기준 변경 등
	\end{itemize}
\end{enumerate}
\subsection{포폴등 약제의 요양급여기준} 
홈페이지 자료입니다: \url{http://www.obgy.org/board/read.php?bid=9&pid=1340829} \par
(포폴 급여기준이 30분에서 2시간----> 30분에서 3시간 으로 변경 되었습니다

\subsection{65 세 이상 의원급 외래진료 등의 본인부담액 관련}
\url{http://www.obgy.org/board/read.php?bid=9&pid=598400&}

\paragraph{주요개정내용}
\begin{itemize}[○]\tightlist
\item 국민건강보험법 시행규칙 별표 3 제 1 호 개정 예정
	\begin{itemize}\tightlist
	\item 65 세 이상 노인이 의원, 치과의원, 한의원 및 보건의료원에서 외래 요양 급여를 받는 경우 본인부담액
	\end{itemize}
\end{itemize}


\subsection{자궁내장치 제거료 급여기준. ​공난포 채취시 요양급여비용 산정방법}
홈페이지 자료입니다(2018년 1월부터 시행)\par
\url{http://www.obgy.org/board/read.php?bid=9&pid=1367838}
\begin{commentbox}{LOOP제거급여기준}
\paragraph{자궁내장치 제거료 급여기준}\par
피임시술 요양급여 대상자 또는 본인이 원하여 자궁내장치삽입술을 시술받은 대상자가 다음과 같은 사유로 제거시 자궁내장치제거료는 요양급여 대상임.\par

\emph{- 다 음 -}
\begin{enumerate}[가.]\tightlist
\item 지속적인 자궁\cntrdot{}질 출혈, 과다월경
\item 골반통, 복통, 월경통
\item 골반염, 자궁\cntrdot{}질염
\item 자궁천공
\item 임신
\item 암 : 자궁체부, 자궁경부, 유방암 ( 레보노르게스트렐 자궁내장치 )
\item  상기 가 .- 바 . 이외 의학적 치료가 필요하여 제거한 사유를 제시한 경우
\end{enumerate}
(2018년 1월 1일)
\end{commentbox}

