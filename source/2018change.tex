\section{2018 CHANGE}
\subsection{검체, 조직검사 개편(코드 변경)}
<보건복지부 고시 제2017 - 222호> 자료입니다\par
\url{http://obgy.org/board/read.php?bid=9&pid=1261952&} \par
\highlightY{일단 청구코드 자동변경(맵핑)하고, 신규처방코드 확인을 해야합니다!!(신규처방코드 맵핑이 안되어있는경우는 새코드작업필요)}

\subsection{「암검진실시기준 일부개정」}
\url{http://www.obgy.org/board/read.php?bid=9&pid=1315263}

\emph{주요내용}
\begin{enumerate}[다.]\tightlist
\item 건강보험 상대가치점수 개편에 따른 암검진 수가 변경(안 별표1)
	\begin{itemize}\tightlist
	\item 유방암검진시 맘모그라피 촬영 수가 청구방법 개선(일괄 4매 청구 → 편측 2매씩 분리하여 청구)
	\end{itemize}
\item 기타 암검진 결과 판정기준 및 결과통보서 등 개선(안 별지 서식)
	\begin{itemize}\tightlist
	\item 암검진 질관리 강화를 위해 검진결과 판정의사 실명제 도입(결과기록지에 판정의사의 의사면허번호 및 성명 기입)
	\item 기타, 위・대장암 2차검사시 내시경검사를 원칙으로 하도록 우선순위 조정, 결과 이상소견 및 의심증상 등 분류기준 변경 등
	\end{itemize}
\end{enumerate}
\subsection{포폴등 약제의 요양급여기준} 
홈페이지 자료입니다: \url{http://www.obgy.org/board/read.php?bid=9&pid=1340829} \par
(포폴 급여기준이 30분에서 2시간----> 30분에서 3시간 으로 변경 되었습니다

\subsection{65 세 이상 의원급 외래진료 등의 본인부담액 관련}
\url{http://www.obgy.org/board/read.php?bid=9&pid=598400&}

\paragraph{주요개정내용}
\begin{itemize}[○]\tightlist
\item 국민건강보험법 시행규칙 별표 3 제 1 호 개정 예정
	\begin{itemize}\tightlist
	\item 65 세 이상 노인이 의원, 치과의원, 한의원 및 보건의료원에서 외래 요양 급여를 받는 경우 본인부담액
	\end{itemize}
\end{itemize}


\subsection{자궁내장치 제거료 급여기준. ​공난포 채취시 요양급여비용 산정방법}
홈페이지 자료입니다(2018년 1월부터 시행)\par
\url{http://www.obgy.org/board/read.php?bid=9&pid=1367838}
\begin{commentbox}{LOOP제거급여기준}
\paragraph{자궁내장치 제거료 급여기준}\par
피임시술 요양급여 대상자 또는 본인이 원하여 자궁내장치삽입술을 시술받은 대상자가 다음과 같은 사유로 제거시 자궁내장치제거료는 요양급여 대상임.\par

\emph{- 다 음 -}
\begin{enumerate}[가.]\tightlist
\item 지속적인 자궁\cntrdot{}질 출혈, 과다월경
\item 골반통, 복통, 월경통
\item 골반염, 자궁\cntrdot{}질염
\item 자궁천공
\item 임신
\item 암 : 자궁체부, 자궁경부, 유방암 ( 레보노르게스트렐 자궁내장치 )
\item  상기 가 .- 바 . 이외 의학적 치료가 필요하여 제거한 사유를 제시한 경우
\end{enumerate}
(2018년 1월 1일)
\end{commentbox}

\subsection{1회용 수술(시술)팩의 급여기준}
1회용 수술(시술)팩은 수술 부위를 오염 등으로부터 보호하기 위해 사용하는 환자용, 의료진용, 수술기구용, 기타 구성품으로 구성된 패키지로 다음의 경우에 요양급여를 인정하고 「치료재료 급여․비급여 목록 및 급여상한금액표」의 해당 치료재료비용을 산정함. 또한, 적응증 이외의 경우에는 소정 행위료에 포함되어 별도 산정하지 아니함.\par
\emph{- 다   음 -}
\begin{enumerate}[가.]\tightlist
\item 적응증
	\begin{enumerate}[1)]\tightlist
	\item 바1, 바2 마취에 의한 수술
	\item 중재적 방사선시술
	\item ECMO(체외순환막형산화요법, Extra Corporeal Membrane Oxygenation)를 위해 Cannula를 삽입하는 시술
	\item 중심정맥관 삽입술
	\item \highlightY{자연분만}
	\end{enumerate}
\item 인정개수 : 수술(시술) 당 1개 인정
   단, 협의 진료로 2가지 이상 수술(시술)을 동시에 시행하여 수술(시술)팩을 추가로 사용하는 경우 별도 인정
\item 산정방법
	\begin{enumerate}[1)]\tightlist
	\item 적응증 ‘가. 1)’의 경우 ‘CABG 수술팩’, ‘Shoulder, Knee, Hip 관절치환 수술팩’, ‘눈 수술팩’, 마취시간별 수술팩(Ⅰ)-(Ⅳ) 중 해당 수술팩 치료재료비용을 산정
	\item 적응증 ‘가. 2)-4)’의 경우 ‘중재적 방사선 시술팩’, ‘ECMO 시술팩’, ‘중심정맥관 삽입 시술팩’ 중 해당 시술팩 치료재료비용을 산정
	\item 적응증 ‘가. \highlightY{5)’의 경우 ‘수술팩(Ⅰ)(마취시간 1시간이하)’ 치료재료비용을 산정}
	\end{enumerate}
\item 다만, 1회용 수술(시술)팩을 사용하지 않고 \highlight{린넨팩을 사용하는 경우 「건강보험 행위 급여․비급여 목록표 및 급여 상대가치점수」의 해당 린넨팩 관리료(자-0)를 별도 산정할 수 있으며, 이는 2018년 12월 31일까지 한시적으로 적용함.}
\end{enumerate}
\leftrod{급여 전환되는 1회용 수술팩 종류}\par 
\tabulinesep =_2mm^2mm
\begin{tabu} to\linewidth {|X[1,l]|X[2,l]|X[1,l]|X[1,l]|} \tabucline[.5pt]{-}
\rowcolor{ForestGreen!40}  코드  & 품명 & 수입(판매)업소 & 상한금액(V.A.T 포함) \\ \tabucline[.5pt]{-}
\rowcolor{Yellow!40} N0101603 & 수술팩(Ⅰ)(마취시간 1시간이하) & ㈜대명화학 & 35,970 \\ \tabucline[.5pt]{-}
\rowcolor{Yellow!40} N0101604 & 수술팩(Ⅰ)(마취시간 1시간이하) & ㈜앤티아이 & 35,970 \\ \tabucline[.5pt]{-}
\rowcolor{Yellow!40} N0101606 & 수술팩(Ⅰ)(마취시간 1시간이하) & ㈜멀티게이트코리아 & 35,970 \\ \tabucline[.5pt]{-}
\rowcolor{Yellow!40} N0101607 & 수술팩(Ⅰ)(마취시간 1시간이하) & ㈜메디웍스 & 35,970 \\ \tabucline[.5pt]{-}
\rowcolor{Yellow!40} N0101608 & 수술팩(Ⅰ)(마취시간 1시간이하) & ㈜서화 & 35,970 \\ \tabucline[.5pt]{-}
\rowcolor{Yellow!40} N0101609 & 수술팩(Ⅰ)(마취시간 1시간이하) & ㈜세종헬스케어 & 35,970 \\ \tabucline[.5pt]{-}
\rowcolor{Yellow!40} N0101610 & 수술팩(Ⅰ)(마취시간 1시간이하) & 웰텍헬스케어주식회사 & 35,970 \\ \tabucline[.5pt]{-}
\rowcolor{Yellow!40} N0101611 & 수술팩(Ⅰ)(마취시간 1시간이하) & ㈜정상메디칼 & 35,970 \\ \tabucline[.5pt]{-}
\rowcolor{Yellow!40} N0101612 & 수술팩(Ⅰ)(마취시간 1시간이하) & ㈜케이엠헬스케어 & 35,970 \\ \tabucline[.5pt]{-}
\rowcolor{Yellow!40} N0101613 & 수술팩(Ⅰ)(마취시간 1시간이하) & ㈜한국디씨티 & 35,970 \\ \tabucline[.5pt]{-}
\end{tabu} \par
적용일자 : 2018-02-01
\par
\medskip

\leftrod{2018년 2월부터 12월 한시적 린넨팩 관리료}\par 
\tabulinesep =_2mm^2mm
\begin{tabu} to\linewidth {|X[1,l]|X[2,l]|X[1,l]|} \tabucline[.5pt]{-}
\rowcolor{ForestGreen!40}  코드  & 품명 & 금액 \\ \tabucline[.5pt]{-}
\rowcolor{Yellow!40} S0006 & 린넨팩 관리료-린넨팩(Ⅰ)(마취시간 1시간초과-3시간이하)을 사용하는 경우 & \myexplfn{113.6} \\ \tabucline[.5pt]{-}
\rowcolor{Yellow!40} S0007 & 린넨팩 관리료-린넨팩(Ⅰ)(마취시간 1시간이하)을 사용하는 경우 & \myexplfn{135.36} \\ \tabucline[.5pt]{-}
\end{tabu}
\par
\medskip
\highlightY{수술팩을 사용하는 행위에 대하여 종별로 대략 하기와 같이 수가 하락이 나타났습니다.}
\begin{itemize}[-]\tightlist
\item 상급종합병원: 29887-4277원
\item 종합병원: 28738-4113원
\item 병원: 27588-3948원
\item 의원: 29268-4197원
\item (단순 계산하여 100원 단위 금액은 차이가 있을 수 있습니다.) 
\end{itemize}
\Que{심평원 질의 결과​}
\Ans{심평원 담당자에게 수가 하락에 대한 질의 결과 수술 시간별로 \emph{행위군을 나누었으며 행위군별 상대가치점수 안에 직접비용으로 녹아들어 있는 린넨값에 대하여 가중평균치를 내서 수술군별로 상대가치점수를 하락}시켰다는 답변}




