\section{100분의 100 본인부담과 비급여 차이}
\Que{진료비 영수증에 보면 100분의 100 본인부담과 비급여 금액으로 나눠져 있는데요, 이 둘의 차이는 무엇인가요?}
\Ans{100분의 100 본인부담은 보건복지부장관이 정하여 고시한 상한금액을 환자가 모두 부담하는 것을 말합니다. 비급여는 보건복지부장관이 정하여 고시한 진료항목에 대하여 해당 진료를 실시하는 병의원에서 정한 금액을 환자가 모두 부담하는 것입니다. 따라서, 100분의 100 본인부담과 비급여의 차이는 100분의 100 본인부담은 법령 등으로 정하여진 상한금액이 있어 어느 병·의원에서든 동일한 금액을 환자에게 징수하여야하나, 비급여는 정하여진 금액이 없어 동일한 진료행위인 경우라도 병·의원별로 금액이 상이할 수 있습니다. 예를 들어 쌍꺼풀수술, 점제거술 등은 비급여대상 진료로 병·의원별로 금액이 상이합니다.}

\begin{commentbox}{100분의 80 본인부담}
종양 표지자 검사인 "인간 부고환 단백 4" 와 "CA125" 검사는 난소암이 의심되는 경우에 진단 목적으로 실시하는 경우 보험적용되고 있으며, 검사비용의 일부를 본인부담하시는 항목입니다. \par
※ 참고사항 
"나-437 인간 부고환 단백 4" : 선별급여 본인부담률 80% 적용 항목(2016.11.1.시행) 
"나-424 CA-125" : 급여 본인일부부담 항목(1999.11.16.시행) 
\end{commentbox}
