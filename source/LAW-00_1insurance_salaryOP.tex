\begin{mdframed}[linecolor=blue,middlelinewidth=2]  
제1편 행위 급여 \cntrdot{}  비급여 목록 및 급여 상대가치점수 >> 제2부 행위 급여 목록\cntrdot{} 상대가치점수 및 산정지침 >> 제9장 처치 및 수술료 등 
\end{mdframed}
\subsection{\newindex{제9장 처치 및 수술료 등}}
\paragraph{제1절 처치 및 수술료〔산정지침〕}
\begin{enumerate}[(1)]\tightlist
\item \uline{18시-09시 또는 공휴일에 응급진료가 불가피하여 처치 및 수술을 행한 경우에는 소정점수의 50\%를 가산한다.(산정코드 두 번째 자리에 18시-09시는 1, 공휴일은 5로 기재)} 이 경우 해당 처치 및 수술을 시작한 시각을 기준하여 산정한다.
\item 「응급의료에 관한 법률」에 의한 응급환자에게 응급의료기관이 응급실에서 응급의료수가기준 “(별표1) 응급의료수가기준액표 나. 응급처치료”의 해당 항목을 실시한 경우에는 소정점수의 50\%를 가산한다.(산정코드 두 번째 자리에 2로 기재)
\item \uline{제1절에 기재되지 아니한 처치 및 수술로서 간단한 처치 및 수술의 비용은 기본진료료에 포함되므로 산정하지 아니한다.}
\item 제1절에 기재되지 아니한 처치 및 수술로서 위 “⑶”에 해당되지 아니하는 처치 및 수술료는 \uline{제1절에 기재되어 있는 처치 및 수술 중에서 가장 비슷한 처치 및 수술 분류항목의 소정점수에 의하여 산정한다.(준용산정)}
\item \uline{대칭기관에 관한 처치 및 수술 중 “양측”이라고 표기한 것은 “양측”을 시술할지라도 소정점수만 산정한다.}
\item \uline{동일 피부 절개 하에 2가지 이상 수술을 동시에 시술한 경우 주된 수술은 소정점수에 의하여 산정하고, 제2의 수술부터는 해당 수술 소정점수의 50\%}(산정코드 세 번째 자리에 1로 기재), 상급종합병원\cntrdot{}종합병원은 해당 수술 소정점수의 70\%(산정코드 세 번째 자리에 4로 기재)를 산정한다. \uline{다만, 주된 수술 시에 부수적으로 동시에 실시하는 수술의 경우에는 주된 수술의 소정점수만 산정한다.}
\item 제1절에 기재된 분류항목 중 상\cntrdot{} 하악골 악성종양 절제술(자-40-나,
자-43-나), 비강, 부비동악성종양적출술(자-96), 비인강 악성종양적출술,(자-104-1), 후두 전적출술(자-122-1-다), 후두 및 하인두 전적출술(자-125), 후두 전적출 및 하인두 부분적출술(자-125-1), 구순암적출술(자-215), 설암 수술(자-218),구강내악성종양적출술(자-220-다), 이하선악성종양적출술(자-223-나), 인두악성종양수술(자-229-1), 부갑상선악성종양절제술(자-454-나), 갑상선 악성종양근치수술(자-456) 시행시 경부의 림프절 청소술을 병행한 경우에는 위 “(6)"에도 불구하고 경부림프절청소술(자-211) “주"의 소정점수를 별도 산정한다.
\item \uline{근접하고 있는 다발성 절종을 수개 처에서 절개한 경우나 동일 검내에 존재하는 맥립종, 산립종의 수술 등은 1회 절개로 간주한다.}
\item 수술은 개시하였으나 병상의 급변 등 부득이한 사유로 인하여 그 수술을 중도에서 중단하여야 할 경우에는 수술의 중단까지와 시술상태가 가장 비슷한 항목의 수술료를 산정한다.
\item 각 분류항목의 \uline{처치 및 수술 등에 레이저를 이용한 경우에도 각 분류항목의 소정점수만을 산정한다.}
\item 각 분류항목의 처치 및 수술 등에\uline{ 내시경을 이용한 경우 내시경료는 소정 시술료에 포함되므로 별도 산정하지 아니한다.}
\item 처치 및 수술시에 사용된 약제 및 치료재료대는 소정점수에 포함되므로 별도 산정하지 아니한다. 다만, 다음에 \uline{열거한 약제 및 치료재료대는 “약제 및 치료재료의 비용에 대한 결정기준”에 의하여 별도 산정한다.}
	\begin{enumerate}[①]\tightlist
	\item 인공식도
	\item 인공심장판막
	\item 인공심폐회로
	\item 인공심박기
	\item ....
	\item 체내고정용 나사, 고정용 금속핀, 고정용 금속선, 고정용 못
	\item 지속적주입, 지속적배액 및 지속적 배기용도관 [체내유치]
	\item 폴리비니루, 호루말 등 충전술 사용재료
		\begin{itemize}\tightlist
		\item 고주파신경자극기 [수술삽입시만 산정]
		\item 고정용 신축성 붕대
		\item 개심술, 안면수술 등 장관이 별도로 정한 처치 및 수술시 사용된 봉합사
		\item \uline{일반처치 또는 수술후처치(자-2-1), 피부과처치(자-18), 화상처치(자-18-1),위세척(자-590)에 사용된 생리식염수 [단, 총사용량이 500ml 이상인 경우에 한함] }
		\item \uline{피부과처치(자-18) 또는 화상처치(자-18-1)시 사용된 연고, 처치 및 수술시 사용된 인체주입용 약제}(단, KMnO4 등의 소독약제는 소정 처치 및 수술료에 포함되므로 별도 산정하지 아니한다.)
		\item 산정지침 ⑽에 해당되는 레이저시술 중 장관이 별도로 인정한 “레이저시술”에 소요된 레이저 재료대
		\item 제1절 및 제2절 분류항목에 별도로 표기한 경우
		\item 기타 장관이 별도로 인정한 약제 및 치료재료(인체조직 포함)
  		\end{itemize}
	\end{enumerate}  
\item (별표 1) 및 (별표 2)에 열거한 항목을 외과 전문의가 시행한 경우에는 해당 항목 소정점수의 (별표 1)은 20\%, (별표 2)는 30\%를 가산한다.(산정코드 첫 번째 자리에 1로 기재)
\end{enumerate}

\subsection{일반처치 또는 수술후처치(자-2-1)} 
\paragraph{자-2-1 일반처치 또는 수술후처치 등 [1일당] Dressing or Post Operative Dressing etc.}
주:
\begin{enumerate}[1.]\tightlist
\item \uline{수술후 처치료는 수술 익일부터 산정한다.}
\item \uline{사용된 거즈, 탈지면, 붕대, 반창고의 비용은 소정 점수에 포함되므로 별도 산정하지 아니한다.}
\item 같은 날에\uline{「다」와「라」, 「마」와「사」,「바」와「자」 또는 「아」와「자」를 실시한 경우에는 둘 중 한 항목의 소정점수만을 산정한다.}
\item \uline{같은 날에 「가」의 (1) 또는 (2)를 여러 부위에 실시한 경우에는 두부, 복부, 배부, 좌\cntrdot{}우\cntrdot{}상\cntrdot{}하지 7부위로 구분하여 각 부위별로 소정점수를 1회만 산정한다.}
\item 다만, 상급종합병원 중환자실에 입원중인 경우에는 [1일당], ’주3‘ 및 ’주4‘에도 불구하고 1일에
「가」는 2회 이내, 「라」와「바」는 3회 이내로 산정한다. (기본코드 5번째 자리에 5로 기재)
\end{enumerate}
%\myexplfn{ } \\ \tabucline[.5pt]{-} %
\tabulinesep =_2mm^2mm
\begin{tabu} to\linewidth {|X[2,l]|X[6,l]|X[1,l]|X[1,l]|} \tabucline[.5pt]{-}
\rowcolor{ForestGreen!40}  코드 &	\centering 분 류 & 점수 & 금액 \\ \tabucline[.5pt]{-}	
\rowcolor{Yellow!40} & 가. 창상처치 Wound Dressing && \\ \tabucline[.5pt]{-}
\rowcolor{Yellow!40} M0111 & (1) 단순처치 Simple Dressing & 58.04 & \myexplfn{58.04} \\ \tabucline[.5pt]{-} %4,320 \\ \tabucline[.5pt]{-}
\rowcolor{Yellow!40} M0121 & (2) 염증성 처치\footnote{주:수술창의 심한 염증 처치, 심한 욕창, 염증이 심한 상처의 처치에 산정한다.} Infectious Wound Dressing & 112.10 & \myexplfn{112.10} \\ \tabucline[.5pt]{-} %8,340 \\ \tabucline[.5pt]{-}
\rowcolor{Yellow!40} M0131 & 나. 장루처치 Stoma Care & 86.75 & \myexplfn{86.75} \\ \tabucline[.5pt]{-} %6,450 \\ \tabucline[.5pt]{-}
\rowcolor{Yellow!40} M0134 & 다. 수술후 튜브삽입에 의한 자연 배액감시 및 처치 Natural Drainage and Care after Operation & 47.57 & \myexplfn{47.57} \\ \tabucline[.5pt]{-} %3,540 \\ \tabucline[.5pt]{-}
\rowcolor{Yellow!40} M0137 & 라. 흡입배농 및 배액처치 Suction Drainage or Tracheostomy Suction etc. & 113.96 &  \myexplfn{113.96} \\ \tabucline[.5pt]{-} %8,480 \\ \tabucline[.5pt]{-}
\rowcolor{Yellow!40} M0141 & 마. 좌욕 Sitz Bath & 19.40 & \myexplfn{19.40} \\ \tabucline[.5pt]{-} %1,440 \\ \tabucline[.5pt]{-}
\rowcolor{Yellow!40} M0143 & 바. 체위변경처치\footnote{주:척수손상, 뇌졸중 환자 등에서 혈액순환 도모 및 욕창방지 등을 위해 피부마사지를 포함한 체위
변경 시에 산정한다.} Position Change & 86.96 & \myexplfn{86.96} \\ \tabucline[.5pt]{-} %6,470 \\ \tabucline[.5pt]{-}
\rowcolor{Yellow!40} M0151 & 사. 회음부 간호 Perineal Care & 56.47 & \myexplfn{56.47} \\ \tabucline[.5pt]{-} %4,200 \\ \tabucline[.5pt]{-}
\rowcolor{Yellow!40} M0153 & 아. 통목욕 간호 Tub Bath & 120.36 & \myexplfn{120.36} \\ \tabucline[.5pt]{-} %8,950 \\ \tabucline[.5pt]{-}
\rowcolor{Yellow!40} M0155 & 자. 침상목욕 간호 Bed Bath & 157.91 & \myexplfn{157.91} \\ \tabucline[.5pt]{-} %11,750 \\ \tabucline[.5pt]{-}
\end{tabu}
%\par

%\paragraph{피부과처치(자-18), 화상처치(자-18-1)}
\paragraph{자-18 피부과처치 [1일당] Dermatologic Dressing}
주:
\begin{enumerate}[1.]\tightlist
\item \uline{피부연고 도포 등 단순한 피부 처치는 기본진료료에포함되므로 별도 산정하지 아니한다.}
\item \uline{사용된 거즈, 탈지면, 붕대, 반창고의 비용은 소정 점수에 포함되므로 별도 산정하지 아니한다.}
\end{enumerate}
\tabulinesep =_2mm^2mm
\begin{tabu} to\linewidth {|X[2,l]|X[6,l]|X[1,l]|X[1,l]|} \tabucline[.5pt]{-}
\rowcolor{ForestGreen!40}  코드 &	\centering 분 류 & 점수 & 금액 \\ \tabucline[.5pt]{-}	
\rowcolor{Yellow!40} & 가. 농가진, 감염성피부질환 등에 Wet Dressing 또는 Soaking을 행한 경우 && \\ \tabucline[.5pt]{-}
\rowcolor{Yellow!40} N0181 & (1) 9\% 이하의 범위 & 72.48 & \myexplfn{72.48} \\ \tabucline[.5pt]{-} %5,390 \\ \tabucline[.5pt]{-}
\rowcolor{Yellow!40} N0182 & (2) 10\%∼18\%의 범위 & 85.26 & \myexplfn{85.26} \\ \tabucline[.5pt]{-} %6,340 \\ \tabucline[.5pt]{-}
\rowcolor{Yellow!40} N0183 & (3) 19\%∼36\%의 범위 & 96.43 & \myexplfn{96.43} \\ \tabucline[.5pt]{-} %7,170 \\ \tabucline[.5pt]{-}
\rowcolor{Yellow!40} N0184 & (4) 37\% 이상의 범위 & 135.14 & \myexplfn{135.14} \\ \tabucline[.5pt]{-} %10,050 \\ \tabucline[.5pt]{-}
\rowcolor{Yellow!40} & 나. 대상포진에 실시한 경우 In Herpes Zoster& &  \\ \tabucline[.5pt]{-}
\rowcolor{Yellow!40} N0061 & (1) 9\% 이하의 범위 & 76.60 & \myexplfn{76.60} \\ \tabucline[.5pt]{-} %5,700 \\ \tabucline[.5pt]{-}
\rowcolor{Yellow!40} N0062 & (2) 10\%∼18\%의 범위 & 88.45 & \myexplfn{88.45} \\ \tabucline[.5pt]{-} %6,580 \\ \tabucline[.5pt]{-}
\rowcolor{Yellow!40} N0063 & (3) 19\%∼36\%의 범위 & 100.99 & \myexplfn{100.99} \\ \tabucline[.5pt]{-} %7,510 \\ \tabucline[.5pt]{-}
\rowcolor{Yellow!40} N0064 & (4) 37\% 이상의 범위 & 139.47 & \myexplfn{139.47} \\ \tabucline[.5pt]{-} %10,380 \\ \tabucline[.5pt]{-}
\end{tabu}
%\par

  
\paragraph{자-18-1 화상처치 Burn Dressing}
주:
\begin{enumerate}[1.]\tightlist
\item 화상부위가 수개 부위일 경우에는 수개 부위의 화상범위를 합하여 아래 항목에 의거하여 산정하되 \uline{화상범위 산정시 1도 화상 범위는 제외한다.}
\item \uline{사용된 거즈, 붕대의 재료대는 별도 산정하되 탈지면, 반창고 등의 비용은 소정점수에 포함되므로 별도 산정하지 아니한다.}
\end{enumerate}
\tabulinesep =_2mm^2mm
\begin{tabu} to\linewidth {|X[2,l]|X[6,l]|X[1,l]|X[1,l]|} \tabucline[.5pt]{-}
\rowcolor{ForestGreen!40}  코드 &	\centering 분 류 & 점수 & 금액 \\ \tabucline[.5pt]{-}	
\rowcolor{Yellow!40} &가. 열탕, 화염, 동상, 화학화상 등의 경우 Scald, Flame, Frostbite, Chemical Burn etc. && \\ \tabucline[.5pt]{-}
\rowcolor{Yellow!40} & (1) 9\% 이하의 범위 & & \\ \tabucline[.5pt]{-}
\rowcolor{Yellow!40} N0011 & (가) 수, 족, 지, 안면, 경부, 성기를 포함하는 경우 Including Hand, Foot, Finger or Toe, Face,Neck.Genitalia & 341.06 & \myexplfn{341.06} \\ \tabucline[.5pt]{-} %25,370 \\ \tabucline[.5pt]{-}
\rowcolor{Yellow!40} N0012 & (나) 수, 족, 지, 안면, 경부, 성기를 포함하지 아니한 경우 Excluding Hand, Foot, Finger or Toe,Face, Neck, Genitalia & 198.46 & \myexplfn{198.46} \\ \tabucline[.5pt]{-} %14,770 \\ \tabucline[.5pt]{-}
\rowcolor{Yellow!40} N0053 & (2) 하지의 1지, 복부 또는 배부에 준하는 범위[10\%-18\%] One Lower Extremity, Abdomen or Back & 913.15 & \myexplfn{913.15} \\ \tabucline[.5pt]{-} %67,940 \\ \tabucline[.5pt]{-}
\rowcolor{Yellow!40} N0054 & (3) 양하지 또는 동체(복부 및 배부)에 준하는 범위 [19\%-36\%] Both Lower Extremities or Trunk & 1,682.54 & \myexplfn{1682.54} \\ \tabucline[.5pt]{-} %125,180 \\ \tabucline[.5pt]{-}
\rowcolor{Yellow!40} NA055 & (4) 상\cntrdot{} 하지 대부분, 양하지와 복부 또는 배부에 준하는범위 [37\%-54\%]
Upper\cntrdot{} Lower Extremities, Both Lower Extremities and Abdomen or Back &  2,633.56 & \myexplfn{2633.56} \\ \tabucline[.5pt]{-} %195,940 \\ \tabucline[.5pt]{-}
\rowcolor{Yellow!40} NA056 & (5) 전신대부분의 범위 [55\% 이상의 범위] more than 55\% of Body Surface Area
& 3,950.32 & \myexplfn{3950.32} \\ \tabucline[.5pt]{-} %293,900 \\ \tabucline[.5pt]{-}
\rowcolor{Yellow!40} & 나. 전기화상의 경우 Electrical Burn N0057 (1) 근육, 골격, 인대의 손상이 포함된 경우
with Injury of Muscle, Skeletal or Tendon & 1,682.54 & 125,180 \\ \tabucline[.5pt]{-}
\rowcolor{Yellow!40} NA057 & 주:섬광 또는 화염이 동반된 경우에는 & 3,365.08 &  \\ \tabucline[.5pt]{-}
\rowcolor{Yellow!40} N0058 & (2) 기타의 경우 Others & 992.39 & \myexplfn{992.39} \\ \tabucline[.5pt]{-} %73,830 \\ \tabucline[.5pt]{-}
\rowcolor{Yellow!40} NA058 & 주:섬광 또는 화염이 동반된 경우에는 & 1519.60점을 산정 & \\ \tabucline[.5pt]{-}
\end{tabu}
%\par