\subsection{검사관련}
※「의료의 질향상을 위한 점검표」중 ‘1.1 수술전 검사 시행여부’ 관련
\Que{수술전 검사항목은 무엇인가요?}
\Ans{척추마취 및 전신마취의 경우\newline
(A) 7개질병군 공통 : CBC(일반혈액검사), U/A(요검사), LFT(간기능검사), Electrolyte(전해질검사), Chest PA(흉부방사선촬영), EKG(심전도), BUN(요소질소), Creatinine(크레아티닌), Coagulation(응고검사), ABO/Rh(혈액형검사)} 
\prezi{\clearpage}
\par
\medskip
\Que{수술전 검사 일부 항목만 시행했을 경우에도 “시행”에 표시하면 되나요?}
\Ans{수술전 검사항목을 모두 시행하였을 경우 “시행”에 표시하고 수술전 검사항목 중 하나라도 시행하지 아니한 경우에는 “미시행”에 표시합니다.
- 수정체수술 또는 편도 및 아데노이드 수술을 전신마취 또는 척추마취하에 실시한 경우는 7개 질병군 공통 검사(A)와 수정체수술(B) 또는 편도 및 아데노이드 수술(C)의 검사를 모두 시행한 경우 “시행”에 표시합니다.\par
※ 예시) 편도 및 아데노이드 수술을 전신마취 하에 실시한 경우
(A) + (C)의 검사를 모두 시행한 경우 “시행”에 표시 }
\par
\medskip
\prezi{\clearpage}
\Que{입원하여 시행한 수술전 검사만 해당되나요? }
\Ans{외래에서 시행한 수술전 검사와 입원하여 시행하는 수술전 검사 모두 해당됩니다.}
\prezi{\clearpage}
\par
\medskip
\Que{수술전 검사 시행여부의 추가코드 표시는 어떻게 하나요?}
\Ans{추가코드 □󰊱 □󰊲 □󰊳은 마취의 종류를 표기하는 코드로서 전신 마취를 시행한 경우에는 추가코드 ☑󰊱에, 척추마취를 시행한 경우 에는 ☑󰊲에, 기타 국소마취를 시행한 경우에는 ☑󰊳에 표시하여야 합니다.}
\prezi{\clearpage}