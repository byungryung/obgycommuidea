\section{자궁경부 용종}
\myde{}{%
\begin{itemize}\tightlist
\item[\dsjuridical] N841 자궁목의 폴립
\item[\dsmedical] R4240 자궁경관점막폴립절제술 / [\myexplfn{365.15} 원]
\item[\dsmedical] R4300 자궁경부(질)약물소작술 Chemical Cauterization of Cervix(Vagina) [\myexplfn{228.47}원]
\item[\dsmedical] 알보칠 비보험
\item[\dsmedical] N841 코드에 \sout{CX541} C5624[액상자궁경부세포검사] 청구가능
\item[\dsmedical] \sout{조직검사는 대개  C5911또는 C5912로 나감.}
\item[\dsmedical] C5602008 LEVEL B 조직병리검사 질가산 포함.
\item[\dsjuridical] Z988 기타 명시된 수술후 상태 : 약물소작시만.
\end{itemize}
}%
{
\emph{polypectomy후 bleeding시에 알보칠 사용후에 R4300추가}
	\begin{enumerate}[가.]\tightlist
	\item 청구사례 : polypectomy+R4300+질강처치를 같이 청구 했는데 R4300이 조정되었습니다. 이의신청해야 하겠지요?\par
		질강처치 조정이 맞습니다. R4300(자궁경부(질)약물소작술)와 질강처치(R4106)는 동일부위 유사행위입니다.자-424(자궁경관점막폴립절제술)와  R4300(자궁경부(질)약물소작술)는 주된 수술(시술) 100\%, 그외 수술 (시술)50\%를 산정이 맞을것 같습니다.
		동일 절개하에서 2가지 이상 수술을 동시에 시술한 경우 주된 수술이란 2가지 이상 수술 중 소정금액이 높은 수술을 기준으로 함. 
		2. 한 절개부위에서 해당과를 달리하여 각각 다른 병변을 수술한 경우, 진료전문과목이 다르더라도 동일 마취하에 연속하여 수술을 하는 것이므로 주된 수술 100\%, 그외 수술 50\%를 산정함.(고시 제2000-73호, `01.1.1. 시행)
	\item \emph{질강처치료와  자궁경부 약물소독술은 유사 행위로 같이 청구하시면 안됩니다.} 질강처치료와 koh미생물검사 가능합니다. koh미생물검사와 자궁경부 약물소독술 청구 가능합니다.
	\item 결과적으로 상병에 N841 자궁목의 폴립, Z988 기타 명시된 수술후 상태 - 청구 메모: 예>> 자궁경부 폴립제거후 출혈이 있어 시행함.을 기재한다.
	\end{enumerate}
}

\tabulinesep =_2mm^2mm
\begin {tabu} to\linewidth {|X[3,l]|X[9,l]|X[1.2,l]|X[1.2,l]|X[3,l]|} \tabucline[.5pt]{-}
\rowcolor{ForestGreen!40} 처방 코드 & \centering 명 칭 &	\centering 수량 & \centering 횟수 & 용법 \\ \tabucline[.5pt]{-}
\rowcolor{Yellow!40} R4240 & 자궁경관점막폴립절제술 & 1.00 & 1 & 보험적용 \\ \tabucline[.5pt]{-}
\rowcolor{Yellow!40} Z988 & 기타 명시된 수술후 상태 &  &  & 약물소작시만 \\ \tabucline[.5pt]{-}
\rowcolor{Yellow!40} \sout{C5911} & 병리조직검사[1장기당]-생검(1-3개) & 1.00 & 1 & 보험적용 \\ \tabucline[.5pt]{-}
\rowcolor{Yellow!40} \sout{C5911} & 병리조직검사[1장기당]-생검(4-6개) & 1.00 & 1 & 보험적용 \\ \tabucline[.5pt]{-}
\rowcolor{Yellow!40} C5602 & LEVEL B 조직병리검사 질가산포함. & 1.00 & 1 & 보험적용 \\ \tabucline[.5pt]{-}
\rowcolor{Yellow!40} \sout{CX541} C5624 & 액상자궁경부세포검사 & 1.00 & 1 & 보험적용 \\ \tabucline[.5pt]{-}
\rowcolor{Yellow!40} VSONO & 질식초음파검사 & 1.00 & 1 & 인정비급여 \\ \tabucline[.5pt]{-}
\rowcolor{Yellow!40} R4300 & 자궁질부약물소작술 & 1.00 & 1 & 보험적용* \\ \tabucline[.5pt]{-}
\end{tabu}
\par
\medskip
* 청구 메모 : 자궁경부 폴립제거후 출혈이 있어 알보칠로 자궁경부약물소작함. F/U시에도 청구가능