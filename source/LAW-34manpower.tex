\section{필요인력 인센티브 관련}
\Que{필요인력 유급휴가 중 대체 고용인력이 인정되나요?}
\Ans{필요인력이 16일 이상 장기유급휴가 시 인력 산정대상에서 제외됩니다.다만 장기유급휴가자를 대체하는 상근자가 있는 경우에는 신고후 인정가능합니다.}


\Que{필요인력의 최종 퇴직일 직전 기간에 연차 휴가인 경우, 인력으로 산정 가능하나요?}
\Ans{필요인력(상근하는 약사, 의무기록사, 방사선사, 임상병리사, 사회복지사, 물리치료사)은 요양병원입원료차등제산정현황통보서 상의 상근자를 의미하며, 16일 이상 장기 유급휴가의 경우에는 산정대상에서 제외합니다.}

\Que{환자수 200명 미만인 요양병원에서 주당 16시간 이상 근무약사를 고용한 경우 근로기준법상 연차가 인정되나요?}
\Ans{필요인력 확보에 따른 별도 보상제와 관련하여 환자수 200명 미만인 요양병원에서 약사가 주 16시간 이상 근무한 경우에는 근로기준법령에 따른 휴가가 인정 가능합니다.}