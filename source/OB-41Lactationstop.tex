\section{분만과 모유수유}
\subsection{수유중단}
\myde{}
{
\begin{itemize}\tightlist
\item[\dsjuridical] o925 억제된 수유
\item[\dsjuridical] o927 기타 및 상세불명의 수유의 장애
%\item[\dsmedical] 
%\item[\dschemical] 
\end{itemize}    
}
{
\begin{enumerate}\tightlist
\item 약: 슈다페드 하루 1알씩 tid  7일 복용, 고용량 에스트로겐 주사나 카버락틴은 분만 후 48시간 이내만 효과 
\item 카버락틴(동구제약) 산후 첫날 1mg(2T) 생산된 유즙분비 억제시 2일간 0.25mg씩 12시간마다 복용하면 된대요.  젖 말리는 것은 비보험
\item 수유전에는 유륜주위 따뜻한 마사지  수유후에는 냉찜질이 매우 효과적
\item 젖 짜내면 안됨(수유 촉진됨) 
\item 아프거나 많이 딱딱할 경우 유륜 주위만 살짝 짜내는 것은 괜찮음 확 짜면 안됨
\item 붕대 감는 것에 대해선 효과 확실치 않음 (논란)
\item 양배추 요법 : 양배추(주로 겉부분 사용, 찌지 않고) 냉장고에 넣어놨다가 유방에 부착, 브라 안에 한장씩 넣음, 2-3시간 후 시들해지면 교체, 유륜 부위는 동그랗게 오려내면 됨
\end{enumerate}
}
\Que{카버락틴 정 CAVERLACTIN TAB (조성 cabergoline 0.5㎎ )의 효능/효과는 다음과 같습니다\par

유즙 분비의 예방 및 억제. \par
이 약은 이미 생산된 젖 분비와 출산직후의 생리적인 젖분비 억제에 사용된다. 
1) 출산 후 산모가 모유수유를 원하지 않거나 산모 또는 신생아와 관련된 의료상의 이유로 모유수유를 할 수 없는 경우. 
2) 사산 또는 유산 후. \par

출산후 모유수유후 본인(산모)가 모유수유를 더 이상 원하지 않아 젖말리기를 원할 경우 
상기 약제(카버락틴)를 보험(급여)적용 가능한가요?
아니면 100/100처방 적용을 해야 하는건가요? }
\Ans{건강보험법령에 따라 의약품은 기본적으로 식품의약품안전처의 허가사항 범위 내에서 사용하여야 하며, 「요양급여의 적용기준 및 방법에 관한 세부사항」고시에서 규정하고 있는 약제의 경우에는 동 고시 범위 내에서 처방ㆍ투여 시 보험 적용됩니다.\par
문의주신 Cabergoline 경구제 (품명: 카버락틴정 0.5mg)의 경우 고시 제2014-190호(‘14.11.1 시행)에 따라 허가사항 범위 내에서 투여 시 요양급여 함을 원칙으로 하고 있으나, \uline{산모와 신생아가 건강상 필요치 않는 단순한 모유억제를 위해 사용하는 경우에는 비급여 대상}으로 하도록 하고 있습니다.
}

\begin{commentbox}{팔로델}
\begin{itemize}\tightlist
    \item 팔로델은 cardiologic complication때문에 젖먹이는 목적으로 사용하지 말 것을 권장합니다(임경빈 원장님)
    \item 스페로프에 팔로델은 젖말리는 용도로는 사용하지 말라고 되어있습니다(박수희원장님)
\end{itemize}    
\end{commentbox}

\subsection{수유량증가약물}
\tabulinesep =_2mm^2mm
\begin {tabu} to\linewidth {|X[1,l]|X[6,l]|} \tabucline[.5pt]{-}
\rowcolor{ForestGreen!40} \centering 약품명 & \centering Motilium \\ \tabucline[.5pt]{-}
\rowcolor{Yellow!40} 성분명 & domperidone \\ \tabucline[.5pt]{-}
\rowcolor{Yellow!40} 처방 & 1일 30mg\#3*7days (2번가능) \\ \tabucline[.5pt]{-}
\rowcolor{Yellow!40} 보험코드 & 구토(R112) \\ \tabucline[.5pt]{-}
\end{tabu}

