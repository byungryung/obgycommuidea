\subsection{부당청구}
\begin{enumerate}[① ]\tightlist
\item 1인실 기준 위반 :
    전체 병동의 50\% 이상을 다인실로 해야만 나머지 병실을 1인실로 인정받을 수 있음. 즉, 이 기준을 위반하게 되면 모든 병실을 1인실로 인정받을 수 없음
\item 모자동실 위반 :
    원칙은 모자동실은 24시간이여야 함. 법원 판례를 보면 최소한 12시간 이상 모자 동실을 한 경우만 ‘모자 동실료’가 인정이 됨.
\item 의료법 및 의료 기사법 위반
	\begin{itemize}\tightlist
	\item UDS는 의사가 직접 해야 한다.(적어도 의사가 지휘 감독이라도 해야 함) 
	\item EKG는 임상 병리사 / 의사가 시행해야 한다.
	\item BMD / CXR / MAMMO는 방사선사가 찍어야 한다.
	\item 방사선 기계와 임상 병리기계가 심평원에 등록된 기종과 다르면 안 된다.
	\item 일반식 가산은 상주하는 영양사 / 조리사가 2명 이상 있어야 한다. 
	\item 약 조제는 의사 / 약사가 해야 한다. (병동 환자 약 포장은 혼합 조제의 개념이므로  간호사 / 간호 조무사 모두 안 된다. 약사 출근 여부 전화로 확인 가능함.)
	\item 수액 line은 의사 / 간호사가 잡아야 한다. (간호조무사는 의사 감독하에만 가능하다.)
	\item 채혈은 의사 / 임상 병리사가 해야 한다.(간호 조무사는 안 된다. 간호사는 ?)
	\item NST는 간호사가 걸어야 한다. (임상병리사 / 방사선사 안 된다. 간호 조무사는 ?)  
	\end{itemize}
\item 비상근 영상의학과 의사 관련
Mammogram은 특수영상 장비로 비근속 영상의학과 의사 가 적어도 일주일에 한번은 출근해서 정도 관리를 해야 한다.(직접 영상의학과 의사에게 전화 걸어 출근 사실 확인함)
\item 진찰료 산정 위반
	\begin{itemize}\tightlist
	\item 국가암 검진 시 당일 급여 진료를 본 경우 진찰료는 50%
   산정해야 함
	\item 대리 처방 후 진찰료는 50\% 산정해야 함
	\end{itemize}	
\item 초/재진 산정 위반
\item 급여 / 비급여 산정 위반
	\begin{itemize}\tightlist
	\item 급여 항목을 비급여로 청구한 경우(임의비급여)
	\item 비급여 항목을 급여로 청구한 경우
	\end{itemize}	
\item 별도 산정할 수 없는 치료 재료대 산정
	\begin{itemize}\tightlist
	\item 행위료에 포함되어 있는 것으로 간주되는 치료대를 별도로 산정하는 경우
	\item 약물 소작술 시 알보칠 질정은 인정 안 됨.(알보칠 액은 비급여 산정 가능)
	\item 소수술시 비용 보전 위해 치료 재료대를 비급여 산정하면 안됨. (수술 중 초음파로 대체)
	\end{itemize}	
\item DRG에서 유착 방지제 / 인소브 / 영양제
	\begin{itemize}\tightlist
	\item C/SEC 환자는 DRG로 유착 방지제 / 영양제 / 인소브 비용 받으면 안 됨
	\item 부인과 수술의 경우 DRG로 유착 방지제 / 인소브 비용 받으면 안 됨(Ectopic preg는 제외) 
	\item 물품의 구매량과 재고량을 카운트하므로 퇴원 영수증에 기입을 하지 않고 수납을 하더라도 문제가 된다.
	\end{itemize}	
\end{enumerate}  

\begin{commentbox}{공단 건강검진 실시 후 산정기준 위반청구}
\begin{description}\tightlist
\item[관련근거] 
\begin{itemize}\tightlist
\item 국민건강보험법 제52조(건강검진), 건강검진기본법('08.3.21 제정, 법률 제8942호)
\item 건강검진실시기준, 암검진실시기준 등
\end{itemize}
\item[부당사례]
\begin{enumerate}[1)]\tightlist
\item B의원은 공단검진당일 검진과는 \highlight{별도 질환에 대한 진찰 시 진찰 이외에 의사의 처방이 발생한 경우 진찰료의 50\%를 산정할 수 있으나, 진찰료 100\%를 요양급여비용으로 청구(산정기준 위반청구)}
\end{enumerate}
\end{description}
\end{commentbox}

\textbf{사회복지시설 내 촉탁의 진료 후 진찰료 부당청구}
\begin{commentbox}{사회복지시설 등의 입소자에게 원외처방전 교부 후 진찰료 100\% 청구}
\begin{description}\tightlist
\item[관련근거] 
\begin{itemize}\tightlist
\item 건강보험 행위급여\cntrdot{}비급여 목록표 및 급여상대가치점수 제1장 기본진료료 가-1-나(재진진찰료) 주:8. 사회복지사업법에 따른 사회복지시설 내에서 의료기관 소속 촉탁의 또는 협약의료기관의사가 시설 입소자에게 원외처방전을 교부한 겨우 \highlightR{진찰료 중 외래관리료 소정점수를 산정하여야 함(AA254080코드로 청구)}
\end{itemize}
\item[부당사례]
\begin{enumerate}[1)]\tightlist
\item A의원은 2013년 9월부터 총 9회 ‘상세불명 원인의 상세불명의 접촉피부염(L259)’ 등의 상병으로 진료한 수진자 ㅇㅇㅇ의 경우 \highlight{사회복지시설 입소자임에도 불구하고 의료기관 소속 촉탁의가 시설에 방문하여 진료 후 원외처방전을 교부해주고 진찰료 100\%를 요양급여비용으로 청구}
\end{enumerate}
\end{description}
\end{commentbox}

\paragraph{환자 가족이 내원하여 처방전 발급 시 진찰료 부당청구}\par
환자 보호자(가족)에게 처방전 등을 발급한 후 재진진찰료 100\% 청구
\begin{description}\tightlist
\item[관련근거] 
\begin{itemize}\tightlist
\item 건강보험 행위급여\cntrdot{}비급여 목록표 및 급여상대가치점수 제1장 기본진료료 가-1-나(재진진찰료) 주:7. 환자가 직접 내원하지 아니하고 \highlight{환자 가족이 내원하여 진료담당의사와 상담한 후 약제를 수령하거나 처방전만을 발급받는 경우에는 재진진찰료 소정점수의 50\%를 산정}하여야 함.
\end{itemize}
\item[부당사례]
\begin{enumerate}[1)]\tightlist
\item  A의원은 수진자 ㅇㅇㅇ의 보호자만 내원하여 진료의사와 상담 후 약제를 수령 했으나 \highlight{재진진찰료 50\%(AA254090)를 산정 청구하여야 함에도 재진진찰료 100\%}를 산정하여 요양급여비용을 청구
\end{enumerate}
\end{description}

\textbf{영상의학과 전문의가 상근하지 아니하면서 판독료 가산 청구}
\begin{description}\tightlist
\item[관련근거] 
\begin{itemize}\tightlist
\item 건강보험 행위급여\cntrdot{}비급여 목록표 및 급여상대가치점수 제3장 영상진단 및 방사선치료료 제1절 방사선단순영상진단료 주1 및 제2절 방사선특수영상진단료 주1 - 당해 요양기관에 상근하는 영상의학과 전문의가 판독을 하고 판독소견서를 작성한 경우에 소정점수의 10\%를 가산함
\end{itemize}
\item[부당사례]
\begin{enumerate}[1)]\tightlist
\item   A의원은 영상의학과 전문의가 판독소견서를 작성하지 않고 \highlight{해당 진료과의 전문의가 진료기록부에 판독소견을 기록한 후 판독료(소정점수) 10\%의 가산료를 산정}하여 요양급여비용을 청구
\end{enumerate}
\end{description}

\paragraph{무자격자가 시행한 방사선단순영상진단료 부당청구}\par
\textbf{무자격자가 촬영한 방사선영상진단료 청구}
\begin{description}\tightlist
\item[관련근거] 
\begin{itemize}\tightlist
\item 의료법 제27조(무면허 의료행위 등 금지) 제1항 - 의료인이 아니면 누구든지 의료행위를 할 수 없으며 의료인도 면허된 것 이외의 의료행위를 할 수 없음. -
\item 의료기사 등에 관한 법률 제3조(업무범위와 한계), 제9조(무면허 업무금지 등) 및 동법 시행령 제2조(의료기사, 의무기록사 및 안경사의 업무범위 등)
\end{itemize}
\item[부당사례]
\begin{enumerate}[1)]\tightlist
\item  A의원은 방사선사가 근무하지 않은 기간 동안, \highlight{방사선사 자격이 없는 원무과장이 방사선영상진단 촬영}을 하고 그 비용을 요양급여비용으로 청구
\end{enumerate}
\end{description}

\paragraph{입원환자 식대-영양사, 조리사 가산 부당청구}\par
\begin{commentbox}{영양사 조리사 가산 관련 동 인력의 실제 근무내용과 다르게 신고}
\begin{description}\tightlist
\item[관련근거] 
\begin{itemize}\tightlist
\item 입원환자 식대 세부산정기준(보건복지부 제2016-91호(행위), 2016.6.15.) 
	\begin{itemize}\tightlist
	\item 입원환자 식대 세부산정기준에 의거 일반식 가산에서 영양사 가산, 조리사 가산에 필요한 인력산정 기준은 환자식 제공업무를 주로 담당하는 \highlight{당해 요양기관에 소속된 인력으로 의원급(보건의료원 포함) 각각 1명, 병원급 이상은 각각 2명 이상인 경우 산정}함
	\item 전일제 영양사 및 조리사로 1주간의 근로시간이 월평균 40시간인 근무자는 1인으로 산정하고 단시간 근무로 1주간의 근로시간이 월평균 32시간(이상) - 40시간(미만) 근무자는 0.8인으로 산정하며, 32시간 미만 근무자는 산정대상에서 제외함.
	\end{itemize}
\end{itemize}
\item[부당사례]
\begin{enumerate}[1)]\tightlist
\item A병원은 [표] 영양사 근무현황과 같이 영양사 4명에 대한 입\cntrdot{}퇴사일을 실제 근무내역과 다르게 신고하여 2013.11월, 2013년 12월, 2014년 2월 영양사 가산을 산정할 수 없음에도 요양급여비용으로 청구함.
\item B병원은 조리사 ㅇㅇㅇ가 2013.1.8. - 2013.10.23. 상근 근무한 것으로 신고하였으나 \highlight{실제로는 근무 사실이 없으며,} 조리사 ㅇㅇㅇ는 2011.10.21. - 2013 1.7.근무한 것으로 신고하였나 실제 2011.10.21.-2012.9.8.까지 근무한 것으로 확인되는 등 2012.9.9.-2013.10.23.까지 \highlight{조리사 1인만 상근으로 근무하였음에도 식대 조리사 가산을 요양급여비용으로 청구함}
\end{enumerate}
\end{description}

영양사 근무현황\par
\tabulinesep =_2mm^2mm
\begin{tabu} to \linewidth {|X[2,c]|X[2,c]|X[2,l]|X[2,l]|X[2,l]|X[2,l]|} \tabucline[.5pt]{-}
\rowcolor{Gray!25}  구분 & 성명 & 신고 & 내역 & 확인 & 내역 \\ \tabucline[.5pt]{-}
\rowcolor{Yellow!5} 영양사 & & 입사일 & 퇴사일 & 입사일 & 퇴사일 \\ \tabucline[.5pt]{-}
\rowcolor{Yellow!5} & ㅇㅇㅇ & 2013.9.1 & 2013.11.30 & 2013.9.9 & 2013.1.30 \\ \tabucline[.5pt]{-}
\rowcolor{Yellow!5} & ㅇㅇㅇ & 2013.9.1 & 2013.9.16 & 2013.9.1 & 2013.9.8 \\ \tabucline[.5pt]{-}
\rowcolor{Yellow!5} & ㅇㅇㅇ & 2013.9.17 & 2013.10.7 & 2013.9.17 & 2013.10.2 \\ \tabucline[.5pt]{-}
\rowcolor{Yellow!5} & ㅇㅇㅇ & 2013.12.1 & 2014.4.30 & 2013.12.3 & 2014.4.30 \\ \tabucline[.5pt]{-}
\end{tabu}
\par
\end{commentbox}

\paragraph{별도 산정할 수 없는 치료재료대 부당징수}\par
\textbf{행위료에 포함되어 별도 산정할 수 없는 치료재료료 본인부담금으로 징수}
\begin{description}\tightlist
\item[관련근거] 
\begin{itemize}\tightlist
\item 요양급여의 비용 중 본인이 부담할 비용의 부담액은 국민건강보험법제41조(요양급여) 및 제44조(비용의 일부부담), 동법 시행령 제19조(비용의 본인부담) 및 [별표2] \highlight{요양급여비용 중 본인이 부담할 비용의 부담률 및 부담액에 따라 징수하고}, 요양급여사항 또는 비급여사항 외의 다른 명목으로 비용청구를 해서는 안됨
\end{itemize}
\item[부당사례]
\begin{enumerate}[1)]\tightlist
\item A의원은 관련 행위료에 포함되어 그 비용을 별도 산정할 수 없는 치료재료 (\highlight{hemoclip})을 사용하고 그 비용을 수진자에게 전액 본인부담금으로 징수 
\item B의원은 경막외신경차단술(LA222) 등을 실시하고 신경차단술 행위료에 포함된 치료재료(\highlight{Epidural needle})의 일부비용을 별도 수진자에게 본인부담금으로 징수 
\item C병원은 요양급여비용의 소정수가에 포함되어 별도산정 불가인 \highlight{수술포} 등을 수진 자에게 본인부담금으로 징수 행위료에 포함되어 별도 산정할 수 없는 치료재료료 본인부담금으로 징수
\end{enumerate}
\end{description}

\paragraph{의사인력 확보수준에 따른 입원료 차등제 산정기준 위반청구}
\textbf{의사인력을 실제 근무 사실과 다르게 신고}
\begin{description}\tightlist
\item[관련근거] 
\begin{itemize}\tightlist
\item 건강보험 행위 급여\cntrdot{}비급여목록표 및 급여상대가치점수 제3부 요양병원 산정지침 4.라.'의사인력 확보수준에 따른 입원료 차등제'
\item 의사인력 확보수준에 따른 요양병원 입원료 차등적용 관련 기준(보건복지부고시 제2009-214호(요양병원), 2010.4.1.)
\end{itemize}

\item[부당사례]
\begin{enumerate}[1)]\tightlist
\item A요양병원은 의사 ㅇㅇㅇ가 2011년 5월 9일부터 2011년 6월 13일까지 근무한 사실이 없으나 상근한 것으로 신고∙적용하여, 의사인력확보수준에 따른 입원료 차등제를 실제 의사등급보다 높게 적용하여 요양급여비용을 청구
\item B요양병원은 주1~2일만 근무한 의사를 상근으로 신고하여 의사인력 확보수준이 2등급임에도 1등급으로 적용하여 (1등급 전문의 수가 50\%이상인 경우 산정가능한) 요양병원 입원료를 20\%가산하여 청구
\item ※ 내과,외과, 신경과,정신건강의학과, 재활의학과, 가정의학과, 신경외과, 정형외과 전문의 수 50\%이상
\end{enumerate}
\end{description}


\paragraph{간호인력 확보수준에 따른 입원료 차등제 산정기준 위반청구}
\textbf{간호인력을 실제 근무사실과 다르게 신고}
\begin{description}\tightlist
\item[관련근거] 
\begin{itemize}\tightlist
\item 건강보험 행위 급여\cntrdot{}비급여목록표 및 급여상대가치점수 제3부 요양병원 산정지침 4.마. '간호인력 확보수준에 따른 입원료 차등제'
\item ※ \highlight{간호사 비율이 간호인력의 3분의 2이상인 경우 1일당 2,000원을 별도산정}
\item 간호인력 확보수준에 따른 요양병원 입원료 차등적용 관련 기준(보건복지부고시 제2009-214호(요양병원), 2010.4.1.)
\end{itemize}

\item[부당사례]
\begin{enumerate}[1)]\tightlist
\item A요양병원은 [표] 간호인력 근무현황과 같이 병동에 전속되어 입원환자 간호업무를 전담하는 인력이 아닌 간호사 ㅇㅇㅇ 등을 실제 근무내역과 다르게 병동 전담 근무인력으로 신고 \highlight{’14년 4분기, ’15년 1분기 간호인력 확보수준이 2등급임에도 1등급으로 적용하여 요양병원 입원료 청구}
\end{enumerate}
\end{description}

