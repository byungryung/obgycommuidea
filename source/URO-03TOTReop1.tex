\section{TOT후 재수술시 청구방법}
\myde{}{
\begin{itemize}\tightlist
\item[\dsjuridical] N393 스트레스요실금
\item[\dsjuridical] R300 배뇨통
\item[\dsjuridical] R391 기타 배뇨곤란
\item[\dsjuridical] T814 달리 분류되지 않은 처치에 따른 감염
\item[\dsmedical] SB023000 \myexplfn{349.22} 원 : 술후 간단한 테이프 길이 조절
\item[\dsmedical] M0031000 \myexplfn{946.42} 원 : 술후 15일 이내에 절단 혹은 제거시 %3개월 미만의 재수술
\item[\dsmedical] R3565 50\%요실금수술. 가. 질강을 통한 수술 (2) 기타의 경우 [\myexplfn{3477.58} 50\% : 15일 초과후 절단 혹은 제거시 %3개월 이상
%\item[\dsmedical] 
\end{itemize}
청구메모>> TOT후 염증, 테이프노출, 통증및 배뇨장애 등의 합병증으로 재수술시행함.
}
{
○ 인조테이프를 이용한 요실금 수술 후 염증, 테이프 노출, 통증 및 배뇨장애 등의 합병증으로 재수술하는 경우 수가산정 방법은 수술의 난이도 등을 감안하여 다음과 같이 적용키로 함.\par
\begin{center}\emph{- 다 음 -}\end{center}15
\begin{enumerate}[1)]\tightlist
\item 수술 후 테이프의 길이 조절만 시행하는 경우
: 자2 창상봉합술 나. 안면과 경부 이외 (1) 단순봉합 (가) 제1범위 3) 길이 5.0cm 이상이거나, 근육에 달하는 것으로 산정함.
\item 수술 후 15일 이내에 절단 혹은 제거하는 경우 \textcolor{red}{SB023000} \myexplfn{349.22} 원
: 자3 피부 및 피하조직, 근육내 이물제거술 가. 근막절개하 이물제거술로 산정함. \textcolor{red}{M0031000} \myexplfn{946.42} 원 
\item 수술 후 15일 초과 후 절단 혹은 제거하는 경우
: 자356 요실금 수술 가. 질강을 통한 수술 (2) 기타의 경우(요실금 수술 소정점수)의 50\%로 산정함.\textcolor{red}{R3565000} \myexplfn{   } 원
\end{enumerate}
(2009.10.1 시행)
}