\section{Triamcinolone acetonide 주사제(품명: 트리암시놀론 주 등)}
\myde{}{
\begin{itemize}\tightlist
\item[\dsjuridical] L639 상세불명의 원형탈모증 
\item[\dsmedical] N0091 가. 25㎠ 미만 \myexplfn{139.61} 원   
\item[\dsmedical] N0092 나. 25㎠ 이상 \myexplfn{229.50} 원  
%\item[\dschemical] 
\end{itemize}
}
{
\leftrod{급여기준}\par
\begin{enumerate}[1.]\tightlist
\item 허가사항 범위 내에서 투여 시 요양급여 함을 원칙으로 함.
\item 허가사항 범위(효능ㆍ효과, 용법ㆍ용량)을 초과하여 아래와 같은 기준으로 투여한 경우에도 요양급여를 인정함.
\end{enumerate}
\begin{center}\textbf{- 아 래 -}\end{center}
\begin{enumerate}[가.]\tightlist
\item 원형탈모증 및 켈로이드반흔 상병 등에 병변 내 주입 시 환자상태에 따라 1일 40mg 이내로 투여한 경우
\item 추간관절차단(Facet joint block/injection)시는 1 level당 20㎎으로 최대 3 level(60mg)까지, 양측은 각각 최대 2 level(80mg)까지 인정
\end{enumerate}
}


\subsection{N0092  병변내주입요법(25㎠ 이상) 행위정의 적응증}
습진성병변, 결절성질환, 손발톱의 이상, 사마귀,원형탈모증, 켈로이드(특히 통증이나 가려움증 등 증상이 있거나, 운동장애가 있는 경우) 등 바이러스 질환, 종양 등 직접 병변에 약물 투여가 필요한 경우\par
\leftrod{실시방법}
\begin{enumerate}[1.]\tightlist
\item 약물의 주입이 필요한 부위를 도안하거나 표시한다.
\item 필요한 약물을 주사기에 희석한다. (필요한 농도를 계산 또는 배합)
\item 근육 주사와 달리 약물에 따라 혈관을 피하거나 연조직을 피하는 방법이 필요한 경우도 있음. 병변내 주사시에 피부위축, 색소침착, 이차감염, 피부괴사 등을 고려하여야 한다.
\end{enumerate}