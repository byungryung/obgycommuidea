\section{임신확인을 위한 진료의 보험급여여부}
\myde{}{
\emph{생리적무월경시} : 비보험진료\par
\emph{병적무월경시}
\begin{itemize}\tightlist
%\item[\dsjuridical] O300 쌍둥이임신
%\item[\dsjuridical] Z372 쌍둥이, 둘 다 생존 출생
\item[\dschemical] 요임신반응검사-[일반면역검사](정성) \sout{B0260} D5701(1780원)
%\item[\dschemical] 요임신반응검사[정량] D5702(5800원)
\end{itemize}
}
{
\sout{병적 무월경이 아닌 \textcolor{red}{생리적인 무월경 상태(수유 무월경, 임신 무월경 등)에서 임신확인만을 위해 내원한 경우}에는 산전진찰의 범주로 볼수 없으며, 국민건강보험요양급여의기준에관한규칙[별표2] 비급여대상 3호 가목에 의한 건강검진의 범주에 해당되므로 \textcolor{red}{비급여}함.}\par
삭제(고시 제2016-190호,2016.10.1시행) 삭제사유 : 초음파 검사 급여기준 개정에 따른 임신확인을 위한 진료(임신 확인을 위한 검사 포함)의 급여 전환
\begin{enumerate}\tightlist
\item 본인이 원해서 하는 임신확인 검사는 여전히 건강검진이므로 비급여 검사입니다
\item 2차성무월경(PCOD 등) 같은 질환 감별을 위한 임신확인 검사는 급여입니다
\item 임상의가 임신 확인을 해야 할 필요가 있다고 판단되면 급여로 검사해야 합니다.(2016년 10월부터-)
예)출혈이 있을때 임신과 관련된 출혈 가능성이 있는 경우, Z32.0확인 안된 임신 상병 활용
\end{enumerate}
}
\par
\medskip
\Que{외과적 질환으로 신경외과에 내원하여 수술을 하게 되는경우 수술전 기본적 검사를 시행하고 혹여 임신일지 가능성을 염두하여 시행되는 임신반응검사의경우 급여 여부가 궁금합니다. 이런경우 건강검진의 범주에 해당되어 비급여로 구분되어지는건지 궁금합니다}
\Ans{요양급여는 가입자 등의 연령.성별.직업 및 심신상태 등의 특성을 고려하여 진료의 필요가 있다고 인정되는 경우에 정확한 진단을 토대로 하여 환자의 건강증진을 위하여 의학적으로 인정되는 범위안에서 최적의 방법으로 실시하여야 한다고 규정되어 있습니다.진료의가 환자의 상태 등에 따라 \textcolor{red}{진료상 필요하여 임신반응검사를 가임기여성에게 실시하였다면 요양급여 대상입니다.}}

\par
\medskip
\begin{commentbox}{병적무월경시 요임신반응검사의 급여}
31세 환자(P:1), 2개월동안 생리없어서(루프 했다는 말은 없었슴). 소변 임신 검사후 초음파후 질경을 넣고 진찰하니 루프가 자궁입구에 걸쳐있었슴 (totally out state). 설명후 루프제거 원하여 루프제거하였고, 피검사설명(금액까지 설명:FSH,E2,TSH, PRL,LH)후 피검사하였슴. 루프제거와 피검사는 보험처리하였고, 초음파와 임신 소변검사(만원)는 비보험처리하였슴\par

\begin{center}\emph{환자가 심평원에 진료비확인 요청}\end{center}\par
심평원답변(요임신검사에 대한 진료비 환수요청) \par
: 해당 수진자는 생리적인 무월경 상태에서 임신확인을 위해 내원한 경우가 아니라 루프를 삽입하고 있는 상태에서 속발성 무월경으로 진단받고 시행한 요 임신반응검사이므로 요양급여로 적용받는 것이 타당한 것으로 사료됨. \par
: 무월경에대한 피검사와 같이 시행된것으로 병적무월경검사로 보여짐
%--소변검사(비보험)에 대한 진료비 환수 결정
\end{commentbox}

