\section{자궁근종:Surgical myomectomy(DRG)}
\myde{}{%
\begin{itemize}\tightlist
\item[\dsjuridical] D250 자궁의 점막하 평활근종(Submucous leiomyoma of uterus)
\item[\dsjuridical] D251 자궁의 벽내 평활근종(Intramural leiomyoma of uterus) 
\item[\dsjuridical] D252 자궁의 장막하 평활근종(Subserosal leiomyoma of uterus)
\item[\dsjuridical] D259 상세불명의 자궁근종(Leiomyoma of uterus, unspecified)
\item[\dsmedical] N04700 기타 자궁 수술(악성종양제외) 
\item[\dsmedical] N04701 기타 자궁 수술(악성종양제외), 중증의 합병증이나 동반상병 동반
\item[\dsmedical] N04702 기타 자궁 수술(악성종양제외), 심각한 합병증이나 동반상병 동반
\item[\dsmedical] N04500 복강경을 이용한 기타 자궁 수술 
\item[\dsmedical] N04501 복강경을 이용한 기타 자궁 수술(악성종양제외), 심각한 합병증이나 동반상병 동반
\item[\dsmedical] 수태력보전에 대한가산, 입원일수에 대한 가산, 야간응급수술에 대한 가산.
\end{itemize}
}%
{
\begin{enumerate}\tightlist
\item 18시- 09시 또는 공휴일에 응급진료가 불가피하여 수술을 행한 경우에는 1호의 질병군별 점수에 다음과 같이 해당 질병군의 야간\bullet 공휴 소정점수를 추가
산정한다. 이 경우 수술 또는 마취를 시작한 시간을 기준으로 산정한다.
\item  임신\bullet 출산을 담당하는 장기의 병변 부위만을 제거\bullet 교정하는 수술을 하여 임신\bullet 출산능력을 보존한 경우  다만, \textcolor{red}{자궁내막증이 있거나 불임(또는 난임) 등으로 임신가능성을 높이기 위해 난소 또는 난관 전절제술을 실시한 경우는 사례별로 인정}
	\begin{enumerate}[가.]\tightlist	
	\item 임신\bullet 출산을 담당하는 장기의 수술을 동시에 실시하여 그 수술결과로 임신\bullet 출산능력이 보존된 경우
	\item 아래의 경우는 가산점수를 산정하지 아니함
		\begin{enumerate}[(1)]\tightlist
		\item  폐경 또는 55세 이상 여성(55세 이상이나 폐경이 아닌 경우 관련자료 첨부시 이를 참조하여 인정)
		\item  기존에 시행한 수술로 임신\bullet 출산 능력을 상실한 경우
		\end{enumerate}
	\end{enumerate}
\item 질병군별 점수는 가입자 등의 입원일수에 따라 다음과 같이 정상군, 하단 및 상단열외군으로 구분하여 그 총합을 산정한다. 이 경우 고정비율과 평균
입원일수, 정상군 하한 및 상한 입원일수는 제3호와 같다.	
\end{enumerate}
}

\clearpage
\tabulinesep =_2mm^2mm
\begin {tabu} to 1.02\linewidth {|X[.9,l]|X[2.9,l]|X[1,l]|X[1,l]|X[1,l]|X[1,l]|X[1,l]|X[.9,1]|} \tabucline[.5pt]{-}
\rowcolor{ForestGreen!40} 분류 & 명 칭 &	\centering 점 수 & 금액(원) & 가산 & 금액 & 야간 공휴 & 금액 \\ \tabucline[.5pt]{-}
\rowcolor{Yellow!40} N04500 & 복강경을 이용한 기타 자궁 수술(악성종양제외),심각한 합병증이나 동반상병 미동반 & 32,384.51 & 2,299,300 & 39,185.26 & 2,782,150 & 3,616.13 & 256,750 \\ \tabucline[.5pt]{-}
\rowcolor{Yellow!40} N04501 & 복강경을 이용한 기타 자궁 수술(악성종양제외),심각한 합병증이나 동반상병 동반 & 37,207.89 & 2,641,760 & 44,463.43 & 3,156,900 & 3,616.13 &  
 256,750 \\ \tabucline[.5pt]{-}
\rowcolor{Yellow!40} N04700 & 기타 자궁 수술(악성종양제외),심각한 혹은 중증의 합병증이나 동반상병 미동반 & 18,347.46 & 1,302,670 & 21,099.58 & 1,498,070 & 2,803.63 & 
 199,060 \\ \tabucline[.5pt]{-}
\rowcolor{Yellow!40} N04701 & 기타 자궁 수술(악성종양제외),중증의 합병증이나 동반상병 동반 & 23,824.93 & 1,691,570 & 27,398.68 & 1,945,310 & 2,803.63 & 199,060 \\ \tabucline[.5pt]{-}
\rowcolor{Yellow!40} N04702 & 기타 자궁 수술(악성종양제외),심각한 합병증이나 동반상병 동반 & 33,714.65 & 2,393,740 & 38,771.86 & 2,752,800 & 2,803.63 &199,060 \\ \tabucline[.5pt]{-}
\end{tabu}
\par
\medskip

\tabulinesep =_2mm^2mm
\begin {tabu} to\linewidth {|X[1,l]|X[1,l]|X[6,l]|X[1,l]|X[1,l]|} \tabucline[.5pt]{-}
\rowcolor{ForestGreen!40}  & 코드 &	\centering 분 류 & 점수 & 금액 \\ \tabucline[.5pt]{-}
\rowcolor{Yellow!40} 자-412 & &  자궁근종절제술 Myomectomy & & \\ \tabucline[.5pt]{-}
\rowcolor{Yellow!40} & & 가. 복부접근 Abdominal Approach & & \\ \tabucline[.5pt]{-}
\rowcolor{Yellow!40} & R4121 & (1) 단순 [장막하근종] Subserosal & 3,276.73 & \myexplfng{3276.73}  \\ \tabucline[.5pt]{-} %243,790 \\ \tabucline[.5pt]{-}
\rowcolor{Yellow!40} & R4122 & (2) 복잡 [근층내, 점막하, 인대간, 간질내, 복막하 근종이나, 결절 2개 이상인 다발성자궁근종인 경우에 산정] Complex & 3,444.62 & \myexplfng{3444.62}  \\ \tabucline[.5pt]{-} %256,280 \\ \tabucline[.5pt]{-}
\rowcolor{Yellow!40} &  R4123 &나. 질부접근 Vaginal Approach & 2,066.70 & \myexplfng{2066.70}  \\ \tabucline[.5pt]{-} %153,760  \\ \tabucline[.5pt]{-}
\rowcolor{Yellow!40}  자-412-1 & & 자궁경하 자궁근종절제술 Hysteroscopic Removal of Leiomyoma& & \\ \tabucline[.5pt]{-}
\rowcolor{Yellow!40}  & R4125 & 가. 3cm 미만 & 2,279.21 & \myexplfng{2279.21}  \\ \tabucline[.5pt]{-} %169,570 \\ \tabucline[.5pt]{-}
\rowcolor{Yellow!40}  & R4126 & 나. 3cm 이상 [다발성 포함] & 2,589.36 & \myexplfng{2589.36}  \\ \tabucline[.5pt]{-} %192,650 \\ \tabucline[.5pt]{-}
\end{tabu}


\subsection{자궁근종절제술-질부접근(R4123) 의 포괄수가제 포함여부}
자궁경부의 근종인경우 \highlight{외래에서 입원없이} 시행하는 자궁근종절제술-질부접근(R4123) 이 포괄수가제 포함 대상인가요?\par
포괄수가제 포함 대상인경우 행위별청구외에 추가로 다른 청구방법이 있나요?

\begin{commentbox}{}
질병군(DRG) 포괄수가는 국민건강보험법시행령 제21조제3항제2호에 따라 복지부장관이 별도 고시하는 7개 질병군으로 입원진료를 받은 경우에 적용하며, 질병군 입원진료는 질병군 급여 일반원칙에 따라 다음의 항목을 포함하고 있습니다.\par
- 다 음 -
\begin{itemize}\tightlist
\item 7개 질병군으로 응급실\bullet 수술실 등에서 수술을 받고 연속하여 6시간 이상 관찰 후 귀가 또는 이송한 경우 
\item 7개 질병군 중 수정체수술(대절개 단안 및 양안, 소절개 단안 및 양안), 기타항문수술, 서혜 및 대퇴부탈장수술 단측 및 양측(복강경 이용 포함)의 수술을 받고 6시간 이상 관찰 후 당일 귀가 또는 이송한 경우
\end{itemize}
따라서 자궁근종으로 자궁근종절제술-질부접근(R4123)을 받고 6시간 미만 관찰 후 당일 귀가 또는 이송한 경우(외래)는 \highlight{행위별청구대상}임을 알려드립니다.
\end{commentbox}
