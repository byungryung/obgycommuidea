\section{의료기관의 제증명수수료 항목 및 금액에 관한 기준}
\emph{(제4조제2항 관련)}
\par
\medskip
\tabulinesep =_2mm^2mm
\begin {tabu} to\linewidth {|X[1,l]|X[3,l]|X[7,l]|X[2,l]|} \tabucline[.5pt]{-}
\rowcolor{ForestGreen!40} 연번 & 항 목 & 기 준 & 상한금액(원) \\ \tabucline[.5pt]{-}
\rowcolor{Yellow!40} 1 & 일반진단서 & 의료법 시행규칙 [별지 제5호2서식]에 따라 의사가 진찰하거나 검사한 결과를 종합하여 작성한 진단서를 말함 & 20,000 \\ \tabucline[.5pt]{-}
\rowcolor{Yellow!40} 10 & 상해진단서(3주미만) & 의료법 시행규칙 [별지 제5호의 3서식]에 따라 질병의 원인이 상해(傷害)로 상해진단기간이 3주 미만일 경우의 진단서를 말함 & 100,000 \\ \tabucline[.5pt]{-}
\rowcolor{Yellow!40} 11 & 상해진단서(3주이상) & 의료법 시행규칙 [별지 제5호의 3서식]에 따라 질병의 원인이 상해(傷害)로 상해진단기간이 3주 이상일 경우의 진단서를 말함 & 150,000 \\ \tabucline[.5pt]{-}
\rowcolor{Yellow!40} 12 & 영문 일반진단서 & 의료법 시행규칙 [별지 제5호2서식]에 따라 의사가 영문으로 작성한 ‘일반 진단서’를 말함 & 20,000 \\ \tabucline[.5pt]{-}
\rowcolor{Yellow!40} 13 & 입퇴원확인서 & 환자의 인적사항(성명, 성별, 생년월일 등)과 입퇴원일을 기재하여, 입원사실에 대하여 행정적으로 발급하는 확인서를 말함 (입원사실증명서와 동일) & 3,000 \\ \tabucline[.5pt]{-}
\rowcolor{Yellow!40} 14 & 통원확인서 & 환자의 인적사항(성명, 성별, 생년월일 등)과 외래 진료일을 기재하여, 외래진료사실에 대하여 행정적으로 발급하는 확인서를 말함 & 3,000 \\ \tabucline[.5pt]{-}
\rowcolor{Yellow!40} 15 & 진료확인서 & 환자의 인적사항(성명, 성별, 생년월일 등)과 특정 진료내역을 기재하여, 특정 진료사실에 대하여 행정적으로 발급하는 확인서를 말함 (방사선 치료, 검사 및 의약품 등) & 3,000 \\ \tabucline[.5pt]{-}
\rowcolor{Yellow!40} 18 & 출생증명서 & 의료법 시행규칙 [별지 제7호의 서식]에 따라 의사 또는 조산사가 작성하는 태아의 출생에 대한 증명서를 말함 & 3,000 \\ \tabucline[.5pt]{-}
\rowcolor{Yellow!40} 21 & 사산(사태)증명서 & 의료법시행규칙 [별지 제8호의 서식]에 따라 의사 또는 조산사가 작성한 태아의 사산(死産) 또는 사태(死胎)에 대한 증명서를 말함 & 10,000 \\ \tabucline[.5pt]{-}
\rowcolor{Yellow!40} 22 & 입원사실증명서 & 환자의 인적사항과 입원일이 기재되어 있는 확인서로 입퇴원확인서 금액기준과 동일함 &  3,000 \\ \tabucline[.5pt]{-}
\rowcolor{Yellow!40} 25 & 진료기록사본 (1-5매) & 의료법 시행규칙 제15조제1항에 따른 진료기록부 등을 복사하는 경우를 말함(1~5매까지, 1매당 금액) & 1,000 \\ \tabucline[.5pt]{-}
\rowcolor{Yellow!40} 26 & 진료기록사본 (6매 이상) & 의료법 시행규칙 제15조제1항에 따른 진료기록부 등을 복사하는 경우를 말함(6매부터, 1매당 금액) & 100 \\ \tabucline[.5pt]{-}
\rowcolor{Yellow!40} 27 & 진료기록영상(필름) & 방사선단순영상, 방사선특수영상, 전산화단층영상(CT) 등 영상 자료를  필름을 이용하여 복사하는 경우를 말함 & 5,000 \\ \tabucline[.5pt]{-}
\rowcolor{Yellow!40} 28 & 진료기록영상(CD) & 영상진단, 내시경사진, 진료 중 촬영한 신체부위 등 영상 자료를 CD를 이용하여 복사하는 경우를 말함 & 10,000 \\ \tabucline[.5pt]{-}
\rowcolor{Yellow!40} 29 & 진료기록영상(DVD) & 영상진단, 내시경사진, 진료 중 촬영한 신체부위 등 영상 자료를 DVD를 이용하여 복사하는 경우를 말함 & 20,000 \\ \tabucline[.5pt]{-}
\rowcolor{Yellow!40} 30 & 제증명서 사본 & 기존의 제증명서를 복사(재발급)하는 경우를 말함(동시에 동일 제증명서를 여러통 발급받는 경우 최초 1통 이외 추가로 발급받는 제증명서도 사본으로 본다) & 1,000 \\ \tabucline[.5pt]{-}
\end{tabu}
\par
\medskip
주) 상한금액은 진찰료 및 각종 검사료 등 진료비용을 포함하지 않음