\section{기형아를 정밀계측한다는 의미는?}
\Large EB514010 \normalsize \menu{First Trimester > Advanced > Abnormaly를 정밀 계측하는 경우}\par
\noindent\Large EB518010 \normalsize \menu{Second, Third Trimester > Advanced > Abnormal를 정밀 계측하는 경우}\par
\noindent\Large EB436 \normalsize \menu{나943 > Detailed Fetal Echocardiography}
\prezi{\clearpage}
%\medskip 
\begin{commentbox}{정밀 초음파의 기형아(Anomaly) 수가 적용 및 추적검사}%범위는 ?}
임산부 정밀 초음파로 \highlight[cyan!70]{태아정밀계측 실시 중 태아 기형이 확인되어 기형과 관련된 추가적인 계측을 시행한 경우, 또는 타 검사 등으로 기형 여부가 확진되어 기형아 정밀계측을 시행한 경우에 산정함}
\begin{itemize}\tightlist
	\item 임신 제 1삼분기 정밀 초음파 검사 중 우연히 태아 기형을 발견하여 정밀 계측을 한 경우: EB513 → EB514 처방코드 변경
	\item 임신 제 2,3삼분기 정밀 초음파 검사 중 우연히 태아 기형을 발견하여 정밀 계측을 한 경우: EB517 → EB518처방코드 변경
	\item 산전진찰 결과 태아의 심장에 이상소견이 있어 정밀검사를 시행하는 경우 산정, 나943다 (EB436) 태아정밀 심초음파 Detailed Fetal Echocardiography, 3,241.07를 산정한다.
	\item 정밀 초음파(또는 기형아 정밀계측 초음파)에서 태아 심장 이상소견이 발견되어 태아정밀 심초음파검사를 한 경우: 정밀초음파 검사로 태아 심장 이상소견이 발견되어 태아정밀 심초음파를 각각 별도로 시행한 경우에는 두 초음파 각각 별도로 수가 산정 가능 (산부인과 또는 소아심장과 시행 여부 상관 없음) 
	\item 태아심장이상 추적검사시에도 EB436001 제한적 심초음파로 하면 됩니다. (7회예외 조항)
	\item 기형아 정밀 계측 초음파 검사 후 F/U 검사하는 경우: 일반 초음파 확인 사항 모두 포함하여 태아 기형 F/U 하는 경우는 해당 삼분기의 일반 초음파, 해당 태아 기형만 F/U 하는 경우 해당 삼분기의 일반의 제한적 초음파로 산정(고시 또는 \textsf{Q\&A}에 없는 내용)
\end{itemize}
\end{commentbox}
\prezi{\clearpage}
\begin{commentbox}{}
기형이 진단된 이후의 초음파는 EB511010/EB515010 만되며 7회이상 제한은 없습니다
\end{commentbox}
\prezi{\clearpage}

\subsection{태아 기형의 구체적인 기준을 정하지는 않았으나 아래에 해당하는 경우 산정 가능 (고시 또는 \textsf{Q\&A}에 없는 내용) by 산부인과학회}
\begin{minipage}{.95\textwidth}
\begin{enumerate}[①]\tightlist
	\item 다태임신으로 인한 합병증: 결합쌍태아, 무심장쌍태아, 태아간 수혈증후군, 태아간 20\% 이상의 체중 차이가 있는 불균형 쌍태아, 일측태아사망, 단일양막성쌍태아등
	\item 이상소견을 가진 태아: 주기형 (major anomaly)\footnote{생명에 지장이 있거나 수술 등 적극적인 치료를 해야 생존이 가능하거나 심각한 형태의 이상으로 인해 정상적인 생활이 어려운 경우} 및 부기형 (minor anomaly)\footnote{주기형을 제외한 나머지 기형, 생명에는 지장이 없고, 정상적인 생활이 어려울 정도의 심각한 기형은 아니라 육안적으로 이상이 보이는 경우 (예) lack of earlobe, middle 5th finger clinodactyly}이 있거나 다발성 기형인 경우
	\item 양수검사, 융모막검사, 제대혈검사에서 염색체 이상이 확인된 경우
	\item 원인 미상의 양수과다증, 양수과소증
	\item 염색체 이상의 Minor or soft marker (nuchal fold thickness포함) 는 제외
\end{enumerate}
\end{minipage}
\prezi{\clearpage}
\subsection{산부인과학회 세부급여기준 Q\&A}
\begin{itemize}\tightlist
\item 임신 12주 NT가 증가되어 있었으나 융모막검사 또는 양수검사에서 정상이었고, 임신 20주에 정밀초음파 검사하는 경우
	\begin{enumerate}[①]\tightlist
	\item 구조적 이상이 없는 경우 : 제 2,3삼분기 정밀 초음파 (EB517)
	\item 심장 기형 등 구조적 이상이 발견된 경우 : 제 2,3삼분기 기형아 정밀계측 초음파 (EB518)
	\end{enumerate}
\item 임신 제 1삼분기 정밀 초음파 검사 중 우연히 태아 기형을 발견하여 정밀 계측을 한 경우: EB513 → EB514 처방코드 변경 \par
\emph{예: 임신 12주에 NT를 측정한 경우}
	\begin{enumerate}[①]\tightlist
	\item  NT가 정상이고 다른 이상도 없는 경우 : 제 1삼분기 정밀 초음파 (EB513)
	\item  NT가 비정상으로 증가되어 있거나 태아기형(예: anencephaly, gastroschisis 등)이 관찰된 경우 : 제 1삼분기 기형아 정밀계측 초음파 (EB514)
	\end{enumerate}
\item 임신 제 2,3삼분기 정밀 초음파 검사 중 우연히 태아 기형을 발견하여 정밀 계측을 한 경우: EB517 → EB518 처방코드 변경\par
\emph{예: 임신 20주에 정밀초음파 검사하는 경우}
	\begin{enumerate}[①]\tightlist
	\item  구조적 이상이 없는 경우 : 제 2,3삼분기 정밀 초음파 (EB517)
	\item  심장 기형 등 구조적 이상이 발견된 경우 : 제 2,3삼분기 기형아 정밀계측 초음파 (EB518)
	\end{enumerate}
\end{itemize}
\prezi{\clearpage}	
\begin{commentbox}{Hydronephrosis}
는 교과서마다 임신주수마다 기준의 차이가 있는데 어떻게 정의하나요? \par
일반적으로 hydronephrosis는 renal pelvis의 anteroposterior diameter가 10 mm 이상인 경우 태아 기형에 포함됩니다. Mild pelvic dilatation(pyelectasis, pelviectasia)의 경우에는 태아기형에 포함되지는 않으나 추적 관찰 시 hydronephrosis로 진행할 가능성이 있으므로 횟수 초과 급여 가능한 태아 이상에는 포함
될 수 있습니다. 그 진단 기준은 다양하지만, 교과서에 있는 기준 중 어느 하나라도 만족하면 나중에 근거자료로 제출할 수 있으니 어느 기준에든 부합하면 가능할 것으로 생각이 됩니다.
\end{commentbox}
\prezi{\clearpage}
\leftrod{Quad 검사에서 high risk, NIPT 검사에서 이상소견시 기형초음파 가능 여부}
\textcolor{red}{EB514 \& EB518 제외}\par
\prezi{\clearpage}
Ventriculomegaly는 어떻게 정의하나요?
\begin{quotebox}
Lateral ventricle이 10 mm 이상인 경우 태아 기형에 포함됩니다. 
\end{quotebox}
\prezi{\clearpage}
\subsection{적절한 상병 코드}
	\begin{itemize}\tightlist
	\item \highlight{O283 산모의 출산전 선별검사의 초음파 이상소견}
	\item O35 알려진 또는 의심되는 태아 이상 및 손상에 대한 산모관리
	\item O350.태아의 (의심되는) 중추신경계통기형에 대한 산모관리
	\item O351.태아의 (의심되는) 염색체이상에 대한 산모관리
	\item O352.태아의 (의심되는) 유전성 질환에 대한 산모관리
%	\item O35.3.모체의 바이러스병으로 인한 (의심되는) 태아손상에 대한 산모관리
%	\item O35.4.알콜로 인한 (의심되는) 태아손상에 대한 산모관리
%	\item O35.5.약물로 인한 (의심되는) 태아손상에 대한 산모관리
%	\item O35.6.방사선으로 인한 (의심되는) 태아손상에 대한 산모관리
%	\item O35.7.기타 의학적 처치로 인한 (의심되는) 태아손상에 대한 산모관리
%	\item O35.8.기타 (의심되는) 태아이상 및 손상에 대한 산모관리
	\item \highlight{O359 상세불명의 (의심되는) 태아 이상 및 손상에 대한 산모관리}
	\item O280 산모의 출산전 선별검사의 혈액학적 이상 소견
	\item O285 산모의 출산전 선별검사의 염색체및 유전성 이상소견
	\item O430 태아간 수혈증후군
	\item O311 일측태아사망
	\item O40 양수과다증
	\item O410 양수과소증
	\end{itemize}
\prezi{\clearpage}	
\subsection{심기형 관찰시}
\leftrod{태아에서 심기형이 관찰 되었습니다. 무엇으로 청구해야 하나요?}
EB436, 다.  태아정밀 심초음파 Detailed Fetal Echocardiography, \myexplfn{3241.07} 원
태아의 심기형이 있는경우는 숫가가 좀더 좋은 태아심장 심초음파로 청구해야  합니다만, ...
\leftrod{태아 정밀 심초음파는 7회 횟수에 포함이 되나요?}
산모초음파 7회 보험 횟수에 포함되지 않습니다 
\leftrod{. 정밀 초음파(또는 기형아 정밀계측 초음파)에서 태아 심장 이상소견이 발견되어 태아정밀 심초음파 검사를 한 경우는 두번 청구 가능한가요?}
두 초음파 각각 별도로 수가 산정 가능 (산부인과 또는 소아심장과 시행 여부 상관 없음) (고시 또는 Q\&A에 없는 내용)
