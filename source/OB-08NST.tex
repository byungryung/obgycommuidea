\section{비자극검사}
\begin{paracol}{2}
\setlength{\columnseprule}{0.4pt}
\setlength{\columnsep}{2em}
\begin{leftcolumn}
\begin{commentbox}{}
\begin{itemize}\tightlist
\item[\dsmedical] 나732-1 (E7325) 
\end{itemize}
\end{commentbox}
%\medskip
%\centering

%\includegraphics[width=0.75\linewidth]{labial-fusion}
\end{leftcolumn}

\begin{rightcolumn}
임신 24주이상 임부에서 실시한 경우 입원,외래 불문하고 1회만 인정하며, 1회를 초과하여 시행한 경우에는 전액 본인부담토록함. 다만 35세 임부에 한하여 1회 추가인정\par
\noindent통상적인 태동시기를 고려하여 \uline{임신24주 이후에 실시한 경우에만 인정함.}
\end{rightcolumn}
\end{paracol} 
\subsection{comments}
\begin{itemize}\tightlist
\item 비자극검사 나732-1 : 자궁수축이 없는 상태에서 태아심박동수 측정으로 태아안녕평가
\item 분만전감시-전자태아감시 나732-가: 분만진통중 자궁수축에 따른 태아심박동의 변화관찰
\item 태아심음자궁수축검사 : 조기진통 및 유도분만을 시도하는 산모에게 자궁수축의 강도와 빈도 측정. 임신 16주이후, 자궁수축여부 확인을 위해 실시함.
\end{itemize}