\section{자궁내막증}
\myde{}{
\begin{itemize}\tightlist
\item[\dsjuridical] N80	자궁내막증	Endometriosis
\item[\dsjuridical] N80.0	자궁의 자궁내막증	Endometriosis of uterus
\item[\dsjuridical] N80.1	난소의 자궁내막증	Endometriosis of ovary
\item[\dsjuridical] N80.3	골반복막의 자궁내막증	Endometriosis of pelvic peritoneum
\item[\dsjuridical] N80.9	상세불명의 자궁내막증	Endometriosis, unspecified
%\item[\dsmedical] 
%\item[\dschemical] 
\end{itemize}
}
{
\leftrod{GnRH 주사제 급여기준입니다. 고시 제2013-127호}\par
1. 허가사항 범위 내에서 아래와 같은 기준으로 투여 시 요양급여를 인정하며, 동 인정기준 이외에는 약값 전액을 환자가 부담토록 함. 
\begin{center}\textbf{- 아 래 -}\end{center}
\begin{enumerate}[가.]\tightlist
\item 급여대상품목 : 허가받은 GnRH (Gonadotropin-Releasing Hormone) Analogue 계열 제제(성분명:Buserelin, Goserelin, Leuprorelin, Nafarelin, Triptorelin 등) 
\item 급여범위 : 각 약제 허가사항 범위 내에서 다음과 같이 인정함. 
	\begin{enumerate}[1)]\tightlist
	\item \uline{자궁내막증 : 복강경 검사등으로 확진된 경우} 
	\item 중추성사춘기조발증 
		\begin{enumerate}[가)]\tightlist
		\item 투여대상 
			\begin{enumerate}[(1)]\tightlist
			\item 이차성징성숙도(Tanner stage) 2이상이면서 골연령이 해당 역연령보다 증가되고, 
			\item GnRH(생식샘자극호르몬분비호르몬) 자극검사에서 황체형성호르몬(LH)이 기저치의 2-3배 증가되면서 최고 농도는 5 IU/L 이상인 경우. 
			\end{enumerate}
		\item 투여기간
			\begin{enumerate}[(1)]\tightlist
			\item 투여시작: 역연령 여아 9세(8세 365일), 남아는 10세(9세 365일) 미만 
			\item 투여종료: 역연령 여아 11세(11세 364일), 남아는 12세(12세 364일)까지 
			\end{enumerate}
		\end{enumerate}
	\end{enumerate}
\end{enumerate}

\leftrod{Dienogest 경구제 (품명:비잔정)}
고시 제2013-127호\par
허가사항 범위 내에서 \uline{복강경 검사 등으로 자궁내막증이 확진된 경우}에 투여 시 요양급여를 인정하며, 동 인정기준 이외에는 약값 전액을 환자가 부담토록 함.
}

\subsection{로렐린 주사}
\textbf{효능/효과}
\begin{enumerate}[1.]\tightlist
\item 자궁내막증. 
\item 과다월경, 하복통, 요통 및 빈혈 등을 수반한 자궁근종에서 근종핵의 축소 및 증상의 개선. 
\item 전립선암. 
\item 폐경전 유방암. 
\item 중추성 사춘기조발증.
\end{enumerate}

\textbf{용법/용량}
\begin{enumerate}[1.]\tightlist
\item 자궁내막증 : 보통 성인에는 4주 1회 류프로렐린아세트산염으로서 3.75 mg을 피하주사한다. 단, 체중이 50 kg 미만의 환자에게는 1.88 mg을 투여한다. 또한 초회 투여는 월경주기 1∼5일째에 한다. 
\item 자궁근종 : 보통 성인에는 4주 1회 류프로렐린아세트산염으로서 1.88 mg을 피하주사한다. 단, 체중과다 환자, 자궁종대가 고도인 환자에는 3.75 mg을 투여한다. 또한 초회투여는 월경주기 1∼5일째에 한다. 
\item 전립선암, 폐경전 유방암 : 보통 성인에는 4주 1회 류프로렐린아세트산염으로서 3.75 mg을 피하주사한다. 
\item 중추성 사춘기조발증 : 보통 4주 1회 류프로렐린아세트산염으로서 체중 kg당 30 μg을 피하주사한다. 또한 증상에 따라 체중 kg당 90 μg까지 증량할 수 있다.
\end{enumerate}

%\begin{commentbox}{}

%\end{commentbox}