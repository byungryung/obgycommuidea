\section{도플러 10\% 가산}%제 5 절 초음파 검사료
\begin{enumerate}[1.]\tightlist
\item 만 8 세 미만의 소아에 대하여는 소정점수의 20\% 를 가산한다(산정코드 첫 번째 자리에 3 으로 기재).
\item \uline{도플러 검사를 실시한 경우 소정점수의 10\% 를 가산한다( 산정코드 두 번째 자리에 1 로 기재 )}. 다만 , 「나 -940」 , 「나 -943」 , 「나 -948」 , 「나 -952」 , 「나 -956」 , 「나 -961」 은 소정점수에 포함되어 있으므로 그러하지 아니한다 .\highlight{태아정밀심초음파는 도플러 가산에서 제외됨}
\item 조영제를 사용하여 검사한 경우 소정점수의 30\% 를 가산하고( 산정코드 두 번째 자리에 2 로 기재 ), 검사 시 사용된 조영제는 별도 산정한다 .
\end{enumerate}

\par
\medskip
\prezi{\clearpage}
\Que{임신확인 초음파 검사 시 도플러 가산이 가능한가요?} 
\Ans{제 1삼분기 임신확인 초음파(EB512)에서 자궁 내 정상 임신낭 또는 난황만 확인이 되는 경우에는 도플 러 가산 산정할 수 없습니다.  \highlight{도플러 가산은 비정상 임신(예: 자궁 외 임신, 유산, 출혈 등)인 경우 또는 의심되는 경우 진단을 위해 도플러 검사를 한 경우}에 적용 가능합니다. }
\prezi{\clearpage}
\Que{도플러 가산은 언제 가능한가요?} 
\Ans{태아심장박동 확인은 초음파 행위정의에 포함되어 있으므로 \highlightR{단순 태아심박동 확인을 위한 도플러 검사는 도플러 가산에 해당되지 않습니다.} 아래와 같이 태아 또는 태반의 이상을 진단하기 위한 목적으로 혈관 및 장기에서 도플러 검사를 한 경우에 (color, power, M-mode, P-W mode 상관 없이) 도플러 가산 적용이 가능합니다. 
\begin{enumerate}[①]\tightlist
\item 선천성 태아 기형 진단을 위한 도플러 검사
\item 비정상 태반 또는 탯줄 진단을 위한 도플러 검사
\item 자궁내 태아성장지연, 양수과소증, 전자간증 등에서 탯줄동맥, 탯줄정맥, 중대뇌동맥, 정맥관, 자궁동맥 등의 혈관에서 혈류 및 파형을 분석하기 위한 도플러 검사 
\item 기타 태아, 태반, 탯줄, 자궁 등의 혈관 또는 혈류 이상을 진단하기 위한 도플러 검사 
\end{enumerate}
}