\section{임산부 일반초음파 QNA}
\subsection{임신 제1삼분기는 언제인가요?}
착상부터 임신 13주까지(13주 6일까지)
\begin{itemize}\tightlist
\item 문제점은 현 KCD-7으로 상병명을 코딩한 경우는 
	\begin{itemize}\tightlist
	\item Z3400 // Z3480 : 기타 정상임신의 관리, 임신 22주 미만
	\item Z3401 // Z3481 : 기타 정상임신의 관리, 임신 22주 이상 - 34주 미만
	\item Z3402 // Z3482 : 기타 정상임신의 관리, 임신 34주 이상 으로 상병명으로 3분기를구분할 수 없다.
	\end{itemize}
\item 그래서 청구시 주수를 적어야 합니다.
\end{itemize}
JT005에 초음파검사시 임신주수를 기재함 → 청구 프로그램 회사에 의뢰
%\end{itemize}
\par
\medskip
\Que{임신 1삼분기에 일반 초음파만 본 경우 3번을 청구할 수 있나요?}
\Ans{없습니다. 2번을 초과한 경우 비급여 입니다. 따라서 \textcolor{red}{꼭 11-13주에}  NT를 측정하시고 정밀 초음파[EB513]을 1번 청구하시기 바랍니다. \par
NT 를 꼭 재시고 EB513로 청구, 만약 3mm 이상이면 EB514 가능합니다.}
\prezi{\clearpage}
\subsection{1분기 일반초음파<검사항목>}
\begin{enumerate}[①]\tightlist
\item 태아수 
\item 임신낭
\item 난황(Yolk sac)
\item 태아심장박동수 (Fetal heart rate)
\item 정둔장(Crown-rump length)또는 태아신체계측(Fetal biometry) : 태아신체계측 세부항목 : 양쪽마루뼈지름(Biperietal diameter), 대퇴골길이(Femur length), 복부둘레(Abdominal circumference)
\item 태반 및 제대평가 (Placenta and umbilical cord)
\item 양수양 적정성 평가(Amniotic Fluid)
\end{enumerate}
\begin{quotebox}
일삼분기에 도플러 가산을 위해선 단순히 태아심장박동수 체크만으론 안될것 같고,\highlight{uterine artery 도플러 추가로 보고 G sac 주위 칼라도플러도 보고, 차팅}을 해야 할것 같습니다.
\end{quotebox}
\prezi{\clearpage}
\subsection{1분기 정밀<검사항목>}
\begin{enumerate}[①]\tightlist
\item 1분기 일반초음파<검사항목>들
\item 태아 구조 screening 평가
	\begin{itemize}[-]\tightlist
	\item 머리와 목 (head \& neck)
	\item 심장과 흉부 (heart \& thorax)
	\item 복부 (abdomen)
	\item 비뇨생식계 (genito-Urinary system)
	\item 복벽 (abdominal wall defect)
	\item 척추 (spine)
	\item 사지 (extremities)
	\end{itemize}
\item 정확한 시상면에서의 태아목덜미투명대 (nuchal translucency)
\end{enumerate}
\emph{기형아 정밀계측에서는 2항에서 (기형이 있거나 기형이 의심되는 기관에 대한 정밀 평가 및 동반 기형 여부 판단을 위한 타 장기 평가)}
\prezi{\clearpage}
\subsection{2,3분기 일반초음파<검사항목>}
\begin{enumerate}[①]\tightlist
\item 태아수
\item 태아선진부
\item 태아심장박동 (fetal heart beat)
\item 태아신체계측(Fetal Biometry) : 아두대횡경(biparietal diameter), 복부둘례(abdominal circumference), 대퇴골길이(femur length)
\item 양수양(Measurement of amnionic fluid) : 단일최대길이(single deepest pocket) 또는 양수지수(amnionic fluid index :AFI)
\item 태반과 탯줄(placenta and umbilical cord)
\end{enumerate}
\prezi{\clearpage}
\subsection{2,3분기 정밀<검사항목>}
\begin{enumerate}[①]\tightlist
\item 2,3분기 일반초음파<검사항목>
\item 태아의 각 장기 및 신체부위별 검사 (parts of the fetus)
	\begin{itemize}[-]\tightlist
	\item 머리와 목 (head \& neck)
	\item 얼굴 (face)
	\item 심장과 흉부 (heart \& thorax)
	\item 복부 (abdomen)
	\item 비뇨생식계 (genito-urinary system)
	\item 복벽 (abdominal wall)
	\item 척추 (spine)
	\item 사지 (extremities)
	\end{itemize}
\end{enumerate}
\emph{기형아 정밀계측에서는 2항에서 (기형이 있거나 기형이 의심되는 기관에 대한 정밀 평가 및 동반 기형 여부 판단을 위한 타 장기 평가)}

