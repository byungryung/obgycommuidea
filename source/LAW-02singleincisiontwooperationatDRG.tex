\section{DRG에서 동일절개하에 다른 수술시 산정방법}
\myde{제왕절개시 난소수술 자궁적출술시 난소수술}
{
\begin{itemize}\tightlist
\item[\dsjuridical] D279\_00 난소의 양성신생물, 상세불명 쪽
\item[\dsmedical] R4421001 부속기종양적출술 양성 제2의 수술(행위별 수가로개별 청구, 50\%가능함.)
\end{itemize}
}
{
19. 질병군 진료시 질병군 분류번호를 결정하는 주된 수술 이외에 제1편 제2부제9장제1절(기본처치 제외) 또는 제10장제3절․제4절의 수술을 실시한 경우에는 해당 수술 소정점수를 추가 산정한다. 다만, 주된 수술과 동일 피부 절개 하에 실시되는 수술은 해당 수술 소정점수의 70\%를 산정 한다.
}

\subsection{질병군 분류번호를 결정하는 주된 수술 이외에 수술을 실시한 경우 수기료 추가 산정방법}
\emph{자궁적출술시 TOT등}\par
(보건복지부 고시 제2015 – 26호(2015년 1월 30일 )\par
질병군 분류번호를 결정하는 주된 수술 이외에 제1편제2부제9장제1절(기본처치 제외) 및 제10장제3절, 제4절의 수술을 실시한 경우의 추가 산정 방법은 다음과 같이 한다.\par
- 다 음 -
\begin{enumerate}[1.]\tightlist
\item 질병군 진료 중 질병군 분류번호를 결정하는 주된 수술과 날을 달리하여 실시하는 수술도 포함함
\item \textcolor{red}{해당 수술 항목의 소정점수만을 산정하고, 야간\bullet 공휴 가산 등을 포함한 모든 가산은 적용하지 아니함}
\item 아래의 경우는 추가 산정하지 아니함
	\begin{enumerate}\tightlist
	\item 합병증 혹은 처치 중의 우발적 천자 및 열상 등으로 실시한 수술
	\item 수정체수술 질병군과 동시에 실시한 유리체흡인술(자505), 유리체내주입술(자507), 유리체절제술-부분절제(자512-나)
	\item 편도절제술과 동시에 실시한 아데노이드절제술(내시경하에서 실시한 경우 포함)
	\item 기타 또는 주요 항문수술 질병군에 해당하는 수술을 2개 이상 실시한 경우
	\end{enumerate}
\item  위 1부터 3까지에서 정하고 있지 않은 내용은 「건강보험 행위 급여\bullet {} 비급여 목록표 및 급여 상대가치점수」 제1편 제2부 제9장, 제10장 및 「요양급여의 적용기준 및 방법에 관한 세부사항」Ⅰ. 행위 제9장, 제10장을 적용한다.
\end{enumerate}