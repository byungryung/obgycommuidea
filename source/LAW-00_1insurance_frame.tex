\section{국민건강보험의 이해: 각 직역별}
\subsection{보건복지부}
건강보험사업의 주관 (법 제2조)
•주관 내용
–국민건강보험종합계획 수립
–‘건강보험정책심의위원회’ 운영
–공단ㆍ심평원 및 요양기관 관리 감독
–요양급여대상 및 그 인정기준 결정
–행위별 상대가치점수, 약제ㆍ치료재료별 상한금액 결정
–요양기관 현지조사 및 업무정지(과징금) 처분
\begin{center}
\includegraphics[width=.9\textwidth]{NGW}
\end{center}
건강보험정책심의위원회 (국민건강보험법 제 4조)
–보건복지부장관 소속
–기능
•요양급여의 기준
•요양급여비용에 관한 사항
•직장가입자의 보험료율
•지역가입자의 보험료부과 점수당 금액
–구성: 위원장 1명과 부위원장 1명을 포함하여 25명의 위원
•위원장 : 보건복지부차관
•부위원장 : 공익위원 증에서 위원장이 지명
•위원
8명 근로자단체 및 사용자단체가 추천 각 2명,
시민단체, 소비자단체, 농어업인단체 및 자영업자단체가 추천하는 각 1명
8명 의료계를 대표하는 단체 및 약업계를 대표하는 단체가 추천하는 8명
8명 공무원 2명, 공단의 이사장 및 심평원의 원장이 추천하는 각 1명,
건강보험에 관한 학식과 경험이 풍부한 4명


\subsection{국민건강보험공단(NHIS)}
\begin{center}
\includegraphics[width=.9\textwidth]{NHIS}
\end{center}
건강보험의 보험자 (법 제13조)
•업무
–가입자 및 피부양자의 자격 관리
–보험료와 그 밖에 이법에 다른 징수금의 부과ㆍ징수
–보험급여의 관리
–가입자 및 피부양자의 건강 유지와 증진을 위하여 필요한 예방사업
–보험급여 비용의 지급
–자산의 관리ㆍ운영 및 증식사업
–의료시설의 운영
–건강보험에 관한 교육훈련 및 홍보
–건강보험에 관한 조사연구 및 국제협력
•이사회(15명)
–이사장
•보건복지부장관의 제청에 의하여 대통령이 임명
–이사
•6명: 노동조합, 사용자단체, 시민단체, 소비자단체, 농어업인단체, 노인단체에서 각 1명
•5명: 이사장 추천
•3명: 관계 공무원
•재정운영위원회(30명)
–요양급여비용의 계약체계, 직장가입자의 보수월액 산정방법 및 지역가입자의 보험료 부과점수 산정방법, 보험료 등의 결손처분 등을 심의 의결
–10명 : 직장가입자를 대표
–10명 : 지역가입자를 대표
–10명 : 공익을 대표

\subsection{건강보험심사평가원(HIRA)}
•심사 및 평가기관 (법 제62조)
•업무
–요양급여비용의 심사
–요양급여의 적정성 평가
–심사기준 및 평가기준의 개발
–관련 조사연구 및 국제협력
–다른 법률에 따른 급여비용 심사 및 적정성평가 위탁 업무
\begin{center}
\includegraphics[width=.9\textwidth]{HIRA}
\end{center}
•이사회(15명)
–원장
•대통령이 임면
–이사
•5명 의약관계단체 추천자
•1명 국민건강보험공단 추천자
•3명 건강보험심사평가원 추천자
•4명 노동조합, 사용자단체, 소비자단체, 농어업인 단체 각 1명
•1명 관계 공무원
•진료심사평가위원회
–1,090명 이내(상근 90명, 비상근 1000명)

\subsection{요양기관}
•요양기관 당연지정제 채택 (법 제42조)
–관련 법률에 의거 개설ㆍ등록된 의료기관 및 약국 등은 별도의 신청 또는 지정 절차 없이 당연 요양기관으로 지정되어 가입자 등의 질병, 부상, 분만에 대한 요양급여를 실시함
–민간의료기관이 대부분 (90\%이상) 인 우리나라 특수성 반영
•종별 요양기관 현황(2016년 기준)

\subsection{가입자}
•전국민 건강보험 가입 (법 제5조, 제6조)
–의료급여 수급권자, 유공자등 의료보호대상자를 제외한 국내에 거주하는 국민 모두가 가입자 또는 피부양자가 됨
–직장가입자(피부양자)와 지역가입자로 구분
•가입자별 대상자 현황(2016년 6월 기준)

\section{건강보험 재정 및 보험료}
건강보험 재원
•보험료 (법 제69조)
•국고지원금 (법 제108조)
•국민건강증진기금에서의 지원금
(국민건강증진법 부칙)
•기타 수입금
\begin{center}
\includegraphics[width=.9\textwidth]{MIincome}
\end{center}

\subsection{건강보험료 부과체계}

\subsubsection{가입자별 상이한 부과체계 (2016년)}
\tabulinesep =_2mm^2mm
\begin{tabu} to\linewidth {|X[1,l]|X[3,l]|X[5,l]|} \tabucline[.5pt]{-}
\rowcolor{ForestGreen!40}  구 분 & 직장가입자 & 지역가입자 \\ \tabucline[.5pt]{-}
\rowcolor{Yellow!40} 부과기준 & \begin{itemize}\tightlist \item 보수 * 사용자 : 사업소득 \item 보수 外 소득 \end{itemize} & \begin{itemize}\tightlist \item 연소득 > 500만원 초과 세대: 소득, 재산, 자동차 점수 합산 \item 연소득 ≤ 500만원 이하 세대: 평가소득*, 재산, 자동차 점수 합산 \item * 성․연령, 소득, 재산, 자동차로 산정\end{itemize} \\ \tabucline[.5pt]{-}
\rowcolor{Yellow!40} 산정방식 & \begin{itemize}\tightlist \item 보수월액 × 정률(6.12\%, ‘17) \item 보수 外 소득 연 7,200만원 초과 : 보수 外 소득의 3.06\% 추가로 부과 \end{itemize} & \begin{itemize}\tightlist \item 보험료 부과점수 × 점수당 금액(179.6원, ‘17) \end{itemize} \\ \tabucline[.5pt]{-}
\rowcolor{Yellow!40} 납부자 & ▪사용자 50\%, 근로자 50\% & ▪지역가입자 세대 100\% \\ \tabucline[.5pt]{-}
\rowcolor{Yellow!40} 최저‧최고 보험료(월) & ▪(본인부담) 8,560원~239만원\newline * 사용자부담 포함시 17,120원~478만원 & ▪3,590원~228만원\newline * 전액 본인부담 \\ \tabucline[.5pt]{-}
\end{tabu}