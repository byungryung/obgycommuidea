\section{임신성 당뇨검사} 
\myde{50g 경구 포도당부하검사}
{
\begin{itemize}\tightlist
\item[\dschemical] 나-371나 (C3711) 당검사 (정량) [\myexplfn{17.32} 원]
\item[\dschemical] 650100140 다이솔에스액(포도당)
\end{itemize}
}
{
임신 24-28주 사이에 1회만 인정하고, 해당 수기료는 나 371나 당검사(정량)으로 산정하며, 부하검사 시 사용된 약제는 별도 인정함.
}
\prezi{\clearpage}

\myde{100g 경구 포도당부하검사}
{
\begin{itemize}\tightlist
\item[\dschemical] 나-371나 (C3711) 당검사 (정량) [\myexplfn{17.32} 원] X 4
\item[\dschemical] 650100140 다이솔에스액(포도당) X 2
\end{itemize}
}
{
임신부에게 실시하는 100g 경구 포도당부하검사의 급여기준 : 임신부에게 실시하는 100g 경구 포도당부하검사는 임신으로 인한 생리적 변화에 의해 발생하는 임신성 당뇨병을 진단하는 검사방법으로 세부 인정기준은 다음과 같이 함. \par
      - 다 음 – 
\begin{enumerate}[가.]\tightlist
\item 적용대상 : 50g 경구 포도당부하검사 결과 140㎎/㎗( ※임신성 당뇨병의 고위험군 은 130㎎/㎗) 이상인 경우에 인정함. 
\item 수가산정방법 : Glucose 측정 시 수기료는 나371나 당검사(정량)으로 산정하고, 부하검사 시 사 용된 약제는 별도 인정함. 
위 1) 이외에 ‘나693나 경구 포도당부하 검사’ 또는 ‘나737 임신성 100g 경구 포도당부하검사 관리료’를 별도 산정하는 경우 세부적인 기준은「건강보험 행위 급여\cntrdot{} 비급여 목록표 및 급여 상대가치점수」제1편 제2부 제2장 제3절 [내분비 기능 검사] ‘주1‧2’ 또는 「임신성 100g 경구 포도당부하검사 관리료 급여기준」에 따름. 
\end{enumerate}
}

\prezi{\clearpage}
\myde{임신성 100g 경구 포도당부하검사 관리료}
{
나-737 (E7370)[\myexplfn{98.00} 원]
}
{
임신성 100g 경구 포도당부하검사 관리료 인정기준 : 임신성 100g 경구 포도당 부하검사 관리료는 아래와 같은 요건에 모두 해당하는 경우에 산정함. \par
     - 아 래 – 
\begin{enumerate}[가.]\tightlist
\item 임신부에게 실시하는 100g 경구 포도당부하검사 과정에서 발생 가능한 부작용(구토, 어지러움), 태아 등을 모니터링 및 관리하고, 검사전\cntrdot{} 후 상태 및 검사결과에 대한 판독소견을 기록한 경우에 산정함. 
\item Glucose를 4회 측정한 경우에 산정하며, 검체검사료는 나371나 당검사(정량)의 소 정점수를 별도 산정함. 
\item 나693나 경구 포도당부하검사와 동시에 산정할 수 없음
\end{enumerate} 
}
%\clearpage
\prezi{\clearpage}
\par
\medskip
\begin{commentbox}{임신성 당뇨검사 환불사태}
보건복지부 고시(제 2007-46호)에 따라 2007년 6월1일부터 고위험군 임산부에 한해 임신성 당뇨검사비를 보험급여로 적용했으나 일선 병원에서 비급여로 비용을 받음\par
발단이 된 것은 2013년 5월 한 산모가 온라인에 올린 체험기.\par
해당 산모는 '건강보험심사평가원에 진료비 환불신청을 해 임당검사 비용을 환불받았으며, 30세 이상 산모와 임신성 당뇨 고위험군 환자의 경우 마찬가지로 비급여로 낸 임당검사 비용을 환불받을 수 있다'는 내용의 게시물을 온라인에 게재했고, 해당 내용이 산모카페 등에 소개되면서 집단적 환불신청 움직임으로 이어졌다.\par
실제 이 같은 글들이 퍼져나가면서 심평원에는 임당검사 비용 환급을 요구하는 진료비 확인민원 증가, 병의원을 상대로 직접 환급을 요청
\end{commentbox}