\section{약제}
\subsection{급여 등재 약물}
%\begin{table}
\tabulinesep =_2mm^2mm
\begin{tabu} to \linewidth {|X[2,l]|X[2,l]|X[2,l]|X[2,l]|} \tabucline[.5pt]{-}
\rowcolor{Gray!25} 보험 인정 범위 & 식약청 약품 허가 범위 & 급여로 산정 & 비급여로 산정 \\ \tabucline[.5pt]{-}
\rowcolor{Yellow!5} 인정범위 내 & 허가내 사용 & 합 법 & 불법(임의비급여) \\ \tabucline[.5pt]{-}
\rowcolor{Yellow!5} 인정범위 내 & 허가외 사용 & 합 법 & 불법(임의비급여) \\ \tabucline[.5pt]{-}
\rowcolor{Yellow!5} 인정범위 외 & 허가내 사용 & 삭 감 & 불법(임의비급여) \\ \tabucline[.5pt]{-}
\rowcolor{Yellow!5} 인정범위 외 & 허가외 사용 & 삭 감 & 불법(임의비급여) \\ \tabucline[.5pt]{-}
\end{tabu}
‘허가외 사용 의약품’이란 약사법 제31조 제2항 및 제3항 또는 제42조 제1항에 의한 허가(신고)를 받았으나, 허가(신고)받지 아니한 효능·효과, 또는 용법·용량으로 사용하고자 하는 의약품을 말한다.
%\end{table}

\begin{enumerate}[①]\tightlist
\item 급여 진료의 경우 심평원 보험 인정 범위 내이면서 식약청 약품 허가 범위 내인 경우 급여로 산정하는 경우는 합법이다.
\item 급여 진료의 경우 심평원 보험 인정 범위 내이면서 식약청 약품 허가 범위를 벗어난 경우 급여로 청구하는 경우 합법입니다. 식약처 허가 범위외라 하더라도 고시상에 인정되는 경우라면 처방 가능합니다 예)아스피린의 절박유산 사용 : 식약처 허가범위는 아니지만, 보험인정 고시가 있음
\item 급여 진료의 경우 심평원 보험 인정 범위를 벗어나면서 식약청 약품 허가범위 내라 하더라도 급여로 산정하면 삭감된다. 예로는 antihistamin계통의 약물들 (2가지 이상의 2세대 이상의 antihistamin계통의 약물 사용시)
\item 급여 진료의 경우 심평원 보험 인정 범위를 벗어나면서 식약청 약품 허가범위 외이면 급여로 산정하면 삭감된다. 이를 막기위해 약제의 보험기준에 맞는 상병을 입력하는 방법이 있습니다
\item 급여 진료의 경우 심평원 보험 인정 범위 여부 및 식약청 약품 허가범위와 상관없이 급여약을 비급여로 산정하면 모두 임의비급여로 불법이다. 
\item 비급여 진료로 진찰료 청구하지 않고 진료를 보는 경우, 보험 인정 범위 및 식약청 약품 허가 범위 상관없이 비급여로 산정해서 처방 가능하다. (ex> 탈모치료로 프로스카를 처방하는 경우, 비만환자에게 마약 성분류 식욕 억제제 사용)
처방 가능합니다. (퇴행성)탈모, 비만은 비급여 대상입니다.. 급여 약제나 비급여 약제 모두 비급여 대상입니다. 비급여 진료 항목인 경우 (비만 , 성형) 식약처 허가초과(향정신성의약품의 식욕억제제로 사용)는 비급여 산정이 가능합니다(처벌할 법적 근거가 없음)
\end{enumerate}

\begin{commentbox}{아스피린의 절박유산 사용 : 식약처 허가범위는 아니지만 보험인정 고시가 있음} 
아스피린 효능/효과:
\begin{enumerate}[1.]\tightlist
\item 혈소판 응집억제 작용에 의한 불안정형 협심증 환자에 있어서 비치명적 심근경색 위험감소 및 일과성 허혈 발작 위험감소에 사용 
\item 최초 심근경색 후 재경색 예방 
\item 다음 경우의 혈전·색전 형성의 억제 - 뇌경색환자, 관상동맥 우회술(CABG) 또는 경피경관 관상동맥 성형술(PTCA)시행 후 
\item 허혈성 심장질환의 가족력, 고혈압, 고콜레스테롤혈증, 비만, 당뇨 같은 복합적 심혈관 위험 인자를 가진 환자에서 관상동맥 혈전증의 예방
\end{enumerate}

Aspirin 경구제 (품명:아스피린 프로텍트정100밀리그람 등) 고시 제2013-127호
\begin{enumerate}[1.]\tightlist
\item 허가사항 범위 내에서 투여 시 요양급여 함을 원칙으로 함. 
\item 허가사항 범위(효능ㆍ효과)를 초과하여 아래와 같은 기준으로 투여한 경우에도 요양급여를 인정함. \par
- 아 래 -
	\begin{enumerate}[가.]\tightlist
	\item 절박유산과 관계가 있다고 추정되는 태반이나 탈락막의 혈전생성을 방지하기 위해 혈소판 응집억제 작용으로 저용량(80~100mg/day)을 투여한 경우 
	\item 말초동맥성질환에 투여한 경우.
	\end{enumerate}
\end{enumerate}
\end{commentbox}
\subsection{임의비급여의 부분적및 제한적인 합법화}
대법원은 ▲의학적 안전성과 유효성을 갖추고 ▲시급성이 있고 ▲환자에게 미리 그 내용과 비용을 설명해 동의를 받는 경우에 한해 임의비급여를 허용할 수 있다고 했다. 이 제한적인 조건에 부합하는 임의비급여인지는 의료기관이 증명해야 한다\par
\medskip
\includegraphics[width=.9\linewidth]{bitogum}

\subsection{비급여 등재 약물}
\tabulinesep =_2mm^2mm
\begin{tabu} to \linewidth {|X[2,l]|X[2,l]|X[2,l]|} \tabucline[.5pt]{-}
\rowcolor{Gray!25} 보험 인정 범위 & 식약청 약품 허가 범위 &  비급여로 산정 \\ \tabucline[.5pt]{-}
\rowcolor{Yellow!5} 없음 & 허가내 사용 & 합 법  \\ \tabucline[.5pt]{-}
\rowcolor{Yellow!5} 없음 내 & 허가외 사용 & 임의비급여(?) \\ \tabucline[.5pt]{-}
\end{tabu}
\begin{enumerate}[①]\tightlist
\item 급여 진료의 경우 심평원 보험 인정 범위 내인 경우 비급여로 산정 가능하다.
\item 급여 진료의 경우 식약청 약품 허가 범위를 벗어나는 경우 비급여 산정이 가능한지요? 급여 진료 항목인 경우 비급여 약품의 식약처 허가범위 초과는 임의비급여(?)입니다. 그러나, 비급여 약품을  사용할 경우 의사의 판단에 의거 환수할 법적 근거 없습니다 예)비급여 비타민제. 태반등의 식약처 허가외 사용
\end{enumerate}
 
\clearpage
\section{행위/치료대}
\subsection{급여 인정 범위가 있는 행위/치료대}
\tabulinesep =_2mm^2mm
\begin{tabu} to \linewidth {|X[4,l]|X[4,l]|X[4,l]|X[4,l]|} \tabucline[.5pt]{-}
\rowcolor{Gray!25}  보험/비보험 진료& 보험인정 범위  & 급여로 산정 & 비급여로 산정 \\ \tabucline[.5pt]{-}
\rowcolor{Yellow!5} 보험 진료 & 인정범위내 & 합법 & 불법(임의비급여)  \\ \tabucline[.5pt]{-}
\rowcolor{Yellow!5} 보험 진료 & 인정범위외 & 삭감 & 불법(임의비급여) \\ \tabucline[.5pt]{-}
\rowcolor{Yellow!5} 비보험 진료 & 인정범위내 & X  & 합법 \\ \tabucline[.5pt]{-}
\rowcolor{Yellow!5} 비보험 진료 & 인정범위외 & X  & 합법 \\ \tabucline[.5pt]{-}
\end{tabu}
\par
\medskip
\begin{enumerate}[①]\tightlist
\item 급여 진료시에는 보험 인정 범위 내에서 급여로 청구한 경우 합법이다.
\item 급여 진료시에는 보험 인정 범위 외에서 급여로 청구한 경우 삭감된다. 예를 들면 질강처치의 경우는 고시에 없는 상병에서나 한달에 두번 하게 되면 삭감된다.
\item 급여 진료시에는 보험인정범위가 있는 행위/치료대등을 임의로 비급여로 받으면 불법이다. 예를 들면 루프부작용으로  N989 상세불명의 질출혈로 루프제거R4275 시행시에 관습대로 루프제거를 비보험으로 받게되면 임의비급여로 불법이다.
\item 비급여 진료시에는 급여 행위/약제/재료는 모두 비급여 진료 행위로 간주하기 때문에 급여 항목은 비급여로 전환해서 받을 수 있게 되어 있습니다. 그래서 타원 IVF 중인 환자가 그 병원에서 받은 약을 가지고 본원에 온 경우에는 주사료를 비급여로 전환해서 받을 수 있습니다. 이렇게 생각한다면 보험 인정 범위 내의 행위/치료대의 경우 비급여 진료로 해서 진찰료를 받지 않고 접수해서 비급여로 산정하는 것은 괜찮지 않습니까?(필요하다면 급여 / 비급여로 분리 접수를 해서 말입니다.) 일단 진료가 급여 항목인경우인지, 비급여 항목인지 분리해야합니다 IVF-ET, IUI등의 시술과 관계없는 급여 진료인 경우 주사료와 진찰료가 급여 산정가능하지만 IVF-ET, IUI 등의 시슬과 관계있는 경우 진찰료와 주사수기료 모두 비급여입니다
\end{enumerate}

\subsection{비급여 인정 범위가 있는 행위/치료대(인정 비급여)}
\tabulinesep =_2mm^2mm
\begin{tabu} to \linewidth {|X[4,l]|X[4,l]|} \tabucline[.5pt]{-}
\rowcolor{Gray!25}  보험인정 범위  & 비급여로 산정 \\ \tabucline[.5pt]{-}
\rowcolor{Yellow!5} 인정범위내 & 합법   \\ \tabucline[.5pt]{-}
\rowcolor{Yellow!5} 인정범위외 & 합법  \\ \tabucline[.5pt]{-}
\end{tabu}

\par
\medskip
\begin{enumerate}[①]\tightlist
\item 급여 진료시 보험인정법위내의 경우엔 비급여 산정이 가능하다.
\item 급여 진료시 보험 인정범위외의 경우중 비급여 산정이 가능하다는 고시가 있는경우는 합볍적으로 비급여 산정이 가능합니다(별도 산정이 불가한 경우 비급여 산정도 임의 비급여) 예) 요실금 수술의 경우 보험기준외의 경우는 비급여  산정 가능
\end{enumerate}

\begin{commentbox}{급여진료시 보험인정외 비급여로 청구가능 한 경우}
인조테이프를 이용한 요실금수술은 요류역학검사(방광내압측정 및 요누출압검사)로 복압성 요실금 또는 복압성 요실금이 주된 혼합성 요실금이 확인되고 요누출압이 120cmH2O 미만인 경우에 .인정하며, 동 인정기준 이외에는 비용효과성이 떨어지고 치료보다 예방적 목적이 크다고 간주하여 시술료 및 치료재료 비용 전액은 환자가 부담토록 함(비급여).\par

인조테이프를 이용한 요실금수술의 구체적 적용기준에 대하여 붙임과 같이 통보하오니 업무에 참고하시기 바랍니다.\par
 
붙임)
\begin{enumerate}[1.]\tightlist
\item 요실금수술 인정기준에 해당되지 않는 경우 수술료 및 치료재료 외의 진료비용에 대하여 ‘인조테이프를 이용한 요실금수술 인정기준’ 에 해당되지 않는 경우에는 수술료 및 치료재료 비용 뿐 아니라 입원료, 마취료 등 제반 진료비용 전액은 환자가 부담토록 함. (비급여).
\item 요류역학검사(방광내압측정 및 요누출압검사)를 실시하지 않고 요실금 수술을 시행시 급여여부 ; 현행 인정기준에 해당되지 않으므로 비급여 대상임.
\end{enumerate}
\end{commentbox}

\subsection{비급여 인정 범위가 없는 행위/치료대(비인정 비급여)}
예를 들면 일회용질경등.. 환수대상임.\par
- 비급여 진료의 경우 진찰료 산정을 못하므로 이를 보전하기 위한 비용들
\begin{enumerate}[①]\tightlist
\item 피임 목적으로 방문 시의 ‘비급여 피임 처방료’ 
\item IUI CYCLE 중 내원하여 클로미펜이나 FSH 처방 받을 시의 ‘불임 환자 상담료’
\item 성의학 진료 시의 ‘성의학 상담료’
\end{enumerate}

피임은 비급여 항목이므로 비급여 진찰료, 상담료 산정 가능합니다 \par
불임은 VF-ET, IUI 등의 시슬과 관계있는 경우 진찰료와 주사수기료 상담료 모두 비급여입니다.\par
“성상담“은 합법 비급여 항목입니다. 

\clearpage
\section{검사}
\subsection{급여 인정 범위가 있는 검사}
\tabulinesep =_2mm^2mm
\begin{tabu} to \linewidth {|X[4,l]|X[4,l]|X[4,l]|X[4,l]|} \tabucline[.5pt]{-}
\rowcolor{Gray!25}  보험/비보험 진료& 보험인정 범위  & 급여로 산정 & 비급여로 산정 \\ \tabucline[.5pt]{-}
\rowcolor{Yellow!5} 보험 진료 & 인정범위내 & 합법 & 불법(임의비급여)  \\ \tabucline[.5pt]{-}
\rowcolor{Yellow!5} 보험 진료 & 인정범위외 & 불법(삭감) & 불법(임의비급여) \\ \tabucline[.5pt]{-}
\rowcolor{Yellow!5} 비보험 진료(건강검진) & 인정범위외 & X  & 합법 \\ \tabucline[.5pt]{-}
\end{tabu}
\par
\medskip

\begin{enumerate}[①]\tightlist
\item 급여 진료 시 보험 인정 범위 내의 경우 급여로 해야 한다. (비급여로 하면 임의 비급여로 불법)
\item 급여 진료 시 보험 인정 범위를 벗어나는 경우는 원칙적으로 급여 / 비급여 모두 불법이다.
\item 급여 항목에 대해서 보험 인정 범위를 벗어나느 경우에도 건강검진 목적으로 시행하는 검사는 예외적으로 비급여 산정이 가능하다. 
\end{enumerate}
\begin{commentbox}{}
비보험 진료를 위한 검사의 경우는 비급여로 해도 되는지요?(성형 수술을 위한 수술전 검사를 시행하는 것을 ‘건강 검진 목적’으로 시행했다고 하는 것 보다는 비급여 진료에 사용된 급여 행위/약제/재료는 모두 비급여 진료 행위로 간주하기 때문에 급여 항목을 비급여로 전환했다고 하는 것이 더 논리에 맞지 않나요? 만일 이것이 인정된다면 보험 인정범위 내 검사도 급여 / 비급여로 분리 접수를 해서 비급여 진료에 의한 검사로 비급여 산정을 하면 괜찮지 않나요?)
\par
보험 등재된 검사 또한 비급여 항목에 해당되는 경우(성형수술 등) 비급여입니다
급여 등재된 혈액검사는 기본적으로 급여 검사입니다
환자가 원하여 하는 건강검진 검사의 범주에 해당되는 경우는 비급여입니다
비록 환자가 동의하였다하여 요양급여 대상을 비급여 또는 환자전액 본인부담 (100/100) 으로 해서는 안됩니다
\end{commentbox}

\subsection{비급여 인정 범위가 있는 검사(인정 비급여)}
\tabulinesep =_2mm^2mm
\begin{tabu} to .5\linewidth {|X[4,l]|X[4,l]|} \tabucline[.5pt]{-}
\rowcolor{Gray!25}  보험인정 범위  & 비급여로 산정 \\ \tabucline[.5pt]{-}
\rowcolor{Yellow!5} 인정범위내 & 합법   \\ \tabucline[.5pt]{-}
\rowcolor{Yellow!5} 인정범위외 & 합법  \\ \tabucline[.5pt]{-}
\end{tabu}
\par
\medskip
① 인정 비급여 검사의 경우 그 허가범위를 넘어가도 합법이다.

\subsection{비급여 인정 범위가 없는 검사(비인정 비급여)}
대표적으로 RT PCR 22종등 입니다. 이 검사의 경우는 신의료기술 등재가 되지 않았습니다. 그럼으로 검사하시게 되면 안 되지만 비급여 진료시 비인정 비급여를 한다고 하더라도 특별히 환자의 민원이 있지 않는한 특별한 문제가 있지는 않습니다.\par

\begin{commentbox}{}
인정 비급여 검사는 신의료 기술등재후 요양급여 결정 신청된 경우입니다.
미결정 검사나 행위, 치료재료는 임의비급여입니다.
환자가 원하여 하는 건강검진 검사의 범주에 해당되는 경우등은  비급여입니다
비급여 진료의 경우 따로 상병기입은 안해도 됩니다,, 진료기록에 비급여 진료항목임을 적어두면 됩니다
급여와 비급여 진료가 혼재된 경우 급여에 관련된 상병후 진찰료등 청구가 가능하면 진료기록에 급여 진료에 대한 내용을 소명해두면 됩니다
\end{commentbox}
\clearpage
국민건강보험법     제41조(요양급여)
\begin{enumerate}[①]\tightlist
\item 가입자와 피부양자의 질병, 부상, 출산 등에 대하여 다음 각 호의 요양급여를 실시한다.
	\begin{enumerate}[1.]\tightlist
	\item 진찰·검사
	\item 약제(藥劑)·치료재료의 지급
	\item 처치·수술 및 그 밖의 치료
	\item 예방·재활
	\item 입원
	\item 간호
	\item 이송(移送)
	\end{enumerate}
\item \highlightY{제1항에 따른} \highlightR{요양급여(이하 "요양급여"라 한다)의 방법·절차·범위·상한 등의 기준은 보건복지부령으로 정한다.}
\item 보건복지부장관은 제2항에 따라 \highlightR{요양급여의 기준을 정할 때 업무나 일상생활에 지장이 없는 질환, 그 밖에 보건복지부령으로 정하는 사항은 요양급여의 대상에서 제외할 수 있다.}
\end{enumerate}


국민건강보험 요양급여의 기준에 관한 규칙  [별표 1]요양급여의 적용기준 및 방법(제5조제1항관련)


\begin{enumerate}[1.]\tightlist
\item 요양급여의 일반원칙
	\begin{enumerate}[가.]\tightlist
	\item  요양급여는 가입자 등의 \highlightR{연령·성별·직업 및 심신상태 등의 특성을 고려하여 진료의 필요가 있다고 인정되는 경우에 정확한 진단을 토대로 하여 환자의 건강증진을 위하여 의학적으로 인정되는 범위 안에서 최적의 방법으로 실시하여야 한다.}
	\item  요양급여를 담당하는 의료인은 의학적 윤리를 견지하여 환자에게 심리적 건강효과를 주도록 노력하여야 하며, 요양상 필요한 사항이나 예방의학 및 공중보건에 관한 지식을 환자 또는 보호자에게 이해하기 쉽도록 적절하게 설명하고 지도하여야 한다.
	\item  \highlightR{요양급여는 경제적으로 비용효과적인 방법으로 행하여야 한다.}
	\item  \highlight{요양기관은 가입자 등의 요양급여에 필요한 적정한 인력·시설 및 장비를 유지하여야 한다. 이 경우 보건복지부장관은 인력·시설 및 장비의 적정기준을 정하여 고시할 수 있다.}
	\item  라목의 규정에 불구하고 가입자 등에 대한 최적의 요양급여를 실시하기 위하여 필요한 경우, 보건복지부장관이 정하여 고시하는 바에 따라 다른 기관에 검사를 위탁하거나, 당해 요양기관에 소속되지 아니한 전문성이 뛰어난 의료인을 초빙하거나, 다른 요양기관에서 보유하고 있는 양질의 시설·인력 및 장비를 공동 활용할 수 있다.
	\end{enumerate}
\item \highlight{진찰·검사, 처치·수술 기타의 치료}
	\begin{enumerate}[가.]\tightlist
	\item  \highlightR{각종 검사를 포함한 진단 및 치료행위는 진료상 필요하다고 인정되는 경우에 한하여야 하며 연구의 목적으로 하여서는 아니된다.}
	\item  영 제21조제3항제2호에 따라 보건복지부장관이 정하여 고시하는 질병군에 대한 입원진료의 경우 그 입원진료 기간동안 행하는 것이 의학적으로 타당한 검사·처치 등의 진료행위는 당해 입원진료에 포함하여 행하여야 한다.
	\end{enumerate}
\item 약제의 지급 
	\begin{enumerate}[가.]\tightlist
	\item  처방·조제
		\begin{enumerate}[(1)]\tightlist
		\item 영양공급·안정·운동 그 밖에 요양상 주의를 함으로써 치료효과를 얻을 수 있다고 인정되는 경우에는 의약품을 처방·투여하여서는 아니되며, 이에 관하여 적절하게 설명하고 지도하여야 한다.
		\item 의약품은 약사법령에 의하여 허가 또는 신고된 사항(효능·효과 및 용법·용량 등)의 범위 안에서 환자의 증상 등에 따라 필요·적절하게 처방·투여하여야 한다. (약제 전산심사)다만, 안전성·유효성 등에 관한 사항이 정하여져 있는 의약품 중 진료상 반드시 필요하다고 보건복지부장관이 정하여 고시하는 의약품의 경우에는 허가 또는 신고된 사항의 범위를 초과하여 처방·투여할 수 있으며, 중증환자에게 처방·투여하는 약제로서 보건복지부장관이 정하여 고시하는 약제의 경우에는 건강보험심사평가원장이 공고한 범위 안에서 처방·투여할 수 있다.
		\item 요양기관은 중증환자에 대한 약제의 처방·투여시 해당약제 및 처방·투여의 범위가 (2)의 허용범위에는 해당하지 아니하나 해당환자의 치료를 위하여 특히 필요한 경우에는 건강보험심사평가원장에게 해당약제의 품목명 및 처방·투여의 범위 등에 관한 자료를 제출한 후 건강보험심사평가원장이 중증질환심의위원회의 심의를 거쳐 인정하는 범위 안에서 처방·투여할 수 있다.
		\item 제10조의2제2항에 따라 식품의약품안전처장이 긴급한 도입이 필요하다고 인정한 품목의 경우에는 식품의약품안전처장이 인정한 범위 안에서 처방·투여하여야 한다.
		\item 항생제·스테로이드제제 등 오남용의 폐해가 우려되는 의약품은 환자의 병력·투약력 등을 고려하여 신중하게 처방·투여하여야 한다.
		\item 진료상 2품목 이상의 의약품을 병용하여 처방·투여하는 경우에는 1품목의 처방·투여로는 치료효과를 기대하기 어렵다고 의학적으로 인정되는 경우에 한한다.
		\end{enumerate}
	\end{enumerate}		
\item 치료재료의 지급
치료재료는 약사법 기타 다른 관계법령에 의하여 허가·신고 또는 인정된 사항(효능·효과 및 사용방법)의 범위 안에서 환자의 증상에 따라 의학적 판단에 의하여 필요·적절하게 사용한다. 다만, 안전성·유효성 등에 관한 사항이 정하여져 있는 치료재료 중 진료에 반드시 필요하다고 보건복지부장관이 정하여 고시하는 치료재료의 경우에는 허가·신고 또는 인정된 사항(효능·효과 및 사용방법)의 범위를 초과하여 사용할 수 있다.
\end{enumerate}


국민건강보험 요양급여 기준에 관한 규칙.
제10조(신의료기술등의 요양급여 결정신청)
\begin{enumerate}[①]\tightlist
\item 요양기관, 의약관련 단체, 치료재료의 제조업자·수입업자(치료재료가 「인체조직 안전 및 관리 등에 관한 법률」 제3조제1호에 따른 인체조직인 경우에는 같은 법 제13조에 따른 조직은행의 장을 말한다)는 제8조제2항에 따른 요양급여대상 또는 제9조제1항에 따른 비급여대상으로 결정되지 아니한 새로운 행위 및 치료재료(이하 "신의료기술등"이라 한다)에 대하여는 다음 각 호에 규정된 날부터 30일 이내에 요양급여대상 여부의 결정을 보건복지부장관에게 신청하여야 한다.  <개정 2001.12.31., 2005.10.11., 2006.12.29., 2007.7.25., 2008.3.3., 2009.7.31., 2010.3.19.>
	\begin{enumerate}[1.]\tightlist
	\item 행위의 경우에는 「의료법」 제53조에 따른 신의료기술평가(이하 "신의료기술평가"라 한다) 결과 안전성·유효성 등을 인정받은 이후 가입자등에게 최초로 실시한 날
	\item 치료재료의 경우에는 다음 각 목에서 정한 날
		\begin{enumerate}[가.]\tightlist
		\item 「약사법」 또는 「의료기기법」에 따른 품목허가 또는 품목신고 대상인 치료재료인 경우에는 식품의약품안전청장으로부터 품목허가를 받거나 품목신고를 한 날. 다만, 품목허가나 품목신고 대상이 아닌 치료재료의 경우에는 해당 치료재료를 가입자등에게 최초로 사용한 날
		\item 「인체조직 안전 및 관리 등에 관한 법률」 제3조제1호에 따른 인체조직(이하 "인체조직"이라 한다)의 경우에는 보건복지부장관으로부터 조직은행 설립허가를 받은 날. 다만, 다음의 어느 하나의 경우에는 그 해당하는 날
			\begin{enumerate}[1)]\tightlist
			\item  수입인체조직의 경우에는 보건복지부장관이 정하는 바에 따라 안전성에 문제가 없다는 통지를 받은 날
			\item  조직은행 설립허가 당시의 취급품목이 변경된 경우에는 보건복지부장관이 그 변경사실을 확인한 날
			\end{enumerate}
		\item 가목 및 나목에도 불구하고 신의료기술평가대상이 되는 치료재료의 경우에는 신의료기술 평가 결과 안전성·유효성 등을 인정받은 이후 해당 치료재료를 가입자등에게 최초로 사용한 날
		\end{enumerate}
	\item 삭제  <2006.12.29.>
	\end{enumerate}
\item 제1항에 따른 결정신청은 그 결정을 신청하려는 자가 다음 각 호의 구분에 따른 평가신청서에 해당 각 목의 서류를 첨부하여 건강보험심사평가원장에게 요양급여대상여부의 평가신청을 함으로써 이를 갈음한다.  <개정 2001.12.31., 2005.10.11., 2007.7.25., 2008.3.3., 2009.7.31., 2010.3.19., 2010.4.30.>
	\begin{enumerate}[1.]\tightlist
	\item 행위의 경우 : 별지 제14호서식의 요양급여행위평가신청서
		\begin{enumerate}[가.]\tightlist
		\item 신의료기술의 안전성·유효성 등의 평가결과통보서
		\item 상대가치점수의 산출근거 및 내역에 관한 자료
		\item 비용효과에 관한 자료(동일 또는 유사 행위와의 장·단점, 상대가치 점수의 비교 등을 포함한다)
		\item 국내외의 실시현황에 관한 자료(최초실시연도·실시기관명 및 실시건수 등을 포함한다)
		\item 소요장비·소요재료·약제의 제조(수입)허가(신고)관련 자료
		\item 국내외의 연구논문 등 기타 참고자료
		\end{enumerate}		
	\item 삭제  <2006.12.29.>
	\item 제1항제2호가목의 경우(제1항제2호다목에 따른 치료재료를 포함한다): 별지 제16호서식의 치료재료평가신청서
		\begin{enumerate}[가.]\tightlist
		\item 제조(수입)품목허가증(신고서)사본(품목허가를 받거나 품목 신고를 한 치료재료에 한한다)
		\item 판매예정가 산출근거 및 내역에 관한 자료
		\item 비용효과에 관한 자료(동일 또는 유사목적의 치료재료와의 장·단점, 판매가의 비교 등을 포함한다)
		\item 국내외의 사용현황에 관한 자료(최초사용연도·사용기관명 및 사용건수 등을 포함한다)
		\item 구성 및 부품내역에 관한 자료 및 제품설명서
		\item 국내외의 연구논문 등 기타 참고자료
		\item 신의료기술의 안전성·유효성 등의 평가결과통보서(제1항제2호다목에 따른 치료재료만 해당한다)
		\end{enumerate}
	\item 제1항제2호나목의 경우(제1항제2호다목에 따른 인체조직을 포함한다): 별지 제16호의2서식의 인체조직평가신청서
		\begin{enumerate}[가.]\tightlist
		\item 조직은행설립허가증 사본(기재사항 변경내역을 포함한다). 다만, 수입인체조직의 경우에는 보건복지부장관이 정하는 바에 따라 안전성에 문제가 없다는 사실을 증명하는 서류를 함께 첨부하여야 한다.
		\item 인체조직가격 산출근거 및 내역에 관한 자료
		\item 비용효과에 관한 자료(동일 또는 유사목적의 인체조직과의 장·단점, 가격 비교 등을 포함한다)
		\item 국내외의 사용현황에 관한 자료(최초 사용연도, 사용기관명 및 사용건수 등을 포함한다)
		\item 인체조직에 대한 설명서
		\item 국내외의 연구논문 등 기타 참고자료
		\item 신의료기술의 안전성·유효성 등의 평가결과 통보서(제1항제2호다목에 따른 인체조직만 해당한다)
		\end{enumerate}
	\end{enumerate}
\item 보건복지부장관은 요양기관이 정당한 사유없이 신의료기술등에 대하여 제1항의 규정에 위반하여 요양급여대상 여부의 결정을 신청하지 아니하고 가입자등에게 실시 또는 사용한 후 그 비용을 부담시킨 신의료기술등이 요양급여대상으로 확인된 경우에는 법 제85조제1항제1호의 규정에 의하여 당해 요양기관의 업무정지를 명하거나 동조제2항의 규정에 의한 과징금처분을 하여야 한다.  <개정 2001.12.31., 2008.3.3., 2010.3.19.>
[제목개정 2001.12.31., 2007.7.25.]
\end{enumerate}
