\section{약제}
\subsection{급여 등재 약물}
%\begin{table}
\tabulinesep =_2mm^2mm
\begin{tabu} to \linewidth {|X[2,l]|X[2,l]|X[2,l]|X[2,l]|} \tabucline[.5pt]{-}
\rowcolor{Gray!25} 보험 인정 범위 & 식약청 약품 허가 범위 & 급여로 산정 & 비급여로 산정 \\ \tabucline[.5pt]{-}
\rowcolor{Yellow!5} 인정범위 내 & 허가내 사용 & 합 법 & 불법(임의비급여) \\ \tabucline[.5pt]{-}
\rowcolor{Yellow!5} 인정범위 내 & 허가외 사용 & 합 법 & 불법(임의비급여) \\ \tabucline[.5pt]{-}
\rowcolor{Yellow!5} 인정범위 외 & 허가내 사용 & 삭 감 & 불법(임의비급여) \\ \tabucline[.5pt]{-}
\rowcolor{Yellow!5} 인정범위 외 & 허가외 사용 & 삭 감 & 불법(임의비급여) \\ \tabucline[.5pt]{-}
\end{tabu}
‘허가외 사용 의약품’이란 약사법 제31조 제2항 및 제3항 또는 제42조 제1항에 의한 허가(신고)를 받았으나, 허가(신고)받지 아니한 효능·효과, 또는 용법·용량으로 사용하고자 하는 의약품을 말한다.
%\end{table}

\begin{enumerate}[①]\tightlist
\item 급여 진료의 경우 심평원 보험 인정 범위 내이면서 식약청 약품 허가 범위 내인 경우 급여로 산정하는 경우는 합법이다.
\item 급여 진료의 경우 심평원 보험 인정 범위 내이면서 식약청 약품 허가 범위를 벗어난 경우 급여로 청구하는 경우 합법입니다. 식약처 허가 범위외라 하더라도 고시상에 인정되는 경우라면 처방 가능합니다 예)아스피린의 절박유산 사용 : 식약처 허가범위는 아니지만, 보험인정 고시가 있음
\item 급여 진료의 경우 심평원 보험 인정 범위를 벗어나면서 식약청 약품 허가범위 내라 하더라도 급여로 산정하면 삭감된다. 예로는 antihistamin계통의 약물들 (2가지 이상의 2세대 이상의 antihistamin계통의 약물 사용시)
\item 급여 진료의 경우 심평원 보험 인정 범위를 벗어나면서 식약청 약품 허가범위 외이면 급여로 산정하면 삭감된다. 이를 막기위해 약제의 보험기준에 맞는 상병을 입력하는 방법이 있습니다
\item 급여 진료의 경우 심평원 보험 인정 범위 여부 및 식약청 약품 허가범위와 상관없이 급여약을 비급여로 산정하면 모두 임의비급여로 불법이다. 
\item 비급여 진료로 진찰료 청구하지 않고 진료를 보는 경우, 보험 인정 범위 및 식약청 약품 허가 범위 상관없이 비급여로 산정해서 처방 가능하다. (ex> 탈모치료로 프로스카를 처방하는 경우, 비만환자에게 마약 성분류 식욕 억제제 사용)
처방 가능합니다. (퇴행성)탈모, 비만은 비급여 대상입니다.. 급여 약제나 비급여 약제 모두 비급여 대상입니다. 비급여 진료 항목인 경우 (비만 , 성형) 식약처 허가초과(향정신성의약품의 식욕억제제로 사용)는 비급여 산정이 가능합니다(처벌할 법적 근거가 없음)
\end{enumerate}

\begin{commentbox}{아스피린의 절박유산 사용 : 식약처 허가범위는 아니지만 보험인정 고시가 있음} 
아스피린 효능/효과:
\begin{enumerate}[1.]\tightlist
\item 혈소판 응집억제 작용에 의한 불안정형 협심증 환자에 있어서 비치명적 심근경색 위험감소 및 일과성 허혈 발작 위험감소에 사용 
\item 최초 심근경색 후 재경색 예방 
\item 다음 경우의 혈전·색전 형성의 억제 - 뇌경색환자, 관상동맥 우회술(CABG) 또는 경피경관 관상동맥 성형술(PTCA)시행 후 
\item 허혈성 심장질환의 가족력, 고혈압, 고콜레스테롤혈증, 비만, 당뇨 같은 복합적 심혈관 위험 인자를 가진 환자에서 관상동맥 혈전증의 예방
\end{enumerate}

Aspirin 경구제 (품명:아스피린 프로텍트정100밀리그람 등) 고시 제2013-127호
\begin{enumerate}[1.]\tightlist
\item 허가사항 범위 내에서 투여 시 요양급여 함을 원칙으로 함. 
\item 허가사항 범위(효능ㆍ효과)를 초과하여 아래와 같은 기준으로 투여한 경우에도 요양급여를 인정함. \par
- 아 래 -
	\begin{enumerate}[가.]\tightlist
	\item 절박유산과 관계가 있다고 추정되는 태반이나 탈락막의 혈전생성을 방지하기 위해 혈소판 응집억제 작용으로 저용량(80~100mg/day)을 투여한 경우 
	\item 말초동맥성질환에 투여한 경우.
	\end{enumerate}
\end{enumerate}
\end{commentbox}
\subsection{임의비급여의 부분적및 제한적인 합법화}
대법원은 ▲의학적 안전성과 유효성을 갖추고 ▲시급성이 있고 ▲환자에게 미리 그 내용과 비용을 설명해 동의를 받는 경우에 한해 임의비급여를 허용할 수 있다고 했다. 이 제한적인 조건에 부합하는 임의비급여인지는 의료기관이 증명해야 한다\par
\medskip
\includegraphics[width=.9\linewidth]{bitogum}

\subsection{비급여 등재 약물}
\tabulinesep =_2mm^2mm
\begin{tabu} to \linewidth {|X[2,l]|X[2,l]|X[2,l]|} \tabucline[.5pt]{-}
\rowcolor{Gray!25} 보험 인정 범위 & 식약청 약품 허가 범위 &  비급여로 산정 \\ \tabucline[.5pt]{-}
\rowcolor{Yellow!5} 없음 & 허가내 사용 & 합 법  \\ \tabucline[.5pt]{-}
\rowcolor{Yellow!5} 없음 내 & 허가외 사용 & 임의비급여(?) \\ \tabucline[.5pt]{-}
\end{tabu}
\begin{enumerate}[①]\tightlist
\item 급여 진료의 경우 심평원 보험 인정 범위 내인 경우 비급여로 산정 가능하다.
\item 급여 진료의 경우 식약청 약품 허가 범위를 벗어나는 경우 비급여 산정이 가능한지요? 급여 진료 항목인 경우 비급여 약품의 식약처 허가범위 초과는 임의비급여(?)입니다. 그러나, 비급여 약품을  사용할 경우 의사의 판단에 의거 환수할 법적 근거 없습니다 예)비급여 비타민제. 태반등의 식약처 허가외 사용
\end{enumerate}
 
