\section{Candida Vulvovaginitis}
\emph{캔디다 검사를 PCR + KOH mount로 하기}
\myde{}{%
\begin{itemize}\tightlist
\item[\dsjuridical] B373 외음 및 질의 칸디다증
\item[\dsjuridical] L292 외음가려움
\item[\dschemical] B0042 요침사현미경검사(wetSmear(DirectSmear)) [\myexplfn{9.14} 원]
\item[\dschemical] B4107 미생물현미경검사[일반염색]-피부진균KOH도말현미경검사 [\myexplfn{75.96} 원]
\item[\dschemical] C5956006 종합효소연쇄반응(기타) [\myexplfn{415.49} 원]
\item[\dsmedical] R4106 질강처치 [\myexplfn{58.04} 원]
\item[\dsmedical] 래피탈 질정 카네스텐 질정, Clotrimazole 100mg (일반약), 세나서트질정 2mg (일반약) \footnote{\pageref{Vag_Tab} 쪽 참조}
\item[\dsmedical] GV(Gentian violet): 현재 구하기도 어렵고, 등재된 약제가 아니라. 치료후 산정 불가함
\item[\dsmedical] 연고 apply: 원내 사용후 사용량만큼 청구 가능, clotrimazole(카네스테 크림 : 바이엘 코리아, 카마졸 크림 : 알보젠코리아 등), flutrimazole (나이트랄 크림 : 일양, 타미트랄 크림 : 태극, 
 프리마솔 크림 : 코오롱, 플리트졸 크림 : 대한뉴팜)
\item[\dsmedical] 가려움으로 인한 Dexa및 antihistamin사용.\footnote{\pageref{anti_histamin} 쪽 참조}
\end{itemize}
청구메모>> 캔디다 질염의심 환자로 KOH현미경 검사에서 균이 확인되지 않아서 검사시행함.
}{
<< charting 예제 >> \par
C/C : pruritus at ext. genitalia\par
P/Ex\par
 Vulva : erythema and edema\par
 curd like discharge\par
}

%\subsection{Recurrent Candida Vulvovaginitis}
\begin{commentbox}{주로 회음부에 얇은 스크레차같은 상처가 세로로 여러 줄이 있는 경우} 
\begin{itemize}\tightlist
\item[\dsjuridical] T141 상세불명의 신체부위의 열린상처
\item[\dsmedical] M0111 단순처치
\end{itemize}
\end{commentbox}

\begin{commentbox}{외음부 질칸디다증 청구팁}
B373과 N72[자궁경관염]혹은 N86[자궁경부 미란], A5900[편모충성 외음질염]이 겹칠경우\par
B373은 B358[기타 백선증]로 대치해서 다음에 B373을 쓸수 있게 SAVE한다.\par
N86인경우 R4310[자궁경부 전기 소작술] 청구 다음날 R4106[질강처치료] 청구하고 1-2주간격 R4311 OR R4310[자궁경부 약물소작술]청구 R4106과 R4310이 겹칠 땐 당근 R4310 청구  
\end{commentbox}
