\begin{myshadowbox}
\begin{enumerate}[2.]\tightlist
\item \textcolor{red}{가입자 또는 피부양자(이하 “가입자 등”이라한다)가 질병군으로 입원진료를 받은 경우에 적용}하되, 다음의 각 항목은 질병군 적용에서 제외하고 제 1편을 적용한다.
	\begin{enumerate}[가.]\tightlist
	\item 혈우병환자, HIV감염자
	\item 입원일수가 30일을 초과할 경우 31일째부터 발생하는 진료분
	\item 차상위 본인부담경감대상자로서 제3호 나목에 해당하는 경우
	\item \textcolor{blue}{질병군 진료 이외의 목적으로 입원하여 입원일수가 6일을 초과한 시점에 예상치 못하게 질병군 수술이 이루어진 경우 입원일로부터 수술시행일 전일까지의 진료분}
	\end{enumerate}
\end{enumerate}
\prezi{\clearpage}
\end{myshadowbox}
\Que{입원기간 중에 자격이 변동된 경우 또는 타법령(산재, 자보)으로 입원 중 질병군 대상 수술을 했을 경우 청구방법은?}
\Ans{타법령으로 입원 중 질병군 진료가 발생한 경우 전체 진료내역을 \textcolor{blue}{행위별수가제로 청구}함\par

 ☞ 급여 65720-1898호(2001.12.29) 「타법령으로 입원진료 중 질병군 진료가 발생한 경우 요양급여비용 산정방법」}
\par
\medskip
\prezi{\clearpage}
\Que{임신 유지목적으로 입원하여 입원일수가 \textcolor{red}{6일을 초과한 시점}에서 예상치 못하게 제왕절개분만이 이루어진 경우(초음파 산정방법)}
\Ans
{\begin{itemize}\tightlist
\item 입원(행위별 청구) :(정상임신부) 7회까지 급여, 그 외 비급여 (태아의 이상이나 이상이 예측되는 경우) 급여
\item 제왕절개 시점 : <분리청구 시점>% 구분 :  
\item 제왕절개분만 입원(DRG 청구) : 분만기간 초음파(비급여)
\end{itemize}}
\par
\medskip
\prezi{\clearpage}
\Que{임신 유지목적으로 입원하여 입원일수가 \textcolor{red}{6일을 이내에} 제왕절개분만이 이루어진 경우(초음파 산정방법)}
\Ans
{제왕절개분만 입원(DRG 청구) : 분만기간 초음파(비급여)}
\prezi{\clearpage}