\section{소견서와 진료확인서}
\Que{국민건강보험법상 고시에 의하여 교부하는 소견서 이외에 \textcolor{blue}{보험회사 등 요구에 의해 소견서를 발급해야 하는지 여부와 발급비용에 대한 적용기준} 있는지?}
\Ans{현재「요양급여의 적용기준 및 방법에 관한 세부사항고시」에는 \emph{각종 증명서 발급비용 (일반진단서, 입원 및 치료 확인서, 진료비추정서, 장해진단서, 추가발급비용 등)은 환자가 부담}토록 되어 있으나 
\emph{소견서, 촉탁서의 비용은 가1  진찰료 또는 가2  입원료의 소정점수에 포함되어 별도 산정하지 못하도록}  되어 있음.\\
이는 원칙으로 같은 증명서의 개념이나 상해진단서, 장해진단서, 진료비추정서, 정신감정서 등은 법적 보상이나 배상과 직접적 관련이 있는 것이며, 소견서는 일반적으로 동일 의료기관에서 다른 과나 다른 의료기관에서 동일 환자의 진료에 참고할 수 있도록 하기 위하여 진료의사가 자신의 소견을 적은 것으로,\textcolor{red}{보험회사 등에서 요청하는 소견서는 일반진단서를 말하는 것}이라 할 수 있음.}
\Que{소견서에 대하여 정해진 소정양식이 있는지 여부 및 양식이 없다면 요양기관에서 임의로 만들어서 환자에게 교부해도 무방한지?}
\Ans{소견서는 일반적으로 동일 의료기관에서 다른 과나 다른 의료기관에서 동일 환자의 진료에 참고할 수 있도록 하기 위하여 진료의사가 자신의 소견을 적은 것으로 \textcolor{red}{특별한 서식이나 기준이 없음}.}
\Que{국민건강보험법상 고시에 의하여 교부하는 소견서는 진찰료에 포함되어 있다는 의미는 진료 종료 후 당일 발급하는 소견서비용은 진찰료에 포함되어 있다는 것으로 해석하여 소견서 비용을 받지 않고 추후 환자나 가족 보험사(환자동의서 지침)등으로부터 요구 시 발급하는 소견서 비용을 받아도 무방한지? }
\Ans{환자나 가족, 보험사 등이 요구하는 소견서라는 것은 일반진단서를 말하는 것이라 보여지며 이를 교부시 본인이 그 비용을 부담해야 하는 사항임.}
\Que{학교나 보험회사 등 요구에 의해 진료확인서를 발급해야 하는지 여부와 발급비용에 대한 적용기준 있는지(병명을 기재하는 경우 또는 병원 내원일만 기재하는 경우 등)?}
\Ans{진료확인서는 본인이 신청하여 교부 받는 것이 타당하며 병명의 기재 등 기재사항에 대하여 규정하고 있지 아니함.}
\Que{진료확인서에 대하여 정해진 소정양식은 있는지 양식이 없다면 요양기관에서 임으로 작성하여 교부해도 문제가 없는지 와 교부 시 병명 등에 대해 자세히 기재해야 하는지?}
\Ans{진료확인서 등에 대하여는 별도의 서식을 정하고 있지 아니함.}
\medskip
\begin{figure}
\centering
\includegraphics[scale=.5]{visitconfirm}
\end{figure}

\begin{figure}
\centering
\includegraphics[scale=.5]{schoolvisitconfirm}\\
\end{figure}
