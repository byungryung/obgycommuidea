\section{RUBELLA IN PREGNANCY}
\subsection{Rubella IgG avidity test}
\begin{mdframed}[linecolor=blue,middlelinewidth=2]
노-395 (CZ395) Rubella 항체 결합력 검사[Rubella IgG Avidity test] \par
자궁내 태아감염을 일으킬 수 있는 rubella virus, CMV, Toxoplasma gondii 등에서 IgM이 양성을 보일 때 최근 또는 현감염과 과거감염이나 재감염과의 감별을 하기 위한 검사임.\par
[실시방법]\par
효소면역측정법(EIA법)-ELISA 검사법이며, 반정량검사임
제2007-36호 (2007.5.1)
\end{mdframed}
\prezi{\clearpage}
\subsection{Rubella IgG avidity test interpretation}
\begin{enumerate}\tightlist
\item IgG avidity< 40: 최근감염 또는 최근 백신 접종자
\item Borderline : 
\item IgG avidity>60: 과거 감염 또는 과거 백신 접종자 
\end{enumerate} 
IgG avidity:M easures the functional binding affinity of the IgG class of antibody in response to infection\par
Helps distinguish between primary and non-primary infection.\par
Acute Phase Reactant (< 2–3 months): Low avidity  Low avidity Index: 1–30\% \par
Chronic Phase Reactant (> 4 months): High Avidity High avidity Index: > 60\% \par
Rubella IgG avidity test < 40미만인 경우엔 유산수술 권유하고 delivery를 원하는 산모의 경우엔 codocentesis recommand.
\prezi{\clearpage}
\subsection{풍진과 예방접종}
\subsection{임신중 풍진예방접종에 대하여}\index{풍진!예방접종}
\Que{임신 7주인데 모르고 5일전에 풍진예방접종했는데  어떻게 하면 좋을까요?}
\Ans{아직까지 임신 초기 풍진 접종으로 기형아 발생은 전세계에  한건도 없었다고 합니다. 그걸 반드시 말해줘야겠네요 
그리고 그런 결과가 없었다  와  안 생긴다는 다름도 설명을 해줘야겠습니다.}
