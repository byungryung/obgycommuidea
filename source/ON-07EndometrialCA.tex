\section{자궁내막암의증}
\myde{}{%
\begin{itemize}\tightlist
\item[\dsjuridical] C541 Malignant neoplasm of corpus uteri, endometrium 의증(or 배제진단)
\item[\dsmedical] C8572 자궁내막조직생검-구획소파생검 \myexplfn{522.51} 원
\item[\dsmedical] R4521 자궁소파수술 \myexplfn{724.43} 원
\item[\dsmedical] \sout{C5915 Biopsy[1장기당]-생검(13개이상)} \myexplfn{687.25} 원
\item[\dsmedical] C5602 조직병리검사 [1장기당] -(Level B)
\item[\dsmedical] EB562 유도초음파 \myexplfn{887.80} 원
\item[\dsmedical] EB455010 여성생식기 일반초음파/도플러 \myexplfn{953.44} 원
\item[\dsmedical] EB457010 여성생식기 정밀초음파/도플러 \myexplfn{1397.03} 원 if 초음파상 (종양이나 기형이 있을때)
\item[\dsmedical] LA271 척수신경말초지 차단술-음부신경(Pudendal block) \myexplfn{225.10} 원
\item[\dsmedical] L0101 정맥마취(전신마취) \myexplfn{546.79} 원
\item[\dsmedical] 기타 lidocain or labo등 
\item[\dsmedical] 피펠 재료대는 임의로 별도 산정 불가합니다.
\item[\dsmedical] 유착방지제 (100/80 본인부담 80\%) :치료재료 신청 필요 

\end{itemize}
}%
{\begin{enumerate}[1.]\tightlist
\item 산정특례 질환이 의심되는 환자로 해당 산정특례 질환이 의심되어 ``초음파 개정 관련 Q\&A'' 16번 「여성생식기 초음파 수가 산정 방법」에 따라 해부학적 이상소견이 있는 경우(여성생식기 종괴, 여성생식기 기형, 종양), ``나-944 라. 여성생식기 초음파 (2) 정밀''로 산정 가능하며, 초음파 유도하 ``나-857 자궁내막조직생검''를 실시한 경우 ``초음파 개정 관련 Q\&A'' 33번 「유도초음파 산정방법」에 따라 ``유도초음파''를 산정할 수 있습니다.
\item 또한, ``보건복지부 고시 제2016-175호(2016.10.1일 시행)'' 「요양급여의 적용기준 및 방법에 관한 세부사항」``초음파 검사의 급여기준'' 2. 산정방법 ``나. 진단 초음파와 유도 초음파를 동시에 시행한 경우에는 각각의 소정점수를 산정함''이라고 명시하고 있어 동시 산정이 가능합니다.
\item 다만, ``초음파 개정 관련 Q\&A'' 44번 ``4대 중증질환이 의심되는 환자의 범위''가 명시되어 있으니 이점 참고하시어 급여대상여부의 판단이 선행되어야 상기 수가를 산정할 수 있음을 안내해드립니다.
\end{enumerate}} 
\subsection{구획 경관확장자궁소파술(Fractional curettage)}
자궁 각 부위를 조사하고 조직을 얻기 위한 검사입니다. 자궁경관을 확장하기 전에 내자궁경부 소파술을 시행하고, 다음에 자궁 경관을 확장하여 자궁내막의 이상 조직을 긁어내는 소파기나 숟가락 모양으로 구부린 기구로 자궁벽에 생긴 것을 조심스럽게 떼어냅니다. 약간의 고통이 있을 수 있어 보통 마취 하에 시행합니다. \par
