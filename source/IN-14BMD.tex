\section{골밀도검사의 인정기준}
\Que{62세 여자인데 골밀도 검사가 조정되었습니다. 이유가 뭔가요?}
\Ans{골밀도검사는 65세 이상의 여성과 70세 이상의 남성, 비정상적으로 1년 이상 무월경을 보이는 폐경전 여성 등에게 급여 기능합니다.}

\begin{commentbox}{골밀도검사의 인정기준}
다334 골밀도검사의 인정기준은 다음과 같이 함
\begin{enumerate}[가.]\tightlist
\item 적응증
	\begin{enumerate}[(1)]\tightlist
	\item 65세 이상의 여성과 70세 이상의 남성
	\item 고위험 요소가 1개 이상 있는 65세 미만의 폐경후 여성
	\item 비정상적으로 1년 이상 무월경을 보이는 폐경전 여성
	\item 비외상성(fragility) 골절
	\item 골다공증을 유발할 수 있는 질환이 있거나 약물을 복용중인 경우
	\item 기타 골다공증 검사가 반드시 필요한 경우
	\end{enumerate}	
\item 고위험요소
	\begin{enumerate}[1.]\tightlist
	\item 저체중(BMI < 18.5)
	\item 비외상성 골절의 과거력이 있거나 가족력이 있는 경우
	\item 외과적인 수술로 인한 폐경 또는 40세 이전의 자연 폐경
	\end{enumerate}
\item 산정횟수
	\begin{enumerate}[(1)]\tightlist
	\item 진단 시 - 1회 인정하되, 말단골 골밀도검사 결과 추가검사의 필요성이 있는 경우 1회에 한하여 central bone(spine, hip)에서 추가검사 인정함
	\item 추적검사
		\begin{enumerate}[(가)]\tightlist
		\item 추적검사의 실시간격은 1년 이상으로 하되, 검사 결과 정상골밀도로 확인된 경우는 2년으로 함
		\item 치료효과 판정을 위한 추적검사는 central bone(spine, hip)에서 실시한 경우에 한하여 인정함
		\item 위 (가), (나)의 규정에도 불구하고 스테로이드를 3개월 이상 복용하거나 부갑상선 기능항진증으로 약물치료를 받는 경우는 종전 골밀도검사 결과에 따라 아래와 같이 할 수 있으며, 이 경우 central bone(spine, hip)에서 시행함\newline
- 아 래 -
			\begin{itemize}\tightlist
			\item 정상골밀도(T-score ≥ -1)인 경우 : 첫 1년에 1회 측정, 그 이후부터는 2년에 1회
			\item T-score ≤ -3 인 경우 : 첫 1년은 6개월에 1회씩, 그 이후부터는 1년에 1회
			\end{itemize}
		\end{enumerate}
	\end{enumerate}
\end{enumerate}
\end{commentbox}

