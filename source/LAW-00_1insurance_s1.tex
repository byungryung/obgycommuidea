\subsection{[별표 2]\newindex{비급여대상(제9조제1항관련)}}
\begin{enumerate}[1.]\tightlist
\item 다음 각목의 질환으로서 \uline{업무 또는 일상생활에 지장이 없는 경우에 실시 또는 사용되는 행위\cntrdot{}약제 및 치료재료}
	\begin{enumerate}[가.]\tightlist
	\item 단순한 피로 또는 권태
	\item 주근깨\cntrdot{}다모(多毛)\cntrdot{}무모(無毛)\cntrdot{}백모증(白毛症)\cntrdot{}딸기코(주사비)\cntrdot{}점(모반)\cntrdot{}사마귀\cntrdot{}여드름\cntrdot{}노화현상으로 인한 탈모 등 피부질환
	\item 발기부전(impotence)\cntrdot{}불감증 또는 생식기 선천성기형 등의 비뇨생식기 질환
	\item 단순 코골음
	\item 질병을 동반하지 아니한 단순포경(phimosis)
	\item 검열반 등 안과질환
	\item 기타 가목 내지 바목에 상당하는 질환으로서 보건복지부장관이 정하여 고시하는 질환
	\end{enumerate}
\item 다음 각목의 진료로서 \uline{신체의 필수 기능개선 목적이 아닌 경우에 실시 또는 사용되는 행위\cntrdot{}약제 및 치료재료}
	\begin{enumerate}[가.]\tightlist
	\item 쌍꺼풀수술(이중검수술), 코성형수술(융비술), 유방확대\cntrdot{}축소술, 지방흡인술, 주름살제거술 등 \uline{미용목적의 성형수술과 그로 인한 후유증치료}
	\item 사시교정, 안와격리증의 교정 등 시각계 수술로써 시력개선의 목적이 아닌 외모개선 목적의 수술
	\item <삭제>
	\item 저작 또는 발음기능개선의 목적이 아닌 외모개선 목적의 악안면 교정술 및 교정치료
	\item 관절운동 제한이 없는 반흔구축성형술 등 외모개선 목적의 반흔제거술
	\item 안경, 콘텍트렌즈 등을 대체하기 위한 시력교정술
	\item 기타 가목 내지 바목에 상당하는 \uline{외모개선 목적의 진료로서 보건복지부장관이 정하여 고시하는 진료}
	\end{enumerate}
\item \uline{다음 각목의 예방진료로서 질병\cntrdot{}부상의 진료를 직접 목적으로 하지 아니하는 경우에 실시 또는 사용되는 행위\cntrdot{}약제 및 치료재료}
	\begin{enumerate}[가.]\tightlist
	\item \uline{본인의 희망에 의한 건강검진}(법 제47조의 규정에 의하여 공단이 가입자등에게 실시하는 건강검진 제외)
	\item \uline{예방접종(파상풍 혈청주사 등 치료목적으로 사용하는 예방주사 제외)}
	\item 구취제거, 치아 착색물질 제거, 치아 교정 및 보철을 위한 치석제거 및 구강보건증진 차원에서 정기적으로 실시하는 치석제거
	\item 불소국소도포, 치면열구전색 등 치아우식증 예방을 위한 진료(치아우식증에 이환되지 않은 순수 건전치아를 가진 만 6세 이상 14세 이하 소아의 제1대구치에 대한 치면열구전색 제외)
	\item 멀미 예방, 금연 등을 위한 진료
	\item \uline{유전성질환 등 태아의 이상유무를 진단하기 위한 세포유전학적검사}
	\item 기타 가목 내지 마목에 상당하는 예방진료로서 보건복지부장관이 정하여 고시하는 예방진료
	\end{enumerate}
\item 보험급여시책상 \uline{요양급여로 인정하기 어려운 경우} 및 그 밖에 \uline{건강보험급여원리에 부합하지 아니하는 경우로서} 다음 각목에서 정하는 \uline{비용\cntrdot{}행위\cntrdot{}약제 및 치료재료}
	\begin{enumerate}[가.]\tightlist
	\item 가입자 등이 다음 각 항목 중 어느 하나의 요건을 갖춘 요양기관에서 \uline{1개의 입원실에 3인 이하가 입원할 수 있는 병상(이하 ``상급병상"이라 한다)}을 이용함에 따라 영 제24조제2항 및 제8조제4항의 규정에 의하여 고시한 상대가치점수로 산정한 입원료(이하 ``기본입원료"라 한다) 외에 추가로 부담하는 입원실 이용 비용
    		\begin{enumerate}[(1)]\tightlist
			\item \uline{의료법령에 따라 허가를 받거나 신고한 병상 중 기본입원료만 산정하는 일반병상(이하 “일반병상”이라 한다)을 다음의 구분에 따른 비율 이상을 확보하여 운영하는 경우.} 다만, 규칙 제12조제3항 또는 제4항에 따라 제출한 요양기관현황통보서 또는 요양기관변경통보서 상의 격리병실, 무균치료실, 특수진료실 및 중환자실과 「의료법」 제27조제3항제2호에 따른 외국인환자를 위한 전용 병실 및 병동의 병상은 일반병상 및 상급병상의 계산에서 제외한다.
     			\begin{enumerate}[(가)]\tightlist
				\item 의료법령에 따라 신고한 \uline{병상이 10병상을 초과하는} 「의료법」 제3조제2항제1호에 따른 \uline{의원급 의료기관과} 같은 항 제2호에 따른 \uline{병원급 의료기관(종합병원 및 상급종합병원은 제외한다)}: \uline{환자 6인 이상이} 입원할 수 있는 일반병상의 입원료(이하 “최저입원료"라 한다)만으로 산정하는 일반병상을 \uline{50퍼센트 이상 확보할 것}. 이 경우 「암관리법」 제22조에 따라 완화의료전문기관으로 지정된 요양기관에서 같은 법 제24조에 따라 완화의료 입원진료를 받는 경우에는 환자 5인이 입원할 수 있는 일반병상의 입원료(이하 ``5인실 입원료"라 한다)를 최저입원료로 한다. 
     			\item 「의료법」 제3조제2항제3호마목에 따른 종합병원(상급종합병원을 포함한다): 70퍼센트
				\end{enumerate}
   			\item \uline{의료법령에 의하여 신고한 병상이 10병상 이하인 경우}
			\end{enumerate}
	\item 「의료법」 제3조제2항제3호마목에 따른 종합병원(상급종합병원을 포함한다): 다음 표에 따른 비율 
	
	\medskip%par
	\tabulinesep =_2mm^2mm
	\begin{tabu} to\linewidth {|X[2,l]|X[2,l]|} \tabucline[.5pt]{-}
	\rowcolor{ForestGreen!40}  \centering 구 분 &	\centering 확보 비율 \\ \tabucline[.5pt]{-}
	\rowcolor{Yellow!40} 보건복지부령 제30호 국민건강보험 요양급여의 기준에 관한 규칙 일부개정령 부칙 제3조제1항 각 호에 해당하는 종합병원
 & 최저입원료만으로 산정하는 일반병상을 50퍼센트 이상 확보할 것 \\ \tabucline[.5pt]{-}
	\rowcolor{Yellow!40} 보건복지부령 제30호 국민건강보험 요양급여의 기준에 관한 규칙 일부개정령 부칙 제3조제2항 각 호에 해당하는 종합병원
 & 최저입원료만으로 산정하는 일반병상을 50퍼센트 이상 확보하고, 2010. 12. 23. 이후 신설 병상 중 일반병상을 70퍼센트 이상 확보할 것 \\ \tabucline[.5pt]{-}
	\rowcolor{Yellow!40} 그 외의 모든 종합병원  & 일반병상을 70퍼센트 이상 확보할 것. 이 경우 최저입원료만으로 산정하는 일반병상이 총 병상의 50퍼센트 이상이어야 한다. \\ \tabucline[.5pt]{-}
	\end{tabu}
	\item 법 제46조에 의하여 장애인에게 보험급여를 실시하는 보장구를 제외한 보조기\cntrdot{}보청기\cntrdot{}안경 또는 콘택트렌즈 등 보장구
	\item \uline{보조생식술(체내\cntrdot{}체외인공수정 포함)시 소요된 비용}
	\item \uline{친자확인을 위한 진단  }
	\item 치과의 보철(보철재료 및 기공료 등 포함)
	\item 및 아. 삭제 <2002.10.24>
	\item 이 규칙 제8조의 규정에 의하여 보건복지부장관이 고시한 약제에 관한 급여목록표에서 정한 일반의약품으로서 「약사법」 제23조에 따른 조제에 의하지 아니하고 지급하는 약제
	\item 삭제 <2006.12.29>
	\item 「의료법」 제46조에 따른 선택진료를 받는 경우에 선택진료에 관한 규칙에 따라 추가되는 비용
	\item 「장기등 이식에 관한 법률」에 따른 장기이식을 위하여 다른 의료기관에서 채취한 골수 등 장기의 운반에 소요되는 비용
	\item 「마약류 관리에 관한 법률」 제40조에 따른 마약류중독자의 치료보호에 소요되는 비용
	\item 이 규칙 제11조제1항 또는 제13조제1항의 규정에 의하여 \uline{요양급여대상 또는 비급여대상으로 결정\cntrdot{}고시되기 전까지의 신의료기술} 등. 다만, 제11조제4항 또는 제13조제1항 후단의 규정에 의하여 소급하여 요양급여대상으로 적용되는 신의료기술 등을 제외한다. 
	\item 그 밖에 요양급여를 함에 있어서 비용효과성 등 \uline{진료상의 경제성이 불분명하여 보건복지부장관이 정하여 고시하는 검사\cntrdot{}처치\cntrdot{}수술 기타의 치료 또는 치료재료}
	\end{enumerate}
\item 삭제 <2006.12.29>
\item 영 별표 2 제2호의 규정에 의하여 보건복지부장관이 정하여 고시하는 질병군에 대한 입원진료의 경우에는 제1호 내지 제4호(제4호 하목을 제외한다), 제7호에 해당되는 행위\cntrdot{}약제 및 치료재료. 다만, 제2호 사목, 제3호 사목, 제4호거목은 다음 각목에서 정하는 경우에 한한다.
	\begin{enumerate}[가.]\tightlist
	\item \uline{보건복지부장관이 정하여 고시하는 행위 및 치료재료}
  	\item \uline{질병군 진료 외의 목적으로 투여된 약제}
	\end{enumerate}
\item 건강보험제도의 여건상 요양급여로 인정하기 어려운 경우
	\begin{enumerate}[가.]\tightlist
	\item 보건복지부장관이 정하여 고시하는 한방물리요법
  	\item 한약첩약 및 기상한의서의 처방 등을 근거로 한 한방생약제제
	\end{enumerate}
\item 약사법령에 따라 허가를 받거나 신고한 범위를 벗어나 약제를 처방\cntrdot{}투여하려는 자가 보건복지부장관이 정하여 고시하는 절차에 따라 의학적 근거 등을 입증하여 비급여로 사용할 수 있는 경우. 다만, 제5조제3항에 따라 중증환자에게 처방\cntrdot{}투여하는 약제 중 보건복지부장관이 정하여 고시하는 약제는 건강보험심사평가원장의 공고에 따른다.
\end{enumerate}
