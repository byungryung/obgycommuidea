\section{습관성유산}
\myde{}{
\begin{itemize}\tightlist
\item[\dsjuridical] N96 습관적 유산자
\item[\dsjuridical] E079 상세불명의 갑상선의 장애
\item[\dsjuridical] E282 다낭성 난소증후군
\item[\dsjuridical] N979 상세불명의 여성불임
\item[\dsjuridical] O089 유산, 자궁외임신 및 기태임신에 따른 상세불명의 합병증
\end{itemize}
}
{}

\tabulinesep =_2mm^2mm
\begin {longtabu} to\linewidth {|X[3,l]|X[6,l]|X[1,l]|X[1,l]|X[4,l]|} \tabucline[.5pt]{-}
\rowcolor{ForestGreen!40} \centering 코드 & \centering 처방명칭 & 수량 & 횟수 & 용법 \\ \tabucline[.5pt]{-}
\rowcolor{Yellow!40} C3500 & 난포자극호르몬 & 1 & 1 & \\ \tabucline[.5pt]{-}
\rowcolor{Yellow!40} C3480 & 황체형성호르몬 & 1 & 1 & \\ \tabucline[.5pt]{-}
\rowcolor{Yellow!40} C3260 & 에스트라디올 & 1 & 1 & \\ \tabucline[.5pt]{-}
\rowcolor{Yellow!40} C3360 & 갑상선자극호르몬TSH & 1 & 1 & E079(추가) \\ \tabucline[.5pt]{-}
\rowcolor{Yellow!40} C3510 & 프로락틴 & 1 & 1 & \\ \tabucline[.5pt]{-}
\rowcolor{Yellow!40} C3530 & 테스토스테론 & 1 & 1 & E282(추가) \\ \tabucline[.5pt]{-}
\rowcolor{Yellow!40} C3640 & DHEA-S & 1 & 1 & E282(추가) \\ \tabucline[.5pt]{-}
\rowcolor{Yellow!40} C3520 & 베타에이취씨지 & 1 & 1 & o089(추가) \\ \tabucline[.5pt]{-}
\rowcolor{Yellow!40} C3290 & 트리요도타이로닌 & 1 & 1 & E079 \\ \tabucline[.5pt]{-}
\rowcolor{Yellow!40} C3460 & 프로게스테론 & 1 & 1 & N979 \\ \tabucline[.5pt]{-}
\rowcolor{Yellow!40} C3340 & 유리싸이록신FreeT4 & 1 & 1 & E079 \\ \tabucline[.5pt]{-}
\rowcolor{Yellow!40} C6001006 & 염색체검사(배양검사포함) & 2 & 1 & 100/100 \\ \tabucline[.5pt]{-}
\rowcolor{Yellow!40} C5031 & 항카디오리핀항체(선별) & 1 & 1 & 12 or 13-16 \\ \tabucline[.5pt]{-}
\rowcolor{Yellow!40} C5032 & 항카디오리핀항체(확진)-IgG & 1 & 1 & 12 or 13-16 \\ \tabucline[.5pt]{-}
\rowcolor{Yellow!40} C5033 & 항카디오리핀항체(확진)-IgM & 1 & 1 & 12 or 13-16 \\ \tabucline[.5pt]{-}
\rowcolor{Yellow!40} C5034 & 항카디오리핀항체(확진)-IgA & 1 & 1 & 12 or 13-16 \\ \tabucline[.5pt]{-}
\rowcolor{Yellow!40} B1711 & 루프스항응고인자(선별검사) & 1 & 1 & 17 or 18 \\ \tabucline[.5pt]{-}
\rowcolor{Yellow!40} B1712006 & 루프스항응고인자(확진검사) & 1 & 1 & 17 or 18 \\ \tabucline[.5pt]{-}
\rowcolor{Yellow!40} BY802 & 플라즈미노겐 항활성체(면역학적) & 1 & 1 & 100/100 \\ \tabucline[.5pt]{-}
\rowcolor{Yellow!40} C3410 & 인슐린 & 1 & 1 & 100/100 \\ \tabucline[.5pt]{-}
\rowcolor{Yellow!40} CZ133 & 호모시스테인 & 1 & 1 & 인정비급 \\ \tabucline[.5pt]{-}
\rowcolor{Yellow!40} CY628006 & 종합효소연쇄반응-제한효소.. & 1 & 1 & 100/100 \\ \tabucline[.5pt]{-}
\rowcolor{Yellow!40} CY633006 & 종합효소연쇄반응-제한효소.. & 1 & 1 & 100/100 \\ \tabucline[.5pt]{-}
\rowcolor{Yellow!40} B1842 & Protein-S(면역학적) & 1 & 1 & 100/100 \\ \tabucline[.5pt]{-}
\rowcolor{Yellow!40} B1691006 & 혈액응고인자정량(기능적).. & 1 & 1 & 100/100 \\ \tabucline[.5pt]{-}
\rowcolor{Yellow!40} C5120006 & 세포표지검사[단세포군항체별로.. & 1 & 1 & 100/100 \\ \tabucline[.5pt]{-}
\rowcolor{Yellow!40} BY301006 & 기타미생물배양-마이코플라즈마 & 1 & 1 & N760 or 100/100 \\ \tabucline[.5pt]{-}
\rowcolor{Yellow!40} BY302006 & 기타미생물배양-클라미디아 & 1 & 1 & 상 동 \\ \tabucline[.5pt]{-}
\rowcolor{Yellow!40} BY303006 & 기타미생물배양-유레아플라즈마 & 1 & 1 & 상 동 \\ \tabucline[.5pt]{-}
\end{longtabu}
\medskip

\begin{enumerate}\tightlist
\item C6001006[비급여] 염색체 검사 : 부모의 혈액을 통해 부모의 염색체 이상을 검사하여  전좌 등의 염색체 이상을 가진 부부의 경우 착상 전 유전진단 (preimplantation genetic diagnosis, PGD)이 효과적인 방법으로 이용 되고 있다.
\item 초음파검사[인정비급여]/HSG[비급여] :자궁의 해부학적 이상에 의한 습관성 유산의 대표적인 기형이 중격자궁 (septated uterus) 이며 이는 쌍각자궁 및 이중자궁에 비교하여 유산 발생율이 60-80\%로 보고 되고 있으나 자궁내시경 등을 이용한 적절한 치료시 유산의 재발 방지율이 90\%에 이르는 효과적인 치료가 가능한 자궁기형이다. 반면 쌍각자궁, 이중자궁, 단각자궁 등의 기형은 수술적 치료가 용이 하지는 않으나 중격자궁과 비교 하여 유산의 발생 정도는 30 - 40\%로 비교적 높지 않다.
\item 호르몬검사[상병입력후 급여] :호르몬 이상 체내에서 발생할 수 있는 여러 호르몬 분비 이상 중 갑상선 호르몬 분비이상, 당뇨, 다낭성 난소 증후군 그리고 황체기 결함 등이 습관성 유산과 연관 된 것으로 알려져 있으며 이는 해당 호르몬의 조절을 통해 치료가 가능하다.
\item 배양검사[상병입력후 급여] : 감염(Infectious causes)
감염과 반복자연유산과의 연관성에 대해서는 많은 연구가 진행되지 못하였으며 아직 논란의 여지가 많다. 여러 가지 감염원 중 mycoplasma, ureaplasma, chlamidia 그리고 β-streptococcus와 자연유산과의 관계에 대한 연구가 진행 되었지만 아직 유산과의 집적적인 연관성 및 그 기전에 대한 명확한 결론은 좀더 연구가 필요한 상황이며 항생제 투약을 통해 치료를 시행하고 있다.
\item 유전성 호혈전증 (inherited thrombophilia)[급여] : 반복자연유산 환자의 일부는 혈전성 경향이 강한 유전적 성향을 가지며 이 경우 자궁-태반 혈류의 감소 및
태반혈관의 혈전형성이 생겨 결국 태아는 혈액공급을 받지 못해 유산 된다는 개념으로서 최근 많은 연구가 진행되고 있으나 논란 또한 많은 원인이며 그 치료로 저 용량 아스피린 (low dose heparin), 저 분자량 헤파린(low molecular weight heparin) 등이 사용되고 있다.(항카디오리핀항체와 루프스항응고인자검사)
\item 기타검사들[급여여부 불확실/인정비급여] : 반복자연유산의 동종면역학적 원인에 대한 본격적인 연구가 시행되기 전까지 전체 환자의 약 50\%의 경우에서 원인을 알 수 없는 원인불명으로 분류 되었다.비교적 최근 면역학적 진단 및 치료에 대한 연구가 진행되면서 과거 원인불명으로 분류되었던 경우의 상당 부분이 동종면역의 면역학적 원인으로 진단 되고 면역조절치료 등 그 원인에 해당되는 치료가 시행되고 있으며 그 이론적 배경 에는 1950년대 Medawar 등의 태아가 모체 내에서 면역계의 공격을 견뎌내고 안전하게 성장할 수 있는 기전에 대한 연구 결과인
	\begin{enumerate}[i)]\tightlist
	\item 모체 면역세포가 태아항원에 대한 무반응(anergy) 또는 면역관용을 획득한다는 가설,
	\item 태아와 모체간 해부학적 장벽에 의해 모체 면역세포가 태아세포에 접근할 수 없다는 가설, 그리고
	\item 태아세포 스스로 동종항체의 발현을 억제한다는 가설 등에 근거하고 있다(Billing RE 등, 1953).
	\end{enumerate}
하지만 Medawar의 가설은 초창기 생식면역학의 기본 개념을 제시하였지만 태아항원과 모체 면역계 사이에 이루어지는 면역관용 현상을 설명하는 데는 부족함이 많았다. 이후, 자연살해세포 (natural killer, NK cell), 자연살해 T 세포(natural killer T, NKT cell), 면역조절 T 세포
(regulatory T, Treg cell), 단핵세포 (monocyte), 수지상세포 (dendritic cell), 대식세포 (Macrophage) 등 다양한 종류의 모체 면역세포들이 면역관용에 관여한다는 연구들이 보고되었고 (Aluvihare VR 등, 2004). 이들 면역세포들은 자궁 내의 태아와 모체가접촉하는 ``태아-모체 접촉면, feto-maternal interface" 에 decidual associated lymphoid tissue (DALT)라는 임신 고유의 조직학적 구조물 내에 존재하며 태아에 대한 모체의 면역관용을 획득하는데 중요한 역할을 하고, 착상 및 임신 유지에 수반되는 혈관생성 및 영양막의 발달에 필수적인 무균성 염증반응 (sterile inflammation)을 유발하는 것으로 보고 되었다(Piccini 등, 2002). 최근 습관성 유산의 면역학적 원인의 진단의 표식자로는
	\begin{enumerate}[ⅰ)]\tightlist
	\item 말초혈액 NK cell의 수적 증가와 독성의 증가 정도(degree of peripheral blood NK cell fraction 및 cytotoxicity),
	\item 말초혈액 염증성 면역세포의 우세 여부(degree of Intracellular Th1/2 T cell ratio)
	\item 말초혈액 염증성 면역반응 우세 여부(degree of Th1/2 cytokine ratio)
	\end{enumerate}
등이 대표적인 표식자로 연구 되고 있지만 임신 중 면역현상에 대한 연구수행의 어려움에 따른 대단위 전향적 연구의 부족 및 연구 결과의 불일치로 인해 현재까지 습관성 유산의 면역학적 진단에 대해서는 많은 논란이 있는 상태이다
\end{enumerate}