\section{자궁근종:전체적인 개관}
\subsection{자궁근종:치료방법 요약}
\tabulinesep =_2mm^2mm
\begin {tabu} to \linewidth {|X[1,l]|X[1,l]|X[1,l]|} \tabucline[.5pt]{-}
\rowcolor{ForestGreen!40} 방 법 & 효 과 &	단 점 \\ \tabucline[.5pt]{-}
\rowcolor{Yellow!40} 피임약 & 생리양 감소, 생리통 감소 & 매일 복용 \\ \tabucline[.5pt]{-}
\rowcolor{Yellow!40} 미레나 & 생리양 감소, 생리통 감소 & 인체 삽입, 부정출혈, IUD고유 합병증 \\ \tabucline[.5pt]{-}
\rowcolor{Yellow!40} GnRH agonist & 일시적 무월경, 근종 크기 감소 & 단기간 사용만 가능, 폐경 증상\\ \tabucline[.5pt]{-}
\rowcolor{Yellow!40} 이니시아 & 생리양 감소, 생리통 감소 & 단기간 사용만 가능\\ \tabucline[.5pt]{-}
\rowcolor{Yellow!40} Tranaxamic acid & 생리양 감소 & 일시적 효과 \\ \tabucline[.5pt]{-}
\rowcolor{Yellow!40} NSAIDS & 생리통 감소 & 일시적 효과 \\ \tabucline[.5pt]{-}
\rowcolor{Yellow!40} 자궁내막소작술 & 생리양 감소 & 비가역적, 가임력 상실\\ \tabucline[.5pt]{-}
\rowcolor{Yellow!40} Myolysis & 증상 호전, 종괴 괴사 & 유착형성, 합병증 \\ \tabucline[.5pt]{-}
\rowcolor{Yellow!40} Myomectomy & 증상호전, 종괴 제거 & 재발 가능성 \\ \tabucline[.5pt]{-}
\rowcolor{Yellow!40} Hysterectomy & 증상호전, 종괴 제거 & 가임력 상실, 수술 합병증  \\ \tabucline[.5pt]{-}
\rowcolor{Yellow!40} Uterine atery embolization & 증상호전, 종괴 감소 & 치료 실패 \\ \tabucline[.5pt]{-}
\rowcolor{Yellow!40} HIFU & 증상호전, 종괴 괴사 & 치료 실패 \\ \tabucline[.5pt]{-}
\end{tabu}%rowcolor{Yellow!40} & & \\ \tabucline[.5pt]{-}

\subsection{치료방법의 비교}
\includegraphics[width=.85\linewidth]{myoma.png}