\section{전산심사의 법적근거}
제5조(요양급여의 적용기준 및 방법)
\begin{enumerate}[①]\tightlist
\item 요양기관은 가입자등에 대한 요양급여를 별표 1의 요양급여의 적용기준 및 방법에 의하여 실시하여야 한다.
\end{enumerate}
【별표 1】 요양급여의 적용기준 및 방법
\begin{enumerate}[1.]\tightlist
\item 요양급여의 일반원칙
	\begin{enumerate}[다.]\tightlist
	\item 요양급여는 경제적으로 비용효과적인 방법으로 행하여야 한다.
	\end{enumerate}
\item 진찰ㆍ검사, 처치ㆍ수술 기타의 치료
	\begin{enumerate}[가.]
	\item 각종 검사를 포함한 진단 및 치료행위는 진료상 필요하다고 인정되는 경우에 한하여야 하며 연구(제8조의2에 따른 임상연구는 제외한다)의 목적으로 하여서는 아니된다.
	\end{enumerate}
\item 약제의 지급
	\begin{enumerate}[가.]\tightlist
	\item 처방ㆍ조제
		\begin{enumerate}[(2)]\tightlist
		\item 의약품은 약사법령에 의하여 허가 또는 신고된 사항(효능․효과 및 용법․용량 등)의 범위 안에서 환자의 증상 등에 따라 필요ㆍ적절 하게 처방․투여 하여야 한다(중략).
		\end{enumerate}
		\begin{enumerate}[(6)]\tightlist
		\item 진료상 2품목 이상의 의약품을 병용 처방․투여하는 경우에는 1품목의 처방․투여로는 치료효과를 기대하기 어렵다고 의학적으로 인정되는 경우에 한한다.
		\end{enumerate}
	\end{enumerate}	
\end{enumerate}

심사평가원에서는「내분비 및 순환계통의 질환」, 「피부 및 피하조직의 질환」, 「관절병증 및 연조직장애」,「손상에 의한 질환」및「구강의 질환」에 대하여 보건복지부 고시, 심사지침, 의약품 허가사항 등을 반영한 전산 심사를 개발하여 모니터링을 실시하고 있으며, 2016년 12월 1일 접수분 부터 심사적용 할 예정입니다.

\subsection{점차로 옥죄오는 전산심사}
\url{http://www.monews.co.kr/news/articleView.html?idxno=92422}