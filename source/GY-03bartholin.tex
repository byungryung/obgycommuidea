\section{바르톨린샘의 낭종/농양}
\myde{}{%
\begin{itemize}\tightlist
\item[\dsjuridical] N750 바르톨린샘의 낭 Cyst of Bartholin’s gland
\item[\dsjuridical] N751 바르톨린샘의 농양 Abscess of Bartholin’s gland
\item[\dsjuridical] Z988 기타 명시된 수술후 상태
\item[\dsmedical] R4050 바도린선농양절개술[\myexplfn{589.45} 원](외음부 또는 회음부농양배농술을 실시한 경우에도 소정점수를 산정한다)
\item[\dsmedical] R4060 바도린선낭종절제술[\myexplfn{1174.26} 원]
\item[\dsmedical] R4065 바돌린선 낭종 조대술 [\myexplfn{1106.29} 원]
\item[\dsmedical] R4103 질식배농술-질벽혈종제거 [\myexplfn{468.41} 원]- 주사기로 aspiration한 경우 가능합니다
\item[\dsmedical] M0111 단순처치는 외래추적조사중에 사용가능.
\end{itemize}
}%
{
\Que{양측 바르톨린 농양 (Bartholin's Gland Abscess)( N750/N751/N758/N759 )이 있어 같은날 바르톨린 조대술 (bartholine marsupialization operation) 양측 시행한 경우 자-406-1 R4065 바도린선낭종조대술 Marsupialization of Batholin's Gland의 200\%(2배) 산정이 가능한가요?}

\Ans{문의하신 자406-1 바도린선낭종조대술은 별도의 인정기준이 없으며, 아래의「건강보험 행위 급여\cntrdot{}비급여 목록표 및 급여 상대가치점수」제1편 제2부 제9장 제1절 [산정지침]을 참고하여 환자의 상태에 따라 청구하여 주시기 바랍니다.\par

\emph{-아 래-}
\begin{enumerate}[(5)]\tightlist
\item 대칭기관에 관한 처치 및 수술 중 "양측"이라고 표기한 것은 "양측"을 시술할지라도 소정점수만 산정한다. 
\item 동일 피부 절개 하에 2가지 이상 수술을 동시에 시술한 경우 주된 수술은 소정점수에 의하여 산정하고, 제2의 수술부터는 해당 수술 소정점수의 50\%, 상급종합병원\cntrdot{}종합병원은 해당 수술 소정점수의 70\%를 산정한다. 다만, 주된 수술 시에 부수적으로 동시에 실시하는 수술의 경우에는 주된 수술의 소정점수만 산정한다
\end{enumerate} 
바르톨린 조대술 (bartholine marsupialization operation)이 양측 표시된 수술(행위)가 아니며,  좌우  바르톨린 조대술 (bartholine marsupialization operation)은 동일 절개 부위가 아니므로 제생각에는  200\%(2배) 산정이 합리적일거 같습니다,}
}