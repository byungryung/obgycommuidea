\section{AUB:PREMENOPAUSE}
\myde{}{%
\begin{itemize}\tightlist
\item[\dsjuridical] N915 상세불명의 희발월경
\item[\dsjuridical] N921 불규칙적 주기를 가진 과다 및 빈발 월경 
\item[\dsjuridical] C541 자궁체부의 악성신생물, 자궁내막 의증(or 배제진단)
\item[\dsjuridical] E079 상세불명의 갑상선의 장애
\item[\dsjuridical] E282 다낭성 난소증후군 
\item[\dsjuridical] E249 상세불명의 쿠싱증후군 
\item[\dsjuridical] O089 유산, 자궁외임신 및 기태임신에 따른 상세불명의 합병증
\item[\dsjuridical] A638 기타 명시된 주로 성행위로 전파되는 질환	
\item[\dsjuridical] D50 철결핍빈혈 
\item[\dsjuridical] D51	비타민B12결핍빈혈, D52 엽산결핍빈혈  
\end{itemize}   
}%
{
\begin{itemize}\tightlist
\item Check point 
	\begin{enumerate}\tightlist
	\item Confirm that bleeding is uterine 
	\item Rule out diseases of urethra, bladder, vagina, vulva, hemorrhoids,  IBS
	\item Rule out bleeding d/t medications  
	\item Drug medication Hx
		\begin{itemize}\tightlist
		\item  anticoagulants, aspirin 
		\item  antidepressants(SSRI) 
		\item   hormone therapy, tamoxifen  
		\item  steroids
		\item  contraceptives 
		\item  thyroxin  
		\item   Herb: gingseng, ginkgo, soy products  
		\end{itemize}
	\item Rule out systemic illness           
		\begin{itemize}\tightlist
		\item  thyroid disease  
		\item   PCOS
		\item  coagulpathies
		\item  leukemia 
		\item   thrombocytopenia  
		\item  pituitary adenoma
		\item   hypothalamic suppression 
		\item  hepatic disease, renal disease, adrenal hyperplasia, Cushing disease  
		\end{itemize}
	\item Pregnancy?
	\end{enumerate}
\item 필요검사류
	\begin{enumerate}\tightlist
	\item  Serum studies
		\begin{itemize}\tightlist
		\item  Pregnancy test 
		\item  CBC: determine degree of anemia 
		\item  TFT, prolactin, coagulation tests 
		\item  Progesterone: day 21 – 23 to verify ovulatory status
		\item  FSH/LH: verify menopausal status or rule out PCOS  
		\end{itemize}
	\item  Cervical cytology and cervical cultures 
		\begin{itemize}\tightlist
		\item  Should be performed on all women to exclude cervical cancers  
		\end{itemize}
	\item  Transvaginal sonography  
		\begin{itemize}\tightlist
		\item  Irregular menstrual bleeding should be investigated for polyps and submucosal fibroids  
		\end{itemize}
	\item   Endometrial biopsy 
		\begin{itemize}\tightlist
		\item  Should be performed on all women >35yrs. to rule out endometrial cancer or premalignant lesion  
		\item  18-35yrs: risk factors(family history of ovarian, breast, colon, endometrial cancer, tamoxifen use, chronic anovulation, obesity, estrogen therapy, prior endometrial hyperplasia, diabetes)  
		\end{itemize}
	\end{enumerate}
\end{itemize}
}