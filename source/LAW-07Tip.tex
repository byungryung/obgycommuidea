\section{급여 대상 환자에게 인정비급여 행약치는 가능하다.}
\myde{자궁경부 미란이 심해서 알보칠로 약물소작술을 시행하고 알보칠 청구가능여부}{
\begin{itemize}\tightlist
\item[\dsjuridical] R4300 자궁경부(질)약물소작술
\item[\dsjuridical] N86 자궁목의 미란 및 외반증
\item[\dsjuridical] N72 자궁목의 염증성 질환
\end{itemize}
}{
산부인과 의원입니다
원내에서 자궁경부 미란이나 만성 자궁경부염, 생검이나 수술후 알보칠 콘센트레이트 액 (720mg/g)을 이용하여 자궁경부 약물 소작술을 시행합니다. 이런 경우 알보칠 콘센트레이트 액 (720mg/g)은 비급여 등재 약품으로 R4300 자궁경부(질)약물소작술의 급여 산정외에 알보칠 콘센트레이트 액 (720mg/g)의 비급여 산정이 가능한가요?\par
알보칠 콘센트레이트 액 (720mg/g)관련 원내 비급여 고지후 산정 가능한가요?\par

\emph{답변]}\par
문의하신 약제는 보건복지부 고시(약제급여 목록 및 상한금액표)되지 않은 비급여 약제이며
비급여대상은 건강보험에 해당되지 않아 병원에서 정하는 진료수가(약가)에 의해 산정하는 항목입니다.\par
아울러, 비급여 고지여부는 「의료법」 제45조 및 동법 「시행규칙」제42조의2의 규정에 따라 환자 또는 환자의 보호자가 쉽게 알수 있도록 보건복지부령으로 정하는 바에 따라 고지하여야 함을 알려드립니다. 
}
\subsection{질염 환자에 비급여 질정 사용하고 원내고시된 비급여 가격으로 받아도 된다.} \label{Vag_Tab}
\begin{itemize}\tightlist
\item 세균성질염 : 세나서트질정 2mg (일반약)
\item 세균성질염, 비특이성 질염(손상된 질내 세균총의 정상화) : 바지씨질정 Ascorbic Acid 250mg (일반약)
\item 칸디다질염 : 래피탈 질정 카네스텐 질정, Clotrimazole 100mg (일반약)
\end{itemize}

\subsection{임질환자에 트리악손을 비급여로 사용한다.}