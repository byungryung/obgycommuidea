\subsection{질병군 급여 일반원칙(주된수술 이외의 수술시)}
\begin{myshadowbox}
\begin{enumerate}[19.]\tightlist
\item 질병군 진료시 질병군 분류번호를 결정하는 \textcolor{red}{주된 수술 이외에 제1편(행위별 수가)제2부제9장제1절(기본처치 제외) 또는 제10장제3절 제4절의 수술을 실시한 경우에는 해당 수술 소정점수를 추가 산정한다. 다만, 주된 수술과 동일 피부 절개 하에 실시되는 수술은 해당 수술 소정점수의 70\%를 산정}한다.
\end{enumerate}
\end{myshadowbox}
\prezi{\clearpage}
\Que{질병군으로 입원진료 중 환자가 원하여 시행하는 요실금수술은 별도 산정 가능한가요?}
\Ans{질병군 수가에는 급여와 보건복지부장관이 정하여 고시하는 비급여가  있으며, 환자가 요실금수술을 원하는 경우라도 수술이 요양급여대상 인지 비급여대상인지를 객관적인 검사를 통하여 결정되어야 하며, 비급여 으로 결정이 되면 해당 수술비용과 비급여 수술에 사용 된 치료재료 등은 별도로 받을 수 있습니다.\par
※ 요실금 수술 보건복지부 고시 제2011-144호(11.11.25)\par
만약 \textcolor{red}{요양급여대상이면 해당수술 소정점수만을 보존받을수 있습니다. 결과적으론 재료에 대해선 보존받을수 없습니다.}}
\prezi{\clearpage}
\par
\medskip
\Que{제왕절개분만 질병군에서 ‘분만 전 처치(자-437)’, ‘분만 후 처치(자437-1)’, ‘제왕절개술 전 질식분만시도(자451-1)’ 등의 처치를 실시한 경우 질병군 소정점수 이외에 추가산정 여부}
\Ans{질병군 진료기간 중 질병군 분류번호를 결정하는 주된 수술 이외에 다른 수술을 실시한 경우 질병군 급여상대가치점수 이외에 다른 수술료의 추가산정이 가능함.\par
‘분만 전 처치(자-437)’, ‘분만 후 처치(자437-1)’, ‘제왕절개술 전 질식분만시도(자451-1)’ 등은 분만을 목적으로 시행되는 처치 등으로 서로 \textcolor{red}{다른 수술에 해당하지 않으므로 관련 수기료는 추가 산정할 수 없음}}
\prezi{\clearpage}
\begin{commentbox}{추가로 산정할 수 있는 수술의 범위}
질병군 진료시 질병군 분류번호를 결정하는 주된 수술이외에 \emph{추가로 산정할 수 있는 수술의 범위}는?\par
질병군 분류번호를 결정하는 주된 수술이외에 제1편제2부제9장제1절(기본처치 제외) 및 제10장제3절, 제4절의 수술을 실시한 경우 해당 수술의 수기료를 추가 산정함. 또한, 질병군 주진단과 다른 상병으로 실시되는 수술에 해당하는 경우 산정하며 비위관삽관술 등 질병군 주된 수술 및 그 외 실시되는 수술의 과정에 발생되는 처치 등은 해당되지 않음. 또한 각종 처치(위세척, 질강처치, 직장맛사지, 피부과처치, 자궁내장치삽입술 등), 체외충격파쇄석술, 혈액투석, 응급처치 등은 산정이 불가함
\end{commentbox}
\prezi{\clearpage}
\leftrod{질병군 진료시 주된 수술 이외에 수술을 추가로 실시한 경우 질병군 요양급여비용 청구방법}
질병군 주된 수술 이외에 실시한 수술은 특정내역 MT007(DRG세부내역)의 내역구분 ‘COP'에 기재하여 청구하고 질병군 수술 이외에 실시한 수술료에 종별가산한 금액을 양급여비용총액1에 합하여 기재함\par
\emph{(작성요령)}\par
기타 자궁적출술(N04200) 질병군 진료시 자궁적출술(R4145)과 난소부분절제술(R4430)을 시행한 경우 

\prezi{\clearpage}\par
\medskip
\Que{부인과적 개복수술 또는 기타 개복수술시 병변 없이 실시한 충수절제술 인정여부}
\Ans{개복 수술시 병변이 없는 상태에서 시행한 충수절제술(Incidental Appendectomy)은 별도 산정 할 수 없음\par
☞ 고시 제2014-126호(‘14.7.30.) 「부인과적 개복수술 또는 기타 개복수술시 병변없이 실시한 충수절제술 인정여부」}
\prezi{\clearpage}