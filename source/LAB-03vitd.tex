\section{Vitamin D}
\myde{}{
\begin{itemize}\tightlist
\item[\dsjuridical] E55.9 상세불명의 비타민 D 결핍증(Vitamin D deficiency, unspecified)
\item[\dsjuridical] M81.9 상세불명의 골다공증 Osteoporosis, unspecified
\item[\dschemical] CY154  기타 비타민[D2]  \myexplfn{85} 원
\item[\dschemical] CY155 기타 비타민[D3] \myexplfn{85} 원
\item[\dschemical] CY170 기타 비타민[총 비타민D] \myexplfn{85} 원
\item[\dschemical] CY704 기타 비타민[D2]-핵의학적방법  \myexplfn{159} 원
\item[\dschemical] CY700 기타 비타민[총 비타민D]-핵의학적방법 \myexplfn{159} 원
\item[\dschemical] CY705 기타 비타민[D3]-핵의학적방법 \myexplfn{159} 원
\end{itemize}
}
{
\leftrod{너153 Vit D3 검사는 다음에 해당되는 환자에게 비타민 D 결핍이 의심되어 실시한 경우에 인정함.}\par
\begin{center}\textbf{- 다          음 -}\end{center}
\begin{enumerate}[가.]\tightlist
\item 적응증
\begin{enumerate}[1)]\tightlist
\item Vit D의 흡수장애를 유발할 수 있는 위장질환 및 흡수장애 질환
\item 항경련제(Phenytoin 이나 Phenobarbital 등) 또는 결핵약제 투여받는 환자
\item 간질환                    
\item 신장질환                 
\item 악성종양
\item Vit D 결핍성 구루병
\item 이차성 골다공증의 원인 감별이 필요한 경우
\item 골다공증 진단 후 약물치료 시작 전 1회, 비타민 D 투여 3~6개월 후 약제 효과 판정을 위해 실시 시 1회 인정함을 원칙으로 하되, 이 후 추적검사는 연 2회까지 인정
\item 체표면적 40\% 이상 화상
\end{enumerate}
\item 기타
\begin{enumerate}[1)]\tightlist
\item 비타민 D (D2, D3 및 total D) 검사는 1종만 인정
\item 선별 검사로 HPLC법(너153주1)은 인정하지 아니함(2016년 11월 1일)
\end{enumerate}
\end{enumerate}
이차성 골다공증의 원인 감별이 필요한 경우는, \uline{비타민디검사CY155는 임상의의 판단하에 의증으로 넣고 검사 진행해도 됩니다만 임상적으로 일괄적인 검사보다는 선별검사}가 바람직합니다. \highlight{본인이 원하는 검사나 상기기준에서 벗어나는 경우  당연히 비급여 입니다,} 비급여를 꼭 100/100 만 받을필요없습니다. 비급여 고시후 적정금액 받으시면 됩니다.\par
비타민D는 30이상이면 정상, 10.0 이하일 때 결핍증에 해당되므로 맞아서 문제가 없고 10.0 이상에서 투여했다면 25(OH)vitD 검사를 내일 바로 보내서 혈중농도를 체크하세요.
}

\subsection{Vit D검사와 임산부}
최근 다양한 임신 합병증(전자간증, 임신성 당뇨병, 조산 등)과 비타민D의 연관성에 대하여 아래와 같은 여러 연구결과들이 보고되고 있다.\par
첫째, 남자와 여자 모두에서 비타민D의 수치와 수태능력(fertility)이 관련되어 있는 것으로 나타났다. 비타민 D 부족은 수태능력의 감소를 유발시키는 것으로 생각되고, 한 동물 실험에서는 비타민 D의 결핍이 암놈 쥐의 수태능력을 75\%까지 떨어뜨리는 것으로 나타났다.

\subsection{VItamin D처방에 대한 tip}
\begin{enumerate}\tightlist
\item 비오엔 주사도 비타민 D 결핍이 20 이하니 그러면 실비
\item 전 폐경기 호르몬 치료 시작전에 채혈할때,   상세불명의 비타민 결핍증 코드 넣고 cy155 routine으로 합니다. 거의 다 결핍나와서 호르몬 처방할때  비오엔  3개월마다 주사합니다. 실비도 되고, 호르몬 치료 효과도 더 좋은듯 해요.
\item 전 검사를 100/100 냅니다..6000 원.
\item 먹는약 써니디로 처방합니다. 성인도 주사보다는 써니디로 많이 처방합니다. 그게 흡수율이 좋다고 들어서요
\item 비타민 d 주사 지용성이라 뻑뻑하고 근육 많이 풀어줘야 되는게 좀 단점이죠
\item 저도 써니디 많이 줬는데, 생각보다 잘 안 챙겨 드시더라구요..폐경 여성은 호르몬제  3개월 단위로 올때 같이 주사 놔주니 compliance가 좋아서요
\item 비오앤도 공급가가 지방이랑, 서울이랑 현저히 차이 나서, 전 비타디본 으로 바꿨어요..휴온스 메리트 d도 좀 더 싸더라구요.
\end{enumerate}


\leftrod{25-OH-비타민 (25-hydroxycholecalciferol)} 검사 의의 \par
비타민 D3는 간에서 hydroxylation되어 25-hydroxycholecalciferol로 전환되며 Vitamin D3는 calcium 및 phosphorus 의 대사에 중요한 역할을 함. 
25-hydroxycholecalciferol의 감소를 보이는 질환은 비타민 D 결핍성 구루병, 흡수부전, phenytoin이나 phenobarbital 복용, 담즙성 간경변증, 신증후군 등이 있으며 본 검사는 비타민 D 결핍성 구루병, 흡수부전,
 phenytoin이나 phenobarbital 복용, 담즙성 간경변증, 신증후군 등에 의한 25-hydroxycholecalciferol의 결핍에 의해 칼슘 및 인의 대사이상이 있을 경우 이를 진단하기 위해 유 용한 검사임.



\leftrod{1,25-Dihydroxy 비타민 D3 (1,25-dihydroxycholecalciferol)} 검사 의의 \par
음식 중의 비타민 D는 생리적으로 활성형이 되려면 간에서 25-hydroxycholecalciferol로 전환되고, 
이것이 신장에서 다 시 1,25-hydroxycholecalciferol로 전환됨.
 1,25-dihydroxy 비타민 D3의 감소를 보이는 질환은 비타민 D결핍성 구루병, 흡수부전, 만성신부전, 부갑상선기능저하증, 종양성 골연화증, 갑상선기능항진증, 신증후군, 미숙아, 노인 등이고 
 증가를 보이는 질환은 원발성 부갑상선기능항진증, 말단거대증, 특발성 갑상선기능저하증, 고칼슘뇨증, 임신등임. 비타민 D 의 결핍시에는 소아에서는 골격의 성장 부위가 칼슘 침착이 없으므로 뼈의 변형을 초래할 수 있으며 성인의 경우는 뼈의 성장이 완료되어 있으므로 골연화증(osteomalacia)이 일어나 게 됨. 검사는 전술한 비타민 D 감소를 보이는 질환과 증가 를 보이는 질환에서 칼슘 및 인의 대사이상을 판정하기 위해
유용한 검사임.


