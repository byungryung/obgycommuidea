\begin{mdframed}[linecolor=blue,middlelinewidth=2]  
제1편 행위 급여 \cntrdot{}  비급여 목록 및 급여 상대가치점수 >> 제2부 행위 급여 목록\cntrdot{} 상대가치점수 및 산정지침 >> 제5장 마취료
\end{mdframed}
\subsection{\newindex{마취료〔산정지침〕}}
\begin{enumerate}[(1)]\tightlist
\item 마취약제 주사 시 사용한 1회용 주사기 및 주사침 등의 재료대는 마취료 소정점수에 포함되므로 별도 산정하지 아니한다.
\item 신생아 마취시에는 마취료 소정점수의 60\%를 가산하며, 만8세 미만의 소아 또는 만70세 이상의 노인의 경우에는 마취료 소정점수의 30\%를 가산한다.(산정코드 첫 번째 자리에 신생아는 1, 만8세 미만은 3, 만70세 이상은 4로 기재)
\item 장기이식수술마취2), 심폐체외순환법마취5), 일측폐환기법마취6), 고빈도제트 환기법마취7), 개흉적 심장수술마취8), 뇌종양, 뇌혈관질환에 대한 개두술마취9)시에는 마취료 소정점수의 50\%를 가산한다.(산정코드 첫 번째 자리에 각각 2, 5, 6, 7, 8, 9로 기재)
\item 18시-09시 또는 공휴일에 응급진료가 불가피하여 마취를 행한 경우에는 소정점수의 50\%를 가산한다.(산정코드 두 번째 자리에 18시-09시는 1, 공휴일은 5로 기재) 이 경우 해당 마취를 시작한 시각을 기준으로 산정한다.
\item 수술 중에 발생하는 우발사고에 대한 처치(산소흡입, 응급적 인공호흡) 또는 주사(강심제) 등의 비용은 별도 산정할 수 있으나, 그 밖의 경우에는 산소 흡입, 응급적 인공호흡비용 및 EKG monitoring료는 산정하지 아니한다.
\item 동일 목적을 위하여 2 이상의 마취를 병용한 경우 또는 마취 중에 다른 마취법으로 변경한 경우에는 주된 마취의 소정점수만 산정한다.
\item 제6장에 분류되지 아니한 표면마취, 침윤마취 및 간단한 전달마취의 비용은 제2장, 제9장 또는 제10장에 분류한 소정 시술료에 포함되므로 별도 산정 하지 아니한다.
\item 마취통증의학과 전문의 초빙료를 산정하는 경우에는 초빙된 마취통증의학과 전문의의 면허종류, 면허번호를 요양급여비용 청구명세서에 기재하고, 마취 통증의학과 전문의가 서명 또는 날인한 마취기록지를 비치하여야 한다.
\end{enumerate}