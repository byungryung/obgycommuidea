\section{포괄수가제}
\subsection{7개 질병군 포괄수가제란? (2013년 7월부터)}
환자가 입원해서 퇴원할 때까지 발생하는 진료에 대하여 질병마다 미리 정해진 금액을 내는 제도입니다. 입원비가 하나로 묶여있다고 생각하시면 됩니다. (\textcolor{red}{같은 질병이라도 환자의 합병증이나 타상병 동반여부에 따라 가격은 달라질 수} 있습니다.)\\
\prezi{\clearpage}
\emph{적용대상질병군}\\ 
현재는 4개 진료과 7개 질병군을 대상으로 시행중
\begin{itemize}\tightlist
\item 안과 : 백내장수술(수정체 수술) 
\item 이비인후과 : 편도수술 및 아데노이드 수술 
\item 외과 : 항문수술(치질 등), 탈장수술(서혜 및 대퇴부), 맹장수술(충수절제술) 
\item 산부인과 : \textcolor{blue}{제왕절개분만, 자궁 및 자궁부속기(난소, 난관 등)수술(악성종양 제외), 자궁외임신제외}
\end{itemize}
※ 수정체수술(백내장수술), 서혜 및 대퇴부 탈장수술(장관절제 미동반) 등 간단한 항문수술의 경우에는 6시간 미만 관찰 후 당일 귀가 또는 이송시에도 포괄수가제(DRG설명보기)가 적용되어 본인부담금은 입원부담률인 20\%로 적용받게 됩니다. 다만 7개 질병군에 해당되는 수술을 받았어도 \textcolor{red}{의료급여 대상자 및 혈우병 환자와 HIV감염자(인체면역결핍바이러스병)}는 포괄수가제(DRG) 적용에서 제외됩니다.
\prezi{\clearpage}
\begin{Cdoing}{우리나라 포괄수가제}
질병군(DRG) 포괄수가는 국민건강보험법시행령 제21조제3항제2호에 따라 복지부장관이 별도 고시하는 7개 질병군으로 입원진료를 받은 경우에 적용하며, 질병군 입원진료는 질병군 급여 일반원칙에 따라 다음의 항목을 포함하고 있습니다.\\
- 다 음 -
\begin{itemize}\tightlist
\item 7개 질병군으로 응급실ㆍ수술실 등에서 수술을 받고 연속하여 6시간 이상 관찰 후 귀가 또는 이송한 경우 
\item 7개 질병군 중 수정체수술(대절개 단안 및 양안, 소절개 단안 및 양안), 기타항문수술, 서혜 및 대퇴부탈장수술 단측 및 양측(복강경 이용 포함)의 수술을 받고 6시간 이상 관찰 후 당일 귀가 또는 이송한 경우
\end{itemize}
\end{Cdoing}

\prezi{\clearpage}
\section{제1부 질병군 급여 일반원칙}
\begin{enumerate}[1.]\tightlist
\item 상급종합병원, 종합병원, 병원(요양병원을 포함한다), 의원(보건의료원을 포함한다)인 요양기관이 국민건강보험법 시행령(이하 “영”이라 한다) 제 21조 제3항제2호 및 국민건강보험 요양급여의 기준에 관한 규칙(이하 “요양급여기준”이라 한다) 제8조제3항에 따라 포괄적인 행위가 적용되는 질병군에 대한 입원진료를 하는 경우에 적용한다.
\item 가입자 또는 피부양자(이하 “가입자 등”이라한다)가 질병군으로 입원진료를 받은 경우에 적용하되, 다음의 각 항목은 질병군 적용에서 제외하고 제 1편을 적용한다.
	\begin{enumerate}[가.]\tightlist
	\item 혈우병환자, HIV감염자
	\item 입원일수가 30일을 초과할 경우 31일째부터 발생하는 진료분
	\item 차상위 본인부담경감대상자로서 제3호 나목에 해당하는 경우
	\item 질병군 진료 이외의 목적으로 입원하여 입원일수가 6일을 초과한 시점에 예상치 못하게 질병군 수술이 이루어진 경우 입원일로부터 수술시행일 전일까지의 진료분
	\end{enumerate}
\item 제2호 규정에 따른 질병군 입원진료에는 다음의 각 항목을 포함한다.
	\begin{enumerate}[가.]\tightlist
	\item 제2부 각 장에 분류된 질병군으로 응급실ㆍ수술실 등에서 수술을 받고 연속하여 6시간 이상 관찰 후 귀가 또는 이송한 경우
	\item 제2부 각 장에 분류된 질병군 중 수정체 소절개 수술 단안, 수정체 소절개 수술 양안, 수정체 대절개 수술 단안, 수정체 대절개 수술 양안, 기타항문 수술, 서혜 및 대퇴부 탈장수술(장관절제 미동반) 단측, 서혜 및 대퇴부 탈장수술(장관절제 미동반) 양측, 복강경을 이용한 서혜 및 대퇴부 탈장수술(장관절제 미동반) 단측, 복강경을 이용한 서혜 및 대퇴부 탈장수술(장관절제 미동반) 양측 질병군으로 수술을 받고 6시간 미만 관찰 후 당일 귀가 또는 이송하는 경우
	\end{enumerate}
\item 제2부 각 장에 분류된 질병군 상대가치점수(이하 “점수”라 한다)는 다음 각목의 행위ㆍ약제 및 치료재료를 포함한다.
	\begin{enumerate}[가.]\tightlist
	\item 제1편 행위 급여ㆍ비급여 목록 및 급여 상대가치점수에서 정한 행위급여목록표에 고시된 행위
	\item 요양급여기준 제8조제2항의 규정에 의하여 고시된 약제 급여 목록 및 급여 상한금액표의 약제와 치료재료 급여ㆍ비급여 목록 및 급여 상한 금액표의 치료재료
	\item 요양급여기준 별표 2의 비급여대상 중 제6호의 비급여대상을 제외한 행위ㆍ약제 및 치료재료
	\item 국민건강보험법 시행규칙 별표 6의 본인이 요양급여비용의 100분의 100을 부담하는 항목 중 제1호 자목에 해당하는 항목을 제외한 행위ㆍ약제 및 치료재료
	\item 다음 항목 중 위 가목 내지 라목에 해당하는 경우
		\begin{enumerate}[(1)]\tightlist
		\item 요양급여기준 별표 1 제1호 마목에서 장관이 정하는 바에 따라 다른 기관에 검사를 위탁하거나 당해 요양기관에 소속되지 아니한 전문성이 뛰어난 의료인을 초빙하거나, 또는 다른 요양기관에서 보유하고 있는 양질의 시설ㆍ인력 및 장비를 공동 사용하는 경우 소요되는 행위ㆍ약제 및 치료재료
		\item 입ㆍ퇴원 당일에 발생한 행위ㆍ약제 및 치료재료로써 외래진료 및 퇴원약제 등을 포함하되 다음 항목은 제외한다.
			\begin{enumerate}[(가)]\tightlist
			\item 질병군 입원을 예견하지 못한 상태에서 입원 당일 외래진료를 받은 경우의 원외처방 약제비
			\item 질병군으로 퇴원 후 질병군과 관계없는 상병으로 퇴원 당일 외래진료를 받은 경우의 원외처방 약제비
			\item 질병군으로 퇴원 후 질병군 질환과 관계없는 상병으로 퇴원 당일 재입원하는 경우의 요양급여비용
			\end{enumerate}
		\item 요양기관의 요구에 의하여 가입자 등이 외부에서 직접 구입한 약제 및 치료재료
		\end{enumerate}
	\end{enumerate}	
\item 질병군에 대한 요양급여비용을 산정할 때에는 제2부 각 장에 분류된 질병군 점수를 기준으로 별표 1의 질병군별 점수 산정요령에 의하여 산정된 점수 총합에 국민건강보험법 제45조제3항과 영 제21조제1항에 따른 점수당 단가를 곱하여 10원 미만을 절사한 금액을 요양급여비용 총액으로 산정한다.\par
이 경우 위 금액 외에 식대를 포함한 별도로 산정하는 비용이 있는 경우에는 각각의 산정방식에 의하여 산정된 금액을 합산한다.
\item 제5호 본문에도 불구하고 질병군별 금액 산정시 점수당 단가는 별표 2의 질병군 행위 및 약제ㆍ치료재료 구성비율에 따른 행위부분 점수와 매년 상한금액 변화를 적용한 약제ㆍ치료재료 금액을 점수당 단가로 나눈 점수를 합한 점수(소수점 이하 셋째 자리에서 4사5입)에 적용한다.\par
<산식> \par
질병군별 금액 = \{질병군별 행위 점수 + (약제ㆍ치료재료 금액 ÷ 점수당단가)\} × 점수당 단가
\item 제5호에 따라 산정한 요양급여비용의 총액이 영 제21조제1항 내지 제3항및 요양급여기준(별표 2 제6호를 제외한다)에 의하여 산정한 총액보다 적고 그 차액이 100만원을 초과하는 경우(이 경우를 요양급여비용열외군이라 한다)에는 위 제5호에 따른 금액에 100만원을 초과하는 금액(10원 미만 절사)을 합한 금액을 요양급여비용 총액으로 산정한다.
\item 가입자 또는 피부양자가 제1호에 따른 요양기관(제3편을 적용받는 요양병원은 제외)에서 「국민건강보험법」제43조에 따라 신고한 일반입원실 및 정신과폐쇄병실의 4인실 또는 5인실을 이용한 경우에는 별표 2의3의 추가 비용 계산식에 따른 금액을 추가 산정하고, 상급종합병원의 일반입원실 및 정신과폐쇄병실의 1인실(보건복지부장관이 정하여 고시하는 불가피한 1인실 입원의 경우 제외)을 이용한 경우에는 제5호 본문에 따른 금액에서 1인실 이용일수에 해당하는 기본입원료(제1편제2부제1장 가-2-가)를 제외 하고 산정한다.
\item 영 별표 2 제2호 나목의 “보건복지부장관이 정하여 고시하는 입원실을 이용한 경우”라 함은 가입자 등이 제1호에 따른 요양기관에서 국민건강보험법 제43조에 따라 신고한 일반입원실 및 정신과폐쇄병실의 4인실 또는 5인실을 이용한 경우를 말하며, 별표 2의3의 본인부담액 계산식에 따른 금액을 더하여 본인부담액을 산정한다.
\item 영 별표 2 제2호 다목의 “그 고시에서 정한 금액”이라 함은 제7호 중 100만원 초과분에 해당하는 금액을 말한다.
\item (별표 2의1)에 열거한 항목을 외과 전문의가 시행한 경우에는 소정점수의 30\%에 대한 각 요양기관별 종별가산율을 적용한 금액을 추가 산정한다.
\item 18시-09시 또는 공휴일에 응급진료가 불가피하여 수술을 행한 경우에는 해당 질병군의 야간ㆍ공휴 소정점수를 추가 산정한다. 이 경우 수술 또는 마취를 시작한 시간을 기준으로 산정한다.
\item 질병군 요양급여를 실시하는 요양기관은 질병군 입원환자의 질병군 분류 번호와 관련한 주진단 및 기타진단, 수술명 등은 진료기록부에 근거하여 정확한 코드를 부여하여야 하며, 진단명이 입원시부터 존재하였는지 여부를 확인할 수 있도록 진료기록부에 기록하고, 의료의 질 향상을 위한 점검표를 별지 서식에 따라 작성하여야 한다.
\item 입원 중인 환자를 제2부 각 장에 분류된 질병군 중 수정체 소절개 수술 단안, 수정체 소절개 수술 양안, 수정체 대절개 수술 단안, 수정체 대절개 수술 양안의 진료를 위해 다른 요양기관으로 의뢰하여 질병군 진료를 실시한 경우 해당 요양급여비용은 의뢰받은 요양기관에서 질병군으로 적용 한다.
\item 질병군 진료 시 초음파검사는 「요양급여의 적용기준 및 방법에 관한 세부사항」제2장 검사료 초음파검사 세부인정기준을 적용하며, \highlight{인정기준에 의한 급여대상에 해당되는 경우에는} 제2부 각 장에 분류된 질병군 점수이외에 제1편 제2부 초음파검사료를 추가 산정한다. 
	\begin{itemize}\tightlist
	\item EZ986 분만기간 초음파 : 분만을 위한 입원기간 동안 발생한 초음파 검사를 모두 의미함. 제왕절개를 위해 입원한 환자들의 경우는 옆의 비급여초음파를 최소한 2번 이상 실시하고 청구한다.
	\item EZ887 초음파를 이용한 태아 생물리학 계수( Biophysical Profile )
	\item 임신 유지목적으로 입원하여 6일 이내에 제왕절개분만이 이루어진 경우 : 분만기간 초음파(비급여)로 청구한다. 조산통으로 입원한 경우엔 2일에 한번씩은 초음파를 본다.
	\item 임신 유지목적으로 입원하여 입원일수가 6일을 초과한 시점에서 예상치 못하게 제왕절개분만이 이루어진 경우 
		\begin{itemize}\tightlist
		\item 입원(행위별 청구) : (정상임신부) 7회까지 급여, 그 외 비급여(태아의 이상이나 이상이 예측되는 경우) 급여
		\item 분리청구 시점 구분
		\item 제왕절개분만 입원(DRG 청구) : 분만기간 초음파(비급여)
         \end{itemize}                               
	\item 분만과 연결된 입원: 분만기간이 장기로 길어진 경우 분리청구 시점 기준으로 적용
		\begin{itemize}\tightlist
		\item 입원(행위별 청구) : (정상임신부) 7회까지 급여, 그 외 비급여(태아의 이상이나 이상이 예측되는 경우) 급여
		\item 분리청구 시점 구분
		\item 자연분만및 제왕절개분만 입원제왕절개분만 입원(DRG 청구) : 분만기간 초음파(비급여)
         \end{itemize} 	
%	\item 질병군 진료 이외의 목적으로 입원하여 입원일수가 6일을 초과한 시점에 예상치 못하게 질병군 수술이 이루어진 경우 입원일로부터 수술시행일 전일까지의 진료분을 제외한 경우의 보험 초음파등(6일전의 조기진통등으로 입원하여 제왕절개분만한 경우 횟수초과 급여초음파 청구 (해당 삼분기의 일반 또는 일반의 제한초음파 산정). 단, 1일 1회만 청구 가능함)
     \end{itemize}
\item 별표 2의4에 열거한 항목에 해당하는 행위 및 치료재료는 제1편 제2부 행위 급여 상대가치점수와 「약제 및 치료재료의 비용에 대한 결정기준」에 의한 금액을 추가 산정한다.
\item 영 별표 2 제4호에 따른 요양급여 항목 및 본인부담률은 별표 2의5와 같다. 이 경우 별표 2의5에 열거한 항목에 해당하는 행위 및 치료재료는 「요양급여의 적용기준 및 방법에 관한 세부사항」을 적용하며, 인정기준에 의한 급여대상에 해당되는 경우에는 제1편 제2부 행위 급여 상대가치 점수와 「약제 및 치료재료의 비용에 대한 결정기준」에 의한 금액을 추가 산정한다.
\item 질병군 진료시 마취통증의학과 전문의를 초빙하여 마취를 실시한 경우에는 제1편제2부제6장 바-2의 마취통증의학과 전문의 초빙료를 추가 산정하며, 제1편제2부제6장 및「요양급여의 적용기준 및 방법에 관한 세부사항」의 마취통증의학과 전문의 초빙료 산정 관련 규정을 적용한다.
\item 질병군 진료시 질병군 분류번호를 결정하는 주된 수술 이외에 제1편제2부 제9장제1절(기본처치 제외) 또는 제10장제3절ㆍ제4절의 수술을 실시한 경우에는 해당 수술 소정점수를 추가 산정한다. 다만, 주된 수술과 동일 피부 절개 하에 실시되는 수술은 해당 수술 소정점수의 70\%를 산정한다.
\item 질병군 진료 시 제1편제2부제1장 5.가에 따른 의료질평가지원금은 가-22의 각 분야별 등급별 ‘입원’의 소정점수를 질병군 입원일수와 동일하게 추가 산정한다.
\item 질병군 진료 시 제1편제2부제19장제2절에 따른 (별표 2) 및 (별표 3)의 응급의료행위를 실시하는 경우에는 제1편에서 정하고 있는 해당 소정점수의 50\%를 추가 산정하고, 제1편제2부제19장제2절의 산정지침 3. 내지 5. 및 「요양급여의 적용기준 및 방법에 관한 세부사항」을 적용한다.
\item「의료법」제3조의5에 따라 전문병원으로 지정받은 의료기관에서 질병군 진료 시 제1편제2부제1장 산정지침 6.에 따른 전문병원 관리료 등은 가-24-가 전문병원 입원관리료와 가-24-1-가 전문병원 입원의료질지원금의 해당 소정점수를 질병군 입원일수에 따라 추가 산정한다.
\item 질병군 진료 시 감염예방ㆍ관리 활동을 실시하는 경우에는 제1편제2부제1장 가-25의 감염예방ㆍ관리료를 추가 산정하고, 「요양급여의 적용기준및 방법에 관한 세부사항」을 적용한다.
\item 질병군 진료시 통증자가조절법(Patient Controlled Analgesia)을 실시한 경우 제1편제2부 행위 급여 상대가치점수와 「약제 및 치료재료의 비용에 관한 결정기준」에 의한 금액을 추가 산정하고, 「요양급여의 적용기준및 방법에 관한 세부사항」을 적용한다.
\end{enumerate}
