\subsection{\newindex{제41조(요양급여)}}
\begin{enumerate}[①]\tightlist
\item 가입자와 피부양자의 질병, 부상, 출산 등에 대하여 다음 각 호의 요양급여를 실시한다.
	\begin{enumerate}[	1.]\tightlist
	\item 진찰\cntrdot{}검사
	\item 약제(藥劑)\cntrdot{}치료재료의 지급
	\item 처치r수술 및 그 밖의 치료
	\item 예방\cntrdot{}재활
	\item 입원
	\item 간호
	\item 이송(移送)
	\end{enumerate}
\item 제1항에 따른 \uline{요양급여(이하 ``요양급여"라 한다)의 방법\cntrdot{}절차\cntrdot{}범위\cntrdot{}상한 등의 기준은 보건복지부령으로 정한다.}
\item 보건복지부장관은 제2항에 따라 요양급여의 기준을 정할 때 \uline{업무나 일상생활에 지장이 없는 질환, 그 밖에 보건복지부령으로 정하는 사항은 요양급여의 대상에서 제외}할 수 있다.
\end{enumerate}

\subsection{\newindex{제5조(요양급여의 적용기준 및 방법)}}
\emph{국민건강보험 요양급여의 기준에 관한 규칙  제5조(요양급여의 적용기준 및 방법)}
\begin{enumerate}[①]\tightlist
\item 요양기관은 가입자등에 대한 요양급여를 \uline{별표 1의요양급여의 적용기준 및 방법에 의하여 실시}하여야 한다.
\item 제1항의 규정에 의한 요양급여의 적용기준 및 방법에 관한 세부사항과 조혈모세포이식 \uline{요양급여의 적용기준 및 방법에 관한 세부사항은} 의약계\cntrdot{}공단 및 건강보험심사평가원의 의견을 들어 \uline{보건복지부장관이 정하여 고시}한다.<개정 2008\cntrdot{}3\cntrdot{}3, 2010\cntrdot{}3\cntrdot{}19, 2010\cntrdot{}12\cntrdot{}23>
\item 제2항의 규정에 불구하고 「국민건강보험법 시행령」(이하 ``영"이라 한다) 별표 2 제3호마목에 따른 중증질환자(이하 ``중증환자"라 한다)에게 처방\cntrdot{}투여하는 약제중 보건복지부장관이 정하여 고시하는 약제에 대한 요양급여의 적용기준 및 방법에 관한 세부사항은 제5조의2의 규정에 의한 중증질환심의위원회의 심의를 거쳐 건강보험심사평가원장이 정하여 공고한다. 이 경우 건강보험심사평가원장은 요양기관 및 가입자등이 해당 공고의 내용을 언제든지 열람할 수 있도록 관리하여야 한다. <신설 2005\cntrdot{}10\cntrdot{}11, 2008\cntrdot{}3\cntrdot{}3, 2010\cntrdot{}3\cntrdot{}19, 2012ㆍ8ㆍ31>
\end{enumerate}

\paragraph{[별표 1]요양급여의 적용기준 및 방법(제5조제1항관련)}
\begin{enumerate}[1.]\tightlist
\item \highlight{요양급여의 일반원칙}
	\begin{enumerate}[가.]\tightlist
	\item 요양급여는 가입자 등의 \uline{연령\cntrdot{}성별\cntrdot{}직업 및 심신상태 등의 특성을 고려하여 진료의 필요가 있다고 인정되는 경우에 정확한 진단을 토대로 하여 환자의 건강증진을 위하여 의학적으로 인정되는 범위 안에서 최적의 방법으로 실시하여야 한다.}
	\item 요양급여를 담당하는 의료인은 의학적 윤리를 견지하여 환자에게 심리적 건강효과를 주도록 노력하여야 하며, 요양상 필요한 사항이나 예방의학 및 공중보건에 관한 지식을 환자 또는 보호자에게 이해하기 쉽도록 적절하게 설명하고 지도하여야 한다.
	\item 요양급여는 \uline{경제적으로 비용 효과적인 방법으로 행하여야 한다.}
	\item 요양기관은 \uline{가입자 등의 요양급여에 필요한 적정한 인력\cntrdot{}시설 및 장비를 유지하여야 한다. 이 경우 보건복지부장관은 인력\cntrdot{}시설 및 장비의 적정기준을 정하여 고시할 수 있다.}
	\item 라목의 규정에 불구하고 가입자 등에 대한 최적의 요양급여를 실시하기 위하여 필요한 경우, 보건복지부장관이 정하여 고시하는 바에 따라 다른 기관에 검사를 위탁하거나, 당해 요양기관에 소속되지 아니한 전문성이 뛰어난 의료인을 초빙하거나, 다른 요양기관에서 보유하고 있는 양질의 시설\cntrdot{}인력 및 장비를 공동 활용할 수 있다.
	\end{enumerate}
\item \highlight{진찰\cntrdot{}검사, 처치\cntrdot{}수술 기타의 치료}
	\begin{enumerate}[가.]\tightlist
	\item \uline{각종 검사를 포함한 진단 및 치료행위는 진료상 필요하다고 인정되는 경우에 한하여야 하며 연구의 목적으로 하여서는 아니된다.}
	\item 영 제21조제3항제2호에 따라 보건복지부장관이 정하여 고시하는 질병군에 대한 입원진료의 경우 그 입원진료 기간동안 행하는 것이 의학적으로 타당한 검사\cntrdot{}처치 등의 진료행위는 당해 입원진료에 포함하여 행하여야 한다.
	\end{enumerate}
\item 약제의 지급 
	\begin{enumerate}[가.]\tightlist
	\item 처방\cntrdot{}조제
		\begin{enumerate}[(1)]\tightlist
		\item 영양공급\cntrdot{}안정\cntrdot{}운동 그 밖에 요양상 주의를 함으로써 치료효과를 얻을 수 있다고 인정되는 경우에는 의약품을 처방\cntrdot{}투여하여서는 아니되며, 이에 관하여 적절하게 설명하고 지도하여야 한다.
		\item 의약품은 약사법령에 의하여 \large{허가 또는 신고된 사항(효능\cntrdot{}효과 및 용법\cntrdot{}용량 등)의 범위 안에서 환자의 증상 등에 따라 필요\cntrdot{}적절하게 처방\cntrdot{}투여하여야 한다. (약제 전산심사)}다만, \uline{안전성\cntrdot{}유효성 등에 관한 사항이 정하여져 있는 의약품 중 진료상 반드시 필요하다고 보건복지부장관이 정하여 고시하는 의약품의 경우에는 허가 또는 신고된 사항의 범위를 초과하여 처방\cntrdot{}투여할 수 있으며}, 중증환자에게 처방\cntrdot{}투여하는 약제로서 보건복지부장관이 정하여 고시하는 약제의 경우에는 건강보험심사평가원장이 공고한 범위 안에서 처방\cntrdot{}투여할 수 있다.
		\item \uline{요양기관은 중증환자에 대한 약제의 처방\cntrdot{}투여시 해당약제 및 처방\cntrdot{}투여의 범위가 (2)의 허용범위에는 해당하지 아니하나 해당환자의 치료를 위하여 특히 필요한 경우에는 건강보험심사평가원장에게 해당약제의 품목명 및 처방\cntrdot{}투여의 범위 등에 관한 자료를 제출한 후 건강보험심사평가원장이 중증질환심의위원회의 심의를 거쳐 인정하는 범위 안에서 처방\cntrdot{}투여할 수 있다.}
		\item 제10조의2제2항에 따라 식품의약품안전처장이 긴급한 도입이 필요하다고 인정한 품목의 경우에는 식품의약품안전처장이 인정한 범위 안에서 처방\cntrdot{}투여하여야 한다.
		\item 항생제\cntrdot{}스테로이드제제 등 오남용의 폐해가 우려되는 의약품은 환자의 병력\cntrdot{}투약력 등을 고려하여 신중하게 처방\cntrdot{}투여하여야 한다.
		\item \uline{진료상 2품목 이상의 의약품을 병용하여 처방\cntrdot{}투여하는 경우에는 1품목의 처방\cntrdot{}투여로는 치료효과를 기대하기 어렵다고 의학적으로 인정되는 경우에 한한다.}
		\end{enumerate}  
	\item 주사
		\begin{enumerate}[(1)]\tightlist
		\item \uline{주사는 경구투약을 할 수 없는 경우, 경구투약시 위장장애 등의 부작용을 일으킬 염려가 있는 경우, 경구투약으로 치료효과를 기대할 수 없는 경우 또는 응급환자에게 신속한 치료효과를 기대할 필요가 있는 경우에 한한다.}
		\item \uline{동일 효능의 내복약과 주사제는 병용하여 처방\cntrdot{}투여하여서는 아니된다. 다만, 경구투약만으로는 치료효과를 기대할 수 없는 불가피한 경우에 한하여 병용하여 처방\cntrdot{}투여할 수 있다.}
		\item 혼합주사는 치료효과를 높일 수 있다고 의학적으로 인정되는 경우에 한한다. 
		\item \uline{당류제제\cntrdot{}전해질제제\cntrdot{}복합아미노산제제\cntrdot{}혈액대용제\cntrdot{}혈액 및 혈액성분제제의 주사는 의학적으로 특히 필요하다고 인정되는 경우에 한한다.}
		\end{enumerate}
	\end{enumerate}
\item 치료재료의 지급 : 
\uline{치료재료는} 약사법 기타 다른 관계법령에 의하여 \uline{허가\cntrdot{}신고 또는 인정된 사항(효능\cntrdot{}효과 및 사용방법)의 범위 안에서 환자의 증상에 따라 의학적 판단에 의하여 필요\cntrdot{}적절하게 사용한다. 다만, 안전성\cntrdot{}유효성 등에 관한 사항이 정하여져 있는 치료재료 중 진료에 반드시 필요하다고 보건복지부장관이 정하여 고시하는 치료재료의 경우에는 허가\cntrdot{}신고 또는 인정된 사항(효능\cntrdot{}효과 및 사용방법)의 범위를 초과하여 사용할 수 있다.}
\item 예방\cntrdot{}재활 : 
재활 및 물리치료(이학요법)는 약물투여 또는 처치 및 수술 등에 의하여 치료효과를 얻기 곤란한 경우로서 재활 및 물리치료(이학요법)가 보다 효과가 있다고 인정되는 경우에 행한다.
\item 입원
	\begin{enumerate}[가.]\tightlist
	\item \uline{입원은 진료상 필요하다고 인정되는 경우에 한하며 단순한 피로회복\cntrdot{}통원불편 등을 이유로 입원지시를 하여서는 아니된다.}
	\item 퇴원은 의학적 타당성과 퇴원계획의 충분성 등을 신중하게 고려하여 적절한 시기에 행하여져야 한다.
	\item 입원환자에 대한 식사는 환자의 치료에 적합한 수준에서 의료법령 및 식품위생법령에서 정하는 기준에 맞게 위생적인 방법으로 제공하여야 한다
	\end{enumerate}
\end{enumerate}