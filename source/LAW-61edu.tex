\section*{법정의무교육}
 %- 개인정보교육, 성희롱예방교육, 산업안전보건교육에 관해서 허심탄회하게 좀 알려주세요.(나00 원장님)

\begin{enumerate}[1)]\tightlist
\item 개인정보보호 교육은 사업장 내 개인정보보호 관련 지식이 있는 자를 통한 원내 교육, 외부 전문교육 기관 의뢰, 온라인 교육 등을 통해 진행할 수 있습니다. 온라인 교육 예시입니다
	\begin{enumerate}[①]\tightlist
	\item 개인정보보호 개인정보보호 종합지원 포털(url{http://privacy.go.kr})을 통한 교육 수강 및 수료증 발급 \newline
     - 4개 파트(210분, 270분, 240분, 120분)로 구성되어 있으며, 1개의 파트만 수강하더라도 수료증이 발급됨
	\item 개인정보보호 포털 홈페이지(http://www.i-privacy.kr)을 통한 교육 수강 및 수료증 발급 \newline
     - 3개 파트(150분, 100분, 30분)로 구성되어 있으며, 1개의 파트만 수강하더라도 수료증이 발급됨
	\end{enumerate}

\item 성희롱예방교육은 직원이10인미만  병의원은 직원들볼수있게 직원들 있는곳에 비치해두면 됩니다. 비치 자료입니다:\url{http://www.obgydoctor.co.kr/xe/index.php?document_srl=4478&mid=m_faq}

\item 산업안전교육은 상시 근로자 5명 이상의 사업장 대상입니다. 5인이상 사업장인 경우 집체, 현장외에 인터넷 교육도 가능합니다. 교육자료는  한국산업안전보건공단(\url{https://www.kosha.or.kr/main.do?chk=1})을 통해 제공받을수 있습니다
\end{enumerate}