\section{ASCUS}
\myde{}{%
\begin{itemize}\tightlist
\item[\dsjuridical] R876: 여성 생식기 검사물의 세포학적 이상소견 - 이상 파파니콜로 도말
\item[\dsmedical] \sout{CX541 액상자궁경부 세포검사(THIN)} [\myexplfn{368.48} 원]
\item[\dsmedical] C5624 액상세포검사(자궁질세포병리검사) 32,900원
\item[\dschemical] \sout{C5959외 인유두종바이러스 유전자형 검사} [\myexplfn{634.66} 원]
\item[\dschemical] D6592016 HPVChip HPV 유전자형검사[DNAMicroarray법] 51,440원
\item[\dschemical] D6586086 HPVMultiplex 유전자형검사[중합효소연쇄반응법] 51,440원
\item[\dschemical] D6586046 HPVReal-Time 유전자형검사[실시간중합효소연쇄반응법] 51,440원
%\item[\dschemical] 
%\item[\dschemical] 
\item[\dsmedical] C8576 자궁경부 착공검사(Punch) [\myexplfn{314.80} 원]
%\item[\dsmedical] 
\item[\dsmedical] C8575 자궁경내 소파술(ECC) [\myexplfn{539.44} 원]
\item[\dsmedical] \sout{C5911 생검 1-3개} [\myexplfn{282.35} 원]
\item[\dsmedical] \sout{C5912 생검 4-6개} [\myexplfn{380.51} 원]
\item[\dsmedical] C5602 조직병리검사[1장기당]-(LevelB) 28,970원
\item[\dsmedical] Cxgram EZ886 자궁경부 확대 촬영 검사 (법정비급여)
\end{itemize}
청구메모>> PAP검사상 ASCUS나와서 정밀검사및 추적검사함.
}%
{액상자궁경부세포검사 인정기준
\begin{enumerate}\tightlist
\item R876 자궁경부 세포진 검사상 미확정 비정형 편평세포 (ASC-US) 또는 비정형 선세포 (AGC) 이상의 변화된 소견을 보여 추적 관찰이 필요한 경우.
\item B977 인유두종 바이러스 검사에서 이상이 있어 추후 관찰이 필요한 경우: HPV양성(저위험군포함)
\item N870-N872 자궁경부암 전단계(자궁경부 이형증) 또는 자궁경부암으로 진단되어 치료를 받은 후 재발여부를 평가하는 경우
: 자궁암 또는 이형증으로 펀치생검, 원추생검 또는 전자궁적출상태
\item N930,N841 자궁경부 출혈(성교 후 접촉출혈 : N930)이나 polyp(N841)이 있는 경우 
\item Cell prep : 심사지침에 " 너-541[CX541] 액상 자궁경부 세포검사의 소정점수를 산정함"으로 thin prep과 동일합니다.
\end{enumerate}

인유두종바이러스 검사의 보험 적응증
\begin{enumerate}\tightlist 
\item R876 자궁경부 세포진 검사상 이상소견(ASCUS이상)이 있는 경우
\item 조직검사상 구인두암 또는 구인두전두암이 확인된 경우
\item 상기 1,2이후 추적검사가 필요한 경우
\item PAP정상, 고위험 저위험 HPV양성 소견으로 추적검사시 액상PAP은 급여, HPV검사는 비급여
\end{enumerate}}

\subsection{ASCUS는 대박}
한번 ASCUS이상의 자궁경부이상세포은 \uline{영원히 고위험이다. 검사간격과 검사빈도에 대한 보험인정은 없다.} 나중에 추가적인 단서조항이 없는 경우에는 언제까지나 가능하다. 산부인과 어떤 선생님말 따라 \textcolor{red}{ASCUS는 대박이다.}이라는 말이 실감이 난다. 다시 한번 강조하지만 심평원에서 진단명과 청구메모만을 보고 심사하고 조정및 지급을 결정하므로 청구메모에 거기에 맞는 타당한 이유를 넣어야 할것 같습니다.

\subsection{자궁경부착공생검}
\begin{enumerate}\tightlist
\item 대한산부인과의사회의 노력으로 자궁경부에서 생검하는 경우 기존의 침생검 -기타 (C8506 : 13,050 원)에서 자궁경부착공생검 (C8576 ; 21,660원)이라는 새로운 진료코드가 2011년 만들어져서 2012년 1월 1일 부터 적용되고 있습니다. 보험청구는 행위료로 자궁경부착공생검과 조직검사료를 청구하시면 됩니다.
\item 가능 진단명
	\begin{itemize}\tightlist
	\item R876 여성 생식기 검사물의 세포학적 이상소견 - 이상 파파니콜로 도말[ASCUS 이상]
	\item N870 경도의 자궁목 형성이상[mild dysplasia]
	\item N871 중등도의 자궁목 형성이상[moderate dysplasia]
	\item N872 달리 분류되지 않은 중증의 자궁목 형성이상[severe dysplasia]
	\item N879 상세불명의 자궁목의 형성이상
	\end{itemize}
\item 진료코드 : 행위료 C8576 : [\myexplfn{314.80} 원], 조직검사료 (c5911-c5912)로 청구하면 됨
\item 시술 당일에 punch site에 출혈있으시엔 알보칠치료후에 자궁경부소작술[R4310]같이 내도 됨.
	\begin{itemize}\tightlist
	\item +청구메모 : 착공생검자리에서 출혈이 있어서 알보칠로 자궁경부소작술 시행함. 
	\end{itemize}
\item 경부 착공생검후 추적조사시에 당일은 단순처치 청구 안됩니다. M0111 단순처치 / 58.04점 가능합니다.
	\begin{itemize}\tightlist
	\item + Z480 외과적 드레싱 및 봉합의 처치
	\end{itemize}
\end{enumerate}
