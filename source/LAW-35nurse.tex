\section{간호인력 차등제 관련}
\Que{요양병원의 간호업무를 전담하나, 직급간호사(감독, 부장, 과장 등)의 경우 간호인력으로 산정 가능한가요?}
\Ans{요양병원 간호인력은 입원환자 간호업무를 전담하는 간호사와 이에 대한 간호업무를 보조하는 간호조무사를 의미하며, 입원병동에 근무하지만 입원환자 간호를 전담하지 않는 간호인력(간호감독, 간호과장 등), 일반병상과 특수병상을 순환 또는 파견(PRN 포함)근무하는 간호인력, 특수병상 중 집중치료실, 인공신장실, 물리치료실에서 근무하는 간호인력, 외래근무자와 분만휴가자(1월 이상 장기유급휴가자 포함)의 경우에는 산정대상에서 제외됩니다. 따라서 해당 간호사가 입원환자 간호업무만 전담하는 경우에는 간호인력으로 산정 가능하되, 입원병동에 근무하지만 타 업무와 겸하여 간호업무를 담당하는 경우에는 간호인력 산정에서 제외하는 것이 타당합니다.\par
 보험급여과-389호 (2010.02.05.)}
\subsection{요양병원의 비정규직 간호인력 산정기준}
\Que{요양병원의 임시직 간호인력은 어떻게 인정되나요?}
\Ans{간호인력중 비정규직 간호인력(기간제, 단시간근로자등) 산정기준은 「기간제 및 단시간근로자 보호에 관한 법률」 제17조(근로조건의 서면명시)를 준수하고, 3개월 이상 고용계약을 체결한 경우에 산정합니다. 임시직 간호사 중 1주간의 근로시간이 휴게시간을 제외하고 20(이상)~30시간(미만)인 경우 0.4인, 30(이상)~40시간(미만)근무자는 0.6인, 40시간(이상)근무자는 0.8인으로 산정하며, 「소득세법」 시행규칙제 7조제4호에 의한 의료취약지역 소재 요양기관은 각각 0.5인, 0.7인, 0.9인으로 산정할 수 있습니다. 다만, 임시직간호사를 고용하는 경우 정규직 간호사 의무고용비율은 100분의 50입니다. 임시직 간호조무사는 1주간의 근로시간이 휴게시간을 제외하고 44시간(다만, 근로기준법에 의한 근로시간이 주40시간인 요양기관은 40시간)인 근무자의 경우에만 3인을 2인으로 산정할 수 있습니다. 출산휴가자를 대체하는 간호사는 1주간의 근로시간이 휴게시간을 제외하고 44시간(다만, 근로기준법에 의한 근로시간이 주40시간인 요양기관은 40시간)인 근무자의 경우 1인으로 산정할 수 있습니다.}

\begin{commentbox}{「기간제 및 단시간 근로자 보호에 관한 법률」 제17조 (근로조건의 서면명시)}
사용자는 기간제근로자 또는 단시간근로자와 근로계약을 체결하는 때에는 다음 각 호의 모든 사항을 서면으로 명시하여야 한다. 다만, 제6호는 단시간근로자에 한한다.
\begin{enumerate}[1.]\tightlist
\item 근로계약기간에 관한 사항
\item 근로시간·휴게에 관한 사항
\item 임금의 구성항목·계산방법 및 지불방법에 관한 사항
\item 휴일·휴가에 관한 사항
\item 취업의 장소와 종사하여야 할 업무에
\end{enumerate}
\end{commentbox}

\Que{임시직 간호사가 3개월 이상 고용계약을 체결하였으나 3개월 근무를 채우지 못한 경우 간호인력으로 산정 가능한가요?}
\Ans{임시직 간호사(시간제, 계약직 등)가 3개월 이상 고용계약을 체결하였으나 부득이한 사유로 실제 근무기간이 3개월이 안되는 경우라도 간호관리료 산정대상 간호사수에 포함할 수 있습니다.}