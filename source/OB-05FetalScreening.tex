\section{기형아 선별검사}
\myde{}{
\begin{itemize}\tightlist
\item[\dsjuridical] Z348 정상임신
\item[\dschemical] Integrated test : 9-13주 PAPP-A \& NT : 16-18주 Quad test
\item[\dschemical] 임신초기 NIPT
\item[\dschemical] 목덜미투명대
\end{itemize}
}
{
\begin{itemize}\tightlist
\item 임신 11-14주 목덜미투명대 : 다운증후군(64-70\%), 터너증후군, 심장기형 : 검사자의 경험에 의존, 20\%의 높은 위양성률
\item 임신 12주 모체혈액 검사 마커의 다운증후군 검사 정확도는 5\% 위양성률에서 PAPP-A는 44\%이고 free beta hCG는 25\%로 매우 낮다.
\end{itemize}
}
\prezi{\clearpage}
\subsection{보험적용항목}
\begin{itemize}\tightlist
\item Triple Test
\item QUAD Test
\end{itemize}
\prezi{\clearpage}
\subsection{인정비급여}
\begin{itemize}\tightlist
\item Integrated test의 PAPP-A
\item FRAGILE X TEST
\item Aminocentesis
\item FISH검사류
\end{itemize}
* FISH
노-598자
(CZ967)기타 검사자.
형광동소교잡반응검사환자의 염색체에 DNA probe를 반응시켜 hybridization 시킨후 특정 유전자의 유무, 염색체의 수적, 구조적 이상을 진단하는 검사법으로 질환의 유전학적 원인파악을 위한 검사임.(프라더-빌리/안젤만 증후군, 묘성 증후군, 디조지 증후군, 칼만 증후군, 밀러-디커 증후군, 망막아세포종, 루빈스타인-타이비 증후군, 스미스-마게니스 증후군, 윌리암스 증후군, 울프-히르쉬혼 증후군, X-관련 어린선증, X-관련 안백색증,bcr-abl 재배열, PML-RARA 재배열, X/Y 염색체, 페인팅 프로브, 중심절 프로브, 13번 삼염색체 증후군, 18번 삼염색체증후군, 21번 삼염색체 증후군 등 희귀난치성질환)제2004-89호(2005.1.1)
\prezi{\clearpage}
\subsection{임의비급여}
\begin{itemize}\tightlist
\item First Trimester의 DOUBLE TEST
\item QF-PCR
\item Tu-see test [DNA methylation]
%\item STD 계열[4종, 10종, 22종]
\end{itemize}
\prezi{\clearpage}
\subsection{임신초기 선별검사에서 NIPT의 활용}
현재는 Trisomy 13,18,21의 산전 선별검사를 위해 빠르게 임상에 적용되고 있다. \par
다운증후군의 경우 99\%이상의 높은 진단 정확성과 0.16\%의 낮은 위양성률을 보이는 효과적인 산전선별검사로 보고되고 있다.\par
최근에는 성염색체 이상과 염색체 미세 결실에까지 NIPT적용 질환의 범위가 확대되고 있으며, 태아의 whole exome sequencing과 열성 유전질환의 genotyping까지 그 적용 분야가 더욱 확대될 전망이다.
\begin{itemize}\tightlist
\item 2012 ACOG : 태아 상염색체 수적 이상의 고위험군에 적용할 수 있는 선별검사로 정의하여, 양성 결과에 대해서는 침습적 산전검사로 반드시 검사결과를 확인하여야 한다
	\begin{enumerate}\tightlist
	\item advanced maternal age
	\item an abnormal serum screen
	\item personal or family history of aneuploidy
	\item abnormal ultrasound
	\end{enumerate}
\item 2015 ISPD : 모든 일반 임산부로 확대할수 있다고 발표
\item 2015 ACOG : 아직까지 NIPT를 일반 산모에게 first line test로 사용하는 것을 권장하지 않는 상황이다. 이는 임신부의 나아에 따라 질환의 발생빈도가 상이하므로 NIPT의 양성예측치 역시 산모의 나이에 따라 상당한 차이를 나타내기 때문이다. 
 NIPT PPV가 35세 이하인 경우엔 33\%, 40세 이상의 경우엔 87\%가 나온다고 함. 꼭 invasive technique로 확진이 필요함.
\end{itemize}
\prezi{\clearpage}
\par
\medskip
\Que{임신전, 결혼전에 임신초기 검사 처럼 피검사 원할때 보험은 어떻게 하며 병명은 어떻게 하십니까?}
\Ans{본인이 원하는경우는 비급여대상이므로 비급여(비보험)입니다}



