\section{노무관계}
\subsection{근로관계의 의의}
\begin{itemize}\tightlist
\item 근로관계란 근로자의 근로제공과 사용자의 임금지급에 관련한 사용자와 근로자 사이의 법률관계를 의미
\item 이를 고용관계 또는 근로계약관계라고도 말함
\item 근로관계는 근로계약에 의해 원칙으로 성립하지만 그 예외가 있음(사실상 근로관계)
\item 근로기준법 제17조는 사용자는 근로계약을 체결할 때에 임금, 소정근로시간, 휴일, 연차유급휴가 등을 서면으로 명시하도록 규정하고 있음
\end{itemize}
\subsection{개별적 근로관계법과 집단적 노사관계법}
\tabulinesep =_2mm^2mm
\begin {tabu} to\linewidth {|X[1,c,m]|X[6,l]|X[6,l]|} \tabucline[.5pt]{-}
\rowcolor{ForestGreen!40} \centering 구분 & \centering 개별적 근로관계법 &	\centering 집단적 노사관계법 \\ \tabucline[.5pt]{-}
\rowcolor{Yellow!40} 의의 & - 근로자 개인과 사용자 간의 근로  관계를 규율 \newline  근로계약 체결, 근로관계의 내용,   변경, 종료절차 등에 일정한 법적 기준을 제시함으로써 근로자의   근로조건을 보호둘중 하나만 청구 & - 노동조합 또는 근로자대표 등   근로자집단과  사용자 간의   노사관계를 규율 \newline  근로자의 단결을 통한 집단적  방법에 의해 근로자의   근로조건을 보호 \\ \tabucline[.5pt]{-}
\rowcolor{Yellow!40} 내 용 & - 근로계약/취업규칙, 임금/퇴직, 휴일/휴가, 근로시간/휴게시간, 징계/해고, 여성/소년보호, 산업안전/보건 등 &  
- 노동조합설립/운영, 단체교섭, 단체협약, 쟁의행위/조정/중재, 부당노동행위, 노사협의회 등 \\ \tabucline[.5pt]{-}
\rowcolor{Yellow!40} 적용 관계 &  -  근로기준법, 최저임금법, 남녀고용평등법, 근로자퇴직급여보장법 등 & - 노동조합 및 노동관계조정법 적용 \\ \tabucline[.5pt]{-}
\end{tabu}

\subsection{취업규칙}
\begin{itemize}[□]\tightlist
\item 임금 등 근로자 전체에 적용될  근로조건에 관한  준칙을 규정
\item 명칭에 상관 없음(인사관리규정,   운영규정, 사칙, 인사규칙 등)
\item 근거 : 근기법 제93조
\item 대상 : 상시 10인 이상의 근로자를  사용하는 사업장
\item 하나의 사업장에 복수의 취업규칙 작성가능
\item 상시근로자의 경우 근기법이 적용되지 않는 자를 제외하고 임시직․정규직․일용직․ 상용직 등을 총 망라한 상시근로자수가 10인 이상인지 여부로 판단 
\item 10인 미만이면 신고 대상에서는\textcolor{red}{ 제외되나 취업규칙을 작성 · 비치하여야 함.}
\end{itemize}
\emph{작성및 변경}\\
\includegraphics[scale=.65]{nonu}

\subsection{근로관계의 유형}
\emph{무기계약근로자}\\
기간의 정함이 없는 근로계약을 체결한 근로자로서 계약기간의 상한을 정한 정년규정을 두고 있음\\
\emph{기간제근로자}
\begin{itemize}\tightlist
\item 기간의 정함이 있는 근로계약을 체결한 근로자로서 2년을 상한으로 계약체결이 가능함
\item 기간제법에서 정한 예외사유에 해당하지 않고 2007. 7. 1일 이후 근로계약이 체결․갱신되거나, 
     기존 근로계약을 연장한 후 2년이 초과한 근로자는 무기계약근로자로 봄.
\end{itemize}
\emph{단시간 근로자}\\
1주간의 소정근로시간이 당해 사업장의 동종업무에 종사하는 통상근로자의 1주간 
   소정근로시간에 비해 짧은 근로자\\
\emph{1주간의 소정근로시간이 현저히 짧은 단시간근로자}
\begin{itemize}\tightlist
\item  4주간을 평균하여 1주간의 소정근로시간이 15시간 미만인 근로자(근기법 제18조)
\item 근기법 일부 적용 제외(주휴일, 연차유급휴가), 퇴직급여제도․고용보험 가입제외 
    <단, 산재보험은 적용>   
\end{itemize}
\textcolor{blue}{정규근로자}
\begin{enumerate}[❶]\tightlist
\item 고용관계와 사용관계가 동일함
\item 근로계약기간의 정함이 없음
\item 법정근로시간의 전일제 근로형태
\item 근로기준법 등의 법적 보호   대상에 해당됨   
\end{enumerate}
정규근로자의 \textcolor{red}{4가지 항목 중에 한 항목이라도 결여 시 비정규근로자로 판단}
       (무기계약으로 전환된 직원의 경우 정규근로자와 차이가 없음)   \\
\textcolor{blue}{비정규근로자}   
\begin{enumerate}[❶]\tightlist
\item 근로계약상의 사용자와 지휘명령  관계상의  사용자가 다른 경우
\item 근로계약기간이 일・월・년 등으로  정해진 경우
\item 근로시간이 통상근로자의 근로 시간에 못  미치는 경우
\item 근로기준법 등의 법적 보호영역에 서 제외되는  경우 (특수 고용직)  
\end{enumerate}
\subsection{근로시간}
\begin{mdframed}[linecolor=blue,middlelinewidth=2]
근무시간이란 직원이 사용자의 지휘・감독 하에서 근로를 제공하는 시간을 의미하며 휴게시간은 근무시간에 미포함(단, 작업을 위한 지휘.감독아래 대기시간은 포함
\end{mdframed}
