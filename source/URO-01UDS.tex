\section{요류역학검사:UDS}
\myde{}{
\begin{itemize}\tightlist
\item[\dsjuridical] N393 스트레스요실금
\item[\dsmedical] E6560 요류역학검사[X-Ray 포함] Urodynamic Study \myexplfn{1729.42} 원
\item[\dsmedical] E6561 방광내압측정 \myexplfn{897.22} 원
\item[\dsmedical] EX751 요도내압측정 \myexplfn{788.69} 원 
\item[\dsmedical] EY521 요류측정 \myexplfn{369.65} 원
\item[\dsmedical] F6116 근전도검사-기타(항문또는요도괄약근) \myexplfn{675.97} 원
\item[\dsmedical] URODYNAMIC CATHETER 2-WAY(M0005012) 1*1*1
\end{itemize}
}{
\Large \textbf{요류역학검사의 수가산정방법} \normalsize
\begin{enumerate}[1.]\tightlist
\item  \textcolor{red}{나656 요류역학검사는} 방광압력,복강압력,요도압력,방광근육압력,소변유출여부 등을 검사하여 배뇨기능 장애의 진단을 판단하는 검사항목으로 \textcolor{red}{주로 나611마(2) 근전도검사, 나656-1 방광내압측정, 너751 요도내압측정, 너752 요류측정 검사항목을 포함하고 있음.} 이중 일부 검사를 시행하는 경우 수가산정방법은 검사 항목과 검사 항목수를 고려하여 다음과 같이 함.\par
\begin{center}\emph{- 다 음 -}\end{center}
	\begin{enumerate}[가.]\tightlist
	\item 나656 요류역학검사는 나611마(2) 근전도검사, 나656-1 방광내압측정, 너751 요도내압측정(또는 요누출압검사 또는 압력요속검사 대체 실시 가능), 너752 요류측정 항목 중 \textcolor{red}{3항목이상을 실시한 경우 나656 요류역학검사의 소정점수를 산정함.}
	\item  3항목[나611마(2) 근전도검사, 나656-1 방광내압측정,너751 요도내압측정(또는 요누출압검사 또는 압력요속검사), 너752 요류측정] 미만으로 실시한 경우 각 검사의 소정점수를 산정함.
  	\end{enumerate}
\item \textcolor{red}{「요누출압검사(leak point pressure)」 또는「압력요속검사(pressure-flow study)」}는 방광내압측정과 실시방법이 유사하므로 \textcolor{red}{나656-1 방광내압측정 소정점수로 준용 산정함.}
(시행일 : 2011.12.1일부터)
\end{enumerate}
}

\subsection{나656 요류역학검사(Urodynamic study) 관련 세부 실시항목별 심사적용에 대하여(5사례)}
나656 요류역학검사(Urodynamic study) 관련 세부 실시항목별 심사적용에 대하여(5사례)
인조테이프를 이용한 요실금수술은 보건복지부 고시 제2007-3호(‘07.1.23)에 의거 요류역학검사 결과 요누출압이 120cmH2O 미만인 경우 인정하고 있으나, 요류역학검사 실시항목에 대한 세부심사방안 마련의 필요성이 제기되어 관련학회 의견 및 전문가 자문회의 결과를 토대로 동 검사의 심사적용 방안에 대하여 아래와 같이 결정함.\par
\par 
\begin{center}\emph{- 아 래 -}\end{center}
 \par
\begin{enumerate}[①]\tightlist
\item 방광내압측정검사 결과(Cystometry 또는 요누출압검사)상 음압을 보이는 경우
	\begin{itemize}[-]\tightlist
	\item 시작시점 : Pves(방광내압), Pabd(복강내압력)의 경우 음압으로 start시는 나656-1 방광내압측정검사료를 인정하지 아니함. 다만, Pdet(배뇨근압)의 경우 관련 제 외국논문(Neurourology and urodynamics, 2007) 참조, > -5cmH2O 범위 내는 인정키로 함.
	\item 검사도중 음압이 나타나는 경우 : 음압이 나타나는 경우 그때마다 “0(zero)”이상으로 보정을 한 경우에는 인정함.
	\end{itemize}  
	\begin{mdframed}[linecolor=blue,middlelinewidth=2]
	※ 방광내압측정술(Cystometry)이란 방광이 충전되는 동안 방광내압을 측정하는 방법으로 양질의 방광내압측정 검사 결과를 얻기 위해서는 방광과 직장 도관에 대한 압력변환기(pressure transducer)는 똑같은 참고위치에서 측정하는 것이 중요하다. 따라서, 직장도관(Pabd)은 검사 시작시 대기압에 영점으로 맞추어진 방광내압(Pves)과 같이 영점으로 맞추고 시작하고 Pves 에서 Pabd을 뺀 값인 Pdet(배뇨근압)이 제로가 되는지 확인하여야 하며 음압의 값을 보이는 경우는 모두 즉시 교정하도록 권고하고 있음(2002, ICS report)
	\end{mdframed}
\item 총 방광용적 참조, 요누출압(VLPP 또는 CLPP) 측정 시점의 생리식염수 infused volume의 적정성여부
: 요누출압 측정은 대개 환자 예상방광용적의 1/2에서 start 하도록 되어있으나 총 방광용적 자체의 정확한 확인이 어렵고 검사실시자에 따라 결과치가 달라질 수 있는 점을 감안하여 요누출압 측정시점의 생리식염수 주입용량이 300㎖ 이하에서 측정된 경우 인정함을 원칙으로 함.
  
\item 방광내압측정검사(Cystometry) 일련의 과정에서 요누출압검사 시행시 방광내압측정검사료 각각 인정여부
: 방광내압측정(Cystometry) 후 방광이 비워진 상태에서 다시 방광을 채워 요누출압검사를 시행하는 경우에는 나656-1 방광내압측정검사료를 각각 인정하나, Filling cystometry 도중에 요누출압검사를 하는 경우는 주된 검사 100\%, 제2의 검사 50\%를 인정함(나656-1 150\%)
  
\item 방광내압측정검사의 전 과정(full cystometry)을 측정하지 아니한 경우
: 방광내압측정 시 최대방광용적까지 full check를 하지 않고 요누출압 측정으로 요실금이 확인되는 순간 검사 종료 시는 방광내압검사료를 인정하지 아니하며 요누출압 측정검사료만 인정키로 함.
\begin{mdframed}[linecolor=blue,middlelinewidth=2]
※ 요누출압측정검사료는 나656-1 방광내압측정 항목에 준용산정
\end{mdframed}
\item 방광내압측정이 부적절하게 시행되어 심사조정 시, 동시 시행된 타 검사(요누출압, 요도내압검사)의 인정여부
: 방광내압측정검사가 잘못된 경우 동시에 실시되고 검사결과에도 영향을 미치는 요누출압측정은 인정하지 아니함이 타당하므로 나656-1 방광내압측정검사료는 인정하지 아니함.
다만, 요도내압검사는 요도의 전장에 걸쳐 요도내 압력을 측정하여 요도 폐색여부를 알기 위한 검사로. 방광내압측정검사와는 별개의 과정이므로 인정함
  
\item 요류역학검사결과 상 EMG가 부적절하게 시행된 것으로 확인될 경우 인정여부
: 동 건은 검사결과상 EMG가 시행되지 않은 것으로 판단되므로 나-611 마(2) 근전도검사료를 인정하지 아니함.
  
\item 나-656 요류역학검사자 자격요건 및 판독소견서에 대한 주)항목 신설 필요성에 대하여
: 의사가 판독한 판독소견서를 작성한 경우에 한하여 나-656 요류역학검사료를 인정함.
[2007.12.17 진료심사평가위원회]
\end{enumerate}

\begin{mdframed}[linecolor=blue,middlelinewidth=2]  
※ 알림
동 내용은 사례에 대한 결정으로 외부공개 하는바 상기항목 중 제 ①, ②, ④, ⑤, ⑥항은 심사지침, 제 ③, ⑦항은 보건복지부에 고시건의 예정이므로 추후 고시안과 동 내용이 다를 수 있음을 알려드림.
\end{mdframed}

\subsection{비뇨기계검사시 사용되는 UDS catheter 인정기준}
요류역학검사 및 신우내압측정검사 등에서 압력 측정 목적으로 사용되는 \textcolor{red}{UDS(Urodynamic Study)catheter는 진단시} 필수적인 치료재료인 점 등을 감안하여 다음의 경우에 별도 \textcolor{red}{인정 함}.
\begin{center}\emph{- 다 음 -}\end{center}
\begin{enumerate}[가.]\tightlist
\item 나656 요류역학검사[X-Ray포함] 
\item 나656-1 방광내압측정 
\item 나656-2 신우내압측정검사
\item 너-751 요도내압측정 
\end{enumerate}
(고시 제2009-200호,'09.11.1 시행)\par
\par
\medskip
\leftrod{나656 요류역학검사에서 압력 측정 목적으로 사용되는 UDS(Urodynamic Study) catheter 인정개수에 대하여}
\begin{itemize}[■]\tightlist
\item 청구내역
	\begin{itemize}[○]\tightlist
	\item A 사례(남/36세)/B 사례(남/60세)
		\begin{itemize}[-]\tightlist
		\item 상병명 : 상세불명의 방광의 신경근육기능장애, 혈뇨를 동반한 급성 전립선염
		\item 주요 청구내역 
			\begin{itemize}\tightlist
			\item 나656 요류역학검사(E6560) 1*1*1
			\item \textcolor{blue}{URODYNAMIC CATHETER 2-WAY(M0005012) 1*1*3}
			\end{itemize}
		\end{itemize}
	\end{itemize}		
\item 심의내용 \newline
비뇨기계검사시 사용되는 UDS(Urodynamic study) catheter 현행 인정기준에 의하면 ‘요류역학검사 및 신우내압측정검사 등에서 압력 측정 목적으로 사용한 경우 별도 인정한다’고 되어 있고 인정개수에 대하여는 명시되어 있지 아니함.(보건복지부 고시 제2009-200호, 2009.10.30.)
  
동 건은 요류역학검사시 UDS catheter(2-way)가 여러 개 산정되어 인정개수에 대하여 논의한 결과, 비뇨기계검사시 사용되는 카테터는 형태에 따라 2-way(방광내압 측정용)와 3-way(방광내압, 요도내압 동시 측정용)로 분류되어 있어 요류역학검사시 검사자가 어떤 항목의 검사를 실시할 것인지에 따라 카테터의 선별사용이 가능함. 
  
따라서, 특별한 사유가 없는 한 \textcolor{red}{요류역학검사시 사용한 UDS catheter는 한 개만 인정키로 함.}
  
\item 참고
	\begin{itemize}[○]\tightlist
	\item 비뇨기계검사시 사용되는 UDS catheter 인정기준(고시 제2009-200호(치료재료), 2009.10.30.)
	\end{itemize}
\end{itemize}