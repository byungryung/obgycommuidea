\section{자궁경관봉축술}
\myde{}{
\begin{itemize}\tightlist
\item[\dsjuridical] o343 자궁경관부전에 대한 산모관리
\item[\dsmedical] 자궁경관봉축술-맥도날드 자428가 R4281 (95340원)
\item[\dsmedical] 자궁경관봉축술-쉬로도카법 자428나 R4282 (188410원)
\item[\dsmedical] 자궁경관봉축술-복식자궁경관봉축술 자428다 R4283 (315030원)
\item[\dsmedical] 자궁경관봉축술-양막복원 후 자궁경부원형봉합술 자428라 R4284 (275540원) : 자궁경부가 개대되고 양막이 돌출되어 양수감압
술과 「가」를 동시에 실시한 경우에 산정한다
\item[\dsmedical] 자궁경부봉축해제술 자428-1 R4285 (24990원)
\end{itemize}
}
{
}
\prezi{\clearpage}
\par
\medskip
\begin{commentbox}{자궁경관무력증에 의한 봉합술과 실비보험}
\emph{N코드가 되는가? }\par
N883 자궁경부의 무력증(임신중이 아닌 여성의 (의심되는) 자궁경관부전의 검사 및 관리) 
제외 : 영향받은 태아 또는 신생아(P01.0)Excludes : Affecting fetus or newborn, 임신에 합병된 경우(O34.3-)\par
\emph{단체들의 입장}\par
\begin{description}\tightlist
\item[통계청] 한국표준질병사인분료(KCD) 에 따르면 임신과 합병된 자궁무력증은 N.88 코드가 아닌 O34가 되어야하고, O코드는 임신/출산 면책이다.
\item[보험사] 임신유지를 위한 목적의 수술,치료이기 때문에 임신/출산 면책이다.
\item[산부인과학회] 자궁경부 무력증은 임신과 관련하여 발생하는 질환이므로 임신과 무관한 상태에서 자궁경부 무력증의 치료 및 안정이 필요한 경우는 없다.
임신이 아닌 경우 자궁봉축술을 시행하는 경우는 매우 드물다. (원추절제술로 인해 경부길이가 짧아진 경우 등등) 원칙적으로 임신 전에 경부무력증이 진단된 경우 N88.3코드를 부여할 수 있으나, 임신 중에 자궁 경부 봉축술을 시행한 경우 분류 코드는 O34.3를 부여하게 된다.​ (이것 역시 KCD 기준)
\item[판례] 부산지방법원 2007가단41269 채무부존재확인 원고: 현대해상 : ‘출산관련’ 손해는 보상하지 아니한다는 내용의 자막을 송출한 사실은 인정된다. 그러나 피고가 얻은 ‘자궁경부 무력증 및 조기진통’이라는 질병으로 인해 입은 손해는 원고가 명시 설명의무를 다하였더라도 \textcolor{red}{원고의 면책대상에 포함되지 아니함}
\end{description}
%\href{http://blog.naver.com/PostView.nhn?blogId=congsnote&logNo=220159341348&parentCategoryNo=7&categoryNo=&viewDate=&isShowPopularPosts=true&from=search}{정리가 잘된 blog}
\end{commentbox}
\prezi{\clearpage}
\par
\medskip
\subsection{자궁경관봉축술 시행시 Mersilene tape 인정여부}
\sout{자궁경관봉축술 시행시 대부분의 요양기관이 값싼 Silk, Nylon 등으로 소기의 목적을 달성하고 있으므로 Mersilene tape은 별도 산정할 수 없음.(2000-12-30)}\par
2015년 10월 삭제(보건복지부 고시 제2015 - 169호) (고시 제2015-169호, 2015.10.1.시행)
\begin{itemize}[▶]\tightlist
\item 삭제 사유 : 상기 급여기준에 해당되는 관련 치료재료인“CERVIX SET”가 「치료재료 급여 비급여 목록 및 상한금액표」에 급여로 등재됨에 따라 별도 산정가능하여 해당 급여기준 삭제
\item 참고사항: 관련 치료재료 : 자궁경관봉축술용 봉합재료 CERVIX SET(B2013004) 
\item B2013004 CERVIX SET (16440원) 제조:B.BRAUN, 수입 :비브라운 코리아
\end{itemize}
\begin{myshadowbox}
문제점으로 비브라운의 Cervix Set를 쓰는 데가 거의 없다. 비브라운 회사측도 별다른 의지 없어보이고.. 그래서 저희 병원은 아직도 임의비급여로 받고 있는 실정임.
\end{myshadowbox}