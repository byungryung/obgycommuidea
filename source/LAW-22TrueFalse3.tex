국민건강보험법     제41조(요양급여)
\begin{enumerate}[①]\tightlist
\item 가입자와 피부양자의 질병, 부상, 출산 등에 대하여 다음 각 호의 요양급여를 실시한다.
	\begin{enumerate}[1.]\tightlist
	\item 진찰·검사
	\item 약제(藥劑)·치료재료의 지급
	\item 처치·수술 및 그 밖의 치료
	\item 예방·재활
	\item 입원
	\item 간호
	\item 이송(移送)
	\end{enumerate}
\item \highlightY{제1항에 따른} \highlightR{요양급여(이하 "요양급여"라 한다)의 방법·절차·범위·상한 등의 기준은 보건복지부령으로 정한다.}
\item 보건복지부장관은 제2항에 따라 \highlightR{요양급여의 기준을 정할 때 업무나 일상생활에 지장이 없는 질환, 그 밖에 보건복지부령으로 정하는 사항은 요양급여의 대상에서 제외할 수 있다.}
\end{enumerate}

국민건강보험 요양급여의 기준에 관한 규칙  [별표 1]요양급여의 적용기준 및 방법(제5조제1항관련)

\begin{enumerate}[1.]\tightlist
\item 요양급여의 일반원칙
	\begin{enumerate}[가.]\tightlist
	\item  요양급여는 가입자 등의 \highlightR{연령·성별·직업 및 심신상태 등의 특성을 고려하여 진료의 필요가 있다고 인정되는 경우에 정확한 진단을 토대로 하여 환자의 건강증진을 위하여 의학적으로 인정되는 범위 안에서 최적의 방법으로 실시하여야 한다.}
	\item  요양급여를 담당하는 의료인은 의학적 윤리를 견지하여 환자에게 심리적 건강효과를 주도록 노력하여야 하며, 요양상 필요한 사항이나 예방의학 및 공중보건에 관한 지식을 환자 또는 보호자에게 이해하기 쉽도록 적절하게 설명하고 지도하여야 한다.
	\item  \highlightR{요양급여는 경제적으로 비용효과적인 방법으로 행하여야 한다.}
	\item  \highlight{요양기관은 가입자 등의 요양급여에 필요한 적정한 인력·시설 및 장비를 유지하여야 한다. 이 경우 보건복지부장관은 인력·시설 및 장비의 적정기준을 정하여 고시할 수 있다.}
	\item  라목의 규정에 불구하고 가입자 등에 대한 최적의 요양급여를 실시하기 위하여 필요한 경우, 보건복지부장관이 정하여 고시하는 바에 따라 다른 기관에 검사를 위탁하거나, 당해 요양기관에 소속되지 아니한 전문성이 뛰어난 의료인을 초빙하거나, 다른 요양기관에서 보유하고 있는 양질의 시설·인력 및 장비를 공동 활용할 수 있다.
	\end{enumerate}
\item \highlight{진찰·검사, 처치·수술 기타의 치료}
	\begin{enumerate}[가.]\tightlist
	\item  \highlightR{각종 검사를 포함한 진단 및 치료행위는 진료상 필요하다고 인정되는 경우에 한하여야 하며 연구의 목적으로 하여서는 아니된다.}
	\item  영 제21조제3항제2호에 따라 보건복지부장관이 정하여 고시하는 질병군에 대한 입원진료의 경우 그 입원진료 기간동안 행하는 것이 의학적으로 타당한 검사·처치 등의 진료행위는 당해 입원진료에 포함하여 행하여야 한다.
	\end{enumerate}
\item 약제의 지급 
	\begin{enumerate}[가.]\tightlist
	\item  처방·조제
		\begin{enumerate}[(1)]\tightlist
		\item 영양공급·안정·운동 그 밖에 요양상 주의를 함으로써 치료효과를 얻을 수 있다고 인정되는 경우에는 의약품을 처방·투여하여서는 아니되며, 이에 관하여 적절하게 설명하고 지도하여야 한다.
		\item 의약품은 약사법령에 의하여 허가 또는 신고된 사항(효능·효과 및 용법·용량 등)의 범위 안에서 환자의 증상 등에 따라 필요·적절하게 처방·투여하여야 한다. (약제 전산심사)다만, 안전성·유효성 등에 관한 사항이 정하여져 있는 의약품 중 진료상 반드시 필요하다고 보건복지부장관이 정하여 고시하는 의약품의 경우에는 허가 또는 신고된 사항의 범위를 초과하여 처방·투여할 수 있으며, 중증환자에게 처방·투여하는 약제로서 보건복지부장관이 정하여 고시하는 약제의 경우에는 건강보험심사평가원장이 공고한 범위 안에서 처방·투여할 수 있다.
		\item 요양기관은 중증환자에 대한 약제의 처방·투여시 해당약제 및 처방·투여의 범위가 (2)의 허용범위에는 해당하지 아니하나 해당환자의 치료를 위하여 특히 필요한 경우에는 건강보험심사평가원장에게 해당약제의 품목명 및 처방·투여의 범위 등에 관한 자료를 제출한 후 건강보험심사평가원장이 중증질환심의위원회의 심의를 거쳐 인정하는 범위 안에서 처방·투여할 수 있다.
		\item 제10조의2제2항에 따라 식품의약품안전처장이 긴급한 도입이 필요하다고 인정한 품목의 경우에는 식품의약품안전처장이 인정한 범위 안에서 처방·투여하여야 한다.
		\item 항생제·스테로이드제제 등 오남용의 폐해가 우려되는 의약품은 환자의 병력·투약력 등을 고려하여 신중하게 처방·투여하여야 한다.
		\item 진료상 2품목 이상의 의약품을 병용하여 처방·투여하는 경우에는 1품목의 처방·투여로는 치료효과를 기대하기 어렵다고 의학적으로 인정되는 경우에 한한다.
		\end{enumerate}
	\end{enumerate}		
\item 치료재료의 지급
치료재료는 약사법 기타 다른 관계법령에 의하여 허가·신고 또는 인정된 사항(효능·효과 및 사용방법)의 범위 안에서 환자의 증상에 따라 의학적 판단에 의하여 필요·적절하게 사용한다. 다만, 안전성·유효성 등에 관한 사항이 정하여져 있는 치료재료 중 진료에 반드시 필요하다고 보건복지부장관이 정하여 고시하는 치료재료의 경우에는 허가·신고 또는 인정된 사항(효능·효과 및 사용방법)의 범위를 초과하여 사용할 수 있다.
\end{enumerate}


국민건강보험 요양급여 기준에 관한 규칙.
제10조(신의료기술등의 요양급여 결정신청)
\begin{enumerate}[①]\tightlist
\item 요양기관, 의약관련 단체, 치료재료의 제조업자·수입업자(치료재료가 「인체조직 안전 및 관리 등에 관한 법률」 제3조제1호에 따른 인체조직인 경우에는 같은 법 제13조에 따른 조직은행의 장을 말한다)는 제8조제2항에 따른 요양급여대상 또는 제9조제1항에 따른 비급여대상으로 결정되지 아니한 새로운 행위 및 치료재료(이하 "신의료기술등"이라 한다)에 대하여는 다음 각 호에 규정된 날부터 30일 이내에 요양급여대상 여부의 결정을 보건복지부장관에게 신청하여야 한다.  <개정 2001.12.31., 2005.10.11., 2006.12.29., 2007.7.25., 2008.3.3., 2009.7.31., 2010.3.19.>
	\begin{enumerate}[1.]\tightlist
	\item 행위의 경우에는 「의료법」 제53조에 따른 신의료기술평가(이하 "신의료기술평가"라 한다) 결과 안전성·유효성 등을 인정받은 이후 가입자등에게 최초로 실시한 날
	\item 치료재료의 경우에는 다음 각 목에서 정한 날
		\begin{enumerate}[가.]\tightlist
		\item 「약사법」 또는 「의료기기법」에 따른 품목허가 또는 품목신고 대상인 치료재료인 경우에는 식품의약품안전청장으로부터 품목허가를 받거나 품목신고를 한 날. 다만, 품목허가나 품목신고 대상이 아닌 치료재료의 경우에는 해당 치료재료를 가입자등에게 최초로 사용한 날
		\item 「인체조직 안전 및 관리 등에 관한 법률」 제3조제1호에 따른 인체조직(이하 "인체조직"이라 한다)의 경우에는 보건복지부장관으로부터 조직은행 설립허가를 받은 날. 다만, 다음의 어느 하나의 경우에는 그 해당하는 날
			\begin{enumerate}[1)]\tightlist
			\item  수입인체조직의 경우에는 보건복지부장관이 정하는 바에 따라 안전성에 문제가 없다는 통지를 받은 날
			\item  조직은행 설립허가 당시의 취급품목이 변경된 경우에는 보건복지부장관이 그 변경사실을 확인한 날
			\end{enumerate}
		\item 가목 및 나목에도 불구하고 신의료기술평가대상이 되는 치료재료의 경우에는 신의료기술 평가 결과 안전성·유효성 등을 인정받은 이후 해당 치료재료를 가입자등에게 최초로 사용한 날
		\end{enumerate}
	\item 삭제  <2006.12.29.>
	\end{enumerate}
\item 제1항에 따른 결정신청은 그 결정을 신청하려는 자가 다음 각 호의 구분에 따른 평가신청서에 해당 각 목의 서류를 첨부하여 건강보험심사평가원장에게 요양급여대상여부의 평가신청을 함으로써 이를 갈음한다.  <개정 2001.12.31., 2005.10.11., 2007.7.25., 2008.3.3., 2009.7.31., 2010.3.19., 2010.4.30.>
	\begin{enumerate}[1.]\tightlist
	\item 행위의 경우 : 별지 제14호서식의 요양급여행위평가신청서
		\begin{enumerate}[가.]\tightlist
		\item 신의료기술의 안전성·유효성 등의 평가결과통보서
		\item 상대가치점수의 산출근거 및 내역에 관한 자료
		\item 비용효과에 관한 자료(동일 또는 유사 행위와의 장·단점, 상대가치 점수의 비교 등을 포함한다)
		\item 국내외의 실시현황에 관한 자료(최초실시연도·실시기관명 및 실시건수 등을 포함한다)
		\item 소요장비·소요재료·약제의 제조(수입)허가(신고)관련 자료
		\item 국내외의 연구논문 등 기타 참고자료
		\end{enumerate}		
	\item 삭제  <2006.12.29.>
	\item 제1항제2호가목의 경우(제1항제2호다목에 따른 치료재료를 포함한다): 별지 제16호서식의 치료재료평가신청서
		\begin{enumerate}[가.]\tightlist
		\item 제조(수입)품목허가증(신고서)사본(품목허가를 받거나 품목 신고를 한 치료재료에 한한다)
		\item 판매예정가 산출근거 및 내역에 관한 자료
		\item 비용효과에 관한 자료(동일 또는 유사목적의 치료재료와의 장·단점, 판매가의 비교 등을 포함한다)
		\item 국내외의 사용현황에 관한 자료(최초사용연도·사용기관명 및 사용건수 등을 포함한다)
		\item 구성 및 부품내역에 관한 자료 및 제품설명서
		\item 국내외의 연구논문 등 기타 참고자료
		\item 신의료기술의 안전성·유효성 등의 평가결과통보서(제1항제2호다목에 따른 치료재료만 해당한다)
		\end{enumerate}
	\item 제1항제2호나목의 경우(제1항제2호다목에 따른 인체조직을 포함한다): 별지 제16호의2서식의 인체조직평가신청서
		\begin{enumerate}[가.]\tightlist
		\item 조직은행설립허가증 사본(기재사항 변경내역을 포함한다). 다만, 수입인체조직의 경우에는 보건복지부장관이 정하는 바에 따라 안전성에 문제가 없다는 사실을 증명하는 서류를 함께 첨부하여야 한다.
		\item 인체조직가격 산출근거 및 내역에 관한 자료
		\item 비용효과에 관한 자료(동일 또는 유사목적의 인체조직과의 장·단점, 가격 비교 등을 포함한다)
		\item 국내외의 사용현황에 관한 자료(최초 사용연도, 사용기관명 및 사용건수 등을 포함한다)
		\item 인체조직에 대한 설명서
		\item 국내외의 연구논문 등 기타 참고자료
		\item 신의료기술의 안전성·유효성 등의 평가결과 통보서(제1항제2호다목에 따른 인체조직만 해당한다)
		\end{enumerate}
	\end{enumerate}
\item 보건복지부장관은 요양기관이 정당한 사유없이 신의료기술등에 대하여 제1항의 규정에 위반하여 요양급여대상 여부의 결정을 신청하지 아니하고 가입자등에게 실시 또는 사용한 후 그 비용을 부담시킨 신의료기술등이 요양급여대상으로 확인된 경우에는 법 제85조제1항제1호의 규정에 의하여 당해 요양기관의 업무정지를 명하거나 동조제2항의 규정에 의한 과징금처분을 하여야 한다.  <개정 2001.12.31., 2008.3.3., 2010.3.19.>
[제목개정 2001.12.31., 2007.7.25.]
\end{enumerate}
