\subsection{(직원) 진료비 할인의 의료법과 세무}
만일 \emph{건강보험급여에 해당하는 진료비를 할인(본인부담금 면제 혹은 할인)할 경우}, 이는 의료법 27조를 위반하는 것으로 처벌받게 됩니다.

\begin{commentbox}{의료법27조}
③누구든지 「국민건강보험법」이나 「의료급여법」에 따른 본인부담금을 면제하거나 할인하는 행위, 금품 등을 제공하거나 불특정 다수인에게 교통편의를 제공하는 행위 등 영리를 목적으로 환자를 의료기관이나 의료인에게 소개·알선·유인하는 행위 및 이를 사주하는 행위를 하여서는 아니 된다. 
\end{commentbox}

본인부담금 할인을 환자 유인, 알선으로 보기 때문이라고 할 수 있다고 합니다. \par
그러나, 비급여의 경우는 얘기가 다릅니다. 

\Que{의료기관에서 비급여 비용 할인 가능한가요?}
\Ans{의료법 제27조 제3항에서는 “누구든지「국민건강보험법」이나「의료급여법」에 따른 본인부담금을 면제하거나 할인하는 행위, 금품 등을 제공하거나 불특정 다수인에게 교통편의를 제공하는 행위 등 영리를 목적으로 환자를 의료기관이나 의료인에게 소개·알선·유인하는 행위 및 이를 사주하는 행위를 하여서는 아니 된다.”고 규정하고 있습니다.\par

다만, 동 조항의 ‘본인부담금’의 범위에 비급여 진료비까지 포함시키는 것은 형벌법규의 지나친 확장해석으로서 죄형법정주의 원칙에 어긋나며, 의료시장의 질서를 근본적으로 해하는 등의 특별한 사정이 없는 한 의료기관 및 의료인이 스스로 자신에게 환자를 유치하는 행위는 의료법 제27조의 ‘유인’이라 할 수 없다는 대법원의 판례(2007도10542,2008.2.28)에 따라 \emph{의료기관이 비급여 진료비용 할인이라는 수단으로 스스로 자신에게 환자를 유치하는 행위는 일반적으로 가능}하다 할 것입니다. }

\subsection{세무}
직원 복리 후생제도로 본인 50\%( 본인부담 만원이하시 무료) 직계가족 50\%, 형제 자매 20\%의 \emph{진료비 할인을 받았을 시는 원칙적으로 급여에 해당하며 근로소득으로 과세대상 총급여에 포함}하셔야 합니다. \par
\Que{질문 1. 직원본인진료시  본인부담 만원이하시 진료비를  무료로 했을때 1년간 할인 받은 금액이 급여로 잡혀 세금을 내야 하는지? \par
질문 2. 직계가족 및 형제자매, 사촌 본인부담 진료비 감면 받았을시 1년간 할인 받은 금액이 급여로 잡혀 세금을 내야 하는지?} 
\Ans{질문1,2의 진료비 할인은 급에에 해당하여 과세대상 총급여에 포함됩니다.}

\Que{질문 3. 지인 또는 단체와 자매결연으로 진료비 감면시 세무법에 유의 해야 할 부분은? }
\Ans{질문 3의 경우는 급여와 관련된 것이 아닌 것으로 판단되며, 병원의 자체규정과 회계처리 원칙에 따라 다른 계정(기부금, 결손금, 등등)으로 처리하시는 것이 좋은 방안이라고 생각됩니다.}