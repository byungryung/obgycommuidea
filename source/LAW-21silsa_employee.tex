\subsection{(직원) 진료비 할인의 의료법과 세무}
만일 \emph{건강보험급여에 해당하는 진료비를 할인(본인부담금 면제 혹은 할인)할 경우}, 이는 의료법 27조를 위반하는 것으로 처벌받게 됩니다.

\begin{commentbox}{의료법27조}
③누구든지 「국민건강보험법」이나 「의료급여법」에 따른 본인부담금을 면제하거나 할인하는 행위, 금품 등을 제공하거나 불특정 다수인에게 교통편의를 제공하는 행위 등 영리를 목적으로 환자를 의료기관이나 의료인에게 소개\cntrdot{}알선\cntrdot{}유인하는 행위 및 이를 사주하는 행위를 하여서는 아니 된다. 
\end{commentbox}

본인부담금 할인을 환자 유인, 알선으로 보기 때문이라고 할 수 있다고 합니다. \par
그러나, 비급여의 경우는 얘기가 다릅니다. 

\Que{의료기관에서 비급여 비용 할인 가능한가요?}
\Ans{의료법 제27조 제3항에서는 “누구든지「국민건강보험법」이나「의료급여법」에 따른 본인부담금을 면제하거나 할인하는 행위, 금품 등을 제공하거나 불특정 다수인에게 교통편의를 제공하는 행위 등 영리를 목적으로 환자를 의료기관이나 의료인에게 소개·알선·유인하는 행위 및 이를 사주하는 행위를 하여서는 아니 된다.”고 규정하고 있습니다.\par

다만, 동 조항의 ‘본인부담금’의 범위에 비급여 진료비까지 포함시키는 것은 형벌법규의 지나친 확장해석으로서 죄형법정주의 원칙에 어긋나며, 의료시장의 질서를 근본적으로 해하는 등의 특별한 사정이 없는 한 의료기관 및 의료인이 스스로 자신에게 환자를 유치하는 행위는 의료법 제27조의 ‘유인’이라 할 수 없다는 대법원의 판례(2007도10542,2008.2.28)에 따라 \emph{의료기관이 비급여 진료비용 할인이라는 수단으로 스스로 자신에게 환자를 유치하는 행위는 일반적으로 가능}하다 할 것입니다. }
\subsection{시술후 complication으로 외래추적 조사 중 환자본인부담금 면제시}
\Que{conization후에 bleeding시 외래에서 albothyl apply하고 R4300 약물소작술하고, N72 자궁경부의 염증성질환등의 진단명하에 R4106 질강처치등을 시행합니다. 이런 경우에 환자에겐 돈을 받지 않고, 공단에만 보험청구하면 안되는가요? 안된다면 어떤 조항때문에 안 되는지 알고 싶습니다.}
\Ans{한 가정의학과 의원에서 의사의 가족 또는 의원에서 일하는 직원들의 가족에게 본인부담금을 면제하여 주었다는 혐의로 수사기관에 고발당했다. \par
의료법 제27조 제3항에서 정하는 본인부담금을 면제하거나 할인하는 행위로 환자를 의료기관에 유인해서는 안된다는 조항을 적용한 것이다. \par
위 의원은 10명의 환자에 대하여 약 9만원 가량의 본인부담금을 면제해 줬고, 이와 같은 사실이 보건소에 알려져 고발당했으나, 경찰조사를 거쳐 결국 가족이나 직원의 가족들에게 인정상 본인부담금을 면제해 주었을 뿐, 이로 인하여 얻을 수 있는 이익도 크지 않다는 것이 확인돼 결국 무혐의(불기소)처분을 받았다. \par
의료법 제27조 제3항은 ‘환자유인’행위를 금지할 뿐, 본인부담금의 할인행위, 금품제공, 교통편의 제공 등은 위 환자유인의 방법 중 하나를 예시한 것에 불과하다. 그럼에도 불구하고 \emph{의료인 또는 의료기관이 본인부담금을 할인해 주거나 금품 또는 교통편의를 제공하는 것에 대해 구체적으로 그 행위가 환자유인행위에 해당하는지 여부에 상관없이 고발이 이뤄지는 경우}가 있다.\par 
이러한 경우 의료기관은 불가피한 수사를 받게 돼 업무에 큰 지장을 겪고, 결국 무혐의결정 또는 무죄판결을 받게 되는 경우에는 행정, 사법의 낭비가 초래 된다. 위와 같은 문제는 의료법 제27조 제3항의 규정형식 및 적용범위가 불분명하다는 점에서 비롯된다. 환자유인금지조항은 1981년 환자들의 어려운 처지를 이용한 환자유인브로커 등의 활동을 특히 막기 위해 최초로 도입됐다. \par
당시 의료법은 “누구든지 영리를 목적으로 환자를 의료기관 또는 의료인에게 소개, 알선, 기타 유인하는 행위를 할 수 없다”고 규정하였을 뿐, 환자유인행위의 예를 구체적으로 들고 있지는 않았다. \par
2002년 개정 의료법에서는 위 환자유인금지규정에 유인행위의 예를 추가했다. 그리고 그 규정형식은 “누구든지 영리를 목적으로 환자를 의료기관 또는 의료인에게 소개, 알선, 유인하기 위해 다음 각 호의 1에 해당하는 행위를 하거나 사주하는 행위를 해서는 안된다”라고 규정했다. 또 각 호로 “본인부담금을 면제 또는 할인하는 행위, 금품 등을 제공하거나 불특정다수인에게 교통편의를 제공하는 행위, 그 밖에 영리를 목적으로 환자를 알선, 소개, 유인 또는 이를 사주하는 행위로서 대통령령이 정하는 행위”를 규정하여 위 행위들이 환자유인행위의 한 예시임을 분명히 했다. \par
그러나 현행 의료법은 위 2002년 의료법이 각 호에서 예시한 환자유인행위의 예시들을 본문에 편입했다. 그 결과 의료법 제27조 제3항의 해석상 본인부담금 할인, 금품 또는 교통편의의 제공행위 자체가 처벌대상인지, 아니면 위 행위들로 인해 환자를 유인하는 것이 처벌대상인지 그 문언적 해석이 불분명해질 가능성이 발생했다. \par
\highlightY{환자유인행위금지 규정의 연혁을 볼 때 위 행위들이 환자유인행위의 한 예시에 불과할 뿐, 행위 자체가 금지되는 것이 아님은 명백}하다. 그러나 위 행위들로 인하여 실제 의료시장의 질서가 교란되는 정도의 환자유인행위가 발생했는지 여부와 상관없이, 위 행위가 있었다는 사정만으로 고발이 이뤄지고 많은 시간과 인력이 낭비되고 있는 것이다. \par
의료법의 개정으로 의료광고도 각종 규제 아래 허용됐고, 의료기관도 고유의 직업수행의 자유를 누리기 위해 스로를 홍보하는 행위가 필요하다. 그리고 그러한 과정에서 어느 정도 환자에 대한 유인행위가 이뤄질 수 밖에 없다. 이러한 현실을 반영해 \emph{대법원도 금품을 제공하는 등의 행위로 환자를 유인하는 효과가 의료시장의 질서를 근본적으로 해할 정도에 이르렀다고 판단되는 경우에 의료법상 환자유인금지규정에 위반된다고 판시}하고 있다(대법원 2008. 2. 28. 선고 2007도10542 판결). 의료기관이 환자유인금지규정에 명시된 행위를 하였다고 하여 곧바로 처벌대상이 되는 것은 아니며 대법원이 판시한바와 같이 의료시장의 질서를 현저히 해할 정도에 이르러야 할 것이다. \par
그러나 이러한 정도가 어느 정도인지는 그 누구도 명확한 선을 그을 수 없다. 결국 환자유인금지규정의 적용은 더욱 신중하게 행해져야 할 것이다.
\par
\medskip
\url{http://www.dailymedi.com/detail.php?number=763117}
}



\subsection{세무}
직원 복리 후생제도로 본인 50\%( 본인부담 만원이하시 무료) 직계가족 50\%, 형제 자매 20\%의 \emph{진료비 할인을 받았을 시는 원칙적으로 급여에 해당하며 근로소득으로 과세대상 총급여에 포함}하셔야 합니다. \par
\Que{질문 1. 직원본인진료시  본인부담 만원이하시 진료비를  무료로 했을때 1년간 할인 받은 금액이 급여로 잡혀 세금을 내야 하는지? \par
질문 2. 직계가족 및 형제자매, 사촌 본인부담 진료비 감면 받았을시 1년간 할인 받은 금액이 급여로 잡혀 세금을 내야 하는지?} 
\Ans{질문1,2의 진료비 할인은 급에에 해당하여 과세대상 총급여에 포함됩니다.}

\Que{질문 3. 지인 또는 단체와 자매결연으로 진료비 감면시 세무법에 유의 해야 할 부분은? }
\Ans{질문 3의 경우는 급여와 관련된 것이 아닌 것으로 판단되며, 병원의 자체규정과 회계처리 원칙에 따라 다른 계정(기부금, 결손금, 등등)으로 처리하시는 것이 좋은 방안이라고 생각됩니다.}