\section{2018년 변경된 조직병리검사 각 항목별 세부내용(산부인과)}

\tabulinesep =_2mm^2mm
\begin{tabu} to\linewidth {|X[1,l]|X[5,l]|} \tabucline[.5pt]{-}
\rowcolor{ForestGreen!40} LEVEL & 설명 \\ \tabucline[.5pt]{-}
\rowcolor{Yellow!40} 가. Level A (C5601) & 염증성, 감염성, 비종양성 병변이 의심되는 소견이 있는 경우 \newline
	난관- 불임시술 (Fallopian Tube- Sterilization)\newline
	바르톨린선- 낭종 (Bartholin’s Gland- Cyst)\newline
	연부조직- 변연절제 (Soft Tissue- Debridement)\newline
	연부조직- 지방종 (Soft Tissue- Lipoma)\newline
	질점막 (부수적제거) (Vaginal Mucosa (Incidental))\newline
	피부- 낭종/쥐젖/변연절제 (Skin- Cyst/Tag/Debridement)\newline
	피부/연부조직- 농양 (skin/soft tissue- Abscess)\newline
	혈종 (Hematoma)\newline
	분류되지 않은 조직 및 장기- 정상, 낭종, 농양, 혈종 (Tissue/organ, unclassified- nomal/cyst/abscess/hematoma)  \\ \tabucline[.5pt]{-}
\rowcolor{Yellow!40} 나. Level B (C5602) &	골, 뇌, 간, 심근, 췌장, 연부조직, 고환, 전립선 이외의 장기에서 생검한 경우 \newline
	난관- 생검 (Fallopian Tube- Biopsy) \newline
	난소- 생검 (Ovary- Biopsy) \newline
	외음/음순- 생검 (Vulva/Labia- Biopsy) \newline
	유방- 생검 (Breast- Biopsy) \newline
	자궁/경부내막- 소파술/생검/폴립절제 (Uterus/Endocervix- Curettage/Biopsy/Polypectomy) \newline
	자궁/자궁내막- 소파술/생검/폴립절제 (Uterus/Endometrium- Curettage/Biopsy/Polypectomy) \newline
	자궁경부- 생검 (Cervix- Biopsy) \newline
	질- 생검 (Vagina- Biopsy)  \\ \tabucline[.5pt]{-}
\rowcolor{Yellow!40} 다. Level C (C5603,4) & 1. 양성종양절제
	2. 위장관 폴립절제
	3. 태아 · 출혈 등의 이상이 있는 태반
	4. 병변 전체를 검색하여 치료방침을 결정해야 하는 비종양성 병변
	5. 골, 뇌, 간, 심근, 췌장, 연부조직, 고환, 전립선을 생검한 경우
	6. 양성종양에서 조직구축학적 방법으로 블록을 제작한 경우 \newline
	난관- 자궁외 임신 (Fallopian Tube- Ectopic Pregnancy)\newline
	난소 (난관포함 상관없이)- 신생물, 쐐기절제 (Ovary (w/ or w/o Tube)- Neoplastic, Wedge Resection(w/ or w/o Tube))\newline
	연부조직- 종괴(지방종외), 생검 (Soft Tissue- Mass(Not Lipoma), Biopsy )\newline
	유방- 맘모톰절제 (Breast- Mammotome Excision)\newline
	유방- 병소절제 (Breast- Excision of Lesion)\newline
	자궁-비종양/양성종양, 적출/절제(Uterus- non tumor/benign tumor, Hysterectomy/Resection)\newline
	자궁경부- 이형성, 원추형절제 (Cervix- Dysplasia, Conization)\newline
	자연/계류 유산 (Spontaneous/Missed- Abortion)\newline
	태반 (Placenta)\newline
	분류되지 않은 조직 및 장기- 양성종양, 절제 (Tissue/organ, unclassified- benign tumor, Resection/Excision)\\ \tabucline[.5pt]{-}
\rowcolor{Yellow!40} 라. Level D (C5605) &	1. 악성종양절제
	2. 경계형 악성 이상의 종양에서 조직구축학적검사를 시행한 경우 \newline
	외음- 악성종양, 아전/전절제 (Vulva- malignant tumor, Total/Subtotal Resection) \newline
	유방- 악성종양, 절제 (Breast- malignant tumor, Mastectomy )\newline
	자궁/난소- 악성종양 (Uterus/Ovaries- malignant tumor, hysterectomy/ Oophorectomy)\newline
	자궁경부- 악성종양, 원추형절제/지도화 원추형절제 (Cervix- malignant tumor, Conization/Mapping)\\ \tabucline[.5pt]{-}
\end{tabu}

\subsection{조직병리 검사 행위 재분류 상세 기준}
\paragraph{용어설명}
\begin{enumerate}\tightlist
\item 염증성, 감염성, 비종양성 병변 절제 (A) - 육앉으로 어느 정도 진단이 가능하고, 고도의 병리학적 지식이 필요하지 않는 경우
\item 생검 : 병변의 일부를 떼어서 병리진단을 요청하는 경우 (주요 악성여부를 확인하기 위해 시행)
\item 조직구축학적 검사 - 종양에서 병변의 위치, 크기, 분화도, 침윤깊이, 절제연 상태등을 판별하는 검사 (conization, EMR, ESD, Mass excision, Partial or total Resection에서 가능)
\end{enumerate}

\begin{center}
\includegraphics[width=.95\textwidth]{2018Bx}
\end{center}
\begin{mdframed}[linecolor=cyan,middlelinewidth=2]
기존의 조직검사는 조직을 채취하는 개수에 따라 별도의 보험코드를 사용하여 보험수가를 산정하였었습니다. 새롭게 변경되는 2018년부터는 조직 개수가 아닌
\begin{enumerate}[1)]\tightlist
\item 채취하는 부위
	\begin{itemize}\tightlist
	\item 양측장기 인정
	\item 양측성 장기의 각각 부위를 인정한다. (단, 감상선, 편도, 성대, 혈관, 피부제외)
	\item 난소와 난관은 자궁부속기로 합쳐서 양측 산정한다
	\end{itemize}
\item 양성 또는 악성 종양을 절제 또는 조직검사를 시행할 때로 구분
	\begin{itemize}\tightlist
	\item 생검에서 악성이 나와도 Level up안함.
	\item 절제(수술)조직에서는 양성종양이나 악성종양 여부- LEVEL변화 가능 ( A ↔ C or D, C ↔ D )
	\end{itemize}
\end{enumerate}
하여 별도의 보험 수가를 산정하고 있습니다
\paragraph{예외} : 연부조직 Lipoma(A), 비종양Appendix(A), 부수적 담낭절제(A), 외상성 수지절단(A), 비종양성 수지절단(B), 외상성 사지절단(B), 난관절제-ETP(C), 난소쐐기 절제- 신생물(C), 치원성 종양/난종 절제(C),  Morgagni cyst(A)\par
\paragraph{문제점} : 양성 VS 악성, 비종양성VS종양 ? (Endometriosis, wart, condyloma, fibromatosis, AK, SK, nodular fasciitis ?) 종양인지, 악성인지의 구분은 ICD-10과 ICD-O 3rd를 기준으로 해야 하고 피부병변에서 작은 조직이 생검인지 여부는 의뢰지에서 악성을 감별하고자 하는지? 판단\par
\paragraph{블록수 결정} : 대처적으로 명확한 것들은 근종과 VTH의 자궁등은 블록수 9개 까지, TAH등이나 ovarian cyst등은 블록수 10개 이상으로  
\end{mdframed}
