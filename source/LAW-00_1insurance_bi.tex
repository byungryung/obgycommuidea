\section{우리나라의 급여system의 이해와 독소조항}
\subsection{요양기관에서 제공되는 행위,약제,치료재료를 요양급여대상과 비급여 대상으로 구분하고 비급여 대상으로 고시되지 않은 항목은 모두 요양급여 대상으로 적용하는 negative system입니다.}
위의 이유로 의해 아무리 좋은치료라 해도 \uline{인정되지않는 비급여치료는 불법입니다.}
\subsection{요양급여의 대상여부 확인 (국민건강보험법 제43조의 2)}
가입자 또는 피부양자는 본인일부부담금
   외에 부당한 비용이 제39조 3항의 규정에 의하여 요양급여의 대상에서 제외되는 것인지를 건강보험심사평가원에 확인을 요청할수 있다
 확인요청을 받은 심평원은 그 결과를 확인 요청자에게 통보하여야하며,확인 요청한 비용이 요양급여의 대상에 해당되는 비용으로 확인되면 그내용을 공단및 관련 요양기관에 통보하여야한다 
통보받은 요양기관은 과다 징수한 금액을 지체없이 확인요청자에게 지급하여야한다
만일 당해 요양기관이 과다본인부담금을 지급하지 아니한경우에는 공단은 당해 요양기관에 지급할 요양급여비용에서 그 과다본인부담금을 공제하여 이를 확인 요청자에게 지급할수있다
\subsection{\newindex{100대100}이란?}
100분의 100 본인부담은 보건복지부장관이 정하여 고시한 상한금액을 환자가 모두 부담하는 것을 말합니다. 
비급여는 보건복지부장관이 정하여 고시한 진료항목에 대하여 해당 진료를 실시하는 병의원에서 정한 금액을 환자가 모두 부담하는 것입니다.
따라서, 100분의 100 본인부담과 비급여의 차이는 100분의 100 본인부담은 법령 등으로 정하여진 상한금액이 있어 어느 병\cntrdot{}의원에서든 동일한 금액을 환자에
게 징수하여야하나, 비급여는 정하여진 금액이 없어 동일한 진료행위인 경우라도병\cntrdot{}의원별로 금액이 상이할 수 있습니다. 예를 들어 쌍꺼풀수술, 점제거술 등은 비
급여대상 진료로 병\cntrdot{}의원별로 금액이 상이합니다.
 

\section{비급여 원내고시의무, 보건소보고의무X}
원내고시 내용만 변경하면 됩니다..비급여 항목,,,보건소 제출은 5년전에 (2010년 1월 의료법 제45조가 개정\cntrdot{}시행).....의무사항에서 제외되었습니다.

\subsection{비급여 대상의 항목(행위․약제․치료재료)과 그 가격을 적은 책자 등을 접수창구 등 환자 또는 환자의 보호자가 쉽게 볼 수 있는 장소에 비치}
\begin{itemize}\tightlist
\item `책자 등'이라 함은 비급여 진료비용이 모두 기재되어 환자들이 쉽게 열람할 수 있도록 의료기관 구내에 비치된 매체라면 폭넓게 인정됨. 
     * 제본된 책자, 제본되지 않은 인쇄물, 메뉴판, 벽보, 비용검색 전용 컴퓨터 등
\item `쉽게 볼 수 있는 장소의 범위'는 일반적으로 환자대기실\cntrdot{}접수창구 및 진료받은 비용을 정산할 수납창구 등이 될 수 있음.
\end{itemize}

\subsection{비급여 대상의 항목(행위․약제․치료재료)을 묶어 1회 비용을 정하여 총액을 표기 가능}
\begin{itemize}\tightlist
\item 건강보험 급여비용까지 포함하여 표기 가능(다만, 급여비용 포함된 가격임을 알 수 있도록 비고란 등에 표기)
\item 비급여 비용은 원칙적으로 단일 가격으로 고지해야 하나, 환자 상태에 따른 행위의 난이도 차이가 발생할 수 있으므로 범위를 설정하여 표기 가능(가급적 항목 분류 세분화, 가격범위 설정 이유 등 표기)
\item 비급여 진료비용이나 항목이 변경된 경우, 변경된 날짜를 기재하고, 변경된 내용을 표기
\item 고지된 가격 이하로 비용을 받는 것은 가능하지만, 이를 초과하여 징수하지 못하며, 위반시 시정명령 처분(「의료법」 제45조제3항 및 제63조)
\end{itemize}
\prezi{\clearpage}

%\clearpage
%\section{\newindex{비급여}란?}
%\subsection{건강보험법 내 비급여대상(제9조제1항관련) 에 대한 규정}
%\subsection{비급여대상(제9조제1항관련)}

\section{\newindex{보험에 관련된 흔한 질문들}}
%\noindent
%\begin{minipage}[t]{0.45\linewidth}% Question goes here
%    \textbf{Q.}
%    성주체성장애 환자는 보험으로 진료비는 청구 가능한지요?\index{보험QA!성주체성장애}
%\end{minipage}
%\hfill% Separate the Question and Answer
%\begin{minipage}{0.45\linewidth}% Answer goes here
%    \begin{mdframed}[linecolor=blue,middlelinewidth=2]
%    건강검진, 예방접종 등 예방진료로서 질병\cntrdot{}부상의 진료를 직접목적으로 하지 아니하는 경우에 실시 또는 사용되는 행위\cntrdot{}약제 및 치료재료 등에 해당되지 않으므로 성주체성장애는 %\colorbox{Yellow!80}{급여 대상}으로 진찰후 진찰료 청구가 가능합니다
%    \end{mdframed}
%\end{minipage}\par
%\bigskip
\Que{성주체성장애 환자는 보험으로 진료비는 청구 가능한지요?}\index{보험QA!성주체성장애}
\Ans{건강검진, 예방접종 등 예방진료로서 질병\cntrdot{}부상의 진료를 직접목적으로 하지 아니하는 경우에 실시 또는 사용되는 행위\cntrdot{}약제 및 치료재료 등에 해당되지 않으므로 성주체성장애는 \highlight{급여 대상}으로 진찰후 진찰료 청구가 가능합니다}

\Que{응급피임약 (또는 비아그라)만 처방하는 경우 진찰료 청구가 가능한가요?}\index{보험QA!응급피임약}
\Ans{응급피임약 (또는 비아그라)만 처방하는 경우는 \highlight{비급여 대상}으로 진찰료 청구가 불가합니다.}

\Que{팔에 심는 피임약(임플라논) 제거시 수가 산정 방법 및 급여 여부} \index{보험QA!임플라논제거}\par
\Ans{피임시술은 고시 제2010-45호(‘10.7.1 시행)에 의거 본인이 원하여 실시한 경우는 \highlight{비급여대상}입니다. 현재 임플라논의 시술 및 제거 관련하여는 사례별로 목적에 따라 급여여부가 결정되어야 할 것으로 사료되오니 이해있으시길바랍니다.\par
다만 시술 및 제거 관련 수기료에 대하여는 현재 별도 정하고 있지는 않으나「건강보험요양급여비용」 제1편2부9장 처치 및 수술료 산정지침 (3) 및 (4)에서는 다음과 같이 언급하고 있으니 업무에 참고하시기 바랍니다.\par
- 다 음 -\par
(3) 제1절에 기재되지 아니한 처치 및 수술로서 간단한 처치 및 수술의 비용은 기본진료료에 포함되므로 산정하지 아니한다.\par
(4) 제1절에 기재되지 아니한 처치 및 수술로서 위 “(3)”에 해당되지 아니하는 처치 및 수술료는 제1절에 기재되어 있는 처치 및 수술 중에서 가장 비슷한 처치 및 수술 분류항목의 소정점수에 의하여 산정한다
}

\Que{ 술전검사중 HCV Ab에 대한 보험적용여부} \index{보험QA!술전검사중 HCV Ab보험적용}
\Ans{국민건강보험 요양급여의 기준에 관한 규칙」[별표1] 요양급여의 적용기준 및 방법 1항 다목에 의하면 요양급여는 경제적으로 비용효과적인 방법으로 행하여야 하며 2항 가목에서는 각종 검사를 포함한 진단 및 치료행위는 진료상 필요하다고 인정되는 경우에 한하여야 하며 연구의 목적으로 하여서는 아니된다고 정하고 있습니다.
  
나487 C형간염항체검사(HCV Ab)는 인정기준(고시 제2009-250호)에 의거, 급여대상 질환이나 타 검사소견등 의학적 타당성이 확인(진료기록 등)되는 경우에는 급여대상이지만 아닌 경우에는 \highlight{비급여}임을 알려드립니다.
}
%\begin{Cdoing}{
\prezi{\clearpage}

\section{\newindex{인정비급여항목}들}
\begin{itemize}\tightlist
\item 초음파
\item 자궁경부확대촬영검사 Cervicography (비급여목록 노886
\item 상급병실료
\item 고주파 자궁근종 융해술
\item 자기공명영상유도하 고강도 초음파 집속술(자궁근종)
\item HAL\&RAR(Hemorrhoidal artery ligation and Rectoanal repair)
\item Rubella IgG avidity test
\item 성기능상담[NZ002]
\item 성기능 장애 평가[FZ684]
\item AMH
\item fragile X test
\item FISH 등입니다.
\end{itemize}