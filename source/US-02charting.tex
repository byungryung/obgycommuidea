\section{임신주수별 급여 횟수 및 차트기재 내용}
\begin{tabu}{|X[2]|X[3]|X[.8]|X[6]|}
\tabucline[.5pt]{-}
\tabulinesep =_1mm^1mm
%\linespread{1.1}
\rowfont{\sffamily} 시행 주수 &	진단 초음파 종류 &	횟수 &	확인 사항 \\\hline
임신 13주 이하 & 임신 제1삼분기 일반 & 2 & 임신 여부 및 자궁 및 부속기의 종합적인 확인을 하는 경우 *주: 임신 여부만을 확인하는 경우에는 460.55점 산정 \\\hline
임신 11-13주 & 임신 제 1삼분기 정밀 & 1 & 태아 목덜미 투명대 확인, 기형 진단 \\\hline
임신 14-19주 & 임신 제 2,3삼분기 일반 &	1 &	태아 안녕, 양수량 확인, 태아 성장 평가 \\\hline
임신 16주 이후 & 임신 제 2,3삼분기 정밀 & 1 & 태아 성장 및 기형여부 진단, 양수량, 태반 이상 유무 진단 \\\hline
임신 20-35주 & 임신 제 2,3삼분기 일반 &	1 &	태아 성장 및 안녕, 양수량, 태반 이상 유무 확인 \\\hline
임신 36주 이후 & 임신 제 2,3삼분기 일반 & 1 & 상 동 with 태아 위치 확인 \\\hline
\tabucline[.5pt]{-}
\end{tabu}

\begin{enumerate}[1.]\tightlist
\item 총7회 산전초음파급여. %\keystroke{Win}+\keystroke{s}\footnote{\url{https://dl.dropboxusercontent.com/u/6869091/2016obgysono/obgySONO.exe} \par\noindent\url{https://dl.dropboxusercontent.com/u/6869091/2016obgysono/obgy.sqlite}}
	\begin{enumerate}[①]\tightlist
	\item 임신1삼분기 2회
	\begin{commentbox}{}%헛갈리기 쉬운 사항들}%mdframed}[linecolor=blue,middlelinewidth=2]  
	\begin{itemize}\tightlist
	\item 요임신반응검사(EPT)와 B-hcg로 임신확인되어도 임신낭 보이지 않으면 급여 아닙니다. 
	\item 자궁외임신은 부인과초음파로 급여 아닙니다\footnote{다만 초음파상 나팔관에서 G-sac이나 Yolk-sac이 확인된경우는 보험 초음파산정가능}%태아확인시는 
	\item 계류유산수술시엔 수술초음파(비보험)입니다. (from 산부인과학회)
	\end{itemize}
	\end{commentbox}%mdframed}
	
	\item 임신1삼분기 정밀(11-13주) 1회
	\item 임신 14-19주 1회
	\item 임신 16주이후 정밀 초음파 1회
	\item 임신 20-35주 1회
	\item 임신 36주 이후 1회
	\end{enumerate}
	\begin{commentbox}{}%헛갈리기 쉬운 사항들}%mdframed}[linecolor=red,middlelinewidth=2]  
	기형아 정밀계측 초음파 검사 후 F/U 검사하는 경우도 태아에게 이상이 있는 경우이므로 횟수제한 초과 일반 또는 일반의 제한적 초음파로 급여 산정(고시 또는 \textsf{Q\&A}에 없는 내용)
	\begin{itemize}\tightlist
	\item 일반 초음파 확인 사항 모두 포함하여 태아 기형 F/U 하는 경우: 일반 초음파
	\item 해당 태아기형만 F/U 하는 경우: 일반의 제한적 초음파 (아래 ‘제한적 초음파’ 참조)
	\end{itemize}
	\end{commentbox}%mdframed}
\item 도플러 10\% 가산
\item 다태의 경우 태아수에 따라 소정점수 산정
\item 분만기간 초음파는 비급여
\item 야간및 응급 가산은 안됩니다.
\end{enumerate}
