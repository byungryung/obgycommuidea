\section{현지조사 거짓\cntrdot{}부당청구 확인사례}
\subsection{거짓청구 VS 부당청구}
\begin{description}\tightlist
\item[거짓청구] \highlightR{진료비 청구의 원인이 되는 진료 행위가 실제 존재하지 않음}에도 관련서류의 거짓작성 또는 속임수 등의 부정한 방법에 의해 진료비를 청구한 행위
	\begin{itemize}\tightlist
	\item 입원 및 내원 거짓 또는 증일청구
	\item 미실시 행위료등 청구
	\item 비급여대상 진료후 이중청구
	\end{itemize}
\item[부당청구] \highlightR{진료비 청구의 원인이 되는 진료행위는 실제 존재}하나, 진료행위가 건강보험법 및 의료법 등 관계법령을 위반하여 부정하게 이루어진 진료비 청구행위
	\begin{itemize}\tightlist
	\item 요양급여 산정기준 위반청구
	\item 행위료, 의약품 등 대체청구
	\item 인력,시설,장비 신고 위반청구
	\item 본인부담금 과다징수등
	\end{itemize}
\end{description}
\subsection{복지부 실사 다빈도 사례 분석표 (2016. 10. 31. 의협신문)} 
\href{http://www.doctorsnews.co.kr/news/articleView.html?idxno=113498}{의원급 실사 '비급여 이중청구' 가장 많아}
\par
\medskip

\tabulinesep =_2mm^2mm
\begin{tabu} to \linewidth {|X[1,c]|X[1,c]|X[2,l]|} \tabucline[.5pt]{-}
\rowcolor{Gray!25}  유형  & 빈 도 & 사 례 \\ \tabucline[.5pt]{-}
\rowcolor{Yellow!5} 비급여 이중청구 & 43건(39.8\%) & \textbf{▶} 비급여 과정에서 진찰료 청구(피부 미용시술, 예방접종, 단순 영양제 투여)\newline \textbf{▶} 비급여 항목을 급여로 청구\\ \tabucline[.5pt]{-}
\rowcolor{Yellow!5} 거래량-청구량 불일치 & 15건 (13.8\%) & \textbf{▶} 주사제, 1회용 겸자, 하기도증기흡입시 흡입제 등의 거래량이 청구량보다 적은 경우 \\ \tabucline[.5pt]{-}
\rowcolor{Yellow!5} 검진당일 대장내시경 청구 & 13건(12.0\%) & \\ \tabucline[.5pt]{-}
\rowcolor{Yellow!5} 임의 비급여 & 11건 (10.2\%) & \textbf{▶} 주사제, 수액제, 검사 등 급여항목을 비급여로 청구 \\ \tabucline[.5pt]{-}
\rowcolor{Yellow!5} 본인부담금 할인 & 9건(8.3\%) & \\ \tabucline[.5pt]{-}
\rowcolor{Yellow!5} 의료법 및 의료기사법 위반 & 9건 (8.3\%) & \textbf{▶} 무자격자 진료보조 \newline \textbf{▶} 진료기록부 허위기재 \newline \textbf{▶} 전화상담후 청구 \newline \textbf{▶} 일회용 주사기나 겸자 재사용 등 \\ \tabucline[.5pt]{-}
\rowcolor{Yellow!5} 미진료 청구 & 3건(2.7\%) & \textbf{▶} 입원중 외출\cntrdot{}외박 환자 청구 \newline \textbf{▶} 가족을 보지 않은 상태에서 가족치료 청구 \\ \tabucline[.5pt]{-}
\rowcolor{Yellow!5} 기타 다빈도 사례 & 21건 (19.4\%) & \textbf{▶} 외이도이물제거 관련 \newline \textbf{▶} 대리청방 관련 \newline \textbf{▶} 직원상근관련 여부 \newline \textbf{▶} 직원 및 친인척 진료 \\ \tabucline[.5pt]{-}
\end{tabu}
%\par
%\medskip
\subsection{거짓청구 유형(허위청구)}
\begin{enumerate}[①]\tightlist
\item 비급여 이중 청구
	\begin{itemize}\tightlist
	\item 비급여 진료(예방 접종, 비만 치료 등) 후 진찰료를 산정하는 경우
	\item 국가암 검진 시 진찰료를 청구하는 경우
	\item 비급여 항목을 비급여로 받고 급여로 다시 청구하는 경우
	\end{itemize}	
\item 미진료 시 진찰료 청구
	\begin{itemize}\tightlist
	\item 특히 수진자의 해외 출국 기록 확인함(출국 기간 중 진료 기록이 있으면 문제가 됨)
	\end{itemize}	
\item 거래량 / 청구량 불일치
	\begin{itemize}\tightlist
	\item 약제 및 물품 관리 시 남아있는 양을 기준으로 재구입하면 안 되고 청구양을 기준으로 재구입해야 한다.
	\item 질정 조심
	\item 혹시 DC를 하신다면 DC 때 사용하시는 유착 방지제 및 영양제의 경우 기록을 남기지 않을텐데, 물품 매입 기록과 재고량이 너무 차이가 나면 의심받을 가능성 있음
	\end{itemize}	
\item 처치 /행위료 허위 청구
	\begin{itemize}\tightlist
	\item 행위료의 경우 행위를 했다는 기록 외에는 다른 어떠한 물증이 없는 상태임. 실사 시 수진자들에게 전화로 확인해서 그런 진료 행위를 받은 기억이 없다는 진술을 받으려 함.
	\end{itemize}	
\item 미실시 검사 허위 청구
	\begin{itemize}\tightlist
	\item 실사 시 방사선 검사 기록(PACS나 필름) / 임상 병리 검사 결과지 등으로 확인함
	\item 기록이 없으면 허위 청구로 몰아가려 함.(특히 원내 검사 결과 관리 잘해야 함)
	\end{itemize}	
\item 본인부담금 할인 : 환자 유인행위로 간주하며, 할인된 본인부담금과 연관된 공단 청구 금액을 허위 청구로 간주함
	\begin{itemize}\tightlist
	\item 특히 직원 할인의 경우 조심해야 한다.
	\item 직원 약 처방 때문에 접수 후 약 처방하고 본인 부담금 안 받으면 안됨.
	\item 직원 할인 / 직원 가족 할인의 경우 모두 불법임(할인을 해줄 경우 그 차액만큼을 상여금을 지급한 것으로 해서 세무상 증빙 서류를 만들어 놓아야만 합법임)
	\end{itemize}
\end{enumerate}

\begin{commentbox}{실제 내원하지 않은 일자에 내원한 것으로 요양급여비용을 청구}
특히 수진자의 \emph{해외 출국 기록 확인함}. (출국 기간 중 진료 기록이 있으면 문제가 됨)
\begin{description}\tightlist
\item[관련근거] 요양급여비용의 청구는 국민건강보험법 제47조(요양급여비용의 청구와 지급 등)와 의료법 제22조(진료기록부 등) 제1항 등에 의거 요양기관에 내원한 수신자에 대하여 실제 진료한 내역을 기록한 진료기록부 등에 의하여 정확히 청구하여야 함.
\item[부당사례]
\begin{enumerate}[1)]\tightlist
\item A의원은 수진자 ㅇㅇㅇ에 대해 4일 외래진료를 받은 것으로 청구하였으나, 실제로는 \highlight{이틀만 내원하였고, 이틀은 내원하지 않았음에도 내원하여 진료를 받은 것으로 거짓기록}하고 진찰료 등을 청구
\item B의원은 수진자 ㅇㅇㅇ의 경우, 2013년 5월 4일, 5월 8일, 5월 13일 ‘어깨의 유착성 관절낭염(M750)’상병으로 내원하여 진료받은 것으로 청구하였으나, 실제로 2013년 5월2일부터 5월 14일까지 \highlight{해외출국 중으로 진료받은 사실이 없음에도 거짓기록 후 청구}
\end{enumerate}
\end{description}
\end{commentbox}

\begin{commentbox}{실제 시행하지 않은 검사료 거짓청구}
\begin{description}\tightlist
\item[관련근거] 요양급여비용의 청구는 국민건강보험법 제47조(요양급여비용의 청구와 지급 등)와 의료법 제22조(진료기록부 등) 제1항 등에 의거 요양기관에 내원한 수신자에 대하여 실제 진료한 내역을 기록한 진료기록부 등에 의하여 정확히 청구하여야 함.
\item[부당사례]
\begin{enumerate}[1)]\tightlist
\item A, B의원은 실시하지 않은 당검사(반정량)를 실시한 것으로 진료기록부에 거짓 기록한 후 검사료 등 요양급여비용을 청구
\item C의원은 인성검사-간이정신진단검사(F6216), 치매척도검사[GDS](F6221)를 실시 하지 않고 해당 검사료를 요양급여비용으로 청구
\end{enumerate}
\end{description}
\end{commentbox}

\begin{commentbox}{실제 시행하지 않은 이학요법료 거짓청구}
\begin{description}\tightlist
\item[관련근거] 건강보험 행위급여\cntrdot{}비급여 목록표 및 급여상대가치점수 제7장 이학요법료 - 해당항목의 물리치료를 실시할 수 있는 일정한 면적의 치료실과 실제 사용할 수 있는 장비를 보유하고 있는 요양기관에서 의사의 처방에 따라 상근물리치료사가 실시하고, 그 결과를 진료기록부에 기록한 경우에 요양급여비용으로 산정
\item[부당사례]
\begin{enumerate}[1)]\tightlist
\item  A의원은 근막염 등의 상병으로 내원한 수진자 ㅇㅇㅇ에게 실제 시행하지 않은 표층열 치료(MM010)와 단순운동치료[일당](MM101) 등을 시행한 것으로 요양 급여비용을 청구
\item B의원은 이학요법 처방은 하였으나 물리치료 실시 기록대장을 확인한 결과 심층 열치료 [1일당](MM020)를 실시하지 않고, 요양급여비용을 청구함
\end{enumerate}
\end{description}
\end{commentbox}

비급여대상(피부관리)을 전액 환자에게 부담시킨 후 요양급여비용으로 이중청구
\begin{description}\tightlist
\item[관련근거] 건강건강보험요양급여의 기준에 관한 규칙 제9조 제1항 관련[별표2] 업무 또는 일상생활에 지장이 없는 경우, 신체의 필수 기능개선 목적이 아닌 경우, 예방진료로써 질병\cntrdot{}부상의 진료를 직접 목적으로 하지 아니하는 경우에 실시 또는 사용되는 행위\cntrdot{}약제 및 치료재료 등은 비급여 대상이므로 요양급여비용으로 청구 할 수 없음.
\item[부당사례]
\begin{enumerate}[1)]\tightlist
\item  A의원은 비급여대상인 모공, 안면홍조 등 \highlightR{피부관리}를 위해 2일간 내원한 수진자 ㅇㅇㅇ에 대해 부분 혈관 레이저 등을 시술하고 그 비용을 \highlightR{비급여로 환자에게 전액 징수하였음}에도 ‘장미색잔비늘증(비강진)(L42)’ 상병으로 진찰료를 요양급여 비용으로 이중청구함
\end{enumerate}
\end{description}

비급여대상(예방접종, 비만)을 전액 환자에게 부담시킨 후 요양급여비용으로 이중청구
\begin{description}\tightlist
\item[관련근거] 건강건강보험요양급여의 기준에 관한 규칙 제9조 제1항 관련[별표2] 업무 또는 일상생활에 지장이 없는 경우, 신체의 필수 기능개선 목적이 아닌 경우, 예방진료로써 질병\cntrdot{}부상의 진료를 직접 목적으로 하지 아니하는 경우에 실시 또는 사용되는 행위\cntrdot{}약제 및 치료재료 등은 비급여 대상이므로 요양급여비용으로 청구 할 수 없음.
\item[부당사례]
\begin{enumerate}[1)]\tightlist
\item  A의원은 독감접종을 위해 내원한 수진자 ㅇㅇㅇ에 대하여 진찰 및 \highlight{독감접종을 실시하고 그 비용을 비급여로 환자에게 전액 징수하였음에도 ‘소화불량(K30)’ 상병으로 내원하여 진료받은 것으로 진찰료를 요양급여비용으로 이중청구함}
\item B의원은 \highlight{단순비만을 진료 후 비용을 비급여로 전액징수 후 ‘상세불명의 고혈당증(R739)’ 상병으로 진찰료 등을 요양급여비용으로 이중청구함}
\end{enumerate}
\end{description}

비급여대상(희망검진)을 전액 환자에게 부담시킨 후 요양급여비용으로 이중청구
\begin{description}\tightlist
\item[관련근거] 건강건강보험요양급여의 기준에 관한 규칙 제9조 제1항 관련[별표2] 업무 또는 일상생활에 지장이 없는 경우, 신체의 필수 기능개선 목적이 아닌 경우, 예방진료로써 질병\cntrdot{}부상의 진료를 직접 목적으로 하지 아니하는 경우에 실시 또는 사용되는 행위\cntrdot{}약제 및 치료재료 등은 비급여 대상이므로 요양급여비용으로 청구 할 수 없음.
\item[부당사례]
\begin{enumerate}[1)]\tightlist
\item  A의원은 \highlight{본인 희망검진으로} 내원한 수진자 ㅇㅇㅇ에 대해 그 비용을 \highlight{비급여로 전액 징수하였음에도 ‘만성 표재성 위염(K293)’ 등의 상병으로 진찰료 및 검사료 등을 요양급여비용으로 이중청구함}
\end{enumerate}
\end{description}

\begin{commentbox}{공단 건강검진 실시 후 검진비용 및 요양급여비용 이중청구}
\begin{description}\tightlist
\item[관련근거] 
\begin{itemize}\tightlist
\item 국민건강보험법 제52조(건강검진), 건강검진기본법('08.3.21 제정, 법률 제8942호)
\item 건강검진실시기준, 암검진실시기준 등
\end{itemize}
\item[부당사례]
\begin{enumerate}[1)]\tightlist
\item  A의원은 검진항목에 포함되어있는 혈당검사 등의 혈액검사를 실시하고 관련 \highlight{검진비용으로 청구하고, 요양급여비용으로 이중청구}
\end{enumerate}
\end{description}
\end{commentbox}
