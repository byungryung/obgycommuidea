\section{고위험임산부 집중관리료}

\subsection{고위험임산부 집중관리료(1일당)}
\tabulinesep =_2mm^2mm
\begin {tabu} to\linewidth {|X[4,l]|X[1,l]|X[1,l]|} \tabucline[.5pt]{-}
\rowcolor{ForestGreen!40} \centering 분류 & \centering 코드 & \centering 금액 \\ \tabucline[.5pt]{-}
\rowcolor{Yellow!40} 고위험임산부 집중관리료6시간 미만 & ac731 & 13,320 \\ \tabucline[.5pt]{-}
\rowcolor{Yellow!40} 고위험임산부 집중관리료6시간 이상 & ac732 & 19,980 \\ \tabucline[.5pt]{-}
\end{tabu}

\subsection{급여기준}
고위험임산부 집중관리료 산정대상
\begin{enumerate}[1)]\tightlist
\item 고위험임산부 집중치료실 입원료 및 고위험임산부 집중관리료는 다음중 하나 이상에 해당되는 임산부로서 집중치료가 필요하다고 의사가 판단한 경우에 인정함.\newline
\emph{ - 다 음 - }
	\begin{enumerate}[가.]\tightlist
	\item 자궁 수축 검사상 20분동안 4회 이상 또는 1시간 동안 8회 이상의 자궁수축이 관찰된 임신 34주 이내의 조기진통
	\item 임신 34주 이내의 조기 양막파열
	\item 자궁경부무력증으로 응급수술 전\cntrdot{}후 경과관찰을 요하는 경우
	\item 중증 전자간증 또는 자간증
	\item 양수과소증 또는 양수과다증
	\item 자궁내 발육지연
	\item 쌍태간 수혈증후군
	\item 산과적 출혈
	\item 38도 이상의 고열이 있는 임산부
	\end{enumerate}
	\begin{itemize}[*]\tightlist
	\item 상기 적응증에 해당함에도 집중치료가 아닌 단순 분만을 위해 대기 또는 검진만을 위해 내원한 경우에는 산정할 수 없음
	\end{itemize}
\item 고위험임산부 집중관리료는 '고위험임산부 집중치료실 입원료 및 집중관리료 산정대상'에 해당하나 '고위험임산부 집중치료실 입원료 급여기준'을 충족하지 못한 경우, 다음의 요건을 갖추고 고위험임산부를 집중치료한 경우에 인정함.
	\begin{enumerate}[가.]\tightlist
	\item 산정기관 : 분만실을 신고\cntrdot{}운영하며, 산부인과 전문의가 1인 이상 상근하는 의료기관
	\item 장비
		\begin{itemize}[-]\tightlist
		\item 의료가스시설
		\item 태아감시장치, 심전도모니터, 맥박산소계측기
		\item 심전도기록기, 도플러, 초음파기기
		\item 지속적 수액 주입기
		\end{itemize}
	\end{enumerate}
	\begin{enumerate}[1.]\tightlist
	\item 단, 고위험임산부 집중치료실을 운영하나 불가피하게 고위험임산부 집중치료실(unit) 외에서 고위험임산부를 집중치료한 경우에도 고위험임산부 집중관리료를 인정함.
	\item 산정횟수 : 고위험임산부 집중관리료는 1일 1회 산정하며, 입원환자의 경우에는 입원기간 중 7회 이내로 산정함. 단, 동일날 고위험임산부 집중치료실 입원료와 동시에 산정 할 수 없음.
	\end{enumerate}	
\end{enumerate}	