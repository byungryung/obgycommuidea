\section{태아심음자궁수축검사}
\begin{paracol}{2}
\setlength{\columnseprule}{0.4pt}
\setlength{\columnsep}{2em}
\begin{leftcolumn}
\begin{commentbox}{}
\begin{itemize}\tightlist
\item[\dsjuridical] O600 분만이 없는 조기진통 (의증 or 배제진단)
\item[\dsjuridical] O470 False labor before 37 weeks
\item[\dsmedical] 나-732-2 (E7326)
\end{itemize}
\end{commentbox}
%\medskip
%\centering

%\includegraphics[width=0.75\linewidth]{labial-fusion}
\end{leftcolumn}

\begin{rightcolumn}
인정기준 
\begin{enumerate}[가.]\tightlist
\item 한시간 당 8회 이상의 자궁수축이 느껴지거나(=OR) 조기진통이 의심되는 37주 이전 산모
\item 유도분만을 시도하는 산모
\end{enumerate}
\end{rightcolumn}
\end{paracol} 
\prezi{\clearpage}
\subsection{유도분만시 태아심음자궁수축검사와 분만전감시의 수가선정방법}
나 732-2 태아심음자궁수축검사는 나732 분만전감시와 같은 날에 실시한 경우에는 별도 산정하지 아니한다고 규정하고 있는 바, 유도분만을 시도하는 산모가 분만한 당일에는 나732 분만전감시의 소정점수만 산정함.\par
다만, 유도분만을 시도하는 산모의 분만이 24시간을 초과하지 않은 경우에 실시된 나732-2 태아심음자궁수축검사는 같은 날이 아니더라도 나732 분만전감시의 소정점수만 산정합니다.\par

\emph{예시>}\par
\begin{itemize}\tightlist
\item 유도분만으로 감시 시작 2.27 7:00am, 유도분만으로 감시 종료 2.28. 1:00pm(총 30시간)
 :  나732 분만전감시 X 1회, 나732-2 태아심음자궁수축검사 X 1회 각각 산정
\item 유도분만으로 감시 시작 2.27. 7:00am, 유도분만으로 감시 종료 2.28. 5:00am(총 22시간)
        :  나732 분만전감시(12시간초과) X 1회 산정
\end{itemize}
\prezi{\clearpage}
\begin{commentbox}{태아심음자궁수축검사와 비자극검사의 차이점}
나732-2 태아심음자궁수축검사는 나732-가 분만전감시-전자태아감시, 나732-1 비자극검사와 동일한 장비를 사용하나 실시대상과 목적이 다름
\begin{itemize}\tightlist
\item 비자극검사: 자궁수축이 없는 상태에서 태아심박동수 측정
\item 분만전감시-전자태아감시: 분만 진통 중 자궁수축에 따른 태아심박동의 변화를 관찰
\item 태아심음자궁수축검사: 조기진통 및 유도분만을 시도하는 산모에게 자궁수축의 강도와 빈도 측정
\end{itemize}
\end{commentbox}