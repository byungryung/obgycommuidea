\section{AUB:ADOLESCENT}
\myde{}{%
\begin{itemize}\tightlist
\item[\dsjuridical] N915 상세불명의 희발월경
\item[\dsjuridical] N921 불규칙적 주기를 가진 과다 및 빈발 월경 
\item[\dsjuridical] C541 자궁체부의 악성신생물, 자궁내막 의증(or 배제진단)
\item[\dsjuridical] E079 상세불명의 갑상선의 장애
\item[\dsjuridical] O089 유산, 자궁외임신 및 기태임신에 따른 상세불명의 합병증
\item[\dsjuridical] A638 기타 명시된 주로 성행위로 전파되는 질환	
\item[\dsjuridical] D50 철결핍빈혈 
\item[\dsjuridical] D51 비타민B12결핍빈혈, D52 엽산결핍빈혈  
\end{itemize}   
}%
{\begin{itemize}\tightlist
\item Check point 
	\begin{enumerate}\tightlist
	\item 초경 <2yrs: anovulation due to neuroendocrine immaturity 
	\item Sexually active -> 임신의 가능성  
	\item 체중 변화, 과한 운동, 질환, stress -> estrogen level 낮아 자궁 내막의 증식이 충분하지 않아 출혈 양은 소량임 
	\item 출혈성 질환의 병력/가족력  
	\item Signs of trauma or sexual abuse? 
	\item Genital tract infections
	\end{enumerate}
	
\item 필요한 검사류
	\begin{enumerate}\tightlist
	\item Anemia evaluation: CBC Ferritin등 
	\item Pregnancy evaluation: Serum or Urine bHCG 
	\item Bleeding tendency evaluation: PT/PTT 
	\item STD6 or Culture 
	\item Hormone evaluation: Female Hormone / TFT
	\item Sono
	\end{enumerate}
\end{itemize}
}
\begin{commentbox}{Normal menstruation}
	\begin{itemize}[-]\tightlist
	\item 주기: 21-35 days  
	\item 기간: 2-7 days (평균 4.7일) 
	\item Amount: 30-40mL 
	\end{itemize}
	\begin{itemize}[-]\tightlist
	\item 임상적으로 의미가 있는 자궁출혈
	\item 주기: <21 or >35 days
	\item 기간: <2 or >7 days
	\item Amount: >80mL
	\item 월경과 관련 없는 자궁출혈 
	\end{itemize}
\end{commentbox}

\subsection{Termiology}
\begin{itemize}\tightlist
\item Amenorrhea(무월경): absence of menstruation for at least three usual cycle lengths
\item Oligomenorrhea(희발 월경): interval >35 days 
\item Polymenorrhea(번발 월경): interval <21 days 
\item Menorrhagia(월경 과다): regular intervals, >7days, or >80cc 
\item Metrorrhagia(부정자궁출혈): light bleeding irregular intervals 
\item Menometrorrhagia(기능성 자궁출혈): irregular intervals, >7days 
\item Intermenstrual bleeding: bleeding occurring between menses 
\item Premenstrual bleeding: light bleeding preceding regular menses 
\item Postcoital bleeding: vaginal bleeding within 24 hours of intercourse 
\end{itemize}

한마디로 정상적인 월경의 패턴에서 벗어난 자궁출혈 
\begin{enumerate}\tightlist
\item Hormone  관점
	\begin{itemize}[-]\tightlist
	\item 생리를 제외한 모든 출혈
	\item 생리가 되려면 배란이 되고, 임신은 안 되는 두 가지 조건이 전제 되어야 합니다.
	\end{itemize}
\item Bleeding pattern  관점
	\begin{itemize}[-]\tightlist
	\item Amenorrhea(무월경): absence of menstruation for at least three usual cycle lengths 
	\item Oligomenorrhea(희발 월경): interval >35 days 
	\item Polymenorrhea(번발 월경): interval < 21 days
	\item Menorrhagia(월경 과다): regular intervals, >7days, or >80cc
	\item Metrorrhagia(부정자궁출혈): light bleeding irregular intervals 
	\item Menometrorrhagia(기능성 자궁출혈): irregular intervals, >7days 
	\end{itemize}
\end{enumerate}

%정상월경이란 ? 생리 14일경 배란이 일어나고 임신이 안된 경우\\
%\begin{center}
%\includegraphics[scale=.7]{mense}
%\end{center}

\subsection{Est breakthrough B vs Est withdrawal B vs Prog. Breakthrough B vs Prog. Withdrawal B}
\begin{enumerate}\tightlist
\item Estrogen breakthrough bleeding
	\begin{itemize}[-]\tightlist
	\item Excess estrogen
	\item Insufficient progesterone: structural support 상실
	\end{itemize}
\item Estrogen withdrawal bleeding
	\begin{itemize}[-]\tightlist
	\item Estrogen 갑작스런 감소
	\item (eg. bilateral oophorectomy,HRT 중단, 배란 직전) 
	\end{itemize}
\item Progesterone breakthrough bleeding
	\begin{itemize}[-]\tightlist
	\item progesterone/estrogen ration 높은 경우 (eg. Progesterone만 함유한 피임제) 
	\item Estrogen 부족으로 atrophic endometrium
	\end{itemize}
\item Progesterone withdrawal bleeding
	\begin{itemize}[-]\tightlist
	\item Normal menstrual cycle
	\end{itemize}
\end{enumerate}	