\section{출산진료비 지원}
\myde{}{%
\emph{급여 항목}
\begin{itemize}\tightlist
\item[\dsmedical] 경막외 마취관리 기본[1시간 기준]: L1214 \myexplfn{916.88} 원%(68200원)
\item[\dsmedical] 경막외 마취유지[1시간초과후 15분 증가할때마다] : L1224 \myexplfn{126.25} 원
\item[\dsmedical] 마취통증의학과 초빙료: L7990 \myexplfn{1415.18} 원 %(105290원)
\item[\dsmedical] 산소흡입: M0040(6280원)
\item[\dsmedical] pulse oxymeter: L1310[마취중 말초산소포화도감시](2580원)
\item[\dsmedical] 펜다닐, 부피바카인등 약물
\item[\dsmedical] Epidural set 재료대
\end{itemize}
}% 
{
자연분만시 본인부담금 면제 대상 적용범주\par
\begin{center}\emph{- 다   음 -}\end{center}
\begin{enumerate}[1.]\tightlist
\item 건강보험 가입자 또는 피부양자인 임산부가 자연분만을 하면 입원진료비 중 본인부담금이 전액 면제되며, 식대는 50\% 면제됩니다
\item 국민건강보험법시행령 [별표2]제3호의 규정에 의하여 본인이 부담할 비용을 면제하는 자연분만은 자435분만, 자436둔위분만, 자438 제왕절개술기왕력이 있는 질식분만, 카1 조산료, 타2 보건진료소 조산료, 타3다 보건지소 조산료, 타4다 보건소 조산료를 말함.
\item 다만, 자연분만을 시도하였으나 제왕절개술을 시행한 경우, 분만을 위해 입원하였으나 분만이 이루어지지 않은 경우는 해당되지 아니함.\newline
(시행일 : 2012.12.1.시행)
\end{enumerate}
}%

\par
\medskip
\begin{commentbox}{무통환불사태}
2004년 8월 산부인과 의사들은 제도적 미비와 낮은 수가에 항의해 무통분만 시술 거부를 선언했다. 일부 산부인과에서 건강보험에서 정한 수가 이상으로 무통분만시술비를 징수한 것이 알려지면서 환자들의 환불요구가 잇달았고 심평원 민원으로 환불결정이 나면서부터 당시 건강보험상 뜨거운 이슈가 되었다. 무통분만시술은 ‘100분의 100’이란 보험적용을 받았다.\par
\medskip

보험 대상으로 지정해 수가를 통제하긴 하되, 비용 전액을 환자가 부담토록 한 것이다. 그런데 책정된 무통분만시술료는 대략 2만8,000원으로 이 비용으로 마취과의사를 초빙하는 비용까지 포함이 된 것이다. 상식적으로 그 금액으로 마취과의사를 초빙할 수 없다는 것을 이해한다면 수가가 현실을 반영하지 못하기 때문에 빚어진 사태이다. 그러다보니 병원에서 15만원 정도를 받았고 정해진 금액 외에는 환불하라는 결정에 의사들이 시술거부를 선언한 것이다.
\end{commentbox}

\par
\medskip
\Que{경막외마취에 의한 무통분만 후 질식 분만에 실패하여 제왕절개만출술을 시행하는 경우} 
\Ans{경막외마취에 의한 무통분만 후 질식 분만에 실패하여 제왕절개만출술을 시행하는 경우 두 가지 마취 행위는 동일 목적으로 볼수 없으므로 각각의 마취료를 인정할 것을 요청한 사항에 대하여,분만전 통증조절 목적으로 실시하는 경막외마취는 현재 건강보험요양급여비용및그상대가치점수 제5장 마취료 산정지침(5)에 의거 " 동일목적을 위하여 2 이상의 마취를 병용한 경우 또는 마취중에 다른 마취법으로 변경한 경우에는 주된 마취의 소정점수만 산정한다."는 기준에 의거 주된 마취 한가지만 인정됨을 알려드립니다.}

\begin{commentbox}{마취통증의학과 전문의가 상근하는 요양기관에서 마취통증의학과 전문의 초빙시 인정여부}
마취통증의학과 전문의가 상근하는 요양기관에서 마취통증의학과 전문의 초빙료를 산정할 수 있는 경우는 다음과 같으며, 요양급여비용 청구 시 부득이한 사유 또는 신고사실을 확인할 수 있도록 마취기록부, 변경신고서 등 객관적인 증빙자료를 첨부하여야 함. 
\begin{center}\emph{- 다 음 -}\end{center}
\begin{enumerate}[가.]\tightlist
\item 상근하는 마취통증의학과 전문의가 예비군 훈련 등 부득이한 사유로 부재중인 경우 수술이 가능한 다른 요양기관으로 환자를 이송 조치함이 원칙이나 이송할 수 없는 상황에서 마취통증의학과 전문의를 초빙하는 경우. 다만, 이 경우 관련 법령에 의거 인력 등에 대한 변경신고(유선신고 포함)가 이루어져야함. 
\item  천재지변, 기타 예기치 못한 구급사태 등으로 인하여 동일 시간대에 2인 이상의 수술이 동시에 이루어져야 할 부득이한 사유로 마취통증의학과 전문의를 초빙하는 경우
\item 마취통증의학과 전문의가 상근하는 산부인과 병의원에서 야간 또는 공휴일에 임신 또는 분만관련 응급수술을 시행하게 되어 부득이하게 마취통증의학과 전문의를 초빙하는 경우\newline
(시행일 : '12.12.1일 부터)
\end{enumerate}  
☞ 변경사유 : 마취통증의학과 전문의가 상근하는 산부인과 병ㆍ의원에서 야간 또는 공휴일 응급수술을 위하여 불가피하게 마취통증의학과 전문의를 초빙시 초빙료를 요양급여로 인정. \emph{마취통증의학과전문의초빙료:L7990(105290원)}
\end{commentbox}

