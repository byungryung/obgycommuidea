\section{Urinary incontinence: 수술적 치료}
\myde{}{
\begin{itemize}\tightlist
\item[\dsmedical] N393 스트레스요실금
\item[\dsmedical] R3565 요실금수술. 가. 질강을 통한 수술 (2) 기타의 경우 [\myexplfn{3477.58}
%\item[\dsmedical] 
%\item[\dsmedical] 
%\item[\dsmedical] 
%\item[\dsmedical] 
%\item[\dsmedical] 
%\item[\dsmedical] 
\end{itemize}
}
{
\Large \textbf{인조테이프를 이용한 요실금수술 인정기준}\normalsize
\begin{enumerate}[1.]\tightlist
\item 인조테이프를 이용한 요실금수술은 \uline{요류역학검사로 복압성 요실금 또는 복압성 요실금이 주된 혼합성 요실금이 확인되어 수술적 치료가 필요한 경우}에 인정함. 다만, 진료담당의사의 요류역학검사 판독소견서는 다음 항목을 포함하여 작성하고, \textcolor{red}{요양급여비용 청구시 판독소견서와 관련 검사결과지를 첨부하여 제출토록 함.}\par
- 다 음 - 
	\begin{enumerate}[가.]\tightlist
	\item 총 방광용적 (Maximal Cystometric Capacity) 
	\item 감각(Sensation)의 증감 혹은 정상 여부 
	\item 유순도(Compliance)의 증감 혹은 정상 여부 
	\item 비억제성배뇨근수축(Uninhibited Detrusor Contraction)의 유무 - 복압(Pabd), 배뇨근압(Pdet) 포함
	\item 요류검사시(Uroflometry) 최고요속 및 배뇨량, 배뇨후 잔뇨량 
	\item 요누출압(Valsalva Leak Point Pressure, VLPP 또는 Coughing Leak Point Pressure, CLPP) 측정시
		\begin{enumerate}[(1)]\tightlist
		\item 충전 방광용적 (Bladder filling Volume) 
		\item 요누출 (Urine Leakage) 유무 
		\item 방광내압(Pves)
		\end{enumerate}
	\item 최대요도폐쇄압(Maximal Urethral Closing Pressure)
  	\end{enumerate}
\item 동 인정기준 이외에는 비용효과성이 떨어지고 \uline{치료보다 예방적 목적이 크다고 간주하여 제반 진료비용(입원료, 마취료 및 치료재료 비용 등) 은 요양급여하지 아니함(비급여)}.
(시행일 : 2011.12.1일부터)
\end{enumerate}
}
%\clearpage
\par
\medskip
\tabulinesep =_2mm^2mm
\begin {tabu} to\linewidth {|X[1,l]|X[1,l]|X[6,l]|X[1,l]|X[1,l]|} \tabucline[.5pt]{-}
\rowcolor{ForestGreen!40}  & 코드 &	\centering 분 류 & 점수 & 금액 \\ \tabucline[.5pt]{-}
\rowcolor{Yellow!40} 자-356 & & 요실금수술 Operation for Urinary Incontinence & & \\ \tabucline[.5pt]{-}
\rowcolor{Yellow!40} & & 가. 질강을 통한 수술 Transvaginal Approach & & \\ \tabucline[.5pt]{-}
\rowcolor{Yellow!40} & R3564 & (1) 자가근막을 이용한 경우 [근막채취료 포함] & 4,237.80 & 315,290 \\ \tabucline[.5pt]{-}
\rowcolor{Yellow!40} & R3565 & (2) 기타의 경우 Others & 3,477.58 & \myexplfn{3477.58}  \\ \tabucline[.5pt]{-} %258,730 \\ \tabucline[.5pt]{-}
\rowcolor{Yellow!40} & R3562 & 나. 개복에 의한 수술 Abdominal Approach  & 6,618.84 & \myexplfn{6618.84}  \\ \tabucline[.5pt]{-} %492,440  \\ \tabucline[.5pt]{-}
\rowcolor{Yellow!40} & R3563 & 다. 인공물질 또는 자가지방 주입 Foreign Material or Autologous Fat Injection & 3,134.19 & \myexplfn{3134.19}  \\ \tabucline[.5pt]{-} %233,180 \\ \tabucline[.5pt]{-}
\rowcolor{Yellow!40} 자-356-1 & R3566 & 인공요도괄약근 제거술 Removal of Artificial Urethral Sphincter & 4,671.43 & \myexplfn{4671.43}  \\ \tabucline[.5pt]{-} %347,550 \\ \tabucline[.5pt]{-}
\rowcolor{Yellow!40} 자-362 & R3620 & 방광류교정술 Repair of Cystocele & 4,220.25 & \myexplfn{4220.25}  \\ \tabucline[.5pt]{-} %\myexplfn {m}  \\ \tabucline[.5pt]{-} %313,990 \\ \tabucline[.5pt]{-}
\rowcolor{Yellow!40} 자-377 & R3770 & 카룬클절제술 Removal of Urethral Caruncle & 2,053.22 & \myexplfn{2053.22}  \\ \tabucline[.5pt]{-} %152,760 \\ \tabucline[.5pt]{-}
\end{tabu}
\par
\medskip
\subsection{인조테이프를 이용한 요실금수술 관련 세부 적용기준}
인조테이프를 이용한 요실금수술의 구체적 적용기준에 대하여 붙임과 같이 통보하오니 업무에 참고하시기 바랍니다.\par
\begin{center}\emph{붙 임 )}\end{center}
%\begin{enumerate}[1.]\tightlist
\Que{요실금수술 인정기준에 해당되지 않는 경우 수술료 및 치료재료 외의 진료비용에 대하여}
\Ans{‘인조테이프를 이용한 요실금수술 인정기준’ 에 해당되지 않는 경우에는 수술료 및 치료재료 비용 뿐 아니라 입원료, 마취료 등 제반 진료비용 전액은 환자가 부담토록 함. (비급여).}

\Que{요류역학검사(방광내압측정 및 요누출압검사)를 실시하지 않고 요실금 수술을 시행시 급여여부} 
\Ans{현행 인정기준에 해당되지 않으므로 비급여 대상임.}

%\end{enumerate}
\Que{요누출압검사의 수기료} \par
\Ans{나656-1 방광내압측정 항목에 준용산정토록 함.}

\Que{복압성 요실금이 주된 혼합성 요실금에 대하여}
\Ans{종전에 절박성 요실금이 주된 혼합성 요실금으로 진단되었었다 하더라도 수술시점에 \uline{복압성 요실금이 주된 혼합성 요실금으로 진단되었다면 급여대상}임}

\Que{진단과 수술 시기에 차이가 있는 경우 수술의 급여여부}
\Ans{요류역학검사 결과 등으로 복압성 요실금 또는 복압성 요실금이 주된 혼합성 요실금으로 진단되었다면,\uline{ 진단 시기에 관계없이 수술 비용 등은 보험급여함이 타당함.}}

\subsection{현수견인법(Sling Procedure)에 의한 요실금 치료재료(SISTEMA REMEEX 등) 인정기준}
1. 압력 재조절이 가능한 슬링(Sling)을 이식하여 요실금을 조절하는 치료재료인 ARGUS와 SISTEMA REMEEX는 다음의 경우에  요양급여를 인정함. \par
\begin{center}\emph{- 다    음 -}\end{center}
\begin{enumerate}[가.]\tightlist
\item 적응증
	\begin{enumerate}[(1)]\tightlist
	\item  남자의 경우
		\begin{enumerate}[(가)]\tightlist
		\item 전립선적출술로 인해 발생한 요실금
		\item 신경인성 방광으로 인해 발생한 요실금
		\item 외상으로 인해 발생한 요실금
		\end{enumerate}		
	\item 여자의 경우(Sistema Remeex만 해당): 복압성 요실금으로 아래에 해당하는 경우에 인정함.
		\begin{enumerate}[(가)]\tightlist
		\item 첫 수술 실패 후 재수술시
		\item 배뇨근 수축력 약화(detrusor underactivity)가  있는 경우
		\item 진료상 필요성이 있는 심한 요실금의 경우(내인성 요도괄약근 부전의 경우)
		\end{enumerate}
	\end{enumerate}
\item 인정개수 : 1개 인정
\end{enumerate}
2. 상기 1항의 급여대상 이외 사용한 치료재료 비용은 「요양급여비용의 100분의 100 미만의 범위에서 본인부담률을 달리 적용하는 항목 및 부담률의 결정 등에 관한 기준」에 따라 본인부담률을 80\%로 적용함.\par
(고시 제2016-147호, '16.9.1. 시행)\par

ARGUS K5510008(2724180원) \par
SISTEMA REMEEX K5510001(2724180원)\par
\leftrod{요실금수술 치료재료 SISTEMA REMEEX 인정기준에 대하여}
\begin{itemize}[■]\tightlist
\item 심의배경 \newline
요실금 수술 치료재료 SISTEMA REMEEX는 현행 인정기준(고시 제2010-56호, 2010.8.1. 시행)에 의거 요양급여하고 있으나, 인정기준 중 ‘배뇨근 수축력 약화와 심한 요실금’에 대한 구체적 적용 방법에 대한 논란이 있는바 이에 대하여 심의함.
\item 참고 
	\begin{itemize}[○]\tightlist
	\item 건강보험 행위 급여ㆍ비급여 목록 및 급여 상대가치점수(보건복지부 고시 제2010 -123호, 2011.1.1. 시행)
	\item 치료재료 급여ㆍ비급여 목록 및 급여상한금액표(보건복지부 고시 제2010-89호, 2011.1.1. 시행)
	\item 현수견인법에 의한 요실금 치료재료(SISTEMA REMEEX) 등 인정기준(보건복지부 고시 제2010-56호, 2010.8.1. 시행)
	\item 대한배뇨장애 및 요실금학. 배뇨장애와 요실금. 일조각. 2004
	\item 대한비뇨기과학회. 비뇨기과학 제4판. 일조각. 2007
	\end{itemize}
\item 심의내용 \newline
SISTEMA REMEEX는 일반적인 요실금수술용 인조테이프에 비해 수술 후 압력을 쉽게 재조정하여 적절한 배뇨상태를 줄 수 있는 장점은 있으나, 월등히 고가인 점 등을 고려하여 ‘배뇨근 수축력 약화와 심한 요실금’에 대한 구체적 적용 방법은 다음과 같이 적용키로 함.
\end{itemize}
\begin{center}\emph{- 다 음 -}\end{center}
\begin{itemize}[○]\tightlist
\item \textcolor{red}{배뇨근 수축력 약화(detrusor underactivity)가 있는 경우 }
	\begin{itemize}[-]\tightlist
	\item Schafer nomogram에서 배뇨근 수축력 약화 확인 
	\item BCI(Bladder Contractility Index) <100 
	\item 확실한 임상증상(최고 요속(Qmax)이 15ml/sec 이하이고 배뇨후 잔뇨량이 100ml 이상인 경우)이 확인되는 경우
	\end{itemize}
\item \textcolor{red}{심한 요실금} 
	\begin{itemize}[-]\tightlist
	\item \textcolor{red}{Grade 3(서 있는 경우에도 있는 경우)}
	\item \textcolor{red}{요누출압(ALPP) 60cmH2O 이하인 경우}
	\item 최대요도폐쇄압 60cmH2O 이하인 경우
	\end{itemize}
\end{itemize}
[2011.6.13. 진료심사평가위원회]

\subsection{골반장기탈출 교정용 mesh 급여기준} 
골반장기탈출을 교정하기 위하여 사용하는 Seratom Implant와 골반장기탈출증 이식용 메쉬(Preshape Type)는 \textcolor{red}{다음의 경우에 1개만 요양급여를 인정함.}\par
\begin{center}\emph{- 다   음 -}\end{center}
\begin{enumerate}[1.]\tightlist
\item \textcolor{red}{자궁적출술을 시행 받은 후 발생한 질원개탈출증}
\item 골반재건술을 시행 받은 후 재발된 경우
\item 자궁적출술 또는 골반재건술을 처음 시행 받는 경우
	\begin{enumerate}[1)]\tightlist
	\item POP-Q\footnote{Pelvic Organ Prolapse Quantification} 검사상 II기에서 다음의 경우 인정 
		\begin{enumerate}[①]\tightlist
		\item 60세 미만 환자
		\item 65kg 이상의 과체중 환자
 		\item 외측 결손이 의심되는 전 질벽 탈출증(방광류) 환자 
		\end{enumerate}
	\item POP-Q 검사상 III 또는 IV기인 경우
	\end{enumerate}
\end{enumerate}	
( 고시 제2016-190호, 2016.10.1 시행)

\subsection{요실금수술과 방광류교정수술을 동시 시행하는 경우}
\begin{itemize}\tightlist
\item 심의내용 
	\begin{itemize}\tightlist
	\item 동 요양기관은 요실금수술과 방광류교정술을 전건 동시 시행하는 경향으로, 스트레스 요실금, 방광류 상병하에 요실금수술(transobturator adjustable, TOA), 방광류교정술(Cystocele repair(Ant. repair))을 시행하고 자356가(2) 요실금수술-질강을 통한 수술-기타의 경우 1*1*1, 자362 방광류교정술 1*1*1을 청구한 사례임. 
	\item 전반적인 진료내역 검토결과, \textcolor{red}{자356가(2) 요실금수술은 요류역학검사상 수술적 치료가 필요하다고 판단되므로 인정하나, 자362 방광류교정술은 진료기록상 방광류에 대한 객관적인 근거(grade, 의학적 사진, 방광류 조영술 등)가 확인되지 않으므로 인정하지 아니함. }
	\end{itemize}
\item 참고
	\begin{itemize}\tightlist
	\item 배뇨장애와 요실금, 제2판, 29장 여성 요실금의 분류와 역학, 30장 여성 요실금의 진단, 34장 여성 요실금의 수술치료, 38장 골반장기탈출증의 진단
	\end{itemize}
\end{itemize}
 [2015.8.12. 진료심사평가위원회(지역심사평가위원회)]



