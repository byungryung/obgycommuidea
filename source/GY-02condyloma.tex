\section{곤지름(condyloma)}
\myde{}{%
\begin{itemize}\tightlist
\item[\dsjuridical] D260 Flat condyloma : 자궁경부에 생긴것은 자궁경부의 기타 양성 신생물
\item[\dsjuridical] A639 회음부에 생긴 것 : 항문생식기의 사마귀 
\item[\dsmedical] R4305 음부콘딜로마의 수술적 치료 [절제술, 전기소작술, 냉동치료] [\myexplfn{490.47} 원]
\item[\dsmedical] R4306 나. 비수술적치료 [포도필린 등 국소도포 포함] [\myexplfn{248.53} 원]
\end{itemize}
청구메모>> 곤지름 치료는 몇차례라도 청구가 가능하지만, 3번 이상이 되는 경우에는 청구메모에 상세한 이유를 써주는 게 좋습니다.
}%
{STD6종등 다른 성병검사 추가할수 있습니다.\par }
\subsection{곤지름(condyloma)가 다발성일때 여러번 청구 가능한가요?}
안됩니다.\par
외음부를 한장기로 보기에..하나의 시술만 인정됩니다\par
자-430-1가 음부콘딜로마치료법-수술적치료(R4305)는 다발성 또는 거대한 경우를 포함하는 점수이므로 다발성으로 발생된 음부콘딜로마절제술을 실시한 경우에도 소정점수의 1회만 인정함
\subsection{곤지름 약물요법}
알다라도 의외로 효과가 좋습니다. 일단 알다라 쓰고 팔로업해서 호전 없으면 수술하기로 했는데 팔로업해보니 싹 없어져서 수술이 취소되는 경우도 여러번 있었어요. 알다라 크림에 대한 허가사항에 보면 \uline{기존치료로 치료되지 않는 경우만 보험인정이므로 조심해서 써야 할것같습니다. 삭감가능합니다.}\par
넓게 퍼진 질 점막에는 포도필린이 편하고 좋은데요.\par 
TCA로하는데 잘안됩니다. 작은 초기때는 되는데요.
\subsection{알다라크림 보험적용 고시}
Imiquimod 12.5mg 외용제(품명: 알다라크림) \par
허가사항 중 아래와 같은 기준으로 투여 시 요양급여를 인정하며, 동 인정기준  이외에는 약값 전액을 환자가 부담토록 함. \par
- 아 래 -
\begin{itemize}\tightlist
\item[○] 성인의 외부 생식기, 항문주위 사마귀/첨형 콘딜로마의 치료에 기존요법으로 
병변이 재발하여 재치료하는 경우
\end{itemize}
(보건복지부 고시 제2013-127호 시행일: 2013.9.1.)

\subsection{알다라크림 최대 보험가능갯수}
주3회 1회당 1포 최대 16주 이므로 3X16주=48포 까지 가능하겠습니다.
