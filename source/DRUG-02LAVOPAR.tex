\section{라보파 주 LAVOPA INJ }% \label{LavoparInj}
\begin{Ddoing}{라보파약전}  
조성:ritodrine hydrochloride 50㎎/5㎖\\ 
보험 정보:644900700(보) 1,257/5ml/앰플 급여\\
효능/효과\\
단순 조숙산통에 대한 단기간 관리 : 
자궁수축억제제치료에 대한 \highlight{의학적 또는 산과적 금기에 해당되지 않는 임신기간 22주에서 37주 사이의 임부 분만억제.}\\
  
용법/용량\\
자궁수축억제제 사용경험이 있는 산부인과전문의/내과의사만이 이 약을 이용한 치료를 시작해야 하며, 이는 임부 및 태아의 건강상태에 대한 지속적인 모니터링을 실시할 장비를 적절히 갖춘 시설에서 시행해야 한다. 
자궁수축억제제치료의 주효과로써 분만을 48시간까지 지연하는 것을 보여주는 자료에 따라, 치료기간은 48시간을 초과하지 않아야 한다. ; 무작위 대조시험에서 주산기 사망률 또는 이환율에 대하여 통계적으로 유의한 효과가 없는 것으로 관찰되었다. 이러한 단기간 지연은 주산기 건강을 향상시키는 것으로 알려진 다른 방법들에 적용하는데 사용할 수 있다.
이 약은 리토드린 사용에 따른 어떠한 금기사항을 배제하도록 환자 평가를 완료한 후 및 조숙산통으로 진단된 후 가능하면 조기에 투여해야 한다. (사용상의 주의사항 중 2. 다음 환자에게 투여하지 말 것 항 참고) 이는 치료기간 내내 심폐기능감시 및 심전도(ECG)모니터링과 함께 환자의 심혈관상태에 대한 적절한 평가를 포함해야 한다. (사용상의 주의사항 중 5. 일반적 주의 항 참고). 
정맥주입에 대한 특별한 주의사항 : 제한인자인 수축억제, 맥박수 증가, 혈압변화와 관련하여 환자 각각의 용량을 적정해야 한다. 이러한 변수들은 치료 중 주의깊게 모니터링해야 한다. 임부의 최대 심박수는 분당 120 회를 넘지 않아야 한다. 
임부의 폐부종 위험성을 피하기 위하여 수화(Hydration) 정도를 주의깊게 조절하는 것이 필수적이다. (사용상의 주의사항 중 5. 일반적 주의 항 참고) 따라서 약액량을 최소한으로 유지하도록 투여해야 하며, 주입장치(가급적이면 시린지펌프)를 사용해야 한다. \\
\end{Ddoing}

2013년 11월 27일부터 라보파서방캡슐은 판매중지 되고 라보파주는 사용기간을 제한하여 임신 22주에서 37주까지의 임부의 분만억제로 48시간을 초과하지 않아야 한다라고 허가사항이 변경된 것으로 알고 있습니다. 
\begin{hemphsentense}{라보파주의 허가사항변동에 대한 질의}
저희가 궁금한 사항은 주사제를 사용할 경우 48시간을 초과하지 않아야 한다라고 하는데 그러면 
\begin{enumerate}\tightlist
\item 입원시 이틀만 입원치료하면서 급여가 가능한것인지? 궁금합니다. 예를 들어 48시간이라고 제한한다고 하는데 23주된 산모가 1월 1일 밤 11시에 조기진통으로 내원하여 라보파주가 처방이 되었습니다. 그러면 48시간이 끝나는 시간은 1월 3일 밤 11시가 되는데 이 시간이 지나면 산모를 퇴원을 시켜야 하나요? 
\item 아니면 48시간동안 주사제를 사용하고 경과를 지켜 보기 위해 며칠을 더 입원하는것은 급여로 인정되나요? 48시간 동안 라보파주를 사용하고 계속 조기진통이 있을 경우 니페디핀제재의 경구약을 처방해야 되는 산모가 있을 경우 입원일수에 제한이 없는 건가요? 
\item 23주된 산모가 1월1일 오전 10시에 조기진통으로 입원하여 라보파주를 사용하고 컨디션이 좋아져 1월3일 오전10시에 퇴원을 하였습니다. 그런데 당일 저녁 8시에 다시 증상이 있어 내원하여 라보파주를 사용해도 급여로 인정이 되나요? 
\item 지금 현재 라보파서방캡슐이 판매 중지 되어 대체약으로 니페디핀제재의 경구제를 사용할려고 하는데 이것은 산모일 경우 급여로 인정이 되는지도 궁금합니다. 혹여 급여가 안되면 산모들에게 전액본인부담을 시켜도 되는지도 궁금합니다.
\end{enumerate}
30년 이상 조산방지제로 처방되던 약들이 허가사항이 변경되고 판매중지된 상태입니다. 조산기 있는 산모들이 두려워하고 힘들어 하고 있습니다. 빠른 시일내에 안전한 대체약이 나와 산모와 아기의 건강을 지켜 주고 싶습니다. 본원은 조산방지제를 사용했던 산모들로부터 조산방지제의 안전성에 대해 문의 전화를 많이 받고 있습니다. 산모들에게 명확한 답변을 할 수 있도록 심평원측의 빠른 답변 부탁드리겠습니다. 
\end{hemphsentense}

\begin{commentbox}{답변}
해당 안전성 속보 내용을 보면 ``유럽의약품청(EMA) 검토사항: 주사제의 경우 임신22주에서 37주 사이 최대 48시간 동안 조기진통을 억제하는 데에만 사용가능" 및 ``리토드린 주사제의 경우 허가변경 절차에 따라 사전예고 후 변경지시할 계획임을 알려드립니다. (12.16 예정)"로 언급되어 있습니다. 이를 근거로 본다면 12월 16일 이후부터 라보파주는 48시간 동안 사용으로 허가사항이 변경되며 이에 따라 급여인정 범위도 허가사항 범위와 동일하게 정해지게 됩니다.
\begin{enumerate}\tightlist
\item 즉 48시간 이후에 라보파주 투여는 허가사항 범위 초과에 해당하여 급여인정이 불가능하게 된다는 의미이며, 이는 질의하신 48시간 이후에 임신부를 퇴원시키라는 의미는 아닙니다. 
\item 임신부의 입원 치료와 라보파주 급여인정은 별개의 의학적 판단이 필요한 사항이라고 여겨집니다. 입원환자에게 라보파주 투여 이외에도 니페디핀 경구제 투여를 포함한 적절한 다른 처치를 하실 수 있습니다.
\item 라보파주의 재투여 여부는 현재 허가사항 변경이 진행 중이므로 변경내용의 확인이 가능한 이후 허가사항 내용을 확인해 보는 것이 필요하겠습니다. 
\item 니페디핀 경구제는 고시 제2013-127호(`13.9.1)에 의거하여 조기진통에 투여한 경우 급여기준에 적합하면 급여인정이 가능합니다. 아래 관련 고시내용 참고하시기 바랍니다. 
\end{enumerate}
\end{commentbox}