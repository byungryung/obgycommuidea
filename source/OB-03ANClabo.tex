\section{임신초기 검사}
\myde{}{%
\emph{급여 항목}
\begin{itemize}\tightlist
\item[\dsjuridical] Z3400 정상최초임신의 관리, 임신 22주 미만
\item[\dsjuridical] Z3480 기타 정상임신의 관리, 임신 22주 미만
\item[\dschemical] CBC → B1040: RBC / B1050: WBC /  B1060: PLT / B1010: Hb / B1020: HCT
\item[\dschemical] RUA with micro → B0030: RUA / B0043: micro
\item[\dschemical] 혈액형검사 → B2000: ABO / B2021: RH
\item[\dschemical] 매독반응검사(매독혈청검사) → C4602: 매독 반응검사[정밀]
\item[\dschemical] 간염검사 → C4801: HBsAg(B형간염 S항원검사)
\item[\dschemical] 풍진검사 → C4681410:  바이러스항체(일반) IgG-Rubella, C4683416: 바이러스항체(정밀)IgM-Rubella
\item[\dschemical] 에이즈검사 → C4711: HIV항체(일반) 또는 C4712:HIV항체(정밀)
\item[\dschemical] 50g 경구 포도당부하검사(임신 24-28주 사이) → C3711:당검사(정량)+약제
\item[\dsmedical] 초음파 검사
\end{itemize}
\emph{비급여 항목}
\begin{itemize}\tightlist
\item[\dschemical] 갑상선기능검사 → : TSH, FT4
\item[\dschemical] 간염검사 → : HBsAb 
\item[\dschemical] 간기능검사 → : GOT/GPT
\item[\dschemical] Rubella IgG avidity Test
\item[\dschemical] 기타 비타민 (D3) 검사(CY155):25-(OH) Vitamin  D
\item[\dschemical] HAV Ig G등
\end{itemize}
}% 
{
「산전진찰」이란 임신부 및 태아의 건강을 평가하여 위험임신을 선별하는 등의 산전관리를 의미하는 것으로, 산전진찰 목적으로 시행하는 검사의 건강보험 요양급여 인정기준은 다음과 같이 함.\par
                                     \begin{center}\emph{         - 다   음 -}\end{center}
\begin{enumerate}[가.]\tightlist
\item 요양급여대상 검사
	\begin{enumerate}[1)]\tightlist
	\item 혈액학검사 
	\item 요검사
	\item 혈액형검사
	\item 매독반응검사(매독혈청검사)
	\item HBsAg(B형간염 S항원검사)
	\item 모체혈청 선별검사 중 Triple Test 또는 Quad Test(α-FP, Estriol, β-HCG, inhibin-A)
	\item 풍진검사(IgG, IgM)
	\item 에이즈검사
	\item 비자극검사
		\begin{enumerate}[가)]\tightlist
		\item 임신24주 이상 자궁수축이 없는 임부에게 임신기간 중에 입원, 외래 불문하고 1회만 인정하며, 다태임신의 경우에도 1회만 산정함. 다만, 35세 이상 임부에 한하여 1회를 추가로 인정함. 
		\item 가)의 인정횟수를 초과하여 시행한 경우에는 전액 본인부담토록 함.
		\end{enumerate} 
	\item 50g 경구 포도당부하검사
     - 임신 24-28주 사이에 1회만 인정하고, 해당 수기료는 나371나 당검사(정량)으로 산정하며, 부하검사 시 사용된 약제는 별도
       인정함.
	\item 초음파검사 :    “초음파 검사의 급여기준”에 따름
	\item 자궁경부세포진검사 *	
	\end{enumerate}
\item 비급여대상검사
	\begin{enumerate}[1)]\tightlist
	\item 유전학적 양수검사
	\item 위 1) 이외 국민건강보험 요양급여의 기준에 관한 규칙 [별표2] 비급여대상 3호 가목에 의한 건강검진의 범주에 속하는 
     검사항목
     \end{enumerate}
\end{enumerate}
* 2017년 10월 1일부터 자궁경부세포진검사가 비급여에서 급여항목으로 됨.
}%

\par
\medskip
\prezi{\clearpage}
\begin{commentbox}{막달검사와 고위험산모 수술전검사의 보험여부}
%\Que{요양급여대상 검사는 임신기간중 시행횟수는 1회 만 보험인가요? 임신초에 한번시행후 분만전 막달에 상태변화등을 진단하기위해 보험처방가능한지요? 아니면, 나.비급여 대상검사인지요? \par
%2. 외래에서 정상분만 예정자이지만 분만시 발생할 수 있는 응급수술을 대비해 수술 lab 을 보험으로 처방가능한지요? 아니면 나.비급여 대상검사 해당사항인지요?(분만은 대량출혈등 위험한 과정으로 응급수술을 예측할수 있는 상황으로 분만전 검사필요)}
%\Ans{.
“산전진찰 목적으로 시행하는 검사의 요양급여 범위(고시 제2013-36호, ’13.3.1 시행)”에 의하면 요양급여대상 검사를 정하고 있는바, 가.요양급여대상 검사 1)~8)까지의 검사의 경우에는 요양급여대상 검사 종류를 명시하고 있으나 \textcolor{red}{검사별 시행 횟수를 따로 정하고 있지는 않습니다.}\par
또한, 국민건강보험 요양급여의 기준에 관한 규칙 [별표1] 요양급여의 적용기준 및 방법(제5조제1항관련)에 의거, \textcolor{blue}{요양급여는 가입자 등의 연령ㆍ성별ㆍ직업 및 심신상태 등의 특성을 고려하여 진료의 필요가 있다고 인정되는 경우에 정확한 진단을 토대로 하여 환자의 건강증진을 위하여 의학적으로 인정되는 범위 안에서 최적의 방법으로, 경제적으로 비용효과적인 방법으로 행하여야 하며, 각종 검사를 포함한 진단 및 치료행위는 진료상 필요하다고 인정되는 경우}에 한하여야 하며 연구의 목적으로 하여서는 아니된다고 규정하고 있습니다.\par
따라서, 산전진찰 검사 및 분만전 검사 등의 진료는 요양급여의 원칙 및 범위 안에서 이루어져야 하며, \textcolor{red}{임신중 임산부 및 태아의 상태를 고려하여 진료의사의 의학적 판단하에 필요하여 시행한 경우 검사의 타당성 등에 따라 사례별로 심사}가 이루어집니다. 다만, 검사의 타당성이 확인되지 않는 경우에는 심사조정 가능함을 알려드립니다. 감사합니다. 끝.
%}
\end{commentbox}

\par
\medskip
\prezi{\clearpage}
\begin{commentbox}{산전진찰 진찰료 관련} 
\begin{enumerate}[1.]\tightlist
\item 건강보험 요양급여 행위 및 그 상대가치점수 제2부 제1장 기본진찰료[산정지침]에 의거,\textcolor{blue}{초진환자를 진찰하였을 경우는 초진진찰료, 재진환자를 진찰하였을 경우는 재진진찰료를 산정하며, 이 경우 초진환자란 해당 상병으로 동일 의료기관의 동일 진료과목 의사에게 진료받은 경험이 없는 환자를 말하고, 재진환자란 해당 상병으로 동일 의료기관의 동일 진료과목 의사에게 계속해서 진료받고 있는 환자를 말하는 것으로 정의}되어 있습니다.
\item 또한, 해당 상병의 치료가 종결되지 아니하여 계속 내원하는 경우에는 내원간격에 상관없이 재진환자로 보며, 하나의 상병에 대한 진료를 계속중에 다른 상병이 발생하여 동일의사가 동시에 진찰을 한 경우는 재진진찰료를 1회 산정토록 되어 있습니다
\item 아울러, 건강보험에서는 질병, 부상자 뿐만 아니라 모든 요양급여대상자를 진찰한 경우 현행 진찰료 산정지침을 동일하게 적용하여야 하므로 \textcolor{red}{계속적으로 산전진찰을 하는 경우 재진진찰료로 산정하고 산전진찰중 다른 상병이 발생하여 동일의사가 동시에 진찰을 한 경우도 재진진찰료를 1회 산정함이 타당함}을 알려드립니다. 
\end{enumerate}
\end{commentbox}  

\par
\medskip
\prezi{\clearpage}
\Que{타병원서 계류유산진단 받고 수술하러오셨는데 임신초기 피검사원하시는데  급여되는지요?}
\Ans{타병원에서 온경우도 , 계류유산이어도 산전검사 보험 적용됩니다}
\prezi{\clearpage}
\begin{commentbox}{나206가 불규칙항체검사(선별) 인정기준}
나206가 불규칙항체검사(선별)는 \textcolor{red}{수혈이나 임신을 통하여 생성될 수 있는 비예기항체(또는 불규칙항체)를 확인}하는 검사로 다음과 같은 경우에 인정함.
\begin{center}\emph{- 다 음 -}\end{center}
\begin{enumerate}[가.]\tightlist
\item 수혈이 예상되는 환자에게 1회 인정
\item 수혈이 계속되는 환자에게 3일마다 1회 인정
\item 지연성 용혈성 수혈반응이 의심되는 환자
\end{enumerate}
점이 있으므로 수혈이 예상되는 환자에게 수혈여부를 불문하고 치료기간중 1회에 한하여 인정함. 

\begin{itemize}\tightlist  
\item 불규칙항체검사(선별):B2061006 \myexplfn{94.20} 원
\item 불규칙항체검사(동정):B2062006 \myexplfn{221.13} 원
\end{itemize}

\end{commentbox}

