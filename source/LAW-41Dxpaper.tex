\section{진단서}
\subsection{각 상병별 치료기간 from 대한의사협회}\index{진단서!치료기간}
\tabulinesep =_2mm^2mm
\begin {tabu} to\linewidth {|X[1,l]|X[6,l]|X[1,l]|X[1,l]|} \tabucline[.5pt]{-}
\rowcolor{ForestGreen!40} \centering 부 위 &  \centering 상 병 명 & 코드 & 기간\footnote{ 완전히 치유될 때까지를 말하며 나이, 합병증, 기타 의학적 상태에 따라 달라 질 수 있다.}  \\ \tabucline[.5pt]{-}
\rowcolor{Yellow!40} 외음부 & 처녀막 파열 & N89.8 & 1  \\ \tabucline[.5pt]{2-4}
\rowcolor{Yellow!40} & 얕은 외음부 찢긴 상처 & S31.4  & 1-2  \\ \tabucline[.5pt]{2-4}
\rowcolor{Yellow!40} & 얕은 외음부 찰과상 & S30.2 & 1-2  \\ \tabucline[.5pt]{2-4}
\rowcolor{Yellow!40} & 외음부 혈종 & N90.8 & 2-3  \\ \tabucline[.5pt]{2-4}
\rowcolor{Yellow!40} & 큰어귀(바르톨린) 샘의 낭포 & N75.0 & 2  \\ \tabucline[.5pt]{-}
\rowcolor{Yellow!40} 질 & 질벽 파열 & S31.4 & 2-3  \\ \tabucline[.5pt]{2-4}
\rowcolor{Yellow!40} & 곧창자질사이막 찢긴상처(직장 질 사이막 열상) & S31.8 & 4  \\ \tabucline[.5pt]{-}
\rowcolor{Yellow!40} 자궁 & 자궁파열 & S37.6 & 6  \\ \tabucline[.5pt]{2-4}
\rowcolor{Yellow!40} & 자궁근종 & D25.9 & 4  \\ \tabucline[.5pt]{-}
\rowcolor{Yellow!40} 자궁부속 & 자궁외 임신 & O009 & 2-4  \\ \tabucline[.5pt]{2-4}
\rowcolor{Yellow!40} 기관 & 양성난소덩어리(양성난소종괴) & N83.2  & 2-4  \\ \tabucline[.5pt]{-}
\rowcolor{Yellow!40} 골반내 & 자궁내막증 & N80.1 & 6개월 이상  \\ \tabucline[.5pt]{2-4}
\rowcolor{Yellow!40} & 골반내 염증 & N73.9 & 2-4  \\ \tabucline[.5pt]{-}
\rowcolor{Yellow!40} 산 과 & 자연분만 & O80.9 & 6  \\ \tabucline[.5pt]{2-4}
\rowcolor{Yellow!40} & 제왕절개 & O82 & 6  \\ \tabucline[.5pt]{2-4}
\rowcolor{Yellow!40} & 태아사망 & O36.4 & 4-6  \\ \tabucline[.5pt]{2-4}
\rowcolor{Yellow!40} & 절박유산 & O20.0 & 2-4  \\ \tabucline[.5pt]{-}
\end{tabu}
