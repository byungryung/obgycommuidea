\section{보조생식술을 위해 초음파를 시행하는 경우} 
\begin{enumerate}[1)]\tightlist
\item 보조생식술 진료시작일에 자궁부속기 및 자궁내막의 상태  등을 보는 경우 나 944 라 (1)  여성생식기 초음파 ( 일반 ) 를 산정함 
\item 보조생식술 관련 약제투여 후 난포의 크기 및 수 ,  자궁내막 두께 등을 관찰하는 경우 나 940 나 단순초음파 ( Ⅱ ) 를 산정함 
\end{enumerate}

\Que{보조생식술 시술을 위해 과배란유도제를 처방받는 날 시행한 초음파는 급여가 적용되나요?}
\Ans{급여 적용됩니다. 보조생식술 진료 시작일에 시행한 초음파는 ‘복부-여성생식기 초음파 일반(EB455)’으로 산정하며, 이후 난포의 크기, 자궁내막 상태 등을 추적 관찰하기 위해 난자채취 전까지 시행하는 초음파도 ‘단순초음파(Ⅱ)(EB402)’를 산정하여 급여 적용합니다. }

\Que{자연주기를 이용한 보조생식술 시술을 위해 생리시작 후 내원하여 초음파를 시행한 경우에도 급여가 적용되나요?}
\Ans{약제 투여 여부와 관계없이 생리시작 후 내원하여 시행한 초음파는 ‘복부-여성생식기 초음파 일반(EB455)’으로 산정하고, 이후 난포의 크기, 자궁내막 상태 등을 추적 관찰하기 위해 난자채취 전까지 시행하는 초음파는 ‘단순초음파(Ⅱ)(EB402)’를 산정하여 급여 적용합니다.}

\Que{난자채취시 초음파를 시행한 경우 초음파 비용은 별도 산정이 가능한가요?}
\Ans{난자채취나 배아이식시 시행한 유도초음파의 경우 해당 시술비용에 포함되어 있으므로 별도 산정이 불가합니다. }

\Que{자연임신을 시도하기 위해 과배란유도제를 투여받는 경우에도 약제 및 초음파가 급여 적용 되나요?}
\Ans{보조생식술을 시술받는 경우에만 급여가 적용됩니다. }

\Que{단순초음파를 시행한 경우 특정내역 JS013을 기재해야 하나요?}
\Ans{과배란유도제 투여 등 후 난포의 크기, 자궁내막 상태 등을 추적 관찰하기 위해 단순초음파(Ⅱ)(EB402)를 시행하는 경우 JS013(단순․유도초음파 세부내역)을 기재해야 합니다.} 
\emph{[예]}
	\begin{itemize}\tightlist
	\item 해부학적 구분코드/수가코드(5단코드)*/구체적 사유
    \item * 수가코드(5단코드)는 단순․유도초음파를 시행하게 된 관련 행위코드(검사, 처치 및 수술료 등)를 기재 
󰋮 	\item 기재형식: X(1)/X(5)/X(200)
󰋮 	\item (예시) JS013 I/ /보조생식술
	\item * 초음파 시행 당일 관련 행위 없으므로 수가코드(5단코드)는 기재 생략
	\end{itemize}