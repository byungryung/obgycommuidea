\section{감기 및 알러지성 비염}
\myde{}{
\begin{itemize}\tightlist
\item[\dsjuridical] J069 상세불명의 급성 상기도감염
\item[\dsjuridical] J304	상세불명의 앨러지비염
\item[\dsjuridical] J209	상세불명의 급성 기관지염
\item[\dsjuridical] R05 기침
\item[\dsjuridical] E860	탈수
\item[\dsjuridical] M6260 근육긴장, 여러 부위
%\item[\dsmedical] 
%\item[\dsmedical] 
\end{itemize}
}{
경구진해거담제에 대한 심평원 고시 \par
\emph{진해거담제에 해당되는 약제(복지부분류 222) : 거담제, 기침 억제제, 기관지확장제, 비충혈 제거제}\par
\uline{진해거담제에 해당되지 않는 약제 :  항히스타민제(141번)\footnote{2세대 항히스타민제 약전에 감기에 대한 허가사항이 없다고 합니다.고로 삭감가능성이 있으므로 처방시에 알러지성비염 상병을 넣어주세요. 1세대 항히스타민의 경우는 관계없음.}, 해열제(114번)}
\begin{enumerate}\tightlist
\item 상기도 질환에선 2종 이내에서 처방가능
\item 상기도 질환이외의 호흡기 질환 : 3종 이내에서 처방가능
\item 천식,COPD : 제한없음
\end{enumerate}
}

\subsection{흔히 쓰는 진해거담제들}
\begin{itemize}\tightlist
\item Antitussive
	\begin{itemize}\tightlist
	\item 시네츄라 45ml \# 3
	\item 코프렐 6T \# 3 (비급여): 비교적 싼약이며, 환자들의 순응도고 좋고, 또한 심평원고시를 피해갈수 있다.
	\item 애니코프 2T \# 2 
	\end{itemize}
\item Mucolytics
	\begin{itemize}\tightlist
	\item 리나치올 (carboxy cysteine) 500mg 3T \# 3 : 부작용은 거의 없고 75kg이상인 환자는 6T를 쓰는게 좋다.
	\item 뮤테란 (n-acetyl cysteine) 3T \# 3  : 거담효과는 적지만, 심한 통증등으로 인해서 많은양의 AAP를 내야 하는 경우에 독성제거 효과가 있다고 하심.
	\item 바리다제, 뮤코라제 3T \# 3 
	\item 셀포라제 2T \# 2
	\item 엘도스 
	\end{itemize}
\end{itemize}

\subsection{콧물을 주증상으로 온 환자 감기(급성비인두염) OR 알러지성비염}
\tabulinesep =_2mm^2mm
\begin{tabu} to\linewidth {|X[1,l]|X[1,l]|X[1,l]|} \tabucline[.5pt]{-}
\rowcolor{ForestGreen!40}    & 감기 & 알러지성 비염 \\ \tabucline[.5pt]{-}
\rowcolor{Yellow!40} 발열 오한 & + or - & - \\ \tabucline[.5pt]{-}
\rowcolor{Yellow!40} 통증 (인후통 근육통 두통)& + or - & - \\ \tabucline[.5pt]{-}
\rowcolor{Yellow!40} 기침 & + or - & + or -  \\ \tabucline[.5pt]{-}
\rowcolor{Yellow!40} 가래 & + or - & + or - \\ \tabucline[.5pt]{-}
\rowcolor{Yellow!40} 코막힘 & - & + or - \\ \tabucline[.5pt]{-}
\rowcolor{Yellow!40} 재채기 & - & + or - \\ \tabucline[.5pt]{-}
\rowcolor{Yellow!40} 증상 지속 기간 & 대부분 7일이내 저절로 좋아짐. & 증상이 지속되는 경향이 있으며 자주 반복됨. \\ \tabucline[.5pt]{-}
\rowcolor{Yellow!40} Pharyngeal injection & ++ or +  & + or - \\ \tabucline[.5pt]{-}
\end{tabu}
\begin{mdframed}[linecolor=blue,middlelinewidth=2]
\emph{심한 콧물을 주소로 온 경우}\par
1 세대 항히스타민제 + 2세대 항히스타민제 or
2 세대 항히스타민제 + 2세대 항히스타민제 or
 + 재진시 부터는 antileukotrients(씽크레어등)
\end{mdframed}
\paragraph{coughing}
감기기침의 기전은 염증으로 예민해져 있는 인후 부위의 기침 반사수용체를 \uline{코 뒤로 넘어가는 콧물이 자극해서 발생한다.} (후비루 현상) \par
고로 원인이 되는 \uline{콧물분비를 억제하는 약 : 항히스타민제가 주가 되어야 한다.} Antitussive 는 보조적으로 사용 될 수있다.\par
효과 : 항히스타민제 >>비마약성진해제
\paragraph{Sputum}
감기 가래의 기전은 \uline{주로 콧물이 목으로 넘어가는 것을 환자들이 가래라고 착각하는 것이다.} 이때 목이 간질간질하면서 가래가 끼는 느낌을 호소하고 때로는 밖으로 나오기도 한다. 가래가 먼저 생기고 이로 인해 기침이 발생하는 것 같다고 말함. \par
만약 기침이 먼저 발생하고 기침과 동시에 가래가 튀어나온다고 호소하면 이는 기관지에서 발생하는 진짜 가래일 가능성이 높으며 기관지염이나 폐렴을 의심하여야 함.\par
가래가 매우 끈적거리고 색깔이 진하다면 대부분 부비동염이 합병되었을 가능성이 높으며, 때론 기관지염과 폐렴을 의심해야 함. 부비동염이 의심된다고 항생제를 사용할 필요는 없음.\par
치료 : 콧물을 억제하는 항히스타민제를 먼저 베이스로 깔고 끈적하고 진한 가래라면 거담제를 병행함. 가래가 투명하면 거담제 필요 없음.

\paragraph{Rhinorrhea}
\uline{감기로 인한 콧물은 대게 양이 많지 않고 투명함. }\par
\uline{양이 많거나 코막힘이나 재채기가 동반되어 있다면 환자가 원래 알러지성비염을 가지고 있었을 가능성이 매우 높으며} 감기로 인해 기존의 비염이 악화된 경우임.\par

발열,오한, 통증(인후통, 두통, 근육통),불쾌감등의 infection의 sign \uline{없이 단순히 콧물과 기침. 또는 콧물이나 기침 한가지만 호소하는 경우}단순비염 증상일 경우가 많음.\par
경우의 수 : 
\begin{enumerate}\tightlist
\item 순수 감기 
\item 순수 비염 (또는 비부비동염)
\item 감기+비염 (또는 비부비동염)
\item 기타 질환 
\end{enumerate}
치료 : 1번 2번 3번 모두 결론은 항히스타민제 (비염의 경우에는 항히스타민제의 강도가 올라가야만 컨트롤 됨)


\paragraph{Nasal Stuffness}
\uline{알러지성 비염이 없이는 감기로 코막힘의 증상이 나오기 어렵다.}\par
따라서 코막힘의 증상이 있다면 비염이 있다고 생각하고 가끔 재채기 한적 있냐고 물어 보세요. 재채기는 대게 비염이 있는 사람만 하는 것입니다.\par
\begin{mdframed}[linecolor=blue,middlelinewidth=2]
처방 : 항히스타민제 + (재진시 류코트린엔 억제제 추가)\par 
비 충혈제거제는(슈다페드, 복합제:액티피드 등) 처방하지 마시기 바랍니다. 
\end{mdframed}

\subsection{Nasal decongestant의 부작용}
\begin{enumerate}\tightlist
\item 6세 미만에선 쓰지말자
\item 장기적으로 부작용이 상당하다.
\item 감기증상에 ephedrine을 처방하지 맙시다.
\item 단독제재 : 슈다페드, 슈다펜
\item 일반약중 ephedrine을 함유한 약 : 엑티피트, 알레그라디, 리노에바스텔 (항히스타민 + ephedrine)
\item 종합기침억제재중 ephedrine을 함유한 약 : 코푸시럽, 코데닝, 코데농, 코뚜시럽등등 (비마약성진해제+항히스타민+ephedrine) 
\end{enumerate}

%\subsection{삭감 조심해야할 약물 - 항히스타민 제제들}

\subsection{항히스타민제}
1세대 항히스타민제재는 감기상병으로 처방가능하지만, 2세대부터는 꼭 알러지성 비염 코드가 있어야 삭감이 안됨. \par
계절성 및 다년성 알러지성 비염, 알러지성 결막염, 만성 특발성 두드러기, 피부 소양증의 병명이 들어 가야 합니다. \uline{감기에는 반드시 삭감입니다.\footnote{1세대 항히스타민제재는 삭감없고, 2세대 항히스타민제재는 삭감됨}}
\par
\medskip

\tabulinesep =_2mm^2mm
\begin {tabu} to\linewidth {|X[2,l]|X[2,l]|X[2,l]|} \tabucline[.5pt]{-}
\rowcolor{ForestGreen!40} & \centering 성분명 & \centering 상품명(하루용량) \\ \tabucline[.5pt]{-}
\rowcolor{Yellow!40}  1 세대 & piprinhydrinate & 푸라콩(3) \\ \tabucline[.5pt]{2-3}
\rowcolor{Yellow!40}  부작용: 졸림,구강건조 & \emph{Pheniramin} & \emph{페니라민(3-6)} \\ \tabucline[.5pt]{-}
\rowcolor{Yellow!40} 2세대 & Ketotifen & 자디텐(2) \\ \tabucline[.5pt]{2-3}
\rowcolor{Yellow!40} 알러지성 비염코드 J304 &  Cetrizine & 지르텍 \\ \tabucline[.5pt]{2-3}
\rowcolor{Yellow!40} & Mizolastine & 미졸렌 \\ \tabucline[.5pt]{2-3}
\rowcolor{Yellow!40} & Loratadine & 클라리틴 \\ \tabucline[.5pt]{2-3}
\rowcolor{Yellow!40} & Azelastine & 아젭틴(2) \\ \tabucline[.5pt]{2-3}
\rowcolor{Yellow!40} & Fexofedine & 알레그라 \\ \tabucline[.5pt]{2-3}
\rowcolor{Yellow!40} & Ebastine & 에바스텔 \\ \tabucline[.5pt]{2-3}
\rowcolor{Yellow!40} & Bepotatatine & 타리온(2) \\ \tabucline[.5pt]{2-3}
\rowcolor{Yellow!40} & Levocetrizine & 씨잘 \\ \tabucline[.5pt]{2-3}
\rowcolor{Yellow!40} & Desloratidine & 에리우스 \\ \tabucline[.5pt]{-}
\end{tabu}

\subsection{발열 오한 인후통 근육통 두통 전신불쾌감}
\tabulinesep =_2mm^2mm
\begin{tabu} to\linewidth {|X[1,l]|X[4,l]|X[4,l]|} \tabucline[.5pt]{-}
\rowcolor{ForestGreen!40} 경구제 & AAP & NSAID	\\ \tabucline[.5pt]{-}
\rowcolor{Yellow!40} 용량,종류 & 타이레놀 325 mg 3T \#3 \newline 타이레놀이알650 mg  3T \#3(가장많이) \newline 타이레놀 325 mg 9T \#3 \newline 타이레놀이알 650 mg 6T\footnote{체중이 많이 나가고, 너무 많이 힘들어 하면 씀. 부작용이 걱정되면 뮤테란등을 같이 쓰면 좋겠다고 하심.} \#3 & Loxprofen 3T \#3 \newline
Ibuprofen \newline 
캐롤에프(비급여) \\ \tabucline[.5pt]{-}
\rowcolor{Yellow!40} 장점 & 다양한 용량으로 조절.
부작용이 적다. & 소염작용 \\ \tabucline[.5pt]{-}
\rowcolor{Yellow!40} 단점 & 간 독성 주의 \newline
  : 간대사 장애 환자. 음주 환자 & 위장장해 \newline 신기능 저하 우려 \newline  : 노인 환자 만성병 환자 주의 \newline 혈액순환 장해 (혈관 수축) 
   : 심혈관질환 주의 \newline
알러지 질환 악화.  \\ \tabucline[.5pt]{-}
\end{tabu}
\medskip

\tabulinesep =_2mm^2mm
\begin{tabu} to\linewidth {|X[1,l]|X[4,l]|X[4,l]|X[4,l]} \tabucline[.5pt]{-}
\rowcolor{ForestGreen!40} 주사제 & Tramadol (트리돌) & Propacetamol(데노간) & Diclofenac (발렌탁) \\ \tabucline[.5pt]{-}
\rowcolor{Yellow!40} 용법 & 근주 & 정맥수액에 혼합 & 근주 \\ \tabucline[.5pt]{-}
\rowcolor{Yellow!40} 용량 & 2/3 - 1 ample & 2/3 -1 ample & 2/3-1 ample \\ \tabucline[.5pt]{-}
\rowcolor{Yellow!40} 효과 & 진통 & 진통\bullet  해열 & 진통\bullet  해열\bullet  소염 \\ \tabucline[.5pt]{-}
\rowcolor{Yellow!40} 부작용 & 메스꺼움, 구토 & 경구제(AAP)와 동일 & 경구제(NSAID)와 동일 \newline 주사부위 통증  \\ \tabucline[.5pt]{-}
\end{tabu}


\begin{mdframed}[linecolor=cyan,middlelinewidth=2]
\begin{itemize}\tightlist
\item 기침의 원인이 코에 있는지 기관지에 있는지 구분하여 한 쪽약만 몰아서 써야한다. 그래야 환자도 잘 치료 되고 기침을 치료 하는 실력이 는다. – J 원장님.
\item 알러지를 알아야 감기를 잘 치료 할 수 있다 –S 원장님.
\end{itemize}
\end{mdframed}
