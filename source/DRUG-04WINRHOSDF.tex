\section{윈로에스디에프 주 1500IU WINRHO SDF INJ 1500IU }	\label{WinRhoInj}
\begin{Adoing}{약전}
\begin{itemize}\tightlist
\item 구분:전문(희귀의약품)
\item 제조사:Cangene Corp.	
\item 판매사:정인약품 Jung In	
\item 생산발매상황
\item 조성:immunoglobulin anti-D(Rh) 1500IU 
\item 보험 정보:655300090(보) 136,870/3ml/관 급여(2010-01-01)
\item 효능/효과
	\begin{enumerate}\tightlist
	\item 모체와 태아 또는 신생아와의 혈액형이 각각 다음과 같거나 태아 또는 신생아의 Rh인자가 확인되지 않는 경우 또는 확인할 수 없는 경우 모체의 D(Rho)항원에 대한 감작의 예방을 위한 투여. 
	\item 특발성 혈소판 감소 자반증의 치료. 
		\begin{itemize}\tightlist
		\item 소아의 급성 또는 만성 특발성 혈소판 감소 자반증. 
		\item 성인의 만성 특발성 혈소판 감소 자반증. 
		\item HIV 감염에 의한 성인 또는 소아의 이차적인 특발성 혈소판 감소 자반증.
		\end{itemize}
	\end{enumerate}
\end{itemize}
\end{Adoing}

\begin{Cdoing}{Immunoglobulin anti-D(Rh) 600, 1500, 5000IU 주사제 (품명: 윈로에스디에프주) 급여기준}
1. Rho(-)형 산모에게 투여시는 허가사항 범위내에서 인정함.\\ 
2. 특발성 혈소판 감소 자반의 치료(소아의 급성 또는 만성 특발성 혈소판 감소 자반증, 성인의 만성특발성 혈소판 감소 자반증, HIV 감염에 의한 성인 또는 소아의 이차적인 특발성 혈소판 감소 자반증)에 아래와 같은 기준으로 투여하는 경우에는 요양급여를 인정하며, 동 상병에 대하여 허가사항 범위이지만 동 인정기준 이외에 투여한 경우에는 약값 전액을 환자가 부담토록 함.\\ 
- 아 래 -
\begin{enumerate}[(1)]\tightlist
\item 성인 
	\begin{enumerate}[(가)]\tightlist
	\item 적응증: 만성 ITP 환자 중 아래의 1항목에 해당하는 경우 인정함 
		\begin{itemize}\tightlist
		\item 심한 혈소판감소증(20,000mm3/이하) 
		\item 중증의 출혈이 있을 때(ex:중추신경계질환, 위장관출혈 등) 
		\item 응급수술을 요하는 경우 
		\item 비장적출수술의 전처치 
		\item 임신 30주 이상의 임산부가 분만에 대비하고자 할 때 
		\item I.V. globulin으로 치료가 있었던 환자에서 증상 재발시 
		\end{itemize}
	\item 용법ㆍ용량 및 인정기간: 허가사항 범위내에서 최대 2일간 투여
  	\end{enumerate}
\item 소아(16세미만 소아에 적용) 
	\begin{enumerate}[(가)]\tightlist
	\item 적응증: 아래의 1항목에 해당하는 경우 인정함 
		\begin{itemize}\tightlist
		\item 급성 ITP에서 출혈경향이 심하거나 중추신경계의 출혈의 위험성이 있을 때, 또한 심한 혈소판 감소증(20,000/mm3이하)이 있을 때 
		\item 급ㆍ만성 ITP 환아가 외과적 수술을 요할 때 
		\item 급ㆍ만성 ITP 환아가 심한 외상을 입었을 때 
		\item Steroid제제를 2-4주간 투여한 후에도 효과가 없을 때 
		\item ITP 환아가 감염이 합병되었을 때 
		\item 5세 이하의 어린이가 만성 ITP로 심한 혈소판 감소증(80,000mm3/이하)으로 비장적출을 연기하고자 할 때 
		\item 비장적출이 요구될 때 
		\item 경태반 모체 항체로 인한 급성 혈소판 감소증인 신생아 
		\end{itemize}
	\item 용법ㆍ용량 및 인정기간: 허가사항 범위내에서 최대 2일간 투여 
	\item [※주:만성 ITP란 발병 6개월내에 관해(Remission)가 나타나지 않는 경우를 말함].
	\end{enumerate}
\end{enumerate}
\end{Cdoing}