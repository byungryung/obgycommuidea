\section{우리나라 진찰료}
\begin{itemize}\tightlist
\item 새롭게 추가되는 행위임에도 
\item 진찰료 상대가치 변화 없이 
\item 진찰료 포함으로 쉽게 결정이 되고 있음
\item 진찰료를 재분류하면서
\item 기존 진찰료에 대한 상대가치의 변화가 없었음
	\begin{itemize}\tightlist
	\item 재진의 50\%
	\item 야간 가산 기준 시간 변경
	\item 초 재진 기준 변경
	\item 차등수가
	\item 만성질환(고혈압, 당뇨병)자의 평생 재진
	\end{itemize}
\end{itemize}

\subsection{진찰료의 보상 수준}
최근 의학의 눈부신 발전으로 인해 새로운 또는 고도의 발달된 의료기술이 의료행위에 미치는 절대적 영향력은 어느 누구도 부인할 수 없지만 진찰이 진료행위(의료행위)의 기본적인 첫 출발임과 동시에 가장 중요한 핵심요소임 역시 아무도 부정하지 못한다. \par

즉 진찰은 의사의 무형적인 사고, 판단, 선택을 절대적으로 필요로 하기 때문에 의사만이 할 수 있는 전문적 고유 업무이다. 
진찰이 갖는 의의가 이렇게 의미심장함에도 불구하고 실제 임상진료현장에서는 진찰의 중요성이 과소평가되고 있으며, 건강보험 측면에서도 진찰료 또는 기본진료료 항목은 항상 평가 절하되어 있다. \PAR \medskip

\emph{진찰료 포함 목록}\par
\begin{itemize}\tightlist
\item 전정 재활 운동치료(Vestibular rehabilitation therapy)
   (전정기능장애시 회복이 촉진되도록 환자에게 재활치료의 목표와 방법을 자세히 설명하고 선택된 방법(눈운동, 걷기운동 등)을 설명, 교육하고 연습시킴) 
\item 입원 또는 외래 환자 (가족 포함)에게 제공되는 각종 교육 및 상담 
\item Video 녹화료(각종 검사, 수술 등의 내용을 녹화한 후 환자 또는 보호자가 타 병원으로 가져가는 목적 등의 이유로 요구시 비디오 복사 제공, 테이프 비용만 실비) 
\item 사시 기능 훈련, 약시 기능 훈련(차안법 및 시기능훈련) 
\item 영양치료팀 자문료 
\item PAD test (요실금 정도 등을 파악하기 위해 운동 전후의 패드 무게를 비교) 
\item M.S.E[Mental Status exam] 
\item 건강위험평가 Health Risk Assessment 
\item 정밀 체성분 측정검사 [impedance법] 
\item 동맥경화 측정기*
\item "교육상담료"의 "주" 및 "별표1"에 의거 기본진료료의 소정점수에 포함됨.(산전후교육)
\item 여성갱년기 질환의  만성질환관리료 [조정신청] :	가14 만성질환관리료 주2. 대상환자에 한하여 산정함.
\item 자궁경관 및 질 검체 채취료[조정신청]:	나592 자궁질도말세포병리검사의 소정점수에 포함됨.
\end{itemize}

\subsection{상대가치 전면개정 작업}
고시가를 환산지수로 나누어 도입한 상대가치점수에 대한 불만이 지속된 상태에서 자원 기준 상대가치 연구에 대한 공감대 형성으로 우리나라도 2003년 대한의사협회의 상대가치연구단과 건강보험심사평가원의 상대가치연구개발단이 공동으로 상대가치 수가의 첫 번째 전면 개정 작업이 시작되었다. 

1) 1차 상대가치 전면개정 작업 대한의사협회 산하 상대가치 개정위원회에서 주도적으로 4,941개 의과 의료행위(비급여 608개, 신설 64개 행위 포함)에 대해서 의료행위 정의와 의사 업무량 산출 작업을 하였다. 진료비용은 건강보험심사평가원의 상대가치 연구개발단에서 수행하였으며 18개 전문과 진료비용 패널(의사, 간호사 및 임상인력 협회 추천 230여명)을 만들어 수행하였다. 동시에 top down valuation을 위해 표본기관 비용상세조사(원가 중심점별 하향식 방법)와 기관단위 비용조사(전국단위 하향식 방법)를 수행하였다. 

2) 2차 상대가치 전면개정 작업 2차 개정작업은 1차와 유사하게 작업을 진행하되, 1차와는 다르게 과별 총점 고정이던 것을 유형별 즉 수술, 처치, 기능, 영상, 검체검사 5가지 유형으로 검증하여 유형별 총점 고정 하에서 과 간의 벽을 허무는 작업을 수행하였다. 의사업무량은 의사협회 산하 상대가치개정위원회에서 수행하였고, 진료비용은 건강보험심사평가원 상대가치개발단에서 수행하였다. 데이터 구축 이후 현재 최종 마무리 작업 중에 있다.

3) 연구 문제점 230여명의 전문가들이 참가하여 진행한 전면개정 연구로, 참가한 사람들이 동일한 기준을 가지고 적용해야 했지만, 사전에 연구에 대한 충분한 동의가 없어 연구 참여자간에 다른 기준을 가지고 자료가 구축되었다는 점이 가장 큰 문제점이라고 할 수 있다. 즉, 대학병원의 이상적 상황을 고려해서 자료를 구축하는 것이 원칙이었으나 참여자들이 이를 충분히 공감하지 못했고, 어떤 패널은 대학병원의 이상적 상황을 고려해서, 어느 패널은 현실을 반영, 또 다른 패널은 의원의 현실을 반영해서 데이터를 구축하는 일이 나타났다. 또한 benchmark를 이용하는데 있어서도 서로 다른 생각을 가지고 이용하였다. 이 결과 진료비용 자료의 경우 회계 조사결과와 대비해 보면 적게는 1.5배, 많게는 10배까지 자료가 구축되었고, 이로 인해 결과적으로 과별 불균형이 오히려 심화되었다(강길원 외, 2007).


 1차 연구에서 지적되었지만, 2차 때도 1차 때의 오류를 반복하였다. 유형별로 나누어 검증하였다는 점에서는 1차 보다는 나은 방법이지만, 1차 때의 문제점은 여전히 개선되지 못하였다. 회계조사에서도 대표성 문제가 제기되었다. top down valuation을 하기 위한 회계조사도 병원계의 충분한 협조가 부족해서 결국 협조 잘해주는 병원에서 회계조사 연구를 수행해야 했다. 따라서 현장 조사 자료나 회계 조사 자료 모두 대표성을 확보하기 어려웠다. 

\subsection{상대가치 제도 운영의 개선 방안}
1. 의사협회가 주도적으로 할 수 있도록 공식화해야 한다.
미국은 행위의 등재와 개정 및 삭제 그리고 의사업무량과 진료비용 점수 모두 미국의사협회에서 주도적으로 하고 있다는 점에서 우리보다 한발 앞서 있다고 볼 수 있다. 발전하는 의료행위의 등재와 개정, 그리고 삭제를 원활하게 하기 위해서는 전문가 단체의 참여가 절대적이며 또한 상대가치점수 산정에서도 마찬가지다. 의료행위의 변화는 의료기술 발전과 같이 빠르게 적용되어야 한다. 이것은 단순히 코드를 부여하는 것이 아닌 현재 우리나라 의료의 현 상황을 파악해서 미래를 준비하게 하는데도 중요한 일이다. 상대가치점수에서 의사 업무량 상대가치는 제공되는 의료 서비스에 투입되는 의사의 시간과 강도에 의해서 결정이 된다. 그리고 이것에 대해서는 의료서비스를 제공하는 의사들이 가장 잘 알 수밖에 없는 부분이다. 진료비용 역시 서비스를 제공하는 현장에 있는 의사들이 가장 잘 알고 있는 부분이다.따라서 이러한 부분에 대해서는 대한의사협회가 주도적으로 할 수 있도록 해야 한다. 그러면서 건강보험심사평가원 상대가치개발팀이 이 모임에 참가 하여 점수 산정 과정의 정당성과 결과 산출의 합리성을 검증하면 된다. 이를 위해서 의사협회 산하에 있는 상대가치개정위원회와 의료행위심의위원회를 공식적으로 인정하고 제도적, 재정적 뒷받침을 해야 한다.

2. 충분한 시간을 투자해야 한다.
많은 참가자들이 참여하는 만큼 컨센서스(consensus) 형성에 많은 시간을 투자를 해야 한다. 먼저 연구 참여자를 선정해서 충분히 교육해야 한다. 상대가치 제도에 대한 이해와 상대가치점수 산정에 대한 이해 그리고 타과를 존중할 수 있도록 하는 등 참가자들이 충분히 제도에 대한 컨센서스 형성을 한 다음에 연구를 진행해야 한다.

3. 원가 보존을 해야 한다.
무엇보다 원가보존이 가능하도록 지금의 원가 이하의 수가를 개선해야 한다. 상대가치는 행위간의 상대가치를 비교해서 만들어지는 점수인데, 환산지수를 곱하면 바로 수가가 나온다는 점에서 모든 행위가 원가보다 수가가 적다면 아무리 상대가치점수를 제대로 산정하려고 해도 머릿속에서 수가가 바로 계산이 되면서 제대로 된 점수를 산정할 수 없게 만들고 있다.이러한 점은 의료의 왜곡을 만든다는 점에서 가장 먼저 개선해야 할 부분이라고 생각하며, 이번 2차 개정 연구에서 과 간의 벽을 허물고 상대가치점수를 산정한다는 점에서 충분한 재정 투입으로 제대로 된 상대가치 연구 점수 산정의 토대가 되었으면 한다. 요즘은 병원에서 ABC 원가 분석 작업을 하고 있으며, 이를 이용해서 의료행위에 대한 상대가치점수와 적정 환산지수를 알아낼 수 있다. 이 역시 의료계와의 긴밀한 협조 속에서 제대로 된 sampling 작업을 거친다면 대표성을 얻을 수 있다고 생각한다. 

결론
상대가치 제도 운용에 대해서 공급자인 의료계와 보험자, 심평원에서 각자의 중요한 역할을 담당해야하며 함께 논의해야 한다. 하지만 전문가 행위인 의료행위의 등재, 개정, 삭제 및 의사업무량 산정과 진료비용 데이터 구축은 전문가 단체가 주도적으로 해야 한다고 생각한다. 대한의사협회 산하 의료행위심의위원회와 상대가치개정위원회는 14년 동안 2번의 전면개정 작업에 적극 참여하였으며, 26개 전문과의 이해가 걸린 상당히 많은 문제를 합리적이고 공정한 방법으로 원만한 합의를 통해서 결정한 경험을 가지고 있다. 현재는 상대가치개정위원회에 건강보험심사평가원 대표가 참석하고 있지 않지만, 향후 제도적 뒷받침과 공식적 기능을 수행하면서 건강보험심사평가원 대표가 참석해서 모든 과정을 함께 한다면 지금의 많은 문제를 해결할 수 있을 것이다.

