\section{불임과 급여}
\subsection{난임 관련 진료의 급여여부}
난임을 진단하기 위한 검사 및 임신 촉진 목적의 배란촉진제 사용은 다음과 같은 경우에 요양급여하며, 환자가 원하여 실시하는 경우는 비급여대상임.\par
\emph{- 다 음 -}
\begin{enumerate}[가.]\tightlist
\item 피임없이 정상적인 부부생활을 하면서 1 년 내에 임신이 되지 않은 경우(1 차성)
\item 유산, 자궁외임신 및 분만 후 1 년 이내에 임신이 되지 않은 경우(2 차성 )
\end{enumerate}

\begin{commentbox}{난소과자극증후군과다태임신}
보조생식술 후 합병증인 \emph{과배란유도에 의한 난소과자극증후군과 다태임신}의 급여여부\par
보조생식술 후 합병증으로 나타나는 과배란유도에 의한 난소과자극증후군과 다태임신은 임신에 수반된 질병치료의 목적 또는 임신된 모체의 건강을 해할 우려가 있어 시행되는 것이므로 급여대상임.
\end{commentbox}

\Que{남은 배아를 냉동\cntrdot{}보관하는 비용도 급여 적용이 되는 건가요?}
\Ans{비급여에 해당됩니다.}
\Que{태아나 배아에 대한 유전학적 검사는 급여 적용이 되는 건가요?}
\Ans{비급여에 해당됩니다.}
\begin{commentbox}{선택적유산술} 
보조생식술 후 선택적 유산 급여여부\par 
보조생식술 후 모자보건법 제 14 조 및 동법 시행령 제 15 조의  규정에  해당되어 시행하는 선택적 유산은 비급여임 . 
\end{commentbox}
\subsection{난임부부 보조생식술 시행시  본인부담률 적용 기준} 
 
난임부부에게 보조생식술 시행시  「 국민건강보험법 시행령 」  [ 별표 2]  제 3 호 카목의 규정에 의하여 요양급여비용의  100 분의  30 을 부담하는 요양급여의 적용 범주는 다음과 같음 . \par
\emph{-  다 음  -}\par 
\begin{enumerate}[가.]\tightlist
\item 적용대상 :  보조생식술 급여기준에 해당하는 자 
\item 적용기간 :  과배란유도가 필요하여  약제를  투여하는 경우 약제  처방일 또는 자연주기를 이용하는 경우  생리시작 후 내원일 부터 배아이식일 ,  자궁강내  정자주입일 또는 시술 중단일까지의 기간 
\item 적용범위 :  보조생식술과 관련하여 발생한 일체의 요양급여비용 ( 진찰료 ,  보조생식술 시술행위료 ,  마취료 ,  약제비 등 ) \newline
 *  다만 ,  약제 ,  행위 ,  치료재료 중  「 요양급여의 적용기준  및  방법에 관한 세부사항 」 ( 또는 기타법령 ) 에서 본인부담률 ( 액 ) 을 별도로  정 한 항목은 해당 고시 ( 또는 법령 ) 에서 정한  본인부담률 ( 액 ) 을 적용함 
\item 입원의 경우 보조생식술 시술행위료 
\end{enumerate} 

\Que{본인부담률 30\%를 적용하는 보조생식술 진료기간의 요양급여비용 청구시 명세서 구분자를 기재해야 하나요?}
\Ans{본인부담률 30\%를 적용하는 보조생식술 진료기간의 요양급여비용을 청구하는 명세서에는 특정내역 MT002(특정기호)란에 'F021'기재하여 청구합니다.}
\Que{의원급 외래인 경우 보조생식술 진료기간과 그 외 진료의 본인부담률이 30\%로 동일한데도 특정기호 F021을 기재해야 하나요?}
\Ans{건보공단 대상자 사전등록 시스템에 등록된 정보와 비교하여 급여 적용여부가 결정되므로 특정기호 F021을 반드시 기재하여야 합니다. }
\Que{보조생식술 진료기간에 타상병과 동시 진료시 청구는 어떻게 하나요?}
\Ans{보조생식술과 관련된 요양급여비용은 본인부담률을 30\% 적용하고, 타상병에 대한 요양급여비용은 분리청구하여 현행 종별 본인부담률을 적용하되, 특정내역 MT001(상해외인)란에 ‘T’를 기재합니다. }

\subsection{제 9 장 처치 및 수술료 등} 
\leftrod{보조생식술 급여기준}\par 
난임부부에게 시행하는 보조생식술은  「 모자보건법 」  제 11 조의 3  및 동법 시행규칙 제 8 조에 따라 난임시술 의료기관으로 지정된  기관에서 다음과 같은 경우에 시행시 요양급여함 .  동 기준 이외  시행한 보조생식술과 잔여배아 등을 동결 \cntrdot{} 보관하는 비용은 비급여임 . \par
\emph{-  다 음  -}\par 
\begin{enumerate}[가.]\tightlist
\item 요양급여 대상자  
	\begin{enumerate}[1)]\tightlist
	\item 법적 혼인상태에 있는 난임부부  
	\item 여성 연령 만  44 세 이하 ( 연령은 과배란유도가 필요 하여 약제를 투여하는 경우 약제  처방일 또는 자연주기를  이용하는 경우 생리시작 후 내원일  당일을 기준으로 함 ) 
	\end{enumerate}
\item 요양급여 인정범위 
	\begin{enumerate}[1)]\tightlist
	\item 신선배아 : ‘ 자 640  정자채취 및 처리 ’ 부터  ‘ 자 645  배아 이식 ’ 까지의 과정 
	\item 동결배아 : ‘ 자 643  해동 ’ 부터  ‘ 자 645  배아 이식 ’ 까지의 과정 
	\item 인공수정 : ‘ 자 640  정자채취 및 처리 ’, ‘ 자 646  자궁강내 정자 주입술 ’ 
	\end{enumerate}	
\item 적응증 
	\begin{enumerate}[1)]\tightlist
	\item 체외수정 ( 신선배아 ,  동결배아 )  
		\begin{enumerate}[가)]\tightlist
		\item 원인불명 난임 : 정액검사 ,  배란기능 ,  자궁강 및 난관검사 결과 의학적 소견상 모두 정상으로 진단되었으나  3 년 이상 임신이 되지 않은 경우 ( 단 ,  여성 연령이  35 세 이상인 경우  1 년 이상 임신이 되지 않은 경우 ) 
		\item 여성요인 
			\begin{enumerate}[(1)]\tightlist
			\item 양측난관 폐색  ( 인공 폐색으로 난관문합술 이후  1 년 이상  임신이 되지 않는 경우 ) 
			\item 중증 자궁내막증 
			\item 난소기능 저하  
			\item 착상전 유전진단이 필요한 경우  
			\end{enumerate}
		\item 남성요인 
			\begin{enumerate}[(1)]\tightlist
			\item 시상하부나 뇌하수체 질환으로 인한 저성선자극호르몬성  성선기능저하증으로 최소한  24 개월간 호르몬 치료를 하였으나 이 기간 중 자연임신이 되지 않은 경우 
			\item 정관절제술을 실시했던 경우 \newline
				%\begin{enumerate}[(가)]\tightlist
				(가) 2 회 반복 정관문합술이 실패한 경우 \par` 
				(나) 정관문합술 후  3 개월 내에 사정액에서 정자가 관찰되지 않거나 ,  정자가 출현한 이후  1 년 내에 임신이  되지 않는 경우 \par 
				(다) 정관문합술이 불가한 경우\par
				%\end{enumerate}		
			\item 정계정맥류제거술 후  6 개월 이내에 정액검사 지표의  향상이 없거나 수술 후 정액검사 지표 향상이 있으나  1 년 이내 임신이 되지 않는 경우 
			\item 폐쇄성 무정자증에 대한 수술적 교정이 실패했거나 불가능한 경우 ( 수술적 교정이 불가능한 폐쇄성 무정 자증은 정관무발생 ,  다발적 정관폐쇄 ,  부고환 전체 폐쇄 를 말함 ) 
			\item 비폐쇄성 무정자증의 경우 현미경하 미세수술적다중고환조직정자추출에서 정자가 발견되어 체외수정이 가능한 경우 
			\end{enumerate}
		\item 체외수정시술 이외의 난임치료에 의해  1 년 이상 임신이 되지 않는 경우 
		\item 기타 체외수정이 필요하다는 의학적 소견이 있는 경우 
		\end{enumerate}
 	\item 인공수정  
		\begin{enumerate}[가)]\tightlist
		\item 원인불명의 난임 :  정액검사 ,  배란기능 ,  자궁강 및 난관검사 결과 의학적 소견상 모두 정상으로 진단되었으나  1 년 이상 임신이  되지 않은 경우 ( 단 ,  여성 연령이  35 세 이상인 경우  6 개월  이상 임신이 되지 않은 경우 ) 
		\item 여성요인 
			\begin{enumerate}[(1)]\tightlist
			\item 과거 자궁내막증 수술 후 자연 임신 시도  6 개월 이상 경과된 경우 
			\item 임상적으로 의심되는 자궁내막증 소견이 있으면서  1 년  이상 자연임신이 되지 않은 경우 
			\end{enumerate}		
		\item 남성요인 
			\begin{enumerate}[(1)]\tightlist
			\item 정계정맥류가 없으나  ‘ 인간정액 검사 및 처리 매뉴얼 ( 제 5 판 ,  세계보건기구 )’ 에 따른 정액 검사 결과 정자수가 적거나 정자의 운동성이 저하되어  있는 경우 
			\item 사정장애 등 기타 남성난임의 경우 
			\end{enumerate}
		\item 기타 인공수정이 필요하다는 의학적 소견이 있는 경우 
		\end{enumerate}
   \end{enumerate}
\item 급여인정 횟수 :  신선배아  4 회 ,  동결배아  3 회 ,  인공수정  3 회  
\end{enumerate}

\Que{난임을 진단받기 위해 검사를 실시한 경우(예: 자궁난관조영술)에도 본인부담률을 30\% 적용하나요?}
\Ans{국민건강보험법 시행령 [별표2] 제3호 카목의 규정에 의하여 요양급여비용의 100분의 30을 부담하는 요양급여의 적용 범주는 보조생식술 진료기간(과배란유도가 필요하여 약제를 투여하는 경우 약제 처방일 또는 자연주기를 이용하는 경우 생리시작 후 내원일부터 배아이식일, 자궁강내 정자주입일 또는 시술 중단일까지의 기간)의 요양급여비용이므로 보조생식술 진료기간 전 난임을 진단받기 위해 실시한 검사는 현행 본인부담률을 적용합니다.} 

\Que{배란유도제 투여 전에 보조생식술 필요여부를 판단하기 위해 검사를 실시한 경우(예: 자궁난관조영술)에도 본인부담률을 30\% 적용하나요?}
\Ans{난임부부에게 보조생식술 시행시 국민건강보험법 시행령 [별표2] 제3호 카목의 규정에 의하여 요양급여비용의 100분의 30을 부담하는 요양급여의 적용 범주는 보조생식술 진료기간(과배란유도가 필요하여 약제를 투여하는 경우 약제 처방일 또는 자연주기를 이용하는 경우 생리시작 후 내원일부터 배아이식일, 자궁강내 정자주입일 또는 시술 중단일까지의 기간)의 요양급여비용이므로 배란유도제 투여 전 보조생식술 필요여부를 판단하기 위해 실시한 검사는 현행 본인부담률을 적용합니다. }

\Que{배아 이식 후 임신여부 확인을 위해 내원한 경우에도 본인부담률을 30\% 적용하나요?}
\Ans{난임부부에게 보조생식술 시행시 국민건강보험법 시행령 [별표2] 제3호 카목의 규정에 의하여 요양급여비용의 100분의 30을 부담하는 요양급여의 적용 범주는 보조생식술 진료기간(과배란유도가 필요하여 약제를 투여하는 경우 약제 처방일 또는 자연주기를 이용하는 경우 생리시작 후 내원일부터 배아이식일, 자궁강내 정자주입일 또는 시술 중단일까지의 기간)의 요양급여비용이므로 배아 이식 후, 배란유도제 투여 전 보조생식술 필요여부를 판단하기 위해 실시한 검사는 현행 본인부담률을 적용합니다.}
 
\Que{과배란유도제 투여 전 조기배란억제제를 투여하는 경우 보조생식술 진료기간의 시작일은 언제부터인가요?}
\Ans{조기배란억제가 필요하여 과배란유도제 투여 전에 조기배란억제제를 투여하는 경우(장기요법) 본인부담률 30\%를 적용하는 보조생식술 진료 시작일은 조기배란억제제 처방일부터입니다.} 

\Que{자연주기를 이용한 보조생식술을 진행하는 경우 보조생식술 진료기간의 시작일은 언제부터인가요?}
\Ans{생리시작 후 2-3일경 내원하여 초음파 검사 등을 실시 후 자연주기를 이용한 보조생식술 시술을 진행하기로 결정하였다면 진료 당일이 진료 시작일이 됩니다.} 

\Que{급여적용 인정횟수를 초과하여 보조생식술을 시술받는 경우 약제를 포함한 보조생식술 과정 전부가 비급여인가요?}
\Ans{급여적용 인정횟수 초과시에는 보조생식술 시술행위료는 비급여이며, 보조생식술에 사용되는 약제에 대해서는 「요양급여의 적용기준 및 방법에 관한 세부사항과 심사지침」을 따릅니다. 이외 보조생식술과 관련하여 발생하는 비용(마취료 등)에 대해서는 급여로 적용되며 현행 종별 본인부담률을 적용합니다.
단, 약제, 행위, 치료재료 중 「요양급여의 적용기준 및 방법에 관한 세부사항」에서 본인부담률(액)을 별도로 고시한 항목은 해당 고시에서 정한 본인부담률(액)을 적용합니다.}

\Que{여성 연령 만 44세가 초과하여 보조생식술을 시술받는 경우 보조생식술 과정 전부가 비급여인가요?}
\Ans{여성 연령이 만 44세가 초과된 경우 보조생식술 시술행위료는 비급여이며, 보조생식술에 사용되는 약제에 대해서는 「요양급여의 적용기준 및 방법에 관한 세부사항과 심사지침」을 따릅니다. 이외 보조생식술과 관련하여 발생하는 비용(마취료 등)에 대해서는 급여로 적용되며 현행 종별 본인부담률을 적용합니다.
단, 약제, 행위, 치료재료 중 「요양급여의 적용기준 및 방법에 관한 세부사항」에서 본인부담률(액)을 별도로 고시한 항목은 해당 고시에서 정한 본인부담률(액)을 적용합니다.}

\Que{보조생식술 당일 합병증이 발생하여 함께 진료받는 경우 본인부담률은 어떻게 적용하나요?}
\Ans{합병증을 포함한 요양급여에 대해 국민건강보험법 시행령 [별표2] 제3호 카목의 규정에 의한 본인부담률 30\%를 적용받습니다. 
※ 하나의 명세서에 보조생식술 및 합병증 관련 요양급여비용 청구}

\Que{보조생식술 당일 타상병에 대해서도 진료받은 경우 본인부담률은 어떻게 적용하나요?}
\Ans{보조생식술과 관련된 요양급여비용은 국민건강보험법 시행령 [별표2] 제3호 카목의 규정에 의한 본인부담률 30\%를 적용하고, 타상병에 대한 진료는 현행 종별 본인부담률을 적용합니다. 
※ 타상병에 대한 요양급여비용은 분리청구}
\Que{보조생식술 진료기간의 요양급여비용은 종별 상관없이 본인부담률을 30\% 적용하나요?}
\Ans{종별 상관없이 보조생식술 진료기간의 요양급여비용은 본인부담률을 30\% 적용합니다. 다만, 상급종합병원의 경우 진찰료를 제외한 요양급여비용에 대해 본인부담률을 30\% 적용합니다.}

\Que{의료급여수급권자나 차상위인 경우에도 보조생식술 진료기간의 요양급여비용은 본인부담률을 30\% 적용하나요?} 
\Ans{보조생식술 진료기간의 요양급여비용에 대해 본인부담률 30\%는 건강보험 가입자에 대해서 적용합니다. 따라서 의료급여수급권자나 차상위의 경우 현행 통상적인 외래본인부담금(률)을 적용합니다.} 

\Que{외래에서 보조생식술 시술 행위를 한 당일 입원을 한 경우에 본인부담률은 어떻게 되나요?}
\Ans{입원본인부담률을 적용하되, 보조생식술 시술행위는 본인부담률 30\%를 적용합니다.} 

\Que{의료급여수급권자나 차상위 해당자가 보조생식술 시술 행위를 한 당일 입원을 한 경우에 본인부담률은 어떻게 되나요?}
\Ans{의료급여수급권자나 차상위의 경우 보조생식술 시술행위를 포함하여 현행 통상적인 입원본인부담금(률)을 적용합니다.}

\Que{보조생식술 합병증이 발생하여 다음날 합병증에 대한 진료만 실시한 경우에도 본인부담률을 30\% 적용할 수 있나요?}
\Ans{보조생식술 급여대상자의 본인부담률 30\% 적용기간 동안에는 보조생식술 진료 당일 이외 내원하여 진료한 합병증 치료에 대해서도 본인부담률 30\%가 적용됩니다. }

\Que{보조생식술 급여기준의 적응증 중 여성요인의 ‘난소기능 저하’의 진단 기준은 어떻게 되나요?}
\Ans{기존 ‘난임부부 시술비 지원사업’에서 적용된 진단기준과 동일하게 적용됩니다.\par
\emph{<참고> 난임부부 시술비 지원사업에서 제시한 난소기능 저하 진단 기준 }\par
아래 3가지 요인 중 2가지 이상에 해당하는 경우에 난소기능 저하로 진단합니다. 
\begin{enumerate}[①]\tightlist
\item 난소기능 검사(Ovarian reserve test)결과 기능저하 <난소기능 검사결과 비정상 기준>
	\begin{itemize}\tightlist	
	\item 초기 난포기 질식 초음파상 양측 난소에 난포수(Antral follicle count:AFC)가 6개 이하
    \item AMH 검사결과 1.0ng/mL이하
    \item FSH 12mIU/ml이상
	\end{itemize}
\item POR(Poor Ovarian Reserve)의 위험인자
	\begin{itemize}\tightlist	
	\item 나이 40이상, 터너증후군(Turner syndrome), FMR1 premutation, 골반염증(Pelvic infection), 난관손상(Tubal damage), 클라미디아 검사 양성(Chlamydia antibody test: +), 자궁내막증(Ovarian endometrioma), 난소낭종 수술력(Ovarian surgery for ovarian  cysts), 골반장기 과거 수술력, 항암치료(Chemotherapy, 특히 alkylating agent), 생리주기가 짧아짐(Shortening of the menstrual cycle) 등
	\end{itemize}
\item POR(Poor Ovarian Response) 과거력
	\begin{itemize}\tightlist	
	\item 3개 미만의 growing follicle로 인하여 cycle이 취소되거나 혹은 적어도 하루에 150IU FSH 이상을 적용한 ovarian stimulation protocol에서 3개 이하의 oocytes가 얻어지는 경우
	\end{itemize}
\end{enumerate}   
}

\Que{보조생식술 급여기준의 적응증 중 원인불명 난임의 검사 기준은 어떻게 되나요?}
\Ans{기존 ‘난임부부 시술비 지원사업’에서 적용된 의학적 기준 가이드라인의 검사기준과 동일하게 적용됩니다.\par
\emph{<참고> 난임부부 시술비 지원사업에서 제시한 검사기준 }
\begin{itemize}\tightlist
\item 정액검사: 인간정액 검사 및 처리 매뉴얼(제5판, 세계보건기구)의  정상기준에 따름(총 사정액 1.5ml 이상, 정자수 1천5백만/ml 이상, 전진성 운동 정자의 비율이 32\% 이상이거나 운동성 있는 정자비율이 40\% 이상, 엄격기준에 따른 정상적인 모양의 정자 4\% 이상)
\item 배란기능: 황체기 중반 혈중 프로게스테론 검사로 확인하는 것을 추천하나, 규칙적인 월경주기를 가지면서 배란증상을 보일 경우 정상배란으로 판단 가능
\item 자궁강 및 난관검사: 자궁난관조영술(HSG) 혹은 HyCoSy, 복강경검사나 개복수술 중 진단하는 것을 원칙으로 하며, 검사 결과 최소한 한쪽 나팔관은 정상이어야 함
\end{itemize}
}

\Que{우리나라에 거주하는 외국남성(건강보험 가입 안 됨)과 결혼한 한국인 여성(건강보험 가입)이 보조생식술 시술시 급여 인정이 가능한가요?}
\Ans{보조생식술 급여 적용 대상자(보조생식술이 필요하다는 진단을 받은 법적 혼인상태의 부부로 여성 연령 만 44세 이하이며 급여횟수가 남아있는 경우)에 해당된다면 건강보험에 가입되어 있는 한국인 여성의 경우 급여 적용됩니다.} 
\Que{외국인과 결혼한 여성이 남편 국가에만 혼인신고를 한 경우에도 법적 부부에 해당하여 급여 적용받을 수 있나요?}
\Ans{법적 부부란 민법 제812조제1항에 의거 「가족관계의 등록 등에 관한 법률」에 정한 바에 따라 법률상 혼인신고를 통해 법률적인 효력이 발생한 경우를 의미하므로 우리나라에 혼인신고가 되어 있지 않은 경우 급여 적용을 받을 수 없습니다. } 

