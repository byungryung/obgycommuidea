\section{불임과 급여}

\subsection{난임부부 보조생식술 시행시  본인부담률 적용 기준} 
 
난임부부에게 보조생식술 시행시  「 국민건강보험법 시행령 」  [ 별표 2]  제 3 호 카목의 규정에 의하여 요양급여비용의  100 분의  30 을 부담하는 요양급여의 적용 범주는 다음과 같음 . \par
\emph{-  다 음  -}\par 
\begin{enumerate}[가.]\tightlist
\item 적용대상 :  보조생식술 급여기준에 해당하는 자 
\item 적용기간 :  과배란유도가 필요하여  약제를  투여하는 경우 약제  처방일 또는 자연주기를 이용하는 경우  생리시작 후 내원일 부터 배아이식일 ,  자궁강내  정자주입일 또는 시술 중단일까지의 기간 
\item 적용범위 :  보조생식술과 관련하여 발생한 일체의 요양급여비용 ( 진찰료 ,  보조생식술 시술행위료 ,  마취료 ,  약제비 등 ) \par
 *  다만 ,  약제 ,  행위 ,  치료재료 중  「 요양급여의 적용기준  및  방법에 관한 세부사항 」 ( 또는 기타법령 ) 에서 본인부담률 ( 액 ) 을 별도로  정 한 항목은 해당 고시 ( 또는 법령 ) 에서 정한  본인부담률 ( 액 ) 을 적용함 
\item 입원의 경우 보조생식술 시술행위료 
\end{enumerate} 

\subsection{제 9 장 처치 및 수술료 등} 
\leftrod{보조생식술 급여기준}\par 
난임부부에게 시행하는 보조생식술은  「 모자보건법 」  제 11 조의 3  및 동법 시행규칙 제 8 조에 따라 난임시술 의료기관으로 지정된  기관에서 다음과 같은 경우에 시행시 요양급여함 .  동 기준 이외  시행한 보조생식술과 잔여배아 등을 동결 \cntrdot{} 보관하는 비용은 비급여임 . \par
\emph{-  다 음  -}\par 
\begin{enumerate}[가.]\tightlist
\item 요양급여 대상자  
	\begin{enumerate}[1)]\tightlist
	\item 법적 혼인상태에 있는 난임부부  
	\item 여성 연령 만  44 세 이하 ( 연령은 과배란유도가 필요 하여 약제를 투여하는 경우 약제  처방일 또는 자연주기를  이용하는 경우 생리시작 후 내원일  당일을 기준으로 함 ) 
	\end{enumerate}
\item 요양급여 인정범위 
	\begin{enumerate}[1)]\tightlist
	\item 신선배아 : ‘ 자 640  정자채취 및 처리 ’ 부터  ‘ 자 645  배아 이식 ’ 까지의 과정 
	\item 동결배아 : ‘ 자 643  해동 ’ 부터  ‘ 자 645  배아 이식 ’ 까지의 과정 
	\item 인공수정 : ‘ 자 640  정자채취 및 처리 ’, ‘ 자 646  자궁강내 정자 주입술 ’ 
	\end{enumerate}	
\item 적응증 
	\begin{enumerate}[1)]\tightlist
	\item 체외수정 ( 신선배아 ,  동결배아 )  
		\begin{enumerate}[가)]\tightlist
		\item 원인불명 난임 : 정액검사 ,  배란기능 ,  자궁강 및 난관검사 결과 의학적 소견상 모두 정상으로 진단되었으나  3 년 이상 임신이 되지 않은 경우 ( 단 ,  여성 연령이  35 세 이상인 경우  1 년 이상 임신이 되지 않은 경우 ) 
		\item 여성요인 
			\begin{enumerate}[(1)]\tightlist
			\item 양측난관 폐색  ( 인공 폐색으로 난관문합술 이후  1 년 이상  임신이 되지 않는 경우 ) 
			\item 중증 자궁내막증 
			\item 난소기능 저하  
			\item 착상전 유전진단이 필요한 경우  
			\end{enumerate}
		\item 남성요인 
			\begin{enumerate}[(1)]\tightlist
			\item 시상하부나 뇌하수체 질환으로 인한 저성선자극호르몬성  성선기능저하증으로 최소한  24 개월간 호르몬 치료를 하였으나 이 기간 중 자연임신이 되지 않은 경우 
			\item 정관절제술을 실시했던 경우 \newline
				%\begin{enumerate}[(가)]\tightlist
				(가) 2 회 반복 정관문합술이 실패한 경우 \par` 
				(나) 정관문합술 후  3 개월 내에 사정액에서 정자가 관찰되지 않거나 ,  정자가 출현한 이후  1 년 내에 임신이  되지 않는 경우 \par 
				(다) 정관문합술이 불가한 경우\par
				%\end{enumerate}		
			\item 정계정맥류제거술 후  6 개월 이내에 정액검사 지표의  향상이 없거나 수술 후 정액검사 지표 향상이 있으나  1 년 이내 임신이 되지 않는 경우 
			\item 폐쇄성 무정자증에 대한 수술적 교정이 실패했거나 불가능한 경우 ( 수술적 교정이 불가능한 폐쇄성 무정 자증은 정관무발생 ,  다발적 정관폐쇄 ,  부고환 전체 폐쇄 를 말함 ) 
			\item 비폐쇄성 무정자증의 경우 현미경하 미세수술적다중고환조직정자추출에서 정자가 발견되어 체외수정이 가능한 경우 
			\end{enumerate}
		\item 체외수정시술 이외의 난임치료에 의해  1 년 이상 임신이 되지 않는 경우 
		\item 기타 체외수정이 필요하다는 의학적 소견이 있는 경우 
		\end{enumerate}
 	\item 인공수정  
		\begin{enumerate}[가)]\tightlist
		\item 원인불명의 난임 :  정액검사 ,  배란기능 ,  자궁강 및 난관검사 결과 의학적 소견상 모두 정상으로 진단되었으나  1 년 이상 임신이  되지 않은 경우 ( 단 ,  여성 연령이  35 세 이상인 경우  6 개월  이상 임신이 되지 않은 경우 ) 
		\item 여성요인 
			\begin{enumerate}[(1)]\tightlist
			\item 과거 자궁내막증 수술 후 자연 임신 시도  6 개월 이상 경과된 경우 
			\item 임상적으로 의심되는 자궁내막증 소견이 있으면서  1 년  이상 자연임신이 되지 않은 경우 
			\end{enumerate}		
		\item 남성요인 
			\begin{enumerate}[(1)]\tightlist
			\item 정계정맥류가 없으나  ‘ 인간정액 검사 및 처리 매뉴얼 ( 제 5 판 ,  세계보건기구 )’ 에 따른 정액 검사 결과 정자수가 적거나 정자의 운동성이 저하되어  있는 경우 
			\item 사정장애 등 기타 남성난임의 경우 
			\end{enumerate}
		\item 기타 인공수정이 필요하다는 의학적 소견이 있는 경우 
		\end{enumerate}
   \end{enumerate}
\item 급여인정 횟수 :  신선배아  4 회 ,  동결배아  3 회 ,  인공수정  3 회  
\end{enumerate}
 

\begin{commentbox}{선택적유산술} 
보조생식술 후 선택적 유산 급여여부\par 
보조생식술 후 모자보건법 제 14 조 및 동법 시행령 제 15 조의  규정에  해당되어 시행하는 선택적 유산은 비급여임 . 
\end{commentbox}
 
\section{급여기준들}
\leftrod{자 640, 정자채취 및 처리}\par 
\emph{감염환자 정자 처리 ,  역행성사정  정자처리 ,  정자 운동성 촉진 처리  급여기준} \par
정자채취 및 처리시 감염환자에게 실시한 경우나 역행성사정을 통해 얻어진 경우 ,  정자운동성 촉진 처리가 필요한 경우로 다음의 경우에 요양급여를 인정함 . \par
\emph{-  다 음  -}\par  
\begin{enumerate}[가.]\tightlist
\item B 형 , C 형 간염보균자나  HIV  보균자의 정자를 처리하는 경우 
\item 농정액 (pyospermia)  정자를 처리 하는 경우 
\item 역행성사정 환자의 소변에서 정자를 확보하여 처리하는 경우 
\item 정자운동성이 저하 ( 전진성 운동 정자의 비율이  10% 이하 이거나 운동성 있는 정자 비율이  20% 이하 ) 되어  운동 촉진  처리를 하는 경우 
\item 전기자극을 이용하여 채취한 정자를 처리하는 경우 
\end{enumerate}

\emph{고환조직정자흡인의 급여 기준}\par 
고환조직에서 정자를 채취하기 위해 가는 바늘로 고환을 찔러  세정관내에 존재하는 정자를 찾는 고환조직정자흡인은 수술적  교정이 불가능한 폐쇄성무정자증으로 확인된 자 중 의학적으로  고환조직정자추출 시행이 불가능한 경우 ( 이전의  고환의 염증성  질환 및 반복적인 수술로 인하여 유착이 심한 경우 ,  고환암이 의심되는 경우 ) 에 요양급여를 인정함 . \par
 

\emph{고환조직정자추출의 급여기준}\par 
정액에서 체외수정에 사용할만한 정자가 없는 경우에 고환조직을  일부 절제하여 정자를 얻는 고환조직정자추출은 다음의 경우에 요양급여를 인정함 . \par

\emph{-  다 음  -}\par 
\begin{enumerate}[가.]\tightlist
\item 수술적 교정이 불가능한 폐쇄성무정자증 
\item 폐쇄성무정자증 환자에서 정관정관문합술 (vasovasostomy) 이나 정관부고환문합술 (vasoepididymostomy)  수술에 실패한  경우 
\item 사정장애가 있는 환자 중에서 약물치료로 교정이 되지 않는  경우 
\item 발기기능장애로 기존의 치료로 교정이 되지 않는 경우 
\item 체외수정 당일 사정된 정액에서 정상모양의 정자가 하나도  없거나 모든 정자가  비활동성인  경우 
\end{enumerate} 

\emph{미세수술적 부고환정자흡인술의 급여기준}\par 
부고환에서 정자를 채취하는 미세수술적 부고환정자흡인술은 다음의 경우에 요양급여를 인정함 . \par

\emph{-  다 음  -}\par 
\begin{enumerate}[가.]\tightlist
\item 고환의 백막에 접근이 불가능하여 고환조직 채취가 불가능한 경우 
\item 고환의 악성종양이 의심되는 경우 
\end{enumerate} 

\emph{현미경하 미세 수술적다중고환 조직정자추출의 급여기준}\par 
현미경하 미세수술적다중고환조직정자추출은 비폐쇄성무정자증으로 진단된 경우에만 요양급여를 인정함 . \par

\leftrod{자 641 난자채취 및 처리}\par  
\emph{난자 활성화 급여기준}\par 
수정을 돕기 위한 난자 활성화는 다음의 경우에 요양급여를 인정함 . \par

\emph{-  다 음  -}\par  
\begin{enumerate}[가.]\tightlist
\item 성숙난자 채취를 시도했으나 모든 난자가 미성숙난자로 채취된 경우 
\item 채취된 성숙난자 중  70%  이상 수정되지 않는 경우 
\item 정자운동성이 없는 경우 ( 전진성 운동 정자의 비율이  10% 이하이거나 운동성 있는 정자 비율이  20% 이하 ) 
\item 이전 체외수정 시술에서 모든 난자의 수정 실패 혹은 수정률  저하 (40%  미만 ) 를 보였던 경우 
\item 이전 체외수정 시술에서 세포질내 정자주입술로 수정이  이루어졌으나 배아발달이 비정상적으로 느리거나 배아질이  많이 떨어지는 경우 
\end{enumerate} 

\leftrod{자 642 수정 및 확인}\par 
\emph{세포질내  정자주입술의  급여기준} \par 
수정률을 높이기 위해 난자의 세포질내에 정자를 직접 주입하여 수정을 유도하는 세포질내 정자주입술은 다음의 경우에 요양급여를 인정함 . \par

\emph{-  다 음  -}\par 
\begin{enumerate}[가.]\tightlist
\item 희소정자증 ,  무력정자증 ,  기형정자증 ,  희소무력정자 ,  희소무운동성기형정자증 등과 같은 심각한 남성인자로 인한 난임인 경우 
\item 항 정자 항체가 존재하는 경우 
\item 척수손상 환자 ,  사정장애가 있는 환자 ,  역방향 사정 환자의 경우 
\item 폐쇄성무정자증인 경우  
\item 성숙정지에 기인한 고환부전 ,  부분 생식세포 무형성증의 경우 
\item 동결보존된 정자나 난자를 이용하는 경우  
\item 유전질환에 대한 착상 전 유전진단이 필요한 경우 
\item 중증의 자궁내막증 ,  난소기능저하가 있는 경우 
\item 미성숙 난자를 수정시키는 경우 
\item 이번 일반 체외수정 실시 후 수정 실패한 경우  
\item 이전 일반 체외수정 실시 후 수정 실패하였거나 배발생률이 낮았던 경우 
\item 이전 체외수정 시술 후  2 회 이상의 반복 임신 실패력이 있는 경우 
\end{enumerate} 

\emph{세포질내 정자 주입술 ( 고배율 현미경 ,  편광 현 미경 이용 ) 의  급여 기준 }\par
수정률을 높이기 위해 고배율 현미경 (IMSI)  등을 이용한 정자  선별이나 편광 현미경을 이용한 정자주입 위치 선별을 통한 세포질내 정자주입술은 다음의 경우에 요양급여를 인정함 . \par

\emph{-  다 음  -}\par 
\begin{enumerate}[가.]\tightlist
\item 이전 체외수정 시술 후 자연 유산  2 회 이상 경험한 경우 
\item 이전 체외수정 시술에서 세포질내 정자주입술을 시행했으나  반복 임신 실패나  2 회 이상의 화학적 임신을 경험한 경우 
\item 이전 체외수정 시술에서 세포질내 정자주입술로도  40%  이하의 낮은 수정률을 보인 경우 
\item 이전 체외수정 시술에서 포배기 배발달률이 낮은 경우 
\item 이전 체외수정 시술에서 모든 난자의 수정 실패가 있었던 경우 
\item 심한 기형정자증이 있는 경우 ( 정상정자의 비율  1%  이하 ) 
\item 수술적으로 채취 후 동결한 정자를 해동하여 세포질내 정자 주입술을 시행하는 경우 
\end{enumerate} 

\emph{히알루론산 결합  정자선별의 급여 기준 }\par
정자 선별시 형태와 운동성 외에 히알루로난이 코팅된 장치를  사용하여 성숙한 정자를 선별하여 수정을 시도하는 히알루론산  결합 정자선별은 다음의 경우에 요양급여를 인정함 . \par

\emph{-  다 음  -}\par  
\begin{enumerate}[가.]\tightlist
\item 이전 체외수정 시술 후 자연 유산을  2 회 이상 경험한 경우 
\item 이전 체외수정 시술에서 세포질내 정자주입술을 시행했으나  반복 임신 실패나  2 회 이상의 화학적 임신을 경험한 경우  
\item 이전 체외수정 시술에서 세포질내 정자 주입술로도  40%  이하의 낮은 수정률을 보인 경우 
\item 이전  세포질내 정자주입술시  포배기 배발달률이 낮거나 지연 발육되었던 경우 
\item 정상적인 형태의 정자가  1%  미만이거나 운동성이 심하게 감소 ( 전진성 운동 정자의 비율이  10% 이하이거나 운동성 있는 정자 비율이  20% 이하 ) 되어 있는 경우 
\item 정자성숙도가 떨어지는 경우 
\end{enumerate} 

\leftrod{자 644 배아 배양 및 관찰 }\par
\emph{배아 활성화의 급여기준}\par
자연적인 배양 과정이 원활히 진행되지 않아 약물 ,  전기 등의  방법을 사용하여 배양을 돕는 배아 활성화는  다음의 경우에 요양급여를 인정함 . \par
\emph{-  다 음  -}\par  
\begin{enumerate}[가.]\tightlist
\item 난소저반응군에 해당하는 경우 
	\begin{enumerate}[1)]\tightlist
	\item 난소저반응군의 과거력이 없는 경우 이번 시술시 난포 자극호르몬 주사제를 최소 하루에  150 IU  이상 사용한  과배란유도로  3 개 이하의 난자가 얻어진 경우 
	\item 난소저반응군의 과거력이 있는 경우 이번 시술시 난포 자극호르몬 주사제를 최소 하루에  150IU  이상 사용한 과배란유도로  5 개 이하의 난자가 얻어진 경우 
	\end{enumerate}
\item 미성숙 난자가 많이 나온 경우 
	\begin{enumerate}[1)]\tightlist
	\item 성숙난자 채취를 시도하였으나 미성숙난자가  70\%  이상 나온 경우 
	\item 미성숙난자 채취를 시도하여 난자를 성숙시켰으나  70\%  이상에서 실패한 경우 
	\end{enumerate}
\item 배아 발달이 심하게 늦거나 ,  발달이 정지된 경우 
\end{enumerate} 

\emph{지속적 관찰을 시행한 경우의 급여기준 및 산정방법} \par
배아의 발달 속도가 불규칙하여  Time-lapse  시스템 등을 이용하여 배아 발달 상태를 지속적으로 모니터링 해야 하는 다음과  같은 경우에 요양급여로 산정하되 국민건강보험법 시행규칙 별표 6 1. 다 . 에  따라 본인부담률을  100 분의  80 으로 적용 함 . \par
\emph{-  다 음  -}\par 
\begin{enumerate}[가.]\tightlist
\item 이전 체외수정 시술 후 반복 임신 실패나  2 회 이상의 화학적  임신을 경험한 경우 
\item 단일 배아이식을 예정할 경우 
\item 기타 지속적 배아 관찰이 필요하다는 의학적 소견이 있는 경우 
\end{enumerate} 

\leftrod{자 645 배아 이식, 보조부화술의  급여기준}\par 
착상률을 향상시키기 위해 배아이식 전 투명대에 인위적으로  절개를 가하여 배아의 부화를 돕는 보조부화술은  다음의 경우에 요양급여를 인정함 . \par
\emph{-  다 음  -}\par  
\begin{enumerate}[가.]\tightlist
\item 여성의 연령이  40 세 이상인 경우 
\item 투명대가 정상보다 두꺼운 경우 ( ≧ 15 μ m)  또는 투명대의 색이 검거나 비정형 모양인 경우 
\item 난포자극호르몬 (FSH)  수치가 정상보다 높은 경우 (FSH ≧ 12) 
\item 난할기 동결배아 이식과 같이 투명대의 경화현상이 발생하는 경우 
\item 이전 체외수정 시술 시 양질의 수정란을 이식하였으나  2 회  이상 착상 실패한 경우 
\item 이전 체외수정 시술 시 배아의 부화가 일어나지 않았던 경우
\end{enumerate} 

\subsection{보조생식술에서의 초음파}
보조 생식술 관련만 급여 적용입니다. \uline{보조생식술과 관련 없는 모니터링 초음파는 비급여}입니다
\Que{보조생식술과 관련된 각종 검사나 초음파 검사도 건강보험이 적용되나요?}
\Ans{보조생식술에 소용된 비용이 급여 적용되므로 해당 검사들은 원칙적으로 모두 급여로 전환됩니다. \par
다만, 개별 검사별 급여기준이 별도 존재하거나 비급여로 명시되어 있는 경우는 이에 따라 건강보험이 적용됩니다.
초음파 검사는 보조생식술 시술을 위해 시행되는 경우 건강보험을 적용합니다\par
보조생식술을 위해 초음파를 시행하는 경우
	\begin{enumerate}[1)]\tightlist
	\item 보조생식술 진료시작일에 자궁부속기 및 자궁내막의 상태 등을 보는 경우 나944라(1) 여성생식기 초음파(일반)를 산정함
	\item 보조생식술 관련 약제투여 후 난포의 크기 및 수, 자궁내막두께 등을 관찰하는 경우 나940나 단순초음파(Ⅱ)를 산정함
	\end{enumerate}
}

\subsection{보조생식술에 사용되는 호르몬 약제}
1. 각 약제의 허가사항 및 급여기준 범위 내에서 투여 시 요양급여를 인정함.\par
2. 허가사항 범위를 초과하여 보조생식술에 아래와 같은 기준으로 투여 시「보조생식술 급여기준」 범위 내에서 요양급여를 인정하며, 비급여로 실시한 경우에는 약값 전액을 환자가 부담토록 함.\par
- 아     래 -\par
	\begin{enumerate}[가.]\tightlist
	\item Estradiol valerate 경구제, Medroxyprogesterone acetate 경구제 
	\item Prednisolone 경구제, Dexamethasone 경구제
		\begin{itemize}[○]\tightlist
		\item 투여 대상
			\begin{enumerate}[1)]\tightlist
			\item 반복 유산과 관련된 자가면역 질환이 있는 환자
			\item 원인불명으로 3회 이상 반복하여 착상 실패를 경험한 환자
			\end{enumerate}
		\end{itemize}
	\end{enumerate}	
3. 허가사항 범위를 초과하여 보조생식술에 아래와 같은 기준으로 투여 시 약값 전액을 환자가 부담토록 함.\par
- 아     래 -\par
	\begin{enumerate}[가.]\tightlist
	\item  Micronized progesterone 경구제의 경구 투여.
		\begin{enumerate}[1)]\tightlist
		\item 체외수정 시 Progesterone의 보충요법
		\item Progesterone 결핍으로 인한 불임 여성의 난모세포 공여프로그램
		\item 반복적인 유산(3회 이상)의 과거력이 있는 환자
		\end{enumerate}
	\item  성장호르몬제(Growth hormone, Somatropin)
		\begin{itemize}[○]\tightlist
		\item 체외수정을 시행하는 여성 중 다음 2가지 이상을 만족하는 경우
			\begin{enumerate}[1)]\tightlist
			\item 만 40세 초과 연령
			\item 난소 예비능이 좋지 않은 과거력 
			\item 난소 기능 저하
			\end{enumerate}
		\end{itemize}
	\end{enumerate}	
     ※ 대상 약제: Estradiol valerate 경구제, Medroxyprogesterone acetate 경구제, Prednisolone 경구제, Dexamethasone 경구제, Micronized progesterone 경구제, Somatropin 서방형 주사제, Somatropin 주사제
