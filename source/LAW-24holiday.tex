\section{대체휴무 등 공휴가산 관련}
\Que{근로자의 날(5월 1일)과 대체공휴일에 진료를 한 경우 공휴가산이 적용되나요?}
\Ans{「건강보험 행위 급여m\cntrdot{}비급여\cntrdot{}비급여 목록표 및 급여 상대가치점수(보건복지부 고시)」에 의한 공휴가산은 관공서의 공휴일에 관한 규정에 의거 공휴일로 정해진 날에 진료를 행한 경우에 산정할 수 있습니다. 근로자의 날(5월 1일)은 근로자의 날 제정에 관한 법률에 의거 유급휴일로 정한 것으로 “관공서의 공휴일에 관한 규정”에 해당되지 아니하므로 근로자의 날에 진료를 실시한 경우에는 공휴가산을 적용할 수 없음을 알려드립니다. 또한, “관공서의 공휴일에 관한 규정”에서 말하는 ‘공휴일’은 제2조에 따른 공휴일 및 제3조에 따른 대체공휴일을 포함하는 개념으로 판단되어 대체공휴일은 공휴가산을 적용할 수 있습니다.}

\begin{commentbox}{관공서의 공휴일에 관한 규정}
\begin{description}\tightlist
\item[제1조(목적)] 이 영은 관공서의 공휴일에 관한 사항을 규정함을 목적으로 한다.
\item[제2조(공휴일)] 관공서의 공휴일은 다음과 같다. 다만, 재외공관의 공휴일은 우리나라의 국경일중 공휴일과 주재국의 공휴일로 한다.
	\begin{enumerate}[1.]\tightlist
	\item 일요일
	\item 국경일 중 3\cntrdot{}1절, 광복절, 개천절 및 한글날
	\item 1월 1일
	\item 설날 전날, 설날, 설날 다음날 (음력 12월 말일, 1월 1일, 2일)
	\item 삭제 <2005.6.30.>
	\item 석가탄신일 (음력 4월 8일)
	\item 5월 5일 (어린이날)
	\item 6월 6일 (현충일)
	\item 추석 전날, 추석, 추석 다음날 (음력 8월 14일, 15일, 16일)
	\item 12월 25일 (기독탄신일)
	\item 「공직선거법」 제34조에 따른 임기만료에 의한 선거의 선거일
	\item 기타 정부에서 수시 지정하는 날
	\end{enumerate}
\item[제3조(대체공휴일)]
	\begin{enumerate}[①]\tightlist
	\item 제2조제4호 또는 제9호에 따른 공휴일이 다른 공휴일과 겹칠 경우 제2조제4호 또는 제9호에 따른 공휴일 다음의 첫 번째 비공휴일을 공휴일로 한다.
	\item 제2조제7호에 따른 공휴일이 토요일이나 다른 공휴일과 겹칠 경우 제2조제7호에 따른 공휴일 다음의 첫 번째 비공휴일을 공휴일로 한다.[본조신설 2013.11.5.]
	\end{enumerate}
\end{description}	
\end{commentbox}