\section{삭감 조심해야할 약물 - 항히스타민 제제들} \label{anti_histamin}
1세대 항히스타민제재는 감기상병으로 처방가능하지만, 2세대부터는 꼭 알러지성 비염 코드가 있어야 삭감이 안됨. \par
계절성 및 다년성 알러지성 비염, 알러지성 결막염, 만성 특발성 두드러기, 피부 소양증의 병명이 들어 가야 합니다. \uline{감기에는 반드시 삭감입니다.\footnote{1세대 항히스타민제재는 삭감없고, 2세대 항히스타민제재는 삭감됨}}
\subsection{항히스타민제}
\tabulinesep =_2mm^2mm
\begin {tabu} to\linewidth {|X[2,l]|X[2,l]|X[2,l]|} \tabucline[.5pt]{-}
\rowcolor{ForestGreen!40} & \centering 성분명 & \centering 상품명(하루용량) \\ \tabucline[.5pt]{-}
\rowcolor{Yellow!40}  1 세대 & piprinhydrinate & 푸라콩(3) \\ \tabucline[.5pt]{2-3}
\rowcolor{Yellow!40}  부작용: 졸림,구강건조 & \emph{Pheniramin} & \emph{페니라민(3-6)} \\ \tabucline[.5pt]{-}
\rowcolor{Yellow!40} 2세대 & Ketotifen & 자디텐(2) \\ \tabucline[.5pt]{2-3}
\rowcolor{Yellow!40} 알러지성 비염코드 J304 &  Cetrizine & 지르텍 \\ \tabucline[.5pt]{2-3}
\rowcolor{Yellow!40} & Mizolastine & 미졸렌 \\ \tabucline[.5pt]{2-3}
\rowcolor{Yellow!40} & Loratadine & 클라리틴 \\ \tabucline[.5pt]{2-3}
\rowcolor{Yellow!40} & Azelastine & 아젭틴(2) \\ \tabucline[.5pt]{2-3}
\rowcolor{Yellow!40} & Fexofedine & 알레그라 \\ \tabucline[.5pt]{2-3}
\rowcolor{Yellow!40} & Ebastine & 에바스텔 \\ \tabucline[.5pt]{2-3}
\rowcolor{Yellow!40} & Bepotatatine & 타리온(2) \\ \tabucline[.5pt]{2-3}
\rowcolor{Yellow!40} & Levocetrizine & 씨잘 \\ \tabucline[.5pt]{2-3}
\rowcolor{Yellow!40} & Desloratidine & 에리우스 \\ \tabucline[.5pt]{-}
\end{tabu}
