\subsection{질병군 급여 일반원칙(비급여)}
\begin{myshadowbox}
\begin{enumerate}[4.]\tightlist
\item 제2부 각 장에 \textcolor{red}{분류된 질병군 상대가치점수(이하 “점수”라 한다)}는 다음 각목의 행위ㆍ약제 및 치료재료를 포함한다.
	\begin{enumerate}[가.]\tightlist
	\item 제1편 행위 급여ㆍ비급여 목록 및 급여 상대가치점수에서 정한 행위급여목록표에 고시된 행위
	\item 요양급여기준 제8조제2항의 규정에 의하여 고시된 약제 급여 목록 및 급여 상한금액표의 약제와 치료재료 급여ㆍ비급여 목록 및 급여 상한 금액표의 치료재료
	\item \textcolor{red}{요양급여기준 별표 2의 비급여대상 중 제6호의 비급여대상}을 제외한 행위ㆍ약제 및 치료재료
	\item 국민건강보험법 시행규칙 별표 6의 본인이 요양급여비용의 100분의 100을 부담하는 항목 중 제1호 자목에 해당하는 항목을 제외한 행위ㆍ약제 및 치료재료
	\item 다음 항목 중 위 가목 내지 라목에 해당하는 경우
		\begin{enumerate}[(1)]\tightlist
		\item 요양급여기준 별표 1 제1호 마목에서 장관이 정하는 바에 따라 다른 기관에 검사를 위탁하거나 당해 요양기관에 소속되지 아니한 전문성이 뛰어난 의료인을 초빙하거나, 또는 다른 요양기관에서 보유하고 있는 양질의 시설ㆍ인력 및 장비를 공동 사용하는 경우 소요되는 행위ㆍ약제 및 치료재료
		\item 입ㆍ퇴원 당일에 발생한 행위ㆍ약제 및 치료재료로써 외래진료 및 퇴원약제 등을 포함하되 다음 항목은 제외한다.
			\begin{enumerate}[(가)]\tightlist
			\item 질병군 입원을 예견하지 못한 상태에서 \textcolor{red}{입원 당일 외래진료를 받은 경우의 원외처방 약제비}
			\item 질병군으로 퇴원 후 \textcolor{red}{질병군과 관계없는 상병으로 퇴원 당일 외래진료를 받은 경우의 원외처방 약제비}
			\item 질병군으로 \textcolor{red}{퇴원 후 질병군 질환과 관계없는 상병으로 퇴원 당일 재입원하는 경우의 요양급여비용}
			\end{enumerate}
		\item 요양기관의 요구에 의하여 가입자 등이 외부에서 직접 구입한 약제 및 치료재료
		\end{enumerate}
	\end{enumerate}	
\end{enumerate}
\end{myshadowbox}
\prezi{\clearpage}
%\begin{commentbox}{질병군 급여 일반원칙(비급여)}
%\end{commentbox}
\Que{질병군 진료기간 중에 환자가 원하여 불임관련 진료를 시행한 경우 별도 산정 여부}
\Ans{질병군 진료기간 중 일차성 불임과 이차성 불임에 해당되어 \textcolor{red}{불임관련 진료를 시행한 경우는 질병군 급여상대가치점수에 포함} 됨. 다만, 일차성 불임과 이차성 불임에 해당되지 않고 환자가 원하여 실시한 불임 관련 진료는 비급여대상임\par
☞ 고시 제2004-36호 (‘04.6.24, ‘04.7.1. 시행)「불임관련 진료의 요양급여여부」}
\prezi{\clearpage}
\par
\medskip
\Que{자궁내장치(IUD)를 교체하고 재삽입하는 경우 질병군 적용방법은?}
\Ans{본인이 원하여 자궁내장치삽입술을 시술받고 동 장치를 교체하기 위하여 기유치된 자궁내장치를 제거하고 재삽입하는 경우의 관련 진료비용은 \textcolor{red}{비급여대상}임. 다만, 피임시술 요양급여 대상자가 기존에 유치된 자궁내 장치를 제거하고 새기구를 삽입하는 경우는 질병군 급여상대가치점수에 포함됨\par
☞ 고시 제2011-50호(‘11.4.29.) 「자궁내장치(IUD) 교체시 제거료 산정방법」}
\prezi{\clearpage}
\par
\medskip
\Que{통증자가조절법(PCA)시 비급여 치료재료나 약제 사용시 적용방법은?}
\Ans{ ‘각종 수술 후 통증관리를 위한 통증자가조절법(PCA)’은 행위별수가제와 더불어 질병군 포괄수가제에서도 요양급여비용의 전액(100분의 100)을  \textcolor{red}{본인이 부담하는 항목으로 산정기준은 행위별과 동일함}. 다만, 치료재료 및 약제가 비급여인 경우 식약처장의 허가사항(효능ㆍ효과 및 용법ㆍ용량 등) 범위 안에서 사용하되 가격은 요양기관이 실제 구입한 금액으로 산정 함.\par
☞ 고시(2005-101호 2005.12.30)「통증자가조절법(PCA)의 급여여부」}
\prezi{\clearpage}
\par
\medskip
\Que{질병군으로 입원진료 중 환자가 원하여 시행하는 요실금수술 별도 산정 여부}
\Ans{환자가 요실금수술을 원하는 경우라도 수술의 요양급여대상 또는 비급여대상여부는 객관적인 검사를 통하여 결정되어야 함. 또한 요실금수술이 \textcolor{red}{비급여 대상으로 결정이 되면 해당 수술 비용과 사용된 치료재료 등은 별도 산정함(비급여)}\par
☞ 고시 제2011-144호(‘11.11.25.) 「인조테이프를 이용한 요실금수술 인정기준」}
\prezi{\clearpage}
\par
\medskip
\Que{질병군 대상 수술 후 기력저하 등의 이유로 환자가 원하여 투여하는 영양제의 별도 산정 여부}
\Ans{질병군 포괄수가에는 급여와 비급여가 포함(보건복지부 장관이 고시한 질병군 비급여목록 제외)되어 있으며, \textcolor{red}{질병군 진료기간에 투여한 영양제 또한 질병군 수가에 포함되어 있으므로 별도 산정할 수 없음}\par
\emph{다만, 일상생활에 지장이 없는 단순피로 및 권태에 투여한 경우에는 별도 산정가능합니다}}
\prezi{\clearpage}
\par
\medskip
\Que{제왕절개 분만시 유착방지제 별도 산정 여부}
\Ans{유착방지제는 보건복지부장관이 정하여 고시하는 \textcolor{red}{비급여목록에 해당하지 않으므로 환자에게 별도로 부담시킬 수 없음.}\par 
☞ 요양기관에서 제출한 자료(비급여)에 근거하여 포괄수가에 발생빈도 만큼 포함하여 산출하였음}
\prezi{\clearpage}
\par
\medskip
\Que{칼슘제제 등 건강보조식품 별도 산정 여부}
\Ans{「건강보험요양급여의 기준에 관한 규칙」 별표1 요양급여의 적용기준 및 방법 3.에 따라  '약제는 약사법령에 의하여 허가 또는 신고된 사항(효능ㆍ효과)의 범위 안에서 처방ㆍ투여하여야 한다’고 규정되어 있음. 따라서, 의약품이 아닌 \textcolor{red}{건강보조식품은 건강보험 급여 여부의 논의 대상이 아님}}
\prezi{\clearpage}
\par
\medskip
\Que{질병군 진료기간 중 수면내시경검사를 실시한 경우 별도 산정  여부}
\Ans{수면내시경검사는 보건복지부장관이 별도로 정하여 고시하는 비급여항목에 해당하지 않으므로 환자에게 \textcolor{red}{별도로 부담시킬 수 없음}\par
☞ 요양기관에서 제출한 자료(비급여)에 근거하여 포괄수가에 발생빈도 만큼 포함하여 산출하였음}
\prezi{\clearpage}
\par
\medskip
\Que{질병군(DRG) 입원진료기간 중 MRI 촬영을 실시한 경우 별도 산정 여부}
\Ans{복지부 고시 제2013-180호(‘13.11.27) “MRI 세부산정기준” 따라 질환별 급여대상 및 산정기준에 해당하는 경우 질병군 상대가치점수에 포함되어 별도 산정할 수 없으나, 질환별 급여대상 및 산정기준에 해당하지 않는 경우에는 비급여 대상임}
\prezi{\clearpage}
\par
\medskip
\Que{「자기공명영상유도하 고강도 초음파 집속술(자궁근종)」시 실시 하는 자기공명영상유도비용 별도 산정 여부}
\Ans{「자기공명영상유도하 고강도 초음파 집속술(자궁근종)」을 실시한 경우 ‘복강경을 이용한 기타자궁수술’ \textcolor{blue}{질병군을 적용}하며, \textcolor{red}{자기공명영상유도비용은 MRI 세부산정기준(복지부고시 제2013-180호, ‘13.11.27.) 및 질병군 전문평가위원회 결정사항에 따라 비급여}로 산정할 수 있음}
\prezi{\clearpage}
\par
\medskip
\Que{혈전방지용 압박스타킹, 창상봉합용 액상접착제, 불투명ㆍ투명 드레싱재료 별도 산정 여부}
\Ans{혈전방지용 압박스타킹, 창상봉합용 액상접착제, 불투명ㆍ투명 드레싱재료는 요양기관에서 제출한 자료를 근거로 발생빈도 만큼 포괄수가에 포함되어 있으며 장관이 고시하는 \textcolor{red}{질병군 비급여 목록이 아니므로 별도로 산정 할 수 없음}}
\prezi{\clearpage}
\begin{commentbox}{신의료기술}
요양급여대상 또는 비급여대상으로 결정되지 않은 새로운 행위 및 치료재료로써 \emph{신의료기술등요양급여 결정 신청한 신의료기술인 경우 질병군에서의 적용방법은?}\par
요양급여대상 또는 비급여대상으로 결정ㆍ고시되기 전까지의 신의료기술 등 결정신청 항목은 \textcolor{red}{질병군 포괄수가제에서는 급여대상임} 따라서, 산정방법은 행위의 내용ㆍ성격이 가장 유사한 수술항목을 적용한 후 분류되는 질병군 점수를 청구함.\par
☞ 국민건강보험 요양급여의 기준에 관한 규칙(별표2) 6호
\end{commentbox}
