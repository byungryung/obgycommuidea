\section{폐경기}

\myde{}{%
\begin{itemize}\tightlist
\item[\dsjuridical] N951 폐경 및 여성의 갱년기상태
\item[\dsjuridical] E785 상세불명의 고지혈증
\item[\dschemical] CBC
\item[\dschemical] FSH (폐경 진단시 1회) 
\item[\dschemical] E2(조기 폐경인 경우 )
\item[\dschemical] TG
\item[\dschemical] LDL
\item[\dschemical] HDL( TC=LDL+HDL+TG/5)
\item[\dschemical] OT/PT
\item[\dschemical] TG > 400경우는 TC모두 청구
\item[\dsmedical] 유방4매:G2704(402.69점)
\end{itemize}
}
{
폐경기 및 폐경기 전후 장애에 2-3종 동시 시행한 호르몬검사의 인정기준입니다
폐경기 및 폐경기 전\cntrdot{} 후장애시 난소의 기능과 에스트로겐의 분비는 수년 동안 등락을 거듭하므로 한 시점에서 에스트로겐 측정으로 난소의 기능을 판정하기는 부정확하며, 폐경 이행기에 혈중 난소자극호르몬(FSH)의 증가소견이 일정하게 나타나므로 폐경 진단에 유용한 검사는 나-350 난포자극호르몬(FSH) 검사임. 따라서, 일률적으로 호르몬검사를 2종(나-350 난포자극호르몬 FSH, 나-326 에스트라디올 E2) 혹은 3종(나-350 난포자극호르몬 FSH, 나-326 에스트라디올 E2, 나-348 황체형성호르몬 LH)을 산정한 경우에는 아래와 같이 심사함.
\begin{center}\emph{- 아 래 -}\end{center}
\begin{enumerate}[가.]\tightlist
\item \textcolor{red}{폐경 진단 시에는 난포자극호르몬 검사만 인정하고 조기 폐경인 경우 에스트라디올(E2) 검사를 추가 인정함.}
\item 첫 1회 검사로 진단이 확실치 않은 경우 1회 추가 인정하나, 일반적으로 연령이 만55세 이상인 경우 이미 폐경이 된 상태라고 볼 수 있으므로 합당한 사유가 있는 경우에만 인정함.
\item \emph{황체형성호르몬은 폐경의 진단 및 치료에 영향을 미치지 않으므로} \textcolor{red}{황체형성호르몬(LH) 검사는 인정하지 아니함}
\item \textcolor{red}{폐경 진단 후 호르몬치료 중에 난포자극호르몬 검사는 의미가 없으므로 인정하지 아니함.}
\item 안면홍조 등의 폐경증상이 나아지지 않는 경우 에스트로겐의 수치를 확인하기 위한 \textcolor{red}{에스트라디올(E2) 검사는 사례별로 인정함.} 시행일 : 2011년3월1일 진료분부터 적용
\end{enumerate}
 
폐경기증후군은 여성의 난소 기능 퇴화로 인해 여성호르몬 분비 감소로 발생되는 질환으로, 안면홍조, 우울한 정서, 월경 장애 등과 같은 증상과 함께 골다공증, 심혈관계질환 등을 유발시킬 수 있어 이에 대한 치료로 여성호르몬 요법(Hormone Replacement Therapy, HRT)이 시행됨. 관련 학회에서는 여성호르몬요법 시행 전에 유방촬영술, 간기능검사, 고지혈증검사 등은 기본검사로 실시하여야 한다는 의견을 제시하였고, 교과서 및 임상가이드라인도 여성호르몬요법을 실시 중인 환자는 지속적 추적관찰을 위한 모니터링을 권고하고 있음. 따라서, \emph{폐경기증후군으로 여성호르몬요법(HRT)을 실시하는 모든 환자의 사전 및 추적검사로 유방촬영을 인정함.} [2010.10.18. 진료심사평가위원회
}
\subsection{폐경기 검사 급여기준}
폐경기 및 폐경기 전후 장애에 2~3종 시행한 호르몬검사에 대하여 (4사례)
  
동 건은 폐경기 및 폐경기 전후 장애 상병에 일률적으로 호르몬검사를 2종(난포자극호르몬 FSH, 에스트라디올 E2) 혹은 3종(난포자극호르몬 FSH, 에스트라디올 E2, 황체형성호르몬 LH)을 청구한 사례로
\begin{itemize}\tightlist
\item 교과서 및 관련 자료를 참조 할 때, 갱년기 장애 시 난소의 기능과 에스트로겐의 분비는 수년 동안 등락을 거듭하므로 한 시점에서 에스트로겐 측정은 난소의 기능을 판정하기는 부정확하며, 폐경 이행기에 혈중 FSH의 증가소견이 일정하게 나타나므로 진단에 유용한 검사는 FSH검사임.
\item 따라서 \textcolor{red}{폐경 진단 시에는 난포자극호르몬(FSH)검사만 인정토록 하고 조기 폐경인 경우 에스트라디올(E2)검사를 추가할 수 있으며,} 첫 1회 검사로 진단이 확실치 않은 경우 1회 추가인정가능 하나, 일반적으로 \textcolor{red}{연령이 만 55세 이상인 경우 이미 폐경이 된 상태라고 볼 수 있으므로 합당한 사유가 없으면 인정하기 곤란함.} 황체형성호르몬(LH)은 폐경의 진단 및 치료에 영향을 미치지 않으므로  \textcolor{red}{LH 검사는 인정하지 아니함.}
\item 또한 폐경 진단 후 호르몬치료(HRT) 중에 난포자극호르몬(FSH)검사는 의미가 없으며, 안면홍조 등의 폐경증상이 나아지지 않는 경우 에스트로겐의 수치를 확인하기 위하여 에스트라디올(E2) 검사는 사례별로 인정 가능함. 동 건(4사례)은 상기 내용에 의거하여 볼 때 인정할 만한 사유가 확인되지 않으므로 에스트라디올(E2)검사 및 황체형성호르몬(LH)검사는 인정하지 아니함.[2008.2.18 진료심사평가위원회]
\end{itemize}

\par
\medskip
폐경기 호르몬 검사는 잘 삭감되는 항목으로 아래 폐경기 호르몬 검사의 인정기준을 잘 숙지 하여야 합니다.
\begin{enumerate}\tightlist 
\item 55세 이상은 FSH, E2, LH 인정안됩니다.
\item E2는 조기 폐경시에만 인정합니다.
\item FSH 검사와 HRT 처방이 동시에 나가면 삭감 잘됩니다.따라서 호르몬검사를 먼저하시고 HRT 처방을 나중에
\item E786 상세불명의 지질축적장애 를 상병에 추가 하시면 고지혈증 검사 보험적용 확실
\item 고헐압, 당뇨, 부종에서의 보험청구를 잘 이용하시면 좋습니다.
\end{enumerate}
\medskip

\tabulinesep =_2mm^2mm
\begin {tabu} to\linewidth {|X[1,l]|X[3,l]|} \tabucline[.5pt]{-}
\rowcolor{ForestGreen!40} \centering 제품명/성분명 & \centering 식약처 허가사항 \\ \tabucline[.5pt]{-}
\rowcolor{Yellow!40} 리브론 정 2.5mg &	자연적인 또는 수술에 의한 폐경 이후의 증상(홍조, 야간 발한), 골절되기 쉬운 폐경 이후 골다공증. \\ \tabucline[.5pt]{-}
\rowcolor{Yellow!40} 안젤릭 정 & 1. 폐경 후 일년이 지난 여성의 에스트로겐 결핍증에 대한 호르몬 대체 요법. 2. 골다공증 예방으로 허가 받은 다른 약제에 불내성이거나 금기이고 골절 가능의 위험성이 증가된 폐경 후 여성의 골다공증 예방.\\ \tabucline[.5pt]{-}
\rowcolor{Yellow!40} 크리멘 28 정 &	폐경 후 ( 마지막 생리 후 최소 1 년이 경과된 시점 ) 여성의 에스트로겐 결핍의 증상경감을 위한 호르몬 대체요법 (HRT) \\ \tabucline[.5pt]{-}
\rowcolor{Yellow!40} 프레미나 정 0.3mg 프레미나 정 0.625mg & 성선기능저하증·난소적출·난소기능부전으로 인한 저에스트로겐증, 위축성 질염·외음위축증, 갱년기장애, 기능성 자궁출혈, 골다공증, 폐경 후의 유방암(경감용), 수술불능의 진행성 전립선암(경감용). \\ \tabucline[.5pt]{-}
\rowcolor{Yellow!40} 디비나 정 &	폐경 후 ( 마지막 생리 후 최소 1 년이 경과된 시점 ) 여성의 에스트로겐 결핍의 증상경감을 위한 호르몬 대체요법 (HRT) \\ \tabucline[.5pt]{-}
\rowcolor{Yellow!40} 리비알 정 &	폐경 후 1년이 경과한 여성의 에스트로겐 결핍 증상, 골절위험성이 높은 폐경 이후 여성의 골다공증 예방. \\ \tabucline[.5pt]{-}
\rowcolor{Yellow!40} 에스디올 하프 정 &	폐경후 1년 이상된 여성의 에스트로겐 결핍증상에 대한 호르몬 대체요법, 폐경후 여성의 골다공증 예방. \\ \tabucline[.5pt]{-}
\rowcolor{Yellow!40} 크리안 정 &	자궁절제가 되지 않은 폐경 후(마지막 생리 후 최소 1년이 경과된 시점) 여성의 에스트로겐 결핍의 증상경감을 위한 호르몬대체요법. \\ \tabucline[.5pt]{-}
\rowcolor{Yellow!40} 클리오벨 정 &	폐경 후 1년 이상 된 여성의 에스트로겐 결핍증상에 대한 호르몬 대체요법(다른 골다공증 예방약이 금기이거나 효과가 없는 경우). 골절을 일으키기 쉬운 폐경 후 여성의 골다공증 예방. \\ \tabucline[.5pt]{-}
\rowcolor{Yellow!40} 페모스톤 정 1/10 &	손상되지 않은 자궁을 가진 여성의 자연적 혹은 의인성 폐경 후(마지막 생리 후 최소 1년이 경과된 시점) 에스트로겐 결핍의 증상경감을 위한 호르몬대체요법, 골절의 위험이 있는, 자궁을 가진 폐경 후 여성의 골다공증 예방. \\ \tabucline[.5pt]{-}
\rowcolor{Yellow!40} 페모스톤 정 2/10 &	손상되지 않은 자궁을 가진 여성의 자연적 혹은 의인성 폐경 후(마지막 생리 후 최소 1년이 경과된 시점) 에스트로겐 결핍의 증상경감을 위한 호르몬대체요법, 골절의 위험이 있는, 자궁을 가진 폐경 후 여성의 골다공증 예방. \\ \tabucline[.5pt]{-}
\rowcolor{Yellow!40} 페모스톤 콘티 정 &	손상되지 않은 자궁을 가진 여성의 자연적 혹은 의인성 폐경 후(마지막 생리 후 최소 1년이 경과된 시점) 에스트로겐 결핍의 증상경감을 위한 호르몬대체요법, 골절의 위험이 있는, 자궁을 가진 폐경 후 여성의 골다공증 예방. \\ \tabucline[.5pt]{-}
\rowcolor{Yellow!40} 프레다 정 1mg &	에스트로겐 결핍 증상(갱년기장애). 폐경 후 골다공증. \\ \tabucline[.5pt]{-}
\rowcolor{Yellow!40} 프로기노바 1mg 정	& 갱년기 증상경감을 위한 호르몬대체요법 \\ \tabucline[.5pt]{-}
\rowcolor{Yellow!40} 프로기노바 2mg 정	& 갱년기 증상 치료 위한 호르몬 대체 요법, 골절을 일으키기 쉬운 여성의 골다공증 예방. \\ \tabucline[.5pt]{-}
\end{tabu}

\par
\medskip
\tabulinesep =_2mm^2mm
\begin {tabu} to\linewidth {|X[3,l]|X[6,l]|X[1,l]|X[1,l]|X[4,l]|} \tabucline[.5pt]{-}
\rowcolor{ForestGreen!40} \centering 코드 & \centering 처방명칭 & 수량 & 횟수 & 용법 \\ \tabucline[.5pt]{-}
\rowcolor{Yellow!40} C3500 & 난포자극호르몬 & 1 & 1 & \\ \tabucline[.5pt]{-}
\rowcolor{Yellow!40} C3260 & 에스트라디올 & 1 & 1 & 조기폐경시청구\\ \tabucline[.5pt]{-}
\rowcolor{Yellow!40} C2411 & 총콜레스테롤정량 & 1 & 1 &  \\ \tabucline[.5pt]{-}
\rowcolor{Yellow!40} C2443 & 지질(트리그리세라이드) & 1 & 1 & \\ \tabucline[.5pt]{-}
\rowcolor{Yellow!40} C2420 & HDL콜레스테롤 & 1 & 1 &  \\ \tabucline[.5pt]{-}
\rowcolor{Yellow!40} B2570 & sGOT & 1 & 1 &  \\ \tabucline[.5pt]{-}
\rowcolor{Yellow!40} B2580 & ALT & 1 & 1 & \\ \tabucline[.5pt]{-}
\rowcolor{Yellow!40} B1050 & 백혈구수 & 1 & 1 & \\ \tabucline[.5pt]{-}
\rowcolor{Yellow!40} B1040 & 적혈구수 & 1 & 1 & \\ \tabucline[.5pt]{-}
\rowcolor{Yellow!40} B1060 & 혈소판수 & 1 & 1 & \\ \tabucline[.5pt]{-}
\rowcolor{Yellow!40} B1010 & 혈색수 & 1 & 1 & \\ \tabucline[.5pt]{-}
\rowcolor{Yellow!40} B1020 & 헤마토크리트 & 1 & 1 & \\ \tabucline[.5pt]{-}
\rowcolor{Yellow!40} B1091 & 백혈구백분율(혈액) & 1 & 1 & \\ \tabucline[.5pt]{-}
\rowcolor{Yellow!40} G2701 & 유방1매 & 4 & 1 & N46(추가)\\ \tabucline[.5pt]{-}
\rowcolor{Yellow!40} 미정 & 안티뮐러리안 호르몬 난소능검사 & 1 & 1 & 인정비급여 \\ \tabucline[.5pt]{-}
\rowcolor{Yellow!40} BSONO & 유방초음파검사 & 1 & 1 & 인정비급여 \\ \tabucline[.5pt]{-}
\end{tabu}