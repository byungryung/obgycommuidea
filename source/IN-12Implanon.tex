\section{Implanon의 제거}
\myde{}{
\begin{itemize}\tightlist
\item[\dsjuridical]  M7952 연조직의 잔류 이물- 위팔
\item[\dsmedical]  M0032 피부,피하조직 또는 근육내 이물 제거술 [\myexplfn{439.24}원] 
\end{itemize}
}{
피임은 비급여항목이므로,임플라논 교체는 비급여이지만,\textcolor{red}{임플라논만 제거하는것은 급여대상입니다.(임신을 위한 시술)}
}

기존 피임장치 제거술을 하면 자궁내 장치제거술로 보험 적용이 가능하지만, 임플라논을 제거를 하면 피부 및 피하조직또는 근육내 이물제거술로 보험적용이 가능한지? 아니면 제거술을 환자본인부담금으로 받아도 되는지 궁금합니다

\begin{commentbox}{임플라논 제거 보험여부} 
「피임시술의 요양 급여 인정기준(고시 제2010-45호, ’10.7.1.시행)」에 의하면, 피임시술인 정관절제술 또는 결찰술(자-389-1-라, R3896), 난관결찰술(자-434, R4341-R4345) 및 자궁내장치삽입술(자427, R4271)을 본인이 원하여 실시한 경우에는 비급여 대상이나, 본인이나 배우자가 우생학적 또는 유전학적 정신장애나 신체질환이 있는 경우, 임신으로 모성건강을 악화시킬 수 있는 질환이 있는 경우, 본인이나 배우자가 태아에 미치는 위험성이 높은 전염성질환이 있는 경우에는 요양급여한다고 명시되어 있습니다.\\
또한, 「자궁내 장치 교체시 제거료 산정방법 관련 질의회신(행정해석 보험급여과-2186호, ’08.10.8)」에 의하면, 저출산 문제를 해결하고 출산을 장려하기 위하여 건강보험 지원을 확대하는 정책의 일환으로 \textcolor{blue}{그동안 비급여대상으로 운영하였던 난관, 정관복원수술 및 자궁내장치 제거는 출산을 목적으로 시행한 경우에는 요양급여대상으로 명시}되어 있습니다.\\
따라서, 상기 규정에 비추어 볼 때, \uline{1. 타원에서 임플라논 삽입술을 시행하고 제거술만을 위해 내원한 경우, 2. 출산을 목적으로 제거술을 시행하였다면 요양급여대상으로 적용함이 타당하나, 실제 이루어진 시술에 따라 해당 비용을 산정하여야 할 것입니다.} 참고로 건강보험 행위 급여 비급여 목록 및 상대가치점수 제1편 제2부 제9장 처치및수술료 등 [산정지침]을 안내드리니 업무에 참고하시기 바랍니다.
\end{commentbox}

%\href{http://www.obgydoctor.com/xe/index.php?mid=m_claim_operation&document_srl=1063}{여기}
