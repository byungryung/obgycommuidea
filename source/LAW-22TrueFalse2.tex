\section{검사}
\subsection{급여 인정 범위가 있는 검사}
\tabulinesep =_2mm^2mm
\begin{tabu} to \linewidth {|X[4,l]|X[4,l]|X[4,l]|X[4,l]|} \tabucline[.5pt]{-}
\rowcolor{Gray!25}  보험/비보험 진료& 보험인정 범위  & 급여로 산정 & 비급여로 산정 \\ \tabucline[.5pt]{-}
\rowcolor{Yellow!5} 보험 진료 & 인정범위내 & 합법 & 불법(임의비급여)  \\ \tabucline[.5pt]{-}
\rowcolor{Yellow!5} 보험 진료 & 인정범위외 & 불법(삭감) & 불법(임의비급여) \\ \tabucline[.5pt]{-}
\rowcolor{Yellow!5} 비보험 진료(건강검진) & 인정범위외 & X  & 합법 \\ \tabucline[.5pt]{-}
\end{tabu}
\par
\medskip

\begin{enumerate}[①]\tightlist
\item 급여 진료 시 보험 인정 범위 내의 경우 급여로 해야 한다. (비급여로 하면 임의 비급여로 불법)
\item 급여 진료 시 보험 인정 범위를 벗어나는 경우는 원칙적으로 급여 / 비급여 모두 불법이다.
\item 급여 항목에 대해서 보험 인정 범위를 벗어나느 경우에도 건강검진 목적으로 시행하는 검사는 예외적으로 비급여 산정이 가능하다. 
\end{enumerate}
\begin{commentbox}{}
비보험 진료를 위한 검사의 경우는 비급여로 해도 되는지요?(성형 수술을 위한 수술전 검사를 시행하는 것을 ‘건강 검진 목적’으로 시행했다고 하는 것 보다는 비급여 진료에 사용된 급여 행위/약제/재료는 모두 비급여 진료 행위로 간주하기 때문에 급여 항목을 비급여로 전환했다고 하는 것이 더 논리에 맞지 않나요? 만일 이것이 인정된다면 보험 인정범위 내 검사도 급여 / 비급여로 분리 접수를 해서 비급여 진료에 의한 검사로 비급여 산정을 하면 괜찮지 않나요?)
\par
보험 등재된 검사 또한 비급여 항목에 해당되는 경우(성형수술 등) 비급여입니다
급여 등재된 혈액검사는 기본적으로 급여 검사입니다
환자가 원하여 하는 건강검진 검사의 범주에 해당되는 경우는 비급여입니다
비록 환자가 동의하였다하여 요양급여 대상을 비급여 또는 환자전액 본인부담 (100/100) 으로 해서는 안됩니다
\end{commentbox}

\subsection{비급여 인정 범위가 있는 검사(인정 비급여)}
\tabulinesep =_2mm^2mm
\begin{tabu} to .5\linewidth {|X[4,l]|X[4,l]|} \tabucline[.5pt]{-}
\rowcolor{Gray!25}  보험인정 범위  & 비급여로 산정 \\ \tabucline[.5pt]{-}
\rowcolor{Yellow!5} 인정범위내 & 합법   \\ \tabucline[.5pt]{-}
\rowcolor{Yellow!5} 인정범위외 & 합법  \\ \tabucline[.5pt]{-}
\end{tabu}
\par
\medskip
① 인정 비급여 검사의 경우 그 허가범위를 넘어가도 합법이다.

\subsection{비급여 인정 범위가 없는 검사(비인정 비급여)}
대표적으로 RT PCR 22종등 입니다. 이 검사의 경우는 신의료기술 등재가 되지 않았습니다. 그럼으로 검사하시게 되면 안 되지만 비급여 진료시 비인정 비급여를 한다고 하더라도 특별히 환자의 민원이 있지 않는한 특별한 문제가 있지는 않습니다.\par

\begin{commentbox}{}
인정 비급여 검사는 신의료 기술등재후 요양급여 결정 신청된 경우입니다.
미결정 검사나 행위, 치료재료는 임의비급여입니다.
환자가 원하여 하는 건강검진 검사의 범주에 해당되는 경우등은  비급여입니다
비급여 진료의 경우 따로 상병기입은 안해도 됩니다,, 진료기록에 비급여 진료항목임을 적어두면 됩니다
급여와 비급여 진료가 혼재된 경우 급여에 관련된 상병후 진찰료등 청구가 가능하면 진료기록에 급여 진료에 대한 내용을 소명해두면 됩니다
\end{commentbox}
%\clearpage
