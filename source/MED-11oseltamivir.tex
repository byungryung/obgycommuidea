\section{Oseltamivir 경구제(품명: 타미플루 캅셀 등)}
\Que{12세 아이에게 타미플루를 처방했는데 약값이 조정되었습니다. 급여기준이 뭔가요?}
\Ans{타미플루 캅셀은 인플루엔자(신종인플루엔자 포함)주의보(해외 유입 인플루엔자주의보 포함)가 발표된 이후나 검사에서 인플루엔자(신종인플루엔자 포함) 바이러스 감염이 확인된 경우 고위험군 환자(1세이상 9세이하 소아, 임신부, 65세 이상, 면역저하자 등)에게 초기증상(기침, 두통, 인후통 등 2개 이상의 증상 및 고열)이 발생한지 48시간 이내에 투여시 요양급여가 인정(외래환자)되며 그 외에는 약값 전액을 환자가 부담해야 합니다.}

\begin{commentbox}{Oseltamivir 경구제(품명: 타미플루 캅셀 등)}
허가사항 범위 내에서 아래와 같은 기준으로 투여 시 요양급여를 인정하며, 동 인정기준 이외에는 약값 전액을 환자가 부담토록 함\par
\begin{center} - 아 래 - \end{center}
\begin{enumerate}[가.]\tightlist
\item 인플루엔자(신종인플루엔자 포함)주의보(해외 유입 인플루엔자주의보 포함)가 발표된 이후나 검사에서 인플루엔자(신종인플루엔자 포함) 바이러스 감염이 확인된 경우 다음의 고위험군 환자에게 초기증상(기침, 두통, 인후통 등 2개 이상의 증상 및 고열)이 발생한지 48시간 이내에 투여 시 요양급여를 인정함. 다만, 입원환자는 증상발생 48시간 이후라도 의사가 투약이 필요한 것으로 판단하여 투여한 경우 요양급여를 인정함 \newline
- 다 음 -
	\begin{enumerate}[1)]\tightlist
	\item 1세 이상 9세 이하 소아
	\item 임신부
	\item 65세 이상
	\item 면역저하자
	\item 대사장애(Metabolic disorders)
	\item 심장질환(Cardiac disease)
	\item 폐질환(Pulmonary disease)
	\item 신장기능장애(Renal dysfunction) 등
	\end{enumerate}
\item 조류인플루엔자의 경우
조류인플루엔자주의보가 발표된 이후나 검사상 조류인플루엔자 바이러스 감염이 확인된 경우에는 허가사항 범위 내(치료 및 예방) 투여 시 요양급여를 인정함
 고시 제2016-21호 (2016.02.01. 시행)
\end{enumerate}
\end{commentbox}
