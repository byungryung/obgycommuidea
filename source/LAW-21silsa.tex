\section{현지조사}
\begin{tcolorbox}[frogbox,title=현지조사는 왜 필요한가?(심평원의 시각)]
\begin{itemize}\tightlist
\item 과거 건강보험 시행前 → 최상의 진료가 의료계 목표, 재정의 제한없음
\item 건강보험시대 → 최상의 진료보다는 재정의 한계로 인한 보편적 진료를 요구 → 국민이익(돈이 없어 죽는 사람 無) → 보편적 진료를 위한 가이드라인 설정(급여기준)
\item 건보시대에는 진료만 생각했던 과거 시절에 비해 기준 등 가이드라인 인지 및 준수를 위한 절차적 행정이 상당부분 필요해짐 → 건강보험을 하는 한 이젠 \highlight{행정에 시간을 투자할 때}
\item 보편적 진료 가이드라인 준수여부에 대한 확인 검증절차가 정책적으로 필요 → 현지조사 태동의 논거 → 따라서, 현지조사는 재정의 제한성 속에서 불가피한 정책적 제도로 기능
\item 건강보험제도하에서 현지조사는 회피하거나 불편해할 것이 아니라 슬기롭고 적극적으로 극복해 나가야 할 과제(의무)임
\end{itemize}
\end{tcolorbox}

\section{진료지표와 실사의 이해및 실제}
\subsection{자율지표연동관리제 확인법}
\begin{figure}
\centering
%\begin{subfigure}{.5\textwidth}
  \centering
  \includegraphics[width=.45\linewidth]{Dx_In01}
%\end{subfigure}%
%\begin{subfigure}{.5\textwidth}
  \centering
  \includegraphics[width=.45\linewidth]{Dx_In02}
%\end{subfigure}
\end{figure}

\begin{figure}
\centering
%\begin{subfigure}{.5\textwidth}
  \centering
  \includegraphics[width=.45\linewidth]{Dx_In03}
%\end{subfigure}%
%\begin{subfigure}{.5\textwidth}
  \centering
  \includegraphics[width=.45\linewidth]{Dx_In04}
%\end{subfigure}
\end{figure}
\leftrod{자율지표 결과해석} 
\begin{itemize}\tightlist
\item 절대지표 : 개별 요양기관 내의 단순 평균값 
\item 상대지표 : 비교그룹의 전국평균과 상대비교한 수치 (ECI, DCI, LI, VI, CMI)
\item 상대지표에서 1이 넘으면 무조건 전국평균보다 높다는 얘기 입니다.
\end{itemize}
\par
\begin{center}%ing
  \includegraphics[width=.45\linewidth]{Tx_In05}
\end{center}%par
\begin{commentbox}{진료지표안내}
 국민 건강증진과 건강보험제도 발전을 위해 항상 협조해 주시는 귀 원에 감사드립니다.\par
 우리원에서는 의료기관의 진료비와 진료내역을 분석하여 안내하는 지표연동자율개선제를 시행하고 있습니다. 이는 의료의 질과 진료비 부담에 영향이 큰 항목에 대한 의료기관별 자료를 제공하여 자율적으로 개선토록 하는 사업으로, 지속적인 모니터링을 통해 적정성 평가와 현지조사 등과 연계하고 있습니다.\par
  - 대상항목은 내원일수, 항생제처방률, 주사제처방률, 약품목수, 외래처방약품비 등으로 각 항목별로 관리지표가 동일 평가군에서 높은 기관에 자료를 제공하고 있습니다. \par 
 귀 원의 분석된 항목에 대한 진료지료를 알려드리니 적극적인 관심과 협조 부탁드립니다.\par
→ 보건복지부의 “자율시정통보제”와 심평원의 “지표연동관리제”가”지표연동관리제”위주로  통합돼 2014년  8월말부터 “지표연동 자율개선제”로 통합 시행 중 
\end{commentbox}
\subsection{고가도 지표}
\begin{figure}
\centering
%\begin{subfigure}{.5\textwidth}
  \centering
  \includegraphics[width=.45\linewidth]{ECI}
%\end{subfigure}%
%\begin{subfigure}{.5\textwidth}
  \centering
  \includegraphics[width=.45\linewidth]{DCI}
%\end{subfigure}
\end{figure}
\leftrod{ECI \& DCI}
\begin{itemize}\tightlist
\item 우리병원이 다른 병원[비교군]평균과 비교해서 진료군별 건당 얼마나 높은지 비교
\item 지표의 헛점
	\begin{enumerate}[1.]\tightlist
	\item 질병군별 건당 진료비라는 점!!!! N760 K297 로  STD 6 종 검사 를 하였다면 K297을 1번 상병[주상병] 으로
	\item 이왕이면 건당 진료비가 높을 것 같은 상병 입력 생리통 보다는자궁근종을 1번 상병으로 타과 상병을 주상병
	\item ***모든 지표는 모두 비교군과[남들과의] 비교로 나오게 되므로 모든 병원이 전체적으로 올라가면 만사형통
	\end{enumerate}
\end{itemize}
\leftrod{VI}
\begin{itemize}\tightlist
\item 우리병원이 다른 병원[비교군] 평균과 비교해서 건당 내원일수를 비교하는 것임
\end{itemize}
\leftrod{CMI}
\begin{itemize}\tightlist
\item 대상기관의 환자구성의 중증도를 반영하는 지표
\item CMI 값이 1.2인 경우 대상기관의 환자구성이 동일 표시과목 그룹과 비교시 환자 구성 중증도가 20\%정도 높게 구성
\end{itemize}

\begin{figure}
\centering
%\begin{subfigure}{.5\textwidth}
  \centering
  \includegraphics[width=.45\linewidth]{VI}
  %\caption{우리병원이 다른 병원[비교군] 평균과 비교해서 건당 내원일수를 비교하는 것임}
%\end{subfigure}%
%\begin{subfigure}{.5\textwidth}
  \centering
  \includegraphics[width=.45\linewidth]{CMI}
  %\caption{대상기관의 환자구성의 중증도를 반영하는 지표}
%\end{subfigure}
\end{figure}



