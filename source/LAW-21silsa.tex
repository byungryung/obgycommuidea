\section{현지조사}
\begin{tcolorbox}[frogbox,title=현지조사는 왜 필요한가?(심평원의 시각)]
\begin{itemize}\tightlist
\item 과거 건강보험 시행前 → 최상의 진료가 의료계 목표, 재정의 제한없음
\item 건강보험시대 → 최상의 진료보다는 재정의 한계로 인한 보편적 진료를 요구 → 국민이익(돈이 없어 죽는 사람 無) → 보편적 진료를 위한 가이드라인 설정(급여기준)
\item 건보시대에는 진료만 생각했던 과거 시절에 비해 기준 등 가이드라인 인지 및 준수를 위한 절차적 행정이 상당부분 필요해짐 → 건강보험을 하는 한 이젠 \highlight{행정에 시간을 투자할 때}
\item 보편적 진료 가이드라인 준수여부에 대한 확인 검증절차가 정책적으로 필요 → 현지조사 태동의 논거 → 따라서, 현지조사는 재정의 제한성 속에서 불가피한 정책적 제도로 기능
\item 건강보험제도하에서 현지조사는 회피하거나 불편해할 것이 아니라 슬기롭고 적극적으로 극복해 나가야 할 과제(의무)임
\end{itemize}
\end{tcolorbox}

\section{진료지표와 실사의 이해및 실제}
\subsection{자율지표연동관리제 확인법}
\begin{figure}
\centering
%\begin{subfigure}{.5\textwidth}
  \centering
  \includegraphics[width=.45\linewidth]{Dx_In01}
%\end{subfigure}%
%\begin{subfigure}{.5\textwidth}
  \centering
  \includegraphics[width=.45\linewidth]{Dx_In02}
%\end{subfigure}
\end{figure}

\begin{figure}
\centering
%\begin{subfigure}{.5\textwidth}
  \centering
  \includegraphics[width=.45\linewidth]{Dx_In03}
%\end{subfigure}%
%\begin{subfigure}{.5\textwidth}
  \centering
  \includegraphics[width=.45\linewidth]{Dx_In04}
%\end{subfigure}
\end{figure}
\leftrod{자율지표 결과해석} 
\begin{itemize}\tightlist
\item 절대지표 : 개별 요양기관 내의 단순 평균값 
\item 상대지표 : 비교그룹의 전국평균과 상대비교한 수치 (ECI, DCI, LI, VI, CMI)
\item 상대지표에서 1이 넘으면 무조건 전국평균보다 높다는 얘기 입니다.
\end{itemize}
\par
\begin{center}%ing
  \includegraphics[width=.45\linewidth]{Tx_In05}
\end{center}%par
\begin{commentbox}{진료지표안내}
 국민 건강증진과 건강보험제도 발전을 위해 항상 협조해 주시는 귀 원에 감사드립니다.\par
 우리원에서는 의료기관의 진료비와 진료내역을 분석하여 안내하는 지표연동자율개선제를 시행하고 있습니다. 이는 의료의 질과 진료비 부담에 영향이 큰 항목에 대한 의료기관별 자료를 제공하여 자율적으로 개선토록 하는 사업으로, 지속적인 모니터링을 통해 적정성 평가와 현지조사 등과 연계하고 있습니다.\par
  - 대상항목은 내원일수, 항생제처방률, 주사제처방률, 약품목수, 외래처방약품비 등으로 각 항목별로 관리지표가 동일 평가군에서 높은 기관에 자료를 제공하고 있습니다. \par 
 귀 원의 분석된 항목에 대한 진료지료를 알려드리니 적극적인 관심과 협조 부탁드립니다.\par
→ 보건복지부의 “자율시정통보제”와 심평원의 “지표연동관리제”가”지표연동관리제”위주로  통합돼 2014년  8월말부터 “지표연동 자율개선제”로 통합 시행 중 
\end{commentbox}
\subsection{고가도 지표}
\begin{figure}
\centering
%\begin{subfigure}{.5\textwidth}
  \centering
  \includegraphics[width=.45\linewidth]{ECI}
%\end{subfigure}%
%\begin{subfigure}{.5\textwidth}
  \centering
  \includegraphics[width=.45\linewidth]{DCI}
%\end{subfigure}
\end{figure}
\leftrod{ECI \& DCI}
\begin{itemize}\tightlist
\item 우리병원이 다른 병원[비교군]평균과 비교해서 진료군별 건당 얼마나 높은지 비교
\item 지표의 헛점
	\begin{enumerate}[1.]\tightlist
	\item 질병군별 건당 진료비라는 점!!!! N760 K297 로  STD 6 종 검사 를 하였다면 K297을 1번 상병[주상병] 으로
	\item 이왕이면 건당 진료비가 높을 것 같은 상병 입력 생리통 보다는자궁근종을 1번 상병으로 타과 상병을 주상병
	\item ***모든 지표는 모두 비교군과[남들과의] 비교로 나오게 되므로 모든 병원이 전체적으로 올라가면 만사형통
	\end{enumerate}
\end{itemize}
\leftrod{VI}
\begin{itemize}\tightlist
\item 우리병원이 다른 병원[비교군] 평균과 비교해서 건당 내원일수를 비교하는 것임
\end{itemize}
\leftrod{CMI}
\begin{itemize}\tightlist
\item 대상기관의 환자구성의 중증도를 반영하는 지표
\item CMI 값이 1.2인 경우 대상기관의 환자구성이 동일 표시과목 그룹과 비교시 환자 구성 중증도가 20\%정도 높게 구성
\end{itemize}

\begin{figure}
\centering
%\begin{subfigure}{.5\textwidth}
  \centering
  \includegraphics[width=.45\linewidth]{VI}
  %\caption{우리병원이 다른 병원[비교군] 평균과 비교해서 건당 내원일수를 비교하는 것임}
%\end{subfigure}%
%\begin{subfigure}{.5\textwidth}
  \centering
  \includegraphics[width=.45\linewidth]{CMI}
  %\caption{대상기관의 환자구성의 중증도를 반영하는 지표}
%\end{subfigure}
\end{figure}
\subsection{관리항목별 선정기준 예시}
\tabulinesep =_2mm^2mm
\begin{tabu} to \linewidth {|X[4,l]|X[6,l]} \tabucline[.5pt]{-}
\rowcolor{Gray!25}  관리항목  & 선정기준 \\ \tabucline[.5pt]{-}
\rowcolor{Yellow!5} 내원일수 & 내원일수지표내원일수지표(VI) 1.1 이상 \& 건강진료비고가도지표(CI) 1.0 이상 \& 개설기관 상위 15\%  \\ \tabucline[.5pt]{-}
\rowcolor{Yellow!5} 급성상기도감염 항생제 처방률 & 항생체 처방률 80\% 이상 기관 \\ \tabucline[.5pt]{-}
\rowcolor{Yellow!5} 주사제 처방률 & 주사제 처방률 60\% 이상 기관 \\ \tabucline[.5pt]{-}
\rowcolor{Yellow!5} 약품목수 & 6품목 이상 처방비율 40\% 이상 기관 \\ \tabucline[.5pt]{-}
\rowcolor{Yellow!5} 외래처방약품비 & 외래처방약품비고가도지표(OPCI) 1.3 이상 기관 \\ \tabucline[.5pt]{-}
\end{tabu}
\par
\medskip

\tabulinesep =_2mm^2mm
\begin{tabu} to \linewidth {|X[1,l]|X[1,c]|X[1,c]|X[1,c]|X[1,c]|X[1,c]|} \tabucline[.5pt]{-}
\rowcolor{Gray!25}  종별  & 내원일수 & 급성상기도감염 항생제처방률 & 주사제 처방률 & 약품목수 & 외래처방 약품비 \\ \tabucline[.5pt]{-}
\rowcolor{Yellow!5} 의원 & \bullet & \bullet & \bullet & \bullet & \bullet  \\ \tabucline[.5pt]{-}
\rowcolor{Yellow!5} 치과의원 & \bullet  &  &  &  &   \\ \tabucline[.5pt]{-}
\rowcolor{Yellow!5} 한의원 & \bullet &  &  &  &   \\ \tabucline[.5pt]{-}
\rowcolor{Yellow!5} 병원 & \bullet &  & \bullet & \bullet &   \\ \tabucline[.5pt]{-}
\rowcolor{Yellow!5} 요양병원 & \bullet & \bullet & \bullet & \bullet &   \\ \tabucline[.5pt]{-}
\rowcolor{Yellow!5} 치과병원 & \bullet &  &  &  &   \\ \tabucline[.5pt]{-}
\end{tabu}
\par
\medskip
\begin{itemize}\tightlist
\item 치과병원, 한방병원 : 종별가산율 20% 해당(이외는 2012년부터) 
\item 종합병원 이상 : 2012년부터 통보 
\end{itemize}
	
\subsection{실사 가능하게 하는 관련법규}
\leftrod{요양급여 대상 여부의 확인(국민건강보험법 제 48조)}
\begin{enumerate}[①]\tightlist
\item 가입자나 피부양자는 본인일부부담금 외에 자신이 부담한 비용이 제41조제3항에 따라 요양급여 대상에서 제외되는 비용인지 여부에 대하여 심사평가원에 확인을 요청할 수 있다. → 병원에서 부당하게 받았는지 확인을 할수 있다. → \highlightR{민원 발생의 요인}
	\begin{itemize}\tightlist
	\item 환자가 자기의 진료비를 확인할 수 있는 제도 → 이런 민원으로 인해 실사가 증가합니다.
	\item 해결책은 진료 현장에서 납득을 시켜야 한다.
	\item 똑같은 사안에서 민원인에게는 YES, 병원에는 No 하는 경우가 종종 있어 이중잣대 논란
	\end{itemize}
\item 제1항에 따른 확인 요청을 받은 심사평가원은 그 결과를 요청한 사람에게 알려야 한다. 이 경우 확인을 요청한 비용이 요양급여 대상에 해당되는 비용으로 확인되면 그 내용을 공단 및 관련 요양기관에 알려야 한다. → \highlightR{급여대상인지 비급여 대상인지 확인후 알려준다}
\item 제2항 후단에 따라 통보받은 요양기관은 받아야 할 금액보다 더 많이 징수한 금액(이하 "과다본인부담금"이라 한다)을 지체 없이 확인을 요청한 사람에게 지급하여야 한다. 다만, 공단은 해당 요양기관이 과다본인부담금을 지급하지 아니하면 해당 요양기관에 지급할 요양급여비용에서 과다본인부담금을 공제하여 확인을 요청한 사람에게 지급할 수 있다. → \highlightR{급여를 비급여로 받는등 과다 본인 부담금을 받으면 비용을 뱉어내거나 공단이 병의원에 뺏어서 지불할 수 있다.}
\end{enumerate}

\leftrod{현지조사 법적 근거(건강보험법)}
\begin{description}\tightlist
\item[현지조사] 법 제97조 제2항 : 요양기관에 대한 보건복지부장관의 현지조사 권한 및 자료제출 명
\item[형정처분] 법 제98조, 법 제99조 : 요양기관 업무정지 처분 또는 과징금 처분
	\begin{itemize}\tightlist
	\item 요양급여비용 거짓ㆍ부당청구 한 때
	\item 서류미제출, 조사거부ㆍ방해 또는 기피 한 때
	\end{itemize}
\item[형사고발] 법 제115조 제3항 4호 : 업무정지기간 중 요양급여를 한 경우 1년이하 징역 또는 1천만원 이하 벌금
\item[형사고발] 법 제116조 : 서류 미제출, 거짓보고, 거짓서류제출 및 조사 거부, 방해, 기피한 경우 1천만원 이하 벌금
\end{description}

\subsection{실사의 유형}
\begin{enumerate}[가.]\tightlist
\item 정기조사
	\begin{itemize}\tightlist
	\item (지표점검기관) 자율시정통보 미시정 기관, 부당청구상시감지시스템(데이터 마이닝), 본인부담금과다징수 다발생 기관 등에 의해 부당청구 개연성이 높다고 판단되는 기관에 대해 실시하는 통상적 조사
	\item (외부의뢰기관) 공단 및 심평원의 급여사후관리 혹은 민원제보 및 타 행정기관의 수사 등의 과정에서 요양급여 비용의 부당청구가 확인 혹은 인지되어 보험급여내역전반에 대해 행정조사를 실시할 필요가 있다고 판단되는 기관에 대해 실시하는 조사
	\end{itemize}
\item 기획조사
	\begin{itemize}\tightlist
	\item 건강보험제도 운영상 또는 사회적 문제가 제기된 분야에 대해 제도개선 및 올바른 청구문화 정착을 도모하기 위해 실시하는 조사 \par
- 조사 공정성$\cdot$객관성 제고를 위해 민간전문가가 포함된 ‘기획조사항목선정협의회’를 통해 조사대상 분야 및 기준 등 심의\par
※ 기획조사 실시 전 조사 분야 및 조사 시기 사전 예고
	\item 2015년 기획실사 조사항목 :  
		\begin{enumerate}[①]\tightlist
		\item 건강보험 기획 현지조사 항목은 : 진료비 이중청구 의심기관
		\end{enumerate}
		\begin{enumerate}[①]\tightlist
		\item 의료급여 기획 현지조사 항목은 : 의료급여 사회복지시설 수급권자 청구기관,상반기 
		\item 의료급여 장기입원 청구기관 조사, 하반기
		\end{enumerate}
	\end{itemize}
\item 긴급 조사
	\begin{itemize}\tightlist
	\item 허위$\cdot$부당청구 개연성이 높은 요양기관이 증거인멸$\cdot$폐업 등 우려가 있거나, 사회적 문제가 야기된 분야 등으로 긴급한 조사가 필요한 경우에 실시하는 조사
	\end{itemize}
\item 이행실태 조사
	\begin{itemize}\tightlist
	\item 건강보험 업무정지처분 기간 중 당해 처분을 편법적으로 회피할 우려가 높은 기관 혹은 불이행이 의심되는 양기관 등에 대해 처분의 사후 이행 여부를 점검하기 위해 실시하는 조사
	\end{itemize}
\end{enumerate}

\subsection{실사[현지조사]의 현황}
\begin{commentbox}{복지부 실사담당 사무관의 증언 中에서}
우리나라에 의료기관(병원.의원.한의원.약국 포함) 숫자가 8만 5000개정도이며, 1년에 실사가 900건~1000건 정도 나간다는데 그중 700-850군데가 소위 말하는 일반적인 실사이고 50-150군데 정도가 "기획실사"랍니다. \par 					
\end{commentbox}
2001년도(250개소) \par
2008년도(1018개소) \par
2009년도(954개소) \par
2010년도(920개소) \par
2011년도(1003개소) \par
2012년도(684개소) \par
2013년도(931개소)\par

\subsection{실사의 원인}
\leftrod{실사조사유형의 분류}%\par
\tabulinesep =_2mm^2mm
\begin{tabu} to .75\linewidth {|X[1,l]|X[6,l]} \tabucline[.5pt]{-}
\rowcolor{Gray!25}  순위  & 유형 \\ \tabucline[.5pt]{-}
\rowcolor{Yellow!5} 1위 & 잦은 민원 \\ \tabucline[.5pt]{-}
\rowcolor{Yellow!5} 2위 & 내부 고발자 \\ \tabucline[.5pt]{-}
\rowcolor{Yellow!5} 3위 & 기타(기획조사, 테이터 마이닝, 자율시정 불응기관) \\ \tabucline[.5pt]{-}
\end{tabu}

\begin{itemize}\tightlist
\item 공단 의뢰기기관 
\item 심평원 의뢰기관 
\item 민원제보기관 (검.경찰 권익위, 국민신고마당)
\item 자율시정 불응기관 :일년에 3-4000개 통보 기관중 40-50개 현지조사
\item 데이터 마이닝(부당감지지표)에 의한 기관 
\end{itemize}
\leftrod{데이터 마이닝}
\begin{enumerate}[①]\tightlist
\item 야간․공휴일 진찰료 청구율 
\item 진찰료 단독 청구빈도
\item 원외처방전 미발행률 
\item 초진료 청구빈도
\item 수진자당 보유상병 개수 
\item 처방투약 일수별 처방건수
\item 진찰료 및 검사료 단독 청구 빈도
\item 정신과 환자당 내원일수 및 정신요법료 청구 횟수
\item 이학요법료 항목별 청구빈도 
\item 처방전 2개소 이상 중복 청구건수
\item 약국 동일처방전 중복청구내역 
\item 처방전 건수 불일치건 발생빈도
\item 처방전 집중률과 고가약 처방빈도 
\item 진료지표 급등기관 
\item 요양급여비용 지연청구기관
\end{enumerate}

\subsection{현지조사 대상기관 선정}
\leftrod{\textbf{심평원 의뢰기관}}
\begin{itemize}\tightlist
\item 요양급여비용에 대한 심사 및 평가, 건강보험 재정지킴이 신고 등
 (위에서 부당청구의 개연성이 높게 나타난 기관)
\item 부당청구감지시스템을 통하여 부당청구 개연성이 높은 요양기관
\item 정당한 사유 없이 2회 이상 자료제출 거부하여 부당사실관계 확인 곤란한 기관
\end{itemize}
\textbf{업무 절차}\par
\menu{현지조사의뢰(본ㆍ지원) > 실익검토(조사운영부) > 보건복지부로 현지조사 의뢰}\par
\menu{ > ‘현지조사 선정 심의위원회’심의 > 현지조사 대상선정}\par
☞ ‘선정심의위원회’ 구성(12명) : 공공위원(3명), 의약단체(5명), 시민단체(1명), 법조계 등(3명)\par \medskip

\leftrod{\textbf{공단 의뢰기관 선정기준}}
\begin{itemize}\tightlist
\item 진료내역 통보 및 수진자조회 등 과정에서 부당청구 개연성이 높게 나타난 기관
\item 요양기관 내부종사자 등에 의해 신고된 요양기관
\item 정당한 사유 없이 2회 이상 자료제출 거부하여 부당사실관계 확인 곤란한 기관
\end{itemize}
\textbf{업무 절차}\par
\menu{현지조사의뢰 공단→복지부 > 실익검토요청 복지부→심평원 > 실익검토보고 심평원→복지부 }\par 
\menu{>‘현지조사 선정심의위원회’심의 > 현지조사 대상선정}\par 


※ 공단 지사 → 공단지역본부 → 공단 본부(급여관리실) → 보건복지부\par \medskip

\leftrod{\textbf{내부종사자 고발}}%\par
부당청구 요양기관 신고포상금제 시행 10년\cntrdots{}지급된 포상금 40억
\begin{itemize}\tightlist
\item 포상금제 시행 10년
\item 지급된 포상금 40억
\item 최고 10억 원 지급
\end{itemize}

\begin{commentbox}{내부종사자 공익신고기관 포상금제도}
법 제104조 (포상금지급)\par
공단은 속임수나 그 밖의 부당한 방법으로 보험급여 비용을 지급받은 요양기관을 신고한 사람에 대하여 대통령령에 따라 포상금을 지급
\end{commentbox}

\leftrod{\textbf{대외 의뢰 기관}}%\par
검ㆍ경찰ㆍ감사원ㆍ관련행정부처ㆍ지방자치단체ㆍ국민권익위원회 등 으로부터 건강보험 부당청구 등 혐의로 현지조사 의뢰된 기관으로 부당청구 개연성이 높은 요양기관 \par

\leftrod{\textbf{민원제보 기관}}%\par
요양기관 거짓ㆍ부당청구에 대해 제보된 기관 중 실명의 제보자가 구체적인 사례와 증거를 제시하는 등 부당청구 개연성이 높은 기관 \par

\leftrod{\textbf{본인부담과다징수 다발생 기관}}%\par
요양급여대상여부 확인 등에 의한 본인부담과다 징수 다발생 기관 중 \par
\begin{itemize}\tightlist
\item 매 분기별 발생건수, 부당건수, 부당금액 등 자율시정 정도 등을 검토하여 현지조사가 필요하다고 판단되는 기관
\item ※ 예: 2010년, 2012년 상급종합병원 전수 기획조사 실시
\end{itemize}

\subsection{2017년 기획조사 항목}
건강보험 분야[’16. 12. 21. 사전예고]
\begin{itemize}\tightlist
\item 본인부담금 과다징수 의심기관 [상반기, 하반기] - 상급종합병원 43개소 전수조사
\end{itemize}
의료급여 분야[’16. 12. 21. 사전예고]
\begin{itemize}\tightlist
\item 의료급여 장기입원 청구기관[상반기]
\item 선택의료급여기관에서 의뢰된 진료 다발생 청구기관[하반기]
\end{itemize}

\subsection{실사 피하기 – 진료지표 관리하기}
\begin{enumerate}[해결책1.]\tightlist
\item 복합 상병시 건당진료비가 높을거 같은 상병을 맨 위에 기재
:질염(N761~N768)보다는 골반염(N730~N739)또는 자궁근종(D250~D259) 
\item 다빈도 상병을 보조상병으로 입력: K30, K297, M5496, R1039등등
\item  진료비가 높은 경우 타과 상병을 입력 
 :동일 과에 대한 다빈도 상병을 기준으로 통계를 냄: 질염(N761~N768)보다는 고혈압(I10), 당뇨(E149)을 위의 상병으로 넣을것.
\item 이왕이면 엄청난 상병을 입력
: 정확한 코딩을 위하여서는 환자의 상태에 대한 상병(코드)을 적어야 하지만 실상 
	심평원에서는 환자의 상태는 별개로 상병(코드)과  처치 내역을 보고 심사를 
	하고 있으므로 광역의  상병 또는 중환 상병이 필요(난소의 양성 신생물(D279)
	보다는 난소의 악성신생물(C569)소신진료후 진료기록포함 증빙자료 작성하여
	보존
\end{enumerate}
\begin{commentbox}{}
\begin{itemize}\tightlist 
\item 한마디로 비쌀거 같은 상병을 주상병으로 
\item 상병은 겁나 많이 넣고 
\item 내과 상병  주상병으로 
\item 엄청난 상병 주상병으로..
\end{itemize}
\end{commentbox}

\subsection{실사 피하기 – 환자 납득시키기}
납득이 되게 영수증 발행하기
\subsection{실사 피하기 – 직원 만족}
퇴사시 따듯한 밥 한그릇이라도….

\subsection{실사의 이해및 실제}
\subsection{행정조사기본법}
제1조(목적)
이 법은 행정조사에 관한 기본원칙·행정조사의 방법 및 절차 등에 관한 공통적인 사항을 규정함으로써 행정의 공정성·투명성 및 효율성을 높이고, \highlightR{국민의 권익을 보호함을 목적}으로 한다.
[[시행일 2007.8.18]]

제4조(행정조사의 기본원칙)
\begin{enumerate}[①]\tightlist
\item 행정조사는 \highlightY{조사목적을 달성하는데 필요한 최소한의 범위 안에서 실시하여야 하며}, 다른 목적 등을 위하여 \highlightY{조사권을 남용하여서는 아니 된다.}
\item 행정기관은 조사목적에 적합하도록 조사대상자를 선정하여 행정조사를 실시하여야 한다.
\item 행정기관은 \highlightY{유사하거나 동일한 사안에 대하여는 공동조사 등을 실시함으로써} \highlightR{행정조사가 중복되지 아니하도록} 하여야 한다.
\item 행정조사는 \highlightY{법령등의 위반에 대한 처벌보다는} \highlightR{법령등을 준수하도록 유도하는 데 중점}을 두어야 한다.
\item 다른 법률에 따르지 아니하고는 행정조사의 대상자 또는 행정조사의 내용을 공표하거나 \highlightY{직무상 알게 된 비밀을 누설하여서는 아니된다.}
\item 행정기관은 행정조사를 통하여 알게 된 정보를 다른 법률에 따라 내부에서 이용하거나 다른 기관에 제공하는 경우를 제외하고는 원래의 조사목적 이외의 용도로 이용하거나 타인에게 제공하여서는 아니 된다.
\end{enumerate}

제11조(현장조사)
\begin{enumerate}[①]\tightlist
\item 조사원이 가택·사무실 또는 사업장 등에 출입하여 현장조사를 실시하는 경우에는 행정기관의 장은 다음 각 호의 사항이 기재된 현장출입조사서 또는 법령등에서 현장조사시 제시하도록 규정하고 있는 문서를 조사대상자에게 발송하여야 한다.
	\begin{enumerate}[1.]\tightlist
	\item \highlight{조사목적}
	\item \highlight{조사기간과 장소}
	\item \highlight{조사원의 성명과 직위}
	\item \highlight{조사범위와 내용}
	\item \highlight{제출자료}
	\item \highlight{조사거부에 대한 제재(근거 법령 및 조항 포함)}
	\item \highlight{그 밖에 당해 행정조사와 관련하여 필요한 사항}
	\end{enumerate}
\item 제1항에 따른 \highlightR{현장조사는 해가 뜨기 전이나 해가 진 뒤에는 할 수 없다.} 다만, 다음 각 호의 어느 하나에 해당하는 경우에는 그러하지 아니하다.
  1. 조사대상자(대리인 및 관리책임이 있는 자를 포함한다)가 동의한 경우
  2. 사무실 또는 사업장 등의 업무시간에 행정조사를 실시하는 경우
  3. 해가 뜬 후부터 해가 지기 전까지 행정조사를 실시하는 경우에는 조사목적의 달성이 불가능하거나 증거인멸로 인하여 조사대상자의 법령등의 위반 여부를 확인할 수 없는 경우
\item 제1항 및 제2항에 따라 현장조사를 하는 조사원은 그 권한을 나타내는 증표를 지니고 이를 조사대상자에게 내보여야 한다.
\end{enumerate}

제14조(공동조사)
\begin{enumerate}[①]\tightlist
\item 행정기관의 장은 \highlightY{다음 각 호의 어느 하나에 해당하는 행정조사를 하는 경우에는 공동조사를 하여야 한다.}
	\begin{enumerate}[1.]\tightlist
	\item 당해 행정기관 내의 2 이상의 부서가 \highlightR{동일하거나 유사한 업무분야에 대하여 동일한 조사대상자에게 행정조사를 실시하는 경우}
	\item 서로 다른 행정기관이 대통령령으로 정하는 분야에 대하여 동일한 조사대상자에게 행정조사를 실시하는 경우
	\end{enumerate}
\item 제1항 각 호에 따른 사항에 대하여 \highlightY{행정조사의 사전통지를 받은 조사대상자는 관계 행정기관의 장에게 공동조사를 실시하여 줄 것을 신청할 수 있다.} 이 경우 조사대상자는 신청인의 성명·조사일시·신청이유 등이 기재된 공동조사신청서를 관계 행정기관의 장에게 제출하여야 한다.
\item 제2항에 따라 \highlightY{공동조사를 요청받은 행정기관의 장은 이에 응하여야 한다.}
\item 국무조정실장은 행정기관의 장이 제6조에 따라 제출한 행정조사운영계획의 내용을 검토한 후 관계 부처의 장에게 공동조사의 실시를 요청할 수 있다. [개정 2008.2.29 제8852호(정부조직법), 2013.3.23 제11690호(정부조직법)]
\item 그 밖에 공동조사에 관하여 필요한 사항은 대통령령으로 정한다.
\end{enumerate}

제15조(중복조사의 제한)
\begin{enumerate}[①]\tightlist
\item 제7조에 따라 정기조사 또는 수시조사를 실시한 행정기관의 장은 \highlightR{동일한 사안에 대하여 동일한 조사대상자를 재조사 하여서는 아니 된다.} 다만, 당해 행정기관이 이미 조사를 받은 조사대상자에 대하여 위법행위가 의심되는 새로운 증거를 확보한 경우에는 그러하지 아니하다.
\item 행정조사를 실시할 행정기관의 장은 행정조사를 실시하기 전에 \highlightR{다른 행정기관에서 동일한 조사대상자에게 동일하거나 유사한 사안에 대하여 행정조사를 실시하였는지 여부를 확인할 수 있다.}
\item 행정조사를 실시할 행정기관의 장이 제2항에 따른 사실을 확인하기 위하여 행정조사의 결과에 대한 자료를 요청하는 경우 요청받은 행정기관의 장은 특별한 사유가 없는 한 관련 자료를 제공하여야 한다.
\end{enumerate}

제17조(조사의 사전통지)
\begin{enumerate}[①]\tightlist
\item 행정조사를 실시하고자 하는 행정기관의 장은 제9조에 따른 출석요구서, 제10조에 따른 보고요구서·자료제출요구서 및 제11조에 따른 현장출입조사서(이하 "출석요구서등"이라 한다)를 \highlightR{조사개시 7일 전까지 조사대상자에게 서면으로 통지하여야 한다.} 다만, 다음 각 호의 어느 하나에 해당하는 경우에는 행정조사의 개시와 동시에 출석요구서등을 조사대상자에게 제시하거나 행정조사의 목적 등을 조사대상자에게 구두로 통지할 수 있다.
	\begin{enumerate}[1.]\tightlist
	\item 행정조사를 실시하기 전에 관련 사항을 미리 통지하는 때에는 증거인멸 등으로 행정조사의 목적을 달성할 수 없다고 판단되는 경우
	\item 「통계법」  제3조제2호에 따른 지정통계의 작성을 위하여 조사하는 경우
	\item 제5조 단서에 따라 조사대상자의 자발적인 협조를 얻어 실시하는 행정조사의 경우
	\end{enumerate}
\item 행정기관의 장이 출석요구서등을 조사대상자에게 발송하는 경우 출석요구서등의 내용이 외부에 공개되지 아니하도록 필요한 조치를 하여야 한다.
\end{enumerate}

제20조(자발적인 협조에 따라 실시하는 행정조사)
\begin{enumerate}[①]\tightlist
\item 행정기관의 장이 제5조 단서에 따라 조사대상자의 \highlight{자발적인 협조를 얻어 행정조사를 실시하고자 하는 경우 조사대상자는 문서·전화·구두 등의 방법으로 당해 행정조사를 거부할 수 있다.}
\item 제1항에 따른 행정조사에 대하여 조사대상자가 조사에 응할 것인지에 대한 \highlight{응답을 하지 아니하는 경우에는 법령등에 특별한 규정이 없는 한 그 조사를 거부한 것으로 본다.}
\item 행정기관의 장은 제1항 및 제2항에 따른 조사거부자의 인적 사항 등에 관한 기초자료는 특정 개인을 식별할 수 없는 형태로 통계를 작성하는 경우에 한하여 이를 이용할 수 있다.
\end{enumerate}

제22조(조사원 교체신청)
\begin{enumerate}[①]\tightlist
\item 조사대상자는 조사원에게 공정한 행정조사를 기대하기 어려운 사정이 있다고 판단되는 경우에는 행정기관의 장에게 당해 \highlightR{조사원의 교체를 신청할 수 있다.}
\item 제1항에 따른 교체신청은 그 이유를 명시한 서면으로 행정기관의 장에게 하여야 한다.
\item 제1항에 따른 \highlightR{교체신청을 받은 행정기관의 장은 즉시 이를 심사하여야 한다.}
\item 행정기관의 장은 제1항에 따른 \highlightR{교체신청이 타당하다고 인정되는 경우에는 다른 조사원으로 하여금 행정조사를 하게 하여야 한다.}
\item 행정기관의 장은 제1항에 따른 교체신청이 조사를 지연할 목적으로 한 것이거나 \highlightR{그 밖에 교체신청에 타당한 이유가 없다고 인정되는 때에는 그 신청을 기각하고 그 취지를 신청인에게 통지하여야 한다.}
\end{enumerate}

제23조(조사권 행사의 제한)
\begin{enumerate}[①]\tightlist
\item 조사원은 제9조부터 제11조까지에 따라 사전에 발송된 사항에 한하여 조사대상자를 조사하되, 사전통지한 사항과 관련된 추가적인 행정조사가 필요할 경우에는 조사대상자에게 추가조사의 필요성과 조사내용 등에 관한 사항을 서면이나 구두로 통보한 후 추가조사를 실시할 수 있다.
\item 조사대상자는 \highlightR{법률·회계 등에 대하여 전문지식이 있는 관계 전문가로 하여금 행정조사를 받는 과정에 입회하게 하거나} 의견을 진술하게 할 수 있다.
\item 조사대상자와 조사원은 \highlightR{조사과정을 방해하지 아니하는 범위 안에서 행정조사의 과정을 녹음하거나 녹화할 수 있다.} 이 경우 녹음·녹화의 범위 등은 상호 협의하여 정하여야 한다.
\item 조사대상자와 조사원이 제3항에 따라 녹음이나 녹화를 하는 경우에는 사전에 이를 당해 행정기관의 장에게 통지하여야 한다.
\end{enumerate}

제29조(행정조사의 점검과 평가)
\begin{enumerate}[①]\tightlist
\item 국무조정실장은 행정조사의 효율성·투명성 및 예측가능성을 제고하기 위하여 각급 행정기관의 \highlight{행정조사 실태, 공동조사 실시현황 및 중복조사 실시 여부 등을 확인·점검하여야 한다.} [개정 2008.2.29 제8852호(정부조직법), 2013.3.23 제11690호(정부조직법)]
\item 국무조정실장은 제1항에 따른 확인·점검결과를 평가하여 대통령령으로 정하는 절차와 방법에 따라 국무회의와 \highlightR{대통령에게 보고하여야 한다.} [개정 2008.2.29 제8852호(정부조직법), 2013.3.23 제11690호(정부조직법)]
\item 국무조정실장은 제1항에 따른 확인·점검을 위하여 각급 행정기관의 장에게 \highlightR{행정조사의 결과 및 공동조사의 현황 등에 관한 자료의 제출을 요구할 수 있다.} [개정 2008.2.29 제8852호(정부조직법), 2013.3.23 제11690호(정부조직법)]
\item 행정조사의 확인·점검 대상 행정기관과 행정조사의 확인·점검 및 평가절차에 관한 사항은 대통령령으로 정한다.
\end{enumerate}

\subsection{행정조사 종류}
\begin{itemize}\tightlist
\item 복지부 실사
\item 공단 조사
\item 심평원 조사 
\item 보건소 조사 
\item 소방서, 세무서, 식약청 등 등
\end{itemize}

\subsection{현지조사 업무 수행}
\textbf{업무수행}
\begin{itemize}\tightlist
\item 보건 복지부
	\begin{itemize}\tightlist
	\item 현지조사 법적 권한, 심평원 및 건보공단 지원 받아 현지조사 관장
	\end{itemize}
\item 건강보험심사평가원
	\begin{itemize}\tightlist
	\item 대상기관선정, 조사 실시, 정산 심사, 행정 처분, 사후 관리 등
	\item 현지조사 업무 전반 지원
	\end{itemize}
\item 국민건강보험공단
	\begin{itemize}\tightlist
	\item 수진자 조회 등 현지조사 업무 지원
	\end{itemize}	
\end{itemize}

\textbf{조사반 구성}
\begin{itemize}\tightlist
\item 의원급(약국 ) : 3명 x 3일 
\item 병원급 : 4명 x 5일
\item 종합병원급 이상 : 5명 이상 x 6일~14일
\end{itemize}
\subsection{심사 및 현지확인과 비교}
\par
\medskip

\tabulinesep =_2mm^2mm
\begin{tabu} to \linewidth {|X[1,c]|X[1,c]|X[2,l]|X[4,l]|} \tabucline[.5pt]{-}
\rowcolor{Gray!25}  구분 &  주관  & 법적 근거 & 조치 사항 \\ \tabucline[.5pt]{-}
\rowcolor{Yellow!5} 현지조사 & 보건복지부 & 법 제97조 제2항 & 부당금액환수\newline 행정처분\newline 형사고발 \\ \tabucline[.5pt]{-}
\rowcolor{Yellow!5} 방문심사 & 심사평가원 & 법 제47조 제2항\newline 시행규칙 제20조 & 심사조정\newline 현지조사 의뢰 \\ \tabucline[.5pt]{-}
\rowcolor{Yellow!5} 현지확인 & 건보 공단 & 법 제14조, 제57조 & 부당금액[환수]\newline 현지조사 의뢰 \\ \tabucline[.5pt]{-}
\end{tabu}
\par
\medskip

\includegraphics[width=\textwidth]{silsaflow}

\subsection{공단 조사}
\begin{enumerate}[1.]\tightlist
\item 어디에서 나왔나?
	\begin{description}
	\item[대답:] ``공단에서 나왔습니다”
	\item[대응:] 공문 봅시다  (자세히 읽어본다)
	\end{description}
\item 조사성격? 증거수집 목적이 강하다
\item 상대방 무기 - \highlightR{공단조사  불응시  복지부에  실사 요청  권한만 있다}
\end{enumerate}
공단의 요양기관에 대한 조사권  → 조사협조요청이다 \par

문제제기
 \begin{enumerate}[1)]\tightlist
 \item 요양급여에 관한 조사 권한이 어디 있나? 유사하거나 동일한 사안은 공동조사 해야(행정조사기본법 4조3항, 14조)
 \item 왜 사전 통보 안 했나? (행정조사기본법  17조 - 7일전 통지의무)
 \item 조사를 한 사안에 대하여 다른 기관에서 조사대상자를 재조사하면 안 된다 (행정조사기본법 15조 중복조사제한) 건보공단 조사 후 심평원 실사 의뢰 동일 사안 재조사
 \item 조사내용과 범위를 왜 고지하지 않나?
 \end{enumerate}
    
 조사 불응시 공단의 무기? → 없다
 \begin{enumerate}[1)]\tightlist
 \item 현지조사 의뢰 가능(조사 응하면 현지조사 없냐? 전혀 그렇지 않음)
 \item 타협해서 좋은 결과 거의 보지 못함 
 \end{enumerate}

공단조사시 타협으로 좋은 결과 얻기 힘든 이유 
\begin{enumerate}[1)]\tightlist
\item 그들의 방문 목적 : 공단 실적        자기 실적
\item 타협 당시 전혀 행정처분 내용을 알 수 없음
\item 조사 이후의 진행 절차는 그들의 권한이 아니므로 조사원이 그 이후를 보장할 수 없다 
\item 즉 타협은 그들의 회유 술책이다!  
\end{enumerate}

공단이 조사 목적과 범위를 밝히지 않는 조사 : Ex) 차트 일체, 수납대장 \par
\begin{description}
\item[일체] 미련 갖지 말고 차라리 실사로 가라!
\item[사유] 차분히 대비할 수 있다 (공부하고 시험치는 것이 낫다)
\end{description}
\subsection{현지조사 절차}
\begin{description}\tightlist
\item[서면조사] 조사원이 조사대상기관에 현장 방문하지 않고 요양급여 사항에 관한 보고 또는 관련 자료를 제출하도록 요구하여 요양급여비용 청구의 적법 타당성을 조사하는 방식
\item[현장조사] 조사원이 조사대상기관에 현장 방문하여 요양급여비용 청구의 적법 타당성을 조사하는 방식
\end{description}

\textbf{사전통지 실시} '선정심의 위원회'에서 증거인멸 등의 우려가 없다고 심의한 요양기관\par
\menu{사전통지 > 도착 > 신분증 제시 > 조사명령서 전달 > 조사내용 통보}\par

\textbf{사전통지 미실시} '선정심의 위원회'에서 증거인멸 등의 우려가 있다고 심의한 요양기관\par
\menu{도착 > 신분증 제시 > 조사명령서 전달 > 조사내용 통보}\par

\includegraphics[width=\textwidth]{silsacommand}\par

\includegraphics[width=\textwidth]{silsacommand2}

\subsection{복지부실사}
\begin{enumerate}[1.]\tightlist
\item 어디서  나왔나?  심평원 → 공문 봅시다
\item 조사성격 : 계도가 아니라 처벌을 목적
\item 거부시 -  국민건강보험법 98조2항   영업정지 1년
\end{enumerate}
심평원 실사 대응
\begin{enumerate}[1.]\tightlist
\item 전체 과정 이해 : 현지조사 이후 3년정도의 쟁송기간
	\begin{enumerate}[⓵]\tightlist
	\item 3-4일 조사 후 돌아감 →
	\item 3-6개월 후 사전처분서 →
	\item 이의신청 →
	\item 3-6개월 후 행정처분(면허정지, 과징금, 업무정지) →
	\item 행정소송 1심 →
	\item 행정소송2심 →
	\item 대법원 
	\end{enumerate}
\item 3-4일 조사시(심리상태: 두려움, 회피), 사전처분서 받았을 때 (심리상태: 분노, 대책찾음), 행정처분시 (절망)
\item 회원들 문제점 – 3단계, 4단계에서 대책을 세우고 연락한다. \highlightR{예후: 굉장히 나쁘다}
\item 핵심: 1단계인 3-4일 조사에서 대책을 세우고 잘 대응해야 한다. 
\end{enumerate}
관계서류 조사
\begin{commentbox}{관련근거}
\begin{itemize}\tightlist
\item 국민건강보험법시행규칙 제58조(서류의보존)
\item 요양급여의기준에 관한규칙 제7조
\end{itemize}
\Large{5년간 보존}
\begin{itemize}\tightlist
\item 요양급여비용청구서 및 명세서
\item 약제 및 치료재료 구입에 관한 서류
\item 요양급여비용의 산정에 필요한 서류 및 이를 증빙하는 서류
\item 계산서ㆍ영수증 부본 또는 본인부담금수납대장
\end{itemize}
\end{commentbox}

\begin{itemize}\tightlist
\item 관계서류제출명령위반
\item 거짓보고
\item 조사 거부·방해·기피
\end{itemize}
-> \highlightY{업무정지1년, 1천 만원 이하 벌금}\par
\begin{commentbox}{서류제출 명령 위반의 범위}
\begin{itemize}\tightlist
\item 현지조사 결과(요양급여 관계서류 제출 명령 위반) - 조제기록부, 본인부담금수납대장 등을 제출하지 않음
\item 요양기관 주장내용 - 조제기록부 등 일체의 서류를 작성∙보관하지 않음을 이유로 서류를 제출 하지 못한 경우 이를 서류제출 명령 위반이라고 볼 수 없음
\item 판결요지 - 관계 법령상의 관계서류 작성 및 보존의무의 존재를 알고 있음에도 위반하여 관계서류를 작성하지 아니하여 제출하지 못하게 된 것인바, 서류제출 명령 위반에 해당함
\item ※ \textcolor{blue}{요양기관이 해당 관계서류를 작성하지 아니하였는지 여부와는 무관 하게 서류제출명령 위반에 해당함}
\end{itemize}
\end{commentbox}

현지실사 대비
\begin{enumerate}[1.]\tightlist
\item 평소 사전 지식을 최대한 알아 두라 (유비무환)
\item 실사 닥쳤을 때 당황하지 말고 멘토와 반드시 상의하라!
\item Yes  or  No 가 아니라 Wait 도 있음
\item 3-4일동안 돈 벌려고 하지 마라!
\item \highlightR{사실확인서 (원장\&직원) 조심} : 쓰지 않는 것이 좋다(확인자의 권리) → 본대로 처분해라
    작성의미는 해당 부분에 대해 향후 이의신청, 행정소송 등의 모든 권리를 포기하는 의미 
\end{enumerate}

사실확인서 (매우 중요)
\begin{itemize}\tightlist
\item 3-4일 조사의 핵심으로 조사관이 돌아가기 전에 요구 : 위의 전체 과정 중 2단계 이후의 모든 것을 포기하겠다는 의미
\item 헌법12조2항   누구든지 자기에게 불리한 진술을 하지 않을 권리, 헌재 판례 : 형사처벌에 연결될 가능성이 있는 행정조사도 포함 
\item 대응책:  \highlightY{보신대로 처분해라,  사실여부를 떠나 안 하고 싶다.} 
\end{itemize}

현지조사시 사실확인서 타협으로 좋은 결과 얻기 힘든 이유 
\begin{enumerate}[1)]\tightlist
\item 그들의 방문 목적 : 처벌(실적) 목적 (o) , 계도 목적(X)
\item 타협 당시 전혀 행정처분 내용을 알 수 없음
\item 조사 이후의 진행 절차는 그들의 권한이 아니므로 조사원이 그 이후를 보장할 수 없다 
\item 사실상 이의절차를 포기하겠다는 의미 
\item 즉 타협은 그들의 실적 확보를 위한 회유 술책이다!  
\end{enumerate}

현지실사의 대응팁
\begin{enumerate}[1)]\tightlist
\item 중요한 것은 원장이 직접 해야 : 의료기관 내부 고발이 많음
\item 평소에 최대한 규정에 맞게 : 나중에 6배 환수
\item 입증 책임은 복지부에게 있음 :  EX) BMD를 간호사가 했다?  → 모든 사례에 대해 건건이 복지부가 누가 했는지 입증을 해야 함 -> 생각없이 포괄 확인서 쓰면 안 됨 
\item 실사 왔을 때는 진료를 최소화하고 실사에 집중해야!
\end{enumerate}

\subsection{현지조사 후 처분 }
\includegraphics[width=\textwidth]{postsilsa}

허위 청구 4중 처분  
\begin{enumerate}[1)]\tightlist
\item 의료법에 근거 행정처분(면허정지)
\item 국민건강보험법에 의거 환수,업무정지,과징금
\item 의료법, 형법에 의해 형사 처벌
\item 허위 청구로 인한 면허정지는 의료업을 할 수 없다 (의료법 66조3항)
\item 하나의 행위에 대해 처벌은 사실상 4번 받음 – 헌법상의 평등권, 형평성, 비례의 원칙, 과잉금지의 원칙에 사실상 위배
\end{enumerate}

부당금액 정산 및 행정처분 산출\par 
조사대상기간 6개월 (진료비1억) → 총 부당금액 300만원 → 부당비율 3\% → 월평균부당금액 50만원 → 업무정지 일수는?\par
\includegraphics[width=\textwidth]{postsilsa2}\par

업무정지 가중 처분
\begin{enumerate}[1)]\tightlist
\item 5년 이내에 업무정지(과징금처분)을 받은 사실이 있는 경우
\item 당해 업무정지기간 또는 과징금의 2배에 해당하는 처분 부과 (업무정지1년, 과징금 5배 초과 불가)
\item 5년 기산 방법 : 행정처분통보문서 송달일자로 부터 부당청구가 다시 확인 된 날(현지조사 확인서 징구일)
\end{enumerate}

\begin{commentbox}{면허자격정지 개념}
면허자격 정지
\begin{itemize}\tightlist
\item 부정한 방법으로 진료비 또는 약제비를 거짓청구 → 1년 범위 내에서 면허자격정지
\item 의료법 제66조(자격 정지 등)제1항 제7호
\item  약사법 제79조(약사·한약사 면허의 취소 등)제2항
\end{itemize}
의료업 정지
\begin{itemize}\tightlist
\item 자격정지 기간 중 의료업 불가
\item 의료법 제66조(자격정지 등)제3항 → 관련 서류를 위조,변조하거나 속임수 등 부정한 방법으로 진료비를 거짓청구 한 때(자격정지)
\end{itemize}
\end{commentbox}
\includegraphics[width=\textwidth]{postsilsa3}\par

\paragraph{현지실사 후 처분선택 }\par
업무정지  VS  과징금 최대 5배 선택시 고려점\par
(1억 낼 것이냐?  VS  5개월 쉴것이냐?)\par
→  최종판결 전 업무정지가 시행된다\par
개선방향 – 면허정지, 업무정지 같은 처벌은 최종심 판결 이후 시행해야

\paragraph{행정처분 절차}
\begin{enumerate}\tightlist
\item 처분사전통지
	\begin{itemize}\tightlist
	\item 처분원인이 되는 사실
	\item 처분 내용 및 근거
	\item 의견 제출 방법
	\end{itemize}
\item 의견청취
	\begin{itemize}\tightlist
	\item 우편 또는 보건복지부를 방문하여 의견제출
	\end{itemize}
\item 의견검토
	\begin{itemize}\tightlist
	\item 제출된 의견에 대해 심평원에서 검토하여 복지부에 보고
	\end{itemize}
\item 행정처분
	\begin{itemize}\tightlist
	\item 업무정지처분 또는 과징금 처분
	\end{itemize}	
\item 관련근거: 행정절차법 제21조제1항, 제22조제3항	
\end{enumerate}
	
\paragraph{형사고발 기준}
\begin{enumerate}\tightlist
\item 업무정지기간 중 요양급여(건보법 제115조)
	\begin{itemize}\tightlist
	\item 1년 이하 징역 및 1천 만원 이하 벌금
	\end{itemize}
\item서류 미제출, 거짓보고, 거짓서류제출, 조사거부ㆍ방해ㆍ기피할 경우(건보법 제 116조)
	\begin{itemize}\tightlist
	\item 1천 만원 이하 벌금
	\end{itemize}
\item요양급여비용 거짓 청구는 형법상 사기죄로 고발
	\begin{itemize}\tightlist
	\item 10년 이하 징역 또는 2천 만원 이하 벌금 (형법 제347조)
	\end{itemize}
\item ※ 고발(내부기준) 거짓청구금액 750만원 또는 거짓청구비율 10\% 이상인 기관
\end{enumerate}	
\begin{commentbox}{거짓청구기관 명단공표(법 제 100조)}
행정처분을 받은 요양기관 중 관련 서류를 위·변조하여 요양급여비용을 거짓으로 청구한 요양기관은 위반행위, 처분내용, 명칭, 주소, 대표자의 성명 등을 공표 \par
건강보험공표심의위원회 구성 : 위원장 1인 포함 9명, 임기 2년(연임 가능)\par
\begin{enumerate}\tightlist
\item <공표대상> - 거짓청구금액 1,500만원 이상 - 거짓청구비율 100분의 20이상인 경우
\item <공표방법> - 보건복지부, 심평원, 공단, 시·군·구 등의 홈페이지에 공표 - 거짓청구가 중대한 위반에 해당하는 경우 신문 또는 방송에 추가 공표
\item <공표절차> - 건강보험 공표 심의위원회에서 공표 대상 선정 → 대상자에게 의견진술 기회부여 → 소명자료 검토 후 최종 대상확정 → 인터넷ㆍ언론 등에 공표	
\end{enumerate}
\end{commentbox}

\subsection{현지실사문제점 }
\begin{enumerate}[1)]\tightlist
\item 실적, 처벌 목적이 아니라 운용 취지에 맞게 올바른 청구문화의 정착위해 운용해야 
\item 행정조사기본법 미준수  
    최소한의 조사원칙,조사권 남용금지 (4조1항)
    공동조사원칙,중복조사금지(4조3항, 15조)
    처벌보다 계도원칙(4조4항)
    조사목적, 조사범위와 내용 명확화의 원칙(11조)
    7일전 사전통지원칙(17조)
\end{enumerate}

\clearpage
\section{제출 요구 서류 및 그 의미}
\begin{enumerate}[①]\tightlist
\item 인력 관련
	\begin{itemize}\tightlist
	\item 직원 명부(현재 근무자 및 \textcolor{red}{1년 이내 퇴사자 포함}) 
	\item 급여 대장 및 근로 계약서
	\item 근무일지(Duty 표)
	\item 의사, 간호사, 의료기사, 약사 면허증 사본
	\end{itemize}
\item 시설 관련
	\begin{itemize}\tightlist
	\item 보건소 등록된 평면도
	\item 식당위탁계약서 / 식사 대장
	\item 적출물 관리대장 / 배출시설 운영 일지
	\end{itemize}	
\item 물품 관련
	\begin{itemize}\tightlist
	\item 의약품 및 진료용 재료 구입 관련 증빙서류\newline
 (의약품 구입 목록 대장/의약품 수불 대장/거래 명세표)\newline
 (의료비품 대장/소모품 수불 대장/거래 명세표) 
	\item 마약관리대장
	\end{itemize}	
\item 검사 관련
	\begin{itemize}\tightlist
	\item 방사선과 : 촬영대장, PACS 데이터, 판독기록 
	\item 임상병리실 : 검사대장 / 수탁 검사 항목표 / 검사 결과지(원내 / 수탁)
	\end{itemize}	
\item 장비 관련
	\begin{itemize}\tightlist
	\item 의료 장비 보유 및 구입에 관한 증빙서류
	\item 방사선 촬영장치 정기검사 성적서(X-ray, Mammo), 설치등록증명서
	\item 임상 병리과 검사 장비 종류 확인해서 심평원 신고 된 내용과 일치하는 지 확인함 
	\item 수술 장비 등도 심평원 신고 내용과 일치하는 확인함(일부 병원에서는 실제 사용이 가능한 장비인지 작동해 보기도 한다고 함)
	\end{itemize}	
\item 입원 관련 서류
	\begin{itemize}\tightlist
	\item 공통 내용 : \textcolor{red}{입원 DB}/입퇴원 장부/일일 수납대장/\textcolor{red}{진료비 영수증(상세 내역)}/본인 부담금 수납대장 
	\item 비급여 항목 관련 : \textcolor{red}{비급여 리스트/수납대장}/수진자별 리스트
	\item 급여 항목 관련 : 요양급여비용 계산서 / 요양급여비용 심사 청구서 및 명세서
	\item \textcolor{red}{진료 기록부(의사 및 간호 차트/개인별 투약 기록/방사선 판독 결과 \& 임상병리 결과지)}
	\end{itemize}	
\item 외래 진료 관련 서류
	\begin{itemize}\tightlist
	\item 공통 내용 : \textcolor{red}{외래 DB}/수진자별 접수 대장/일일 수납대장/\textcolor{red}{진료비 영수증(상세 내역)}/본인 부담금 수납대장 
	\item 비급여 항목 관련 : \textcolor{red}{비급여 항목 리스트/수납대장}/수진자별 리스트
	\item 급여 항목 관련 : 요양급여비용 계산서 / 요양급여 비용 심사 청구서 및 명세서 
	\item \textcolor{red}{진료 기록부(의사 차트/오더 기록/방사선 판독 결과 \& 임상병리 결과지)}
	\end{itemize}	
\item 외래 및 입원 DB 관련 설명
	\begin{itemize}\tightlist
	\item DB에서 조사관들이 보는 자료는 우리가 보는 차트가 아님.
 ( 의사 차트는 보지 못하고 처방 기록만 볼 수 있는 것 같음.그래서 의사 차트 열람이 필요한 경우에는 해당 환자의 차트를 출력해 달라고 함)
	\item 처방이나 차팅 시간을 확인하는 것 같음.(추후에 수정한 내용들은 무시하고 실제 진료 시기에 기록한 내용들을 기준으로 판단함)
	\item DB는 반드시 줘야 하는지 ? (안주면 1년 이하의 업무 정지)
	\item 전자 차트 업체는 우리편이 아니라 심평원 편임
	\end{itemize}	
\end{enumerate}

\clearpage
\section{현지조사 거짓\cntrdot{}부당청구 확인사례}
\subsection{거짓청구 VS 부당청구}
\begin{description}\tightlist
\item[거짓청구] \highlightR{진료비 청구의 원인이 되는 진료 행위가 실제 존재하지 않음}에도 관련서류의 거짓작성 또는 속임수 등의 부정한 방법에 의해 진료비를 청구한 행위
	\begin{itemize}\tightlist
	\item 입원 및 내원 거짓 또는 증일청구
	\item 미실시 행위료등 청구
	\item 비급여대상 진료후 이중청구
	\end{itemize}
\item[부당청구] \highlightR{진료비 청구의 원인이 되는 진료행위는 실제 존재}하나, 진료행위가 건강보험법 및 의료법 등 관계법령을 위반하여 부정하게 이루어진 진료비 청구행위
	\begin{itemize}\tightlist
	\item 요양급여 산정기준 위반청구
	\item 행위료, 의약품 등 대체청구
	\item 인력,시설,장비 신고 위반청구
	\item 본인부담금 과다징수등
	\end{itemize}
\end{description}
\subsection{복지부 실사 다빈도 사례 분석표 (2016. 10. 31. 의협신문)} 
\href{http://www.doctorsnews.co.kr/news/articleView.html?idxno=113498}{의원급 실사 '비급여 이중청구' 가장 많아}
\par
\medskip

\tabulinesep =_2mm^2mm
\begin{tabu} to \linewidth {|X[1,c]|X[1,c]|X[2,l]|} \tabucline[.5pt]{-}
\rowcolor{Gray!25}  유형  & 빈 도 & 사 례 \\ \tabucline[.5pt]{-}
\rowcolor{Yellow!5} 비급여 이중청구 & 43건(39.8\%) & \textbf{▶} 비급여 과정에서 진찰료 청구(피부 미용시술, 예방접종, 단순 영양제 투여)\newline \textbf{▶} 비급여 항목을 급여로 청구\\ \tabucline[.5pt]{-}
\rowcolor{Yellow!5} 거래량-청구량 불일치 & 15건 (13.8\%) & \textbf{▶} 주사제, 1회용 겸자, 하기도증기흡입시 흡입제 등의 거래량이 청구량보다 적은 경우 \\ \tabucline[.5pt]{-}
\rowcolor{Yellow!5} 검진당일 대장내시경 청구 & 13건(12.0\%) & \\ \tabucline[.5pt]{-}
\rowcolor{Yellow!5} 임의 비급여 & 11건 (10.2\%) & \textbf{▶} 주사제, 수액제, 검사 등 급여항목을 비급여로 청구 \\ \tabucline[.5pt]{-}
\rowcolor{Yellow!5} 본인부담금 할인 & 9건(8.3\%) & \\ \tabucline[.5pt]{-}
\rowcolor{Yellow!5} 의료법 및 의료기사법 위반 & 9건 (8.3\%) & \textbf{▶} 무자격자 진료보조 \newline \textbf{▶} 진료기록부 허위기재 \newline \textbf{▶} 전화상담후 청구 \newline \textbf{▶} 일회용 주사기나 겸자 재사용 등 \\ \tabucline[.5pt]{-}
\rowcolor{Yellow!5} 미진료 청구 & 3건(2.7\%) & \textbf{▶} 입원중 외출\cntrdot{}외박 환자 청구 \newline \textbf{▶} 가족을 보지 않은 상태에서 가족치료 청구 \\ \tabucline[.5pt]{-}
\rowcolor{Yellow!5} 기타 다빈도 사례 & 21건 (19.4\%) & \textbf{▶} 외이도이물제거 관련 \newline \textbf{▶} 대리청방 관련 \newline \textbf{▶} 직원상근관련 여부 \newline \textbf{▶} 직원 및 친인척 진료 \\ \tabucline[.5pt]{-}
\end{tabu}
%\par
%\medskip
\subsection{거짓청구 유형(허위청구)}
\begin{enumerate}[①]\tightlist
\item 비급여 이중 청구
	\begin{itemize}\tightlist
	\item 비급여 진료(예방 접종, 비만 치료 등) 후 진찰료를 산정하는 경우
	\item 국가암 검진 시 진찰료를 청구하는 경우
	\item 비급여 항목을 비급여로 받고 급여로 다시 청구하는 경우
	\end{itemize}	
\item 미진료 시 진찰료 청구
	\begin{itemize}\tightlist
	\item 특히 수진자의 해외 출국 기록 확인함(출국 기간 중 진료 기록이 있으면 문제가 됨)
	\end{itemize}	
\item 거래량 / 청구량 불일치
	\begin{itemize}\tightlist
	\item 약제 및 물품 관리 시 남아있는 양을 기준으로 재구입하면 안 되고 청구양을 기준으로 재구입해야 한다.
	\item 질정 조심
	\item 혹시 DC를 하신다면 DC 때 사용하시는 유착 방지제 및 영양제의 경우 기록을 남기지 않을텐데, 물품 매입 기록과 재고량이 너무 차이가 나면 의심받을 가능성 있음
	\end{itemize}	
\item 처치 /행위료 허위 청구
	\begin{itemize}\tightlist
	\item 행위료의 경우 행위를 했다는 기록 외에는 다른 어떠한 물증이 없는 상태임. 실사 시 수진자들에게 전화로 확인해서 그런 진료 행위를 받은 기억이 없다는 진술을 받으려 함.
	\end{itemize}	
\item 미실시 검사 허위 청구
	\begin{itemize}\tightlist
	\item 실사 시 방사선 검사 기록(PACS나 필름) / 임상 병리 검사 결과지 등으로 확인함
	\item 기록이 없으면 허위 청구로 몰아가려 함.(특히 원내 검사 결과 관리 잘해야 함)
	\end{itemize}	
\item 본인부담금 할인 : 환자 유인행위로 간주하며, 할인된 본인부담금과 연관된 공단 청구 금액을 허위 청구로 간주함
	\begin{itemize}\tightlist
	\item 특히 직원 할인의 경우 조심해야 한다.
	\item 직원 약 처방 때문에 접수 후 약 처방하고 본인 부담금 안 받으면 안됨.
	\item 직원 할인 / 직원 가족 할인의 경우 모두 불법임(할인을 해줄 경우 그 차액만큼을 상여금을 지급한 것으로 해서 세무상 증빙 서류를 만들어 놓아야만 합법임)
	\end{itemize}
\end{enumerate}

\begin{commentbox}{실제 내원하지 않은 일자에 내원한 것으로 요양급여비용을 청구}
특히 수진자의 \emph{해외 출국 기록 확인함}. (출국 기간 중 진료 기록이 있으면 문제가 됨)
\begin{description}\tightlist
\item[관련근거] 요양급여비용의 청구는 국민건강보험법 제47조(요양급여비용의 청구와 지급 등)와 의료법 제22조(진료기록부 등) 제1항 등에 의거 요양기관에 내원한 수신자에 대하여 실제 진료한 내역을 기록한 진료기록부 등에 의하여 정확히 청구하여야 함.
\item[부당사례]
\begin{enumerate}[1)]\tightlist
\item A의원은 수진자 ㅇㅇㅇ에 대해 4일 외래진료를 받은 것으로 청구하였으나, 실제로는 \highlight{이틀만 내원하였고, 이틀은 내원하지 않았음에도 내원하여 진료를 받은 것으로 거짓기록}하고 진찰료 등을 청구
\item B의원은 수진자 ㅇㅇㅇ의 경우, 2013년 5월 4일, 5월 8일, 5월 13일 ‘어깨의 유착성 관절낭염(M750)’상병으로 내원하여 진료받은 것으로 청구하였으나, 실제로 2013년 5월2일부터 5월 14일까지 \highlight{해외출국 중으로 진료받은 사실이 없음에도 거짓기록 후 청구}
\end{enumerate}
\end{description}
\end{commentbox}

\begin{commentbox}{실제 시행하지 않은 검사료 거짓청구}
\begin{description}\tightlist
\item[관련근거] 요양급여비용의 청구는 국민건강보험법 제47조(요양급여비용의 청구와 지급 등)와 의료법 제22조(진료기록부 등) 제1항 등에 의거 요양기관에 내원한 수신자에 대하여 실제 진료한 내역을 기록한 진료기록부 등에 의하여 정확히 청구하여야 함.
\item[부당사례]
\begin{enumerate}[1)]\tightlist
\item A, B의원은 실시하지 않은 당검사(반정량)를 실시한 것으로 진료기록부에 거짓 기록한 후 검사료 등 요양급여비용을 청구
\item C의원은 인성검사-간이정신진단검사(F6216), 치매척도검사[GDS](F6221)를 실시 하지 않고 해당 검사료를 요양급여비용으로 청구
\end{enumerate}
\end{description}
\end{commentbox}

\begin{commentbox}{실제 시행하지 않은 이학요법료 거짓청구}
\begin{description}\tightlist
\item[관련근거] 건강보험 행위급여\cntrdot{}비급여 목록표 및 급여상대가치점수 제7장 이학요법료 - 해당항목의 물리치료를 실시할 수 있는 일정한 면적의 치료실과 실제 사용할 수 있는 장비를 보유하고 있는 요양기관에서 의사의 처방에 따라 상근물리치료사가 실시하고, 그 결과를 진료기록부에 기록한 경우에 요양급여비용으로 산정
\item[부당사례]
\begin{enumerate}[1)]\tightlist
\item  A의원은 근막염 등의 상병으로 내원한 수진자 ㅇㅇㅇ에게 실제 시행하지 않은 표층열 치료(MM010)와 단순운동치료[일당](MM101) 등을 시행한 것으로 요양 급여비용을 청구
\item B의원은 이학요법 처방은 하였으나 물리치료 실시 기록대장을 확인한 결과 심층 열치료 [1일당](MM020)를 실시하지 않고, 요양급여비용을 청구함
\end{enumerate}
\end{description}
\end{commentbox}

비급여대상(피부관리)을 전액 환자에게 부담시킨 후 요양급여비용으로 이중청구
\begin{description}\tightlist
\item[관련근거] 건강건강보험요양급여의 기준에 관한 규칙 제9조 제1항 관련[별표2] 업무 또는 일상생활에 지장이 없는 경우, 신체의 필수 기능개선 목적이 아닌 경우, 예방진료로써 질병\cntrdot{}부상의 진료를 직접 목적으로 하지 아니하는 경우에 실시 또는 사용되는 행위\cntrdot{}약제 및 치료재료 등은 비급여 대상이므로 요양급여비용으로 청구 할 수 없음.
\item[부당사례]
\begin{enumerate}[1)]\tightlist
\item  A의원은 비급여대상인 모공, 안면홍조 등 \highlightR{피부관리}를 위해 2일간 내원한 수진자 ㅇㅇㅇ에 대해 부분 혈관 레이저 등을 시술하고 그 비용을 \highlightR{비급여로 환자에게 전액 징수하였음}에도 ‘장미색잔비늘증(비강진)(L42)’ 상병으로 진찰료를 요양급여 비용으로 이중청구함
\end{enumerate}
\end{description}

비급여대상(예방접종, 비만)을 전액 환자에게 부담시킨 후 요양급여비용으로 이중청구
\begin{description}\tightlist
\item[관련근거] 건강건강보험요양급여의 기준에 관한 규칙 제9조 제1항 관련[별표2] 업무 또는 일상생활에 지장이 없는 경우, 신체의 필수 기능개선 목적이 아닌 경우, 예방진료로써 질병\cntrdot{}부상의 진료를 직접 목적으로 하지 아니하는 경우에 실시 또는 사용되는 행위\cntrdot{}약제 및 치료재료 등은 비급여 대상이므로 요양급여비용으로 청구 할 수 없음.
\item[부당사례]
\begin{enumerate}[1)]\tightlist
\item  A의원은 독감접종을 위해 내원한 수진자 ㅇㅇㅇ에 대하여 진찰 및 \highlight{독감접종을 실시하고 그 비용을 비급여로 환자에게 전액 징수하였음에도 ‘소화불량(K30)’ 상병으로 내원하여 진료받은 것으로 진찰료를 요양급여비용으로 이중청구함}
\item B의원은 \highlight{단순비만을 진료 후 비용을 비급여로 전액징수 후 ‘상세불명의 고혈당증(R739)’ 상병으로 진찰료 등을 요양급여비용으로 이중청구함}
\end{enumerate}
\end{description}

비급여대상(희망검진)을 전액 환자에게 부담시킨 후 요양급여비용으로 이중청구
\begin{description}\tightlist
\item[관련근거] 건강건강보험요양급여의 기준에 관한 규칙 제9조 제1항 관련[별표2] 업무 또는 일상생활에 지장이 없는 경우, 신체의 필수 기능개선 목적이 아닌 경우, 예방진료로써 질병\cntrdot{}부상의 진료를 직접 목적으로 하지 아니하는 경우에 실시 또는 사용되는 행위\cntrdot{}약제 및 치료재료 등은 비급여 대상이므로 요양급여비용으로 청구 할 수 없음.
\item[부당사례]
\begin{enumerate}[1)]\tightlist
\item  A의원은 \highlight{본인 희망검진으로} 내원한 수진자 ㅇㅇㅇ에 대해 그 비용을 \highlight{비급여로 전액 징수하였음에도 ‘만성 표재성 위염(K293)’ 등의 상병으로 진찰료 및 검사료 등을 요양급여비용으로 이중청구함}
\end{enumerate}
\end{description}

\begin{commentbox}{공단 건강검진 실시 후 검진비용 및 요양급여비용 이중청구}
\begin{description}\tightlist
\item[관련근거] 
\begin{itemize}\tightlist
\item 국민건강보험법 제52조(건강검진), 건강검진기본법('08.3.21 제정, 법률 제8942호)
\item 건강검진실시기준, 암검진실시기준 등
\end{itemize}
\item[부당사례]
\begin{enumerate}[1)]\tightlist
\item  A의원은 검진항목에 포함되어있는 혈당검사 등의 혈액검사를 실시하고 관련 \highlight{검진비용으로 청구하고, 요양급여비용으로 이중청구}
\end{enumerate}
\end{description}
\end{commentbox}

\subsection{부당청구}
\begin{enumerate}[① ]\tightlist
\item 1인실 기준 위반 :
    전체 병동의 50\% 이상을 다인실로 해야만 나머지 병실을 1인실로 인정받을 수 있음. 즉, 이 기준을 위반하게 되면 모든 병실을 1인실로 인정받을 수 없음
\item 모자동실 위반 :
    원칙은 모자동실은 24시간이여야 함. 법원 판례를 보면 최소한 12시간 이상 모자 동실을 한 경우만 ‘모자 동실료’가 인정이 됨.
\item 의료법 및 의료 기사법 위반
	\begin{itemize}\tightlist
	\item UDS는 의사가 직접 해야 한다.(적어도 의사가 지휘 감독이라도 해야 함) 
	\item EKG는 임상 병리사 / 의사가 시행해야 한다.
	\item BMD / CXR / MAMMO는 방사선사가 찍어야 한다.
	\item 방사선 기계와 임상 병리기계가 심평원에 등록된 기종과 다르면 안 된다.
	\item 일반식 가산은 상주하는 영양사 / 조리사가 2명 이상 있어야 한다. 
	\item 약 조제는 의사 / 약사가 해야 한다. (병동 환자 약 포장은 혼합 조제의 개념이므로  간호사 / 간호 조무사 모두 안 된다. 약사 출근 여부 전화로 확인 가능함.)
	\item 수액 line은 의사 / 간호사가 잡아야 한다. (간호조무사는 의사 감독하에만 가능하다.)
	\item 채혈은 의사 / 임상 병리사가 해야 한다.(간호 조무사는 안 된다. 간호사는 ?)
	\item NST는 간호사가 걸어야 한다. (임상병리사 / 방사선사 안 된다. 간호 조무사는 ?)  
	\end{itemize}
\item 비상근 영상의학과 의사 관련
Mammogram은 특수영상 장비로 비근속 영상의학과 의사 가 적어도 일주일에 한번은 출근해서 정도 관리를 해야 한다.(직접 영상의학과 의사에게 전화 걸어 출근 사실 확인함)
\item 진찰료 산정 위반
	\begin{itemize}\tightlist
	\item 국가암 검진 시 당일 급여 진료를 본 경우 진찰료는 50%
   산정해야 함
	\item 대리 처방 후 진찰료는 50\% 산정해야 함
	\end{itemize}	
\item 초/재진 산정 위반
\item 급여 / 비급여 산정 위반
	\begin{itemize}\tightlist
	\item 급여 항목을 비급여로 청구한 경우(임의비급여)
	\item 비급여 항목을 급여로 청구한 경우
	\end{itemize}	
\item 별도 산정할 수 없는 치료 재료대 산정
	\begin{itemize}\tightlist
	\item 행위료에 포함되어 있는 것으로 간주되는 치료대를 별도로 산정하는 경우
	\item 약물 소작술 시 알보칠 질정은 인정 안 됨.(알보칠 액은 비급여 산정 가능)
	\item 소수술시 비용 보전 위해 치료 재료대를 비급여 산정하면 안됨. (수술 중 초음파로 대체)
	\end{itemize}	
\item DRG에서 유착 방지제 / 인소브 / 영양제
	\begin{itemize}\tightlist
	\item C/SEC 환자는 DRG로 유착 방지제 / 영양제 / 인소브 비용 받으면 안 됨
	\item 부인과 수술의 경우 DRG로 유착 방지제 / 인소브 비용 받으면 안 됨(Ectopic preg는 제외) 
	\item 물품의 구매량과 재고량을 카운트하므로 퇴원 영수증에 기입을 하지 않고 수납을 하더라도 문제가 된다.
	\end{itemize}	
\end{enumerate}  

\begin{commentbox}{공단 건강검진 실시 후 산정기준 위반청구}
\begin{description}\tightlist
\item[관련근거] 
\begin{itemize}\tightlist
\item 국민건강보험법 제52조(건강검진), 건강검진기본법('08.3.21 제정, 법률 제8942호)
\item 건강검진실시기준, 암검진실시기준 등
\end{itemize}
\item[부당사례]
\begin{enumerate}[1)]\tightlist
\item B의원은 공단검진당일 검진과는 \highlight{별도 질환에 대한 진찰 시 진찰 이외에 의사의 처방이 발생한 경우 진찰료의 50\%를 산정할 수 있으나, 진찰료 100\%를 요양급여비용으로 청구(산정기준 위반청구)}
\end{enumerate}
\end{description}
\end{commentbox}

\textbf{사회복지시설 내 촉탁의 진료 후 진찰료 부당청구}
\begin{commentbox}{사회복지시설 등의 입소자에게 원외처방전 교부 후 진찰료 100\% 청구}
\begin{description}\tightlist
\item[관련근거] 
\begin{itemize}\tightlist
\item 건강보험 행위급여\cntrdot{}비급여 목록표 및 급여상대가치점수 제1장 기본진료료 가-1-나(재진진찰료) 주:8. 사회복지사업법에 따른 사회복지시설 내에서 의료기관 소속 촉탁의 또는 협약의료기관의사가 시설 입소자에게 원외처방전을 교부한 겨우 \highlightR{진찰료 중 외래관리료 소정점수를 산정하여야 함(AA254080코드로 청구)}
\end{itemize}
\item[부당사례]
\begin{enumerate}[1)]\tightlist
\item A의원은 2013년 9월부터 총 9회 ‘상세불명 원인의 상세불명의 접촉피부염(L259)’ 등의 상병으로 진료한 수진자 ㅇㅇㅇ의 경우 \highlight{사회복지시설 입소자임에도 불구하고 의료기관 소속 촉탁의가 시설에 방문하여 진료 후 원외처방전을 교부해주고 진찰료 100\%를 요양급여비용으로 청구}
\end{enumerate}
\end{description}
\end{commentbox}

\paragraph{환자 가족이 내원하여 처방전 발급 시 진찰료 부당청구}\par
환자 보호자(가족)에게 처방전 등을 발급한 후 재진진찰료 100\% 청구
\begin{description}\tightlist
\item[관련근거] 
\begin{itemize}\tightlist
\item 건강보험 행위급여\cntrdot{}비급여 목록표 및 급여상대가치점수 제1장 기본진료료 가-1-나(재진진찰료) 주:7. 환자가 직접 내원하지 아니하고 \highlight{환자 가족이 내원하여 진료담당의사와 상담한 후 약제를 수령하거나 처방전만을 발급받는 경우에는 재진진찰료 소정점수의 50\%를 산정}하여야 함.
\end{itemize}
\item[부당사례]
\begin{enumerate}[1)]\tightlist
\item  A의원은 수진자 ㅇㅇㅇ의 보호자만 내원하여 진료의사와 상담 후 약제를 수령 했으나 \highlight{재진진찰료 50\%(AA254090)를 산정 청구하여야 함에도 재진진찰료 100\%}를 산정하여 요양급여비용을 청구
\end{enumerate}
\end{description}

\textbf{영상의학과 전문의가 상근하지 아니하면서 판독료 가산 청구}
\begin{description}\tightlist
\item[관련근거] 
\begin{itemize}\tightlist
\item 건강보험 행위급여\cntrdot{}비급여 목록표 및 급여상대가치점수 제3장 영상진단 및 방사선치료료 제1절 방사선단순영상진단료 주1 및 제2절 방사선특수영상진단료 주1 - 당해 요양기관에 상근하는 영상의학과 전문의가 판독을 하고 판독소견서를 작성한 경우에 소정점수의 10\%를 가산함
\end{itemize}
\item[부당사례]
\begin{enumerate}[1)]\tightlist
\item   A의원은 영상의학과 전문의가 판독소견서를 작성하지 않고 \highlight{해당 진료과의 전문의가 진료기록부에 판독소견을 기록한 후 판독료(소정점수) 10\%의 가산료를 산정}하여 요양급여비용을 청구
\end{enumerate}
\end{description}

\paragraph{무자격자가 시행한 방사선단순영상진단료 부당청구}\par
\textbf{무자격자가 촬영한 방사선영상진단료 청구}
\begin{description}\tightlist
\item[관련근거] 
\begin{itemize}\tightlist
\item 의료법 제27조(무면허 의료행위 등 금지) 제1항 - 의료인이 아니면 누구든지 의료행위를 할 수 없으며 의료인도 면허된 것 이외의 의료행위를 할 수 없음. -
\item 의료기사 등에 관한 법률 제3조(업무범위와 한계), 제9조(무면허 업무금지 등) 및 동법 시행령 제2조(의료기사, 의무기록사 및 안경사의 업무범위 등)
\end{itemize}
\item[부당사례]
\begin{enumerate}[1)]\tightlist
\item  A의원은 방사선사가 근무하지 않은 기간 동안, \highlight{방사선사 자격이 없는 원무과장이 방사선영상진단 촬영}을 하고 그 비용을 요양급여비용으로 청구
\end{enumerate}
\end{description}

\paragraph{입원환자 식대-영양사, 조리사 가산 부당청구}\par
\begin{commentbox}{영양사 조리사 가산 관련 동 인력의 실제 근무내용과 다르게 신고}
\begin{description}\tightlist
\item[관련근거] 
\begin{itemize}\tightlist
\item 입원환자 식대 세부산정기준(보건복지부 제2016-91호(행위), 2016.6.15.) 
	\begin{itemize}\tightlist
	\item 입원환자 식대 세부산정기준에 의거 일반식 가산에서 영양사 가산, 조리사 가산에 필요한 인력산정 기준은 환자식 제공업무를 주로 담당하는 \highlight{당해 요양기관에 소속된 인력으로 의원급(보건의료원 포함) 각각 1명, 병원급 이상은 각각 2명 이상인 경우 산정}함
	\item 전일제 영양사 및 조리사로 1주간의 근로시간이 월평균 40시간인 근무자는 1인으로 산정하고 단시간 근무로 1주간의 근로시간이 월평균 32시간(이상) - 40시간(미만) 근무자는 0.8인으로 산정하며, 32시간 미만 근무자는 산정대상에서 제외함.
	\end{itemize}
\end{itemize}
\item[부당사례]
\begin{enumerate}[1)]\tightlist
\item A병원은 [표] 영양사 근무현황과 같이 영양사 4명에 대한 입\cntrdot{}퇴사일을 실제 근무내역과 다르게 신고하여 2013.11월, 2013년 12월, 2014년 2월 영양사 가산을 산정할 수 없음에도 요양급여비용으로 청구함.
\item B병원은 조리사 ㅇㅇㅇ가 2013.1.8. - 2013.10.23. 상근 근무한 것으로 신고하였으나 \highlight{실제로는 근무 사실이 없으며,} 조리사 ㅇㅇㅇ는 2011.10.21. - 2013 1.7.근무한 것으로 신고하였나 실제 2011.10.21.-2012.9.8.까지 근무한 것으로 확인되는 등 2012.9.9.-2013.10.23.까지 \highlight{조리사 1인만 상근으로 근무하였음에도 식대 조리사 가산을 요양급여비용으로 청구함}
\end{enumerate}
\end{description}

영양사 근무현황\par
\tabulinesep =_2mm^2mm
\begin{tabu} to \linewidth {|X[2,c]|X[2,c]|X[2,l]|X[2,l]|X[2,l]|X[2,l]|} \tabucline[.5pt]{-}
\rowcolor{Gray!25}  구분 & 성명 & 신고 & 내역 & 확인 & 내역 \\ \tabucline[.5pt]{-}
\rowcolor{Yellow!5} 영양사 & & 입사일 & 퇴사일 & 입사일 & 퇴사일 \\ \tabucline[.5pt]{-}
\rowcolor{Yellow!5} & ㅇㅇㅇ & 2013.9.1 & 2013.11.30 & 2013.9.9 & 2013.1.30 \\ \tabucline[.5pt]{-}
\rowcolor{Yellow!5} & ㅇㅇㅇ & 2013.9.1 & 2013.9.16 & 2013.9.1 & 2013.9.8 \\ \tabucline[.5pt]{-}
\rowcolor{Yellow!5} & ㅇㅇㅇ & 2013.9.17 & 2013.10.7 & 2013.9.17 & 2013.10.2 \\ \tabucline[.5pt]{-}
\rowcolor{Yellow!5} & ㅇㅇㅇ & 2013.12.1 & 2014.4.30 & 2013.12.3 & 2014.4.30 \\ \tabucline[.5pt]{-}
\end{tabu}
\par
\end{commentbox}

\paragraph{별도 산정할 수 없는 치료재료대 부당징수}\par
\textbf{행위료에 포함되어 별도 산정할 수 없는 치료재료료 본인부담금으로 징수}
\begin{description}\tightlist
\item[관련근거] 
\begin{itemize}\tightlist
\item 요양급여의 비용 중 본인이 부담할 비용의 부담액은 국민건강보험법제41조(요양급여) 및 제44조(비용의 일부부담), 동법 시행령 제19조(비용의 본인부담) 및 [별표2] \highlight{요양급여비용 중 본인이 부담할 비용의 부담률 및 부담액에 따라 징수하고}, 요양급여사항 또는 비급여사항 외의 다른 명목으로 비용청구를 해서는 안됨
\end{itemize}
\item[부당사례]
\begin{enumerate}[1)]\tightlist
\item A의원은 관련 행위료에 포함되어 그 비용을 별도 산정할 수 없는 치료재료 (\highlight{hemoclip})을 사용하고 그 비용을 수진자에게 전액 본인부담금으로 징수 
\item B의원은 경막외신경차단술(LA222) 등을 실시하고 신경차단술 행위료에 포함된 치료재료(\highlight{Epidural needle})의 일부비용을 별도 수진자에게 본인부담금으로 징수 
\item C병원은 요양급여비용의 소정수가에 포함되어 별도산정 불가인 \highlight{수술포} 등을 수진 자에게 본인부담금으로 징수 행위료에 포함되어 별도 산정할 수 없는 치료재료료 본인부담금으로 징수
\end{enumerate}
\end{description}

\paragraph{의사인력 확보수준에 따른 입원료 차등제 산정기준 위반청구}
\textbf{의사인력을 실제 근무 사실과 다르게 신고}
\begin{description}\tightlist
\item[관련근거] 
\begin{itemize}\tightlist
\item 건강보험 행위 급여\cntrdot{}비급여목록표 및 급여상대가치점수 제3부 요양병원 산정지침 4.라.'의사인력 확보수준에 따른 입원료 차등제'
\item 의사인력 확보수준에 따른 요양병원 입원료 차등적용 관련 기준(보건복지부고시 제2009-214호(요양병원), 2010.4.1.)
\end{itemize}

\item[부당사례]
\begin{enumerate}[1)]\tightlist
\item A요양병원은 의사 ㅇㅇㅇ가 2011년 5월 9일부터 2011년 6월 13일까지 근무한 사실이 없으나 상근한 것으로 신고∙적용하여, 의사인력확보수준에 따른 입원료 차등제를 실제 의사등급보다 높게 적용하여 요양급여비용을 청구
\item B요양병원은 주1~2일만 근무한 의사를 상근으로 신고하여 의사인력 확보수준이 2등급임에도 1등급으로 적용하여 (1등급 전문의 수가 50\%이상인 경우 산정가능한) 요양병원 입원료를 20\%가산하여 청구
\item ※ 내과,외과, 신경과,정신건강의학과, 재활의학과, 가정의학과, 신경외과, 정형외과 전문의 수 50\%이상
\end{enumerate}
\end{description}


\paragraph{간호인력 확보수준에 따른 입원료 차등제 산정기준 위반청구}
\textbf{간호인력을 실제 근무사실과 다르게 신고}
\begin{description}\tightlist
\item[관련근거] 
\begin{itemize}\tightlist
\item 건강보험 행위 급여\cntrdot{}비급여목록표 및 급여상대가치점수 제3부 요양병원 산정지침 4.마. '간호인력 확보수준에 따른 입원료 차등제'
\item ※ \highlight{간호사 비율이 간호인력의 3분의 2이상인 경우 1일당 2,000원을 별도산정}
\item 간호인력 확보수준에 따른 요양병원 입원료 차등적용 관련 기준(보건복지부고시 제2009-214호(요양병원), 2010.4.1.)
\end{itemize}

\item[부당사례]
\begin{enumerate}[1)]\tightlist
\item A요양병원은 [표] 간호인력 근무현황과 같이 병동에 전속되어 입원환자 간호업무를 전담하는 인력이 아닌 간호사 ㅇㅇㅇ 등을 실제 근무내역과 다르게 병동 전담 근무인력으로 신고 \highlight{’14년 4분기, ’15년 1분기 간호인력 확보수준이 2등급임에도 1등급으로 적용하여 요양병원 입원료 청구}
\end{enumerate}
\end{description}
\clearpage
\section{실사가 나왔을떄 전략}
\Large{그들의 특성}\normalsize
\par
\medskip
\leftrod{그들은 공무원이다.}
\begin{itemize}\tightlist
\item 나온 이상 반드시 기본 이상은 해야 한다.(기본이 안 될 것 같으면 억지 조항을 들이대서라도 일정 성과는 가지고 가야 함)
\item 사명감을 가지고 모든 문제를 낱낱이 밝힐 의지가 있는 것은 아니다.
\item 하지만 조사 과정에서 감정이 너무 틀어져 버리면 없던 열의가 생길 수 있다.
\end{itemize}

\leftrod{3사(복지부 / 심평원 / 공단) 연합 팀이고 급조된 팀이다.}
\begin{itemize}\tightlist
\item 초기부터 손발이 잘 맞는 것이 아니고 서서히 손발을 맞춰가면서 일을 한다.
\item 서로 감시가 되므로 눈에 보이는 것을 넘어갈 수는 없다. 
\end{itemize}

\Large{그들이 찾으려고 하는 내용들}\normalsize
\par
\medskip
\leftrod{1) 조사 }
\begin{itemize}\tightlist
\item 원칙적으로 현지 조사는 전수 조사이다.(불똥이 어디로든 튈 수 있으며 일단 번지기 시작하면 끝까지 갈 수도 있다.)
\item 우선 현지 조사가 나오게 된 부분에 대한 조사가 먼저 이루어진다.
\item 그들의 입장에서는 현지 조사 나온 부분에 대해서는 반드시 성과를 내야 한다.
\item 그 이후 다른 부분에 대해서 조사한다.(시간이 없으면 더 조사 안 한다.)
\end{itemize}

\leftrod{2) 진료 관련}

\begin{enumerate}[①]\tightlist
\item 기본 항목(가장 먼저 확인하는 내용)
	\begin{itemize}\tightlist
	\item 현지 조사를 나오게 된 직접적인 원인 
	\item 실사 나오기 전, 그 병원의 청구 내역을 분석해서 타기관에 비해서 특이해 보이는 부분들 \newline
-> 이 부분들은 그들이 생각하는 ‘기본’이다. 이 부분에서는 반드시 성과를 내야 하기 때문에 그들도 많은 시간을 할애한다. 나중에 밝혀지더라도 이 부분에서 최대한 시간을 끌어서 더 이상 진도를 못 나가게 막아야 한다.
	\end{itemize}
\item 추가 항목
	\begin{itemize}\tightlist
	\item DB를 이용해 뽑아낸 자료들
	\item 다빈도 청구 항목 
	\item 비급여 리스트 중 임의비급여 항목
	\item 비급여 리스트 중 다빈도 처방 항목
	\item 비급여 리스트 중 문제가 있을 만한 항목(유착방지제/영양제 등) \newline
-> DB를 분석하면 이 항목들과 그 수진자 리스트를 뽑을 수 있다. 이에 대해 서류 / 수진자 통화 / 직원 면담 등을 통해 사실 확인을 한다.
	\end{itemize}
\end{enumerate} 
\leftrod{3) 시설, 인력, 장비 등이 신고 된 내역과 실제가 일치하는 지 확인}

\leftrod{4) 함정 항목}
\begin{itemize}\tightlist
\item 2), 3)을 통해서 만족한 만한 성과가 나오지 않을 경우
\item 흔히 착오 청구를 할 만한 항목
\item 현실과 동떨어진 법 때문에 문제가 될 수 밖에 없는 항목 \newline
 - > 이들을 통해 어떻게든 성과를 맞추려 한다. 
\end{itemize}

\Large{실제적인 대응전략}\normalsize
\par
\medskip 
\leftrod{1) 눈에 보이면 잡는다.(-> 눈에 보이지 않으면 넘어간다.)}
\begin{itemize}\tightlist
\item 실사 나오면 일상으로 하던 병원 내 모든 업무를 멈춘다.
\item 모든 답변은 원장을 통해서만 한다.\newline
 원장 : ‘모든 내용은 내가 답변할테니 직원들에게는 묻지 말라.’ \newline
 직원 :  ‘저는 직원이라 드릴 말씀이 없으니 모든 내용은 원장님께 여쭈어보시라’
\item 간단한 자료라도 반드시 원장이 직접 확인 후 제출한다.
\item 자료 검토 후 필요하면 수정해서 준다.(그들은 수사관이 아니다. 자료의 진위를 밝힐 정도의 능력도, 의지도 있지 않다. 그냥 준 자료를 가지고 검토한다.)
\end{itemize}

\leftrod{2) 시간을 끌어야 한다.}
\begin{enumerate}[①]\tightlist
\item 자료 제출은 최대한 천천히 한다. 
	\begin{itemize}\tightlist
	\item 자료를 안 주면 안 된다.(업무 정지 1년)
	\item 그러나 천천히 주는 것은 문제되지 않는다.  ‘요청 자료 찾고 있고, 정리하고 있다.’
	\end{itemize}
\item 시간이 되면 가야 한다.(그들도 다음 스케쥴 있음)
	\begin{itemize}\tightlist
	\item ‘기본 항목’에서 싸우다가 가게 해야 한다.
	\item 자료를 빨리 줘서 ‘기본 항목’이 끝나고 나면 다음 항목으로 진도가 넘어간다.
	\item ‘기본 항목’에 대해서 잘못을 인정하는 순간, 이에 대해서는 상황이 종결되어 버리고 다음 진도로 넘어가게 된다.(그들은 ‘기본 항목’은 확실히 건져야 하므로, 인정 안 하면 다음으로 넘어갈 수 없다.)
	\end{itemize}
\end{enumerate} 
\leftrod{3) 험악한 분위기 조성하면 안 된다.}
\begin{itemize}\tightlist
\item 그들도 사람이다.
\item 감정 상하면 얼마든지 열의를 가지고 조사할 수 있다.
\item 태업은 하되, 읍소도 하고 능글거리며 둘러대기도 해서 상황을 모면해야 한다. 
\end{itemize}

\leftrod{4) 패턴을 인정하면 안 된다.(사실 확인서 포함)}
\begin{enumerate}[①]\tightlist
\item 행정 소송의 특성
	\begin{itemize}\tightlist
	\item 조사가 끝나면 보건복지부 사무관이 정리해서 사전 처분
    -> 이의 신청 -> 본 처분 -> 행정소송으로 진행됨
	\item 행정 소송 시 부당 청구의 입증 책임은 복지부에 있음
	\item  원칙적으로 복지부에서는 부당 청구자 명단 전체에 대해 그 사실을 입증해야 함(복지부 사무관 혼자서 감당 못함)
	\item 만일 병원 측에서 그 중 일부가 정당한 청구였음을 입증한다면 일부만 무혐의가 되는 것이 아니라 전체가 무혐의로 처리됨
	\item  잘못된 패턴을 인정하고 사실확인서를 쓰면 그 자체가 입증 자료가 되므로 상황이 종결되어 버림
	\end{itemize}
\item 대처법
	\begin{itemize}\tightlist
	\item 평소에 문제의 소지가 있는 진료를 루틴으로 하면 안 된다.
	\item 조사자가 곤란한 질문을 하면 그 자리에서 잘못을 인정해버리지 말고, 일단 내용 확인 후 답변 하겠다고 둘러댄다.(일단은 넘어가고 관계 자료를 검토 후 답변을 찾는다.)
	\item 문제 case에 대해서만 인정하고 패턴은 인정하지 않는다.
     ‘ 이런 처방이 나간 환자가 1000명이 있는데 그 중 무작위로 30명에게 전화를 걸었더니 사실과 다르더라. 1000명에 대해 잘못을 인정해라’ -> ‘그 30명에게 왜 그런 잘못된 처방이 나갔는지 정확히는 모르겠지만, 나는 항상 제대로 했고, 그 30명에게만 잘못된 처방이 나간 것 같다
	 \end{itemize}
\end{enumerate}
	 
\leftrod{5) 실사가 나온 순간 바로 도움을 청해야 한다.}
\begin{itemize}\tightlist
\item 가장 중요한 내용임
\item 모르는 상태에서는 그들의 함정에 빠질 수밖에 없고 일단 잘못 진행된 부분은 돌이킬 수 없다.
\item 실사가 나온 순간 바로 ‘산의회 법제 이사팀’에게 연락해서 실시간으로 조언을 받도록 한다.
\end{itemize}	 