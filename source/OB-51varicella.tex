\section{Varicella-zoster}
\myde{}{
\begin{itemize}\tightlist
\item[\dsjuridical] B019 Chicken pox
\item[\dsjuridical] B02 Zoster
\item[\dsjuridical] A60 항문생식기의 헤르페스바이러스
%\item[\dsmedical] 
\item[\dschemical] Varicella zoster IgG C4682496
\item[\dschemical] Varicella zoster IgM C4683496
\end{itemize}
}
{
\leftrod{아시클로버 정}\par
\begin{center}\textbf{- 다          음 -}\end{center}
\begin{enumerate}[가.]\tightlist
\item 효능/효능 : 초발성·재발성 생식기포진을 포함한 피부·점막조직의 단순포진 바이러스 감염증의 치료·예방.\newline 
대상포진 바이러스 감염증의 치료(특히 급성시의 통증). 포진후 신경통에 대한 효과는 아직 증명되지 않음. \newline
2세 이상 소아의 수두 치료.

\item 용법/용량입니다
	\begin{enumerate}[1)]\tightlist
	\item 성인(단순포진바이러스 감염증 치료) : 1회 200mg, 1일 5회(4시간 간격), 5일간, 중증 초발성 감염증인 경우 치료 연장 가능, 
	\item 성인(면역기능 정상인 환자의 단순포진 감염증 예방) : 1회 200mg, 1일 4회(약 6시간 간격) 또는 1회 400mg, 1일 2회(12시간 간격). 그 후 1회 200mg씩 1일 2-3회로 감량하여 유효성 확인 후 감량. 
	\item 성인(면역기능 저하자의 단순포진 감염증 예방) : 1회 200mg, 1일 4회(약 6시간 간격). 
	\item 성인(대상포진 감염증 치료) : 1회 800mg, 1일 5회(취침시간 제외, 약 4시간 간격), 총 7일간. 
	\item 소아 (단순포진 감염증 치료 및 면역기능 저하자의 단순포진 감염증 예방) : 2세 이상: 성인에 준함, 2세 미만: 성인용량의 1/2. 
	\item 소아 (수두치료(2세 이상)): 1회 20mg/kg(1회 최대 800mg)씩 1일 4회, 5일간. 
	\end{enumerate}
\end{enumerate}
}
\subsection{예방접종과 임신}
예방접종은 금기입니다. AU TGA 임신 카테고리 : B2 미국 FDA 임신 카테고리 : 임신 카테고리에 공식적으로 지정되지 않았습니다. 의견 :이 백신 접종 후 1 - 3 개월 동안 임신을 피하십시오.

동물 생식 연구는 수행되지 않았다. 이 백신이 임산부에게 투여 될 때 태아에게 해를 끼칠 수 있는지 여부는 알려지지 않았습니다. 자연적으로 발생하는 수두 - 대상 포진 바이러스 (VZV) 감염은 때때로 태아에게 해를 끼치는 것으로 알려져 있습니다.\par
그러므로이 백신은 임산부에게 투여해서는 안됩니다. 임신 중 또는 임신 3 개월 이내에 수두 백신을 부주의하게 투여 한 후 태아의 결과를 모니터링하기위한 임신 레지스트리는 19 년 동안 유지되었습니다. \par
2011 년 3 월 현재 수두를 포함한 백신을 접종 한 800 명 이상의 여성 중 선천성 수두 증후군과 일치하는 이상 소견은 없었습니다.약. 동물에 대한 연구가 부적절하거나 부족할 수도 있지만, 이용 가능한 데이터에는 태아 손상이 증가했다는 증거는 없습니다. 
기형의 빈도가 증가하거나 인간 태아에 대한 직접 또는 간접적 인 유해한 영향이 관찰되지 않았다. 동물에 대한 연구가 부적절하거나 부족할 수도 있지만, 이용 가능한 데이터에는 태아 손상이 증가했다는 증거는 없습니다. 기형의 빈도가 증가하거나 인간 태아에 대한 직접 또는 간접적 인 유해한 영향이 관찰되지 않았다. 동물에 대한 연구가 부적절하거나 부족할 수도 있지만, 이용 가능한 데이터에는 태아 손상이 증가했다는 증거는 없습니다.\par
\url{https://www.drugs.com/pregnancy/zoster-vaccine-live.html}

\subsection{수두면역글로블린(varicella-zoster Immunoglobulin)}
\url{http://www.druginfo.co.kr/cp/msdNew/ingredient/ingre_view_cp.aspx?cppid=40515&cpingPid=1747&cpingPid_List=4535}

\begin{commentbox}{varicella}
Pregnant women who get varicella are at risk for serious complications; they are at increased risk for developing pneumonia, and in some cases, may die as a result of varicella.\par
If a pregnant woman gets varicella in her 1st or early 2nd trimester, her baby has a small risk (0.4 – 2.0 percent) of being born with congenital varicella syndrome. The baby may have scarring on the skin, abnormalities in limbs, brain, and eyes, and low birth weight.\par
If a woman develops varicella rash from 5 days before to 2 days after delivery, the newborn will be at risk for neonatal varicella. In the absence of treatment, up to 30% of these newborns may develop severe neonatal varicella infection.
\end{commentbox}

\begin{commentbox}{zoster}
Herpes zoster infection during pregnancy is \highlight{not associated with increased risk of congenital malformations} above the general population baseline risk or of CVS. 
Individuals with HZ should cover lesions in order to reduce the risk of transmitting VZV to susceptible pregnant women.\par
If a susceptible pregnant woman (in any stage of pregnancy) is exposed to VZV, passive antibody prophylaxis with immunoglobulin preparation containing VZV immunoglobulin G is indicated within 96 hours of exposure.
\end{commentbox}
