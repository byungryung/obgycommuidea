\section{외래주사만}
\myde{}{%
\begin{itemize}\tightlist
\item[\dsjuridical] N970 배란과 관련된 여성 불임증
\item[\dsmedical] KK010 피하또는근육내주사
\end{itemize}
청구메모>> 타병원에서 배란유도 약제 가지고 와서 근육 주사 시행함
}%
{\begin{enumerate}\tightlist
\item 불임진료와 관련하여 보건복지부 고시 제2004-36호(‘04.7.1시행)에 따라 일정기간 임신이 되지 않아 불임이 의심되는 경우에 그 원인을 알기 위한 검사 또는 임신촉진 목적의 배란촉진제 사용 등 세부 인정기준에 따라 급여\bullet 비급여를 정하고 있습니다.
\item 또한 인공수정 등 보조생식술(체내ㆍ체외인공수정 포함)시 소요되는 비용(행위, 약제 치료재료)은「국민건강보험 요양급여의 기준에 관한 규칙」(별표2)에 비급여대상에 해당하여 비급여로 운영하고 있습니다.
\item 아울러 진찰료는 외래에서 환자를 진찰한 경우에 산정하는 수가이며, 원내에서 주사제를 주사한 경우에는 해당 수기료 산정이 가능하다고 판단됩니다.
\item 따라서, 상기 규정등을 고려하시어 환자의 급여여부에 따라 적용하시기 바랍니다. 감사합니다. 끝.
\end{enumerate}
}
