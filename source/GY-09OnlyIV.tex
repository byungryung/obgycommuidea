\section{외래주사만}
\myde{}{%
\begin{itemize}\tightlist
\item[\dsjuridical] N970 배란과 관련된 여성 불임증
\item[\dsmedical] KK010 피하또는근육내주사
\end{itemize}
청구메모>> 타병원에서 배란유도 약제 가지고 와서 근육 주사 시행함
}%
{\begin{enumerate}\tightlist
\item 불임진료와 관련하여 보건복지부 고시 제2004-36호(‘04.7.1시행)에 따라 일정기간 임신이 되지 않아 불임이 의심되는 경우에 그 원인을 알기 위한 검사 또는 임신촉진 목적의 배란촉진제 사용 등 세부 인정기준에 따라 급여\bullet 비급여를 정하고 있습니다.
\item \sout{또한 인공수정 등 보조생식술(체내ㆍ체외인공수정 포함)시 소요되는 비용(행위, 약제 치료재료)은「국민건강보험 요양급여의 기준에 관한 규칙」(별표2)에 비급여대상에 해당하여 비급여로 운영하고 있습니다.}
\item 기존 외부 주사제를 가져온경우 (불임) 타 불임병의원에서 불임 시술을 (IVF-ET, IUI)하는 중간이라면,비급이  이던게, 2016년 10월부터는 불임시술 보험적용되면서, 급여로 바뀌어야 할것 같습니다.
\item 아울러 진찰료는 외래에서 환자를 진찰한 경우에 산정하는 수가이며, 원내에서 주사제를 주사한 경우에는 해당 수기료 산정이 가능하다고 판단됩니다.
\item 따라서, 상기 규정등을 고려하시어 환자의 급여여부에 따라 적용하시기 바랍니다. 감사합니다. 끝.
\end{enumerate}
}

\Que{본인부담률 30\%를 적용하는 보조생식술 진료기간의 요양급여비용 청구시 명세서 구분자를 기재해야 하나요?}
\Ans{본인부담률 30\%를 적용하는 보조생식술 진료기간의 요양급여비용을 청구하는 명세서에는 특정내역 MT002(특정기호)란에 'F021'기재하여 청구합니다.}
\Que{의원급 외래인 경우 보조생식술 진료기간과 그 외 진료의 본인부담률이 30\%로 동일한데도 특정기호 F021을 기재해야 하나요?}
\Ans{건보공단 대상자 사전등록 시스템에 등록된 정보와 비교하여 급여 적용여부가 결정되므로 특정기호 F021을 반드시 기재하여야 합니다. }
\Que{보조생식술 진료기간에 타상병과 동시 진료시 청구는 어떻게 하나요?}
\Ans{보조생식술과 관련된 요양급여비용은 본인부담률을 30\% 적용하고, 타상병에 대한 요양급여비용은 분리청구하여 현행 종별 본인부담률을 적용하되, 특정내역 MT001(상해외인)란에 ‘T’를 기재합니다. }