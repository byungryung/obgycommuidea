\section{의료분쟁시 의료인의 자세}
\begin{enumerate}\tightlist
\item 환자의 요구(need)가 뭔가?
  위로의 문제(주관적접근) VS 배상의 문제(객관적 접근)  
\item 배상의 문제 -> 객관화(key word)
\item 가져야 할 자세
	\begin{mdframed}[linecolor=red,middlelinewidth=2]
	\begin{enumerate}[1)]\tightlist
	\item 차트 정리부터  철저 (가장 긴급)
	\item 녹음한다는 것을  항상  명심 – 대답은  매뉴얼대로  반복해야
	\item 직원들 대답, 행동 매뉴얼대로 반복 VS 창의적,즉흥적 대응 금물
	\item Yes, NO 의 접근 버려야 (Ex: 잘못했어요? 안 했어요? 책임질거에요? 안 질거에요?) –
	\item 절대 조급함 버려야 (세월호)  시간이 해결..       
	\end{enumerate}
	\end{mdframed}
\end{enumerate}

\subsection{배상요구 분쟁 문제 팁}
\begin{enumerate}\tightlist
\item 진료시간에 절대 분쟁에 응하지 마라
   \begin{mdframed}[linecolor=blue,middlelinewidth=2]
	-> 대응방법: 따로 시간을 낼테니 현재는 진료시간이니 협조해 달라. 진료하게 도와달라만 반복   (불응시 업무방해죄로 대응) 
   \end{mdframed}	
	\begin{enumerate}[1)]\tightlist
	\item 저녁에 따로 시간을 낸 경우 – 대화 매뉴얼대로 반복
		\begin{enumerate}[①]\tightlist
		\item 의사의 책임이라고 나온다면 모든 책임지겠다   
   		\item 절차에 따라 객관적 손해액을 산정하자.
		\end{enumerate}
	\item 5분이상 불응시 – 매뉴얼대로 신고 
   \begin{mdframed}[linecolor=blue,middlelinewidth=2]
  원내 대응매뉴얼에 따라 직원이 112 신고 -> 경찰출동 -> 경찰은 반드시 현행범으로 입건해야 하고 직원이 경찰서 가서 본대로 현행범에 대해 고발인 조사 -> 원장은 진료
 계속
   \end{mdframed}
	\end{enumerate}
\end{enumerate}

\begin{Ddoing}{의료분쟁사례1 : 주사후 걷지못하는 80세 할머니}
	\begin{itemize}\tightlist
	\item 본원에서  NSAID 엉덩이  근육 주사를  맞은  후  걷지를  못한다고  아들, 며느리, 사위 총출동하여 배상을 요구한 사례
	\item 가족들 요구
		\begin{enumerate}\tightlist 
		\item 주사맞고  잘못되었으니 당신 잘못이다.  
    		\item 어떻게  책임질거냐?  
		\end{enumerate}
	\item  흔히  하는   잘못된 대응
		\begin{enumerate}\tightlist
		\item 우리  잘못  아니다
		\item 책임질 것 없다 \textcolor{red}{-> 분쟁 격화}
		\end{enumerate}
	\item 
	\begin{mdframed}[linecolor=blue,middlelinewidth=2]
	객관화:  (긍정도 부정도 하면 안 됨)
아! 그러셨어요?  심려가 크시겠군요. 일단 왜 이런지 대학병원에 가서 원인을 규명해 봅시다 -> 병원 잘못이라고 나오면 모든 책임을 지겠습니다
	\end{mdframed}
	\item 결과:  그 이후에 오지 않았음
	\end{itemize}
\end{Ddoing}


\begin{Ddoing}{의료분쟁사례2 : 폐색전증 사망 사건}
	\begin{itemize}\tightlist
	\item POD \# 1    병실에서  갑자기  쓰러져 사망한  사건 
	\item 가족들 요구 
		\begin{enumerate}\tightlist
		\item 병원측  잘못이다.   
		\item 어떻게  책임질거냐?  
		\end{enumerate}
	\item 흔히  하는   잘못된 대응
		\begin{enumerate}\tightlist
		\item 우리  잘못  아니다
		\item 책임질 것 없다 \textcolor{red}{-> 분쟁 격화} 
 		\end{enumerate}
	\item 해법
	\begin{mdframed}[linecolor=blue,middlelinewidth=2]
객관화:  (긍정도 부정도 하면 안 됨)
모든 책임진다 -> 원인규명 및 과실 인과관계 절차에 따르자
 	\end{mdframed}
	\item 결과:1년이후 1000만원 위로금 판정
	\end{itemize}  
\end{Ddoing}

\subsection{경찰서 갔을 때 진술에 유의해야 한다}
\begin{enumerate}[1)]\tightlist
\item 묵비권!!
\begin{mdframed}[linecolor=blue,middlelinewidth=2]
 :  피의자가 자기에게 불리한 진술을 강요당하지 않는 권리. 보통은 형사 피의자가 수사기관의 조사나 공판에 있어서 각개의 심문에 대하여 진술을 거부할 수 있는 헌법상 권리를 말함
\end{mdframed}
\item 헌법 12조 2항 (묵비권!!)
\begin{mdframed}[linecolor=blue,middlelinewidth=2]
 :  모든 국민은 고문을 받지 아니하며, 형사상 자기에게 불리한 진술을 강요당하지 아니한다   
\end{mdframed}
\item 미란다 원칙 
\begin{mdframed}[linecolor=blue,middlelinewidth=2]
  : 피의자가 진술을 거부할 수 있는 권리 고지 하지 않고 수사하면 무효,  자발적, 자의적 진술이 아니면 무효 
\end{mdframed}
\end{enumerate}
\section{의료분쟁으로 피고소시 알아야 할 사항 }
\begin{enumerate}[1)]\tightlist
\item 의사는 무슨 죄인가? : 
   업무상 과실치상, 업무상 과실치사 
\item 핵심 쟁점은? 
   \begin{mdframed}[linecolor=blue,middlelinewidth=2]
	과실 유무\\
   다시 해도 그렇게 했을 것이고 최선을 다했으나 나의 능력으로는 불가항력이었다.
   원인?왜 그렇게 되었는지도 이해가 안 되고 모르겠다.
   \end{mdframed}	    
\item 가급적 피해야 할 진술
   \begin{mdframed}[linecolor=blue,middlelinewidth=2]
       Ex)장손상 등의 사건\\
      생각해 보니 이런게 아쉬운 부분이 있다. 
      다시 한다면 이렇게 했을  것 같다 -> \textcolor{red}{처벌됨}
   \end{mdframed}
\end{enumerate}

\begin{hemphsentense}{병원난동시  어떤 죄가 성립?}
\begin{enumerate}[1)]\tightlist
\item 의료법 12조 2항: 방조자도 처벌함 
 : 누구든지 의료기관의 의료용 시설·기재·약품, 그 밖의 기물 등을 파괴·손상하거나 의료기관을 점거하여 진료를 방해하여서는 아니 되며, 이를 교사하거나 방조하여서는 아니 된다.
\item 의료법 제87조 1항 :  다음 각 호의 어느 하나에 해당하는 자는 5년 이하의 징역이나 2천만원 이하의 벌금에 처한다.  의료법 12조2항
\item 업무방해죄 (형법 314조)
\item 모욕죄 (형법 311조):  욕,조롱,악평 등으로 모욕감을 느끼게 하면 성립
\item 명예훼손죄(형법307조):  명예를 훼손시키면 성립
\item 정보통신망 이용 촉진 및 정보보호 등에 관한 법률 70조 
      사람을 비방할 목적으로 정보통신망을 통하여 공공연하게 사실을 드러내어 다른 사람의 명예를 훼손한 자는 3년 이하의 징역 또는 3천만원 이하의 벌금에 처한다
\item 주거침입,퇴거불응죄 (형법 319조):  주거권자의  의사에 반하면 성립 
\item 폭행죄 (형법 260 조) : 폭력적 행동
\item 상해죄 (형법  257조) : 다쳤을 때  
\item 손괴죄 (형법 371조) : 병원 기물 파손
\end{enumerate}
\end{hemphsentense}

\subsection{내용증명}
\begin{description}\tightlist
\item[정의]  어떤  내용의  문서를  언제 누가 누구에게  발송하였는가 하는 사실을  우체국장이  공적인  입장에서  증명하는  제도
\item[의미]  양 당사자 사이에  어떤  의사  표시를  했다는  증거
\item[방법] 내용증명  우편물은  보통 3통을  작성하여  1통을  작성해서  2부를  복사해서  우체국에  가서  내용증명으로  발송해  달라고 하면 됨
\end{description}