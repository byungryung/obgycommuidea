\section{환자분류체계}
\subsection{환자분류체계(Patient Classification System,PCS)의 정의}
\begin{itemize}\tightlist
\item 상병, 시술, 기능상태 등을 이용해서 외래나 입원 환자를 임상적 의미와 의료자원 소모 측면에서 유사한 그룹으로 분류하는 체계
\item 대표적인 환자분류체계 : DRG(Diagnosis Related Groups, 진단명 기준 환자군)
\item 주로 지불 도구, 환자구성(Case-mix) 보정 도구 등으로 활용
\end{itemize}

\leftrod{Case-Mix (mix of cases)}
\par
\medskip
\begin{itemize}\tightlist
\item 과학적 방법으로 patient care episodes를 분류하는 information tool
\item 세계적으로 Clinical management \& Funding 에 광범위하게 활용
\item 병원간 의미 있는 activity 비교에 활용
\item 미국 Fetter and Thompson 개발 (60년대 후반 - 70년대)
	\begin{itemize}\tightlist
	\item 당초 병원 output 기술 및 의료의 질과 자원소모를 모니터링 하기 위해 개발
	\item 급성기 입원 영역에 초점을 둠
	\end{itemize}
\end{itemize}
\prezi{\clearpage}
\begin{center}
\includegraphics[width=.9\textwidth]{PCSrequest}
\end{center}
\prezi{\clearpage}
%\subsection{환자분류체계 사용 정보}
\begin{tcolorbox}[frogbox,title=환자분류체계 사용 정보]
\begin{itemize}\tightlist
\item 행위 분류
	\begin{itemize}\tightlist
	\item 환자에게 행해진 시술을 표준화된 코드로 기록한 ‘건강보험요양급여 행위 목록’ 활용
	\item 현행 : 건강보험요양급여 행위 급여 목록표( 2016.2월판)
	\end{itemize}
\item 질병(진단, 상병) 분류
	\begin{itemize}\tightlist
	\item 환자의 입ㆍ내원 이유와 동반상병, 합병증을 표준화된 진단(상병)코드로 기록한 ‘한국표준질병ㆍ사인분류(KCD)’ 활용
	\item 현행 : KCD-7차(‘16.1.1 시행, 통계청 고시)
	\item 주진단(principal diagnosis)
		\begin{itemize}\tightlist
		\item 검사 후 밝혀진 \textcolor{red}{최종 진단}, 병원치료를 필요로 하게 만든 가장 중요 병태
		\item 다만, 진료개시 후 입원 시 병태와는 관련 없는 새로운 병태가 발견되고 이로 인한 자원소모가 더 클 경우 이를 주진단으로 선정
		\end{itemize}
	\item 기타진단(secondary diagnosis)
		\begin{itemize}\tightlist
		\item 입원 당시부터 주진단과 함께 있었거나 발생된 병태로,
		\item 치료나 입원기간에 영향을 준 모든 진단 (동반상병 및 합병증 등)
		\item 과거의 입원과는 관련 있지만 현재 입원과는 관련 없는 병태는 제외
		\end{itemize}
	\end{itemize}
\end{itemize}
\end{tcolorbox}
\prezi{\clearpage}
\subsection{환자분류체계 구성요소}
\begin{center}
\includegraphics[width=.9\textwidth]{PCSYe}
\end{center}
\prezi{\clearpage}
\begin{center}
\includegraphics[width=.9\textwidth]{PCSYe2}
\end{center}
\prezi{\clearpage}
\subsection{우리나라 환자분류체계 종류}
\begin{itemize}\tightlist
\item 의과
	\begin{itemize}\tightlist
	\item KDRG (입원환자분류체계) : Korean Diagnosis Related Group
	\item KOPG (외래환자분류체계) : Korean Outpatient Group
	\item KRPG (재활환자분류체계) : Korean Rehabilitation Patient Group
	\end{itemize}
\item 한의	
	\begin{itemize}\tightlist
	\item KDRG-KM (한의 입원환자분류체계) : Korean Diagnosis Related Group-Korean Medicine
	\item KOPG-KM (한의 외래환자분류체계) : Korean Outpatient Group-Korean Medicine	
	\end{itemize}
\end{itemize}
\prezi{\clearpage}
\subsection{환자분류체계의 활용}
\begin{enumerate}\tightlist
\item 진료비 지불
	\begin{itemize}\tightlist
	\item 포괄수가 지불단위(7개 질병군, 신포괄)
	\item 맹장염으로 충수절제술을 받은 입원 환자 → 맹장염(질병분류)과 충수절제술(행위분류) 이용하여 분류 → 충수절제술 환자분류(12개)를 토대로 포괄수가 산출(요양기관별, 일자별)
	\item 포괄수가제도: 입원기간 동안 제공된 진료량과 관계없이 어떤 질병의 진료를 위해 입원했는지에 따라 미리 정해진 일정액을 지불하는 제도
	\end{itemize}
\item 심사ㆍ현지조사
	\begin{itemize}\tightlist
	\item 심사대상 선정
	\item 종합정보서비스
	\item 지표연동자율 개선제
	\end{itemize}
\item 평가
	\begin{itemize}\tightlist
	\item 평가지표 산출
	\end{itemize}
\item 급여관리
	\begin{itemize}\tightlist
	\item 처방ㆍ조제 약품비 절감
	\item 장려금 지표
	\end{itemize}
\item 의료기관 기능평가
	\begin{itemize}\tightlist
	\item 상급종합병원 지정ㆍ평가 : 현행 지정 기준(복지부 고시, ‘15.1시행)은 KDRG V3.5의 ADRG 질병군 분류를 활용
		\begin{description}\tightlist
		\item[전문진료질병군] >\par
			\begin{description}\tightlist
			\item[분류기준] 희귀성, 합병증↑, 치사율↑,진단난이도↑등
			\item[KDRG수] (ADRG기준) 245개
			\item[비고] 환자구성비율 17\% 이상
			\end{description}
		\item[일반 진료 질병군] >\par
			\begin{description}\tightlist
			\item[분류기준] 모든 의료기관에서 진료 가능하거나 진료를 하여도 되는 질병군
			\item[KDRG수] (ADRG기준) 362개
			\item[비고] 환자구성비율 17\% 이상
			\end{description}	
		\item[단순 진료 질병군] >\par
			\begin{description}\tightlist
			\item[분류기준] 진료가 간단한 질병, 그밖에 상급종합병원에서 진료를 받지 않아도 되는 질병군
			\item[KDRG수] (ADRG기준) 94개
			\item[비고] 환자구성비율 16\% 이하
			\end{description}				
		\end{description}
	\item 전문병원 지정ㆍ평가
	\end{itemize}
\end{enumerate}
\prezi{\clearpage}
