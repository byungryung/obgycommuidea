\section{7회 이상의 초음파 급여가능한 경우}  %\keystroke{Win}+\keystroke{F}\footnote{\url{https://dl.dropboxusercontent.com/u/6869091/2016obgysono/obgySONO.exe} \par\noindent\url{https://dl.dropboxusercontent.com/u/6869091/2016obgysono/obgy.sqlite}}
\begin{commentbox}{}
임신 과정 중 의학적 판단 하에 \uline{태아에게 이상이 있거나 이상이 예상되어 상기 산정횟수를 초과하여 시행해야 하는 경우에는 해당 삼분기의 일반 또는 일반의 제한적 초음파로 산정하며(주항 제외), 입원 중 동일 목적으로 1일 수회 시행하는 경우에도 1일 1회만 산정함.}[고시 제2016 - 175호]\par
\noindent 비정상 임신 또는 의학적 필요에 의해 시행하는 초음파의 경우: 횟수 초과하여도 급여 산정 가능한 경우와 JX999 기타내역에 기입할 사항입니다. 예: \highlight[cyan!70]{출혈(O200,O209,O441,O46,O469), 태동의 현저한 변화(Z368,O368), 발열(R509), 복통(R1049), 조기진통(O47,O600), 외상(S코드외상), 위해성 약물 노출(O355), 태아 이상(O28)},조기양막파수(O421), 분만예정일 초과(O48)등 으로 태아이상여부 확인위해 추가적인 초음파 검사시행합니다. \highlight[cyan!70]{부분은 산부인과확회추천}
\end{commentbox}
\Large \textcolor{red}{최대한 숫가가 높은 코드를 본후 추가되는 것은 EB511 / EB515로 봅니다.} 심장 초음파는 예외입니다 \normalsize
\prezi{\clearpage}
\subsection{제한적 초음파 적용에 대한 심평원답과 산부인과학회의 답}\par
%\begin{mdframed}[linecolor=red,middlelinewidth=2]  
\begin{itemize}\tightlist
\item 고시 내용
	\begin{itemize}\tightlist
	\item 치료 전•후와 같이 환자 상태변화를 확인하기 위하여 이전 초음파영상과 비교목적으로 시행할 경우 산정 
	\item 해당검사 소정점수의 50\%를 산정(산정코드 세 번째 자리에 1로 기재)
	\end{itemize}
\item 산부인과 학회
	\begin{itemize}\tightlist
	\item 기형아 정밀계측 초음파 검사 후 F/U 검사하는 경우도 태아에게 이상이 있는 경우이므로 횟수제한 초과 일반 또는 일반의 제한적 초음파로 급여 산정(고시 또는 \textsf{Q\&A}에 없는 내용)
	\item 일반 초음파 \uline{확인 사항 모두 포함하여 태아 기형 F/U 하는 경우}: \highlightR{일반 초음파}
	\item 해당 \uline{태아기형만 F/U 하는 경우}: \highlightR{일반의 제한적 초음파} (아래 ‘제한적 초음파’ 참조)
	\item 단, 횟수제한 초과 초음파를 시행한 사유와 초음파 결과를의무기록에 반드시 기재해야 함. (고시 또는 \textsf{Q\&A}에 없는 내용)
	\item 횟수제한 초과 일반초음파는 고위험 가산 적용 안 됨 
	\end{itemize}
\item 심평원 해당 삼분기의 일반 또느 일반의 제한적 초음파의 범위는 별도로 정하고 있지 않으며, 환자의 개별상태에 대한 의학적 판단에 따라 \uline{유선으로 안(16.9.27)내} 한 바와 같이 사례별로 판단되어져야 하며, 관련 급여기준을 아래와 같이...
	\begin{itemize}\tightlist
	\item 산전 초음파 7회 인정은 일반 초음파 100\% 산정가능하다. 7회 초과되는 경우 일반, 제한적 초음파 선택은 병의원 몫이다. 산과관련 제한적 초음파 기준은 따로 없다.
	\item 태아심장에 이상이 있는 경우는 EB436 태아정밀 심초음파를 산정하는 경우 추가(추적검사)산정은 EB436002(제한적) 태아정밀 심초음파로 하고 이는 임신중 7회에 포함되지 않는다.
	\end{itemize}
\end{itemize}	
%\end{mdframed}
\prezi{\clearpage}

\subsection*{7회 이상 초음파가 가능할것 같은 진단명}
%출혈, 태동의 현저한 변화, 발열, 복통, 조기진통, 외상, 위해성 약물 노출, 태아 이상,조기양막파수, 분만예정일 초과등입니다.\par
%아래의 진단명과 JX999 기타내역에 적절한 이유 기입.
\begin{itemize}\tightlist
\item 출혈 
	\begin{itemize}[-]\tightlist
	\item O200 Threatened abortion
	\item O209 Haemorrhage in early pregnancy, unspecified
	\item O441 출혈을 동반한 전치태반
	\item O459 태반조기분리출혈
	\item O469 상세불명의 분만전 출혈
%	\item O24.0.Pre-existing diabetes mellitus, insulin-dependent
%	\item O24.1.Pre-existing diabetes mellitus, noninsulin-dependent
%	\item O24.2.Pre-existing malnutrition-related diabetes mellitus
%	\item O24.3.Pre-existing diabetes mellitus, unspecified
%	\item O24.4.Diabetes mellitus arising in pregnancy
%	\item O24.9.Diabetes mellitus in pregnancy, unspecified
	\end{itemize}
\item 태동의 감소\footnote{한국표준질병•사인분류 질병코딩지침서(Ver.2016)}
	\begin{itemize}[-]\tightlist
	\item Z368 기타 출산전 선별검사(other antenatal screening) : 태아 움직임의 감소라는 진단이 있으나 기저 원인이 전혀 기록되어 있지 않으며 환자가 분만하지 않고 집으로 퇴원하는 경우
	\item O368 기타 명시된 태아문제에 대한 산모관리(Maternal care for other specified fetal problems) : 태아 움직임의 감소라는 진단으로 입원하였는데 아무런 기전 원인도 기록되어 있지 않고 재원 중에 분만 한 환자
	\item 태아 움직임의 감소에 대한 기저 원인이 기록되어 있다면, 그 원인을 주진단으로 분류한다.
	\end{itemize}	
\item 발열 (R509)
	\begin{itemize}[-]\tightlist
	\item O230 Infections of kidney in pregnancy
	\end{itemize}
\item 복통 (R1049)
\item 조기 진통 및 분만 (O60) False labour (O47)
	\begin{itemize}[-]\tightlist
	\item O600 분만이 없는 조기진통(preterm labour without delivery)
	\item O601 조기분만을 동반한 조기자연진통(preterm spontaneous labour with preterm delivery)
	\item O602 만삭분만을 동반한 조기자연진통(preterm spontaneous labour with term delivery) : 37주 이전에 진통이 발생하였으나 진통이 중단되었다가 37주 이후에 분만이 이루어진 경우. 진통으로 입원 후 퇴원하였다가 37주이후 재입원하여 분만한 경우는 분만을 위한 입원 에피소드에 O602를 분류한다.
		\begin{itemize}[-]\tightlist
		\item 임신시점을 나타내기 위해 O60X에 다음의 5단위로 세분류한다.
		\item 0. 임신 22주 미만
		\item 1. 임신 22주 이상, 34주 미만
		\item 2. 임신 34주 이상
		\item 9. 상세불명의 임신기간
		\end{itemize}
	\end{itemize}	
\item 외상 S코드(부분별다름)외상
\item 위해성 약물 (O355)	
\item 태아이상 (O28) 산모의 출산전 선별검사의 이상소견(Abnormal findings on antenatal screening of mother) : 산전 선별검사의 이상 소견이 발견되었으나 명확한 진단이 내려지지 않은경우		
\item 조기양막파수
	\begin{itemize}[-]\tightlist
	\item O420 양막의 조기파열 후 24시간 이내에 진통이 시작되면, 
	\item O421 양막의 조기파열 후 24시간 이후에 진통이 시작되면,
		\begin{itemize}[-]\tightlist
		\item 양막조기파열의 시기를 구분하기 위해서 O42X 에 다음의 5단위 코드를 세분류한다.
		\item 0. 초기조산(34주 미만)
		\item 1. 후기조산(34주 이상, 37주 미만)
		\item 2. 만삭(37주 이상)
		\item 9. 상세불명
		\end{itemize}
	\end{itemize}	
\item 지연임신 (O48) 
\item O10-O16.임신, 출산 및 산후기의 부종, 단백뇨 및 고혈압성 장애(O10-O16)
	\begin{itemize}[-]\tightlist
%	\item O10.Pre-existing hypertension complicating pregnancy, childbirth and the puerperium
%	\item O11.Pre-existing hypertensive disorder with superimposed proteinuria
%	\item O12.Gestational [pregnancy-induced] oedema and proteinuria without hypertension
	\item O13 Gestational [pregnancy-induced] hypertension without significant proteinuria
	\item O14 Gestational [pregnancy-induced] hypertension with significant proteinuria
%	\item O15.Eclampsia
%	\item O16.Unspecified maternal hypertension
	\end{itemize}
\end{itemize}
\prezi{\clearpage}
\par
\includegraphics[scale=.8]{limUS}
\prezi{\clearpage}
\subsection{7회 이상 초음파 급여 가능할것 같은 임신부 증상과 청구메모}
\begin{itemize}\tightlist
\item \emph{질출혈} : 산모분 현재 00주 이신 상태로 질출혈이 매우 심하여 추가 초음파 검사하심
\item \emph{태동변화} : 산모분 현재 00주 이신 상태로 어제 저녁부터 태동이 없어 초음파 검사하심
\item \emph{발열} : 산모분 현재 00주 이신 상태로 39도 이상의 고열이 있어 초음파 검사하심
\item \emph{복통} : 산모분 현재 00주 이신 상태로 복부 통증 및 자궁수축이 심하여 초음파 검사하심
\item \emph{조기양막파수} : 산모분 현재 00주 이신 상태로 Watery vaginal discharge가 관찰되어 조기양막파수가 의심되어 초음파 검사하심
\item \emph{외상} : 산모분 현재 00주 이신 상태로 자동차 접촉사고후 배에 심한 압박이 가해져 초음파 검사하심
\item \emph{고혈압, 두통,시야장애,경련, 부종,단백뇨등} : 산모분 현재 00주 이신 상태로 혈압이 높아 PIH 의심되어 초음파 검사하심
\item \emph{기타검사이상} : 간기능, 신장기능 이상, 혈소판 감소등
\item \emph{자궁경관 무력증의심} : 산모분 현재 00주 이신 상태로 과거력상 조산의 과거력이 있고 초음파상 자궁경부의 길이가 짧아져 자궁경관 무력증이 의심되어 초음파 검사 하심
\item \emph{태아기형(수신증등)} : 산모분 현재 00주 이신 상태로 hydronephrosis 있고 양수양이 적어 초음파 검사하심
\end{itemize}
%\Ans{임신 과정 중 의학적 판단 하에 태아에게 이상이 있거나 이상이 예상되어 7회를 초과하여 일반이나 일반의 제한적 초음파를 산정하는 경우에는 “JX999”에 초과 사유를 free text로 기재함.}

%\Que{태아에게 이상이 있거나 이상이 예상 되는 경우의 판단은 순전히 의사에게 만 있는 건지?}
%\Ans{항상 심평원의 경우는 환자측 편 또한 너무나 많은 보험청구의 경우는 삭감되지않을까 하는 걱정! 일단은 청구해 보는것이...}
\prezi{\clearpage}
\subsection{산부인과학회 Q\&A}
\prezi{\clearpage}
1. 자궁경부길이가 짧아 F/U 하는 경우에는 횟수제한 초과 급여 가능한가요?
\begin{quotebox}
자궁경부길이가 짧은 경우 조기진통 또는 자궁경관무력증이 의심되는 상황이므로 횟수제한 초과 급여 가능합니다. 단, 의무기록과 상병에 조기진통 또는 자궁경관무력증을 기재하는 것이 좋습니다. 만약
자궁경부길이만 F/U 하는 경우에는 '제한초음파'로 처방하고, 일반초음파에 해당하는 검사를 다 하면서 자궁경부길이도 보는 경우에는 일반 초음파 수가를 처방하면 됩니다.
\end{quotebox}
2. 짧은 자궁경부길이의 기준은 무엇인가요?
\begin{quotebox}
자궁경부길이가 2.5 cm 이하인 경우입니다.
\end{quotebox}
\prezi{\clearpage}
3. 임신 40 주 이후는 횟수제한 초과 급여에 포함되나요?
\begin{quotebox}
임신 40 주 이후만으로는 횟수제한 초과 급여에 해당되지 않습니다. 횟수제한 초과 급여는 태아에게 이상이 있거나 이상이 의심되는 경우에 가능합니다. 임신 40 주 이후이든 이전이든 임신 주수와
상관없이 태아의 이상(예: 과대체중아(LGA), 과소체중아(SGA), 양수과소증, 태아 곤란증 등)이 의심되어
초음파를 시행한 경우에는 횟수를 초과해도 급여가 됩니다. 단, 이 경우에 의무기록에 초음파를 시행한
이유를 잘 기록하고 관련 초음파 기록을 남기고, 해당 상병을 입력해야 합니다. 
\end{quotebox}
\prezi{\clearpage}
\subsection{초음파 고위험군 산모는 모두 산정외초음파가 가능하는지?}
\textcolor{blue}{산정초과 태아초음파의 경우는 태아에게 이상이 있거나 이상이 예상되는 경우}등으로 주로 태아쪽에 focus가 있읍니다. (예외적으로 Cx. length가 짧은경우도 속합니다.) \par
거기에 비해서 \textcolor{red}{고위험임신이라고 하는것은 임산부의 질환이나 자궁의 이상, 태반이상, 양수이상등의 태아외적인 환경인 경우}입니다. 예외적으로 SGA만 태아와 관련되어 있습니다. \par
물론 초음파에서 고위험산모의 경우는 태아의 이상이 있거나 이상이 예상되는 경우가 많습니다. 
\begin{enumerate}[1.]\tightlist
\item 태아에게 문제를 초래하는 임부이 질환이므로 \highlight{DM, PIH, thrombocytopenia등은 적절히 JX999에 이유}를 기입하면 될것 같습니다.
\item 자궁의 이상중에 \highlight{IIOC는 자궁경부길이 측정}등의 이유로 가능할것 같습니다.
\item 태반이상중 \highlight{previa등은 출혈등이 있을때는 가능}할것 같습니다. 
\item \highlight{양수과소증이나 과다증이나 SGA는 가능할것 같습니다. 태아 안녕 평가목적}으로 가능할것 같습니다. 
\end{enumerate}
결론적으로 적절한 JX999에 기입하면 될수 있는 상병이 많지만 다 되는 것은 아닙니다.
%물론 JX999에 어떻게 기입하느냐에 따라서 모두 가능할수 있지만, 
%여태까지 생각없이 아니면 본인의 소신대로 적응증이 되지 않는다 생각하거나, 심평원의 삭감이 무서워서 안 내고 있을수 있는 산부인들도 보호해야 한다고 생각합니다
\begin{commentbox}{고위험 산모인 경우 초과 초음파를 산정할수 있는 기준}
고위험 산모의 진료라고 해서 일괄적으로 초과 초음파를 산정 할 수는 없다. 단 의학적인 판단하에 정말 태아에서 이상이 있거나 예상이 되오 초음파를 시행한 경우에는 초과 초음파를 산정할 수 있다.
\end{commentbox}
\begin{commentbox}{태아에게 문제를 초래하는 임부의 질환}
임신성 당뇨, 고혈압이외에 태아에게 문제를 초래하는 임부의 질환상태라고 하면 고위험 산모에 포함되기는 하나 의사마다 의견이 다를 수 있으니 시간이 지난 후 자문이 필요한 내용인거 같다.
\end{commentbox}
