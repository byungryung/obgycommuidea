\section{Gonorrhea}
\myde{}{%
\begin{itemize}\tightlist
\item[\dsjuridical] A638 기타 명시된 주로 성행위로 전파되는 질환(유레아플라스마)
\item[\dsjuridical] A542 임균
\item[\dsjuridical] \sout{B4101 미생물현미경검사(일반염색, Gram stain)} D5802 관찰판정-현미경-일반염색 (Gram stain)
\item[\dschemical] D591112C	핵산증폭-정성그룹1 Neisseriagonorrhoeae	32,170 STD1종Single
%\item[\dschemical] C5956006 종합효소연쇄반응(기타) [\myexplfn{415.49} 원]
%\item[\dschemical] C6014006 하부요로생식기 및 성매개 감염원인균(다중종합효소연쇄반응법) STD 6종, STD 12종
%\item[\dschemical] C5896006 하부요로생식기 및 성매개 감염원인균(다중실시간 종합효소연쇄반응법) STD 6종(RT-PCR), STD 7종(RT-PCR)
\item[\dschemical] D6802016 누680나 핵산증폭-다종그룹2 하부요로생식기및성매개감염원인균[다중실시간중합효소연쇄반응법] STD7종Real-Time 76,640원
%\item[\dschemical] \sout{C5896006 하부요로생식기 및 성매개 감염원인균(다중실시간 종합효소연쇄반응법) STD 6종(RT-PCR), STD 7종(RT-PCR)}
\item[\dschemical] D6802026 누680나 핵산증폭-다종그룹2 하부요로생식기및성매개감염병원체[다중중합효소연쇄반응법] STD12종multiplex 76,640원
\end{itemize}
}
{
\emph{Ceftriaxone 보험 사용으로는 삭감우려가 있음.}
세프트리악손을 급여로 사용하고자 한다면\par
세프트리악손 약전에 있는 효능 및 효과[허가사항범위]
\begin{itemize}\tightlist
\item 폐렴, 기관지염 등 호흡기계 감염증, 
\item 이비인후과 감염증, 
\item 신장 및 요로감염증, 
\item 임질 등 생식기 감염증, 
\item 패혈증, 
\item 수술 전후 감염예방, 
\item 골 및 관절 감염증, 
\item 피부, 상처 및 연조직감염증, 
\item 복막염, 담낭염, 담관염 등 위장관감염증, 
\item 면역기능저하 환자의 감염증, 수막염.
\end{itemize}
청구메모>>  \emph{"1세대 2세대 세파에 효과가 없고 경구 투약만으로 치료효과를 기대할 수 없어 세프트리악손 사용함"} 이라고 적어 주시면 될 것입니다.
}
\begin{commentbox}{Recommended Regimen}
Ceftriaxone 250 mg IM in a single dose \par
PLUS Azithromycin 1g orally in a single dose \par
As dual therapy, ceftriaxone and azithromycin should be administered together on the same day, preferably simultaneously and under direct observation.
\end{commentbox}

\subsection{효과적인 medication}
\begin{enumerate}\tightlist
\item 비급여 Ceftriaxone를 사용. : 대한뉴팜 세프트리악손 주 500mg, 세트리손 주 500mg, 슈넬 세프트리악손 주 500mg
\item 지스로맥스캅셀(아지스로마이신)  ZITHROMAX CAPS.[Azithromycin] 250mg 4T QD 
\end{enumerate}
