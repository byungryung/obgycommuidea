\section{Vaginal laceration}
\myde{}{%
\begin{itemize}\tightlist
\item[\dsjuridical] S314 질, 외음부 열린상처
\item[\dsmedical] SB021[\myexplfn{182.65} 원] - SC021(debridement)[\myexplfn{240.18} 원] 창상봉합술 (2.5cm미만)
\item[\dsmedical] SB022[\myexplfn{213.28} 원] - SC022(debridement)[\myexplfn{300.72} 원] 창상봉합술 (2.5cm미상-5cm미만)
\item[\dsmedical] SB023[\myexplfn{349.22} 원] - SC023(debridement)[\myexplfn{405.45} 원] 창상봉합술 (5cm미상이거나, 근육에 달하는것)
\item[\dsmedical] 봉합사 청구
\item[\dsmedical] 하이퍼테트주 - 파상풍, KK045
\item[\dsmedical] M0111 단순처치[1일당] : 추적검사시
\end{itemize}
}
{주:1. 근접하지 아니한 여러 부위에 창상봉합술을 시행하는 경우에는 전신을 두부, 복부, 배부, 좌․우․상․하지의 7부위로 구분하여 각 부위 별로 소정점수를 각각 산정한다.\par
2. “주1"의 각 부위내에 창상봉합부위가 둘 이상일 때 여러 창상봉합부위가 4″×4″거즈 범위내에 포함되는 경우에는 제1범위 분류항목을 산정하고, 4″×4″거즈 한장 범위를 초과하는 경우에는 두장째 범위부터 1범위당 제2범위의 분류항목으로 각각 산정한다. 즉 SB024,SB025,SB026과 SC024,SC025,SC026으로 합니다.}
