\subsection{행정조사 종류}
\begin{itemize}\tightlist
\item 복지부 실사
\item 공단 조사
\item 심평원 조사 
\item 보건소 조사 
\item 소방서, 세무서, 식약청 등 등
\end{itemize}

\subsection{현지조사 업무 수행}
\textbf{업무수행}
\begin{itemize}\tightlist
\item 보건 복지부
	\begin{itemize}\tightlist
	\item 현지조사 법적 권한, 심평원 및 건보공단 지원 받아 현지조사 관장
	\end{itemize}
\item 건강보험심사평가원
	\begin{itemize}\tightlist
	\item 대상기관선정, 조사 실시, 정산 심사, 행정 처분, 사후 관리 등
	\item 현지조사 업무 전반 지원
	\end{itemize}
\item 국민건강보험공단
	\begin{itemize}\tightlist
	\item 수진자 조회 등 현지조사 업무 지원
	\end{itemize}	
\end{itemize}

\textbf{조사반 구성}
\begin{itemize}\tightlist
\item 의원급(약국 ) : 3명 x 3일 
\item 병원급 : 4명 x 5일
\item 종합병원급 이상 : 5명 이상 x 6일~14일
\end{itemize}
\subsection{심사 및 현지확인과 비교}
\par
\medskip

\tabulinesep =_2mm^2mm
\begin{tabu} to \linewidth {|X[1,c]|X[1,c]|X[2,l]|X[4,l]|} \tabucline[.5pt]{-}
\rowcolor{Gray!25}  구분 &  주관  & 법적 근거 & 조치 사항 \\ \tabucline[.5pt]{-}
\rowcolor{Yellow!5} 현지조사 & 보건복지부 & 법 제97조 제2항 & 부당금액환수\newline 행정처분\newline 형사고발 \\ \tabucline[.5pt]{-}
\rowcolor{Yellow!5} 방문심사 & 심사평가원 & 법 제47조 제2항\newline 시행규칙 제20조 & 심사조정\newline 현지조사 의뢰 \\ \tabucline[.5pt]{-}
\rowcolor{Yellow!5} 현지확인 & 건보 공단 & 법 제14조, 제57조 & 부당금액[환수]\newline 현지조사 의뢰 \\ \tabucline[.5pt]{-}
\end{tabu}
\par
\medskip

\includegraphics[width=\textwidth]{silsaflow}

\subsection{공단 조사}
\begin{enumerate}[1.]\tightlist
\item 어디에서 나왔나?
	\begin{description}
	\item[대답:] ``공단에서 나왔습니다”
	\item[대응:] 공문 봅시다  (자세히 읽어본다)
	\end{description}
\item 조사성격? 증거수집 목적이 강하다
\item 상대방 무기 - \highlightR{공단조사  불응시  복지부에  실사 요청  권한만 있다}
\end{enumerate}
공단의 요양기관에 대한 조사권  → 조사협조요청이다 \par

문제제기
 \begin{enumerate}[1)]\tightlist
 \item 요양급여에 관한 조사 권한이 어디 있나? 유사하거나 동일한 사안은 공동조사 해야(행정조사기본법 4조3항, 14조)
 \item 왜 사전 통보 안 했나? (행정조사기본법  17조 - 7일전 통지의무)
 \item 조사를 한 사안에 대하여 다른 기관에서 조사대상자를 재조사하면 안 된다 (행정조사기본법 15조 중복조사제한) 건보공단 조사 후 심평원 실사 의뢰 동일 사안 재조사
 \item 조사내용과 범위를 왜 고지하지 않나?
 \end{enumerate}
    
 조사 불응시 공단의 무기? → 없다
 \begin{enumerate}[1)]\tightlist
 \item 현지조사 의뢰 가능(조사 응하면 현지조사 없냐? 전혀 그렇지 않음)
 \item 타협해서 좋은 결과 거의 보지 못함 
 \end{enumerate}

공단조사시 타협으로 좋은 결과 얻기 힘든 이유 
\begin{enumerate}[1)]\tightlist
\item 그들의 방문 목적 : 공단 실적        자기 실적
\item 타협 당시 전혀 행정처분 내용을 알 수 없음
\item 조사 이후의 진행 절차는 그들의 권한이 아니므로 조사원이 그 이후를 보장할 수 없다 
\item 즉 타협은 그들의 회유 술책이다!  
\end{enumerate}

공단이 조사 목적과 범위를 밝히지 않는 조사 : Ex) 차트 일체, 수납대장 \par
\begin{description}
\item[일체] 미련 갖지 말고 차라리 실사로 가라!
\item[사유] 차분히 대비할 수 있다 (공부하고 시험치는 것이 낫다)
\end{description}
\subsection{현지조사 절차}
\begin{description}\tightlist
\item[서면조사] 조사원이 조사대상기관에 현장 방문하지 않고 요양급여 사항에 관한 보고 또는 관련 자료를 제출하도록 요구하여 요양급여비용 청구의 적법 타당성을 조사하는 방식
\item[현장조사] 조사원이 조사대상기관에 현장 방문하여 요양급여비용 청구의 적법 타당성을 조사하는 방식
\end{description}

\textbf{사전통지 실시} '선정심의 위원회'에서 증거인멸 등의 우려가 없다고 심의한 요양기관\par
\menu{사전통지 > 도착 > 신분증 제시 > 조사명령서 전달 > 조사내용 통보}\par

\textbf{사전통지 미실시} '선정심의 위원회'에서 증거인멸 등의 우려가 있다고 심의한 요양기관\par
\menu{도착 > 신분증 제시 > 조사명령서 전달 > 조사내용 통보}\par

\includegraphics[width=\textwidth]{silsacommand}\par

\includegraphics[width=\textwidth]{silsacommand2}

\subsection{복지부실사}
\begin{enumerate}[1.]\tightlist
\item 어디서  나왔나?  심평원 → 공문 봅시다
\item 조사성격 : 계도가 아니라 처벌을 목적
\item 거부시 -  국민건강보험법 98조2항   영업정지 1년
\end{enumerate}
심평원 실사 대응
\begin{enumerate}[1.]\tightlist
\item 전체 과정 이해 : 현지조사 이후 3년정도의 쟁송기간
	\begin{enumerate}[⓵]\tightlist
	\item 3-4일 조사 후 돌아감 →
	\item 3-6개월 후 사전처분서 →
	\item 이의신청 →
	\item 3-6개월 후 행정처분(면허정지, 과징금, 업무정지) →
	\item 행정소송 1심 →
	\item 행정소송2심 →
	\item 대법원 
	\end{enumerate}
\item 3-4일 조사시(심리상태: 두려움, 회피), 사전처분서 받았을 때 (심리상태: 분노, 대책찾음), 행정처분시 (절망)
\item 회원들 문제점 – 3단계, 4단계에서 대책을 세우고 연락한다. \highlightR{예후: 굉장히 나쁘다}
\item 핵심: 1단계인 3-4일 조사에서 대책을 세우고 잘 대응해야 한다. 
\end{enumerate}
관계서류 조사
\begin{commentbox}{관련근거}
\begin{itemize}\tightlist
\item 국민건강보험법시행규칙 제58조(서류의보존)
\item 요양급여의기준에 관한규칙 제7조
\end{itemize}
\Large{5년간 보존}
\begin{itemize}\tightlist
\item 요양급여비용청구서 및 명세서
\item 약제 및 치료재료 구입에 관한 서류
\item 요양급여비용의 산정에 필요한 서류 및 이를 증빙하는 서류
\item 계산서ㆍ영수증 부본 또는 본인부담금수납대장
\end{itemize}
\end{commentbox}

\begin{itemize}\tightlist
\item 관계서류제출명령위반
\item 거짓보고
\item 조사 거부·방해·기피
\end{itemize}
-> \highlightY{업무정지1년, 1천 만원 이하 벌금}\par
\begin{commentbox}{서류제출 명령 위반의 범위}
\begin{itemize}\tightlist
\item 현지조사 결과(요양급여 관계서류 제출 명령 위반) - 조제기록부, 본인부담금수납대장 등을 제출하지 않음
\item 요양기관 주장내용 - 조제기록부 등 일체의 서류를 작성∙보관하지 않음을 이유로 서류를 제출 하지 못한 경우 이를 서류제출 명령 위반이라고 볼 수 없음
\item 판결요지 - 관계 법령상의 관계서류 작성 및 보존의무의 존재를 알고 있음에도 위반하여 관계서류를 작성하지 아니하여 제출하지 못하게 된 것인바, 서류제출 명령 위반에 해당함
\item ※ \textcolor{blue}{요양기관이 해당 관계서류를 작성하지 아니하였는지 여부와는 무관 하게 서류제출명령 위반에 해당함}
\end{itemize}
\end{commentbox}

현지실사 대비
\begin{enumerate}[1.]\tightlist
\item 평소 사전 지식을 최대한 알아 두라 (유비무환)
\item 실사 닥쳤을 때 당황하지 말고 멘토와 반드시 상의하라!
\item Yes  or  No 가 아니라 Wait 도 있음
\item 3-4일동안 돈 벌려고 하지 마라!
\item \highlightR{사실확인서 (원장\&직원) 조심} : 쓰지 않는 것이 좋다(확인자의 권리) → 본대로 처분해라
    작성의미는 해당 부분에 대해 향후 이의신청, 행정소송 등의 모든 권리를 포기하는 의미 
\end{enumerate}

사실확인서 (매우 중요)
\begin{itemize}\tightlist
\item 3-4일 조사의 핵심으로 조사관이 돌아가기 전에 요구 : 위의 전체 과정 중 2단계 이후의 모든 것을 포기하겠다는 의미
\item 헌법12조2항   누구든지 자기에게 불리한 진술을 하지 않을 권리, 헌재 판례 : 형사처벌에 연결될 가능성이 있는 행정조사도 포함 
\item 대응책:  \highlightY{보신대로 처분해라,  사실여부를 떠나 안 하고 싶다.} 
\end{itemize}

현지조사시 사실확인서 타협으로 좋은 결과 얻기 힘든 이유 
\begin{enumerate}[1)]\tightlist
\item 그들의 방문 목적 : 처벌(실적) 목적 (o) , 계도 목적(X)
\item 타협 당시 전혀 행정처분 내용을 알 수 없음
\item 조사 이후의 진행 절차는 그들의 권한이 아니므로 조사원이 그 이후를 보장할 수 없다 
\item 사실상 이의절차를 포기하겠다는 의미 
\item 즉 타협은 그들의 실적 확보를 위한 회유 술책이다!  
\end{enumerate}

현지실사의 대응팁
\begin{enumerate}[1)]\tightlist
\item 중요한 것은 원장이 직접 해야 : 의료기관 내부 고발이 많음
\item 평소에 최대한 규정에 맞게 : 나중에 6배 환수
\item 입증 책임은 복지부에게 있음 :  EX) BMD를 간호사가 했다?  → 모든 사례에 대해 건건이 복지부가 누가 했는지 입증을 해야 함 -> 생각없이 포괄 확인서 쓰면 안 됨 
\item 실사 왔을 때는 진료를 최소화하고 실사에 집중해야!
\end{enumerate}
