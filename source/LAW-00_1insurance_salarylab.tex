\begin{mdframed}[linecolor=blue,middlelinewidth=2]  
제1편 행위 급여 \cntrdot{}  비급여 목록 및 급여 상대가치점수 >> 제2부 행위 급여 목록\cntrdot{} 상대가치점수 및 산정지침 >> 제2장 검사료
\end{mdframed}

\subsection{\newindex{검사료〔산정지침〕}}
\begin{enumerate}[(1)]\tightlist
\item 제2장에 기재되지 아니한 검사로서 외관, 취기, 색도 등의 간단한 검사 또는 계산방법에 의하여 검사치를 얻는 경우에는 검사료를 산정하지 아니한다.
\item \uline{대칭기관에 대한 양측검사를 하였을 때에도 “편측”이라는 표기가 없는 한 소정점수만 산정한다.}
\item \uline{검사에 사용된 약제 및 재료대(1회용 주사침 및 주사기 포함)는 소정점수에 포함되므로 별도 산정하지 아니한다. 다만, 다음의 경우에는 “약제 및 치료 재료의 비용에 대한 결정기준”에 의하여 별도 산정한다.}
	\begin{enumerate}[(가)]\tightlist
	\item 인체에 주입된 약제
	\item 부하시험시 사용된 약제
	\item 안기능검사시 사용된 필름, 형광물질, 사진현상 및 인화료
	\item 내시경검사시 사용된 슬라이드 필름 및 사진현상료, 포라로이드필름 또는 칼라프린터 인화지
	\item 핵의학 기능검사시 사용된 방사성 동위원소 및 약제
	\item 제2장 분류항목에 별도로 규정한 약제 및 재료대
	\item 기타 장관이 별도로 인정한 약제 및 재료대
	\end{enumerate}
\item \uline{인체에서 채취한 가검물에 대한 검사를} “(부록) 검체검사 위탁에 관한 기준"에서 정한 \uline{수탁기관으로 위탁하는 경우에는} 제2장 제1절 및 제2절 분류항목 소정점수(가감률 적용 포함)에 \uline{수탁기관의 점수당 단가를 곱하여 계산한 금액의 10\%를 “위탁검사관리료"로 산정한다}
\item (별표)에 열거한 항목은 다음 중 어느 하나에 해당하는 경우에만 산정하되,\uline{㈎, ㈏ 및 ㈐의 경우에는 소정점수의 10\%를 가산하여 산정}한다.(산정코드 세 번째 자리에 6으로 기재)
	\begin{enumerate}[(가)]\tightlist
	\item \uline{진단검사의학과 전문의가 판독하고 판독소견서를 작성\cntrdot{} 비치한 경우}
	\item B세포 표면면역글로불린, 세포표지검사, 면역조직(세포)화학검사, 세포 주기 및 핵산분석검사(유세포측정법), 분자병리검사에 대하여 병리과 전문의가 판독하고 판독소견서를 작성\cntrdot{} 비치한 경우
	\item \uline{분자병리검사에 대하여 관련분야에 대하여 인증 받은 전문의가 판독하고 판독소견서를 작성\cntrdot{} 비치한 경우}
	\item 면역조직(세포)화학검사에 대하여 구강병리과가 설치된 요양기관의 치과의사가 판독하고 판독소견서를 작성\cntrdot{} 비치한 경우
	\end{enumerate}
\end{enumerate}
	
