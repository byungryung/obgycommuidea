\section{PROM검사}
\myde{}{
\begin{itemize}\tightlist
\item[\dsjuridical] O42  양막의 조기 파열
\item[\dsjuridical] O4290 상세불명의 양막의 조기 파열, 조산
\item[\dsjuridical] O4291 상세불명의 양막의 조기 파열, 만삭
\item[\dschemical] B0561 태아피브로넥틴정량검사 / 234.85점 / \myexplfn{234.85} 원
\item[\dschemical] BX013 태아 피브로넥틴 정성검사 [현장검사] / 476.11점 /  \myexplfn{476.11} 원
\item[\dschemical] E7330 니트라진검사 / 30.32점 / \myexplfn{30.32} 원
\item[\dschemical] B0562 태반알파마이크로글로불린-1[현장검사] / 214.52점 / \myexplfn{214.52} 원[AMNISURE TEST]
\item[\dschemical] BX014 인슐린양성장인자결합단백질-1 정성검사[현장검사] / 214.52점 / \myexplfn{214.52} 원
\end{itemize}
}
{\begin{itemize}\tightlist
\item 태아 피브로넥틴 정량검사의 인정기준
\item 조기 양막파수 진단을 위해 시행하는 나56-1 태반알파마이크로글로불린-1 검사, 너-14 인슐린양성장인자결합단백-1 검사 인정여부 
\item 자궁경관점액양치상검사 Fern test of cervical mucus
\end{itemize}
}

\prezi{\clearpage}
\leftrod{태아 피브로넥틴 정량검사의 인정기준}
\begin{enumerate}[1.]\tightlist
\item 나56 태아 피브로넥틴 정량검사는 조기 진통의 발견, 조기 양막파수와 조산의 가능성을 평가하는데 유용한 검사로써 임신 만22주이상 34주이하의 임산부에서 인정기준은 다음과 같이함.\\
-다 음-
	\begin{enumerate}[가.]\tightlist
	\item 조기분만의 경력이 있는 경우
	\item 자궁기형이나 자궁근종 등의 병변이 확인된 경우
	\item 다태임신으로 진통이 있는 경우
	\item 수술이나 Damage 등으로 자궁경부에 30\%이상의 결손이 있는 경우
	\item Monitor상 자궁수축이 명확한 경우
	\item 자궁경관무력증 등으로 자궁경부결찰술을 받은 경우
	\item 기타 동 검사의 실시 사유가 명확한 진료담당의사 소견서를 첨부한 경우
	\end{enumerate}
\item 다만, 임신 만34주를 초과하여 동검사를 실시한 경우 검사료는 전액본인부담토록 함. (2008.1.1 시행)
\item 태아 피브로넥틴 정성검사 Fetal Fibronectin Qualitative Test : 너-13 태아피브로넥틴 정성검사[현장검사] 소정점수를 산정함.(고시 시행일 : 2012.12.1.)
\end{enumerate}
\par
\medskip

\prezi{\clearpage}
\leftrod{조기 양막파수 진단을 위해 시행하는 나56-1 태반알파마이크로글로불린-1 검사, 너-14 인슐린양성장인자결합단백-1 검사 인정여부}\par
\uline{분만진통을 동반하지 않은 조기 양막파수(PROM)시 일차적으로 나-733 니트라진 검사를 시행하고, 임상적으로 추가 검사가 필요하다고 판단되는 경우 나56-1 태반알파마이크로글로불린-1 검사 또는 너-14 인슐린양성장인자결합단백-1 검사를 시행하는 것이 바람직하므로},나56-1 태반알파마이크로글로불린-1 검사 또는 너-14 인슐린양성장인자결합단백-1 검사는 이차적으로 시행 시 인정함.
\begin{itemize}[☞]\tightlist
\item 신설 사유 조기양막 파수(PROM) 진단을 위해 시행하는 나56-1태반알파마이크로글로불린-1 검사, \textcolor{red}{너-14 인슐린양성장인자결합단백-1 검사}는
\item 이차적 검사로 인정하는 기준을 마련함. 시행일 : 2011년3월1일 진료분부터 적용
\end{itemize}

\prezi{\clearpage}
\Que{임신시 조기파수(Premature rupture of membrame), 조기진통(preterm laber)시 일차적으로 나-733 니트라진 검사를 시행하고 있습니다. \par 나-733 니트라진 검사의 급여(보험)적용되는 임신 주수 기준이 있나요? 인정 횟수와 간격기준이 있나요?}
\Ans{나-733 니트라진 검사와 관련하여, 동 검사는 \textcolor{red}{분만진통을 동반하지 않은 조기 양막파수(PROM)시 일차적으로 시행토록 하고 있으며, 인정 횟수와 간격기준에 대해서는 급여기준 및 내부규정 등에 정해진 부분은 없으나, 필요\cntrdot{}적절}하게 이루어졌는지에 대해 진료내역 참고하에 사례별 심사가 이루어질 수 있음을 알려드립니다. }

\prezi{\clearpage}
\leftrod{자궁경관점액양치상검사 Fern test of cervical mucus}\par
보험분류번호 : 너891, 보험EDI코드 : EX891
\begin{description}\tightlist
\item[적응증] 일반적으로 산모의 양막파열 여부는 질강 내에서 양수의 존재를 확인하는 것이나, 양수의 양이 적은 경우에는 양수의 존재를 증명하기 어려우며 또한 양수가 혈액, 태변 등으로 오염되어 있는 경우에는 양수를 확인하기 어렵게 됨. 이러한 경우에는 양수내에 포함되어 있는 고농도의 여성호르몬의 영향을 받아 자궁경부 점액으로 fern형성 여부를 검사하여야 함. 불임 - 정상적으로 자궁경관에서 estrogen 이 분비되는지 확인하는 검사
\item[실시방법] \href{http://www.hira.or.kr/rd/reval/revalueActDefView.do?pgmid=HIRAA030075000000}{행위정의}
\item[전형적 사례]
	\begin{itemize}\tightlist
	\item 성별/연령: 여자/28세
	\item 상병명: 무월경 34주. 임신 34주. 평소보다 많은 질강내 액체로 조기양막파수 의심되어 검사 실시함.
	\end{itemize}
\end{description}
