\leftrod{제6차 개정 한국표준질병사인분류 코딩지침서}\par
 다음의 내용 \href{http://kostat.go.kr/kssc/common/CommonAction.do?method=download&attachDir=bm90aWNl&attachName=JUVEJTk1JTlDJUVBJUI1JUFEJUVEJTkxJTlDJUVDJUE0JTgwJUVDJUE3JTg4JUVCJUIzJTkxJUVDJTgyJUFDJUVDJTlEJUI4JUVCJUI2JTg0JUVCJUE1JTk4XyVFQyVBNyU4OCVFQiVCMyU5MSVFQyVCRCU5NCVFQiU5NCVBOSVFQyVBNyU4MCVFQyVCOSVBOCVFQyU4NCU5QyUyODIwMTIuMDMlMjkucGRm}{제6차 개정 한국표준질병사인분류 코딩지침서}에 따릅니다.\par
유용한 KCD code를 찾을수 있는 Link입니다. \url{http://www.kcdcode.co.kr/}
\begin{itemize}[▷]\tightlist
\item 산과진단코드 부여 순서
	\begin{enumerate}\tightlist
	\item \uline{제왕절개나 기구를 사용하여 분만한 경우} \textcolor{red}{중재술을 하게 된 원인이 되는 병태}를 주된 병태로 부여한다.
		\begin{mdframed}[linecolor=blue,middlelinewidth=2]
			\begin{itemize}\tightlist
			\item 임신성당뇨로 유도분만하였으나 7시간 진통후 CPD로 제왕절개하여 건강한 아이 분만한 경우 :  주된병태 : O65.4 (CPD) 기타병태 : O24.4 (GDM), Z370 (Single live birth) 
			\item Full dilatation후 태아의 P position으로 vaccume extraction시도 했으나 태아의 하강이 없어 흡입분만 포기하고 제왕절개분만한 경우 :  주된병태 : O64.0 (Obstructed labour d/t incomplete rotation of fetal head) 기타병태 : O66.5 (상세불명의 집게및 진공흡착기 적용실패), Z370 (Single live birth) .\index{진단코드!POPP}
			
			\end{itemize}
		\end{mdframed}
	\item \uline{기구의 도움을 받지 않고 질식분만을 하였으나 산모가 출산전 산전병태로 입원}하였다면 산전병태를 주된병태로 분류한다. \textcolor{red}{유도분만의 이유가 주 진단이다}
		\begin{mdframed}[linecolor=blue,middlelinewidth=2]
		\begin{itemize}\tightlist
		\item 임신성고혈압으로 induction delivery하여 1st degree laceration있는 경우는  : 주된병태 : O13 (PIH) 기타병태 : O70.0 (1st degree laceration), Z370 (Single live birth) 
		\end{itemize}
		\end{mdframed}
		
	\item \uline{정상 분만진통으로 입원하여 정상질식분만을 한경우}에는 \textcolor{red}{분만을 주진단으로} 선정한다.
	\begin{mdframed}[linecolor=blue,middlelinewidth=2]
	\begin{description}\tightlist
	\item[O800] 자연두정태위분만
	\item[O801] 자연둔부태위분만
	\item[O814] 진공흡착기분만
	\item[O840] 모두질식분만에의한 다태분만
	\item[O842*] 모두제왕절개에의한 다태분만
	\end{description}
	 \end{mdframed}
	 
	\item 즉 \uline{쌍둥이로 제왕절개분만을 한 경우}는 주된병태는 \textcolor{red}{제왕절개를 한 이유}
	\begin{mdframed}[linecolor=blue,middlelinewidth=2]
	즉. 하나 이상의 태아의 태위장애를 동반한 다태임신의 산모관리(O32.5)이고, 기타병태로 O32.1(Bx 산모관리), O30.0(쌍둥이임신), Z37.2(쌍둥이, 둘 다 생존 출생)등이 있게 된다..\index{진단코드!쌍둥이 제왕절개}
	\end{mdframed}
	\item 분만문제가 진통 전에 발견되었는지, 아니면 진통후에 발견되었는지의 여부에 따라서 `O32-O34'또는 `O64-O66'으로 코딩한다.
		\begin{mdframed}[linecolor=blue,middlelinewidth=2]
		\begin{itemize}\tightlist
		\item 둔위로 Elective c-sec를 하게된경우는 ? 주 진단명이 O32.1 (둔부태위의 산모관리)이지, O64.1 (둔부태위로 인한 난산)이 아니다.
		\item 1분간 지속된 견갑난산을 가진 여아를 질식 분만하였다. 주진단명은 \dotemph{O66.0 (어깨난산으로 인한 난산)} 이다. 
		\end{itemize}
		\end{mdframed}
		
	\item 이전 제왕절개에 따른 분만의 경우에 넣는 코드는 다음과 같다.\index{진단코드!선행제왕절개}
		\begin{mdframed}[linecolor=blue,middlelinewidth=2]
		\begin{description}\tightlist
		\item[O75.7] 이전 제왕절개후 질분만
		\item[O66.4] 상세불명의 분만 시도의 실패 : 위의 두 경우는 TOL(Trial of Labor)를 시도하다가 성공하거나 실패한 경우이고 
		\item[O34.20] 이전의 제왕절개로 인한 흉터의 산모관리
		\item[O34.28] 이전의 기타 외과수술로 인한 자궁흉터의 산모관리 : 위의 두 경우는 Elective로 repeate c-sec를 한 경우로 \dotemph{분만문제가 진통전에 발견되었기 때문에 O32-O34를 쓴다는 원칙을 따른것임.}
		\end{description}
		\end{mdframed}
		
	\item O80-O82의 분류 : 이 코드는 기록되어 있는 정보가 분만이거나 분만 방법에 대해서만 국한되어 있을때 제한적으로 주된병태의 코드로 사용할 수 있다.또한 아무런 문제 없이 정상적인 분만을 하였을때 O80으로 할수 있다.
		\begin{mdframed}[linecolor=blue,middlelinewidth=2]
		\begin{itemize}\tightlist
		\item 산모가 만삭 정상분마을 위해 입원하여 산전, 분만중, 산후에 아무런 합병증이 없고, 기구나 기술을 필요로 하지 않고 정상분만을 한 경우 O80코드를 주진단명으로 할 수 있다.
		\item 특별한 합병증없이 다태분만을 한 경우 주된병태는 ``쌍둥이임신(O30.0)"으로 코드를 부여한다. ``O840" 모두 자연적인 다태분만은 분만의 방법을 나타내 주기 위하여 임의적인 추가코도로 부여 할 수 있다.
		\item 제왕절개술을 받은 경우 선택적이던 응급이던 상관없이 제왕절개술을 받은 이유를 주된병태로 우선 부여한다. 하지만 제왕절개술을 받은 이유가 불명확할 경우 ``O82.-제왕절개에 의한 단일 분만" 코드를 주된 병태로 부여 할 수 있다. \emph{현재와 같이 Repeated c-sec의 이유가 되지는 않는다.}
		\end{itemize}
		\end{mdframed}
	\item 만약 유도분만중 수술한 경우에는 수술한 원인이 주된 병태입니다. 그러나 induction failure는 주된 병태가 될수 없고, induction failure가 생긴 이유가 주된병태입니다. (예로 CPD나 first stage prolongation등)
	\item 모든 분만산모의 경우에는 신생아의 상태를 부진단으로 한다.
		\begin{mdframed}[linecolor=blue,middlelinewidth=2]
		\begin{description}\tightlist
		\item[Z370] 단일생산아 (single liveborn)
		\item[Z371] 단일사산아 (Single stillbirth)
		\item[Z372] 쌍둥이, 둘다 생존 출생 (Twins, both liveborn)
		\end{description}
		\end{mdframed}
	\end{enumerate}
%\item 제왕절개는 \dotemph{제왕절개의 적응증}을 주진단으로 한다. 
\item 입원환자 치료 중 기저질환이 밝혀지면 기저질환을 주진단으로 선정한다. (ex : 병명 - O001 난관임신)
\item 기저질환이 입원 시 알려져 있고, 문제에 대해서만 치료가 이루어지면 그 문제를 주진단으로 선정한다. (ex : 병명 - N833 자궁경부무력증)
\item 급만성이 동시에 발생한 경우 \emph{급성질환}을 주진단으로 선정한다. (ex : 병명 – O140 중등도의 전자간(급성), O249 임신성 당뇨(만성))
\end{itemize}