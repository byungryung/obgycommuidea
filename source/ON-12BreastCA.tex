\section{유방의 악성종양의증}
\myde{}{%
\begin{itemize}\tightlist
\item[\dsjuridical] C509 상세불명의 유방의 악성신생물의증(or 배제진단)
\item[\dsjuridical] N63 달리 명시되지 않은 유방의 결절
\item[\dsjuridical] N644 유방통
\item[\dsmedical] C8506 침생검(표재성)-유방 림프절
\item[\dsmedical] C5912 병리조직검사-생검(4-6개)
\item[\dsmedical] EB562 유도초음파 \myexplfn{887.80} 원
\item[\dsmedical] EB421 유방\bullet 액와부 초음파 Breast\bullet Axilla Ultrasonography \myexplfn{1037.52}원
\item[\dsmedical] 기타 lidocain or labo등 
\end{itemize}
}%
{\begin{itemize}\tightlist
\item 유방조직검사도 청구 가능합니다
\item 유방초음파는 부인과 초음파보다 더 쌉니다. 81,960원 
\item 유도초음파는 자궁내막조직검사 시와 같은 유도초음파 II 입니다
\item 상세불명의 유방의 악성신생물 C509 상병 넣으시면 됩니다.
\end{itemize}
}

\Que{35세 유방암 수술하고 타목시펜 복용중인데 질건조함이 심해 평상시도 불편하고 성교통 및 출혈로 힘들어 하는데 오베스틴 사용하면 안되겠죠?}
\Ans{
\begin{itemize}\tightlist
\item 저같은경우에는 질의문제만 해결하는거고  로칼로만 작용하는거라  쓰는데요 별 문제 없다고 봅니다
\item 전 유방수술한데 물어보라합니다.원래는 상관없는거라고 알고 있는데..전에 한번 처방냈다가 유방암교수님이 뭐라하셔서 저만 나쁜의사 된 이후로는요
\item N수가 많진 않지만,보고된 논문에선 국소치료시 재발과는 큰 영향은 없는걸루. 물론 과다한 사용은 자제해야겠죠
\item 유방전문외과 선생님께 여쭤 보니 국소적으로 쓰는 거는 전혀 문제없다고 하시더군요
\item 방엔 상관없지만 자궁내막은 확인해야된다고되있습니다 기간은 1년인지 6개월인지 기억이안나네요
\end{itemize}
}