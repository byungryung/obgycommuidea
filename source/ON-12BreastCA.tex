\section{유방의 악성종양의증}
\myde{}{%
\begin{itemize}\tightlist
\item[\dsjuridical] C509 상세불명의 유방의 악성신생물의증(or 배제진단)
\item[\dsjuridical] N63 달리 명시되지 않은 유방의 결절
\item[\dsjuridical] N644 유방통
\item[\dsmedical] \sout{C8506 침생검(표재성)-유방 림프절}
\item[\dsmedical] C8641 유방생검(편측)-침생검 FNA, Core Bx, Gun Bx, Mammotome
\item[\dsmedical] C8642 유방생검(편측)-절개생검
\item[\dsmedical] \sout{C5912 병리조직검사-생검(4-6개)}
\item[\dsmedical] 조직병리검사(1장기당) LEVEL B.
\item[\dsmedical] EB562 유도초음파 \myexplfn{887.80} 원
\item[\dsmedical] EB421 유방\bullet 액와부 초음파 Breast\bullet Axilla Ultrasonography \myexplfn{1037.52}원
\item[\dsmedical] M0011036  mission disposable core biopsy instruement *
\item[\dsmedical] M0012036 Trueguide disposable coxaxial biopsy needel *
\item[\dsmedical] 기타 lidocain or labo등 
\end{itemize}
* mission 과 true guide 는 각각 CNB needle 과 guidance
}%
{\begin{itemize}\tightlist
\item 유방조직검사도 청구 가능합니다
\item 유방초음파는 부인과 초음파보다 더 쌉니다. 81,960원 
\item 유도초음파는 자궁내막조직검사 시와 같은 유도초음파 II 입니다
\item 상세불명의 유방의 악성신생물 C509 상병 넣으시면 됩니다.
\end{itemize}
}

\Que{35세 유방암 수술하고 타목시펜 복용중인데 질건조함이 심해 평상시도 불편하고 성교통 및 출혈로 힘들어 하는데 오베스틴 사용하면 안되겠죠?}
\Ans{
\begin{itemize}\tightlist
\item 저같은경우에는 질의문제만 해결하는거고  로칼로만 작용하는거라  쓰는데요 별 문제 없다고 봅니다
\item 전 유방수술한데 물어보라합니다.원래는 상관없는거라고 알고 있는데..전에 한번 처방냈다가 유방암교수님이 뭐라하셔서 저만 나쁜의사 된 이후로는요
\item N수가 많진 않지만,보고된 논문에선 국소치료시 재발과는 큰 영향은 없는걸루. 물론 과다한 사용은 자제해야겠죠
\item 유방전문외과 선생님께 여쭤 보니 국소적으로 쓰는 거는 전혀 문제없다고 하시더군요
\item 방엔 상관없지만 자궁내막은 확인해야된다고되있습니다 기간은 1년인지 6개월인지 기억이안나네요
\end{itemize}
}

\subsection{Cytology 행위료 in OBGY}
\tabulinesep =_2mm^2mm
\begin{tabu} to\linewidth {|X[1,l]|X[1,l]|X[2,l]|X[3,l]|} \tabucline[.5pt]{-}
\rowcolor{ForestGreen!40} 장기   & 코드 & 이름 & 설명 \\ \tabucline[.5pt]{-}
\rowcolor{Yellow!40} 피부 & C8501 & 침생검(표재성) & FNA \\ \tabucline[.5pt]{-}
\rowcolor{Yellow!40} 유방 & C8641 & 유방생검(편측)-침생검 & FNA, core Bx, Gun Bx, Mammotome \\ \tabucline[.5pt]{2-4}
\rowcolor{Yellow!40}  & C8642 & 유방생검(편측)-절개생검 &  \\ \tabucline[.5pt]{-}
\rowcolor{Yellow!40} 갑상선 & C8591 & 갑상선생검-침생검 & FNA \\ \tabucline[.5pt]{-}
\end{tabu}
\par
대칭기관에 대한 양측검사를 하였을 때에도 “편측”이라는 표기가 없는 한 소정 점수만 산정한다 \par

\leftrod{액상흡인세포병리검사의 인정기준}\par

폐, 갑상선, 췌장 종양 진단 목적으로 실시하는 액상흡인세포병리검사는 다음과 같은 경우에 요양급여를 인정함. 다만, 동일 날 실시한 흡인세포병리검사와 중복 산정은 인정하지 아니함.\par
\emph{-다    음-}
\begin{enumerate}[가.]\tightlist
\item 영상의학적 검사에서 이상 소견이 확인된 폐 병변 또는 췌장 병변에 시행한 경우
\item 영상의학적 검사에서 확인된 갑상선 결절에 시행한 경우
\end{enumerate}
보건복지부 고시 제2017-152호(2017.9.1.)\par

액상흡인세포병리검사:C5626
