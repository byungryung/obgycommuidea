\section{미레나}
\myde{}{
\begin{itemize}\tightlist
\item[\dsjuridical] N920 규칙적 주기를 가진 과다 및 빈발 월경	Excessive and frequent menstruation with regular cycle
\item[\dsjuridical] N945 이차성 월경통	Secondary dysmenorrhoea
\item[\dsjuridical] D500 (만성) 실혈에 따른 이차성 철겹핍빈혈	Iron deficiency anaemia secondary to blood loss (chronic)
\item[\dsjuridical] D259 상세불명의 자궁의 평활근종	Leiomyoma of uterus, unspecified
\item[\dsjuridical] N800 선근증
\item[\dsmedical] R4271 자궁내장치삽입술 Insertion of Intrauterine Device [\myexplfn{190.88}원]
\item[\dsmedical] 641100600 미레나 200mcg
\item[\dsmedical] R4277 자궁내장치제거료(실이보이지않는경우) (\myexplfn{605.66} 원) 50\%
\item[\dsjuridical] Ultrasonography(비보험) : 보험으로 하면 미레나는 원가로 청구하며 실 청구는 루프 시술료만 매출생각하면 손해죠. 그래서 초음파 \& 유도 초음파로 보전합니다.
\end{itemize}
}
{levonorgestrel 제제 : 품명(미레나 20mcg))의 요양급여기준은 다음과 같습니다
\begin{enumerate}\tightlist
\item 허가사항중 월경과다증, 월경곤란증, 에스트로겐 대체요법시 프로게스틴의 국소적용에 투여시에는 아래와 같은 기준으로 투여시 요양급여를 인정하며, 허가사항 범위이지만 동인정기준 이외에 투여한 경우에는 약값의 100분의 100을 본인부담토록 함.
	\begin{itemize}\tightlist
	\item 월경과다증으로 확진된 경우
	\item 생리주기당 1-2일간 일상생활이 불가능한 월경곤란증인 경우
	\item 에스트로겐 대체요법시 경구 progestin제제를 사용할 수 없거나 간 등의 대사기능에 문제가 있는 경우
	\end{itemize}
\item 피임 목적으로 투여시에는 비급여대상으로 함.
\end{enumerate}
}
\subsection{인정비급여 항목}
본인이 원하여 자궁내장치삽입술(자427,R4271)을 시술받고 동 장치가 교체하기 위하여 기유치된 자궁내장치를 제거하고 새기구를 재삽입하는 경우는 관련 진찰료 및 시술료등은 비급여 대상임.

\subsection{보)자궁내장치제거}
\myde{}{%
\begin{itemize}\tightlist
\item[\dsjuridical] z3008 출산준비
\item[\dsjuridical] N939 출혈
\item[\dsjuridical] R1039 하복통
\item[\dsjuridical] 루프 위치 변동 등 루프부작용
\item[\dsmedical] R4275 실이 보이는 경우
\item[\dsmedical] R4277 실이 보이지 않는 경우
\item[\dsmedical] R4276 자궁경을 이용한 경우 \footnote{심평원에 자궁경, 방광경 신고된 경우} 
\end{itemize}
}
{}

\subsection{보)자궁내장치삽입술}
의료보험적응증 
\begin{enumerate}[1)]\tightlist
\item 본인이나 배우자가 우생학적 또는 유전학적 정신장애나 신체질환이 있는 경우
\item 임신으로 모성건간을 악화시킬 수 있는 질환이 있는 경우
\item 본인이나 배우자가 태아에 미치는 위험성이 높은 전염성질환이 있는경우
\item 미레나 설치시 : 생리과다, 생리통.. : 청구메모란에 기록이 필요 가 되면 루프제거나 설치단독은 100\%인정
\end{enumerate}

\subsection{보)자궁내장치교체술}
위의 두경우에 해당하는 사유로 루프를 교체하는 경우에는 자427-1 자궁내장치 제거료를 50\% 산정하도록 하고 있음.\\
미레나를 보험교체하는 경우에는 자궁내장치제거 50\%와 주입청구가능합니다.

\subsection{LNG-IUS시의 tip들}
일반적인 spotting과 bleeding양상의 변화과정
\begin{enumerate}\tightlist
\item LNG-IUS 시술후 월경주기에 변화가 나타날 수 있으며, 삽입후 첫 3개월 동안 종종 점상 출혈이 나타나지만 가끔 6개월까지도 지속될 수도 있다.호르몬의 국소작용에 의해 월경량은 점차 줄어 들고 결국은 완전히 없어질 수 있다.
\item LNG-IUS는 초기에 자궁내막의 표면/선상피세포를 억제하고 위축시키면서 탈락막화해서 소혈관 벽이 확장되어 얇아지게 되며 파열되기 쉬운 상태로 변해 파탄성 출혈을 일으킨다.
	\begin{itemize}\tightlist
	\item 12 개월 후에는 혈중 프로게스테론 수치가 192±140pg/mL
	\item 24 개월 후에는 180±66pg/ml, 60개월 후에는 159±59pg/ml로 낮아지게 됨.
	\end{itemize}
\item 시간이 지나면서 자궁내막이 얇아지고 위축됨에 따라 확장된 혈관들이 점차적으로 사라지게 된다.
\item 출혈이 지속되는경우 자궁내막 증식증/자궁내막암, 자궁내 용종/근종, 자궁 내/외 임신, 자궁경부염이나 기타 병변의 가능성도 반드시 고려되어야 하며 초음파, Pap smear, vaginal discharge examination등을 실시하여 감별진단을 시행할 수 있다.	
\end{enumerate}

\begin{commentbox}{bleeding/spotting이 지속되는 경우}
기존에 없는 출혈이 있을시는 반드시 자궁내에 LNG-IUS가 정확한 위치에 있는지 확인해야 한다.\\
문헌상 placebo와 비교해서 더 나은 효과를 보이지 않는다는 문헌도 있지만 환자의 순응도를 위해서 사용해 볼 수 있느 방법으로는 아래와 같다.
\begin{itemize}\tightlist
\item Estradiol valerate 2mg (ex, 프로기노바)투여 :2T/day 또는 에스트로겐 데포 주사(1회)
\item MPA(ex, 프로베라) : 10mg/day로 7일- 14일간
\item Tranexamic acid 3-4T/d (3-5 일) : 혈전증, 신부전증 금기
\item 경구피임약 추가 투여 : 3주 정도 투여함.
\item NSAID : PG 생산 억제, 월경량 30-50\% 감소 월경통 70\% 감소
	\begin{itemize}\tightlist
	\item ibuprofen : 이부프로펜, 캐롤에프, 디캐롤 -> 3T/day
	\item Mefenamic acid :매페남산, 폰탈, 폰스텔 -> 250-500mgX2-4회
	\item Naproxen :  낙센 에프 -> 3T/day
	\end{itemize}
\end{itemize}
\end{commentbox}
