\section{질염관련 균배양검사}
\myde{}{%
\begin{itemize}\tightlist
\item[\dschemical] BY301 가. 마이코플라즈마 Mycoplasma \myexplfn{378.23} 원 
\item[\dschemical] BY302 나. 클라미디아 Chlamydia \myexplfn{357.69} 원  
\item[\dschemical] BY303 다. 유레아플라즈마 Ureaplasma \myexplfn{378.23} 원 
\item[\dschemical] BY309 라. 기타 Others \myexplfn{375.63} 원 
%\item[\dsmedical] 
%\item[\dsmedical] 
%\item[\dsmedical] 
%\item[\dsmedical] 
\end{itemize}
}{
\begin{enumerate}[1.]\tightlist
\item 산부인과 영역에서 시행하는 너 301 기타미생물 배양검사는 다음과 같은 경우에 요양급여를 인정함.\par
- 다   음 - 
	\begin{enumerate}[가.]\tightlist
	\item 골반염의 제 증상(CRP상승, WBC 상승, 복통, 발열 등)이 있는 경우
	\item 임신 제 2분기 이상에서 조산의 위험 증상 (조기 양막파수, 조기 진통 등)이 있는 경우
	\item 질 분비물이 현저히 증가하거나 악취가 나는 등 부인과적 감염이 의심되는 경우	
	\end{enumerate}
\item 상기 1항의 급여대상 이외 산부인과 영역에서 시행하는 경우에는 「요양급여비용의 100분의 100 미만의 범위에서 본인부담률을 달리 적용하는 항목 및 부담률의 결정 등에 관한 기준」에 따라 본인부담률을 80\%로 적용함.(고시 제2016-147호, '16.9.1. 시행)
\end{enumerate} }
