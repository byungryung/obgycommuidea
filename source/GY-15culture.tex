\section{질염관련 균배양검사}
\myde{}{%
\begin{itemize}\tightlist
\item[\dschemical] \sout{BY301 가. 마이코플라즈마 Mycoplasma} D5821-8 특수배양(배양 및 동정) 
\item[\dschemical] \sout{BY302 나. 클라미디아 Chlamydia} D5821-1 특수배양(배양 및 동정) 
\item[\dschemical] \sout{BY303 다. 유레아플라즈마 Ureaplasma} D5821-5	특수배양(배양 및 동정)
\item[\dschemical] \sout{BY309 라. 기타 Others} D5821-6	특수배양(배양 및 동정) 캄필로박터(Campylobactor) 
%\item[\dsmedical] 
%\item[\dsmedical] 
%\item[\dsmedical] 
%\item[\dsmedical] 
\end{itemize}
}{
\begin{enumerate}[1.]\tightlist
\item 산부인과 영역에서 시행하는 너 301 기타미생물 배양검사는 다음과 같은 경우에 요양급여를 인정함.\par
- 다   음 - 
	\begin{enumerate}[가.]\tightlist
	\item 골반염의 제 증상(CRP상승, WBC 상승, 복통, 발열 등)이 있는 경우
	\item 임신 제 2분기 이상에서 조산의 위험 증상 (조기 양막파수, 조기 진통 등)이 있는 경우
	\item 질 분비물이 현저히 증가하거나 악취가 나는 등 부인과적 감염이 의심되는 경우	
	\end{enumerate}
\item 상기 1항의 급여대상 이외 산부인과 영역에서 시행하는 경우에는 「요양급여비용의 100분의 100 미만의 범위에서 본인부담률을 달리 적용하는 항목 및 부담률의 결정 등에 관한 기준」에 따라 본인부담률을 80\%로 적용함.(고시 제2016-147호, '16.9.1. 시행)
\end{enumerate} }

\sout{기존에는 culture(B4051) 내면 ( 나 -405 미생물배양 및 동정검사 Microrganism Culture and Identification
B4051 가 . 미생물배양 및 동정검사 Microrganism Culture and Identification 167.87)
균 안 자라면 추가 검사 불가 균 자라면 , 약제감수성검사 (B4061) 추가 가능 ( 나 -406 미생물 약제 감수성 검사 
Microrganism Antibiotics Sensitivity Test B4061 가 . 디스크 확산법 Disk Difusion 112.94) } 
\par
\medskip
2017 년 9 월부터는 \emph{균자라든 안자라든} B4133 \par
2018년 1월 부터 코드변경으로 \sout{B4133} D5851 일반배양-배양,동정및약제감수성(디스크확산법) 으로  
