\section{산부인과 의사가 알아야할 예방접종}
\begin{itemize}\tightlist
\item 기본접종
	\begin{enumerate}\tightlist
	\item 인플루엔자
	\item 파상풍/디프테리아
	\end{enumerate}
\item 따라잡기 접종
	\begin{enumerate}\tightlist
	\item A형 간염
	\item B형 간염
	\item 자궁경부암바이러스
	\item 백일해
	\item 수두
	\item 홍역/볼거리/풍진
	\end{enumerate}
\end{itemize}

\subsection{인플루엔자}
\begin{commentbox}{접종시기 및 방법}
\begin{itemize}\tightlist
\item 접종시기
	\begin{itemize}\tightlist
	\item 매년 10-12월
	\item 유행 이전에 맞도록 하며 가급적 2주 전까지(고위험군은 유행시기에도 접종 권장)
	\end{itemize}
\item 접종용량
	\begin{itemize}\tightlist
	\item 만 3세 이상 : 0.5ml
	\item 생후 6개월 이상-3세미만 : 0.25ml
	\item 6개월에서 9세 미만은 처음 접종 시에는 1개월 간격으로 2회 접종
	\end{itemize}
\item 임산부, 유행시기에 14주 이상은 접종을 권장,임신 초기(3개월)는 접종을 피한다. 단, 고위험군 임신부는 유행시기 이전에 개월 수에 관계없이 접종을 권장
\item 고위험군 수유부도 접종 권장
\end{itemize}
\end{commentbox}
\paragraph{권장대상}
\begin{enumerate}\tightlist
\item 전파위험자
	\begin{itemize}\tightlist
	\item 의료인
	\item 모유 수유 중인 산모
	\item 고위험군 대상자를 간병하거나 함께 거주하는 가족
	\item 0-59개월 유아와 함께 거주하거나 돌보는 가사 도우미
	\item 기타: 위험지역 여행자, 예방접종을 원하는 일반인
	\end{itemize}	
\item 합병증 발생 고위험군
	\begin{itemize}\tightlist
	\item 50세 이상 성인, 생후 6-59개월 소아, 임산부
	\item 만성 폐질환(만성 기관지염, 폐기종, 기관지 확장증 등, 천식)
	\item 만성 심장 질환(심부전, 허혈성 심질환등, 단순 고혈압 제외)
	\item 만성 대사성 질환(당뇨병등), 만성 간질환, 만성 신장질환, 면역 저하자(항암제 투여, HIV 감염등), 혈색소병증, 장기간 아스피린을 복용하는 6개월-18세 소아
	\item 만성 질환으로 집단 시설에 치료/요양 중인 사람.
	\end{itemize}
\end{enumerate}		

\subsection{디프테리아/파상풍/백일해}
\emph{백신 종류별 톡소이드 함유량}\par
\medskip
\tabulinesep =_2mm^2mm
\begin {tabu} to\linewidth {|X[5,l]|X[2,l]|X[2,l]|X[2,l]|} \tabucline[.5pt]{-}
\rowcolor{ForestGreen!40} & \centering DTap & \centering Td & \centering Tdap \\ \tabucline[.5pt]{-}
\rowcolor{Yellow!40}  Diphtheria toxoid(Lf) & 10-25 & 2 & 2 \\ \tabucline[.5pt]{-}
\rowcolor{Yellow!40}  Tetanus toxoid(Lf) & 5-12.5 & 5-8 & 5 \\ \tabucline[.5pt]{-}
\rowcolor{Yellow!40}  Pertussis toxin(μg) & 25 &  & 2.5 \\ \tabucline[.5pt]{-}
\end{tabu}
%\medskip
%\tabulinesep =_2mm^2mm
%\begin {tabu} to\linewidth {|X[5,l]|X[2,l]|X[2,l]|X[2,l]|} \tabucline[.5pt]{-}
%\rowcolor{ForestGreen!20} & \centering DTap & \centering Td & \centering Tdap \\ \tabucline[.5pt]{-}
%\rowcolor{Yellow!5}  Diphtheria toxoid(Lf) & 10-25 & 2 & 2 \\ \tabucline[.5pt]{-}
%\rowcolor{Yellow!5}  Tetanus toxoid(Lf) & 5-12.5 & 5-8 & 5 \\ \tabucline[.5pt]{-}
%\rowcolor{Yellow!5}  Pertussis toxin(μg) & 25 &  & 2.5 \\ \tabucline[.5pt]{-}
%\end{tabu}

\subsection{여성에서 파상풍 예방의 필요성}
\begin{itemize}\tightlist
\item 임신부의 경우 모체의 항체가 태아에게 전달되어 신생아파상풍을 예방할 수 있음. WHO에서는 Maternal and neonatal tetanus를 예방하기 위해 임산부에게 Tetanus toxoid를 포함하는 백신 (T or Td) 2회 접종을 권장 \par
 최소한 4주간격, 출산 2주전에 완료.
\item 영유아기에 DTap백신을 접종받은 경우에도 성인이 되면 파상풍 항체가가 떨어지는데 남성은 군대에서 파상풍톡소이드를 접종받지만 여성은 파상풍백신의 접종기회가 없어 병원에서 백신접종을 권장하는 하는것이 더욱 필요.
\end{itemize}

\subsection{파상풍/디프테리아 백신(Td) 이상 반응과 주의/금기사항}
\begin{itemize}\tightlist
\item 접종 부위 발열, 동통, 홍반, 경화 등 국소 반응
\item 간혹 \emph{Arthus-like reaction : 10년 이내 접종 금기!} : 일반적으로 접종 후 2-8시간 후에 나타나며 재접종이 정해진 시기 보다 앞당겨 투여된 성인에서 흔하다.
\item 심한 전신 반응 : 두드러기, 아나필락시스, 신경계 합병증
\item 매우 드물게 갈랑-바레 증후군이나 말초신경병증 보고
\item 중등도 이상 급성 질환자 : 접종 시기 연기
\item 금기 : 백신 구성 성분에 중증 알러지 있는 사람
\item 제조사에서 임신부를 금기로 명기하고 있으나, 근거 없음.
\end{itemize}

\subsection{Td/Tdap 접종용량 및 방법}
\begin{itemize}\tightlist
\item Tdap으로 1회 접종(64세 이하), 매 10년마다 Td 추가 접종
	\begin{itemize}\tightlist
	\item 가임기 여성의 경우 Tdap은 임신 전에 접종, 임신 중이라면 출산 직후 접종
	\item 2009년 백일해 항원이 추가된 Tdap백신이 만11-64세 연령에서 사용이 가능하도록 허가되었으나, 필요한 경우(백일해 유행 등) 65세 이상의 연령에서도 접종 가능함.
	\end{itemize}
\item 0.5ml 1회 삼각근 부위에 근육주사 또는 상완외측면에 피하주사
	\begin{itemize}\tightlist
	\item 40세 이상 성인중 DTap 접종력이 없는 경우 0,1,6개월 간격으로 Td 3회 접종. 이중 1회는 Tdap으로 접종
	\item DTap기초 접종력이 확인된 성인의 경우 마지막 접종일로 부터 10년 이상 경과하였으면 Tdap 혹은 Td 1회 접종
	\end{itemize}
\end{itemize}
\tabulinesep =_2mm^2mm
\begin {tabu} to\linewidth {|X[1,l]|X[2,l]|X[2,l]|X[2,l]|} \tabucline[.5pt]{-}
\rowcolor{ForestGreen!40} & \centering Td-pur(Td) & \centering Boostrix(Tdap) & \centering Adacel(Tdap) \\ \tabucline[.5pt]{-}
\rowcolor{Yellow!40}  판매원 & SK & GSK & Sanofi-pasteur \\ \tabucline[.5pt]{-}
\rowcolor{Yellow!40}  Type & Pre-filled syringe & Pre-filled syringe & vial \\ \tabucline[.5pt]{-}
\rowcolor{Yellow!40}  예방질환 & 파상풍, 디프테리아 & 파상풍, 디프테리아, 백일해 &  좌 동 \\ \tabucline[.5pt]{-}
\rowcolor{Yellow!40}  접종연령 & 7세 이상 & 10세 이상\newline *65세 이상도 접종 가능 & 11-64세 \\ \tabucline[.5pt]{-}
\end{tabu}
\par
\medskip
매 10년마다 1번씩 접종, 1회에 한해서 Td백신을 대신하여 Tdap백신으로 접종/IM
\par
\medskip
\highlight{2013년 2월 새로운 권고안은 이전의 백신 접종 여부와 상관없이 매 임신마다 Tdap을 접종할 것을 권장하였다.}
 또한 모체의 항체 형성반응과 신생아로 수동 항체 전달을 최대화 하기 위해 접종시기는 27주에서 36주 사이로 권장하였다.
 임신중 백신을 접종하지 못했다면 출산 직후 Tdap을 반드시 접종하며 다른 가족구성원 또한 신생아 접촉 2주 전에 접종을 완료하도록 권장하였다.

\subsection{성인에서 백일해 Booster백신 접종의 필요성}
\begin{itemize}\tightlist
\item 나이가 들면서 백신에(DTaP)에 의한 획득면역 약화
	\begin{itemize}\tightlist
	\item vaccine acquired immunity : 4-12 yrs
	\item natural infection acquired immunity : 4-20 yrs
	\end{itemize}
\item 군집면역 (herd immunity)획득 실패
	\begin{itemize}\tightlist
	\item 청소년 및 성인에서의 발생이 증가(reservoir) → 영유아 백일해 발생의 주요 원인
	\end{itemize}
\item Vaccine-resistant strain출현
\item 질병 인지도 증가 및 진단법 보급
\end{itemize}

\subsection{A형 간염 백신}
\begin{itemize}\tightlist
\item 1-2세 이상 소아
\item 20대 : 항체 검사 없이 접종
\item 30대 : 항체 음성 확인 후 접종
\item 2회 접종 : 첫 접종후 6-12개월 사이 2차 접종
\item A/B형 간염 혼합백신 : 3회 접종(0,1,6개월)
\item 18세 이하 0.5ml, 19세 이상 1.0ml을 삼각근에 근주
\end{itemize}

\subsection{자궁경부암 백신}
\emph{서바릭스 2D허가 현황}\par
9-14세에 1차 접종을 할 경우 2회 일정은 6개월으로 접종할수 있으며, 일정에 유동성이 필요하다면 2차 접종을 1차 접종을 1차 접종후 5-7개월 사이에 투여할 수 있다.
