\section{환자분류체계}
\subsection{환자분류체계(Patient Classification System,PCS)의 정의}
\begin{itemize}\tightlist
\item 상병, 시술, 기능상태 등을 이용해서 외래나 입원 환자를 임상적 의미와 의료자원 소모 측면에서 유사한 그룹으로 분류하는 체계
\item 대표적인 환자분류체계 : DRG(Diagnosis Related Groups, 진단명 기준 환자군)
\item 주로 지불 도구, 환자구성(Case-mix) 보정 도구 등으로 활용
\end{itemize}

\leftrod{Case-Mix (mix of cases)}
\par
\medskip
\begin{itemize}\tightlist
\item 과학적 방법으로 patient care episodes를 분류하는 information tool
\item 세계적으로 Clinical management \& Funding 에 광범위하게 활용
\item 병원간 의미 있는 activity 비교에 활용
\item 미국 Fetter and Thompson 개발 (60년대 후반 - 70년대)
	\begin{itemize}\tightlist
	\item 당초 병원 output 기술 및 의료의 질과 자원소모를 모니터링 하기 위해 개발
	\item 급성기 입원 영역에 초점을 둠
	\end{itemize}
\end{itemize}
\prezi{\clearpage}
\begin{center}
\includegraphics[width=.9\textwidth]{PCSrequest}
\end{center}
\prezi{\clearpage}
%\subsection{환자분류체계 사용 정보}
\begin{tcolorbox}[frogbox,title=환자분류체계 사용 정보]
\begin{itemize}\tightlist
\item 행위 분류
	\begin{itemize}\tightlist
	\item 환자에게 행해진 시술을 표준화된 코드로 기록한 ‘건강보험요양급여 행위 목록’ 활용
	\item 현행 : 건강보험요양급여 행위 급여 목록표( 2016.2월판)
	\end{itemize}
\item 질병(진단, 상병) 분류
	\begin{itemize}\tightlist
	\item 환자의 입ㆍ내원 이유와 동반상병, 합병증을 표준화된 진단(상병)코드로 기록한 ‘한국표준질병ㆍ사인분류(KCD)’ 활용
	\item 현행 : KCD-7차(‘16.1.1 시행, 통계청 고시)
	\item 주진단(principal diagnosis)
		\begin{itemize}\tightlist
		\item 검사 후 밝혀진 \textcolor{red}{최종 진단}, 병원치료를 필요로 하게 만든 가장 중요 병태
		\item 다만, 진료개시 후 입원 시 병태와는 관련 없는 새로운 병태가 발견되고 이로 인한 자원소모가 더 클 경우 이를 주진단으로 선정
		\end{itemize}
	\item 기타진단(secondary diagnosis)
		\begin{itemize}\tightlist
		\item 입원 당시부터 주진단과 함께 있었거나 발생된 병태로,
		\item 치료나 입원기간에 영향을 준 모든 진단 (동반상병 및 합병증 등)
		\item 과거의 입원과는 관련 있지만 현재 입원과는 관련 없는 병태는 제외
		\end{itemize}
	\end{itemize}
\end{itemize}
\end{tcolorbox}
\prezi{\clearpage}
\subsection{환자분류체계 구성요소}
\begin{center}
\includegraphics[width=.9\textwidth]{PCSYe}
\end{center}
\prezi{\clearpage}
\begin{center}
\includegraphics[width=.9\textwidth]{PCSYe2}
\end{center}
\prezi{\clearpage}
\subsection{우리나라 환자분류체계 종류}
\begin{itemize}\tightlist
\item 의과
	\begin{itemize}\tightlist
	\item KDRG (입원환자분류체계) : Korean Diagnosis Related Group
	\item KOPG (외래환자분류체계) : Korean Outpatient Group
	\item KRPG (재활환자분류체계) : Korean Rehabilitation Patient Group
	\end{itemize}
\item 한의	
	\begin{itemize}\tightlist
	\item KDRG-KM (한의 입원환자분류체계) : Korean Diagnosis Related Group-Korean Medicine
	\item KOPG-KM (한의 외래환자분류체계) : Korean Outpatient Group-Korean Medicine	
	\end{itemize}
\end{itemize}
\prezi{\clearpage}
\subsection{환자분류체계의 활용}
\begin{enumerate}\tightlist
\item 진료비 지불
	\begin{itemize}\tightlist
	\item 포괄수가 지불단위(7개 질병군, 신포괄)
	\item 맹장염으로 충수절제술을 받은 입원 환자 → 맹장염(질병분류)과 충수절제술(행위분류) 이용하여 분류 → 충수절제술 환자분류(12개)를 토대로 포괄수가 산출(요양기관별, 일자별)
	\item 포괄수가제도: 입원기간 동안 제공된 진료량과 관계없이 어떤 질병의 진료를 위해 입원했는지에 따라 미리 정해진 일정액을 지불하는 제도
	\end{itemize}
\item 심사ㆍ현지조사
	\begin{itemize}\tightlist
	\item 심사대상 선정
	\item 종합정보서비스
	\item 지표연동자율 개선제
	\end{itemize}
\item 평가
	\begin{itemize}\tightlist
	\item 평가지표 산출
	\end{itemize}
\item 급여관리
	\begin{itemize}\tightlist
	\item 처방ㆍ조제 약품비 절감
	\item 장려금 지표
	\end{itemize}
\item 의료기관 기능평가
	\begin{itemize}\tightlist
	\item 상급종합병원 지정ㆍ평가 : 현행 지정 기준(복지부 고시, ‘15.1시행)은 KDRG V3.5의 ADRG 질병군 분류를 활용
		\begin{description}\tightlist
		\item[전문진료질병군] >\par
			\begin{description}\tightlist
			\item[분류기준] 희귀성, 합병증↑, 치사율↑,진단난이도↑등
			\item[KDRG수] (ADRG기준) 245개
			\item[비고] 환자구성비율 17\% 이상
			\end{description}
		\item[일반 진료 질병군] >\par
			\begin{description}\tightlist
			\item[분류기준] 모든 의료기관에서 진료 가능하거나 진료를 하여도 되는 질병군
			\item[KDRG수] (ADRG기준) 362개
			\item[비고] 환자구성비율 17\% 이상
			\end{description}	
		\item[단순 진료 질병군] >\par
			\begin{description}\tightlist
			\item[분류기준] 진료가 간단한 질병, 그밖에 상급종합병원에서 진료를 받지 않아도 되는 질병군
			\item[KDRG수] (ADRG기준) 94개
			\item[비고] 환자구성비율 16\% 이하
			\end{description}				
		\end{description}
	\item 전문병원 지정ㆍ평가
	\end{itemize}
\end{enumerate}
\prezi{\clearpage}
\clearpage
\section{KDRG (입원환자분류체계) : Korean Diagnosis Related Group}
\leftrod{질병군 분류란 ?}
\par
\medskip   
질병군 분류(진단명 기준 환자군, Diagnosis Related Group(DRG))는 입원 환자를 자원소모 유사성과 임상적 유사성에 기초하여 분류하는 입원환자 분류체계이다.
\prezi{\clearpage}
\leftrod{질병군 분류번호는 ?}
\par
\medskip
질병군 분류번호는 총 6자리이며, 첫 4자리는 질병군범주, 5번째 자리는 연령구분, 6번째 자리는 합병증 및 동반상병(기타진단)에 의한 분류 (중증도)이다. \par
\begin{center}
\includegraphics[width=.95\textwidth]{DRGprocess}
\end{center}
\par
\medskip
\prezi{\clearpage}
\subsection{주진단 범주(MDC) 분류}
\begin{description}\tightlist
\item[\cntrdots{}] A - M
\item[13 N] 여성 생식기계의 질환및 장애\index{severity}
\item[14 O] 임신,출산, 산욕
\end{description}
\prezi{\clearpage}
\subsection{질병군 분류번호 결정의 이해}
\begin{itemize}\tightlist
\item 환자가 수술을 받았는지 여부에 따라 외과계와 내과계 질병군으로 구분되며, 외과계 질병군은 환자가 받은 수술에 따라 질병군이 결정되고 내과계 질병군은 주진단명에 의해 결정된다.
\item 외과계 그룹은 시술명에 따라 세분화되며, 한 환자가 동일 입원기간 내에 여러 시술을 받은 경우 ‘외과적 우선순위(별표7 질병군범주 우선 순위)’에 따라 우선순위가 가장 높은 외과 질병군으로 배정된다.
\item 개복이나 내시경수술(복강경이나 흉강경)의 구분 및 단측과 양측 등 질병군 분류의 구분이 필요한 경우에는 부가코드(ADC)를 이용하여 질병군을 결정한다.
\item 주진단과 수술에 따라 ADRG(질병군 분류번호 4째자리)까지 분류한 다음, 필요시 연령에 따라서 ADRG를 추가로 세분화한다.
\item 기타진단을 이용한 중증도 분류 과정은 3가지로 구분된다. 
	\begin{itemize}\tightlist
	\item 첫번째 단계는 기타진단의 중증도 점수를 결정하는 것이고 외과환자의 경우 0점 에서 4범까지 중증도 점수를 부여하고 있다.(참고 별표4 기타진단의 중증도 점수)
	\item 두번째 단계는 한 환자가 2개 이상의 기타진단을 가질 경우 환자 단위의 중증도 점수를 결정하는 것이다. 개별 기타진단의 중증도 점수를 통합하는 공식이 있어서 이 공식을 이용해서 환자단위 중증도 점수 (PCCL, Patient Clinical Complexity Level)를 결정하게 된다.
	\item 환자단위 중증도 점수를 이용하여 ADRG별로 중증도 분류단계(최종 질병군 분류번호)를 결정하게 된다. 
	\end{itemize}
\item 이때 결정된 중증도 분류는 ADRG 별로 중증도 분류의 단계를 달리하기 때문에 환자단위 중증도 점수가 ADRG의 중증도 분류와 일치하지는 않는다.
\end{itemize}
\prezi{\clearpage}
\leftrod{질병군 분류번호 결정 요령II}
\par
\medskip
\begin{enumerate}[가.]\tightlist
\item 질병군 분류번호는 주진단, 외과계 시술, 연령 및 기타진단 등에 의하여 6자리로 구성하며, 앞의 4자리는 “질병군범주”를, 5번째 자리는 “연령구분”을, 6번째 자리는 “합병증 및 동반상병 분류”를 나타낸다.
	\begin{itemize}\tightlist
	\item 질병군범주는 ‘주진단’과 ‘외과계 시술’ 등에 의하여 결정되며, 질병군 범주의 결정 및 그 분류번호는 별표3과 같다. 단, 주진단과 첫 번째 기타진단이 「한국표준질병 사인분류」의 다중코딩 지침에 따라 ‘검표(†)와 별표(*) 체계’ 에 해당할 경우 첫 번째 기타진단에 의하여 질병군 범주가 결정된다.
	\item 연령구분은 ‘연령’에 따라 다음 질병군 범주에 한하여 아래와 같이 결정되며, (가)~(다) 이외의 질병군 범주는 연령에 관계없이 분류번호 “0”으로 결정된다.
	\item 합병증 및 동반상병 분류(이하 “합병증분류”라 한다)는 기타진단에 의하여 다음과 같이 결정된다.
		\begin{itemize}\tightlist
		\item 합병증분류에 이용되는 기타진단은 각각의 중증도 점수(별표4 참조)를 가지고 있으나, 주진단 및 기타진단 상호간에 관련성이 높은 경우에는 중증도 점수가 1점 이상이더라도 0점으로 결정된다. (별표5 참조)
		\item 위(가)에 의한 기타진단별 중증도 점수를 반영하여 환자단위 중증도 점수를 결정하며, 동 점수를 이용하여 질병군별로 합병증 분류를 0, 1, 2, 3으로 결정한다. (별표6 참조)
		\end{itemize}
	\end{itemize}	
\item 위 가-⑴ 중 별표3의 각 주진단범주(안과계, 이비인후과계, 소화기계, 여성생식기계, 임신․분만․산욕)에 명시된 질병군범주에 해당되는 경우 로서 질병군범주 우선순위(별표7 참조)에서 당해 질병군범주 보다 높은 범주에 분류된 시술을 함께 행한 경우는 질병군적용에서 제외한다
\end{enumerate}
\prezi{\clearpage}
\begin{center}
\includegraphics[width=.95\textwidth]{DRGcode}
\end{center}
\prezi{\clearpage}
\leftrod{별표 7 임신분만산욕기계 질병군}
\par
\medskip
\tabulinesep =_2mm^2mm
\begin {longtabu} to\linewidth {|X[1,l]|X[3,l]|X[7,l]|} \tabucline[.5pt]{-}
\rowcolor{ForestGreen!40}  순위 & 질병군범주 &	해당 주진단 시술코드 및 부가코드 \centering 총진료비 \\ \tabucline[.5pt]{-}
\rowcolor{Yellow!40} 1 &	자궁적출술을 동반한 \newline 제왕절개분만 &	(R4507, R4508, R4509, R4510, R5001, R5002) \newline
or \newline
(R4517, R4518, R4514, R4519, R4520, R4516 and R4143, R4144,R4145, R4146)  \\ \tabucline[.5pt]{-}
\rowcolor{Yellow!40} 2 & 제왕절개분만(다태아) &	R4519, R4520, R4516 \\ \tabucline[.5pt]{-}
\rowcolor{Yellow!40} 3 & 제왕절개분만(단태아) &	R4517, R4518, R4514  \\ \tabucline[.5pt]{-}
\rowcolor{Yellow!40} 4 &	질식분만(기타 복잡 수술 시행) &	R4351, R4353, R4356, R4358, R3131, R3133, R3136, R3138, R3141,R3143, R3146, R3148, RA431, RA432, RA433, RA434, RA311,RA312, RA313, RA314, RA315, RA316, RA317, RA318, R4361,R4362, RA361, RA362, R4380, RA380 \newline
and \newline
E7691, E7690, O2045, P2141, Q2440, Q2450, Q3012, Q3013, Q3014,Q3017, R4130, R4143, R4144, R4145, R4146, R4154, R4155, R4157,R4170, R4181, R4183, R4202, R4203, R4221, R4223, R4224, R4250,R4261, R4262, R4295, R4331, R4332, R4390, R4400, R4405, R4411,R4412, R4413, R4421, R4423, R4424, R4427, R4428, R4425, R4426,M6650
\\ \tabucline[.5pt]{-}
\end{longtabu}
\par
\medskip
\clearpage
\section{주진단명 부여방법}
\leftrod{주진단의 정의}
\par
\medskip
주진단은 검사 후 밝혀진 최종 진단으로 병원 치료(또는 의료시설 방문)를 필요로 하게 만든 가장 중요한 병태이다. 단, 진료 개시 후 의료시설을 방문하게 만든 병태와는 관련이 없는 새로운 병태가 발견되고, 이로 인한 자원 소모가 더 클 때에는 새로운 병태를 주진단으로  선정한다. 진료 후 밝혀진 진단은 입원 시 진단과 일치할 수도 있고 일치하지 않을 수도 있다. \par
진단이 내려지지 않은 경우에는 주증상이나 검사의 이상소견 또는 문제점을 주진단으로 선정한다. 진료기간 동안 검사나 치료를 받은 병태 중 ‘주진단’은 단일병태 질병이환 분석시 사용된다.
\prezi{\clearpage}
\leftrod{주진단 선정원칙}
\par
\medskip
\textcolor{red}{검사 후 밝혀진 최종 진단으로 병원 치료 또는 의료기관 방문을 필요로 하게 만든 가장 중요한 병태}를 주진단으로 선정한다.
환자가 여러 질환을 동시에 가지고 내원한 경우에는 진단이나 치료에 대한 환자의 요구가 가장 컸던 질환, 즉 의료자원을 가장 많이 사용하게 했던 질환을 주진단으로 선정한다.\par
여기서 의료자원이란 단순히 해당 질병과 관련된 진료비의 크기만을 의미하는 것은 아니며, 해당 질병으로 인해 유발된 재원일수, 시술비용, 약품 및 치료 재료비 등을 종합적으로 고려하여 판단한다. 즉 진료비가 높다고 하더라도 그것이 전체 서비스 제공량과는 무관하게 몇개의 고가 약이나 치료재료 때문에 발생한 것이라면 해당 질병의 자원 소모량이 반드시 높다고 말할 수 없다. 약이나 치료재료의 가격은 나라마다 상이하기 때문에 이로 인한 비용만을 기준으로 자원소모량을 판단하는 것은 국제비교 측면에서 타당하지 않기 때문이다. 자원소모량의 크기에 대해서 논란이 있는 경우에는 자원소모량에 대한 진료의사의 판단에 따른다\par
진료 개시 후 \textcolor{red}{주진단과 관련된 질환이나 합병증이 발생하였을 경우에는 이로 인한 자원소모가 많다고 할지라도 기존 주진단을 유지}한다.\par
단 진료 개시 후 \textcolor{blue}{의료시설을 방문하게 만든 병태와는 관련이 없는 새로운 병태가 발견되고, 이로 인한 자원 소모가 더 클 때에는 새로운 병태를 주진단}으로 선정한다.\par
\textcolor{red}{진단이 내려지지 않은 경우에는 주증상이나 검사의 이상소견 또는 문제점을 주진단}으로 선정한다.\par
\prezi{\clearpage}
\subsection{DRG에서 주진단에 대한 논란}
\begin{hemphsentense}{산부인과 ``DRG 손실 커" 복지부 ``새 패러다임 협조"}
\href{https://dailymedi.com/news/view.html?section=1&category=5&no=771918}{주진단에 대한 복지부의 입장}
산부인과가 포괄수가제 병원급 확대 적용 이후 제왕절개술 주진단명 코딩 방법 변경에 따라 막대한 손실을 입고 있다는 우려가 제기됐다.
대한산부인과학회는 27일 제99차 학술대회를 개최하고 포괄수가제 등 ‘산부인과 건강보험의 과제’에 대해 논의하는 시간을 가졌다. 관동의대 산부인과 민응기 교수는 “우여곡절을 겪으면서 지난 7월부터 모든 의료기관을 대상으로 7개 질병군에 대한 포괄수가제 강제적용이 시작됐다”며 “건강보험심사평가원에 제왕절개술의 보험급여를 청구하면서 심각한 문제가 발생했다”고 전했다.\\
\textcolor{blue}{1997년 2월 포괄수가제 시범사업을 시작하면서부터 올 6월까지 제왕절개의 ‘주진단’명은 O820-O829로 코딩을 하고 제왕절개술을 한 주 사유를 `기타진단’으로 코딩해 중증도 보정을 받아왔다.} 또한 2007년도 포괄수가제 실무지침서에서도 똑같이 주진단을 O820-O829로 코딩하는 청구방법을 공지했고 그렇게 시행해왔다.\\
포괄수가제를 전면 시행하기로 한 지난 7월 1일 직전인 6월 15일 심평원 교육자료에서도 마찬가지였다.민 교수는 “그러나 7월 포괄수가제 전면실시 후 심평원에서는 \textcolor{red}{갑자기 한국표준질병사인분류 질병코딩지침서에 의거해 O820-O829를 주진단이 아닌 ‘기타진단’으로 코딩해 제왕절개술을 시행한 주 사유를 ‘주진단’명으로 코딩해야 한다고 불과 보름 만에 말을 바꿨다”고 지적했다.} 이로 인해 많은 환자에서 중증도가 반영되지 않아 제왕절개술을 시행한 모든 의료기관은 큰 손실을 보게 됐다는 것이다. 학회에 따르면 실제 의료기관이 똑같은 진단명으로 제왕절개술을 하더라도 평균 입원일수를 7일로 산정할 때 단태아는 10만1810원-26만1350원, 다태아의 경우 25만7170원-41만3510원의 금액을 손해 볼 수밖에 없다는 분석이다. 그는 “분만 전문 병의원의 경우 어림잡아 연간 수천만원에서 억대에 이르는 순익 손실을 보게 되는 금액”이라면서 “심평원은 갑작스런 코딩 방법 변경에 대한 납득할만한 해명을 내놓지 못하고 있다”고 강조했다.\\
민 교수는 이어 \uline{“오랜 기간 동안 시행해 온 대로 기존 코딩방법을 유지해야 한다”면서도 “기존 방법에 문제가 발견됐다면 그 틀을 벗어나지 않는 범위에서 문제점 해결 방안을 찾거나 중증도 반영이 달라지지 않도록 서둘러서 보완을 해야 할 것”이라고 피력했다.}\\
복지부 ``과거 방식 고치는데 있어 나타난 전환기적 불편함" 양해 구해\\
보건복지부는 이에 대해 과거 문제가 있었던 부분을 고치는 과정에서 발생한 문제점이라면서 바람직한 정착에 노력하겠다는 입장이다. 보건복지부 보험급여과 배경택 과장은 “과거 방식에 문제가 있어 고치는 과정에서 나타난 전환기적 불편함이라 생각한다. 심평원 업무처리가 더디게 느껴질 수 있으나 이는 복지부에서 의사결정 하는데 시간이 걸렸기 때문”이라고 양해를 구했다. 배 과장은 이어 “바람직하게 개선하는데 심평원, 복지부 모두가 노력할 것”이라고 말했다. 이와 함께 산부인과가 포괄수가제라는 새로운 패러다임을 구축하는데 동행자가 돼 줄 것을 당부했다.\\
그는 “포괄수가제는 다른 패러다임”이라며 “산부인과가 기존 행위별수가제에 안주할 수 있을까에 대해 고민을 해봐야 한다. 새로운 패러다임을 만드는데 적극 참여해 긍정적으로 구축할 수 있도록 노력할 필요가 있다”고 덧붙였다.
\end{hemphsentense}
예를 들면 한국표준질병사인분류 질병코딩지침서에 의거해 제왕절개를 하게된 주된 사유인 Placenta previa를 주진단으로 하게되면, 중등도가 반영되지 않아서 손해를 보게된다고 하는데 이게 어떤 의미일까? \\ DRG에서 마지막 단계로 합병증 및 동반상병 분류로 각 질병군 범주의 특성에 따라 구분된 환자단위 중증도 점수별로 \uline{최종 질병군 분류번호를 결정 하게 된다}. 여기에 쓰이는 것이 기타진단(부진단)입니다. 다시말하면, \textcolor{red}{기타진단의 중요도는 기타진단(부진단)에 의해서 중등도가 결정되게 됩니다.}\\
이와 같은 이류로 \textcolor{red}{중등도(=부진단 NOT주진단)}는 보험급여금에 차이를 주는데요. 여태까지는 o820등의 선택적제왕절개등을 주진단으로 하고 주된이유를 부진단으로 해서 어려운수술에 대한 보전을 받았는데, placenta previa같이 어려운 수술을 해도 현 DRG 주진단 system에서는 보전받을수 없게 되었다는 것입니다. \\
물론 severe PIH로 제왕절개를 한경우에 주진단으로 severe PIH 한가지만 내 놓으면 기본적인 DRG금액만 받을수 있지만, CPD를 주진단으로 하고 severe PIH, hemorrhage등등의 기타진단을 기입하게 되면 severity등급에 따라서 최소 9만원이상의 금액을 더 받을수 있습니다. 실제로 따져봐도 severe PIH는 직접적으로 제왕절개를 해야할 주된 이유는 아닙니다. CPD나 failure to progression등이지요!!\\
\clearpage

\leftrod{제6차 개정 한국표준질병사인분류 코딩지침서}\par
 다음의 내용 \href{http://kostat.go.kr/kssc/common/CommonAction.do?method=download&attachDir=bm90aWNl&attachName=JUVEJTk1JTlDJUVBJUI1JUFEJUVEJTkxJTlDJUVDJUE0JTgwJUVDJUE3JTg4JUVCJUIzJTkxJUVDJTgyJUFDJUVDJTlEJUI4JUVCJUI2JTg0JUVCJUE1JTk4XyVFQyVBNyU4OCVFQiVCMyU5MSVFQyVCRCU5NCVFQiU5NCVBOSVFQyVBNyU4MCVFQyVCOSVBOCVFQyU4NCU5QyUyODIwMTIuMDMlMjkucGRm}{제6차 개정 한국표준질병사인분류 코딩지침서}에 따릅니다.\par
유용한 KCD code를 찾을수 있는 Link입니다. \url{http://www.kcdcode.co.kr/}
\begin{itemize}[▷]\tightlist
\item 산과진단코드 부여 순서
	\begin{enumerate}\tightlist
	\item \uline{제왕절개나 기구를 사용하여 분만한 경우} \textcolor{red}{중재술을 하게 된 원인이 되는 병태}를 주된 병태로 부여한다.
		\begin{mdframed}[linecolor=blue,middlelinewidth=2]
			\begin{itemize}\tightlist
			\item 임신성당뇨로 유도분만하였으나 7시간 진통후 CPD로 제왕절개하여 건강한 아이 분만한 경우 :  주된병태 : O65.4 (CPD) 기타병태 : O24.4 (GDM), Z370 (Single live birth) 
			\item Full dilatation후 태아의 P position으로 vaccume extraction시도 했으나 태아의 하강이 없어 흡입분만 포기하고 제왕절개분만한 경우 :  주된병태 : O64.0 (Obstructed labour d/t incomplete rotation of fetal head) 기타병태 : O66.5 (상세불명의 집게및 진공흡착기 적용실패), Z370 (Single live birth) .\index{진단코드!POPP}
			
			\end{itemize}
		\end{mdframed}
	\item \uline{기구의 도움을 받지 않고 질식분만을 하였으나 산모가 출산전 산전병태로 입원}하였다면 산전병태를 주된병태로 분류한다. \textcolor{red}{유도분만의 이유가 주 진단이다}
		\begin{mdframed}[linecolor=blue,middlelinewidth=2]
		\begin{itemize}\tightlist
		\item 임신성고혈압으로 induction delivery하여 1st degree laceration있는 경우는  : 주된병태 : O13 (PIH) 기타병태 : O70.0 (1st degree laceration), Z370 (Single live birth) 
		\end{itemize}
		\end{mdframed}
		
	\item \uline{정상 분만진통으로 입원하여 정상질식분만을 한경우}에는 \textcolor{red}{분만을 주진단으로} 선정한다.
	\begin{mdframed}[linecolor=blue,middlelinewidth=2]
	\begin{description}\tightlist
	\item[O800] 자연두정태위분만
	\item[O801] 자연둔부태위분만
	\item[O814] 진공흡착기분만
	\item[O840] 모두질식분만에의한 다태분만
	\item[O842*] 모두제왕절개에의한 다태분만
	\end{description}
	 \end{mdframed}
	 
	\item 즉 \uline{쌍둥이로 제왕절개분만을 한 경우}는 주된병태는 \textcolor{red}{제왕절개를 한 이유}
	\begin{mdframed}[linecolor=blue,middlelinewidth=2]
	즉. 하나 이상의 태아의 태위장애를 동반한 다태임신의 산모관리(O32.5)이고, 기타병태로 O32.1(Bx 산모관리), O30.0(쌍둥이임신), Z37.2(쌍둥이, 둘 다 생존 출생)등이 있게 된다..\index{진단코드!쌍둥이 제왕절개}
	\end{mdframed}
	\item 분만문제가 진통 전에 발견되었는지, 아니면 진통후에 발견되었는지의 여부에 따라서 `O32-O34'또는 `O64-O66'으로 코딩한다.
		\begin{mdframed}[linecolor=blue,middlelinewidth=2]
		\begin{itemize}\tightlist
		\item 둔위로 Elective c-sec를 하게된경우는 ? 주 진단명이 O32.1 (둔부태위의 산모관리)이지, O64.1 (둔부태위로 인한 난산)이 아니다.
		\item 1분간 지속된 견갑난산을 가진 여아를 질식 분만하였다. 주진단명은 \dotemph{O66.0 (어깨난산으로 인한 난산)} 이다. 
		\end{itemize}
		\end{mdframed}
		
	\item 이전 제왕절개에 따른 분만의 경우에 넣는 코드는 다음과 같다.\index{진단코드!선행제왕절개}
		\begin{mdframed}[linecolor=blue,middlelinewidth=2]
		\begin{description}\tightlist
		\item[O75.7] 이전 제왕절개후 질분만
		\item[O66.4] 상세불명의 분만 시도의 실패 : 위의 두 경우는 TOL(Trial of Labor)를 시도하다가 성공하거나 실패한 경우이고 
		\item[O34.20] 이전의 제왕절개로 인한 흉터의 산모관리
		\item[O34.28] 이전의 기타 외과수술로 인한 자궁흉터의 산모관리 : 위의 두 경우는 Elective로 repeate c-sec를 한 경우로 \dotemph{분만문제가 진통전에 발견되었기 때문에 O32-O34를 쓴다는 원칙을 따른것임.}
		\end{description}
		\end{mdframed}
		
	\item O80-O82의 분류 : 이 코드는 기록되어 있는 정보가 분만이거나 분만 방법에 대해서만 국한되어 있을때 제한적으로 주된병태의 코드로 사용할 수 있다.또한 아무런 문제 없이 정상적인 분만을 하였을때 O80으로 할수 있다.
		\begin{mdframed}[linecolor=blue,middlelinewidth=2]
		\begin{itemize}\tightlist
		\item 산모가 만삭 정상분마을 위해 입원하여 산전, 분만중, 산후에 아무런 합병증이 없고, 기구나 기술을 필요로 하지 않고 정상분만을 한 경우 O80코드를 주진단명으로 할 수 있다.
		\item 특별한 합병증없이 다태분만을 한 경우 주된병태는 ``쌍둥이임신(O30.0)"으로 코드를 부여한다. ``O840" 모두 자연적인 다태분만은 분만의 방법을 나타내 주기 위하여 임의적인 추가코도로 부여 할 수 있다.
		\item 제왕절개술을 받은 경우 선택적이던 응급이던 상관없이 제왕절개술을 받은 이유를 주된병태로 우선 부여한다. 하지만 제왕절개술을 받은 이유가 불명확할 경우 ``O82.-제왕절개에 의한 단일 분만" 코드를 주된 병태로 부여 할 수 있다. \emph{현재와 같이 Repeated c-sec의 이유가 되지는 않는다.}
		\end{itemize}
		\end{mdframed}
	\item 만약 유도분만중 수술한 경우에는 수술한 원인이 주된 병태입니다. 그러나 induction failure는 주된 병태가 될수 없고, induction failure가 생긴 이유가 주된병태입니다. (예로 CPD나 first stage prolongation등)
	\item 모든 분만산모의 경우에는 신생아의 상태를 부진단으로 한다.
		\begin{mdframed}[linecolor=blue,middlelinewidth=2]
		\begin{description}\tightlist
		\item[Z370] 단일생산아 (single liveborn)
		\item[Z371] 단일사산아 (Single stillbirth)
		\item[Z372] 쌍둥이, 둘다 생존 출생 (Twins, both liveborn)
		\end{description}
		\end{mdframed}
	\end{enumerate}
%\item 제왕절개는 \dotemph{제왕절개의 적응증}을 주진단으로 한다. 
\item 입원환자 치료 중 기저질환이 밝혀지면 기저질환을 주진단으로 선정한다. (ex : 병명 - O001 난관임신)
\item 기저질환이 입원 시 알려져 있고, 문제에 대해서만 치료가 이루어지면 그 문제를 주진단으로 선정한다. (ex : 병명 - N833 자궁경부무력증)
\item 급만성이 동시에 발생한 경우 \emph{급성질환}을 주진단으로 선정한다. (ex : 병명 – O140 중등도의 전자간(급성), O249 임신성 당뇨(만성))
\end{itemize}
\prezi{\clearpage}
%\subsection{주진단과 기타진단에 대한이해}
\begin{tcolorbox}[frogbox,title=주진단과 기타진단에 대한이해]
\begin{enumerate}[가.]\tightlist
\item 주진단
	\begin{enumerate}[(1)]\tightlist
	\item 한번 입원한 건에 대하여는 하나의 주진단을 부여한다. 둘이상의 병태가 주진단 정의에 똑같이 부합될 때는 둘 중 어느 진단을 선택하여도 무방하나 하나의 진단만을 주진단으로 부여한다.
	\item 비급여대상 질환(「국민건강보험 요양급여의 기준에 관한 규칙」별표2 제6호에 해당하는 질환)이 주진단에 해당될 경우는 기타진단 중 가장 주된 진료를 받은 진단을 주진단으로 선정한다.
	\item 진단이 확립되지 아니한 경우 \textcolor{blue}{의심되는 진단(의증)을 주진단으로 부여할 수 있다.} 입원기간 중 생성된 진단 정보가 없어서 진료 후에도 주진단이 여전히 ‘의심되는’, ‘의문나는’ 등으로 기록되어 있는 경우 의심되는 진단을 확진된 것처럼 부여할 수 있다.
	\end{enumerate}
\item 기타진단
\begin{enumerate}[(1)]\tightlist
\item \textcolor{red}{확립된 진단만 부여하고 의심되는 진단(의증)은 기타진단으로 부여 하지 아니한다.} 기타진단은 확진된 경우만 부여할 수 있으며, 의심되는 진단(의증)은 부여하지 아니한다. 의심되는 진단(의증)의 경우는 그 진단과 관련되는 증상 및 증후〔ⅩⅧ장. 달리 분류되지 않은 증상, 징후와 임상 및 검사의 이상 소견에 해당되는 분류기호로 부여하여야 한다.
\item 비급여 대상 질환은 기타진단으로 부여하지 아니한다.
\item 이번 입원과 관련 없는 이전 병태는 기타진단으로 부여하지 아니한다. 진료기록부의 최종진단명란에 기재되어 있는 진단명은 주진단 이외 에는 일반적으로 모두 기타진단으로 간주할 수 있으나, 그 중 과거의 진료 또는 병력에 해당되는 병태로서 이번 입원과 관련 없는 경우는 기타진단으로 부여하지 아니한다.
\item 전신적인 만성질환은 기타진단으로 부여할 수 있다. 고혈압, 파킨슨병, 당뇨병\footnote{하지만 고혈압,당뇨등도 합병증이 없는 경우에는 기타진단.즉 중등도에 올라가기는 힘들다. severity점수가 없다} 등과 같은 만성질환은 지속적인 임상적 평가, 추가적인 간호 및 관찰이 요구될 수 있으므로 기타진단으로 부여할 수 있다.
\item 질병진행 과정중의 한 부분으로의 병태는 기타진단으로 부여하지 아니한다. 질병의 진행과정에 반드시 수반되는 병태는 기타진단으로 별도 부여하지 아니한다.
\item \uline{비정상적인 검사결과만으로(진료의가 임상적인 의미를 부여하지 않은 경우) 기타진단으로 부여하지 아니한다.}
\end{enumerate}
\end{enumerate}
\end{tcolorbox}
\prezi{\clearpage}

\begin{myshadowbox}
\begin{enumerate}[5.]\tightlist
\item 질병군에 대한 \textcolor{red}{요양급여비용을 산정}할 때에는 제2부(실무안내 제2장) 각 장에 분류된 질병군 점수를 기준으로 별표 1의 질병군별 점수 산정요령에 의하여 산정된 점수 총합에 국민건강보험법 제45조제3항과 영 제21조제1항에 따른 점수 당 단가를 곱하여 10원 미만을 절사한 금액을 요양급여비용 총액으로 산정한다. 이 경우 위 금액 외에 식대를 포함한 별도로 산정하는 비용이 있는 경우에는 각각의 산정방식에 의하여 산정된 금액을 합산한다
\end{enumerate}
\begin{enumerate}[13.]\tightlist
\item  질병군 요양급여를 실시하는 요양기관은 \textcolor{red}{질병군 입원환자의 질병군 분류 번호와 관련한 주진단 및 기타진단, 수술명 등은 진료기록부에 근거하여 정확한 코드를 부여}하여야 하며, \textcolor{red}{진단명이 입원시부터 존재하였는지 여부를 확인할 수 있도록 진료기록부에 기록}하고, 의료의 질 향상을 위한 점검표를 별지 서식에 따라 작성하여야 한다
\end{enumerate}
\end{myshadowbox}
\prezi{\clearpage}
\par
\medskip
\Que{급성출혈 후 빈혈(D62), 분만 후 출혈(O72), 분만 중 출혈(O67)의 진단분류기호 부여기준 중 “분만(수술)전(입원당시)” 의 의미는?}입원하여 분만(수술) 전 시행한 혈액검사 및 통상 외래에서 분만(수술)전 시행한 검사를 의미함
\Ans{입원하여 분만(수술) 전 시행한 혈액검사 및 통상 외래에서 분만(수술)전 시행한 검사를 의미함}
\prezi{\clearpage}
\par
\medskip
\Que{항문수술 후 퇴원한 환자가 수술합병증으로 15일 이내에 재입원하여 재수술을 시행한 경우 질병군 적용 대상여부}
\Ans{수술합병증으로 내원한 경우 「한국표준질병ㆍ사인분류」의 “달리 분류되지 않은 처치의 합병증(T81)”이 주진단으로 항문수술 질병군의 주진단 범주에 속하지 않으므로 DRG 대상이 아님(\textcolor{red}{행위별청구 대상})\par
☞ DRG분류는 주진단, 외과계 시술 등에 의해 결정되며 「질병군 급여ㆍ비급여 목록 및 급여 상대가치점수」 제3부 질병군 분류번호 결정요령의「질병군 범주의 결정 및 그 분류번호(별표3)」참조}
\prezi{\clearpage}
\par
\medskip
\Que{전치태반 또는 전자간증 등이 있는 임신부가 제왕절개술을 시행한 경우 주진단 부여원칙은?}
\Ans{제왕절개술을 받은 경우 선택적이던 응급이던 상관없이 \textcolor{red}{제왕절개술을 받은 이유를 주된 병태로 우선 부여함}. 제왕절개술을 받은 이유가 불명확할 경우 O82 제왕절개에 의한 단일분만 코드를 주된 병태로 부여함 \par
 ☞ 관련근거 : 한국표준질병ㆍ사인분류 질병 코딩지침서 (P.113) O80 ~ O84의 분류}
\prezi{\clearpage}
\par
\medskip
\Que{질병군 대상 수술 후 합병증으로 패혈증 등이 발생한 경우 합병증을 주진단으로 선정할 수 있는지?}
\Ans{진료 개시 후 주된 병태와 관련된 질환이나 합병증이 발생하였을 경우에는 이로 인한 자원소모가 많다고 할지라도 기타진단으로 부여함(기존 주된 병태를 주진단으로 적용) \par
 ☞ 관련근거: 한국표준질병ㆍ사인분류 질병 코딩지침서 (P.2~4) 주된병태 선정원칙}
\prezi{\clearpage}
\par
\medskip
\Que{분만 후 출혈(O72) 상병 부여시 자원소모가 있어야 하나요?}
\Ans{분만 후 출혈 상병을 부여할 수 있는 기준은 분만전(입원당시)에 실시한 혈액검사 결과와 비교하여 Hct가 10\%이상 감소했거나 수혈이 필요하여 수혈을 실시한 경우에 가능합니다}
\prezi{\clearpage}
\par
\medskip
\Que{Hct만 단지 떨어져 있다는 이유로 기타진단이 될수 있는지?}
\Ans{분만전 Hct 37.9\%, 제왕절개분만 후 Hct 34.0\% 로 10\% 이상 Hct 감소한 경우이나 환자의 심신상태 등이 양호하여 특별한 처치ㆍ치료를 필요로 하지 않는 경우 O72 분만 후 출혈을 기타진단으로 코딩함은 오류임(복지부고시 「기타진단부여기준(별표8)」에 맞지 않음)}
\prezi{\clearpage}

\par
\medskip
\Que{질병군 분류번호를 결정하는 주된 수술을 양측으로 실시한 경우 질병군 요양급여비용 산정방법}
\Ans{질병군 분류번호를 결정하는 주된 수술을 양측으로 실시한 경우 ‘수정체수술 질병군’ 과 ‘서혜 및 대퇴부 탈장수술 질병군’은 양측 수술 질병군으로 분류되며, ‘편도 및 아데노이드절제술 질병군’과 ‘자궁부속기수술 질병군’은 편측ㆍ양측 수술에 불문하고 해당 질병군의 소정점수를 적용하여야 함
또한, 질병군 분류번호를 결정하는 주된 수술을 양측으로 실시한 경우 양측 수술 질병군 \textcolor{red}{분류여부에 관계없이 수술료를 추가 산정할 수 없음}}
\prezi{\clearpage}
\par
\medskip
\Que{질병군 분류번호를 결정하는 \textcolor{red}{주된 수술 이외에 대칭기관의 양측수술을 실시한 경우} 수술료 추가산정방법}
\Ans{질병군 분류번호를 결정하는 주된 수술 이외에 대칭기관의 양측수술을 실시한 경우 질병군의 소정점수 이외에 대칭기관의 양측 수술료를 추가 산정함. 다만 양측수술 불문하고 해당 소정점수를 산정토록 한 경우에는 해당하지 않음\par
\begin{description}\tightlist
\item[(예시1)] 편도전적출술(자 -230)과 양측 하비갑개절제술(자 -101)을 동시 시술한 경우 ⇒ 편도 및 아데노이드절제술 질병군 급여상대가치점수와 하비갑개절제술의 수술료 200\% 산정
\item[(예시2)] 충수절제술(자 -286)과 양측 *위축성비염수술(자 -98)을 동시 시술한 경우 ⇒ 충수절제술 질병군 급여상대가치점수와 위축성비염수술의 수술료 100\% 산정
\item[*] 자98 위축성비염수술(양측): 양측을 시술할지라도 소정점수만 
\end{description}
}
\prezi{\clearpage}
\begin{shaded}
기타진단의 중요도는 기타진단에 의해서 중등도가 결정되게 됩니다. 예를 들어보면, severe PIH로 제왕절개를 한경우에 주진단으로 severe PIH 한가지만 내 놓으면 기본적인 DRG금액만 받을수 있지만, CPD를 주진단으로 하고 severe PIH, hemorrhage등등의 기타진단을 기입하게 되면 severity등급에 따라서 최소 9만원이상의 금액을 더 받을수 있습니다. 실제로 따져봐도 severe PIH는 직접적으로 제왕절개를 해야할 주된 이유는 아닙니다. CPD나 failure to progression등이지요!!\\
to be continue--
\end{shaded}
\prezi{\clearpage}
\subsection{고위험 기타진단 산정전/산정후 비교}
-조건 : 8박 9일/6인실/식사제외된 사항입니다.\\
\noindent

\tabulinesep =_2mm^2mm
\begin {tabu} to\linewidth {|X[4,c]|X[3,c]|X[3,c]|X[3,c]|} \tabucline[.5pt]{-}
\rowcolor{ForestGreen!40}  & \centering 본인부담 & \centering 공단부담 & \centering 총진료비 \\ \tabucline[.5pt]{-}
\rowcolor{Yellow!40} 산정전(O01600) & 394,834 & 1,388,528 & 1,783,362  \\ \tabucline[.5pt]{-}
\rowcolor{Yellow!40} 산정후(O01601-3) & 408,219 & 1,465,144 & 1,873,363 \\ \tabucline[.5pt]{-}
\rowcolor{Yellow!40} 차 액 & 13,385 & 76,616 & 90,001  \\ \tabucline[.5pt]{-}
\end{tabu}
\prezi{\clearpage}
\begin{shaded}
분만전 hct 37.9\%, 제왕절개분만 후 Hct 34.0\%로 10\%이상 Hct감소한 경우이나 환자의 심신상태 등이 양호하여 특별한 처치\cntrdot{}치료를 필요로 하지 않는 경우 O72 분만 후 출혈을 기타진단으로 코딩함음 오류임(복지부고시 \snm{기타진단부여기준(별표8)}에 맞지 않음)\\

현재 까지 알아낸 특별한 처치들이란?
Nalador usage, 부르탈 usage, GDM에서 BST check등...
\end{shaded}
\prezi{\clearpage}
\tabulinesep =_2mm^2mm
\begin {longtabu} to\linewidth {|X[1,l]|X[6,l]|X[1,l]|X[1,l]|} \tabucline[.5pt]{-}
\rowcolor{ForestGreen!40}  기호 & \centering 한글명칭 & \centering \% & \centering 기준 \\ \tabucline[.5pt]{-}
\rowcolor{Yellow!40} D62 & 급성출혈후 빈혈 & 29.0 & O  \\ \tabucline[.5pt]{-}
\rowcolor{Yellow!40} Z355 & 고령 초임산부의 관리 & 17.7 & O  \\ \tabucline[.5pt]{-}
\rowcolor{Yellow!40} Z358 & 기타 고위험 임신의 관리 & 16.0 & O  \\ \tabucline[.5pt]{-}
\rowcolor{Yellow!40} O721 & 기타 분만직후 출혈 & 10.4 & O  \\ \tabucline[.5pt]{-}
\rowcolor{Yellow!40} O244 & 임신중 생긴 당뇨병 & 4.5 & X  \\ \tabucline[.5pt]{-}
\rowcolor{Yellow!40} O720 & 제3기 출혈 & 2.9 & O  \\ \tabucline[.5pt]{-}
\rowcolor{Yellow!40} O440 & 출혈이 없다고 명시된 전치태반 & 2.5 & X  \\ \tabucline[.5pt]{-}
\rowcolor{Yellow!40} O249 & 상세불명의 임신중 당뇨병 & 2.2 & X  \\ \tabucline[.5pt]{-}
\rowcolor{Yellow!40} O678 & 기타 분만중 출혈 & 2.1 & O  \\ \tabucline[.5pt]{-}
\rowcolor{Yellow!40} O679 & 상세불명의 분만중 출혈 & 2.0 & O  \\ \tabucline[.5pt]{-}
\rowcolor{Yellow!40} O441 & 출혈을 동반한 전치태반 & 1.9 & X  \\ \tabucline[.5pt]{-}
\rowcolor{Yellow!40} O459 & 상세불명의 태반조기분리 & 1.8 & X  \\ \tabucline[.5pt]{-}
\rowcolor{Yellow!40} O13 & 유의한 단백뇨를 동반하지 않은 임신성[임신-유발성]고혈압 & 1.8 & O  \\ \tabucline[.5pt]{-}
\rowcolor{Yellow!40} D500 & (만성)실혈에 이차성 분만후 철결핍빈혈 & 1.6 & X  \\ \tabucline[.5pt]{-}
\rowcolor{Yellow!40} O722 & 지연성 및 이차성 분만후 출혈 & 1.1 & O  \\ \tabucline[.5pt]{-}
\rowcolor{Yellow!40} O140 & 중증도의 전자간 & 0.7 & O  \\ \tabucline[.5pt]{-}
\rowcolor{Yellow!40} O149 & 상세불명의 전자간 & 0.7 & O  \\ \tabucline[.5pt]{-}
\rowcolor{Yellow!40} T810 & 달리 분류되지 않은 처치에 합병된 출혈 및 혈종 & 0.4 & X  \\ \tabucline[.5pt]{-}
\rowcolor{Yellow!40} O141 & 중증의 전자간 & 0.4 & O  \\ \tabucline[.5pt]{-}
\rowcolor{Yellow!40} K661 & 복강내 출혈 & 0.3 & X  \\ \tabucline[.5pt]{-}
\end{longtabu}
\par
\prezi{\clearpage}
\subsection{합병증 및 동반상병 분류 결정 단계}
\begin{enumerate}[(1)]\tightlist 
\item 기타진단의 중증도 점수
	\begin{itemize}\tightlist
	\item 합병증 분류에 이용되는 기타진단은 진단별로 2∼4까지의 중증도 점수를 갖는다.(「4.기타진단의 중증도 점수」 참조)
	\item 주진단 및 기타진단간 상호 연관성이 높은 기타진단은 중증도 점수가 2점 이상이더라도 0점이 된다.(「5.기타진단의 중증도 점수를 0으로 결정되게 하는 주진단」 참조)
	\end{itemize}
\item 환자단위 중증도 점수
	\begin{itemize}\tightlist
	\item 최종적으로 중증도 점수를 갖는 여러 개의 기타 진단들이 있을 경우 이를 통합하여 환자단위 중증도 점수를 결정하게 된다. 환자단위 중증도 점수는 아래와 같은 공식을 이용해서 계산된다.
	\item 환자단위의 중증도점수는  = 0 if there is no 기타진단, = 4 if x >4 , = x otherwise
	\item 점수의 의미는 0 : no CC effect, 1 : minor CC, 2 : moderate CC, 3 : severe CC, 4 : catastrophic CC 입니다. ※ CC(Complication and Comorbidity) : 합병증 및 동반상병
	\end{itemize}

\item 질병군범주별 합병증 및 동반상병 분류
	\begin{itemize}\tightlist
	\item 합병증 및 동반상병 분류의 마지막 단계로 각 질병군 범주의 특성에
따라 구분된 환자단위 중증도 점수별로 최종 질병군 분류번호를 결정
하게 된다. [표1 참조]
	\end{itemize}
\end{enumerate}
\prezi{\clearpage}
%\begin{figure}
%\centering
\includegraphics{severity2}	
\par
\medskip
%\caption{
다른 부인과나 산과의 질환의 경우는 severity에 따라서 2-3단계만 있는데에 비해서, 단태아제왕절개를 보면 severity에 따라서 4단계나 나누어 지는것을 볼수 있습니다. 중증도점수가 올라가면 갈수록 급여받을수 있는 금액이 올라갑니다. 한단계마다 거의 10만원가까이 올라갑니다.%}
%\end{figure}

\clearpage
\section{DRG 기타진단의 진단기준}
\subsection{유의한 단백뇨를 동반하지 않은 임신성(임신-유발성)고혈압(O13)}\label{SUPPIH}
 : 정상혈압을 갖고 있던 여성에서 임신20주 이후에 수축기 혈압이 140 mmHg 이상이거나 확장기 혈압이 90mmHg 이상, 6시간 이상의 간격으로 최소한 2번 이상 증명되고 분만 후까지 단백뇨가 동반되지 않고 고혈압으로 남아 있는 경우

\subsection*{임신중독증(O14)}\label{PIH}
 : 아래의 혈압과 단백뇨의 조건이 모두 충족되는 경우
\begin{itemize}\tightlist
\item 혈압 : 정상혈압을 갖고 있던 여성에서 임신20주 이후에 수축기 혈압이 140mmHg 이상이거나 확장기 혈압이 90mmHg 이상, 6시간 이상의 간격으로 최소한 2번 이상 증명된 경우
\item 단백뇨 : 6시간 이상의 간격으로 2+이상(또는 100mg/dl이상) 2번 이상 증명된 경우 또는 24시간 요중에 단백질이 300mg 이상 존재가 확인된 경우
\end{itemize}

\subsection*{중증의 전자간 (O141)}\label{severePIH}
 - 전자간증이면서 다음의 기준 중 1개 이상 충족
\begin{itemize}\tightlist
\item 환자가 침상 안정 상태에서 적어도 6시간 간격으로 2회에 걸쳐 수축기 혈압 160mmHg 이상 또는 확장기 혈압 110mmHg 이상
\item 24시간 채뇨 소변에서 5gm이상의 단백뇨 또는 적어도 4시간 간격으로 2회 채뇨 점적뇨에서 3+이상
\item 24시간 500ml 이하의 핍뇨
\item 대뇌 장애 또는 시력 장애
\item 폐부종 또는 청색증
\item 상복부 또는 우상복부통증
\item 간기능 장애
\item 혈소판 감소증
\item 태아발육지연
\end{itemize}

\subsection*{헬프(HELLP) 증후군 (O142)}\label{HELLP}
 - 다음의 기준을 모두 충족
\begin{itemize}\tightlist
\item 용혈(hemolysis) : Abnormal peripheral blood smear (microangiopathic anemia), Increased bilirubin ≥ 1.2mg/dl, Increased LDH> 600 IU/L
\item 간효소치 상승 : Increased AST ≥ 72 IU/L, Increased LDH as above
\item 저혈소판혈증 Platelet count < 100×103/μl
\end{itemize}

\subsection*{자간증(O15)}\label{eclampsia}
 :임신중독증(O14)의 조건을 충족하면서 임신성 고혈압에 의해 경련(Convulsion)이 동반된 경우
\prezi{\clearpage}
\subsection*{분만 전 출혈(O46)}
 : 분만 전에(활발한 진통이 시작되기 전) 출혈이 있어 입원한 경우 또는 입원하여 분만 전에 출혈량에 관계없이 출혈이 있었던 경우(혈성이슬 제외)
\begin{itemize}\tightlist
\item 응고장애를 동반한 경우 응고장애를 동반한 분만 전 출혈(O460) 부여 가능
\end{itemize}

\subsection*{분만 중 출혈(O67)} : 분만 중 (활발한 진통이 시작된 후부터 태아의 만출까지 : 분만 제 1.2기)에 과다출혈이 있었던 경우로 분만전(입원당시)과 비교 Hct가 10\%이상 감소한 경우이거나 수혈이 필요하여 수혈을 실시한 경우
\begin{itemize}\tightlist
\item 응고장애를 동반한 경우에는 응고장애를 동반한 분만 중 출혈(O670) 부여 가능
\end{itemize}
\subsection*{분만 후 출혈(O72)} : 분만 제3기부터 분만 후 6주 이내(조기산후출혈과 지연산후출혈을 모두 포함)에 과다 출혈이 있었던 경우로 분만전(입원당시)과 비교 Hct가 10\%이상 감소한 경우이거나 수혈이 필요하여 수혈을 실시한 경우
\begin{itemize}\tightlist
\item 응고장애를 동반한 경우에는 O723(분만 후 응고 결여) 부여 가능
\end{itemize}
\prezi{\clearpage}
\subsection*{급성 출혈 후 빈혈(D62)} : 외과적 수술, 처치 후 다량의 출혈로 수술전(입원당시)과 비교 Hgb과 Hct 수치의 10\% 이상 감소 및 Hb 10g/dl 미만으로 저하되어 이에 대한 치료가 이루어진 경우(약제투여, 수혈 등)

\begin{shaded}
단지 출혈이 있었다고 해서 이러한 기타진단을 쓸수 있는것을 아니고, 이러한 출혈에 대한 어떠한 행위가 있어야 기타진단으로 인정됩니다. 분만전,중 출혈에서는 nalador와 merthergin사용, 분만후 출혈에서는 베노훼럼주 사용등이 있습니다. 
\end{shaded}

\subsection*{어느 개인병원에서는}
\begin{enumerate}[가.]\tightlist
\item 제왕절개시에 출혈이 많다고 생각되면 nalador와 merthergin사용하고, 분만중 출혈(o67)코드 넣자. 부인과 수술시 출혈이 많으면 수술중 적절한 처치후 D62 (급성출혈) 상병추가.
\item 3층 입원실에서는 수술후 Hgb이 8이하 이면(절개시 nalador사용과 관계없이) 
\item 3층 입원실에서는 수술전과 후 Hct비교하여서 10\%이상 떨어져 있고, 수술중 nalador사용을 하지 않을시는 
	\begin{itemize}\tightlist
	\item 분만후 출혈(o72)코드 집어 넣고(회진닥터)
	\item 베노훼럼주 1A + N/S 100cc (15분 이상 slowly IV)
	\item 그 이후 볼그래액 1pack를 저녁으로 복용하게 한다. (입원시에는)
	\item 5일째 CBC 추적조사하게 함.(Hct 10\%이상 감소변화 없거나 퇴원후에도 심한 빈혈로 철분제가 필요하면 훼로바-유서방정 아침저녁으로 7일간 복용처방) 
	\item 퇴원 1주후 CBC F/U
	\item 처방의가 필요에 따라서 베노훼럼주의 용량은 늘리수 있습니다. 아래 참조.
	\end{itemize}
\end{enumerate}
\subsection*{\newindex{베노훼럼주} 사용} 
다음의 \pageref{VenoferrumInj}를 참조하세요.

\clearpage

\subsection*{가진통(O47)}
 : 임신 만기 전에 자궁의 불규칙적인 수축으로 인한 통증으로 수축이 자연 소실되거나 자궁경관의 개대가 없는 상태로 분만으로 이어지지 않은 진통으로 확인된 경우
\prezi{\clearpage}
\subsection*{산후기 패혈증(O85)} : 분만 후 첫 24시간을 제외한 산후 10일 이내에 2일간 계속하여 38°C(100.4°F) 이상의 체온상승이 확인된 경우
\prezi{\clearpage}
\subsection*{고령 초임산부의 관리(Z355)} : 초임산부로서 만 35세 이상인 경우 \label{oldprimi} \index{진단코드!고령초임산부의 관리}
\subsection*{어린 초임산부의 관리(Z356)} : 초임산부로서 만 16세 미만인 경우 \label{youngprimi} \index{진단코드!어린 초임산부의 관리}
\subsection*{기타 고위험 임신의 관리(Z358)} : 경산으로 만 40세 이상인 경우와 만 35세 이상인 경산으로 전 출산과 만 5년 이상 Interval이 있는 경우 \label{otherhigh} \index{진단코드!기타 고위험 임신의 관리}
\prezi{\clearpage}
\subsection*{달리 분류되지 않은 처치에 의한 감염(T814)} : 수술부위의 통증, 국소 종창, 발적, 열감 등의 감염 징후를 동반하면서
\begin{itemize}\tightlist
\item 표재성 창상, 심부 절개 부위 및 기관/강 등 외과수술 부위에서 농성 분비물이 나오는 경우
\item 무균 처치시 획득된 체액이나 조직에서 미생물의 배양이 확인된 경우
\item 무균 처치시 획득된 체액이나 조직에서 미생물이 분리된 경우 등으로 외과의사나 주치의사의 판단에 의해 감염으로 진단한 경우
\end{itemize}
\prezi{\clearpage}
\subsection*{인슐린-의존 당뇨병(E10)} : 한국표준질병사인분류에 의하여 당뇨병이 불안정형(brittle), 연소성발병형(juvenile-onset), 케토증경향(ketosis-prone) (typeI)인 경우 또는I형
\begin{shaded}
GDM등도 기타진단으로 가능하다는 최근에 심평원의 대답을 듣었습니다. 물론 GDM때문에 추가적인 BST등을 했을때 인정됩니다.
\end{shaded}
\clearpage

\subsection{기타진단에 따른 중증도점수}
\leftrod{산과}
\tabulinesep =_2mm^2mm
\begin {longtabu} to\linewidth {|X[1.5,l]|X[6.5,l]|X[1.5,l]|X[.7,l]|} \tabucline[.5pt]{-} 
\rowcolor{ForestGreen!40} 상병코드 & 상병명 & 심사기준 & 점수 \\ \tabucline[.5pt]{-} \endhead
\rowcolor{Yellow!40} O11 & 동반된 단백뇨를 동반한 전에 있던 고혈압성 장애 \index{severity!동반된 단백뇨를 동반한 전에 있던 고혈압성 장애} &  & 2 \\ \tabucline[.5pt]{-}
\rowcolor{Yellow!40} O121 & 임신단백뇨\index{severity!임신단백뇨}  & & 2 \\ \tabucline[.5pt]{-}
\rowcolor{Yellow!40} O13 & 유의한 단백뇨를 동반하지 않은 임신성[임신-유발성]고혈압\index{severity!유의한 단백뇨를 동반하지 않은 임신성[임신-유발성]고혈압} & \pageref{SUPPIH} & 2 \\ \tabucline[.5pt]{-}
\rowcolor{Yellow!40} O140 & 중증도의 전자간\index{severity!중증도의 전자간} & \pageref{severePIH} & 2 \\ \tabucline[.5pt]{-}
\rowcolor{Yellow!40} O149 & 상세불명의 전자간\index{severity!상세불명의 전자간} & \pageref{PIH} & 2 \\ \tabucline[.5pt]{-}
\rowcolor{Yellow!40} O230 & 임신중 신장의 감염\index{severity!임신중 신장의 감염} & & 2 \\ \tabucline[.5pt]{-}
\rowcolor{Yellow!40} O231 & 임신중 방광의 감염\index{severity!임신중 방광의 감염} & & 2 \\ \tabucline[.5pt]{-}
\rowcolor{Yellow!40} O232 & 임신중 요도의 감염\index{severity!임신중 요도의 감염} & & 2 \\ \tabucline[.5pt]{-}
\rowcolor{Yellow!40} O233 & 임신중 요로의 기타 부분 감염\index{severity!임신중 요로의 기타 부분 감염} & & 2 \\ \tabucline[.5pt]{-}
\rowcolor{Yellow!40} O234 & 임신중 요로의 상세불명의 감염\index{severity!임신중 요로의 상세불명의 감염} & & 2 \\ \tabucline[.5pt]{-}
\rowcolor{Yellow!40} O235 & 임신중 생식관의 감염\index{severity!임신중 생식관의 감염} & & 2 \\ \tabucline[.5pt]{-}
\rowcolor{Yellow!40} O239 & 기타 및 상세불명의 임신중 비뇨생식관 감염\index{severity!기타 및 상세불명의 임신중 비뇨생식관 감염} & & 2 \\ \tabucline[.5pt]{-}
\rowcolor{Yellow!40} O240 & 전에 있던 인슐린-의존성 당뇨병\index{severity!전에 있던 인슐린-의존성 당뇨병} & & 2 \\ \tabucline[.5pt]{-}
\rowcolor{Yellow!40} O241 & 전에 있던 인슐린-비의존성 당뇨병\index{severity!전에 있던 인슐린-비의존성 당뇨병} & & 2 \\ \tabucline[.5pt]{-}
\rowcolor{Yellow!40} O242 & 전에 있던 영양실조-관련 당뇨병\index{severity!전에 있던 영양실조-관련 당뇨병} & & 2 \\ \tabucline[.5pt]{-}
\rowcolor{Yellow!40} O243 & 상세불명의 전에 있던 당뇨병\index{severity!상세불명의 전에 있던 당뇨병} & & 2 \\ \tabucline[.5pt]{-}
\rowcolor{Yellow!40} O244 & 임신중 생긴 당뇨병\index{severity!임신중 생긴 당뇨병} & & 2 \\ \tabucline[.5pt]{-}
\rowcolor{Yellow!40} O249 & 상세불명의 임신중 당뇨병\index{severity!상세불명의 임신중 당뇨병} & & 2 \\ \tabucline[.5pt]{-}
\rowcolor{Yellow!40} O266 & 임신, 출산 및 산후기중 간 장애\index{severity!임신, 출산 및 산후기중 간 장애} & & 2 \\ \tabucline[.5pt]{-}
\rowcolor{Yellow!40} O294 & 임신중 척추 및 경막외 마취로 유발된 두통\index{severity!임신중 척추 및 경막외 마취로 유발된 두통} & & 2 \\ \tabucline[.5pt]{-}
\rowcolor{Yellow!40} O295 & 임신중 척추 및 경막외 마취의 기타 합병증\index{severity!임신중 척추 및 경막외 마취의 기타 합병증} & & 2 \\ \tabucline[.5pt]{-}
\rowcolor{Yellow!40} O296 & 임신중 실패한 또는 어려운 삽관\index{severity!임신중 실패한 또는 어려운 삽관} & & 2 \\ \tabucline[.5pt]{-}
\rowcolor{Yellow!40} O298 & 임신중 마취의 기타 합병증\index{severity!임신중 마취의 기타 합병증} & & 2 \\ \tabucline[.5pt]{-}
\rowcolor{Yellow!40} O299 & 상세불명의 임신중 마취의 합병증\index{severity!상세불명의 임신중 마취의 합병증} & & 2 \\ \tabucline[.5pt]{-}
\rowcolor{Yellow!40} O360 & 리서스 동종면역의 산모관리\index{severity!리서스 동종면역의 산모관리} & & 2 \\ \tabucline[.5pt]{-}
\rowcolor{Yellow!40} O361 & 기타 동종면역의 산모관리\index{severity!기타 동종면역의 산모관리} & & 2 \\ \tabucline[.5pt]{-}
\rowcolor{Yellow!40} O411 & 양막낭 및 양막의 감염\index{severity!양막낭 및 양막의 감염} & & 2 \\ \tabucline[.5pt]{-}
\rowcolor{Yellow!40} O440 & 출혈이 없다고 명시된 전치태반\index{severity!출혈이 없다고 명시된 전치태반} & & 2 \\ \tabucline[.5pt]{-}
\rowcolor{Yellow!40} O441 & 출혈을 동반한 전치태반\index{severity!출혈을 동반한 전치태반} & & 2 \\ \tabucline[.5pt]{-}
\rowcolor{Yellow!40} O468 & 기타 분만전 출혈\index{severity!기타 분만전 출혈} & & 2 \\ \tabucline[.5pt]{-}
\rowcolor{Yellow!40} O469 & 상세불명의 분만전 출혈\index{severity!상세불명의 분만전 출혈} & & 2 \\ \tabucline[.5pt]{-}
\rowcolor{Yellow!40} O4700 & 임신 37주 전 제2 삼분기의 가진통\index{severity!임신 37주 전 제2 삼분기의 가진통} & & 2 \\ \tabucline[.5pt]{-}
\rowcolor{Yellow!40} O4701 & 임신 37주 전 제3 삼분기의 가진통\index{severity!임신 37주 전 제3 삼분기의 가진통} & & 2 \\ \tabucline[.5pt]{-}
\rowcolor{Yellow!40} O4709 & 임신 37주 전 상세불명의 삼분기의 가진통\index{severity!임신 37주 전 상세불명의 삼분기의 가진통} & & 2 \\ \tabucline[.5pt]{-}
\rowcolor{Yellow!40} O471 & 임신 37주 후의 가진통\index{severity!임신 37주 후의 가진통} & & 2 \\ \tabucline[.5pt]{-}
\rowcolor{Yellow!40} O479 & 상세불명의 가진통\index{severity!상세불명의 가진통} & & 2 \\ \tabucline[.5pt]{-}
\rowcolor{Yellow!40} O678 & 기타 분만중 출혈\index{severity!기타 분만중 출혈} & & 2 \\ \tabucline[.5pt]{-}
\rowcolor{Yellow!40} O679 & 상세불명의 분만중 출혈\index{severity!상세불명의 분만중 출혈} & & 2 \\ \tabucline[.5pt]{-}
\rowcolor{Yellow!40} O720 & 제3기 출혈\index{severity!제3기 출혈} & & 2 \\ \tabucline[.5pt]{-}
\rowcolor{Yellow!40} O721 & 기타 분만직후 출혈\index{severity!기타 분만직후 출혈} & & 2 \\ \tabucline[.5pt]{-}
\rowcolor{Yellow!40} O722 & 지연성 및 이차성 분만후 출혈\index{severity!지연성 및 이차성 분만후 출혈} & & 2 \\ \tabucline[.5pt]{-}
\rowcolor{Yellow!40} O723 & 분만후 응고결손\index{severity!분만후 응고결손} & & 2 \\ \tabucline[.5pt]{-}
\rowcolor{Yellow!40} O745 & 진통 및 분만중 척수 및 경막외마취-유발성 두통\index{severity!진통 및 분만중 척수 및 경막외마취-유발성 두통} & & 2 \\ \tabucline[.5pt]{-}
\rowcolor{Yellow!40} O746 & 진통 및 분만중 척수 또는 경막외마취의 기타 합병증\index{severity!진통 및 분만중 척수 또는 경막외마취의 기타 합병증} & & 2 \\ \tabucline[.5pt]{-}
\rowcolor{Yellow!40} O747 & 진통 및 분만중 실패한 또는 어려운 삽관\index{severity!진통 및 분만중 실패한 또는 어려운 삽관} & & 2 \\ \tabucline[.5pt]{-}
\rowcolor{Yellow!40} O748 & 기타 진통 및 분만중 마취의 합병증\index{severity!기타 진통 및 분만중 마취의 합병증} & & 2 \\ \tabucline[.5pt]{-}
\rowcolor{Yellow!40} O749 & 상세불명의 진통 및 분만중 마취의 합병증\index{severity!상세불명의 진통 및 분만중 마취의 합병증} & & 2 \\ \tabucline[.5pt]{-}
\rowcolor{Yellow!40} O753 & 진통중 기타 감염\index{severity!진통중 기타 감염} & & 2 \\ \tabucline[.5pt]{-}
\rowcolor{Yellow!40} O860 & 산과수술 상처의 감염\index{severity!산과수술 상처의 감염} & & 2 \\ \tabucline[.5pt]{-}
\rowcolor{Yellow!40} O861 & 분만에 따른 생식관의 기타감염\index{severity!분만에 따른 생식관의 기타감염} & & 2 \\ \tabucline[.5pt]{-}
\rowcolor{Yellow!40} O862 & 분만에 따른 요로감염\index{severity!분만에 따른 요로감염} & & 2 \\ \tabucline[.5pt]{-}
\rowcolor{Yellow!40} O863 & 분만에 따른 기타 비뇨생식관감염\index{severity!분만에 따른 기타 비뇨생식관감염} & & 2 \\ \tabucline[.5pt]{-}
\rowcolor{Yellow!40} O864 & 분만후 원인 불명 열\index{severity!분만후 원인 불명 열} & & 2 \\ \tabucline[.5pt]{-}
\rowcolor{Yellow!40} O868 & 기타 명시된 산후기 감염\index{severity!기타 명시된 산후기 감염} & & 2 \\ \tabucline[.5pt]{-}
\rowcolor{Yellow!40} O870 & 산후기중 표재성 혈전정맥염\index{severity!산후기중 표재성 혈전정맥염} & & 2 \\ \tabucline[.5pt]{-}
\rowcolor{Yellow!40} O871 & 산후기중 심부정맥혈전증\index{severity!산후기중 심부정맥혈전증} & & 2 \\ \tabucline[.5pt]{-}
\rowcolor{Yellow!40} O894 & 산후기중 척수 및 경막외마취-유발두통\index{severity!산후기중 척수 및 경막외마취-유발두통} & & 2  \\ \tabucline[.5pt]{-}
\rowcolor{Yellow!40} O895 & 산후기중 척수 및 경막외마취의 기타 합병증\index{severity!산후기중 척수 및 경막외마취의 기타 합병증} & & 2 \\ \tabucline[.5pt]{-}
\rowcolor{Yellow!40} O896 & 산후기중 실패한 또는 어려운 삽관\index{severity!산후기중 실패한 또는 어려운 삽관} & & 2 \\ \tabucline[.5pt]{-}
\rowcolor{Yellow!40} O898 & 기타 산후기중 마취의 합병증\index{severity!기타 산후기중 마취의 합병증} & & 2 \\ \tabucline[.5pt]{-}
\rowcolor{Yellow!40} O899 & 상세불명의 산후기중 마취의 합병증\index{severity!상세불명의 산후기중 마취의 합병증} & & 2 \\ \tabucline[.5pt]{-}
\rowcolor{Yellow!40} O900 & 제왕절개상처의 파열\index{severity!제왕절개상처의 파열} & & 2 \\ \tabucline[.5pt]{-}
\rowcolor{Yellow!40} O901 & 산과적 회음상처의 파열\index{severity!산과적 회음상처의 파열}& & 2 \\ \tabucline[.5pt]{-}
\rowcolor{Yellow!40} O902 & 산과적 상처의 혈종\index{severity!산과적 상처의 혈종} & & 2 \\ \tabucline[.5pt]{-}
\rowcolor{Yellow!40} O903 & 산후기 심근병증\index{severity!산후기 심근병증} & & 2 \\ \tabucline[.5pt]{-}
\rowcolor{Yellow!40} O904 & 분만후 급성 신부전\index{severity!분만후 급성 신부전} & & 2 \\ \tabucline[.5pt]{-}
\rowcolor{Yellow!40} O910 & 출산과 관련된 유두의 감염\index{severity!출산과 관련된 유두의 감염} & & 2 \\ \tabucline[.5pt]{-}
\rowcolor{Yellow!40} O911 & 출산과 관련된 유방의 농양\index{severity!출산과 관련된 유방의 농양} & & 2 \\ \tabucline[.5pt]{-}
\rowcolor{Yellow!40} Z355 & 고령 초임산부의 관리\index{severity!고령 초임산부의 관리} & \pageref{oldprimi} & 2 \\ \tabucline[.5pt]{-}
\rowcolor{Yellow!40} Z356 & 어린 초임산부의 관리\index{severity!어린 초임산부의 관리} & \pageref{youngprimi} & 2 \\ \tabucline[.5pt]{-}
\rowcolor{Yellow!40} Z358 & 기타 고위험 임신의 관리\index{severity!기타 고위험 임신의 관리} & \pageref{otherhigh} & 2 \\ \tabucline[.5pt]{-}
\rowcolor{Yellow!40} O141 & 중증의 전자간\index{severity!중증의 전자간} & & 3 \\ \tabucline[.5pt]{-}
\rowcolor{Yellow!40} O142 & 헬프증후군\index{severity!헬프증후군} & & 3 \\ \tabucline[.5pt]{-}
\rowcolor{Yellow!40} O150 & 임신중 자간\index{severity!임신중 자간} & & 3 \\ \tabucline[.5pt]{-}
\rowcolor{Yellow!40} O151 & 분만중 자간\index{severity!분만중 자간} & & 3 \\ \tabucline[.5pt]{-}
\rowcolor{Yellow!40} O152 & 산후기 자간\index{severity!산후기 자간} & & 3 \\ \tabucline[.5pt]{-}
\rowcolor{Yellow!40} O159 & 시기 상세불명의 자간\index{severity!시기 상세불명의 자간} & & 3 \\ \tabucline[.5pt]{-}
\rowcolor{Yellow!40} O293 & 임신중 국소 마취에 대한 독성 반응\index{severity!임신중 국소 마취에 대한 독성 반응} & & 3 \\ \tabucline[.5pt]{-}
\rowcolor{Yellow!40} O744 & 진통 및 분만중 국소마취에 대한 독성 반응\index{severity!진통 및 분만중 국소마취에 대한 독성 반응} & & 3 \\ \tabucline[.5pt]{-}
\rowcolor{Yellow!40} O85 & 산후기 패혈증\index{severity!산후기 패혈증} & & 3 \\ \tabucline[.5pt]{-}
\rowcolor{Yellow!40} O893 & 산후기중 국소마취에 대한 독성 반응\index{severity!산후기중 국소마취에 대한 독성 반응} & & 3 \\ \tabucline[.5pt]{-}
\rowcolor{Yellow!40} O290 & 임신중 마취의 폐 합병증\index{severity!임신중 마취의 폐 합병증} & & 4 \\ \tabucline[.5pt]{-}
\rowcolor{Yellow!40} O291 & 임신중 마취의 심장 합병증\index{severity!임신중 마취의 심장 합병증} & & 4 \\ \tabucline[.5pt]{-}
\rowcolor{Yellow!40} O292 & 임신중 마취의 중추신경계통 합병증\index{severity!임신중 마취의 중추신경계통 합병증} & & 4 \\ \tabucline[.5pt]{-}
\rowcolor{Yellow!40} O450 & 응고장애를 동반한 태반조기분리\index{severity!응고장애를 동반한 태반조기분리} & & 4 \\ \tabucline[.5pt]{-}
\rowcolor{Yellow!40} O458 & 기타 태반조기분리\index{severity!기타 태반조기분리} & & 4 \\ \tabucline[.5pt]{-}
\rowcolor{Yellow!40} O459 & 상세불명의 태반조기분리\index{severity!상세불명의 태반조기분리} & & 4 \\ \tabucline[.5pt]{-}
\rowcolor{Yellow!40} O460 & 응고 장애를 동반한 분만전출혈\index{severity!응고 장애를 동반한 분만전출혈} & & 4 \\ \tabucline[.5pt]{-}
\rowcolor{Yellow!40} O670 & 응고 장애를 동반한 분만중 출혈\index{severity!응고 장애를 동반한 분만중 출혈} & & 4 \\ \tabucline[.5pt]{-}
\rowcolor{Yellow!40} O710 & 진통 시작 전의 자궁 파열\index{severity!진통 시작 전의 자궁 파열} & & 4 \\ \tabucline[.5pt]{-}
\rowcolor{Yellow!40} O711 & 분만중 자궁파열\index{severity!분만중 자궁파열} & & 4 \\ \tabucline[.5pt]{-}
\rowcolor{Yellow!40} O740 & 진통 및 분만중 마취제로 인한 흡입폐렴\index{severity!진통 및 분만중 마취제로 인한 흡입폐렴} & & 4 \\ \tabucline[.5pt]{-}
\rowcolor{Yellow!40} O741 & 진통 및 분만중 마취로 인한 기타 폐합병증\index{severity!진통 및 분만중 마취로 인한 기타 폐합병증} & & 4 \\ \tabucline[.5pt]{-}
\rowcolor{Yellow!40} O742 & 진통 및 분만중 마취로 인한 심장합병증\index{severity!진통 및 분만중 마취로 인한 심장합병증} & & 4 \\ \tabucline[.5pt]{-}
\rowcolor{Yellow!40} O743 & 진통 및 분만중 마취로 인한 중추신경계통 합병증\index{severity!진통 및 분만중 마취로 인한 중추신경계통 합병증} & & 4 \\ \tabucline[.5pt]{-}
\rowcolor{Yellow!40} O751 & 진통 및 분만중 또는 그 후에 뒤따르는 쇼크\index{severity!진통 및 분만중 또는 그 후에 뒤따르는 쇼크} & & 4 \\ \tabucline[.5pt]{-}
\rowcolor{Yellow!40} O873 & 산후기중 대뇌정맥 혈전증\index{severity!산후기중 대뇌정맥 혈전증} & & 4 \\ \tabucline[.5pt]{-}
\rowcolor{Yellow!40} O880 & 산과적 공기 색전증\index{severity!산과적 공기 색전증} & & 4 \\ \tabucline[.5pt]{-}
\rowcolor{Yellow!40} O881 & 양수색전증\index{severity!양수색전증} & & 4 \\ \tabucline[.5pt]{-}
\rowcolor{Yellow!40} O882 & 산과적 피떡 색전증\index{severity!산과적 피떡 색전증} & & 4 \\ \tabucline[.5pt]{-}
\rowcolor{Yellow!40} O883 & 산과적 농혈성 및 패혈성 색전증\index{severity!산과적 농혈성 및 패혈성 색전증} & & 4 \\ \tabucline[.5pt]{-}
\rowcolor{Yellow!40} O888 & 기타 산과적 색전증\index{severity!기타 산과적 색전증} & & 4 \\ \tabucline[.5pt]{-}
\rowcolor{Yellow!40} O890 & 산후기중 마취의 폐 합병증\index{severity!산후기중 마취의 폐 합병증} & & 4 \\ \tabucline[.5pt]{-}
\rowcolor{Yellow!40} O891 & 산후기중 마취의 심장합병증\index{severity!산후기중 마취의 심장합병증} & & 4 \\ \tabucline[.5pt]{-}
\rowcolor{Yellow!40} O892 & 산후기중 마취의 중추신경계통 합병증\index{severity!산후기중 마취의 중추신경계통 합병증} & & 4 \\ \tabucline[.5pt]{-}
\end{longtabu}
\prezi{\clearpage}
\leftrod{부인과}
%\subsection{산과}
\tabulinesep =_2mm^2mm
\begin {longtabu} to\linewidth {|X[1.5,l]|X[6.5,l]|X[1.5,l]|X[.7,l]|} \tabucline[.5pt]{-} 
\rowcolor{ForestGreen!40} 상병코드 & 상병명 & 심사기준 & 점수 \\ \tabucline[.5pt]{-} \endhead 
\rowcolor{Yellow!40} D62 & 급성 출혈후 빈혈\index{severity!급성출혈후 빈혈} & CBC F/U other.. & 2 \\ \tabucline[.5pt]{-}
\rowcolor{Yellow!40} T814 & 처치후 봉합농양\index{severity!처치후 봉합농양} & & 2 \\ \tabucline[.5pt]{-}
\end{longtabu}
\prezi{\clearpage}

\begin{tcolorbox}[frogbox,title=기타진단 기준tip]
\begin{itemize}\tightlist
\item Gestational DM : 입원하여 FBS, BS check
\item PPH : hysterotonic drug사용하고, postoperative CBC 체크상 Hct가 10이상 떨어져 있는 경우
\item 고위험산모에 대한 정확한 이해 및 숙지
	\begin{itemize}\tightlist
	\item 고령 초임산부의 관리 (Z355) : 초임산부로서 만 35세 이상
	\item 기타고위험 임신의 관리 (Z358) : 경산으로 만 40세 이상인 경우와 만 35세 이상인 경산으로 만 5년 이상 interval이 있는 경우
	\end{itemize}
\item Preterm입원후 6일내에 제왕절개시 부가진단 입력(O47) http://goo.gl/7YBexP
\item severe PIH등으로 제왕절개시 주진단으로 상병 넣지 말고 다른 CPD등 주진단으로 하고 부진단 http://goo.gl/q9p5G6
\item 무통분만으로 인한 leakage시 o295 임신중 척추및 경마괴 마취로 유발된 통증
\item 검사하는 것을 귀찮아하지 말고 무엇이든 근거를 남겨라. 
\end{itemize}
\end{tcolorbox}

\clearpage
\section{포괄수가제}
\subsection{7개 질병군 포괄수가제란? (2013년 7월부터)}
환자가 입원해서 퇴원할 때까지 발생하는 진료에 대하여 질병마다 미리 정해진 금액을 내는 제도입니다. 입원비가 하나로 묶여있다고 생각하시면 됩니다. (\textcolor{red}{같은 질병이라도 환자의 합병증이나 타상병 동반여부에 따라 가격은 달라질 수} 있습니다.)\\
\prezi{\clearpage}
\emph{적용대상질병군}\\ 
현재는 4개 진료과 7개 질병군을 대상으로 시행중
\begin{itemize}\tightlist
\item 안과 : 백내장수술(수정체 수술) 
\item 이비인후과 : 편도수술 및 아데노이드 수술 
\item 외과 : 항문수술(치질 등), 탈장수술(서혜 및 대퇴부), 맹장수술(충수절제술) 
\item 산부인과 : \textcolor{blue}{제왕절개분만, 자궁 및 자궁부속기(난소, 난관 등)수술(악성종양 제외), 자궁외임신제외}
\end{itemize}
※ 수정체수술(백내장수술), 서혜 및 대퇴부 탈장수술(장관절제 미동반) 등 간단한 항문수술의 경우에는 6시간 미만 관찰 후 당일 귀가 또는 이송시에도 포괄수가제(DRG설명보기)가 적용되어 본인부담금은 입원부담률인 20\%로 적용받게 됩니다. 다만 7개 질병군에 해당되는 수술을 받았어도 \textcolor{red}{의료급여 대상자 및 혈우병 환자와 HIV감염자(인체면역결핍바이러스병)}는 포괄수가제(DRG) 적용에서 제외됩니다.
\prezi{\clearpage}
\begin{Cdoing}{우리나라 포괄수가제}
질병군(DRG) 포괄수가는 국민건강보험법시행령 제21조제3항제2호에 따라 복지부장관이 별도 고시하는 7개 질병군으로 입원진료를 받은 경우에 적용하며, 질병군 입원진료는 질병군 급여 일반원칙에 따라 다음의 항목을 포함하고 있습니다.\\
- 다 음 -
\begin{itemize}\tightlist
\item 7개 질병군으로 응급실ㆍ수술실 등에서 수술을 받고 연속하여 6시간 이상 관찰 후 귀가 또는 이송한 경우 
\item 7개 질병군 중 수정체수술(대절개 단안 및 양안, 소절개 단안 및 양안), 기타항문수술, 서혜 및 대퇴부탈장수술 단측 및 양측(복강경 이용 포함)의 수술을 받고 6시간 이상 관찰 후 당일 귀가 또는 이송한 경우
\end{itemize}
\end{Cdoing}

\clearpage
\section{제1부 질병군 급여 일반원칙}
\begin{enumerate}[1.]\tightlist
\item 상급종합병원, 종합병원, 병원(요양병원을 포함한다), 의원(보건의료원을 포함한다)인 요양기관이 국민건강보험법 시행령(이하 “영”이라 한다) 제 21조 제3항제2호 및 국민건강보험 요양급여의 기준에 관한 규칙(이하 “요양급여기준”이라 한다) 제8조제3항에 따라 포괄적인 행위가 적용되는 질병군에 대한 입원진료를 하는 경우에 적용한다.
\item 가입자 또는 피부양자(이하 “가입자 등”이라한다)가 질병군으로 입원진료를 받은 경우에 적용하되, 다음의 각 항목은 질병군 적용에서 제외하고 제 1편을 적용한다.
	\begin{enumerate}[가.]\tightlist
	\item 혈우병환자, HIV감염자
	\item 입원일수가 30일을 초과할 경우 31일째부터 발생하는 진료분
	\item 차상위 본인부담경감대상자로서 제3호 나목에 해당하는 경우
	\item 질병군 진료 이외의 목적으로 입원하여 입원일수가 6일을 초과한 시점에 예상치 못하게 질병군 수술이 이루어진 경우 입원일로부터 수술시행일 전일까지의 진료분
	\end{enumerate}
\item 제2호 규정에 따른 질병군 입원진료에는 다음의 각 항목을 포함한다.
	\begin{enumerate}[가.]\tightlist
	\item 제2부 각 장에 분류된 질병군으로 응급실ㆍ수술실 등에서 수술을 받고 연속하여 6시간 이상 관찰 후 귀가 또는 이송한 경우
	\item 제2부 각 장에 분류된 질병군 중 수정체 소절개 수술 단안, 수정체 소절개 수술 양안, 수정체 대절개 수술 단안, 수정체 대절개 수술 양안, 기타항문 수술, 서혜 및 대퇴부 탈장수술(장관절제 미동반) 단측, 서혜 및 대퇴부 탈장수술(장관절제 미동반) 양측, 복강경을 이용한 서혜 및 대퇴부 탈장수술(장관절제 미동반) 단측, 복강경을 이용한 서혜 및 대퇴부 탈장수술(장관절제 미동반) 양측 질병군으로 수술을 받고 6시간 미만 관찰 후 당일 귀가 또는 이송하는 경우
	\end{enumerate}
\item 제2부 각 장에 분류된 질병군 상대가치점수(이하 “점수”라 한다)는 다음 각목의 행위ㆍ약제 및 치료재료를 포함한다.
	\begin{enumerate}[가.]\tightlist
	\item 제1편 행위 급여ㆍ비급여 목록 및 급여 상대가치점수에서 정한 행위급여목록표에 고시된 행위
	\item 요양급여기준 제8조제2항의 규정에 의하여 고시된 약제 급여 목록 및 급여 상한금액표의 약제와 치료재료 급여ㆍ비급여 목록 및 급여 상한 금액표의 치료재료
	\item 요양급여기준 별표 2의 비급여대상 중 제6호의 비급여대상을 제외한 행위ㆍ약제 및 치료재료
	\item 국민건강보험법 시행규칙 별표 6의 본인이 요양급여비용의 100분의 100을 부담하는 항목 중 제1호 자목에 해당하는 항목을 제외한 행위ㆍ약제 및 치료재료
	\item 다음 항목 중 위 가목 내지 라목에 해당하는 경우
		\begin{enumerate}[(1)]\tightlist
		\item 요양급여기준 별표 1 제1호 마목에서 장관이 정하는 바에 따라 다른 기관에 검사를 위탁하거나 당해 요양기관에 소속되지 아니한 전문성이 뛰어난 의료인을 초빙하거나, 또는 다른 요양기관에서 보유하고 있는 양질의 시설ㆍ인력 및 장비를 공동 사용하는 경우 소요되는 행위ㆍ약제 및 치료재료
		\item 입ㆍ퇴원 당일에 발생한 행위ㆍ약제 및 치료재료로써 외래진료 및 퇴원약제 등을 포함하되 다음 항목은 제외한다.
			\begin{enumerate}[(가)]\tightlist
			\item 질병군 입원을 예견하지 못한 상태에서 입원 당일 외래진료를 받은 경우의 원외처방 약제비
			\item 질병군으로 퇴원 후 질병군과 관계없는 상병으로 퇴원 당일 외래진료를 받은 경우의 원외처방 약제비
			\item 질병군으로 퇴원 후 질병군 질환과 관계없는 상병으로 퇴원 당일 재입원하는 경우의 요양급여비용
			\end{enumerate}
		\item 요양기관의 요구에 의하여 가입자 등이 외부에서 직접 구입한 약제 및 치료재료
		\end{enumerate}
	\end{enumerate}	
\item 질병군에 대한 요양급여비용을 산정할 때에는 제2부 각 장에 분류된 질병군 점수를 기준으로 별표 1의 질병군별 점수 산정요령에 의하여 산정된 점수 총합에 국민건강보험법 제45조제3항과 영 제21조제1항에 따른 점수당 단가를 곱하여 10원 미만을 절사한 금액을 요양급여비용 총액으로 산정한다.\par
이 경우 위 금액 외에 식대를 포함한 별도로 산정하는 비용이 있는 경우에는 각각의 산정방식에 의하여 산정된 금액을 합산한다.
\item 제5호 본문에도 불구하고 질병군별 금액 산정시 점수당 단가는 별표 2의 질병군 행위 및 약제ㆍ치료재료 구성비율에 따른 행위부분 점수와 매년 상한금액 변화를 적용한 약제ㆍ치료재료 금액을 점수당 단가로 나눈 점수를 합한 점수(소수점 이하 셋째 자리에서 4사5입)에 적용한다.\par
<산식> \par
질병군별 금액 = \{질병군별 행위 점수 + (약제ㆍ치료재료 금액 ÷ 점수당단가)\} × 점수당 단가
\item 제5호에 따라 산정한 요양급여비용의 총액이 영 제21조제1항 내지 제3항및 요양급여기준(별표 2 제6호를 제외한다)에 의하여 산정한 총액보다 적고 그 차액이 100만원을 초과하는 경우(이 경우를 요양급여비용열외군이라 한다)에는 위 제5호에 따른 금액에 100만원을 초과하는 금액(10원 미만 절사)을 합한 금액을 요양급여비용 총액으로 산정한다.
\item 가입자 또는 피부양자가 제1호에 따른 요양기관(제3편을 적용받는 요양병원은 제외)에서 「국민건강보험법」제43조에 따라 신고한 일반입원실 및 정신과폐쇄병실의 4인실 또는 5인실을 이용한 경우에는 별표 2의3의 추가 비용 계산식에 따른 금액을 추가 산정하고, 상급종합병원의 일반입원실 및 정신과폐쇄병실의 1인실(보건복지부장관이 정하여 고시하는 불가피한 1인실 입원의 경우 제외)을 이용한 경우에는 제5호 본문에 따른 금액에서 1인실 이용일수에 해당하는 기본입원료(제1편제2부제1장 가-2-가)를 제외 하고 산정한다.
\item 영 별표 2 제2호 나목의 “보건복지부장관이 정하여 고시하는 입원실을 이용한 경우”라 함은 가입자 등이 제1호에 따른 요양기관에서 국민건강보험법 제43조에 따라 신고한 일반입원실 및 정신과폐쇄병실의 4인실 또는 5인실을 이용한 경우를 말하며, 별표 2의3의 본인부담액 계산식에 따른 금액을 더하여 본인부담액을 산정한다.
\item 영 별표 2 제2호 다목의 “그 고시에서 정한 금액”이라 함은 제7호 중 100만원 초과분에 해당하는 금액을 말한다.
\item (별표 2의1)에 열거한 항목을 외과 전문의가 시행한 경우에는 소정점수의 30\%에 대한 각 요양기관별 종별가산율을 적용한 금액을 추가 산정한다.
\item 18시-09시 또는 공휴일에 응급진료가 불가피하여 수술을 행한 경우에는 해당 질병군의 야간ㆍ공휴 소정점수를 추가 산정한다. 이 경우 수술 또는 마취를 시작한 시간을 기준으로 산정한다.
\item 질병군 요양급여를 실시하는 요양기관은 질병군 입원환자의 질병군 분류 번호와 관련한 주진단 및 기타진단, 수술명 등은 진료기록부에 근거하여 정확한 코드를 부여하여야 하며, 진단명이 입원시부터 존재하였는지 여부를 확인할 수 있도록 진료기록부에 기록하고, 의료의 질 향상을 위한 점검표를 별지 서식에 따라 작성하여야 한다.
\item 입원 중인 환자를 제2부 각 장에 분류된 질병군 중 수정체 소절개 수술 단안, 수정체 소절개 수술 양안, 수정체 대절개 수술 단안, 수정체 대절개 수술 양안의 진료를 위해 다른 요양기관으로 의뢰하여 질병군 진료를 실시한 경우 해당 요양급여비용은 의뢰받은 요양기관에서 질병군으로 적용 한다.
\item 질병군 진료 시 초음파검사는 「요양급여의 적용기준 및 방법에 관한 세부사항」제2장 검사료 초음파검사 세부인정기준을 적용하며, \highlight{인정기준에 의한 급여대상에 해당되는 경우에는} 제2부 각 장에 분류된 질병군 점수이외에 제1편 제2부 초음파검사료를 추가 산정한다. 
	\begin{itemize}\tightlist
	\item EZ986 분만기간 초음파 : 분만을 위한 입원기간 동안 발생한 초음파 검사를 모두 의미함. 제왕절개를 위해 입원한 환자들의 경우는 옆의 비급여초음파를 최소한 2번 이상 실시하고 청구한다.
	\item EZ887 초음파를 이용한 태아 생물리학 계수( Biophysical Profile )
	\item 임신 유지목적으로 입원하여 6일 이내에 제왕절개분만이 이루어진 경우 : 분만기간 초음파(비급여)로 청구한다. 조산통으로 입원한 경우엔 2일에 한번씩은 초음파를 본다.
	\item 임신 유지목적으로 입원하여 입원일수가 6일을 초과한 시점에서 예상치 못하게 제왕절개분만이 이루어진 경우 
		\begin{itemize}\tightlist
		\item 입원(행위별 청구) : (정상임신부) 7회까지 급여, 그 외 비급여(태아의 이상이나 이상이 예측되는 경우) 급여
		\item 분리청구 시점 구분
		\item 제왕절개분만 입원(DRG 청구) : 분만기간 초음파(비급여)
         \end{itemize}                               
	\item 분만과 연결된 입원: 분만기간이 장기로 길어진 경우 분리청구 시점 기준으로 적용
		\begin{itemize}\tightlist
		\item 입원(행위별 청구) : (정상임신부) 7회까지 급여, 그 외 비급여(태아의 이상이나 이상이 예측되는 경우) 급여
		\item 분리청구 시점 구분
		\item 자연분만및 제왕절개분만 입원제왕절개분만 입원(DRG 청구) : 분만기간 초음파(비급여)
         \end{itemize} 	
%	\item 질병군 진료 이외의 목적으로 입원하여 입원일수가 6일을 초과한 시점에 예상치 못하게 질병군 수술이 이루어진 경우 입원일로부터 수술시행일 전일까지의 진료분을 제외한 경우의 보험 초음파등(6일전의 조기진통등으로 입원하여 제왕절개분만한 경우 횟수초과 급여초음파 청구 (해당 삼분기의 일반 또는 일반의 제한초음파 산정). 단, 1일 1회만 청구 가능함)
     \end{itemize}
\item 별표 2의4에 열거한 항목에 해당하는 행위 및 치료재료는 제1편 제2부 행위 급여 상대가치점수와 「약제 및 치료재료의 비용에 대한 결정기준」에 의한 금액을 추가 산정한다.
\item 영 별표 2 제4호에 따른 요양급여 항목 및 본인부담률은 별표 2의5와 같다. 이 경우 별표 2의5에 열거한 항목에 해당하는 행위 및 치료재료는 「요양급여의 적용기준 및 방법에 관한 세부사항」을 적용하며, 인정기준에 의한 급여대상에 해당되는 경우에는 제1편 제2부 행위 급여 상대가치 점수와 「약제 및 치료재료의 비용에 대한 결정기준」에 의한 금액을 추가 산정한다.
\item 질병군 진료시 마취통증의학과 전문의를 초빙하여 마취를 실시한 경우에는 제1편제2부제6장 바-2의 마취통증의학과 전문의 초빙료를 추가 산정하며, 제1편제2부제6장 및「요양급여의 적용기준 및 방법에 관한 세부사항」의 마취통증의학과 전문의 초빙료 산정 관련 규정을 적용한다.
\item 질병군 진료시 질병군 분류번호를 결정하는 주된 수술 이외에 제1편제2부 제9장제1절(기본처치 제외) 또는 제10장제3절ㆍ제4절의 수술을 실시한 경우에는 해당 수술 소정점수를 추가 산정한다. 다만, 주된 수술과 동일 피부 절개 하에 실시되는 수술은 해당 수술 소정점수의 70\%를 산정한다.
\item 질병군 진료 시 제1편제2부제1장 5.가에 따른 의료질평가지원금은 가-22의 각 분야별 등급별 ‘입원’의 소정점수를 질병군 입원일수와 동일하게 추가 산정한다.
\item 질병군 진료 시 제1편제2부제19장제2절에 따른 (별표 2) 및 (별표 3)의 응급의료행위를 실시하는 경우에는 제1편에서 정하고 있는 해당 소정점수의 50\%를 추가 산정하고, 제1편제2부제19장제2절의 산정지침 3. 내지 5. 및 「요양급여의 적용기준 및 방법에 관한 세부사항」을 적용한다.
\item「의료법」제3조의5에 따라 전문병원으로 지정받은 의료기관에서 질병군 진료 시 제1편제2부제1장 산정지침 6.에 따른 전문병원 관리료 등은 가-24-가 전문병원 입원관리료와 가-24-1-가 전문병원 입원의료질지원금의 해당 소정점수를 질병군 입원일수에 따라 추가 산정한다.
\item 질병군 진료 시 감염예방ㆍ관리 활동을 실시하는 경우에는 제1편제2부제1장 가-25의 감염예방ㆍ관리료를 추가 산정하고, 「요양급여의 적용기준및 방법에 관한 세부사항」을 적용한다.
\item 질병군 진료시 통증자가조절법(Patient Controlled Analgesia)을 실시한 경우 제1편제2부 행위 급여 상대가치점수와 「약제 및 치료재료의 비용에 관한 결정기준」에 의한 금액을 추가 산정하고, 「요양급여의 적용기준및 방법에 관한 세부사항」을 적용한다.
\end{enumerate}

\clearpage
\subsection{제4장 산부인과 [적용지침]}
\begin{enumerate}[1.]\tightlist
\item 요양기관종별로 「복강경을 이용한 자궁적출술(악성종양제외)」,「기타 자궁적출술(악성종양제외)」,「복강경을 이용한 기타 자궁 수술(악성종양제외)」,「기타 자궁 수술(악성종양제외)」,「복강경을 이용한 자궁부속기 수술(악성종양제외)」,「자궁부속기 수술(악성종양제외)」,「제왕절개분만(단태아)」,「제왕절개분만(다태아)」의 각 질병군 소정점수를 적용한다.
\item 위 “1”의 규정에도 불구하고 「복강경을 이용한 기타 자궁 수술(악성종양 제외)」,「기타 자궁 수술(악성종양제외)」,「복강경을 이용한 자궁부속기 수술(악성종양제외)」,「자궁부속기 수술(악성종양제외)」의 각 질병군에 해당하는 수술을 실시한 경우 해당 질병군의 가산점수를 산정한다. 다만, 절개생검(심부[장기절개생검]-개복에 의한 것, 나-853-나-2), 유착성자궁부속기절제술(자-433)과 난소를 전적출하는 부속기종양적출술([양측]-양성, 자-442-가)은 가산점수를 산정하지 아니한다.
\item 「제왕절개분만(단태아)」,「제왕절개분만(다태아)」질병군 대상 중 출혈로 인해 혈관색전술(기타혈관, 자-664-나), 자궁내 풍선카테터 충전술[자궁용적측정 포함](자-402-3)을 실시한 경우 질병군 점수를 적용하지 아니하며 제1편을 적용한다.
\item 각 질병군은 동 질병군에 해당하는 수술의 종목수 및 편ㆍ양측 수술에 불문하고 해당 소정점수를 적용한다.
\item 복강경을 이용한 수술 중 부득이한 사유로 중도에 개복술로 전환하여 수술을 종결한 경우에는 복강경을 이용하지 아니한 질병군에 해당하는 소정점수를 적용하고 복강경 등 내시경하 수술시 보상하는 239,000원(100분의 20에 해당 하는 47,800원은 본인부담)의 금액을 추가 산정한다.
\item 제4부 비급여 목록 2. 신의료기술등의 비급여 제9장 처치 및 수술료 등의 (1) 다빈치 로봇 수술을 실시한 경우에는, 제2편제1부제5호에 따라 산정한 복강경을 이용한 자궁 및 자궁부속기 수술 질병군 요양급여비용의 총액에서 별표 2의2의 질병군별 다빈치 로봇 수술시 제외금액표의 금액을 제외하고 산정한다. 다만, 야간ㆍ공휴 및 “2.” 등의 가산은 적용하지 아니한다.
\item 자궁근종, 자궁선근증에 초음파 유도하 고강도초음파집속술(조-566)을 실시한 경우 질병군 점수를 적용하지 아니하며 제1편을 적용한다.
\item 22시- 06시에 제왕절개분만을 행한 경우에는 질병군 야간ㆍ공휴 소정점수를 2회 산정한다. 이 경우 수술 또는 마취를 시작한 시각을 기준으로 산정한다.
\item 분만취약지에서 제왕절개분만을 행한 경우에는 질병군 야간ㆍ공휴 소정점수를 4회 산정하고, 분만취약지는 제1편에서 정하고 있는 「요양급여의 적용기준 및 방법에 관한 세부사항」을 적용한다.
\end{enumerate}

\clearpage
\subsection{질병군 급여의 별도 산정 항목}
\begin{enumerate}[가.]\tightlist
\item 식대 \par
국민건강보험법 시행령 제19조제1항 관련 별표2 요양급여비용 중 본인이 부담할 비용의 부담률 및 부담액 제3호에 따른 입원기간중의 식대
\item 별표 2의1에 열거한 항목에 해당하는 외과전문의 가산
\item 복강경 수술 중 개복하여 수술 종결시 추가 산정 비용 \par
복강경을 이용한 수술 중 부득이한 사유로 개복술로 전환하여 수술을 종결한 경우에는 복강경 등 내시경하 수술시 보상하는 239,000원 추가 산정
\item 초음파검사료\par
질병군 진료 시 초음파검사는 「요양급여의 적용기준 및 방법에 관한 세부사항」의 세부인정기준을 적용하며, 인정기준에 의한 급여대상에 해당되는 경우에는 초음파검사료를 추가 산정
\item 4인실 또는 5인실 이용 시 추가비용 \par
4인실 또는 5인실 이용 시 기본입원료와의 차액을 추가 산정
\item 별표2의4에 열거한 항목에 해당하는 행위 및 치료재료
\item 별표2의5에 열거한 「요양급여비용의 100분의 100미만의 범위에서 본인부담률을 달리 적용하는 항목 및 부담률의 결정 등에 관한 기준」에 따른 행위 및 치료재료
\item 마취통증의학과 전문의 초빙료\par
질병군 진료 시 마취통증의학과 전문의 초빙한 경우 마취통증의학과 전문의 초빙료는 「요양급여의 적용기준 및 방법에 관한 세부사항」의 세부인정기준을 적용하여 추가산정
\item 질병군 분류번호를 결정하는 주된 수술 이외에 수술\par
질병군 분류번호를 결정하는 주된 수술 이외에 수술을 실시한 경우 「요양급여의 적용기준 및 방법에 관한 세부사항」의 세부인정기준을 적용하여 추가산정
\item 의료의 질 평가 지원금 \par
질병군 진료 시 의료의질평가지원금은 가-22의 각 분야별 등급별 ‘입원’의 소정점수를 질병군 입원일수와 동일하게 추가산정
\item 응급의료행위료 \par
제1편제2부제19장제2절에 따른 (별표 2) 및 (별표 3)의 응급의료행위를 실시하는 경우 제1편제2부제19장제2절의 산정지침 3. 내지 5. 및 「요양급여의 적용기준 및 방법에 관한 세부사항」을 적용하여 추가산정
\item 전문병원관리료\par
전문병원에서 지정받은 의료기관에서 질병군 진료시 제1편제2부제1장 산정지침 6. 및「요양급여의 적용기준 및 방법에 관한 세부사항」을 적용하여 추가산정
\item 제왕절개분만 심야가산\par
제2편 제4장 산부인과 적용지침 제9호에 따라 22시-06시에 제왕절개분만을 행한 경우에는 질병군 야간ㆍ공휴 소정점수를 2회 추가산정
\item 분만취약지역 가산\par
제2편제4장 산부인과[적용지침] 10. 에 따라 분만취약지에서 제왕절개 분만을 행한 경우에는 질병군 야간ㆍ공휴 소정점수를 4회 추가산정(단, 분만취약지는 제1편에서 정하고 있는 「요양급여의 적용기준 및 방법에 관한 세부사항」을 적용)
\end{enumerate}

\clearpage
\subsection{보건복지부 고시 제2015 - 26호(2015.01.30)}
\leftrod{질병군 분류번호를 결정하는 주된 수술 이외에 수술을 실시한 경우 수기료 추가 산정방법}
질병군 분류번호를 결정하는 주된 수술 이외에 제 1편제2부제9장제1절(기본처치 제외) 및 제10장제3절, 제4절의 수술을 실시한 경우의 추가 산정 방법은 다음과 같이 한다.\par
\begin{center}\emph{- 다 음 -}\end{center}
\begin{enumerate}[1.]\tightlist
\item 질병군 진료 중 질병군 분류번호를 결정하는 주된 수술과 날을 달리하여 실시하는 수술도 포함함
\item 해당 수술 항목의 소정점수만을 산정하고, 야간ㆍ공휴 가산 등을 포함한 모든 가산은 적용하지 아니함
\item 아래의 경우는 추가 산정하지 아니함
	\begin{enumerate}[가.]\tightlist
	\item 합병증 혹은 처치 중의 우발적 천자 및 열상등으로 실시한 수술
	\item 수정체수술 질병군과 동시에 실시한 유리체 흡인술(자505), 유리체내주입술(자507), 유리체절제술-부분절제(자512-나)
	\item 편도절제술과 동시에 실시한 아데노이드절제술(내시경하에서 실시한 경우 포함)
	\item 기타 또는 주요 항문수술 질병군에 해당하는 수술을 2개 이상 실시한 경우
	\item 위 1부터 3까지에서 정하고 있지 않은 내용은 「건강보험 행위 급여 비급여 목록표 및 급여 상대가치점수」제1편 제2부 제9장, 제10장 및 「요양급여의 적용기준 및 방법에 관한 세부사항」Ⅰ. 행위 제9장, 제10장을 적용한다.
	\end{enumerate}
\end{enumerate}

\clearpage		
\subsection{보건복지부 고시 제2014 - 240호(2014.12.30)}
\leftrod{「기타 자궁 수술」및 「자궁부속기 수술」질병군의 가산점수 인정기준}
「복강경을 이용한 기타 자궁 수술(악성종양제외)」, 「기타 자궁 수술(악성종양제외)」,「복강경을 이용한 자궁부속기 수술(악성종양제외)」,「자궁부속기 수술(악성종양제외)」질병군의 가산점수는 진료담당의사의 의학적 판단 하에 임신ㆍ출산능력을 보존하는 수술을 시행한 경우 산정함을 원칙으로 하며 인정기준은 다음과 같이함\par
\begin{center}\emph{- 다 음 -}\end{center}
\begin{enumerate}[가.]\tightlist
\item 임신ㆍ출산을 담당하는 장기의 병변 부위만을 제거ㆍ교정하는 수술을 하여 임신ㆍ출산능력을 보존한 경우\par
다만, 자궁내막증이 있거나 불임(또는 난임) 등으로 임신가능성을 높이기 위해 난소 또는 난관 전절제술을 실시한 경우는 사례별로 인정
\item 임신ㆍ출산을 담당하는 장기의 수술을 동시에 실시하여 그 수술결과로 임신ㆍ출산능력이 보존된 경우
\item 아래의 경우는 가산점수를 산정하지 아니함
	\begin{enumerate}[(1)]\tightlist
	\item 폐경 또는 55세 이상 여성(55세 이상이나 폐경이 아닌 경우 관련자료 첨부시 이를 참조하여 인정)
	\item 기존에 시행한 수술로 임신ㆍ출산 능력을 상실한 경우
	\end{enumerate}
\end{enumerate}

\clearpage
\begin{myshadowbox}
\begin{enumerate}[2.]\tightlist
\item \textcolor{red}{가입자 또는 피부양자(이하 “가입자 등”이라한다)가 질병군으로 입원진료를 받은 경우에 적용}하되, 다음의 각 항목은 질병군 적용에서 제외하고 제 1편을 적용한다.
	\begin{enumerate}[가.]\tightlist
	\item 혈우병환자, HIV감염자
	\item 입원일수가 30일을 초과할 경우 31일째부터 발생하는 진료분
	\item 차상위 본인부담경감대상자로서 제3호 나목에 해당하는 경우
	\item \textcolor{blue}{질병군 진료 이외의 목적으로 입원하여 입원일수가 6일을 초과한 시점에 예상치 못하게 질병군 수술이 이루어진 경우 입원일로부터 수술시행일 전일까지의 진료분}
	\end{enumerate}
\end{enumerate}
\prezi{\clearpage}
\end{myshadowbox}
\Que{입원기간 중에 자격이 변동된 경우 또는 타법령(산재, 자보)으로 입원 중 질병군 대상 수술을 했을 경우 청구방법은?}
\Ans{타법령으로 입원 중 질병군 진료가 발생한 경우 전체 진료내역을 \textcolor{blue}{행위별수가제로 청구}함\par

 ☞ 급여 65720-1898호(2001.12.29) 「타법령으로 입원진료 중 질병군 진료가 발생한 경우 요양급여비용 산정방법」}
\par
\medskip
\prezi{\clearpage}
\Que{임신 유지목적으로 입원하여 입원일수가 \textcolor{red}{6일을 초과한 시점}에서 예상치 못하게 제왕절개분만이 이루어진 경우(초음파 산정방법)}
\Ans
{\begin{itemize}\tightlist
\item 입원(행위별 청구) :(정상임신부) 7회까지 급여, 그 외 비급여 (태아의 이상이나 이상이 예측되는 경우) 급여
\item 제왕절개 시점 : <분리청구 시점>% 구분 :  
\item 제왕절개분만 입원(DRG 청구) : 분만기간 초음파(비급여)
\end{itemize}}
\par
\medskip
\prezi{\clearpage}
\Que{임신 유지목적으로 입원하여 입원일수가 \textcolor{red}{6일을 이내에} 제왕절개분만이 이루어진 경우(초음파 산정방법)}
\Ans
{제왕절개분만 입원(DRG 청구) : 분만기간 초음파(비급여)}
\prezi{\clearpage}
%\subsection{자궁근종절제술-질부접근(R4123) 의 포괄수가제 포함여부}

\begin{myshadowbox}
\begin{enumerate}[3.]\tightlist
\item 제2호 규정에 따른 질병군 입원진료에는 다음의 각 항목을 포함한다.
	\begin{enumerate}[가.]\tightlist
	\item 제2부 각 장에 분류된 질병군으로 \textcolor{red}{응급실ㆍ수술실 등에서 수술을 받고 연속하여 6시간 이상 관찰} 후 귀가 또는 이송한 경우
	\item 제2부 각 장에 분류된 질병군 중 수정체 소절개 수술 단안, 수정체 소절개 수술 양안, 수정체 대절개 수술 단안, 수정체 대절개 수술 양안, 기타항문 수술, 서혜 및 대퇴부 탈장수술(장관절제 미동반) 단측, 서혜 및 대퇴부 탈장수술(장관절제 미동반) 양측, 복강경을 이용한 서혜 및 대퇴부 탈장수술(장관절제 미동반) 단측, 복강경을 이용한 서혜 및 대퇴부 탈장수술(장관절제 미동반) 양측 질병군으로 수술을 받고 6시간 미만 관찰 후 당일 귀가 또는 이송하는 경우
	\end{enumerate}
\end{enumerate}
\end{myshadowbox}
\prezi{\clearpage}		
\Que{자궁경부의 근종인경우 \highlight{외래에서 입원없이} 시행하는 자궁근종절제술-질부접근(R4123) 이 포괄수가제 포함 대상인가요?\par
포괄수가제 포함 대상인경우 행위별청구외에 추가로 다른 청구방법이 있나요?}

\begin{commentbox}{자궁근종절제술-질부접근(R4123) 의 포괄수가제 포함여부}
질병군(DRG) 포괄수가는 국민건강보험법시행령 제21조제3항제2호에 따라 복지부장관이 별도 고시하는 7개 질병군으로 입원진료를 받은 경우에 적용하며, 질병군 입원진료는 질병군 급여 일반원칙에 따라 다음의 항목을 포함하고 있습니다.\par
\begin{center}\emph{- 다 음 -}\end{center}
\begin{itemize}\tightlist
\item 7개 질병군으로 응급실ㆍ수술실 등에서 수술을 받고 연속하여 6시간 이상 관찰 후 귀가 또는 이송한 경우 
\item 7개 질병군 중 수정체수술(대절개 단안 및 양안, 소절개 단안 및 양안), 기타항문수술, 서혜 및 대퇴부탈장수술 단측 및 양측(복강경 이용 포함)의 수술을 받고 6시간 이상 관찰 후 당일 귀가 또는 이송한 경우
\end{itemize}
따라서 자궁근종으로 자궁근종절제술-질부접근(R4123)을 받고 \textcolor{red}{6시간 미만 관찰 후 당일 귀가 또는 이송한 경우(외래)}는 \highlight{행위별청구대상}임을 알려드립니다.
\end{commentbox}
\prezi{\clearpage}
\par
\medskip
\Que{제왕절개분만 후 산후출혈 등의 합병증으로 당일 이송한 경우(6시간 미만 진료) 질병군 적용여부}
\Ans{제왕절개분만 후 당일 귀가 또는 이송한 경우 \textcolor{blue}{6시간미만 진료도 질병군 적용 대상}임}
\prezi{\clearpage}
\par
\medskip
\Que{쌍둥이 임산부가 자연분만과 제왕절개분만으로 분만방법을 달리하여 출산한 경우 청구방법은?}
\Ans{전체 진료내역을 \textcolor{red}{행위별수가제로 청구함.} 자연분만과 제왕절개분만은 본인부담률이 다르므로 명세서를 분리하여 청구하고 제왕절개분만명세서에는 상해외인에 “D"를 기재하여 청구함}
\prezi{\clearpage}
\par
\medskip
\Que{제왕절개분만의 질병군요양급여비용 청구시 신생아진료비용 청구방법은?}
\Ans{제왕절개분만의 질병군 요양급여비용 청구와 별도로 신생아 진료비용은 \textcolor{red}{행위별로 청구}함(2010년 7월부터 적용)}
\clearpage

\subsection{질병군 급여 일반원칙(비급여)}
\begin{myshadowbox}
\begin{enumerate}[4.]\tightlist
\item 제2부 각 장에 \textcolor{red}{분류된 질병군 상대가치점수(이하 “점수”라 한다)}는 다음 각목의 행위ㆍ약제 및 치료재료를 포함한다.
	\begin{enumerate}[가.]\tightlist
	\item 제1편 행위 급여ㆍ비급여 목록 및 급여 상대가치점수에서 정한 행위급여목록표에 고시된 행위
	\item 요양급여기준 제8조제2항의 규정에 의하여 고시된 약제 급여 목록 및 급여 상한금액표의 약제와 치료재료 급여ㆍ비급여 목록 및 급여 상한 금액표의 치료재료
	\item \textcolor{red}{요양급여기준 별표 2의 비급여대상 중 제6호의 비급여대상}을 제외한 행위ㆍ약제 및 치료재료
	\item 국민건강보험법 시행규칙 별표 6의 본인이 요양급여비용의 100분의 100을 부담하는 항목 중 제1호 자목에 해당하는 항목을 제외한 행위ㆍ약제 및 치료재료
	\item 다음 항목 중 위 가목 내지 라목에 해당하는 경우
		\begin{enumerate}[(1)]\tightlist
		\item 요양급여기준 별표 1 제1호 마목에서 장관이 정하는 바에 따라 다른 기관에 검사를 위탁하거나 당해 요양기관에 소속되지 아니한 전문성이 뛰어난 의료인을 초빙하거나, 또는 다른 요양기관에서 보유하고 있는 양질의 시설ㆍ인력 및 장비를 공동 사용하는 경우 소요되는 행위ㆍ약제 및 치료재료
		\item 입ㆍ퇴원 당일에 발생한 행위ㆍ약제 및 치료재료로써 외래진료 및 퇴원약제 등을 포함하되 다음 항목은 제외한다.
			\begin{enumerate}[(가)]\tightlist
			\item 질병군 입원을 예견하지 못한 상태에서 \textcolor{red}{입원 당일 외래진료를 받은 경우의 원외처방 약제비}
			\item 질병군으로 퇴원 후 \textcolor{red}{질병군과 관계없는 상병으로 퇴원 당일 외래진료를 받은 경우의 원외처방 약제비}
			\item 질병군으로 \textcolor{red}{퇴원 후 질병군 질환과 관계없는 상병으로 퇴원 당일 재입원하는 경우의 요양급여비용}
			\end{enumerate}
		\item 요양기관의 요구에 의하여 가입자 등이 외부에서 직접 구입한 약제 및 치료재료
		\end{enumerate}
	\end{enumerate}	
\end{enumerate}
\end{myshadowbox}
\prezi{\clearpage}
%\begin{commentbox}{질병군 급여 일반원칙(비급여)}
%\end{commentbox}
\Que{질병군 진료기간 중에 환자가 원하여 불임관련 진료를 시행한 경우 별도 산정 여부}
\Ans{질병군 진료기간 중 일차성 불임과 이차성 불임에 해당되어 \textcolor{red}{불임관련 진료를 시행한 경우는 질병군 급여상대가치점수에 포함} 됨. 다만, 일차성 불임과 이차성 불임에 해당되지 않고 환자가 원하여 실시한 불임 관련 진료는 비급여대상임\par
☞ 고시 제2004-36호 (‘04.6.24, ‘04.7.1. 시행)「불임관련 진료의 요양급여여부」}
\prezi{\clearpage}
\par
\medskip
\Que{자궁내장치(IUD)를 교체하고 재삽입하는 경우 질병군 적용방법은?}
\Ans{본인이 원하여 자궁내장치삽입술을 시술받고 동 장치를 교체하기 위하여 기유치된 자궁내장치를 제거하고 재삽입하는 경우의 관련 진료비용은 \textcolor{red}{비급여대상}임. 다만, 피임시술 요양급여 대상자가 기존에 유치된 자궁내 장치를 제거하고 새기구를 삽입하는 경우는 질병군 급여상대가치점수에 포함됨\par
☞ 고시 제2011-50호(‘11.4.29.) 「자궁내장치(IUD) 교체시 제거료 산정방법」}
\prezi{\clearpage}
\par
\medskip
\Que{통증자가조절법(PCA)시 비급여 치료재료나 약제 사용시 적용방법은?}
\Ans{ ‘각종 수술 후 통증관리를 위한 통증자가조절법(PCA)’은 행위별수가제와 더불어 질병군 포괄수가제에서도 요양급여비용의 전액(100분의 100)을  \textcolor{red}{본인이 부담하는 항목으로 산정기준은 행위별과 동일함}. 다만, 치료재료 및 약제가 비급여인 경우 식약처장의 허가사항(효능ㆍ효과 및 용법ㆍ용량 등) 범위 안에서 사용하되 가격은 요양기관이 실제 구입한 금액으로 산정 함.\par
☞ 고시(2005-101호 2005.12.30)「통증자가조절법(PCA)의 급여여부」}
\prezi{\clearpage}
\par
\medskip
\Que{질병군으로 입원진료 중 환자가 원하여 시행하는 요실금수술 별도 산정 여부}
\Ans{환자가 요실금수술을 원하는 경우라도 수술의 요양급여대상 또는 비급여대상여부는 객관적인 검사를 통하여 결정되어야 함. 또한 요실금수술이 \textcolor{red}{비급여 대상으로 결정이 되면 해당 수술 비용과 사용된 치료재료 등은 별도 산정함(비급여)}\par
☞ 고시 제2011-144호(‘11.11.25.) 「인조테이프를 이용한 요실금수술 인정기준」}
\prezi{\clearpage}
\par
\medskip
\Que{질병군 대상 수술 후 기력저하 등의 이유로 환자가 원하여 투여하는 영양제의 별도 산정 여부}
\Ans{질병군 포괄수가에는 급여와 비급여가 포함(보건복지부 장관이 고시한 질병군 비급여목록 제외)되어 있으며, \textcolor{red}{질병군 진료기간에 투여한 영양제 또한 질병군 수가에 포함되어 있으므로 별도 산정할 수 없음}\par
\emph{다만, 일상생활에 지장이 없는 단순피로 및 권태에 투여한 경우에는 별도 산정가능합니다}}
\prezi{\clearpage}
\par
\medskip
\Que{제왕절개 분만시 유착방지제 별도 산정 여부}
\Ans{유착방지제는 보건복지부장관이 정하여 고시하는 \textcolor{red}{비급여목록에 해당하지 않으므로 환자에게 별도로 부담시킬 수 없음.}\par 
☞ 요양기관에서 제출한 자료(비급여)에 근거하여 포괄수가에 발생빈도 만큼 포함하여 산출하였음}
\prezi{\clearpage}
\par
\medskip
\Que{칼슘제제 등 건강보조식품 별도 산정 여부}
\Ans{「건강보험요양급여의 기준에 관한 규칙」 별표1 요양급여의 적용기준 및 방법 3.에 따라  '약제는 약사법령에 의하여 허가 또는 신고된 사항(효능ㆍ효과)의 범위 안에서 처방ㆍ투여하여야 한다’고 규정되어 있음. 따라서, 의약품이 아닌 \textcolor{red}{건강보조식품은 건강보험 급여 여부의 논의 대상이 아님}}
\prezi{\clearpage}
\par
\medskip
\Que{질병군 진료기간 중 수면내시경검사를 실시한 경우 별도 산정  여부}
\Ans{수면내시경검사는 보건복지부장관이 별도로 정하여 고시하는 비급여항목에 해당하지 않으므로 환자에게 \textcolor{red}{별도로 부담시킬 수 없음}\par
☞ 요양기관에서 제출한 자료(비급여)에 근거하여 포괄수가에 발생빈도 만큼 포함하여 산출하였음}
\prezi{\clearpage}
\par
\medskip
\Que{질병군(DRG) 입원진료기간 중 MRI 촬영을 실시한 경우 별도 산정 여부}
\Ans{복지부 고시 제2013-180호(‘13.11.27) “MRI 세부산정기준” 따라 질환별 급여대상 및 산정기준에 해당하는 경우 질병군 상대가치점수에 포함되어 별도 산정할 수 없으나, 질환별 급여대상 및 산정기준에 해당하지 않는 경우에는 비급여 대상임}
\prezi{\clearpage}
\par
\medskip
\Que{「자기공명영상유도하 고강도 초음파 집속술(자궁근종)」시 실시 하는 자기공명영상유도비용 별도 산정 여부}
\Ans{「자기공명영상유도하 고강도 초음파 집속술(자궁근종)」을 실시한 경우 ‘복강경을 이용한 기타자궁수술’ \textcolor{blue}{질병군을 적용}하며, \textcolor{red}{자기공명영상유도비용은 MRI 세부산정기준(복지부고시 제2013-180호, ‘13.11.27.) 및 질병군 전문평가위원회 결정사항에 따라 비급여}로 산정할 수 있음}
\prezi{\clearpage}
\par
\medskip
\Que{혈전방지용 압박스타킹, 창상봉합용 액상접착제, 불투명ㆍ투명 드레싱재료 별도 산정 여부}
\Ans{혈전방지용 압박스타킹, 창상봉합용 액상접착제, 불투명ㆍ투명 드레싱재료는 요양기관에서 제출한 자료를 근거로 발생빈도 만큼 포괄수가에 포함되어 있으며 장관이 고시하는 \textcolor{red}{질병군 비급여 목록이 아니므로 별도로 산정 할 수 없음}}
\prezi{\clearpage}
\begin{commentbox}{신의료기술}
요양급여대상 또는 비급여대상으로 결정되지 않은 새로운 행위 및 치료재료로써 \emph{신의료기술등요양급여 결정 신청한 신의료기술인 경우 질병군에서의 적용방법은?}\par
요양급여대상 또는 비급여대상으로 결정ㆍ고시되기 전까지의 신의료기술 등 결정신청 항목은 \textcolor{red}{질병군 포괄수가제에서는 급여대상임} 따라서, 산정방법은 행위의 내용ㆍ성격이 가장 유사한 수술항목을 적용한 후 분류되는 질병군 점수를 청구함.\par
☞ 국민건강보험 요양급여의 기준에 관한 규칙(별표2) 6호
\end{commentbox}

\clearpage
\begin{myshadowbox}
\begin{enumerate}[15.]\tightlist
\item 질병군 진료 시 초음파검사는 「요양급여의 적용기준 및 방법에 관한 세부사항」제2장 검사료 초음파검사 세부인정기준을 적용하며, \highlight{인정기준에 의한 급여대상에 해당되는 경우에는} 제2부 각 장에 분류된 질병군 점수이외에 제1편 제2부 초음파검사료를 추가 산정한다. 
\end{enumerate}
\end{myshadowbox}
\prezi{\clearpage}
\begin{itemize}\tightlist
	\item EZ986 분만기간 초음파 : 분만을 위한 입원기간 동안 발생한 초음파 검사를 모두 의미함. 제왕절개를 위해 입원한 환자들의 경우는 옆의 비급여초음파를 최소한 2번 이상 실시하고 청구한다.
	\item EZ887 초음파를 이용한 태아 생물리학 계수( Biophysical Profile )
	\item 임신 유지목적으로 입원하여 6일 이내에 제왕절개분만이 이루어진 경우 : 분만기간 초음파(비급여)로 청구한다. 조산통으로 입원한 경우엔 2일에 한번씩은 초음파를 본다.
	\item 임신 유지목적으로 입원하여 입원일수가 6일을 초과한 시점에서 예상치 못하게 제왕절개분만이 이루어진 경우 
		\begin{itemize}\tightlist
		\item 입원(행위별 청구) : (정상임신부) 7회까지 급여, 그 외 비급여(태아의 이상이나 이상이 예측되는 경우) 급여
		\item 분리청구 시점 구분
		\item 제왕절개분만 입원(DRG 청구) : 분만기간 초음파(비급여)
         \end{itemize}                               
	\item 분만과 연결된 입원: 분만기간이 장기로 길어진 경우 분리청구 시점 기준으로 적용
		\begin{itemize}\tightlist
		\item 입원(행위별 청구) : (정상임신부) 7회까지 급여, 그 외 비급여(태아의 이상이나 이상이 예측되는 경우) 급여
		\item 분리청구 시점 구분
		\item 자연분만및 제왕절개분만 입원제왕절개분만 입원(DRG 청구) : 분만기간 초음파(비급여)
         \end{itemize} 	
%	\item 질병군 진료 이외의 목적으로 입원하여 입원일수가 6일을 초과한 시점에 예상치 못하게 질병군 수술이 이루어진 경우 입원일로부터 수술시행일 전일까지의 진료분을 제외한 경우의 보험 초음파등(6일전의 조기진통등으로 입원하여 제왕절개분만한 경우 횟수초과 급여초음파 청구 (해당 삼분기의 일반 또는 일반의 제한초음파 산정). 단, 1일 1회만 청구 가능함)
\end{itemize}
\prezi{\clearpage}
\par
\medskip
\Que{○○종합병원에 충수암 의증으로 입원한 환자가 2014년 2월 1일 초음파검사 등을 실시 후 충수암 진단으로 충수절제술을 시행한 경우 급여대상인 초음파검사의 특정내역 기재방법은?}
\Ans{
\begin{itemize}\tightlist
\item 초음파검사 비용의 추가 산정은 특정내역 MT007의 내역구분 ‘SON'으로 기재
\item 특정내역 기재형식 및 설명
	\begin{itemize}\tightlist
	\item X(3)/ccyymmdd/X/X(9)/9(10)/9(5).V9(2)/9(3)/9(10)/X(200)/X(1)/X(100)
	\item 내역구분/투여(실시)일자/코드구분/코드/단가/1일투여량(실시횟수)/총투여일수(실시횟수)/금액/준용명/면허종류/면허번호
	\end{itemize}
\item ※ 초음파검사가 급여대상이나 산정횟수를 초과하는 경우에는 특정내역 MT007 내역구분 ‘All'(보훈환자의 경우 ’100‘)에 기재
\end{itemize}}

\clearpage
\subsection{질병군 급여 일반원칙(주된수술 이외의 수술시)}
\begin{myshadowbox}
\begin{enumerate}[19.]\tightlist
\item 질병군 진료시 질병군 분류번호를 결정하는 \textcolor{red}{주된 수술 이외에 제1편(행위별 수가)제2부제9장제1절(기본처치 제외) 또는 제10장제3절 제4절의 수술을 실시한 경우에는 해당 수술 소정점수를 추가 산정한다. 다만, 주된 수술과 동일 피부 절개 하에 실시되는 수술은 해당 수술 소정점수의 70\%를 산정}한다.
\end{enumerate}
\end{myshadowbox}
\prezi{\clearpage}
\Que{질병군으로 입원진료 중 환자가 원하여 시행하는 요실금수술은 별도 산정 가능한가요?}
\Ans{질병군 수가에는 급여와 보건복지부장관이 정하여 고시하는 비급여가  있으며, 환자가 요실금수술을 원하는 경우라도 수술이 요양급여대상 인지 비급여대상인지를 객관적인 검사를 통하여 결정되어야 하며, 비급여 으로 결정이 되면 해당 수술비용과 비급여 수술에 사용 된 치료재료 등은 별도로 받을 수 있습니다.\par
※ 요실금 수술 보건복지부 고시 제2011-144호(11.11.25)\par
만약 \textcolor{red}{요양급여대상이면 해당수술 소정점수만을 보존받을수 있습니다. 결과적으론 재료에 대해선 보존받을수 없습니다.}}
\prezi{\clearpage}
\par
\medskip
\Que{제왕절개분만 질병군에서 ‘분만 전 처치(자-437)’, ‘분만 후 처치(자437-1)’, ‘제왕절개술 전 질식분만시도(자451-1)’ 등의 처치를 실시한 경우 질병군 소정점수 이외에 추가산정 여부}
\Ans{질병군 진료기간 중 질병군 분류번호를 결정하는 주된 수술 이외에 다른 수술을 실시한 경우 질병군 급여상대가치점수 이외에 다른 수술료의 추가산정이 가능함.\par
‘분만 전 처치(자-437)’, ‘분만 후 처치(자437-1)’, ‘제왕절개술 전 질식분만시도(자451-1)’ 등은 분만을 목적으로 시행되는 처치 등으로 서로 \textcolor{red}{다른 수술에 해당하지 않으므로 관련 수기료는 추가 산정할 수 없음}}
\prezi{\clearpage}
\begin{commentbox}{추가로 산정할 수 있는 수술의 범위}
질병군 진료시 질병군 분류번호를 결정하는 주된 수술이외에 \emph{추가로 산정할 수 있는 수술의 범위}는?\par
질병군 분류번호를 결정하는 주된 수술이외에 제1편제2부제9장제1절(기본처치 제외) 및 제10장제3절, 제4절의 수술을 실시한 경우 해당 수술의 수기료를 추가 산정함. 또한, 질병군 주진단과 다른 상병으로 실시되는 수술에 해당하는 경우 산정하며 비위관삽관술 등 질병군 주된 수술 및 그 외 실시되는 수술의 과정에 발생되는 처치 등은 해당되지 않음. 또한 각종 처치(위세척, 질강처치, 직장맛사지, 피부과처치, 자궁내장치삽입술 등), 체외충격파쇄석술, 혈액투석, 응급처치 등은 산정이 불가함
\end{commentbox}
\prezi{\clearpage}
\leftrod{질병군 진료시 주된 수술 이외에 수술을 추가로 실시한 경우 질병군 요양급여비용 청구방법}
질병군 주된 수술 이외에 실시한 수술은 특정내역 MT007(DRG세부내역)의 내역구분 ‘COP'에 기재하여 청구하고 질병군 수술 이외에 실시한 수술료에 종별가산한 금액을 양급여비용총액1에 합하여 기재함\par
\emph{(작성요령)}\par
기타 자궁적출술(N04200) 질병군 진료시 자궁적출술(R4145)과 난소부분절제술(R4430)을 시행한 경우 

\prezi{\clearpage}\par
\medskip
\Que{부인과적 개복수술 또는 기타 개복수술시 병변 없이 실시한 충수절제술 인정여부}
\Ans{개복 수술시 병변이 없는 상태에서 시행한 충수절제술(Incidental Appendectomy)은 별도 산정 할 수 없음\par
☞ 고시 제2014-126호(‘14.7.30.) 「부인과적 개복수술 또는 기타 개복수술시 병변없이 실시한 충수절제술 인정여부」}
\prezi{\clearpage}
\clearpage
\begin{myshadowbox}
\begin{enumerate}[4.]\tightlist
\item 복강경을 이용한 수술 중 부득이한 사유로 중도에 개복술로 전환하여 수술을 종결한 경우에는 복강경을 이용하지 아니한 질병군에 해당하는 소정점수를 적용하고 복강경 등 내시경하 수술시 보상하는 239,000원(100분의 20에 해당하는 47,800원은 본인부담)의 금액을 추가 산정한다.
\end{enumerate}
\end{myshadowbox}
\prezi{\clearpage}
\Que{복강경을 이용한 수술 중 부득이한 사유로 \textcolor{red}{중도}에 개복술로 전환하여 수술을 종결한 경우 “복강경 등 내시경하 수술시 보상하는 239,000원의 금액을 추가 산정함. 이 경우 본인일부부담금 산정특례 대상자의 본인부담률 적용 방법은?}
\Ans{본인일부부담금 산정특례 대상자와 차상위 본인부담 경감대상자 등은 각각의 해당 본인부담률을 적용하며, 본인부담률이 5\%인 경우에는 239,000의 5\%에 해당하는 금액인 11,950원을 본인이 부담함\par
☞ 「국민건강보험법시행령」 별표2 제3호에 해당하는 대상자인 경우에는 그 각목에서 정한 본인부담률을 적용}
\prezi{\clearpage}
\par
\medskip
\Que{복강경을 이용한 LAVH시도 할려고 봤더니 골반장기의 유착이 심해서 개복하여 TAH한 경우에 추가 산정 가능여부.}
\Ans{단지 봤다는 소견으로만은 추가산정이 불가하다고 합니다.  \textcolor{red}{“도중”} 이란 말의 뜻대로 뭔가를 하다가 안되었다는 차트의 기록이 필요하다고 합니다. \par 
"질병군 요양급여비용 모니터링"이란 방법을 통해서 위의 과정을 통제한다고 합니다.}

\clearpage
\subsection{질병군 급여 일반원칙(마취 초빙료관련)}
\begin{myshadowbox}
\begin{enumerate}[18.]\tightlist
\item 질병군 진료시 마취통증의학과 전문의를 초빙하여 마취를 실시한 경우에는 제1편(행위별 수가)제2부제6장 바-2 마취통증의학과 전문의 초빙료를 추가 산정하며, 제1편(행위별 수가)제2부제6장 및 「요양급여의 적용기준 및 방법에 관한 세부사항」의 마취통증의학과 전문의 초빙료 산정 관련 규정을 적용한다.
\end{enumerate}
\end{myshadowbox}

마취통증의학과 전문의를 초빙하여 마취를 실시한 경우 특정내역 MT007(DRG 세부내역)의 내역구분 ‘ANE' 에 마취통증의학과 전문의 초빙료(바 2, L7990)를 질병군 요양급여비용 이외에 추가 산정함\par
이 경우 마취통증의학과 전문의 초빙료 및 면허종류, 면허번호를 기재하여야 함 \par
(작성요령) ‘15년 의원 단가 기준
\prezi{\clearpage}
\par
\medskip
\Que{마취통증의학과 전문의가 상근하는 요양기관에서 마취통증의학과 전문의 초빙료를 산정할 수 있는지?}
\Ans{마취통증의학과 전문의가 상근하는 요양기관은 마취통증의학과 전문의 초빙료를 산정할 수 없음 \par
다만, 고시 제2012-153호(‘12.11.27.)에 따라 일부 예외적인 상황에 한하여 마취통증의학과 전문의 초빙료를 산정할 수 있으며 관련 법령에 의거 인력 등에 대한 변경신고(유선신고 포함)가 이루어져야함\par
☞「마취통증의학과 전문의가 상근하는 요양기관에서 마취통증의학과 전문의 초빙시 인정여부」고시 제2012-153호, 12.11.27.) \par
마취통증의학과 전문의가 상근하는 요양기관에서 마취통증의학과 전문의 초빙료를 산정할 수 있는 경우는 다음과 같으며, 요양급여비용 청구 시 부득이한 사유 또는 신고사실을 확인할 수 있도록 마취기록부, 변경신고서 등 객관적인 증빙자료를 첨부하여야 함\par
\begin{center}\emph{- 다 음 -}\end{center}
\begin{enumerate}[가.]\tightlist
\item 상근하는 마취통증의학과 전문의가 예비군 훈련 등 부득이한 사유로 부재중인 경우 수술이 가능한 다른 요양기관으로 환자를 이송 조치함이 원칙이나 이송할 수 없는 상황에서 마취통증의학과 전문의를 초빙하는 경우. 다만, 이 경우 관련 법령에 의거 인력 등에 대한 변경신고(유선신고 포함)가 이루어져야함
\item  천재지변, 기타 예기치 못한 구급사태 등으로 인하여 동일 시간대에 2인 이상의 수술이 동시에 이루어져야 할 부득이한 사유로 마취통증의학과 전문의를 초빙하는 경우
\item  마취통증의학과 전문의가 상근하는 산부인과 병ㆍ의원에서 야간 또는 공휴일에 임신 또는 분만관련 응급수술을 시행하게 되어 부득이하게 마취통증의학과 전문의를 초빙하는 경우
\end{enumerate}
}
\prezi{\clearpage}
\par
\medskip
\Que{7개 질병군 수술 후 또는 통증자가조절법(PCA) 실시로 인한 구역ㆍ구토 치료를 위해 항구토제를 사용한 경우 별도 산정  여부}
\Ans{수술 후 발생한 구역 및 구토의 치료를 위해사용한 항구토제는 요양급여 대상이며 포괄수가 급여비용에 포함되어 \textcolor{red}{별도로 산정할 수 없음.}  또한, 7개질병군 수술 후 통증관리를 위한 통증자가조절법(PCA)은 전액본인부담(100분의100) 항목이며, 이로 인한 구역 및 구토 방지ㆍ치료를 위해 항구토제를 사용한 경우도 별도 산정할 수 없음}
 \prezi{\clearpage}
\par
\medskip
\Que{질식분만 전 통증조절 목적으로 “무통분만 경막외 마취”를 실시하였으나, 질식분만에 실패하여 제왕절개분만을 실시한 경우 “무통분만 경막외 마취” 비용을 환자에게 별도 부담시킬 수 있는지 여부}
\Ans{질식분만 전 통증조절 목적으로 실시한 “무통분만 경막외 마취”는 질식분만에 실패하여 제왕절개분만을 실시한 경우에도 질병군 급여상대가치점수에 포함되어 \textcolor{red}{별도로 비용을 부담시킬 수 없음}\par
다만, 무통분만 경막외 마취를 실시하여 무통분만을 시도하다가 실패하여 제왕절개분만 후, 이미 가지고 있는 경막외 마취 카테터를 통해  통증자가조절법(PCA)을 시행한 경우에는 수기료(바22나(3)(나)LA227) 및 재충전 약제비 등은 100/100본인부담이 가능함}


\clearpage
\subsection{질병군 급여 일반원칙(가산)}
\begin{myshadowbox}
\begin{enumerate}[12.]\tightlist
\item 18시~09시 또는 공휴일에 응급진료가 불가피하여 수술을 행한 경우에는 해당 질병군의 야간 공휴 소정점수를 추가 산정한다. 이 경우 수술 또는 마취를 시작한 시간을 기준으로 한다.
\end{enumerate}
\end{myshadowbox}
\prezi{\clearpage}
\Que{통증자가조절법(PCA)의 야간ㆍ공휴 시술 가산 적용여부}
\Ans{100/100 본인부담항목인 통증자가조절법(PCA)은 환자에게 다른 통증관리방법(근육주사, 경구투여 등)과 PCA에 대해서 충분한 설명을 하고 환자가 동의한 경우에 실시하고 있어 응급진료에 인정하는 공휴 또는 야간가산을 적용하지 않음\par
☞ 급여65720-1787호(‘02.12.18) 「PCA관련 야간 공휴가산 인정여부에 대한 질의회신」}
\prezi{\clearpage}
\par
\medskip
\Que{마취와 수술시간이 모두 야간(또는 공휴)이어야 야간가산을 산정할 수 있는지 여부 }
\Ans{응급진료가 불가피하여 수술을 시행한 경우에는 마취나 수술 중 하나만 야간(또는 공휴)에 해당되어도 산정 가능함}
\prezi{\clearpage}
\par
\medskip
\Que{질병군 수술 후 수술부위의 출혈로 18시 이후에 응급으로 bleeding control을 시행한 경우 해당 질병군의 야간·공휴 소정점수를 추가 산정할 수 있는지 여부}
\Ans{질병군 수술에 따른 합병증으로 출혈이 발생하여 야간에 응급으로 bleeding control을 시행한 경우는 야간ㆍ공휴 소정점수 추가산정에 해당되지 않음
☞ “18~09시 또는 공휴일에 응급진료가 불가피하여 질병군 대상 수술을 행한 경우” 해당 질병군의 야간·공휴 소정점수를 추가 산정}
\prezi{\clearpage}
\begin{myshadowbox}
\begin{enumerate}[2.]\tightlist
\item 위 “1”의 규정에도 불구하고「복강경을 이용한 기타 자궁 수술(악성종양제외)」,「기타 자궁 수술(악성종양제외)」,「복강경을 이용한 자궁부속기 수술(악성종양제외)」,「자궁부속기 수술(악성종양제외)」의 각 질병군에 해당하는 수술을 실시한 경우 해당 질병군의 가산점수를 산정한다. 다만, 절개생검(심부[장기절개생검]-개복에 의한 것, 나-853-나-2), 유착성자궁부속기절제술(자-433)과 난소를 전적출하는 부속기종양적출술([양측]-양성, 자-442-가)은 가산점수를 산정하지 아니한다.
\end{enumerate}
\end{myshadowbox}
\prezi{\clearpage}
\Que{일측 난소의 전절제술을 시행한 기왕력이 있던 환자가 금번 질병군 진료기간 중에 자궁근종수술과 편측난소전절제술을 실시한 경우 가산점수 산정 가능여부}
\Ans{임신·출산을 담당하는 장기의 수술을 동시에 실시하여 그 수술결과로 임신·출산능력이 보존된 경우 「기타 자궁 수술」 및 「자궁부속기수술」 질병군의 가산점수를 산정할 수 있으므로 동 사례는 금번 수술로써 양측 난소를 절제한 경우에 해당되어 가산점수를 산정할 수 없음}
\prezi{\clearpage}
\par
\medskip
\Que{「복강경을 이용한 기타 자궁 수술(악성종양제외)」,「기타 자궁 수술(악성종양제외)」,「복강경을 이용한 자궁부속기 수술(악성종양제외)」,「자궁부속기 수술(악성종양제외)」의 각 질병군에 해당하는 수술을 실시한 경우 가산수가는 어떻게 계산하나요?}
\Ans{산부인과 질병군의 가산점수는 질병군별 소정점수와 해당 질병군의 고정비율을 이용하여 아래와 같이 산출합니다. 
{(질병군별 점수×고정비율)×1.3}+{질병군별 점수×(1-고정비율)}
위 식에서 구해진 가산점수를 다음의 질병군별 점수산정요령에 따라 질병군별 상대가치점수를 구하고 점수당 단가를 곱하여 급여비용을 산출합니다.
< 정상군의 경우 >
【{질병군별 점수×고정비율}+{질병군별 점수×(1-고정비율) × 가입자 등의 입원일수/질병군별 평균 입원일수}】× 20/100 +【질병군별 점수】× 80/100
(예시) 상급종합병원의 복강경을 이용하여 자궁근종절제술(복잡)을 실시한 경우, 질병군은 N04500(복강경을 이용한 기타 자궁 수술)이며, 30% 가산 수가적용시 가산점수 산출
※ 산부인과 가산점수 산정시에는 특정내역 ‘MT041 산부인과 가산점수 산정’에 Y'를 기재하여 청구합니다.\par
\begin{description}\tightlist
\item[질병군] N04500 복강경을 이용한 기타 자궁 수술(악성종양제외)
\item[점수] 37,504.24
\item[고정비율] 0.7
\item[가산점수] 45,380.13 {(37,504.24×0.7)×1.3} + {37,504.24 × (1-0.7)}
\end{description}
}
\prezi{\clearpage}
\par
\medskip
\Que{양측 난관전절제술을 실시한 경우에도 가산점수 산정이 가능한지 여부}
\Ans{자궁내막증이 있거나 불임(또는 난임) 등으로 임신가능성을 높이기 위해 난소 또는 난관 전절제술을 실시한 경우는 사례별로 인정할 수 있으며 환자의 연령, 출산력, 질환상태 등을 참고하여 진료담당의사의 의학적 판단 하에 임신·출산능력을 보존하는 수술을 시행한 경우 산정함을 원칙으로 함}
\prezi{\clearpage}
\par
\medskip
\Que{불임(또는 난임)으로 임신가능성을 높이기 위해 난소 또는 난관 전절제술을 실시한 경우는 사례별로 「기타 자궁 수술」 및 「자궁부속기수술」 질병군의 가산점수를 인정할 수 있음. 이 때 불임(또는 난임)의 정의는?}
\Ans{피임없이 정상적인 부부생활을 하면서 1년 내에 임신이 되지 않은 경우(일차성 불임)와 유산, 자궁 외 임신 및 분만 후 1년 이내에 임신이 되지 않은 경우(이차성 불임)를 의미함 }

\clearpage
\subsection{병실료등}
\begin{myshadowbox}
\begin{enumerate}[8.]\tightlist
\item 가입자 또는 피부양자가 제1호에 따른 요양기관(제3편을 적용받는 요양병원은 제외)에서 「국민건강보험법」제43조에 따라 신고한 일반입원실 및 정신과폐쇄병실의 4인실 또는 5인실을 이용한 경우에는 별표 2의3의 추가 비용 계산식에 따른 금액을 추가 산정하고, 상급종합병원의 일반입원실 및 정신과폐쇄병실의 1인실(보건복지부장관이 정하여 고시하는 불가피한 1인실 입원의 경우 제외)을 이용한 경우에는 제5호 본문에 따른 금액에서 1인실 이용일수에 해당하는 기본입원료(제1편(행위별 수가)제2부제1장 가-2-가)를 제외하고 산정한다.
\item 영 별표 2 제2호 나목의 “보건복지부장관이 정하여 고시하는 입원실을 이용한 경우”라 함은 가입자 등이 제1호에 따른 요양기관에서 국민건강보험법 제43조에 따라 신고한 일반입원실 및 정신과폐쇄병실의 4인실 또는 5인실을 이용한 경우를 말하며, 별표 2의3의 본인부담액 계산식에 따른 금액을 더 하여 본인부담액을 산정한다.
\end{enumerate}
\end{myshadowbox}
\prezi{\clearpage}
\Que{4인실 또는 5인실 이용에 따른 추가비용 산정 시 간호등급이나 입원료 체감제 등도 적용되는지?}
\Ans{4인실 또는 5인실입원료는 제1편(행위)의 종별에 따른 입원료(가-2)를 말하며, 입원료관련 가산 또는 감산은 적용하지 않으므로 간호등급이나 입원료체감제 등은 \textcolor{red}{적용하지 않음}\par
☞ (별표 2의3) ‘주1’ 참조}
\prezi{\clearpage}
\par
\medskip
\Que{본인부담경감대상자의 4인실 또는 5인실 이용에 따른 추가비용 산정 시 본인부담률 적용 방법은?}
\Ans{「국민건강보험법시행령」 별표2 제3호에 해당하는 대상자인 경우에는 그 각목에서 정한 본인부담률을 적용함 (중증질환 5\%, 희귀난치질환 10\%, 6세미만 10\%, 신생아 면제 등)\par
☞ (별표2의3) ‘주2’ 참조}
\prezi{\clearpage}
\par
\medskip
\Que{상급병실 입원료 개정 관련 신설된 심사불능 코드 항목은?}
\Ans{
\begin{description}\tightlist
\item[코드] 60-36 \emph{내용} 질병군(DRG)입원료 산정과 요양기관 현황신고내역 불일치 * '14.9.10 이후 접수분 적용
\item[코드] 60-37 \emph{내용} 질병군(DRG)입원료 산정착오 또는 기재착오 *'14.9.1 이후 접수분 적용
\end{description}
}
\clearpage
\subsection{기타 QA}
\begin{commentbox}{‘요양급여비용열외군’이란?}
진료비가 예외적으로 많이 발생한 질병군 진료건에 대해 추가 보상을 하기 위한 제도로 질병군으로 산정한 요양급여비용 총액이 행위별로 산정한 금액보다 적고 그 차액이 100만원을 초과하는 경우 초과 금액을 추가로 지급함
\begin{itemize}\tightlist
\item 열외군보상금액 = (행위별총진료비 - DRG 총진료비)-100만원
\item 예시
	\begin{itemize}\tightlist
	\item  DRG 총진료비: 100만원
	\item 행위별 총진료비: 400만원
	\item DRG 요양급여비용 총액: 300만원(200만원 추가지급) 
	\end{itemize}
\end{itemize}	
\end{commentbox}
\prezi{\clearpage}
\Que{‘요양급여비용 열외군’ 환자인 경우 ‘행위별 총진료비’ 산정방법은?}
\Ans{‘요양급여비용 열외군’ 환자인 경우 ‘행위별 총진료비’를 산정방법은 다음과 같음\par
\begin{center}\emph{- 다 음 -}\end{center}
\begin{enumerate}[가)]\tightlist
\item 행위별수가제의 100/100본인부담 및 비급여사항을 포함하되, 질병군에서 환자에게 별도 징수가 가능한 이송처치료, PCA(통증자가조절법)와 비급여대상 등은 제외
\item 행위는 상대가치점수표에서 정한 기준에 의해 산정
\item 약제 및 치료재료는 약제급여목록 및 급여상한금액표 또는 치료재료급여ㆍ비급여목록 및 급여상한금액표의 상한금액을 초과하지 않는 범위내에서 실구입가로 산정 
\item 질병군의 급여범위에 해당되나 행위별에서의 비급여대상에 해당하는 행위는 해당 요양기관의 관행수가를 적용하여 산정하고 약제 및 치료재료는 실구입가로 산정 ⇒ 위 가~라에 의거 계산된 금액을 행위별 총진료비란에 기재\newline
☞ 요양급여비용 심사청구서·명세서 세부작성요령 (Ⅸ 질병군요양급여비용 작성요령)
\end{enumerate}
}
\prezi{\clearpage}
\begin{commentbox}{질병군 진료기간 중 질병군 수술과 전혀 다른 상병에 대해 협진 시 다른 상병으로 인한 진료비용의 별도산정 여부}
질병군 급여비용에는 동반상병 및 \textcolor{red}{합병증 진료에 대한 비용이 포함되어 있음. 
다만, 동반상병 및 합병증 진료에 따른 자원소모 등을 고려하여 기타진단을 진단명부여원칙에 따라 코딩}하는 경우 진단명에 따라 중증도 분류가 다르게 적용될 수 있음
\end{commentbox}
\prezi{\clearpage}
\begin{commentbox}{질병군 진료기간 중 수혈을 실시한 환자가 현혈증을 제시한 경우 청구방법은?}
질병군 진료기간 중 수혈을 실시한 경우 질병군 상대가치점수는 수혈비용 등을 포함하여 산출된 것이므로 환자에게 별도로 수혈비용을 부담시킬 수 없음
또한,「헌혈증서 소지자의 수혈비용에 대한 본인부담 산정방법」(급여 65720-1898호, '01.12.29.) 에 따라 질병군에 대한 본인부담액에서 헌혈증서에 의한 대한적십자사의 보상금액을 공제 후 환자에게 징수하고 행위별청구와 동일하게 수혈비용을 대한적십자사총재에게 청구함
\end{commentbox}

\clearpage
\subsection{검사관련}
※「의료의 질향상을 위한 점검표」중 ‘1.1 수술전 검사 시행여부’ 관련
\Que{수술전 검사항목은 무엇인가요?}
\Ans{척추마취 및 전신마취의 경우\newline
(A) 7개질병군 공통 : CBC(일반혈액검사), U/A(요검사), LFT(간기능검사), Electrolyte(전해질검사), Chest PA(흉부방사선촬영), EKG(심전도), BUN(요소질소), Creatinine(크레아티닌), Coagulation(응고검사), ABO/Rh(혈액형검사)} 
\prezi{\clearpage}
\par
\medskip
\Que{수술전 검사 일부 항목만 시행했을 경우에도 “시행”에 표시하면 되나요?}
\Ans{수술전 검사항목을 모두 시행하였을 경우 “시행”에 표시하고 수술전 검사항목 중 하나라도 시행하지 아니한 경우에는 “미시행”에 표시합니다.
- 수정체수술 또는 편도 및 아데노이드 수술을 전신마취 또는 척추마취하에 실시한 경우는 7개 질병군 공통 검사(A)와 수정체수술(B) 또는 편도 및 아데노이드 수술(C)의 검사를 모두 시행한 경우 “시행”에 표시합니다.\par
※ 예시) 편도 및 아데노이드 수술을 전신마취 하에 실시한 경우
(A) + (C)의 검사를 모두 시행한 경우 “시행”에 표시 }
\par
\medskip
\prezi{\clearpage}
\Que{입원하여 시행한 수술전 검사만 해당되나요? }
\Ans{외래에서 시행한 수술전 검사와 입원하여 시행하는 수술전 검사 모두 해당됩니다.}
\prezi{\clearpage}
\par
\medskip
\Que{수술전 검사 시행여부의 추가코드 표시는 어떻게 하나요?}
\Ans{추가코드 □󰊱 □󰊲 □󰊳은 마취의 종류를 표기하는 코드로서 전신 마취를 시행한 경우에는 추가코드 ☑󰊱에, 척추마취를 시행한 경우 에는 ☑󰊲에, 기타 국소마취를 시행한 경우에는 ☑󰊳에 표시하여야 합니다.}
\prezi{\clearpage}


%\subsection{타 법령(산재, 자보 등)으로 입원진료 중 질병군 진료가 발생한 경우 요양급여비용 산정방법}
%\begin{itemize}[▶]\tightlist
%\item 산업재해보상보험법에 의거 요양급여를 받고 있는 자에게 국민건강 보험법에 해당하는 요양급여가 발생한 경우 진찰료, 입원료, 조제료, 주사료, 방사선진단료 등 산재 요양급여와 중복되는 항목은 건강보험으로 산정할 수 없으나 그 외 건강보험 요양급여에 필요한 요양급여 비용은 산정할 수 있도록 되어 있음.(급여1492-46323호, ‘82.6.23)
%\item 그러나 질병군별 포괄수가는 질병군 진료를 위하여 필요한 모든 비용(진찰료, 입원료, 조제료, 주사료, 수술료, 약제료, 치료재료비용 등)이 포함되어 있어 동 포괄수가에서 타 법령의 요양급여와 중복되는 항목의 비용을 제외하기는 곤란하므로 타 법령(산재, 자보 등)으로 입원진료 중 질병군 진료가 발생한 경우에는 행위별수가제를 적용토록 함.	
%\end{itemize}

%\subsection{<질병군 청구명세서에서 사용하는 특정내역 코드 및 설명>}
%※ 「요양급여비용 청구방법, 심사청구서 명세서서식 및 작성요령」의 별표8.특정내역 구분코드 중 질병군에서 사용하는 항목을 표기
%
%구분 코드 & 특정내역 & 특정내역 기재형식 & 설 명
%MS004 & 신생아체중(*) & 9(4) & 모든 분만 명세서와 신생아 명세서의 경우 신생아 체중을 기재\newline 분만 명세서에는 출생 당시의 신생아 체중으로 기재하고, 신생아 명세서에는 입원(또는 출생) 당시 신생아 체중을 기재하되 그램(gram) 단위로 기재
%MT036 & 의료의 질 점검 내용(**) & ccyymmdd/X(1)/9(1)/X(1)/X(1)/X(1)/X(1)/X(2)/X(1)/X(1)/9(2)/X(1)/X(1)/9(1)/X(1)/X(1)/X(1)/X(1)/X(1) & 건강보험요양급여비용 제2편 질병군 급여 비급여 목록 및 급여 상대가치점수의 별지 서식 “의료의 질 향상을 위한 점검표”의 수술일과 점검사항을 작성요령에 따라 순서대로 기재(미시행, 없음 및 이상의 경우 N, 시행, 있음 및 정상의 경우 Y로 표기) 
%MT039 & 복강경 수술중 개복하여 수술(**) & X(1) & 복강경을 이용한 수술 중 부득이한 사유로 개복하여 수술을 종결한 경우 “Y"로 기재
%MT041 & 산부인과 가산점수 산정(**) & X(1) & 건강보험 행위 급여 비급여 목록표 및 급여 상대가치점수 제2편 질병군 급여 비급여 목록 및 급여 상대가치점수 제2부 제4장 산부인과 적용지침 2.에 따라 산부인과 가산점수를 산정한 경우 'Y'를 기재
%MT042 & 다빈치 로봇을 이용한 수술(**) & 9(8)/X(1) & 건강보험 행위 급여 비급여 목록표 및 급여 상대가치점수 제2편 질병군 급여 비급여 목록 및 급여 상대가치점수 제2부 제4장 산부인과 적용 지침 6. 등에 따른 다빈치 로봇 수술을 실시한 경우에는 해당 비급여 비용과 환자동의 여부를 기재 ※ 환자동의서를 작성 비치한 경우 “Y"
%MT034 & 행위 질병군 분리청구의 경우 최초입원 개시일(**) & ccyymmdd & 행위별과 질병군 분리청구 기준에 따라 질병군 진료 이외의 목적으로 입원하여 진료 중 질병군 진료를 실시하여 분리청구하는 경우 최초입원개시일을 기재

