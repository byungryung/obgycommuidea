\section{여성 청소년 대상 사람유두종바이러스 예방접종 및 진찰 \cntrdot{} 상담 사업에 따른 진찰료 산정방법에 관한 질의 응답}
\myde{}{
\begin{itemize}\tightlist
\item R688 기타 명시된 전신 증상 및 징후
\end{itemize} 
}
{
\begin{itemize}\tightlist
\item 특정기호 F012(여성 청소년 대상 사람유두종바이러스(HPV) 예방접종 및 진찰\cntrdot{} 상담사업 지원  대상자) 기재함
\end{itemize}
}

\subsection{일반사항}
건강보험 이외 의료급여, 보훈 대상자 포함여부
\begin{quotebox}
 ○ 건강보험(차상위 포함) 이외 의료급여, 보훈 포함
\end{quotebox} 
외래 및 입원시 여성청소년 대상 사람유두종바이러스(HPV) 예방접종 및 진찰\cntrdot{}  상담 사업에 따라 예방접종과 진찰\cntrdot{} 상담이 실시된 경우 본인부담률 산정방법
\begin{quotebox}
○ 외래-입원 구분에 따른 각각의 본인부담률 적용
\end{quotebox}

\subsection{수가산정관련}
여성청소년 대상 사람유두종바이러스(HPV) 예방접종 및 진찰\cntrdot{}  상담 사업에 따른 진찰료 산정 시 야간\cntrdot{} 공휴 가산 적용 여부 \par
\begin{quotebox}
○ 현행「행위 급여\cntrdot{} 비급여 목록 및 급여 상대가치점수」가-1-가. 초진 진찰료 산정과 동일하게   야간\cntrdot{} 공휴\cntrdot{} 토요 가산 적용함
\end{quotebox}
같은날 동일 의사에게 여성청소년대상 사람유두종바이러스(HPV) 예방접종 사업에 따른 진찰\cntrdot{} 상담 이외 별도로 질환에 대하여 진료받은 경우 진찰료 산정방법
\begin{quotebox}
○ 현행 진찰료 산정기준에 따라, HPV예방접종 사업에 따른 진찰\cntrdot{} 상담에 대한 진찰료 1회만 산정  \par
※ 다만, 2개 이상의 진료과목이 설치되어 있고 해당 과의 전문의가 상근하는 요양기관에서 전문과목 또는 전문분야가 다른 진료담당 의사가 별도 질환을 진찰한 경우는 진찰료를 각각 산정 가능 (분리청구)
\end{quotebox}
여성청소년 대상 사람유두종바이러스(HPV) 예방접종 및 진찰\cntrdot{}  상담 사업에 따른 진찰료 산정 시 의료질평가지원금 및 전문  병원 관리료, 전문병원 의료질 지원금 산정 여부
\begin{quotebox}
○ 산정할 수 없음. 다만, 같은 날 예방접종 사업에 따른 진찰\cntrdot{} 상담 이외 별도로 질환에 대한 진료를 동시에 실시한 경우는 산정 가능(분리 청구)
\end{quotebox}
여성청소년 대상 사람유두종바이러스(HPV) 예방접종 및 진찰\cntrdot{}  상담 사업에 따른 진찰료 산정 시 선택진료 추가비용 산정 여부
\begin{quotebox}
○ 산정할 수 없음. 다만, 같은 날 예방접종 사업에 따른 진찰\cntrdot{} 상담 이외 별도로 질환에 대한 진료를 동시에 실시한 경우는 산정 가능(분리 청구)
\end{quotebox}
예외 인정자(면역저하자에게 3회 접종하거나 조혈모세포 이식 등 타당한 의학적 사유로 재접종이 필요한 경우)의 경우 진찰료도  3회 산정 가능한지 여부
\begin{quotebox}
○ 표준 여성 청소년 건강 상담 시 발생하는 진찰료는 대상자당 최대 2회까지 인정
\end{quotebox}
사람유두종바이러스(HPV) 예방접종 시행일 외 다른 날짜에   방문하여 표준 여성 청소년 건강 상담을 하는 경우 진찰료 인정 여부
\begin{quotebox}
○ 여성청소년 대상 사람유두종바이러스(HPV) 예방접종 및 진찰\cntrdot{} 상담 사업에 따른 진찰료는 예방 접종 시행 당일 동시에 표준 여성 청소년 건강 상담을 제공한 경우에만 인정
\end{quotebox}

\subsection{청구관련}
여성청소년 대상 사람유두종바이러스(HPV) 예방접종 및 진찰\cntrdot{}  상담 사업에 따른 진찰료 청구시 기재된 본인일부부담금은?
\begin{quotebox}
○ 실제 본인이 부담하는 금액은 환자에게 징수하지 않음 (사업예산으로 지급)
\end{quotebox}
 <건강보험, 의원 외래 작성예시>
\par
\medskip
\tabulinesep =_2mm^2mm
\begin {tabu} to \linewidth {|X[1,l]|X[1,l]|X[1,l]|X[1,l]|X[1,l]|} \tabucline[.5pt]{-}
요양급여비용총액 2 & 요양급여비용총액 1 & 본인일부부담금 & 청구액 & 특정내역구분(MT002) \\ \tabucline[.5pt]{-}
주1)14,410 & 14,410 & 주2)4,300 & 주3)10,110 & F012 \\ \tabucline[.5pt]{-}
\end{tabu}
\par
주1) 요양급여비용총액2 = 진찰료 금액\par
주2) 국민건강보험법 시행령 별표2 및 같은법 시행규칙  별표 3에 따른 법정 본인부담금을 기재 \par
     14,410원(요양급여비용총액1) X 30\%(외래 본인부담률) = 4,300원(100원미만 절사) \par 
    여성 청소년 대상 사람유두종 바이러스(HPV) 예방  접종 및 진찰\cntrdot{} 상담 사업 지원대상자에게 징수하지   않음(사업예산으로 지급)\par
주3) 요양급여비용총액1-본인일부부담금을 기재 \par

 <의료급여 2종 수급권자, 의원 외래 작성예시>
\par
\medskip
\tabulinesep =_2mm^2mm
\begin {tabu} to \linewidth {|X[1,l]|X[1,l]|X[1,l]|X[1,l]|X[1,l]|} \tabucline[.5pt]{-}
요양급여비용총액 2 & 요양급여비용총액 1 & 본인일부부담금 & 청구액 & 특정내역구분(MT002) \\ \tabucline[.5pt]{-}
주1)14,410 & 14,410 & 주2)1,000 & 주3)13,410 & F012 \\ \tabucline[.5pt]{-}
\end{tabu} \par
주1) 요양급여비용총액2 = 진찰료 금액 \par
주2) 의료급여법 시행령 별표1의 2호에 따른 본인일부부담금 기재 \par
     = 2종 수급권자 그밖의 외래진료시 본인부담금=1,000원 \par
    여성 청소년 대상 사람유두종 바이러스(HPV) 예방  접종 및 진찰\cntrdot{} 상담 사업 지원대상자에게 징수하지   않음(사업예산으로 지급) \par
주3) 요양급여비용총액1-본인일부부담금= 14,410원-1,000원=13,
\par
\medskip

여성청소년 대상 사람유두종바이러스(HPV) 예방접종 시행 당일 진찰\cntrdot{} 상담 외 다른 행위(검사,  처치 등)가 동시에 발생한 경우 청구방법
\begin{quotebox}
○ 명세서를 구분하여 각각 작성함
\end{quotebox}
\tabulinesep =_2mm^2mm
\begin {tabu} to .75\linewidth {|X[1,l]|X[1,l]|X[1,l]|} \tabucline[.5pt]{-}
\rowcolor{ForestGreen!40} 구분 & 특정내역구분 & 특정내역  \\ \tabucline[.5pt]{-}
\rowcolor{Yellow!40} 예방접종 & MT002 & F012  \\ \tabucline[.5pt]{-}
\rowcolor{Yellow!40} 다른 행위(검사, 처치 등) & MT001 & R  \\ \tabucline[.5pt]{-}
\end{tabu}