\section{입덧}
\myde{}{
\begin{itemize}\tightlist
\item[\dsjuridical]  0210 경도의 임신 입덧 
\item[\dsjuridical]  0211 대사 장애를 동반한 임신 입덧
%\item[\dsmedical] 
%\item[\dschemical] 
\end{itemize}
}
{
%\leftrod{}\par
%%\begin{center}\textbf{- 다          음 -}\end{center}
%\begin{enumerate}[가.]\tightlist
%\item 
%%	\begin{enumerate}[1)]\tightlist
%	\item 
%	\end{enumerate}
%\end{enumerate}
}

\subsection{디클렉틴(Diclectin Enteric Coated Tab.)}
처방하는데 효과가 좋더라구요. 비급여약입니다
\href{https://openwiki.kr/med/%EB%94%94%ED%81%B4%EB%A0%89%ED%8B%B4}{openwiki 관련site}\par
\leftrod{용 법}
환자 상태에 따라 2-4정 복용하게 된다. 1일 최대권장용량은 4정이다(아침에 1정, 오후 중반에 1정, 취침 전에 2정).\par
첫날 자기 전 2정 복용.다음 날 효과가 있으면 계속해서 자기 전 2정을 복용함.
만약 둘째날 효과가 없으면 둘째날 똑같이 자기 전 2정을 복용하고 셋째날 아침1정 자기 전 2정을 복용하며 효과가 있으면 계속해서 3정을 복용.
셋째날도 효과가 미비하다면 넷째날부터 아침 1정, 오후(공복) 1정, 자기전 2정 복용. 
\begin{commentbox}{Vs 아론정}
    \begin{itemize}\tightlist
    \item 입덧에 \highlight{아론정} 비급여  (독살라민) 반알 하루두번 주시면 됩니다 디클렉틴과 똑같은 성분이고 가격은 10분의 일입니다 피라독신만 추가하셔도 되고요
    \item 디클렉틴 카테고리A이고요, 비급여 약인데 하루 2정씩 저녁 자기 전에 먹게 하고요, 약국에서 1주일에 21,500원 2017년 
    \item 피리독신과 독살라민 합성 하는게 어려운 과정인가봐요 가격이 이렇게나 차이나니 
    \item 그냥 따로 2알 먹는 것은 상관이 없는데, 하나로 혼합한 약은 새로 허가 받아야 되니 다시 임상실험 해서 자료 내고 그래야 될겁니다.
    \item 피라독신은 별 효용이 크게 없습니다 주작용은 독실라민인데 그냥 만든거죠 독실라민이 예전에 다른약과 섞어사용하다 큰화를 당해서 근데 디클렉틴은 가격이너무 사악해서
    \item \uline{slow releasing이고 장용정} 이라는 점이 다르다고 하네요
    \end{itemize}
\end{commentbox}

\subsection{ondansetron at pregnancy}
\Que{입덧 심하신 분 온단세트론 비급여로(1만5천원) 영양제나 Fluid 에 섞어서 많이 씁니다만, 효과는 엄청 좋은 것 같네요. 근데 산모에게 쓰면 금기인가요? 임부 등급 B 등급이고 수유모 등급 2 등급이던데요}
\Ans{
    \begin{enumerate}\tightlist
    \item 온단세트론 한동안 산모에게 많이 사용했었는데 birth defect발견되어서 미국에서 소송이 많이 발생하는 바람에 사용이 좀 어려워진 점이 있습니다. 미국에서도 입덧시 사용허가가 나지는 않았었지만 의사들 사이에서 많이 사용했었거든요
    \item 종병있을때 DM산모 한명이 임신기간 열달내내 구토 증상 심해서 내분비내과랑 산부인과 협진으로 입원치료 계속 했던 산모분 있었는데 어쩔수없이 맥페란 조프란 다 사용했었어요. 다른 원인 없나 내시경까지 했는데 문제 없었고 결국 너무 힘들어해서 37주 넘어서 유도분만 성공했고 임신 마치니까 다이나믹하게 nausea증상 좋아지더라구요. 임신 오조에 쓸만한 약이 없는게 사실이고 환자에게 잘 설명후 사용하셔야 할것 같긴합니다
    \end{enumerate}
}
\begin{commentbox}{}
\textbf{QUESTION:}\par
While I usually prescribe doxylamine-pyridoxine for morning sickness, some of my patients with severe nausea and vomiting of pregnancy (NVP) receive ondansetron in hospital. I have read some new precautions recommended by the US Food and Drug Administration (FDA). Is ondansetron safe to use during pregnancy?\par
\textbf{ANSWER:}\par
During the past decade ondansetron has been increasingly used in the United States for NVP, owing to the lack of an FDA-approved drug for this condition. While fetal safety data for doxylamine-pyridoxine are based on more than a quarter of a million pregnancies, the fetal safety data for ondansetron are based on fewer than 200 births. Moreover, a recent case-control study suggested there was an \highlight{increased risk of cleft palate associated with ondansetron}. Recently, the FDA issued a warning about \highlight{potentially serious QT prolongation and torsade de pointes associated with ondansetron use}; the warning included a list of precautions and tests that must be followed. The drug is not labeled for use in NVP in either the United States or Canada. Based on the data available today, ondansetron use \uline{cannot be assumed to be safe during pregnancy.}
\end{commentbox}