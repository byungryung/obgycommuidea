\begin{description}\tightlist	
\item[가-2] 입원료 Inpatient Care\footnote{주 : 1. 내과질환자, 정신질환자, 만8세 미만의 소아환자에 대하여는 소정점수의 30\%를 가산(산정코드 세 번째 자리에 4로 기재)한다.(주2에 해당하는 경우 제외) 2. 강내치료를 위하여 밀봉소선원치료실에 입원한 경우에는 3일 이내의 기간 동안 소정점수의 100\%를 가산한다.(산정코드 세 번째 자리에 3으로 기재)}
	\begin{enumerate}[가.]\tightlist
	\item 기본입원료 
	
	\medskip
	\tabulinesep =_2mm^2mm
	\begin{tabu} to\linewidth {|X[2,l]|X[6,l]|X[1,l]|X[1,l]|} \tabucline[.5pt]{-}
	\rowcolor{ForestGreen!40}  코드 &	\centering 분 류 & 점수 & 금액 \\ \tabucline[.5pt]{-}
	\rowcolor{Yellow!40} AB100(15100) & (1) 상급종합병원 & 522.27 & \myexplfng{522.27} \\ \tabucline[.5pt]{-}
	\rowcolor{Yellow!40} AB200(15200)  & (2) 종합병원 & 480.64 & \myexplfng{480.64} \\ \tabucline[.5pt]{-}
	\rowcolor{Yellow!40} AB300 & (3) 병원, 치과병원, 한방병원 내 치\cntrdot{}의과 & 421.01 & \myexplfng{421.01} \\ \tabucline[.5pt]{-}
	\rowcolor{Yellow!40} 15300 & (4) 한방병원, 병원\cntrdot{}치과병원 내 한의과 & 417.36 &  \myexplfng{417.36} \\ \tabucline[.5pt]{-}
	\rowcolor{Yellow!40} AB400 & (5) 의원, 치과의원, 보건의료원 치\cntrdot{}의과 & 358.86 & \myexplfng{358.86} \\ \tabucline[.5pt]{-}
	\rowcolor{Yellow!40} 15400 & (6) 한의원, 보건의료원 한방과 & 355.29 & \myexplfng{355.29} \\ \tabucline[.5pt]{-}
	\end{tabu}
	
	\item 5인실 입원료 
	
	\medskip
	\tabulinesep =_2mm^2mm
	\begin{tabu} to\linewidth {|X[2,l]|X[6,l]|X[1,l]|X[1,l]|} \tabucline[.5pt]{-}
	\rowcolor{ForestGreen!40}  코드 &	\centering 분 류 & 점수 & 금액 \\ \tabucline[.5pt]{-}
	\rowcolor{Yellow!40} AB120(15120) & (1) 상급종합병원 & 678.95 & \myexplfng{678.95}  \\ \tabucline[.5pt]{-}
	\rowcolor{Yellow!40} AB220(15220) & (2) 종합병원 & 624.83 & \myexplfng{624.83} \\ \tabucline[.5pt]{-}
	\rowcolor{Yellow!40} AB320 & (3) 병원, 치과병원, 한방병원 내 치\cntrdot{}의과 & 547.31 & \myexplfng{547.31} \\ \tabucline[.5pt]{-}
	\rowcolor{Yellow!40} 15320 & (4) 한방병원, 병원\cntrdot{}치과병원 내 한의원 & 542.57 & \myexplfng{542.57} \\ \tabucline[.5pt]{-}
	\rowcolor{Yellow!40} AB420 & (5) 의원, 치과의원, 보건의료원 치\cntrdot{}의과 & 466.52 & \myexplfng{466.52} \\ \tabucline[.5pt]{-}
	\rowcolor{Yellow!40} 15420 & (6) 한의원, 보건의료원 한방과 & 61.88 & \myexplfng{61.88} \\ \tabucline[.5pt]{-}
	\end{tabu}
	
	\item 4인실 입원료 
	
	\medskip
	\tabulinesep =_2mm^2mm
	\begin{tabu} to\linewidth {|X[2,l]|X[6,l]|X[1,l]|X[1,l]|} \tabucline[.5pt]{-}
	\rowcolor{ForestGreen!40}  코드 &	\centering 분 류 & 점수 & 금액 \\ \tabucline[.5pt]{-}	
	\rowcolor{Yellow!40} AB140(15140) & (1) 상급종합병원 & 835.63 & \myexplfng{835.63} \\ \tabucline[.5pt]{-}
	\rowcolor{Yellow!40} AB240(15240) & (2) 종합병원 & 769.02 & \myexplfng{769.02} \\ \tabucline[.5pt]{-}
	\rowcolor{Yellow!40} AB340 & (3) 병원, 치과병원, 한방병원 내 치\cntrdot{}의과 & 673.62 & \myexplfng{673.62} \\ \tabucline[.5pt]{-}
	\rowcolor{Yellow!40} 15340 & (4) 한방병원, 병원\cntrdot{}치과병원 내 한의과 & 667.78 & \myexplfng{667.78} \\ \tabucline[.5pt]{-}
	\rowcolor{Yellow!40} AB440 & (5) 의원, 치과의원, 보건의료원 치\cntrdot{}의과 & 574.18 & \myexplfng{574.18} \\ \tabucline[.5pt]{-}
	\rowcolor{Yellow!40} 15440 & (6) 한의원, 보건의료원 한방과 & 568.46 & \myexplfng{568.46} \\ \tabucline[.5pt]{-}
	\end{tabu}
	\end{enumerate}

\item[가-6] 낮병동 입원료 Day Care \par
\tabulinesep =_2mm^2mm
\begin{tabu} to\linewidth {|X[2,l]|X[6,l]|X[1,l]|X[1,l]|} \tabucline[.5pt]{-}
\rowcolor{ForestGreen!40}  코드 &	\centering 분 류 & 점수 & 금액 \\ \tabucline[.5pt]{-}
\rowcolor{Yellow!40} AF100(18100) & 가. 상급종합병원 & 522.27 & \myexplfng{522.27} \\ \tabucline[.5pt]{-}
\rowcolor{Yellow!40} AF200(18200) & 나. 종합병원 & 480.64 & \myexplfng{480.64} \\ \tabucline[.5pt]{-}
\rowcolor{Yellow!40} AF300 & 다. 병원, 치과병원, 한방병원 내 의\cntrdot{}치과 & 421.01 & \myexplfng{421.01} \\ \tabucline[.5pt]{-}
\rowcolor{Yellow!40} 18300 & 라. 한방병원, 병원\cntrdot{}치과병원 내 한의과 & 417.36 & \myexplfng{417.36} \\ \tabucline[.5pt]{-}
\rowcolor{Yellow!40} AF400 & 마. 의원, 치과의원, 보건의료원 의\cntrdot{}치과 & 358.86 & \myexplfng{358.86} \\ \tabucline[.5pt]{-}
\rowcolor{Yellow!40} 18400 & 바. 한의원, 보건의료원 내 한의원 & 355.29 & \myexplfng{355.29} \\ \tabucline[.5pt]{-}
\end{tabu}
%\myexplfng{ } \\ \tabucline[.5pt]{-} %
\item[가-7] 신생아 입원료 Neonatal Care \footnote{주:신생아제대처치, 기저귀 교환, 혈압, 맥박, 호흡 측정, 목욕 등의 비용과 기저귀 비용이 포함되어 있으므로 그 비용을 별도 산정하지 아니한다.}
	\begin{enumerate}[가.]\tightlist
	\item 신생아실 입원료\footnote{주:질병이 없는 신생아를 신생아실에서 진료\cntrdot{} 간호한 경우에 산정한다.} 
	
	\medskip
	\tabulinesep =_2mm^2mm
	\begin{tabu} to\linewidth {|X[2,l]|X[6,l]|X[1,l]|X[1,l]|} \tabucline[.5pt]{-}
	\rowcolor{ForestGreen!40}  코드 &	\centering 분 류 & 점수 & 금액 \\ \tabucline[.5pt]{-}
	\rowcolor{Yellow!40} AG111 & (1) 상급종합병원 & 856.41 & \myexplfng{856.41} \\ \tabucline[.5pt]{-} %59,950 \\ \tabucline[.5pt]{-}	
	\rowcolor{Yellow!40} AG211 & (2) 종합병원 & 789.87 & \myexplfng{789.87} \\ \tabucline[.5pt]{-} %55,290 \\ \tabucline[.5pt]{-}
	\rowcolor{Yellow!40} AG311 & (3) 병원, 치과병원\cntrdot{} 한방병원 내 의과 & 481.46 & \myexplfng{481.46} \\ \tabucline[.5pt]{-} %33,700 \\ \tabucline[.5pt]{-}
	\rowcolor{Yellow!40} AG411 & (4) 의원, 보건의료원 의과 & 448.94 & \myexplfng{448.94} \\ \tabucline[.5pt]{-} %33,400 \\ \tabucline[.5pt]{-}
	\end{tabu}
	
	\item 모자동실 입원료\footnote{주:질병이 없는 신생아를 모자동실에서 진료\cntrdot{}간호한 경우에 산정한다.} 
	
	\medskip
	\tabulinesep =_2mm^2mm
	\begin{tabu} to\linewidth {|X[2,l]|X[6,l]|X[1,l]|X[1,l]|} \tabucline[.5pt]{-}
	\rowcolor{ForestGreen!40}  코드 &	\centering 분 류 & 점수 & 금액 \\ \tabucline[.5pt]{-}	
	\rowcolor{Yellow!40} AG112 & (1) 상급종합병원 & 1,145.19 & \myexplfng{1145.19} \\ \tabucline[.5pt]{-} %80,160 \\ \tabucline[.5pt]{-}
	\rowcolor{Yellow!40} AG212 & (2) 종합병원 & 1,061.00 & \myexplfng{1061.00} \\ \tabucline[.5pt]{-} %74,270 \\ \tabucline[.5pt]{-}
	\rowcolor{Yellow!40} AG312 & (3) 병원, 치과병원\cntrdot{} 한방병원 내 의과 & 625.94 & \myexplfng{625.94} \\ \tabucline[.5pt]{-} %43,820 \\ \tabucline[.5pt]{-}
	\rowcolor{Yellow!40} AG412 & (4) 의원, 보건의료원 의과 & 572.60 & \myexplfng{572.60} \\ \tabucline[.5pt]{-} %42,600 \\ \tabucline[.5pt]{-}
	\end{tabu}
	
	\item 신생아 모유수유간호관리료\footnote{주:「가」 또는 「나」를 산정하는 신생아에게 모유수유를 한 경우에 산정한다.} 
	
	\medskip
	\tabulinesep =_2mm^2mm
	\begin{tabu} to\linewidth {|X[2,l]|X[6,l]|X[1,l]|X[1,l]|} \tabucline[.5pt]{-}
	\rowcolor{ForestGreen!40}  코드 &	\centering 분 류 & 점수 & 금액 \\ \tabucline[.5pt]{-}	
	\rowcolor{Yellow!40} AG113 & (1) 상급종합병원 & 425.57 & \myexplfng{425.57} \\ \tabucline[.5pt]{-} %29,790 \\ \tabucline[.5pt]{-}
	\rowcolor{Yellow!40} AG213 & (2) 종합병원 & 374.84 & \myexplfng{374.84} \\ \tabucline[.5pt]{-} %26,240 \\ \tabucline[.5pt]{-}
	\rowcolor{Yellow!40} AG313 & (3) 병원, 치과병원\cntrdot{} 한방병원 내 의과 & 213.46 & \myexplfng{213.46} \\ \tabucline[.5pt]{-} %14,940 \\ \tabucline[.5pt]{-}
	\rowcolor{Yellow!40} AG413 & (4) 의원, 보건의료원 의과 & 187.53 & \myexplfng{187.53} \\ \tabucline[.5pt]{-} %13,950 \\ \tabucline[.5pt]{-}
	\end{tabu}
	\end{enumerate}

\item[가-8] 협의진찰료 Consultation \footnote{주: 「의료법」 제47조에 의한 감염관리위원회 및 감염관리실을 설치\cntrdot{}운영하는 요양기관에서 감염전문관리를 실시한 경우에도 소정점수를 산정한다. (기본코드 다섯 번째 자리에 2로 기재)} 
	\begin{enumerate}[가.]\tightlist
	\item 상급종합병원, 상급종합병원에 설치된 치과대학부속치과병원 
	
	\medskip
	\tabulinesep =_2mm^2mm
	\begin{tabu} to\linewidth {|X[2,l]|X[6,l]|X[1,l]|X[1,l]|} \tabucline[.5pt]{-}
	\rowcolor{ForestGreen!40}  코드 &	\centering 분 류 & 점수 & 금액 \\ \tabucline[.5pt]{-}		
	\rowcolor{Yellow!40} AH500 & (1) 의과, 치과 & 155.57 & \myexplfng{155.57} \\ \tabucline[.5pt]{-} %10,890 \\ \tabucline[.5pt]{-}
	\rowcolor{Yellow!40} 11500 & (2) 한의과 & 150.05 & \myexplfng{150.05} \\ \tabucline[.5pt]{-} % \\ \tabucline[.5pt]{-}
	\end{tabu}
	
	\item 종합병원, 상급종합병원에 설치된 경우를 제외한 치과대학부속치과병원 
	
	\medskip
	\tabulinesep =_2mm^2mm
	\begin{tabu} to\linewidth {|X[2,l]|X[6,l]|X[1,l]|X[1,l]|} \tabucline[.5pt]{-}
	\rowcolor{ForestGreen!40}  코드 &	\centering 분 류 & 점수 & 금액 \\ \tabucline[.5pt]{-}		
	\rowcolor{Yellow!40} AH600 & (1) 의과, 치과 & 141.25 & \myexplfng{141.25} \\ \tabucline[.5pt]{-} %9,890 \\ \tabucline[.5pt]{-}
	\rowcolor{Yellow!40} 11600 & (2) 한의과 & 136.24 &  \myexplfng{136.24} \\ \tabucline[.5pt]{-} %\\ \tabucline[.5pt]{-}
	\end{tabu}
	
	\item 병원, 한방병원, 치과병원 
	
	\medskip
	\tabulinesep =_2mm^2mm
	\begin{tabu} to\linewidth {|X[2,l]|X[6,l]|X[1,l]|X[1,l]|} \tabucline[.5pt]{-}
	\rowcolor{ForestGreen!40}  코드 &	\centering 분 류 & 점수 & 금액 \\ \tabucline[.5pt]{-}	
	\rowcolor{Yellow!40} AH700 & (1) 의과, 치과 & 127.02 & \myexplfng{127.02} \\ \tabucline[.5pt]{-} %8,890 \\ \tabucline[.5pt]{-}
	\rowcolor{Yellow!40} 11700 & (2) 한의과  & 122.51 &  \myexplfng{122.51} \\ \tabucline[.5pt]{-} %\\ \tabucline[.5pt]{-}
	\end{tabu}
	
	\item 요양병원, 보건의료원 
	
	\medskip
	\tabulinesep =_2mm^2mm
	\begin{tabu} to\linewidth {|X[2,l]|X[6,l]|X[1,l]|X[1,l]|} \tabucline[.5pt]{-}
	\rowcolor{ForestGreen!40}  코드 &	\centering 분 류 & 점수 & 금액 \\ \tabucline[.5pt]{-}	
	\rowcolor{Yellow!40} AH800 & (1) 의과, 치과 & 69.63 & \myexplfng{69.63} \\ \tabucline[.5pt]{-} %4,870원 \\ \tabucline[.5pt]{-}
	\rowcolor{Yellow!40} 11800 & (2) 한의과  & 67.16 & \myexplfng{67.16} \\ \tabucline[.5pt]{-} %  \\ \tabucline[.5pt]{-}
	\rowcolor{Yellow!40} AH900 & 마. 의원, 치과의원 & 69.63 & \myexplfng{69.63} \\ \tabucline[.5pt]{-} %5,180 \\ \tabucline[.5pt]{-}
	\rowcolor{Yellow!40} 11900 & 바. 한의원 & 67.16 & \myexplfng{67.16} \\ \tabucline[.5pt]{-} % \\ \tabucline[.5pt]{-}
	\end{tabu}
  	\end{enumerate}
  
\item[가-8-1] 집중영양치료료 Therapy by Nutrition Support Team 

\medskip
\tabulinesep =_2mm^2mm
\begin{tabu} to\linewidth {|X[2,l]|X[6,l]|X[1,l]|X[1,l]|} \tabucline[.5pt]{-}
\rowcolor{ForestGreen!40}  코드 &	\centering 분 류 & 점수 & 금액 \\ \tabucline[.5pt]{-}	
\rowcolor{Yellow!40} AI600 & 가. 상급종합병원 & 535.87 & \myexplfng{535.87} \\ \tabucline[.5pt]{-} %37,510 \\ \tabucline[.5pt]{-}
\rowcolor{Yellow!40} AI700 & 나. 종합병원 & 402.57 & \myexplfng{402.57} \\ \tabucline[.5pt]{-} %28,180 \\ \tabucline[.5pt]{-}
\end{tabu}

\item[가-9] AJ001(19001) 중환자실 입원료 ICU Patient Care
	\begin{enumerate}[가.]\tightlist
	\item 성인 또는 소아 중환자실 입원료 Adult or Pediatric \footnote{주:1.일반 중환자실에 전담의를 두는경우에는 272.06점을 별도 산정한다.} : 2015년 9월 신설 
	
	\medskip
	\tabulinesep =_2mm^2mm
	\begin{tabu} to\linewidth {|X[2,l]|X[6,l]|X[2,l]|} \tabucline[.5pt]{-}
	\rowcolor{ForestGreen!40}  코드 &	\centering 분 류 & 점수  \\ \tabucline[.5pt]{-}	
	\rowcolor{Yellow!40} AJ003 (19003) & 2. 중환자실 1Unit당 1인 이상의 전문의를 포함하여 전담의를 두는 경우에는  & 421.71점을 별도 산정한다.  \\ \tabucline[.5pt]{-}
	\rowcolor{Yellow!40} AJ100 (19400) & (1) 상급종합병원 & 2,648.30점   \\ \tabucline[.5pt]{-}
	\rowcolor{Yellow!40} AJ200 (19200) & (2) 종합병원 & 1,545.86점 \\ \tabucline[.5pt]{-}
	\rowcolor{Yellow!40} AJ300 & (3) 병원, 치과병원, 한방병원 내 의\cntrdot{}치과 & 1,133.68점   \\ \tabucline[.5pt]{-}
	\rowcolor{Yellow!40} 19300 & (4) 한방병원, 병원\cntrdot{}치과병원 내 한의과 & 1,128.39점  \\ \tabucline[.5pt]{-}
	\end{tabu}
	
	\item 신생아 중환자실 입원료 Neonatal \footnote{주:신생아 중환자실에는 전담전문의를 두어야 한다.}(2015년 9월 신설) 
	
	\medskip
	\tabulinesep =_2mm^2mm
	\begin{tabu} to\linewidth {|X[2,l]|X[6,l]|X[1,l]|X[1,l]|} \tabucline[.5pt]{-}
	\rowcolor{ForestGreen!40}  코드 &	\centering 분 류 & 점수 & 금액 \\ \tabucline[.5pt]{-}	
	\rowcolor{Yellow!40} AJ101 & (1) 상급종합병원 & 4,074.04 & \myexplfng{4074.04} \\ \tabucline[.5pt]{-} %285,180 \\ \tabucline[.5pt]{-}
	\rowcolor{Yellow!40} AJ201 & (2) 종합병원 & 3,755.24 & \myexplfng{3755.24} \\ \tabucline[.5pt]{-} %262,870 \\ \tabucline[.5pt]{-}
	\rowcolor{Yellow!40} AJ301 & (3) 병원, 치과병원\cntrdot{} 한방병원 내 의과 & 3,025.80 & \myexplfng{3025.80} \\ \tabucline[.5pt]{-} %211,810 \\ \tabucline[.5pt]{-}
	\end{tabu}
	
	\item 소아 중환자실 입원료 Pediatric \footnote{주 : 1. 별도의 Unit으로 운영하는 소아 중환자실에서 만18세 미만 소아청소년을 입원 치료한 경우에 산정한다. } 
	
	\medskip
	\tabulinesep =_2mm^2mm
	\begin{tabu} to\linewidth {|X[2,l]|X[6,l]|X[2,l]|} \tabucline[.5pt]{-}
	\rowcolor{ForestGreen!40}  코드 &	\centering 분 류 & 점수  \\ \tabucline[.5pt]{-}	
	\rowcolor{Yellow!40} AJ004 (19004) & 2. 중환자실 1Unit 당 1인 이상의 전담의를 두는 경우에는 & 272.06 점을 별도 산정한다.  \\ \tabucline[.5pt]{-}	
	\rowcolor{Yellow!40} AJ005 (19005) & 3. 중환자실 1Unit 당 1인 이상의 전문의를 포함하여 전담의를 두는 경우에는 & 421.71점을 별도산정한다.  \\ \tabucline[.5pt]{-}	
	\rowcolor{Yellow!40} AJ102 (19402) & (1) 상급종합병원 &  3,442.79점  \\ \tabucline[.5pt]{-}	
	\rowcolor{Yellow!40} AJ202 (19202) & (2) 종합병원 &  2,311.06점  \\ \tabucline[.5pt]{-}	
	\rowcolor{Yellow!40} AJ302 & (3) 병원, 치과병원, 한방병원 내 의.치과 &  1,694.85점 \\ \tabucline[.5pt]{-}	
	\rowcolor{Yellow!40} 19302 & (4) 한방병원, 병원\cntrdot{}치과병원 내 한의과 &  1,686.95점   \\ \tabucline[.5pt]{-}	
	\end{tabu}
	\end{enumerate}

\item[가-10] 격리실 입원료 Isolation Room Patient Care
	\begin{enumerate}[가.]\tightlist
	\item 일반 격리실 입원료
		\begin{enumerate}[(1)]\tightlist
		\item 상급종합병원 
		
		\medskip
		\tabulinesep =_2mm^2mm
		\begin{tabu} to\linewidth {|X[2,l]|X[6,l]|X[1,l]|X[1,l]|} \tabucline[.5pt]{-}
		\rowcolor{ForestGreen!40}  코드 &	\centering 분 류 & 점수 & 금액 \\ \tabucline[.5pt]{-}	
		\rowcolor{Yellow!40} AK100 & (가) 1인용 & 3,048.78 & \myexplfng{3048.78} \\ \tabucline[.5pt]{-} %213,410 \\ \tabucline[.5pt]{-}
		\rowcolor{Yellow!40} AK101 & (나) 다인용 & 1,219.51 & \myexplfng{1219.51} \\ \tabucline[.5pt]{-} %85,370 \\ \tabucline[.5pt]{-}
		\end{tabu}
		
		\item 종합병원 
		
		\medskip
		\tabulinesep =_2mm^2mm
		\begin{tabu} to\linewidth {|X[2,l]|X[6,l]|X[1,l]|X[1,l]|} \tabucline[.5pt]{-}
		\rowcolor{ForestGreen!40}  코드 &	\centering 분 류 & 점수 & 금액 \\ \tabucline[.5pt]{-}	
		\rowcolor{Yellow!40} AK200 & (가) 1인용 & 2,236.00 & \myexplfng{2236.00} \\ \tabucline[.5pt]{-} %156,520 \\ \tabucline[.5pt]{-}
		\rowcolor{Yellow!40} AK201 & (나) 다인용 & 1,118.00 & \myexplfng{1118.00 } \\ \tabucline[.5pt]{-} %78,260 \\ \tabucline[.5pt]{-}
		\end{tabu}
		
		\item 병원, 치과병원\cntrdot{} 한방병원 내 치\cntrdot{}의과 
		
		\medskip
		\tabulinesep =_2mm^2mm
		\begin{tabu} to\linewidth {|X[2,l]|X[6,l]|X[1,l]|X[1,l]|} \tabucline[.5pt]{-}
		\rowcolor{ForestGreen!40}  코드 &	\centering 분 류 & 점수 & 금액 \\ \tabucline[.5pt]{-}	
		\rowcolor{Yellow!40} AK300 & (가) 1인용 & 1,350.72 & \myexplfng{1350.72} \\ \tabucline[.5pt]{-} %94,550 \\ \tabucline[.5pt]{-}
		\rowcolor{Yellow!40} AK301 & (나) 다인용 & 900.48 & \myexplfng{900.48} \\ \tabucline[.5pt]{-} %63,030 \\ \tabucline[.5pt]{-}
		\end{tabu}
		
		\item 의원, 치과의원, 보건의료원 치\cntrdot{}의과 
		
		\medskip
		\tabulinesep =_2mm^2mm
		\begin{tabu} to\linewidth {|X[2,l]|X[6,l]|X[1,l]|X[1,l]|} \tabucline[.5pt]{-}
		\rowcolor{ForestGreen!40}  코드 &	\centering 분 류 & 점수 & 금액 \\ \tabucline[.5pt]{-}	
		\rowcolor{Yellow!40} AK400 & (가) 1인용 & 1,169.45 & \myexplfng{1169.45} \\ \tabucline[.5pt]{-} %87,010 \\ \tabucline[.5pt]{-}
		\rowcolor{Yellow!40} AK401 & (나) 다인용 & 779.63 & \myexplfng{779.63} \\ \tabucline[.5pt]{-} %58,000 \\ \tabucline[.5pt]{-}
		\end{tabu}
  		\end{enumerate}
  		
	\item 음압 격리실 입원료 \par
		\begin{enumerate}[(1)]\tightlist
		\item 상급종합병원 
		
		\medskip
		\tabulinesep =_2mm^2mm
		\begin{tabu} to\linewidth {|X[2,l]|X[6,l]|X[1,l]|X[1,l]|} \tabucline[.5pt]{-}
		\rowcolor{ForestGreen!40}  코드 &	\centering 분 류 & 점수 & 금액 \\ \tabucline[.5pt]{-}	
		\rowcolor{Yellow!40} AK110 & (가) 1인용 & 4,573.18 & \myexplfng{4573.18} \\ \tabucline[.5pt]{-} %320,120 \\ \tabucline[.5pt]{-}
		\rowcolor{Yellow!40} AK111 & (나) 다인용 & 1,829.27 & \myexplfng{1829.27} \\ \tabucline[.5pt]{-} %128,050 \\ \tabucline[.5pt]{-}
		\end{tabu}
		
		\item 종합병원 
		
		\medskip
		\tabulinesep =_2mm^2mm
		\begin{tabu} to\linewidth {|X[2,l]|X[6,l]|X[1,l]|X[1,l]|} \tabucline[.5pt]{-}
		\rowcolor{ForestGreen!40}  코드 &	\centering 분 류 & 점수 & 금액 \\ \tabucline[.5pt]{-}			
		\rowcolor{Yellow!40} AK210 & (가) 1인용 & 2,683.20 & \myexplfng{2683.20} \\ \tabucline[.5pt]{-} %187,820 \\ \tabucline[.5pt]{-}
		\rowcolor{Yellow!40} AK211 & (나) 다인용 & 1,341.60 & \myexplfng{1341.60} \\ \tabucline[.5pt]{-} %93,910 \\ \tabucline[.5pt]{-}
		\end{tabu}
		
		\item 병원, 치과병원\cntrdot{} 한방병원 내 치\cntrdot{}의과 
		
		\medskip
		\tabulinesep =_2mm^2mm
		\begin{tabu} to\linewidth {|X[2,l]|X[6,l]|X[1,l]|X[1,l]|} \tabucline[.5pt]{-}
		\rowcolor{ForestGreen!40}  코드 &	\centering 분 류 & 점수 & 금액 \\ \tabucline[.5pt]{-}			
		\rowcolor{Yellow!40} AK310 & (가) 1인용 & 1,485.80 & \myexplfng{1485.80} \\ \tabucline[.5pt]{-} %104,010 \\ \tabucline[.5pt]{-}
		\rowcolor{Yellow!40} AK311 & (나) 다인용 & 990.53 & \myexplfng{990.53} \\ \tabucline[.5pt]{-} %69,340 \\ \tabucline[.5pt]{-}
		\end{tabu}
		
		\item 의원, 치과의원, 보건의료원 치\cntrdot{}의과 
		
		\medskip
		\tabulinesep =_2mm^2mm
		\begin{tabu} to\linewidth {|X[2,l]|X[6,l]|X[1,l]|X[1,l]|} \tabucline[.5pt]{-}
		\rowcolor{ForestGreen!40}  코드 &	\centering 분 류 & 점수 & 금액 \\ \tabucline[.5pt]{-}			
		\rowcolor{Yellow!40} AK410 & (가) 1인용 & 1,286.39 & \myexplfng{1286.39 } \\ \tabucline[.5pt]{-} %95,710 \\ \tabucline[.5pt]{-}
		\rowcolor{Yellow!40} AK411 & (나) 다인용 & 857.59 & \myexplfng{857.59} \\ \tabucline[.5pt]{-} %63,800 \\ \tabucline[.5pt]{-}
		\end{tabu}
		\end{enumerate}
  	\end{enumerate}
  	
\item[가-10-1] 납차폐특수치료실 입원료(Lead-Shielded Room Patient Care) \footnote{주:방사성옥소를 이용한 개봉선원치료를 위하여 원자력 안전법령에 의한 시설을 갖춘 요양기관에서 납으로 차폐된 특수치료실에서 관리하는 경우 산정한다. (2015년 9월 수가변경)} 

\medskip
\tabulinesep =_2mm^2mm
\begin{tabu} to\linewidth {|X[2,l]|X[6,l]|X[2,l]|} \tabucline[.5pt]{-}
\rowcolor{ForestGreen!40}  코드 &	\centering 분 류 & 점수  \\ \tabucline[.5pt]{-}	
\rowcolor{Yellow!40} AQ600 & 가. 상급종합병원 & 3,895.58점   \\ \tabucline[.5pt]{-}
\rowcolor{Yellow!40} AQ700 & 나. 종합병원 & 3,561.64점   \\ \tabucline[.5pt]{-}
\rowcolor{Yellow!40} AQ800 & 다. 병원, 치과병원, 한방병원 내 의\cntrdot{}치과 &  1,833.17점   \\ \tabucline[.5pt]{-}
\rowcolor{Yellow!40} AQ900 & 라. 의원, 치과의원, 보건의료원 내 의\cntrdot{}치과 & 1,571.26점   \\ \tabucline[.5pt]{-}
\end{tabu}

\item[가-17] 회복관리료 Fee of Postanesthesia Care(2015년 9월 신설) 

\medskip
\tabulinesep =_2mm^2mm
\begin{tabu} to\linewidth {|X[2,l]|X[6,l]|X[2,l]|} \tabucline[.5pt]{-}
\rowcolor{ForestGreen!40}  코드 &	\centering 분 류 & 점수  \\ \tabucline[.5pt]{-}	
\rowcolor{Yellow!40} AP501 & 가. 상급종합병원 & 313.36점   \\ \tabucline[.5pt]{-}
\rowcolor{Yellow!40} AP601 & 나. 종합병원 & 288.38점  \\ \tabucline[.5pt]{-}
\rowcolor{Yellow!40} AP701 & 다. 병원, 치과병원, 요양병원, 한방병원 & 252.61점  \\ \tabucline[.5pt]{-}
\rowcolor{Yellow!40} AP801 & 라. 의원, 치과의원, 보건의료원 & 215.32점   \\ \tabucline[.5pt]{-}
\end{tabu}

	\begin{Cdoing}{가17 회복관리료(Fee of Postanesthesia Care) 인정기준(2015년 9월 신설)}
회복관리료는 아래와 같은 요건을 모두 충족한 회복실에서 회복관리를 시행한 경우 인정함
	\begin{enumerate}[가.]\tightlist
	\item 산정기준
		\begin{enumerate}[(1)]\tightlist
		\item 인력
			\begin{enumerate}[(가)]\tightlist
			\item 회복실의 회복관찰 업무를 총괄하는 상근하는 마취통증의학과 전문의가 1인 이상
			\item 회복실 내 환자 회복관리 업무만을 전담하는 간호사가 2인 이상 (정규직 전일제 근무 간호사로 1주간의 근로시간이 월평균 40시간인 근무자를 말함)
			\end{enumerate}
		\item 장비
			\begin{enumerate}[(가)]\tightlist
			\item 회복실내에 반드시 갖추어야 하는 장비 - 병상당 기본시설(산소공급장치, 흡인기), - 모니터링 장비: 말초산소포화도측정기, 심전도감시기, 비침습적 혈압측정기, 호기말이산화탄소분압감시기, - 체온조절기, - 호흡보조 장비 등(Nasal prong, Facial Mask, Ambu bag set), - 응급장비(기도삽관 장비 일체) 
			\item 필요시 즉시 사용가능하도록 수술실 또는 회복실에 갖추어야 하는 장비 - Emergency Cart, - 인공호흡기, - 제세동기
			\end{enumerate}
		\end{enumerate}
	\item 산정대상 : 바2가(1) 기관내 삽관에 의한 폐쇄순환식 전신마취 또는 바2가(2) 마스크에 의한 폐쇄순환식 전신마취 후 회복관리만을 목적으로 별도로 설치된 회복실에서 15분 이상 집중 회복관리를 한 경우
	\item 기타 : 회복관리가 종료되기 전에 출혈 등의 이유로 재수술 후 회복실에 다시 입실하여 회복관리가 이루어진 경우에는 회복관리료는 1회만 산정함
	\end{enumerate}
	\end{Cdoing}
\end{description}

\begin{hemphsentense}{3대 비급여 보장성 강화 (비급여 축소)에 따른 풍선 수가신설(보존책)}
\begin{enumerate}[1)]\tightlist
\item 2015년 9월부터 선택의사 지정비율을 80\% \MVRightarrow 67\%(2015년) \MVRightarrow 33\%(2016년) \MVRightarrow 비급여 선택진료 폐지, 건강보험적용(2017년)
\item 2014년 9월부터 4인실과 5인실 건강보험 적용 : 2015년부터 대형병원 의무 병상을 50\% \MVRightarrow 70\%(다만, 종합병원중 산부인과 전문병원은 현행 50\% 유지)
\end{enumerate}
\end{hemphsentense}

\begin{mdframed}[linecolor=blue,middlelinewidth=2]  
제1편 행위 급여 ․ 비급여 목록 및 급여 상대가치점수 >> 제1부 행위 급여 일반원칙 >>  5.의료질평가지원금(2015년 9월 신설)
\end{mdframed}
\begin{enumerate}[5.]\tightlist
\item  \uline{의료질평가지원금(2015년 9월 신설)}
	\begin{enumerate}[가.]\tightlist
	\item \uline{「의료질평가지원금 산정을 위한 기준」의 평가결과에 따라 상급종합병원, 종합병원에 한하여} 3개 분야(의료 질과 환자안전․공공성․의료전달체계 분야, 교육수련 분야, 연구개발 분야)별 \uline{최종등급에 해당하는 소정점수를 산정한다. }
	\item 입원진료의 경우에는 각 분야의 등급별 의료질평가지원금을 아래 항목의 산정횟수와 동일하게 산정한다. 다만, 입원료 중 병원관리료만을 산정하는 경우에는 제외한다.
		\begin{enumerate}[(1)]\tightlist
		\item 입원료(가-2)		\item 무균치료실 입원료(가-4)
		\item 낮병동 입원료(가-6)
		\item 신생아 입원료(가-7가, 가-7나)
		\item 중환자실 입원료(가-9), 다만 AJ001, AJ003, AJ004, AJ005, 19001, 19003, 19004, 19005는 제외
		\item 격리실 입원료(가-10)
		\item 납차폐특수치료실 입원료(가-10-1)
		\end{enumerate}
	\item 외래진료의 경우에는 각 분야의 등급별 의료질평가지원금을 외래환자 진찰료(가-1)의 산정횟수와 동일하게 산정하되, 재진진찰료(가-1나)의 “주6” 및 “주8”은 제외한다.
	\end{enumerate}  
\end{enumerate}

