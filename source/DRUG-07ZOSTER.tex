\section{대상포진약물}
\subsection{\newindex{Gabapentin 경구제}\label{GabepentinOral} (품명: 뉴론틴 등)}
허가사항 범위 내에서 아래와 같은 기준으로 투여 시 요양급여를 인정하며, 동 인정기준 이외에는 약값 전액을 환자가 부담토록 함.\par
- 아 래 -
\begin{enumerate}[1.]\tightlist
\item 간질(Epilepsy): 각 약제의 허가사항 범위 내 인정
\item 신경병성 통증(Neuropathic pain) 중 다음 각호 중 하나에 해당하는 경우
\end{enumerate}
- 다 음 -
\begin{enumerate}[가.]\tightlist
\item 당뇨병성 말초 신경병증(Diabetic neuropathy)
	\begin{enumerate}[1)]\tightlist
	\item  Thioctic acid(또는 α-lipoic acid) 경구제와 병용투여 시 Gabapentin 경구제의 약값 전액을 환자가 부담토록 함. 
	\item  당뇨병성 말초 신경병증성 통증치료제(예 : Pregabalin 경구제, Duloxetine 경구제 등)간의 병용투여는 인정하지 아니함.
	\end{enumerate}
\item 대상포진 후 신경통(Post-herpetic neuralgia) : Lidocaine 패취제(품명: 리도탑카타플라스마)와 병용투여 시 저렴한 약제의 약값 전액을 환자가 부담토록 함.
\item 척수손상에 따른 신경병증성 통증(Spinal cord injury)
\item 복합부위 통증증후군(CRPS, Compelx Regional Pain Syndrome)
\item 다발성 경화증(Multiple sclerosis), 파브리병 (Fabry's disease)
\item 척추 수술후 통증증후군(Post spinal surgery syndrome)
\item 절단 등으로 인한 신경병성 통증(환상통, 단단통)
\item 삼차신경통(1차적으로 다른 약제에 반응하지 않거나 부작용으로 인해 사용하기 어려운 경우)
\item 암성 신경병증성 통증 (건강보험심사평가원장이 공고한「암성통증 관련 사용 권고안」참조 인정)
\item 인간면역결핍바이러스(HIV) 감염인의 신경병성 통증
\end{enumerate}
* 시행일: 2013.9.1.
* 종전고시: 고시 제2009-59호(2009.4.1.)
* 변경사유: 용어정비

\subsection{\newindex{Pregabalin 경구제}\label{PregabalinOral}(품명: 리리카캡슐 등)}
각 약제별 허가사항 범위 내에서 아래와 같은 기준으로 투여 시 요양급여 함을 원칙으로 하며, 동 인정기준 이외에는 약값 전액을 환자가 부담토록 함. \par
- 아 래 -
\begin{enumerate}[가.]\tightlist
\item 간질(Epilepsy) : 허가사항 범위 내에서 인정
\item 신경병성 통증
	\begin{enumerate}[1)]\tightlist
	\item  당뇨병성 말초 신경병증성 통증 
		\begin{enumerate}[가)]\tightlist
		\item Thioctic acid(또는 α-lipoic acid) 경구제와 병용투여 시 Pregabalin 경구제 약값 전액을 환자가 부담토록 함. 
		\item 당뇨병성 말초 신경병증성 통증치료제(예: Gabapentin 경구제, Duloxetine 경구제 등)간의 병용투여는 인정하지 아니함
		\end{enumerate}
	\item 대상포진 후 신경통 Lidocaine 패취제(품명: 리도탑패취)와 병용투여 시 투약비용이 저렴한 약제의 약값 전액을 환자가 부담토록 함.
	\item 척수손상에 따른 신경병증성 통증(Spinal cord injury)
	\item 복합부위 통증증후군(CRPS; Compelx Regional Pain Syndrome)
	\item 암성 신경병증성 통증 (「암환자에게 처방투여하는 약제에 대한 요양급여의 적용기준 및 방법에 관한 세부사항」의 ‘Ⅲ.암성통증치료제’ 범위 내에서 인정)
	\item 척추수술 후 통증증후군(Post spinal surgery syndrome)
	\end{enumerate}
\item 섬유근육통(Fibromyalgia)에는 다음과 같이 요양급여를 인정함.
\end{enumerate}
- 다 음 -
\begin{enumerate}[1)]\tightlist
\item 섬유근육통으로 확진되고 삼환계 항우울제(TCA : Amitriptyline, Nortriptyline 등) 또는 허가사항 중 근골격계 질환에 수반하는 동통의 증상완화에 사용할 수 있는 근이완제(Cyclobenzaprine 등)를 적어도 1달 이상 사용한 후에도 효과가 불충분한 경우
\item Duloxetine(품명: 심발타캡슐)과의 병용투여는 인정하지 아니함.
\end{enumerate}
※ 섬유근육통 확진은 2010년 미국 류마티스학회 발표 진단기준에 부합하고 섬유근육통 영향척도(FIQ;Fibromyalgia Impact Questionnaire) 점수가 40점 이상이며, 시각적 아날로그 동통 스케일(pain VAS; pain Visual Analog pain Scale)이 40mm 이상인 경우로 하며, 투여개시 13주 후 Pain VAS와 FIQ의 호전이 없는 경우 투여중단을 고려해야 함.\par
* 시행일: 2015.8.1
* 종전고시:고시 제2013-127호(2013.9.1)
* 변경사유:교과서, 가이드라인, 임상연구문헌 등에서 척추수술 후 발생한 신경병성 통증에 동 약제가 유의한 통증감소 효과를 보이며, Gabapentin에 비하여 우수한 효과를 나타낸 점 등을 고려하여 급여인정함. 
  
\subsection{\newindex{Lidocaine 패취제}\label{Lidocainpathy}(품명: 리도탑카타플라스마 등)}
대상포진 후 신경통증에는 허가사항 범위 내에서 투여 시 요양급여를 인정하며, Gabapentin 경구제(품명 : 뉴론틴캅셀 등) 또는 Pregabalin 경구제(품명 : 리리카캅셀)와 병용투여 시는 아래와 같이 요양급여를 인정함. \par
- 아 래 -
\begin{itemize}[○]\tightlist
\item 병용 약제 중 투약비용이 저렴한 약제의 약값 전액을 환자가 부담토록 함.
\end{itemize}  
* 시행일: 2013.9.1.
* 종전고시: 고시 제2011-163호(2012.1.1.)

\subsection{\newindex{Famciclovir}\label{FamciclovirOral} 250mg 경구제 (품명 : 팜비어정 250mg 등)}
1. 허가사항 범위 내에서 아래와 같은 기준으로 투여 시 요양급여를 인정하며, 동 인정기준 이외에는 유사효능의 투약비용이 저렴한 약제 투여 후 효과가 없는 경우에 한하여 약값 전액을 환자가 부담토록 함.\par
- 아 래 -\par
○ 대상포진 감염증과 재발성인 생식기포진 감염증의 치료\par
2. 허가사항 범위(효능ㆍ효과 등)를 초과하여 아래와 같은 기준으로 투여 시 약값 전액을 환자가 부담토록 함.\par
- 아 래 -\par
○ 단순포진에 의한 재발성 각막감염시 기존의 Antiviral agent(Acyclovir)로 치료에 실패하였거나, 투여가 어려운 경우\par
* 시행일: 2013.9.1.
* 종전고시: 고시 제2009-73호(2009.4.24.)
