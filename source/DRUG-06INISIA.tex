\section{\newindex{이니시아}, INISIA[Uripristal acetate] 자궁근종 치료제}
\colorbox{red}{주상병}
\begin{itemize}\tightlist
\item D25 자궁내 평활근종
​\item D250 자궁의 점막하 평활근종
\item D251 자궁의 벽내 평활근종
\item D252 자궁의 장막하 평활근종
\item D259  상세불명의 자궁의 평활근종
\end{itemize}
 
\colorbox{red}{보조상병}
\begin{itemize}\tightlist
\item 출혈관련
	\begin{enumerate}\tightlist
	\item N921 불규칙적 주기를 가진 과다 및 빈발월경
	\item N938  기타 명시된 이상 자궁 및 질 출혈
	\end{enumerate}	
\item 통증관련
	\begin{enumerate}\tightlist
	\item N940 중간통
	\item R103 기타 하복부에 국한된 동통
	\item R520 급성통증
	\item N945 속발성 월경통
	\end{enumerate}	
\item 불임
	\begin{enumerate}\tightlist
	\item N978 기타 요인에서 기원한 여성불임
	\end{enumerate}
\end{itemize}

\colorbox{red}{청구메모}
가임기 여성(18세 - 폐경전)으로 자궁근종의 크기 감소와 증상개선을 목적으로  수술전 약물 투여함\\

\subsection{투여방법}
\begin{enumerate}\tightlist
\item 2014년 6월25일 식약청 의약품 수입품목 허가사항 변경
\item 1일 1회 1정[5mg]을 3개월까지 연속 경구 투여
\item 투약은 월경시작 1주 이내에 시작
\item 3개월연속 투여는 한번더 반복 될 수 있슴
\item 첫 3개월 연속 투여 종료 후 첫번째 월경을 온전히 지내고 두번째 월경이 시작되면 일주일 이내에 재투여
\item 2cycle 이상은 안정성 확보 안됨
\end{enumerate}

\subsection{청구시 주의사항}
\textcolor{red}{자궁근종 상병1 + 보조상병중 출혈관련1+ 보조상병중 통증관련1+ 청구메모 }
 기입을 해야만 \emph{삭감이 없습니다. 꼭 유념..}
 
\subsection{청구사례}
\begin{itemize}[■]\tightlist
\item 청구내역
	\begin{itemize}[○]\tightlist
	\item 상병명(53세/여)(외래 1일) : 상세불명의 자궁의 평활근종 
	\item 주요청구내역 : 이니시아정 0.04*1*28(2014.8.7.) ▶ 인정
	\end{itemize}`

\item 진료내역
	\begin{itemize}[○]\tightlist
	\item 경과기록: 참기 힘든 생리통이 있고 생리량이 많으며, 빈혈 있음
	\item 검사내역: 2014.8.7.: Hgb 8.4g/dl, 초음파: D1 4.93cm  
	\end{itemize}		 

\item 심사결과
\emph{ <결  과>}\par 
 53세 연령으로 규칙적인 월경을 하고 있으며 자궁초음파영상에서 4.9cm의  
 자궁근종이 확인되었고 경과기록 상 ‘참기 힘든 생리통’과 ‘생리량 과다’등의 
 증상 및 중등증의 빈혈소견(Hgb 8.4g/dl )이 확인되어 수술전 빈혈치료 목적으로 
 투여한 경우이므로, 이니시아정은 인정함\par

 \emph{ <사  유>}\par 
	\begin{itemize}[○]\tightlist
	\item 이니시아정은 식품의약품안전처 허가사항 중「가임기 성인 여성에서 
    중등도-중증 증상을 가진 자궁근종 환자의 수술전 치료」에 효능·효과 있음
	\item 세계보건기구(WHO)에서는 빈혈을 혈색소치의 정도에 따라 경증(9g/dl이상),
    중등증(6-9g/dl), 심증(6g/dl미만)으로 분류하였음 
	\end{itemize}
\item 참고
	\begin{itemize}[○]\tightlist
	\item 식품의약품안전처의 허가사항 
	\item  세계보건기구(WHO)-빈혈
	\end{itemize}
\end{itemize}
 
\begin{itemize}[■]\tightlist
\item 청구내역
	\begin{itemize}[○]\tightlist
	\item 상병명(53세/여)(외래 1일)
    상세불명의 자궁의 평활근종 
	\item 주요청구내역
    이니시아정 0.04*1*28(2014.8.26.) ▶ 조정  
	\end{itemize}

\item 진료내역
	\begin{itemize}[○]\tightlist
	\item 경과기록: 생리통 +++, 생리량 +++
	\item 검사내역 
    2014.6.30. CT Abdomen-Pelvis: r/o adenomyosis 
	\end{itemize}
\item 심사결과
 \emph{<결  과> }\par
‘상세불명의 자궁의 평활근종’ 진단명으로 청구되었으나, 자궁근종으로 인한 
  증상이 불명확하고  월경과다로 인한 빈혈소견 등 근종 치료 사유가 확인되지 않으며 
  복부CT에서도 R/O 자궁선근종(adenomyosis)으로 확인되므로 이니시아정은 인정  
  하지 아니함

 \emph{<사  유> } \par
 이니시아정은 식품의약품안전처 허가사항 중「가임기 성인 여성에서 중등도-
 중증 증상을 가진 자궁근종 환자의 수술전 치료」에 효능·효과 있음 

\item 참고
	\begin{itemize}[○]\tightlist
	\item  식품의약품안전처의 허가사항 
	\end{itemize}
\end{itemize} 
 
\begin{itemize}[■]\tightlist
\item 청구내역
	\begin{itemize}[○]\tightlist
	\item 상병명(47세/여)(재진외래 1일)
    상세불명의 자궁의 평활근종 
	\item 주요청구내역
    이니시아정 0.04*1*28(2014.8.27.) ▶ 조정  
	\end{itemize}
\item 진료내역
	\begin{itemize}[○]\tightlist
	\item 경과기록:
		\begin{itemize}[-]\tightlist
		\item 2014.7.30.: 생리통+, sono myoma 1~3cm 6개 정도 있음, 미레나 삽입
                 4주후 내원하여 미레나 확인 
		\item 2014.8.27.: 생리양 줄었음, IUD(+)
                이니시아정 => 현청구분
		\end{itemize}		
	\item검사내역 
  - 2014.7.30. 초음파: D1 9.09cm, D2 6.97cm 
	\end{itemize}
\item 심사결과
  \emph{<결  과> }\par
 미레나는 월경과다증, 월경곤란증, 에스트로겐 대체요법시 프로게스틴의 국소
 적용시 치료목적으로 자궁내 삽입(5년 효과)하는 약제이고, 
 이니시아정은 중등도-중증 증상(월경과다, 월경통 등)을 가진 자궁근종 환자에서
 수술전 치료로 투여하는 약제이므로, 두 약제는 자궁근종과 관련된 월경과다,
 월경통 치료에 동일목적으로 사용되고 있음.
 따라서, 이미 자궁내 장치(미레나)를 통해 생리량이 줄었음에도 불구하고 동일
 목적의 약제를 병용 투여한 경우이므로 이니시아정은 인정하지 아니함  \par

 \emph{<사  유>}\par
 이니시아정은 식품의약품안전처 허가사항 중「가임기 성인 여성에서 중등도-
 중증 증상을 가진 자궁근종 환자의 수술전 치료」에 효능·효과가 있고 
 미레나는 피임, 월경과다증, 월경곤란증, 에스트로겐 대체요법시 프로게스틴의 
 국소적용에 효능·효과가 있음 

\item 참고
	\begin{itemize}[○]\tightlist
	\item 식품의약품안전처의 허가사항
	\end{itemize}
\end{itemize}