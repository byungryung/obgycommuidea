\section{초음파 급여화 \textsf{QA}}
\textsf{임신 제 1 삼분기, 제 2,3 삼분기 기준은?}
\begin{commentbox}{}
임신 제 1 삼분기는 착상부터 임신 13 주까지 (13 주 6 일까지 ), 제 2,3 삼분기는 임신 14 주부터 출산 시까지 해당됨. \par \noindent
\textsf{문제점은 현 KCD-7으로 상병명을 코딩한 경우}는 z3480 : 기타 정상임신의 관리, 임신 22주 미만, z3481 : 기타 정상임신의 관리, 임신 22주 이상 - 34주 미만, z3482 : 기타 정상임신의 관리, 임신 34주 이상으로 상병명으로 구분할수 없다.
\end{commentbox}

\textsf{임산부 초음파 산정 시 기재사항}
\begin{commentbox}{}
`'JT005"에 초음파검사 시 임신주수를 기재함.\par \noindent JT005 에 산정횟수 기재로 봐서,,, 청구 프로그램에 입력후 청구하면 월별로해서 환자당 초음파 보험적용 횟수가 합산되나 봅니다!!
\end{commentbox}

\textsf{「산전진찰 목적으로 시행하는 검사의 요양급여 범위」 에서 비급여대상검사의 유전학적 양수검사시 유도초음파는 ?}
\begin{commentbox}{}
유전학적 양수검사는 비급여 대상 검사이므로 이에 시행하는 유도초음파도 비급여임 .
\end{commentbox}

\textsf{특정내역구분코드란?}
\begin{commentbox}{}
JX999, JT005 특정내역구분코드입니다\par\noindent\url{http://www.obgydoctor.co.kr/xe/index.php?mid=m_faq&document_srl=24693}
\end{commentbox}

\textsf{산모인경우,야간이나 휴일에 응급으로 초음파 시행한경우, 야간, 휴일 가산이 추가 가능한가요?}
\begin{commentbox}{}
초음파 검사의 경우 제2장 검사료에 분류된 항목으로 야간이나 휴일 가산의 적용 대상이 아님을 알려드립니다.
\end{commentbox}

\textsf{임신중 각 주수당,보험횟수사가 정해져있고.,7회 보험적용으로 급여기준이 되어있는데.. 타 요양기관에서 보험적용 횟수를 초과한 경우, 본원에서 조회가 가능한가요? 
7회는 초과되어있지는 않지만 주수당 보험적용 횟수를 타 요양기관에서 초과한 경우 본원에서 보험급여가 가능한가요?}
\begin{commentbox}{}
초음파 검사 횟수는 요양기관 불문하고 해당 주수별로 정하고 있는 횟수별로 총 7회 건강보험적용이 되며 횟수를 초과한 사실이 확인된 경우에는 비급여로 산정하여야 합니다.
\end{commentbox}

\textsf{나-951 EB514 , EB518 주: 기형아(Anomaly)를 정밀 계측한 경우 에서 기형아(Anomaly)에 해당되는 경우는 어떠 경우(상병)인가요?}
\begin{commentbox}{}
EB514 주항 및 EB518 주항에 명시된 ‘기형아(Anomaly)’의 상병에 대하여는 따로 정하여 운영하고 있지는 않습니다. 다만, 이처럼 세부인정기준이 별도로 마련되어 있지 않은 항목에 대해서는 교과서적 범위 내에서 진단명 및 환자의 전반적인 상태 등에 따라 적절하게 실시하고 그 비용을 산정하시기 바라며, 다만, 심사 시 청구내역 및 의학적 타당성 등을 고려하여 급여여부를 판단하고 있음을 알려드리오니 너른 이해 바랍니다.
\end{commentbox}

\textsf{제한적 초음파는 어떠한 경우 산정하는가?}
\begin{commentbox}{}
치료 전\bullet 후와 같이 환자 상태변화를 확인하기 위하여 이전 초음파영상과 비교목적으로 시행할 경우 제한적초음파(해당 검사의 소정점수의 50\%)를 산정함.으로 되어있습니다.
\end{commentbox}

\textsf{각주수별 초과보험산정되는경우(출혈, 태동의 현저한 변화, 발열, 복통, 조기진통, 조기양막파수, 외상, 위해성 약물 노출, 태아이상, 분만예정일 초과 등) 주수별 추가 보험 산전초음파 검사하는경우에 일반 초음파 를 산정하나요? 제한적 초음파 산정을 하나요?}
\begin{commentbox}{}
일반 또는 일반의 제한적 초음파의 범위는 별도로 정하고 있지 않으며, 환자의 개별상태에 대한 의학적 판단에 따라 유선으로 (16.9.27.)내 한 바와 같이 사례별로 판단되어져야 하며, 관련 급여기준을 아래와 같이 안내하오니 업무에 참고하시기 바랍니다. 감사합니다.
\begin{itemize}\tightlist
\item 산전초음파 7회인정은 일반 초음파 100\% 산정가능하다,..7회 초과되는 경우 일반(정밀) , 제한적 초음파 선택은 병의원 몫이다, 산과관련 제한적 초음파 기준은 따로 없다
\item 태아심장에 이상이 있는경우는 EB436 태아정밀 심초음파를   산정하는 경우 추가 (추적검사)산정은 EB436001(제한적)태아정밀 심초음파로 하고,, 이는 임신중 7회에 포함되지 않는다!!
\end{itemize} 
\end{commentbox}