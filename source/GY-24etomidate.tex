\section{Etomidate 주사제 (품명:에토미데이트리푸로주)}
\myde{}{
\begin{itemize}
	\item 효능/효과 :  전신 마취 유도제.
	\item 용법/용량
	\begin{enumerate}\tightlist
		\item 성인: 에토미데이트로서 1회 체중 kg당 0.15-0.3mg을 정맥투여 한다. 
		\item 2세 이상 어린이부터 15세 이하의 청소년 및 고령자: 에토미데이트로서 1회 체중 kg당 0.15-0.2mg을 정맥투여 한다. 
		\item 연령, 증상에 따라 적절히 증감한다. 
		\item 반드시 정맥 투여하고 일반적으로 서서히(1회 투여 시간은 30초) 투여하고 필요에 따라 분할 투여한다
	\end{enumerate}
\end{itemize}
}
{
\leftrod{Etomidate 주사제 (품명:에토미데이트리푸로주)}\par
허가사항 범위 내에서 아래와 같은 환자에게 전신마취유도 목적으로 투여 시 요양급여를 인정하며, 동 인정기준 이외에는 약값 전액을 환자가 부담토록 함. 
\begin{center}\textbf{- 아        래 -}\end{center}

\begin{enumerate}[가.]\tightlist
\item 심혈관계 질환 
\item 반응성 기도질환(천식, 만성기관지염 등) 
\item 두개강내압 상승이 있는 경우.
\end{enumerate}
}
\textbf{에토미데이트 사용시에는 100/100} 으로 하십시요.\par
에토미데이트리프로주는 \textbf{따로 교육 필요 없습니다}.
\begin{commentbox}{경련방지책}
\begin{itemize}\tightlist
    \item 에토미 10cc 인데 천천히 포폴처럼 주면 되고 간단한 건 몸무게적으면 7cc정도 서서히 주시면 됩니다 단 경련이 심합니다 포폴보다
    \item 에토미데이트 처음에 1cc 천천히 주고 1분 기다렸다가 나머지  7cc를  천천히  주사하면됩니다. 빨리주면 경련이 심한데 천천히 주면 훨씬 경련이 적어요
    \item 에토미데이트 쓸때 바리움주사나 트라마돌 주사 IV로 먼저 주고 2-3분 지나서 에토미데이트 쓰면 경련이 거의 안 옵니다
    \item 에토미데이트 나쁘지 않습니다.경련 자주 있어요. 대부분 지켜보면 지나갑니다. 하지만 혹시해서 발륨을 준비해 놓는게 좋습니다. 그랜드말시져에 대비해서..
    \item 발륨은 향정입니다 향정이지만 주목을 크게 안받는 향정이죠. 
    \item 춘계학술대회 연자 왈 포폴 비해 호흡억제 심억제 거의 없으나 근경련이 30\% 있다네요
    \item 전 개인적으로 3.8 - 4.0 cc 사용합니다.  경련이 아주 미미합니다
    \item 저는 준비되면 에토미데이트 3cc 먼저주고 1분정도 기다리는 동안 글러브 끼고 슬슬 시간끌고 나서 나머지 주면 좋던데요. 거의 경련도 없구요.  그래도 부족할땐 펜토 3cc 씩 주고 있습니다.
    \item 그러려니 마음 먹고 줍니다. 전 6 - 7 cc  정도 줍니다. 체중이 많을수록 많이 주라는데 제 경험상 오히려 적은 양으로 될 때도 많습니다
    \item 에토미데이트 쓸때 트라마돌을 미리 정맥주사후 조금 지나서 에토미데이트를 천천히 주면 경련도 거의 없고 통증 조절도 잘됩니다.
\end{itemize}    
\end{commentbox}