\section{입원 중 협의진찰료 급여기준}
\Que{수술 시 마취통증의학과와 협의한 협의진찰료가 조정된 사유가 무엇인가요?}
\Ans{협의진찰료는 입원 중인 환자의 특별한 문제에 대한 평가 및 관리를 위하여 그 환자의 주치의가 아닌 다른 진료과목 [또는 세부전문과목(분야)] 의사의 견해나 조언을 얻는 경우 산정하며, 협의진료를 요청하는 특별한 문제 및 협의진료의사의 견해 등을 의무기록에 명시하여야 합니다. 특별한 문제가 없거나 일률적으로 마취통증의학과 협의진찰료를 산정하거나, 예방적 목적으로 협의했을 때에는 산정하면 안됩니다.}

\begin{commentbox}{입원 중 협의진찰료 급여기준}
\begin{enumerate}[1.]\tightlist
\item 산정기준
협의진찰료는 입원 중인 환자의 특별한 문제에 대한 평가 및 관리를 위하여 그 환자의 주치의가 아닌 다른 진료과목 [또는 세부전문과목(분야)] 의사의 견해나 조언을 얻는 경우 산정하며, 협의진료를 요청하는 특별한 문제 및 협의진료의사의 견해 등을 의무기록에 명시하여야 함
\item 진료과목당 산정횟수
	\begin{enumerate}[가.]\tightlist
	\item 상급종합병원, 상급종합병원에 설치된 치과대학부속치과병원 : 입원기간 중 30일에 5회 이내
다만, 중환자실 입원환자의 경우 환자상태 변화 등으로 인해 협진이 필요한 경우 추가산정 가능 (기본코드 다섯 번째 자리에 1로 기재)
	\item 종합병원, 상급종합병원에 설치된 경우를 제외한 치과대학부속치과병원
: 입원기간 중 30일에 3회 이내
	\item 병원·한방병원·치과병원
: 입원기간 중 30일에 2회 이내
	\item 요양병원·의원·한의원·치과의원·보건의료원
: 입원기간 중 30일에 1회
	\end{enumerate}
 고시 제2014-126호 (2014.08.01. 시행)
\end{enumerate}
\end{commentbox}

