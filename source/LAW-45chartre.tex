\section{기록 열람의 요건}
\leftrod{의료법시행규칙 제13조의2 (기록 열람 등의 요건)} 
\begin{itemize}[-]\tightlist
\item  환자의 배우자, 직계 존속·비속 또는 배우자의 직계 존속이 환자에 관한 기록의 열람이나 그 사본의 발급을 요청할 경우 확인 서류
	\begin{enumerate}\tightlist	
	\item 기록 열람이나 사본 발급을 요청하는 자의 신분증 사본
	\item 가족관계증명서, 주민등록표 등본 등 친족관계임을 확인할 수 있는 서류
	\item 환자가 자필 서명한 동의서 (환자가 만 14세 미만의 미성년자인 경우에는 제외)
	\item 환자의 신분증 사본 (환자가 만 17세 미만으로 주민등록증이 발급되지 아니한 경우에는 제외)
	\end{enumerate}
\item  환자가 지정하는 대리인이 환자에 관한 기록의 열람이나 그 사본의 발급을 요청할 경우의 제출 받아야할 서류
	\begin{enumerate}\tightlist	
	\item 기록열람이나 사본발급을 요청하는 자의 신분증 사본
	\item 환자가 자필 서명한 위임장 (만 14세 미만의 미성년자인 경우에는 환자의 법정대리인이 작성, 가족관계증명서 등 법정대리인임을 확인할 수 있는 서류를 첨부)
	\item 환자의 신분증 사본 (만 17세 미만으로 주민등록증이 발급되지 아니한 자는 제외)
	\end{enumerate}
\end{itemize}
형사사건 수사 협조를 위해 진료기록 사본을 의무적으로 제공해야 하는 경우라 하면,
\begin{itemize}[-]\tightlist
\item 법원이 압수 또는 제출을 명하거나, 검사 또는 사법경찰관이 지방법원판사가 발부한 영장에 의하여 압수, 수색 또는 검증을 하는 경우 (환자 본인 동의 불필요) 
\item 위의 방법이 아닌 일반적인 공문형태의 수사 협조 요청일 경우에는 의료인이 그 요청에 따를 의무는 없음
\item 입․퇴원 및 외래내원 여부 같은 환자의 행적, 연락처 등 긴급하게 수사에 필요하다고 판단되는 이외에 진료과목, 처치내용 등 질병 치료와 직접적으로 관계된 내역은 일반적으로 민감한 프라이버시에 해당되므로, 환자의 동의 없이 진료기록 사본을 임의로 제출하였다면 당사자가 「의료법」 및 「개인정보보호법」에 의거하여 소제기가 가능할 것임
\end{itemize}

