\section{골밀도 검사의 인정기준}
\myde{}{
\emph{다-334 골밀도검사 [재료대 포함] Bone Densitometry}
\begin{enumerate}[가.]\tightlist
\item 양방사선(광자) 골밀도검사 Dual-Energy(Photon) Absorptiometry
	\begin{enumerate}[(1)]\tightlist
	\item \textcolor{red}{HC341 1부위} \myexplfn{449.3} 원
	\item HC342 2부위 이상 \myexplfn{531.08} 원
	\end{enumerate}
\item 정량적 전산화단층골밀도검사
	\begin{enumerate}[(1)]\tightlist
	\item \textcolor{red}{HC343 QCT} \myexplfn{470.05} 원
	\item HC346 PQCT \myexplfn{470.05} 원
	\end{enumerate}
\item HC345 방사선흡수측정기 방식 \myexplfn{176.3} 원
\item HC34 기타 방법에 의한 것 [단광자 골밀도측정(SPA),양방사선 말단 골밀도측정(PDEXA), 단에너지 골밀도측정(SXA), \textcolor{red}{초음파 골밀도측정(QUS)] OtherMethods \myexplfn{284.00} 원}
\end{enumerate}
\emph{요양기관 장비신고 필요(심평원 홈페이지)}
}{
다334 골밀도검사의 인정기준은 다음과  같이함.  
                                                                                       
\begin{enumerate}[가.]\tightlist
\item 적응증
	\begin{enumerate}[(1)]\tightlist
	\item 65세 이상의 여성과 70세 이상의 남성
	\item 고위험 요소* 가 1개 이상 있는 65세 미만의 폐경후 여성
	\item 비정상적으로 1년 이상 무월경을 보이는 폐경전 여성
	\item 비외상성(fragility) 골절: 압박골절 
	\item 골다공증을 유발할 수 있는 질환이 있거나 약물을 복용중인 경우: 장기간의 스테로이드 치료를 받고 있는 환자 , 부갑상선기능 항진증, 갑상선기능 항진증 , 다발성 류마티스관절염환자  유방암수술후 에스트로겐제제 복용(페마라등)
	\item 기타 골다공증 검사가 반드시 필요한 경우: 대장암수술또는 기타암수술후 복부 항암치료등,이전 골밀도 검사가 골다공증인경우
    \item ※ 고위험요소
		\begin{enumerate}[1.]\tightlist
		\item 저체중(BMI < 18.5)
		\item 비외상성 골절의 과거력이 있거나 가족력이 있는 경우
		\item 외과적인 수술로 인한 폐경 또는 40세 이전의 자연 폐경:전자궁적출술(TAH),아전자궁적출술(SubTAH),복강경하 전자궁적출술(LAVH,TLH),양측난소절제술(BSO),조기폐경
		\end{enumerate}
	\end{enumerate}	
\item 산정횟수
    \begin{enumerate}[(1)]\tightlist
	\item 진단 시
         - 1회 인정하되, 말단골 골밀도검사 결과 추가검사의 필요성이 있는 경우 1회에 한하여 central bone(spine, hip)에서 추가검사 인정함.
	\item 추적검사
		\begin{enumerate}[(가)]\tightlist
		\item 추적검사의 실시간격은 1년 이상으로 하되, 검사 결과 정상골밀도로 확인된 경우는 2년으로 함.
		\item 치료효과 판정을 위한 추적검사는 central bone(spine, hip)에서 실시한 경우에 한하여 인정함.             
		\item 위 (가), (나)의 규정에도 불구하고 스테로이드를 3개월 이상 복용하거나 부갑상선기능항진증으로 약물치료를 받는 경우는 종전 골밀도검사 결과에 따라 아래와 같이 할 수 있으며, 이 경우 central bone(spine, hip)에서 시행함.\newline
         - 아    래 -
			\begin{itemize}[●]\tightlist
			\item 정상골밀도(T-score ≥ -1)인 경우
            : 첫 1년에 1회 측정, 그 이후부터는 2년에 1회
			\item T-score ≤ -3 인 경우
            : 첫 1년은 6개월에 1회씩, 그 이후부터는 1년에 1회
			\end{itemize}
		\end{enumerate}
	\end{enumerate}	
\end{enumerate}
}

\subsection{골다공증에 실시한 생화학적 골표지자 검사의 인정기준}
골다공증에 실시한 생화학적 \textcolor{red}{골표지자검사는 아래와 같은 경우에 골흡수표지자검사와 골형성표지자검사를 각 1종씩 인정함.}
\begin{enumerate}[가.]\tightlist
\item 골다공증 약물치료 시작 전 1회
\item 골다공증 약물치료 3~6개월 후 약제 효과 판정을 위해 실시 시 1회
\end{enumerate}
\begin{itemize}[*]\tightlist
\item 골흡수표지자
	\begin{itemize}[-]\tightlist
	\item 나-393(디옥시피리디놀린):\textcolor{red}{C3930}
	\item 나-393-1 (N-telopeptide of Collagen Type 1 (NTX)):C3931
	\item 나-393-2 (C-telopeptide of Collagen Type 1 (CTX)):C3932
	\end{itemize}
\item 골형성표지자
	\begin{itemize}[-]\tightlist
	\item 나-363(오스테오칼신):\textcolor{red}{C3630}
	\item 나-398(골특이성알카리성포스파타제):C3980
	\end{itemize}
\end{itemize} 

\subsection{너153 기타 비타민 (D3) 검사(CY155)의 인정기준}
너153 기타 비타민 검사 중 비타민 D (D2, D3 및 total D) 검사의 급여기준은 다음과 같이 함. \par
\begin{center}\emph{- 다  음 -}\end{center}
\begin{enumerate}[가.]\tightlist
\item 적응증
	\begin{enumerate}[1)]\tightlist
	\item 비타민 D 흡수장애를 유발할 수 있는 위장질환 및 흡수장애 질환
	\item 항경련제(Phenytoin 이나 Phenobarbital 등) 또는 결핵약제 투여 받는 환자
	\item 간부전, 간경변증
	\item 만성 신장병
	\item 악성종양
	\item 구루병
	\item \textcolor{red}{이차성 골다공증의 원인 감별이 필요한 경우}
	\item \textcolor{red}{골다공증 진단 후 약물치료 시작 전 1회, 비타민 D 투여 3-6개월 후 약제 효과 판정을 위해 실시 시 1회 인정함을 원칙으로 하되, 이 후 추적검사는 연 2회까지 인정}
	\item 체표면적 40 이상 화상
	\end{enumerate}
\item 기타
	\begin{enumerate}[1)]\tightlist
	\item 비타민 D (D2, D3 및 total D) 검사는 1종만 인정
	\item 선별 검사로 HPLC법(너153주1)은 인정하지 아니함
	\end{enumerate}
\end{enumerate}

\Que{골밀도검사 후 골다공증약을 투약하는 경우 \textcolor{red}{1년 뒤 fu 검사}를 한 뒤 투약지속여부를 판단하게 되는데요,골밀도검사를 1년 단위로 딱 맞추기가 어렵습니다.어디선가 1년보다 며칠 앞당겨 fu 검사를 해도 괜찮다는 말을 들은 적이 있는데 정말 그런가요? 아니면 \textcolor{red}{반드시 1년 경과 후 추적검사를 해야 하나요?}}
\Ans{
골밀도검사는「골밀도검사의 인정기준(고시 제2007-92호, ’07.11.1. 시행)」에 의거, 적응증과 산정횟수를 정하고 있으며 이에 추적검사는 1년 이상으로 실시간격을 두되 검사결과 정상 골밀도로 확인된 경우에는 2년으로 하도록 명시되어 있습니다.\par
또한, 스테로이드를 3개월 이상 복용하거나 부갑상선기능항진증으로 약물치료를 받는 경우는 종전 골밀도검사 결과에 따라 정상 골밀도인 경우 첫 1년에 1회, 그 이후부터는 2년에 1회 시행하고 T-score가 -3이하인 경우 첫 1년은 6개월에 1회, 그 이후부터는 1년에 1회 시행하도록 명시되어 있습니다.\par
\textcolor{red}{실시일로부터 1년 미만에 시행한 골밀도검사는 인정하지 아니함을 알려드립니다}}

\Que{60세 여자가 본인이 원하여 골밀도 검사상 -2.6 check되어 1년간 골다공증약 복용후에 1년후에 BMD F/U하게 된경우 골밀도 이런경우에도 골밀도 검사(DXA검사)를 비급여로 해야하는지 아니면 골다공증이 있었으므로 급여로 해줘도 되는지 문의드립니다}
\Ans{\textcolor{red}{골다공증이 있으므로 급여입니다}(기타 골다공증 검사가 반드시 필요한 경우)\par
골밀도검사의 급여 대상은 고시 제2007-92호(‘07.11.1. 시행)에서 65세 이상의 여성과 70세 이상의 남성, 고위험 요소가 1개 이상 있는 65세 미만의 폐경 후 여성, 비정상적으로 1년 이상 무월경을 보이는 폐경전 여성, 비외상성 골절, 골다공증을 유발할 수 있는 질환이 있거나 약물을 복용중인 경우, 기타 골다공증 검사가 반드시 필요한 경우로 정하고 있습니다. \par
고객님께서 문의하신 “비급여 대상인 건강검진 목적으로 시행한 \textcolor{red}{골밀도검사 결과, T-score가 정상 골밀도 이하로 확인되어 골다공증약 복용 후 추적검사가 필요한 경우”는 상기 적응증 중 기타 골다공증 검사가 반드시 필요한 경우로 급여대상에 해당됨}을 알려드립니다.}

\Que{\textcolor{red}{타병원에서 골다공증약을 복용하고 있던 환자가 본 병원에 내원하였을경우}에 어떡게 하냐고 질문했었고, 심평원에서는 차트 기록및 특정내역란에 기록해두라고 답변이 왔습니다.\par
그래서 그렇게 기록해두면 삭감없냐고 질의를 했었는데, 귀원은 yes or no 로 답 하지 않고 기록해두라고만 답변을 주셨고요.자, 다시 묻습니다. 기록만 그렇게 해두면 삭감은 없는 겁니까??? 만약, 그렇게 기록을 해도 삭감을 한다면, 기록의 의미가 없는 거 아닙니까?? yes or no 로 답해주세요}
\Ans{\textcolor{red}{차트 기록후 특정내역란에 기록하면 골다공증약 급여청구 가능합니다}\par
타병원에서 골다공증약을 복용하고 있던 환자의 진술을 차트에 기록해 두고 그에 기반하여 골다공증약을 처방한 경우 요양급여비용 청구시 특정내역(MX999, JX999)에 현행 고시기준에 부합되는 날짜와 수치가 기재되면 요양급여를 인정합니다.}






