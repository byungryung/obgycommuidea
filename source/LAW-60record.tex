\section{진료기록부}
\subsection{진료기록부 작성}
진료기록부, 조산기록부, 간호기록부, 그 밖의 진료에 관한 기록
\begin{itemize}\tightlist
\item 환자의 주된 증상, 진단 및 치료 내용 등 보건복지부령으로 정하는 의료행위에 관한 사항과 의견을 상세히 기록하고 서명
\item 진료기록부등을 거짓으로 작성하거나 고의로 사실과 다르게 추가기재\cntrdot{}수정 불가
\end{itemize}

\subsection{진료에 관한 기록의 보존 기간}
\begin{enumerate}\tightlist
\item 환자 명부 : 5년
\item 진료기록부 : 10년
\item 처방전 : 2년
\item 수술기록 : 10년
\item 검사소견기록 : 5년
\item 방사선사진 및 그 소견서 : 5년
\item 간호기록부 : 5년
\item 조산기록부: 5년
\item 진단서 등의 부본 (진단서\cntrdot{}사망진단서 및 시체검안서 등을 따로 구분하여 보존할 것) : 3년
\end{enumerate}
- 진료에 관한 기록은 마이크로필름이나 광디스크 등에 원본대로 수록하여 보존 가능; 필름촬영 책임자가 필름의 표지에 촬영 일시와 본인의 성명을 적고, 서명 또는 날인

\subsection{전자의무기록}
\begin{itemize}\tightlist
\item 진료기록부등을 전자서명이 기재된 전자문서로 작성\cntrdot{}보관가능
\item 전자의무기록을 안전하게 관리\cntrdot{}보존하는 데에 필요한 시설과 장비를 갖추어야 함
\item 정당한 사유 없이 전자의무기록에 저장된 개인정보를 탐지하거나 누출\cntrdot{}변조 또는 훼손 불가
\item 전자의무기록의 관리\cntrdot{}보존에 필요한 장비 1. 전자의무기록의 생성과 전자서명을 검증할 수 있는 장비
2. 전자서명이 있은 후 전자의무기록의 변경 여부를 확인할 수 있는 장비
3. 네트워크에 연결되지 아니한 백업저장시스템
\end{itemize}

\subsection{국민건강보험법 관련 서류보존} 
요양기관은 가입자 또는 피부양자에게 요양급여를 한 경우에는 요양급여가 끝난 날부터 5년 동안 보존 
\begin{enumerate}\tightlist
\item 요양급여비용 심사청구서 및 요양급여비용 명세서
\item 약제\cntrdot{}치료재료, 그 밖의 요양급여의 구성 요소의 구입에 관한 서류
\item 개인별 투약기록 및 처방전(약국 및 한국희귀의약품센터의 경우만 해당)
\item 그 밖에 간호관리 등급료의 산정자료 등 요양급여비용 산정에 필요한 서류 및 이를 증명하는 서류 
\end{enumerate}

\subsection{폐업\cntrdot{}휴업시 진료기록부 이관}
\begin{enumerate}\tightlist
\item 의료업을 폐업하거나 1개월 이상 휴업;  관할 시장\cntrdot{}군수\cntrdot{}구청장에게 신고 의무
\item 기록\cntrdot{}보존하고 있는 진료기록부; 보건소장에게 이관 
\item 진료기록부등의 보관계획서를 제출하여 관할 보건소장의 허가를 받은 경우에는 직접 보관
\end{enumerate}