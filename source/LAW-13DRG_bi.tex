\subsection{질병군 급여의 별도 산정 항목}
\begin{enumerate}[가.]\tightlist
\item 식대 \par
국민건강보험법 시행령 제19조제1항 관련 별표2 요양급여비용 중 본인이 부담할 비용의 부담률 및 부담액 제3호에 따른 입원기간중의 식대
\item 별표 2의1에 열거한 항목에 해당하는 외과전문의 가산
\item 복강경 수술 중 개복하여 수술 종결시 추가 산정 비용 \par
복강경을 이용한 수술 중 부득이한 사유로 개복술로 전환하여 수술을 종결한 경우에는 복강경 등 내시경하 수술시 보상하는 239,000원 추가 산정
\item 초음파검사료\par
질병군 진료 시 초음파검사는 「요양급여의 적용기준 및 방법에 관한 세부사항」의 세부인정기준을 적용하며, 인정기준에 의한 급여대상에 해당되는 경우에는 초음파검사료를 추가 산정
\item 4인실 또는 5인실 이용 시 추가비용 \par
4인실 또는 5인실 이용 시 기본입원료와의 차액을 추가 산정
\item 별표2의4에 열거한 항목에 해당하는 행위 및 치료재료
\item 별표2의5에 열거한 「요양급여비용의 100분의 100미만의 범위에서 본인부담률을 달리 적용하는 항목 및 부담률의 결정 등에 관한 기준」에 따른 행위 및 치료재료
\item 마취통증의학과 전문의 초빙료\par
질병군 진료 시 마취통증의학과 전문의 초빙한 경우 마취통증의학과 전문의 초빙료는 「요양급여의 적용기준 및 방법에 관한 세부사항」의 세부인정기준을 적용하여 추가산정
\item 질병군 분류번호를 결정하는 주된 수술 이외에 수술\par
질병군 분류번호를 결정하는 주된 수술 이외에 수술을 실시한 경우 「요양급여의 적용기준 및 방법에 관한 세부사항」의 세부인정기준을 적용하여 추가산정
\item 의료의 질 평가 지원금 \par
질병군 진료 시 의료의질평가지원금은 가-22의 각 분야별 등급별 ‘입원’의 소정점수를 질병군 입원일수와 동일하게 추가산정
\item 응급의료행위료 \par
제1편제2부제19장제2절에 따른 (별표 2) 및 (별표 3)의 응급의료행위를 실시하는 경우 제1편제2부제19장제2절의 산정지침 3. 내지 5. 및 「요양급여의 적용기준 및 방법에 관한 세부사항」을 적용하여 추가산정
\item 전문병원관리료\par
전문병원에서 지정받은 의료기관에서 질병군 진료시 제1편제2부제1장 산정지침 6. 및「요양급여의 적용기준 및 방법에 관한 세부사항」을 적용하여 추가산정
\item 제왕절개분만 심야가산\par
제2편 제4장 산부인과 적용지침 제9호에 따라 22시-06시에 제왕절개분만을 행한 경우에는 질병군 야간ㆍ공휴 소정점수를 2회 추가산정
\item 분만취약지역 가산\par
제2편제4장 산부인과[적용지침] 10. 에 따라 분만취약지에서 제왕절개 분만을 행한 경우에는 질병군 야간ㆍ공휴 소정점수를 4회 추가산정(단, 분만취약지는 제1편에서 정하고 있는 「요양급여의 적용기준 및 방법에 관한 세부사항」을 적용)
\end{enumerate}