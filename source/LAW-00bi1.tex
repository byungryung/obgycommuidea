보통 비급여에 대한 논의를 할 때 건강보험 심사에 대해 잘알지 못하는 경우는 \par
첫번째 항목외 비급여만을 떠올리게 된다. 그러나 이러한 형태의 비급여는 많지 않으며 다른 형태의 비급여가 더 많다. \par

\textcolor{red}{심사기준초과나 급여를 비급여로 받는 형태} 등이다. 그러므로 같은 의료행위를 하더라도 상황에 따라 급여가 될 수도 있고, 비급여가 될 수도 있다. 예를 들어 위궤양 치료제로 많이 사용되는 PPI(프로톤 펌프 억제제)의 경우
위궤양 또는 십이지장이 있는 경우에만 급여로 인정한다. 단순 속쓰림으로 처방했을 경우에는 비급여가 된다. 이런 경우를 급여기준 초과의 예라고 할 수 있다.

허가사항 초과는 항암제를 예로 들수 있다. 약은 안전성, 유효성에 대해 식품의약품안전처로부터 허가를 받아야 사용할 수 있다. 위암에 대한 치료제로 허가받은 약은 말 그대로 위암에 대해서만 유효성이 입증된 약이다. 이 약을 폐암 환자에게 사용하였다면 허가사항 초과가 된다. 실제 병원에서 의사가 폐암환자에게 기존의 약을 다 써보고도 효과가 없으면 다른 항암제의 사용을 고려해 보는데 이는 외국의 논문 등 근거로 하여 사용하게 된다. 또는 외국에는 이미 폐암치료제로 승인을 받았지만 아직 우리나라 식약처에 승인을 받지 않은 경우가 있다. 식약처 승인을 위해서는 비용이 많이 든다. 우리나라 환자를 대상으로 한 연구 결과가 있어야 하는데 임상시험에는 적지 않은 비용이 들기 때문이다. 따라서 수요가 많지 않은 적응증에 대해서는 제약사에서 굳이 비용을 들여서 식약처 승인을 받지 않으려 하기 때문이다.

마지막은 가장 설명하기 어려운 이유인데 병원에서 심평원의 삭감이 우려되어 비급여로 처리하는 경우이다. 이는 심평원의 기준에 대한 불신이나 정보부족등에 의해서 발생한다. 성모병원 사건에서도 정당한 비급여로 인정되지 않은 부분이다. 
\begin{commentbox}{임의 비급여 인정 요건}
임의 비급여 인정요건: [안정성, 유효성, 의학적 필요성, 동의서를 갖추어서 증명해야 합니다.]

\begin{itemize}\tightlist
\item 그 진료행위가 의학적 안전성과 유효성뿐 아니라 요양급여 인정기준 등을 벗어나 진료하여야 할 의학적 필요성을 갖추었고,
\item 가입자 등에게 미리 그 내용과 비용을 충분히 설명하여 본인 부담으로 진료받는 데 대하여 동의를 받았다면, 
\item 이러한 경우까지 “사위 기타 부당한 방법으로 가입자 등으로부터 요양급여비용을 받거나 가입자 등 에게 이를 부담하게 한 때”에 해당 한다고 볼 수는 없다.
\item 다만, 요양기관이 임의로 비급여 진료행위를 하고 그 비용을 가입자 등으로부터 지급받더라도  그것을 부당하다고 볼수 없는 사정은  이를 주장하는 측인 요양기관이 증명하여야 한다.
\end{itemize}
\end{commentbox}

\begin{commentbox}{재진}
\begin{itemize}\tightlist
\item 하나의 상병에 대한 진료를 계속 중에 다른 상병이 발생하여 동일 의사가 동시에 진찰을 한경우
\item 즉 \emph{상병을 두개 넣은 경우네는 재진에 해당}
\item 또는 상병을 고지혈증만 넣는다 해도 30일 이내라면 재진
\end{itemize}
\highlight{초진}
\begin{itemize}\tightlist
\item 고지혈증 \emph{상병만 넣고} N951 상병으로 진찰 받은지 한달이 넘은 경우.
\end{itemize}
\end{commentbox}

2월 1일 질염환자가 3월1일에 다시 질염으로 재발되어 왔다면? 
\begin{quotebox}
초진. 해당 상병의 \emph{치료가 종결된 후 동일 상병이 재발하여 진료를 받기 위해서 내원한 경우에는 초진후 30일 이내에 내원한 경우에는 재진환자}로 본다. 치료의 종결이라 함음 해당 상병의 치료를 위한 내우언이 종경되었거나, 투약이 종결되었을때를 말함
\end{quotebox}

한달 이내에 트리코모나스 질염과 캔디다 질염으로 각각 다른 상병으로 내원하면? 
\begin{quotebox}
재진.하나의 상병에 대한 진료를 계속중에 다른 상병이 발생하여 동일 의사가 동시에 진찰을 한 경우(재진진찰료)
\end{quotebox}

엄청나 산부인과의 엄청난 선생님에게 진료를 본 환자가 한달이 채 안되어서 같은 산부인과 더엄청난 선생님이 다른 상병으로 진료 받는 경우?
\begin{commentbox}{\highlight{초진}}
\begin{itemize}\tightlist
\item 재진의 조건 : 같은 진료과목, 같은 병원이지만 같은 상병이 아니므로 초진
\item 초재진 구분의 3대 기준 
	\begin{enumerate}\tightlist
	\item 상병이 같으냐? 다르냐?
	\item 같은 병원이냐? 아니냐?
	\item 같은 진료과목 의사이냐? 아니냐?[같은 선생님이냐? 아니냐가 아닙니다]
	\end{enumerate}
\end{itemize}
\end{commentbox}

\subsection{비급여 행위 관련 진료시 요양급여 인정범위}
\Que{하지 정맥류 환자에게 비급여 행위로 결정된 레이저정맥폐쇄술(Endovenous Laser Treatment)을 실시한 경우 관련 진료내역 전체를 비급여 대상으로 적용하나요?}
\Ans{질병 자체가 건강보험요양급여 대상에 해당되는 경우에는 해당 비급여 행위를 제외한 모든 진료비에 대해서는 보험급여가 적용됩니다. 따라서 수술료인 레이저정맥폐쇄술료를 제외한 모든 내역을 보험급여로 청구하면 됩니다.}
\begin{commentbox}{신의료행위 관련 진료시 요양급여 범위}
건강보험적용대상자의 진료목적이 건강진단, 미용목적의 성형수술 등과 같이 명백하게 비급여 대상으로 신의료행위를 시술할 경우에는 진찰료를 포함한 모든 진료비는 요양급여대상이 될 수 없으나, 질병자체가 건강보험요양급여대상에 해당하는 경우로 진료담당의사가 진찰, 처치 및 수술 등을 실시할 경우에는 신의료행위(비급여대상 진료)를 제외한 모든 진료비에 대해서는 보험급여 대상이 됨
\end{commentbox}
