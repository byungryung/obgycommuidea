\section{환자기록열람}
환자의 진료를 위하여 불가피하다고 인정한 경우에는 거부 가능 
\begin{enumerate}\tightlist
\item 환자가 아닌 다른 사람에게 환자에 관한 기록을 열람하게 하거나 그 사본을 내주는 등 내용 확인 불가
\item 진료기록을 보관하고 있는 의료기관이나 진료기록이 이관된 보건소에 근무하는 의사는 자신이 직접 진료하지 아니한 환자의 과거 진료 내용의 확인 요청을 받은 경우 - 진료기록을 근거로 하여 사실을 확인 가능
\item 의료인은 응급환자를 다른 의료기관에 이송하는 경우 - 지체 없이 내원 당시 작성된 진료기록의 사본 등을 이송할 의무
\end{enumerate}
\subsection{본인외 기록열람}
\begin{enumerate}\tightlist
\item 환자의 배우자, 직계 존속\cntrdot{}비속 또는 배우자의 직계 존속이 환자 본인의 동의서와 친족관계임을 나타내는 증명서 등을 첨부하는 경우
\item 환자가 지정하는 대리인이 환자 본인의 동의서와 대리권이 있음을 증명하는 서류를 첨부하는 경우
\item 환자가 사망하거나 의식이 없는 등 환자의 동의를 받을 수 없어 환자의 배우자, 직계 존속\cntrdot{}비속 또는 배우자의 직계 존속이 친족관계임을 나타내는 증명서 등을 첨부하는 경우
\item 급여비용 심사\cntrdot{}지급\cntrdot{}대상여부 확인\cntrdot{}사후관리 및 요양급여의 적정성 평가\cntrdot{}가감지급 등을 위하여 국민건강보험공단 또는 건강보험심사평가원에 제공하는 경우
\item 의료급여 수급권자 확인, 급여비용의 심사\cntrdot{}지급, 사후관리 등 의료급여 업무를 위하여 보장기관, 국민건강보험공단, 건강보험심사평가원에 제공하는 경우
\item 「형사소송법」「민사소송법」「의료사고 피해구제 및 의료분쟁 조정 등에 관한 법률」에 따른 경우
\item 근로복지공단, 자동차보험진료수가를 청구 받은 보험회사, 지방병무청장, 공제회, 보훈병원장(고엽제 관련), 국민연금공단에서 각각의 사유에 따라 교부를 요청하는 경우
\item  의료인은 다른 의료인으로부터 진료기록의 내용 확인이나 환자의 진료경과에 대한 소견 등을 송부할 것을 요청 받은 경우; 환자나 환자 보호자의 동의를 받아 송부.
\item 해당 환자의 의식이 없거나 응급환자인 경우 또는 환자의 보호자가 없어 동의를 받을 수 없는 경우; 환자나 환자 보호자의 동의 없이 송부
\end{enumerate}

\subsection{기록열람 요건}
\begin{itemize}\tightlist
\item 환자의 배우자, 직계 존속\cntrdot{}비속 또는 배우자의 직계 존속이 환자에 관한 기록의 열람이나 그 사본의 발급을 요청할 경우 
	\begin{mdframed}[linecolor=blue,middlelinewidth=2]
	\begin{enumerate}\tightlist
	\item 기록 열람이나 사본 발급을 요청하는 자의 신분증 사본
	\item 가족관계증명서, 주민등록표  등본 등 친족관계임을 확인할 수 있는 서류
	\item 환자가 자필 서명한 동의서 (환자가 만 14세 미만의 미성년자인 경우에는 제외)
	\item 환자의 신분증 사본 (환자가 만 17세 미만으로 주민등록증이 발급되지 아니한 경우에는 제외)
	\end{enumerate}
	\end{mdframed}
\item 환자가 지정하는 대리인이 환자에 관한 기록의 열람이나 그 사본의 발급을 요청할 경우
	\begin{mdframed}[linecolor=blue,middlelinewidth=2]
	\begin{enumerate}\tightlist
	\item  기록열람이나 사본발급을 요청하는 자의 신분증 사본
	\item  환자가 자필 서명한 위임장 (만 14세 미만의 미성년자인 경우에는 환자의 법정대리인이 작성, 가족관계증명서 등 법정대리인임을 확인할 수 있는 서류를 첨부)
	\item  환자의 신분증 사본 (만 17세 미만으로 주민등록증이 발급되지 아니한 자는 제외)
	\end{enumerate}
	\end{mdframed}	
\item 환자가 사망하거나 의식이 없는 등 환자의 동의를 받을 수 없는 경우 - 환자의 배우자, 직계 존속\cntrdot{}비속 또는 배우자의 직계 존속이 친족관계임을 나타내는 증명서 
\item 환자가 본인에 관한 진료기록 등을 열람하거나 그 사본의 발급을 원하는 경우에는 본인임을 확인할 수 있는 신분증 제시		
\end{itemize}

\subsection{환자진료기록 사본 교부에 대해}
보건복지부 질의 회신서
\Que{ 의료법 제20조(기록열람 등)에 의하면 가족이 전혀 없는 경우에 한해 환자가 지정하는 대리인에게 기록사본을 교부할 수 있게 되어 있습니다. 그런데 환자가족이 있는 상황에서 보험회사 직원이 위임장을 제시하며 기록 열람 및 차트(진료기록부) 복사 요구시 교부를 해주어야 하는지 여부?}
\Ans{ 우리부에서는 2003. 9. 1. 진료기록사본발급지침상 “환자의 가족 또는 그 대리인이 진료기록 사본 발급을 요청할 때에는 신청서를 작성하고 환자가 직접 작성, 날인한 위임장이 첨부되어야 하며, 위임장에는 위임자와 피위임자의 인적사항 및 \textcolor{red}{위임의 내용 등이 구체적으로 명시}되어야 하며, 이를 객관적으로 입증할 수 있는 인감증명서가 제시되어야 한다”고 한 바 있음.따라서 환자의 가족이 있는 경우에도 환자자신이 진료기록 사본교부를 대리인에게 위임하여 위임장을 직접 작성, 날인하여 교부한 경우에는, 환자편의를 위하여 교부해줄 수 있음.}

\Que{교부를 해야 한다면 보험과 관련없는 환자의 과거력 등 비밀이 누설될 수 있는데 비밀누설에 따라 의사가 처벌받을 수 있는지 여부?}
\Ans{환자 자신이 직접 작성\cntrdot{}날인하여 교부한 위임장에는 위임내용을 구체적으로 명기하도록 규정하고 있는바, 환자가 명기하여 교부를 위임한 부분에 대한 진료기록 사본 발급은 비밀누설과 관련이 없을 것이나, 위임내용을 벗어난 기록의 교부는 비밀누설에 해당될 수 있음.} 

\Que{아울러 위임장의 진위여부를 환자본인에게 유선으로 확인해야 하는지, 확인하지 않고 객관적 증빙자료인 인감증명서만 확인 후 교부했을시 차후 환자본인으로부터 이의를 제기받았을 때 의사가 처벌받을 수 있는지 여부?}
\Ans{현행 우리부 진료기록사본발급지침상 환자의 가족 또는 그 대리인이 진료기록사본발급 요청할 때에는 신청서를 작성하고 환자가 직접 작성, 날인한 위임장이 첨부되어야 하며, 
위임장에는 위임자와 피위임자의 인적사항 및 위임 내용 등이 구체적으로 명시되어야 하며, 이를 객관적으로 입증할 수 있는 인감증명서가 제시되어야 한다고 규정하고 있는 바 이와같은 \textcolor{red}{제출서류 확인으로도 가능할 것}임. 
그러나 다른 사람의 인장 도용 등에 의해 허위로 위임장을 작성, 신청하는 경우에는 형법 제231조와 제232조의 규정에 의해 사문서 위\cntrdot{}변조죄로 5년이하의 징역에 처해질 수 있음.}

\Que{또한 보건복지부가 고시한 진료기록부 복사, 필름 복사 등 환자제공을 위한 복사비용은 실비로 환자가 부담토록 규정(2003.1.27 복지부 인터넷 민원 회신 보험급여과 유권해석)하고 있는데, 적정교부 비용은 얼마인지?}
\Ans{진료기록 등 사본교부와 관련된 비용에서 실비를 징수할 수 있도록 한바, 의료기관은 실비 범위 내에서 사본교부 수수료에 적정한 금액을 징수할 수 있을 것임.}

\Que{환자의 진료목적이 아닌 보험회사의 영업행위와 관련된 요청으로 법률적 쟁론시 전문가로서의 정확성을 보장해야 하는 것을 감안해 보험회사측이 환자의 상황을 상담하는 경우 변호사처럼 상담료를 받을 수 있는지 여부?}
%\begin{mdframed}[linecolor=blue,middlelinewidth=2]
\Ans{- 의료법 제19조에 “의료인은 이법 또는 다른 법령에서 특히 규정된 경우를 제외하고는 그 의료 조산 또는 간호에 있어서 지득한 타인의 비밀을 누설하거나 발표하지 못한다”고 규정하고 있음. 따라서 \textcolor{red}{의료인이 보험회사측과 환자의 상황에 관해 상담하는 것은 의료법 비밀누설의 금지에 위배}될 수 있음.}
%\end{mdframed}

\subsection{수사 협조를 위한 진료기록 사본 제공 지침}
형사사건 수사 협조를 위해 진료기록 사본을 의무적으로 제공해야 하는 경우
\begin{mdframed}[linecolor=blue,middlelinewidth=2]
\begin{itemize}\tightlist
\item 법원이 압수 또는 제출을 명하거나, 검사 또는 사법경찰관이 지방법원판사가 발부한 영장에 의하여 압수, 수색 또는 검증을 하는 경우 (환자 본인 동의 불필요) 
\item 위의 방법이 아닌 \textcolor{red}{일반적인 공문형태의 수사 협조 요청일 경우에는 의료인이 그 요청에 따를 의무는 없음}
\item 그러나 의료법 제21조 제2항 제6호의 <형사소송법 제218조> 관련, 진료기록 사본 제공에 따르는 공/사익의 이익 형량을 의료인 스스로 판단하여 공익을 위해 임의로 진료기록 사본을 제공하려는 경우  해당 환자의 이익이 부당하게 침해될 우려가 있는지를 검토해야 하며 그러한 우려가 없는 경우에 한하여 제공
\item 입․퇴원 및 외래 내원 여부 같은 환자의 행적, 연락처 등 긴급하게 수사에 필요하다고 판단되는 이외에 진료과목, 처치내용 등 질병 치료와 직접적으로 관계된 내역은 일반적으로 민감한 프라이버시에 해당되므로,  환자의 동의 없이 진료기록 사본을 임의로 제출하였다면 당사자가 「의료법」 및 「개인정보보호법」에 의거하여 소제기가 가능할 것임
\end{itemize}
\end{mdframed}
\subsubsection*{개인정보보호법}
개인정보처리자는 다음에 해당하는 경우에는 정보주체 또는 제3자의 이익을 부당하게 침해할 우려가 있을 때를 제외하고는 개인정보를 목적 외의 용도로 이용하거나 이를 제3자에게 제공 가능 (제5호부터 제9호까지의 경우는 공공기관의 경우로 한정)
\begin{enumerate}\tightlist
\item 정보주체로부터 별도의 동의를 받은 경우
\item  다른 법률에 특별한 규정이 있는 경우
\item 정보주체 또는 그 법정대리인이 의사표시를 할 수 없는 상태에 있거나 주소불명 등으로 사전 동의를 받을 수 없는 경우로서 명백히 정보주체 또는 제3자의 급박한 생명, 신체, 재산의 이익을 위하여 필요하다고 인정되는 경우
\item 통계작성 및 학술연구 등의 목적을 위하여 필요한 경우로서 특정 개인을 알아볼 수 없는 형태로 개인정보를 제공하는 경우
\item 개인정보를 목적 외의 용도로 이용하거나 이를 제3자에게 제공하지 아니하면 다른 법률에서 정하는 소관 업무를 수행할 수 없는 경우로서 보호위원회의 심의\cntrdot{}의결을 거친 경우
\item 조약, 그 밖의 국제협정의 이행을 위하여 외국정부 또는 국제기구에 제공하기 위하여 필요한 경우
\item 범죄의 수사와 공소의 제기 및 유지를 위하여 필요한 경우
\item 법원의 재판업무 수행을 위하여 필요한 경우
\item 형(刑) 및 감호, 보호처분의 집행을 위하여 필요한 경우
\end{enumerate}
\subsubsection*{결론}
수사협조 목적의 개별 환자기록을 임의 제출
\begin{mdframed}[linecolor=blue,middlelinewidth=2]
\begin{itemize}\tightlist
\item 진료과목, 처치내용 등 당사자의 프라이버시와 관련된 내용은 의료법상  당사자의 동의 원칙이 준수 
\item 개인정보 보호법이 새롭게 제정된 점을 감안하여 환자 이익 침해 여부에 관한 검토가 반드시 선행
\end{itemize}
\end{mdframed}