\section{난소암 의증}
\myde{}{%
\begin{itemize}\tightlist
\item[\dsjuridical] C560 C561 C569[난소의 악성신생물]
\item[\dsjuridical] D270 D271 D279[난소의 낭종]
\item[\dsmedical] EB457010 여성생식기 초음파 정밀/도플러 
\item[\dschemical] D3720 hCG - 242.70점 - \myexplfn{242.70} 원
\item[\dschemical] D2420 AFP - 49.85점 - \myexplfn{49.85} 원
\item[\dschemical] D4290 CEA - 230.59점 - \myexplfn{230.59} 원
\item[\dschemical] D4311 CA125 - 253.63점 - \myexplfn{253.63} 원
\item[\dschemical] D4350 CA19-9 - 263.56점 - \myexplfn{263.56} 원
\item[\dschemical] D4370 HE4 - 225.66점 - \myexplfn{225.66} 원 : 본인부담 80\%인 선별급여
\item[\dschemical] D4370 \& D4311 ROMA - CA 125는 기존대로 건강보험적용(보호 제외한 보험은 본인부담금30\%)와 HE-4는 급여구분:100/100미만(80)  
\item[\dschemical] D4360 CA72-4 \textcolor{red}{하나더}
\end{itemize}

}%
{
나420-나435, 나437, 너321, 너322종양표지자 검사 및 나352를 종양표지자로서 검사할 경우의 급여기준은 다음과 같이 함.
\begin{enumerate}[가.]
\item 악성종양이 원발장기에 있는 경우 : 최대 2종 인정
\item 악성종양이 원발장기와 속발(전이)장기에 있거나 악성종양이 의심되는 경우 : 원발장기 2종을 포함하여 최대 3종 인정
\item 원발장기가 확인이 안된 상태에서 암이 의심되어 실시하는 경우 : 장기별로 1종씩 인정하되, 최대 3종까지만 인정. 다만, \textcolor{red}{난소암이 의심되는 경우는 조직학적 타입에 따라 specific tumor marker가 각기 다를 수 있으므로, 치료전 검사로 1회에 한하여 최대 5종까지 인정함.} 각 장기의 specific tumor marker는 아래와 같으며, specific tumor marker가 없는 장기의 경우도 상기 인정기준을 적용함.
\item 종양표지자 중 ‘나437 인간부고환 단백4’은 「요양급여비용의 100분의 100미만의 범위에서 본인부담률을 달리 적용하는 항목 및 부담률의 결정 등에 관한 기준」에 따라 본인부담률을 80\%로 적용함.
\item ovary : hCG, AFP, CEA, CA125, CA130,CA19-9
\end{enumerate}
}

\subsection{의증 vs 배제진단}
의증과 배제상병은 완전히 다른개념
\begin{itemize}\tightlist
\item[•] 의증: 최종진단이 나오기 전에 의심된다는 개념
\item[•] 배제상병: 이전에 고려되었지만 최종상병이 확진된 후 배제된 상병 즉 의증은 질병 가능성이 있다는 의미이고 배제된 상병은 그 병이 아니라는 말.
\item[•] 배제상병 의 활용  
	\begin{itemize}[▲]\tightlist
	\item 확진된 진단명과 관계없는 검사들을 청구할 때 사용 
	\item 초재진 산정 불이익 방지에 이용 
	\item 진단명과 관련된 불필요한 민원 방지에 유용
	\end{itemize}
\end{itemize}	
\subsection{선별급여}
「요양급여비용의 100분의 100 미만의 범위에서 본인부담률을 달리 적용하는 항목 및 부담률의 결정 등에 관한 기준」 일부개정\par
\begin{description}\tightlist	
\item[제1조] 중 “요양급여의 적용기준 및 세부사항”을 “요양급여의 적용기준 및 방법”으로 한다.
\item[제3조] 제1항 및 제2항을 다음과 같이 하며, 제3항 및 제4항을 제4항 및 제5항으로 하고, \uline{제3항을 다음과 같이 신설한다.} 
	\begin{enumerate}[①]\tightlist
	\item 영 별표2 제4호에 따른 요양급여 항목 및 본인부담률은 별표2와 같다. 다만, 「요양급여의 적용기준 및 방법에 관한 세부사항」에 급여대상 이외 본인부담률을 별로도 정하여 실시하는 경우는 해당 항목의 세부인정사항에 따른다. 
	\item 제1항에 따른 요양급여 항목 및 본인부담률은 「국민건강보험 요양급여의 기준에 관한 규칙」(이하 “요양급여기준”이라 한다) 제11조의3에 따른 급여평가위원회(이하 “급여평가위원회”라 한다)의 평가와 다음 각 호의 구분에 따른 위원회의 평가를 거쳐 보건복지부 장관이 정한다. 
		\begin{enumerate}[1.]\tightlist
		\item 행위, 치료재료 : 요양급여기준 제11조제7항에 따른 행위 및 치료재료별 전문평가위원회(이하 “전문평가위원회”라 한다)
		\item 약제(암환자에게 처방\cntrdot{} 투여되는 약제를 제외한다) : 「국민건강보험법」제66조에 따른 진료심사평가위원회(이하 “진료심사평가위원회”라 한다)
		\item 암환자에게 처방\cntrdot{} 투여되는 약제 : 요양급여기준 제5조의2에 따른 중증질환심의위원회(이하 “중증질환심의위원회”라 한다)
		\end{enumerate}
	\item 제2항에도 불구하고 요양급여기준 제5조에 따른 요양급여의 적용기준 및 방법에 대해 본인부담률을 달리 적용하는 항목 및 부담률의 결정에 관한 사항은 급여평가위원회 평가를 거쳐 보건복지부장관이 정한다. 
	\end{enumerate}
\item[제9조] 제3항을 다음과 같이 한다. 
	\begin{enumerate}[③]\tightlist
	\item 제3조제2항의 각 위원회는 평가대상 항목이 영 별표2 제5호에 따라 고시된 당시의 치료효과성, 비용효과성, 대체가능성, 사회적 요구도 등에 대한 평가결과와 비교하여 개선되었는지 여부 등을 고려하여 평가를 실시하고 그 결과를 보건복지부장관에게 보고한다. 
	\end{enumerate}
\item[제11조] 중 “제248호)”를 “제334호)”로, “2016년 12월 17일까지로”를 “2019년 12월 31일까지로”로 각각 한다. 
\end{description}
이 고시는 2016년 9월 1일부터 시행한다.\par
즉 100/100과 유사한 다른 보험급여 방법으로 생각됩니다. 100/100이란 보험급액으로 하되 환자가 100\% 부담하는 방법이라면, \emph{80/100란 환자가 100\%보험금액중 80\%부담하고 나머지 20\%는 보험회사가 부담하는 방법입니다.} 산부인과에서 대표적으로 HE4가 80/100으로 됨으로 인해 ROMA검사도 80/100으로 되었습니다.
