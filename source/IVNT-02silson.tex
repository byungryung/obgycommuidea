\section*{비타민 D 주사}
\begin{commentbox}{VitD}
[효능/효과] 비타민D가 결핍된 고령자및 청소년에서의 비타민D결핍의 \emph{예방과 치료}, 비타민D가 결핍된 성인에서의 비타민D 결핍의 \emph{치료} → 실손보험} 적용가능 \par
\emph{Protocol} 혈액 중 25-히드록시 비타민 D량을 확인하여 용량을 조절한다.\par
이 약 투여 후 고령자의 경우 3개월, 청소년은 6개월 내에 치료효과와 내약성이 검토되어야 하며, 이를 바탕으로 재 투여 여부를 결정한다. 과량 투여에 따라 이상반응이 증가하므로 필요량 이상으로 투여 되지 않도록 한다.\par
- 고령자 : 콜레칼시페롤로서 100,000IU을 근육주사한다. \par
- 청소년 : 콜레칼시페롤로서 200,000IU을 근육주사한다. \par
연간 투여량 최대 600,000IU, 진료소견서/영수증 발급

\begin{description}\tightlist
\item[적합상병] E559 상세불명의 비타민 D 결핍증
\item[소견서 내용] 상기 환자분 25 OH vit D 수치가 ng/ml로 부족하여 비타민 D 주사 맞으심.
\end{description}
\end{commentbox}

\highlight{본인이 원하는 검사나 상기기준에서 벗어나는 경우  당연히 비급여 입니다,} 비급여를 꼭 100/100 만 받을필요없습니다. 비급여 고시후 적정금액 받으시면 됩니다.\par
비타민D는 30이상이면 정상, 10.0 이하일 때 결핍증에 해당되므로 맞아서 문제가 없고 10.0 이상에서 투여했다면 25(OH)vitD 검사를 내일 바로 보내서 혈중농도를 체크하세요.
\par
\medskip
\begin{center}
\includegraphics{vitD}
\end{center}
\leftrod{VItamin D처방에 대한 tip}
\begin{enumerate}\tightlist
\item 비오엔 주사도 비타민 D 결핍이 20 이하니 그러면 실비
\item 전 폐경기 호르몬 치료 시작전에 채혈할때,   상세불명의 비타민 결핍증 코드 넣고 cy155 routine으로 합니다. 거의 다 결핍나와서 호르몬 처방할때  비오엔  3개월마다 주사합니다. 실비도 되고, 호르몬 치료 효과도 더 좋은듯 해요.
\item 전 검사를 100/100 냅니다..6000 원.
\item 먹는약 써니디로 처방합니다. 성인도 주사보다는 써니디로 많이 처방합니다. 그게 흡수율이 좋다고 들어서요
\item 비타민 d 주사 지용성이라 뻑뻑하고 근육 많이 풀어줘야 되는게 좀 단점이죠
\item 저도 써니디 많이 줬는데, 생각보다 잘 안 챙겨 드시더라구요..폐경 여성은 호르몬제  3개월 단위로 올때 같이 주사 놔주니 compliance가 좋아서요
\item 비오앤도 공급가가 지방이랑, 서울이랑 현저히 차이 나서, 전 비타디본 으로 바꿨어요..휴온스 메리트 d도 좀 더 싸더라구요.
\end{enumerate}
