\section{Fosamax 70mg or Fosamax plus 70mg}
Alendronic acid (INN) or alendronate sodium (USAN) — sold as Fosamax by Merck — is a bisphosphonate drug used for osteoporosis and several other bone diseases\\


\href{http://en.wikipedia.org/wiki/Alendronic_acid}{wikimedia} \url{http://en.wikipedia.org/wiki/Alendronic_acid}\\ 
\href{http://www.kmle.co.kr/viewDrug.php?m=%B6%F3%BD%C4&inx=46666&c=bce61e0d3ab9c85b21de3f44585a9077}{drug info} \url{http://www.kmle.co.kr/viewDrug.php?m=%B6%F3%BD%C4&inx=46666&c=bce61e0d3ab9c85b21de3f44585a9077}
\begin{shaded}
 “이를 뽑은 뒤 새로운 이가 자라지 않은 채 구멍이 매워지지 않고 뼈가 드러나 염증이 악화되는 것”이라며 “이 때문에 턱이 썩는 것”이라고 설명했다.
세기자데 교수는 “포사맥스의 비스포스포네이트 성분과 관련된 턱괴사 환자를 심심치 않게 볼 수 있다”며 “의사들은 치주질환이 있는 여성에게 골다공증 치료제를 처방할 때 특히 주의를 당부해야 한다”고 말했다.
\end{shaded}
\subsection{\newindex{하드칼츄어블이지정}}
\begin{itemize}\tightlist
\item ca 300mg with vitamin D3 200 IU
\item 급여가능 적응증 : 
	\begin{enumerate}[1.]\tightlist
	\item 골다공증(M80,M81,M82) 골밀도 검사상 같은 성, 젊은 연령의 정상치보다 1표준편차 이상 감소된 경우 (즉 T-score가 -1이하인 경우) 
	\item 폐경기증후군 (N95) 갱년기증상(폐경기 증후군)의 증상완화를 위해 투여하는 경우 
	\item 식사성 칼슘결핍(E58)
	\end{enumerate}
\item 급여일수 : 제한없음
\item 검사지 첨부여부 : 미첨부, 골밀도 수치는 명세서 여백이나 메모란, 참조란을 이용하여 기재
\item 병용투여인정기준 :
	\begin{itemize}\tightlist
	\item 하드칼+호르몬 요법(Estrogen, estrogen derivatives등) 
	\item 하드칼+비호르몬 요법 (Bisphosphonate, Raloxifene, calcitonin, Lpriflavon등)
	\end{itemize}
\end{itemize}
