\section{주진단명 부여방법}
\leftrod{주진단의 정의}
\par
\medskip
주진단은 검사 후 밝혀진 최종 진단으로 병원 치료(또는 의료시설 방문)를 필요로 하게 만든 가장 중요한 병태이다. 단, 진료 개시 후 의료시설을 방문하게 만든 병태와는 관련이 없는 새로운 병태가 발견되고, 이로 인한 자원 소모가 더 클 때에는 새로운 병태를 주진단으로  선정한다. 진료 후 밝혀진 진단은 입원 시 진단과 일치할 수도 있고 일치하지 않을 수도 있다. \par
진단이 내려지지 않은 경우에는 주증상이나 검사의 이상소견 또는 문제점을 주진단으로 선정한다. 진료기간 동안 검사나 치료를 받은 병태 중 ‘주진단’은 단일병태 질병이환 분석시 사용된다.
\prezi{\clearpage}
\leftrod{주진단 선정원칙}
\par
\medskip
\textcolor{red}{검사 후 밝혀진 최종 진단으로 병원 치료 또는 의료기관 방문을 필요로 하게 만든 가장 중요한 병태}를 주진단으로 선정한다.
환자가 여러 질환을 동시에 가지고 내원한 경우에는 진단이나 치료에 대한 환자의 요구가 가장 컸던 질환, 즉 의료자원을 가장 많이 사용하게 했던 질환을 주진단으로 선정한다.\par
여기서 의료자원이란 단순히 해당 질병과 관련된 진료비의 크기만을 의미하는 것은 아니며, 해당 질병으로 인해 유발된 재원일수, 시술비용, 약품 및 치료 재료비 등을 종합적으로 고려하여 판단한다. 즉 진료비가 높다고 하더라도 그것이 전체 서비스 제공량과는 무관하게 몇개의 고가 약이나 치료재료 때문에 발생한 것이라면 해당 질병의 자원 소모량이 반드시 높다고 말할 수 없다. 약이나 치료재료의 가격은 나라마다 상이하기 때문에 이로 인한 비용만을 기준으로 자원소모량을 판단하는 것은 국제비교 측면에서 타당하지 않기 때문이다. 자원소모량의 크기에 대해서 논란이 있는 경우에는 자원소모량에 대한 진료의사의 판단에 따른다\par
진료 개시 후 \textcolor{red}{주진단과 관련된 질환이나 합병증이 발생하였을 경우에는 이로 인한 자원소모가 많다고 할지라도 기존 주진단을 유지}한다.\par
단 진료 개시 후 \textcolor{blue}{의료시설을 방문하게 만든 병태와는 관련이 없는 새로운 병태가 발견되고, 이로 인한 자원 소모가 더 클 때에는 새로운 병태를 주진단}으로 선정한다.\par
\textcolor{red}{진단이 내려지지 않은 경우에는 주증상이나 검사의 이상소견 또는 문제점을 주진단}으로 선정한다.\par
\prezi{\clearpage}
\subsection{DRG에서 주진단에 대한 논란}
\begin{hemphsentense}{산부인과 ``DRG 손실 커" 복지부 ``새 패러다임 협조"}
\href{https://dailymedi.com/news/view.html?section=1&category=5&no=771918}{주진단에 대한 복지부의 입장}
산부인과가 포괄수가제 병원급 확대 적용 이후 제왕절개술 주진단명 코딩 방법 변경에 따라 막대한 손실을 입고 있다는 우려가 제기됐다.
대한산부인과학회는 27일 제99차 학술대회를 개최하고 포괄수가제 등 ‘산부인과 건강보험의 과제’에 대해 논의하는 시간을 가졌다. 관동의대 산부인과 민응기 교수는 “우여곡절을 겪으면서 지난 7월부터 모든 의료기관을 대상으로 7개 질병군에 대한 포괄수가제 강제적용이 시작됐다”며 “건강보험심사평가원에 제왕절개술의 보험급여를 청구하면서 심각한 문제가 발생했다”고 전했다.\\
\textcolor{blue}{1997년 2월 포괄수가제 시범사업을 시작하면서부터 올 6월까지 제왕절개의 ‘주진단’명은 O820-O829로 코딩을 하고 제왕절개술을 한 주 사유를 `기타진단’으로 코딩해 중증도 보정을 받아왔다.} 또한 2007년도 포괄수가제 실무지침서에서도 똑같이 주진단을 O820-O829로 코딩하는 청구방법을 공지했고 그렇게 시행해왔다.\\
포괄수가제를 전면 시행하기로 한 지난 7월 1일 직전인 6월 15일 심평원 교육자료에서도 마찬가지였다.민 교수는 “그러나 7월 포괄수가제 전면실시 후 심평원에서는 \textcolor{red}{갑자기 한국표준질병사인분류 질병코딩지침서에 의거해 O820-O829를 주진단이 아닌 ‘기타진단’으로 코딩해 제왕절개술을 시행한 주 사유를 ‘주진단’명으로 코딩해야 한다고 불과 보름 만에 말을 바꿨다”고 지적했다.} 이로 인해 많은 환자에서 중증도가 반영되지 않아 제왕절개술을 시행한 모든 의료기관은 큰 손실을 보게 됐다는 것이다. 학회에 따르면 실제 의료기관이 똑같은 진단명으로 제왕절개술을 하더라도 평균 입원일수를 7일로 산정할 때 단태아는 10만1810원-26만1350원, 다태아의 경우 25만7170원-41만3510원의 금액을 손해 볼 수밖에 없다는 분석이다. 그는 “분만 전문 병의원의 경우 어림잡아 연간 수천만원에서 억대에 이르는 순익 손실을 보게 되는 금액”이라면서 “심평원은 갑작스런 코딩 방법 변경에 대한 납득할만한 해명을 내놓지 못하고 있다”고 강조했다.\\
민 교수는 이어 \uline{“오랜 기간 동안 시행해 온 대로 기존 코딩방법을 유지해야 한다”면서도 “기존 방법에 문제가 발견됐다면 그 틀을 벗어나지 않는 범위에서 문제점 해결 방안을 찾거나 중증도 반영이 달라지지 않도록 서둘러서 보완을 해야 할 것”이라고 피력했다.}\\
복지부 ``과거 방식 고치는데 있어 나타난 전환기적 불편함" 양해 구해\\
보건복지부는 이에 대해 과거 문제가 있었던 부분을 고치는 과정에서 발생한 문제점이라면서 바람직한 정착에 노력하겠다는 입장이다. 보건복지부 보험급여과 배경택 과장은 “과거 방식에 문제가 있어 고치는 과정에서 나타난 전환기적 불편함이라 생각한다. 심평원 업무처리가 더디게 느껴질 수 있으나 이는 복지부에서 의사결정 하는데 시간이 걸렸기 때문”이라고 양해를 구했다. 배 과장은 이어 “바람직하게 개선하는데 심평원, 복지부 모두가 노력할 것”이라고 말했다. 이와 함께 산부인과가 포괄수가제라는 새로운 패러다임을 구축하는데 동행자가 돼 줄 것을 당부했다.\\
그는 “포괄수가제는 다른 패러다임”이라며 “산부인과가 기존 행위별수가제에 안주할 수 있을까에 대해 고민을 해봐야 한다. 새로운 패러다임을 만드는데 적극 참여해 긍정적으로 구축할 수 있도록 노력할 필요가 있다”고 덧붙였다.
\end{hemphsentense}
예를 들면 한국표준질병사인분류 질병코딩지침서에 의거해 제왕절개를 하게된 주된 사유인 Placenta previa를 주진단으로 하게되면, 중등도가 반영되지 않아서 손해를 보게된다고 하는데 이게 어떤 의미일까? \\ DRG에서 마지막 단계로 합병증 및 동반상병 분류로 각 질병군 범주의 특성에 따라 구분된 환자단위 중증도 점수별로 \uline{최종 질병군 분류번호를 결정 하게 된다}. 여기에 쓰이는 것이 기타진단(부진단)입니다. 다시말하면, \textcolor{red}{기타진단의 중요도는 기타진단(부진단)에 의해서 중등도가 결정되게 됩니다.}\\
이와 같은 이류로 \textcolor{red}{중등도(=부진단 NOT주진단)}는 보험급여금에 차이를 주는데요. 여태까지는 o820등의 선택적제왕절개등을 주진단으로 하고 주된이유를 부진단으로 해서 어려운수술에 대한 보전을 받았는데, placenta previa같이 어려운 수술을 해도 현 DRG 주진단 system에서는 보전받을수 없게 되었다는 것입니다. \\
물론 severe PIH로 제왕절개를 한경우에 주진단으로 severe PIH 한가지만 내 놓으면 기본적인 DRG금액만 받을수 있지만, CPD를 주진단으로 하고 severe PIH, hemorrhage등등의 기타진단을 기입하게 되면 severity등급에 따라서 최소 9만원이상의 금액을 더 받을수 있습니다. 실제로 따져봐도 severe PIH는 직접적으로 제왕절개를 해야할 주된 이유는 아닙니다. CPD나 failure to progression등이지요!!\\
\prezi{\clearpage}

\begin{tcolorbox}[frogbox,title=주진단과 기타진단에 대한이해]
\begin{enumerate}[가.]\tightlist
\item 주진단
	\begin{enumerate}[(1)]\tightlist
	\item 한번 입원한 건에 대하여는 하나의 주진단을 부여한다. 둘이상의 병태가 주진단 정의에 똑같이 부합될 때는 둘 중 어느 진단을 선택하여도 무방하나 하나의 진단만을 주진단으로 부여한다.
	\item 비급여대상 질환(「국민건강보험 요양급여의 기준에 관한 규칙」별표2 제6호에 해당하는 질환)이 주진단에 해당될 경우는 기타진단 중 가장 주된 진료를 받은 진단을 주진단으로 선정한다.
	\item 진단이 확립되지 아니한 경우 \textcolor{blue}{의심되는 진단(의증)을 주진단으로 부여할 수 있다.} 입원기간 중 생성된 진단 정보가 없어서 진료 후에도 주진단이 여전히 ‘의심되는’, ‘의문나는’ 등으로 기록되어 있는 경우 의심되는 진단을 확진된 것처럼 부여할 수 있다.
	\end{enumerate}
\item 기타진단
\begin{enumerate}[(1)]\tightlist
\item \textcolor{red}{확립된 진단만 부여하고 의심되는 진단(의증)은 기타진단으로 부여 하지 아니한다.} 기타진단은 확진된 경우만 부여할 수 있으며, 의심되는 진단(의증)은 부여하지 아니한다. 의심되는 진단(의증)의 경우는 그 진단과 관련되는 증상 및 증후〔ⅩⅧ장. 달리 분류되지 않은 증상, 징후와 임상 및 검사의 이상 소견에 해당되는 분류기호로 부여하여야 한다.
\item 비급여 대상 질환은 기타진단으로 부여하지 아니한다.
\item 이번 입원과 관련 없는 이전 병태는 기타진단으로 부여하지 아니한다. 진료기록부의 최종진단명란에 기재되어 있는 진단명은 주진단 이외 에는 일반적으로 모두 기타진단으로 간주할 수 있으나, 그 중 과거의 진료 또는 병력에 해당되는 병태로서 이번 입원과 관련 없는 경우는 기타진단으로 부여하지 아니한다.
\item 전신적인 만성질환은 기타진단으로 부여할 수 있다. 고혈압, 파킨슨병, 당뇨병\footnote{하지만 고혈압,당뇨등도 합병증이 없는 경우에는 기타진단.즉 중등도에 올라가기는 힘들다. severity점수가 없다} 등과 같은 만성질환은 지속적인 임상적 평가, 추가적인 간호 및 관찰이 요구될 수 있으므로 기타진단으로 부여할 수 있다.
\item 질병진행 과정중의 한 부분으로의 병태는 기타진단으로 부여하지 아니한다. 질병의 진행과정에 반드시 수반되는 병태는 기타진단으로 별도 부여하지 아니한다.
\item \uline{비정상적인 검사결과만으로(진료의가 임상적인 의미를 부여하지 않은 경우) 기타진단으로 부여하지 아니한다.}
\end{enumerate}
\end{enumerate}
\end{tcolorbox}
\prezi{\clearpage}

\begin{myshadowbox}
\begin{enumerate}[5.]\tightlist
\item 질병군에 대한 \textcolor{red}{요양급여비용을 산정}할 때에는 제2부(실무안내 제2장) 각 장에 분류된 질병군 점수를 기준으로 별표 1의 질병군별 점수 산정요령에 의하여 산정된 점수 총합에 국민건강보험법 제45조제3항과 영 제21조제1항에 따른 점수 당 단가를 곱하여 10원 미만을 절사한 금액을 요양급여비용 총액으로 산정한다. 이 경우 위 금액 외에 식대를 포함한 별도로 산정하는 비용이 있는 경우에는 각각의 산정방식에 의하여 산정된 금액을 합산한다
\end{enumerate}
\begin{enumerate}[13.]\tightlist
\item  질병군 요양급여를 실시하는 요양기관은 \textcolor{red}{질병군 입원환자의 질병군 분류 번호와 관련한 주진단 및 기타진단, 수술명 등은 진료기록부에 근거하여 정확한 코드를 부여}하여야 하며, \textcolor{red}{진단명이 입원시부터 존재하였는지 여부를 확인할 수 있도록 진료기록부에 기록}하고, 의료의 질 향상을 위한 점검표를 별지 서식에 따라 작성하여야 한다
\end{enumerate}
\end{myshadowbox}
\prezi{\clearpage}
\par
\medskip

\Que{급성출혈 후 빈혈(D62), 분만 후 출혈(O72), 분만 중 출혈(O67)의 진단분류기호 부여기준 중 “분만(수술)전(입원당시)” 의 의미는?}입원하여 분만(수술) 전 시행한 혈액검사 및 통상 외래에서 분만(수술)전 시행한 검사를 의미함
\Ans{입원하여 분만(수술) 전 시행한 혈액검사 및 통상 외래에서 분만(수술)전 시행한 검사를 의미함}
\prezi{\clearpage}
\par
\medskip
\Que{항문수술 후 퇴원한 환자가 수술합병증으로 15일 이내에 재입원하여 재수술을 시행한 경우 질병군 적용 대상여부}
\Ans{수술합병증으로 내원한 경우 「한국표준질병ㆍ사인분류」의 “달리 분류되지 않은 처치의 합병증(T81)”이 주진단으로 항문수술 질병군의 주진단 범주에 속하지 않으므로 DRG 대상이 아님(\textcolor{red}{행위별청구 대상})\par
☞ DRG분류는 주진단, 외과계 시술 등에 의해 결정되며 「질병군 급여ㆍ비급여 목록 및 급여 상대가치점수」 제3부 질병군 분류번호 결정요령의「질병군 범주의 결정 및 그 분류번호(별표3)」참조}
\prezi{\clearpage}
\par
\medskip
\Que{전치태반 또는 전자간증 등이 있는 임신부가 제왕절개술을 시행한 경우 주진단 부여원칙은?}
\Ans{제왕절개술을 받은 경우 선택적이던 응급이던 상관없이 \textcolor{red}{제왕절개술을 받은 이유를 주된 병태로 우선 부여함}. 제왕절개술을 받은 이유가 불명확할 경우 O82 제왕절개에 의한 단일분만 코드를 주된 병태로 부여함 \par
 ☞ 관련근거 : 한국표준질병ㆍ사인분류 질병 코딩지침서 (P.113) O80 ~ O84의 분류}
\prezi{\clearpage}
\par
\medskip
\Que{질병군 대상 수술 후 합병증으로 패혈증 등이 발생한 경우 합병증을 주진단으로 선정할 수 있는지?}
\Ans{진료 개시 후 주된 병태와 관련된 질환이나 합병증이 발생하였을 경우에는 이로 인한 자원소모가 많다고 할지라도 기타진단으로 부여함(기존 주된 병태를 주진단으로 적용) \par
 ☞ 관련근거: 한국표준질병ㆍ사인분류 질병 코딩지침서 (P.2~4) 주된병태 선정원칙}
\prezi{\clearpage}
\par
\medskip
\Que{분만 후 출혈(O72) 상병 부여시 자원소모가 있어야 하나요?}
\Ans{분만 후 출혈 상병을 부여할 수 있는 기준은 분만전(입원당시)에 실시한 혈액검사 결과와 비교하여 Hct가 10\%이상 감소했거나 수혈이 필요하여 수혈을 실시한 경우에 가능합니다}
\prezi{\clearpage}
\par
\medskip
\Que{Hct만 단지 떨어져 있다는 이유로 기타진단이 될수 있는지?}
\Ans{분만전 Hct 37.9\%, 제왕절개분만 후 Hct 34.0\% 로 10\% 이상 Hct 감소한 경우이나 환자의 심신상태 등이 양호하여 특별한 처치ㆍ치료를 필요로 하지 않는 경우 O72 분만 후 출혈을 기타진단으로 코딩함은 오류임(복지부고시 「기타진단부여기준(별표8)」에 맞지 않음)}
\prezi{\clearpage}

\par
\medskip
\Que{질병군 분류번호를 결정하는 주된 수술을 양측으로 실시한 경우 질병군 요양급여비용 산정방법}
\Ans{질병군 분류번호를 결정하는 주된 수술을 양측으로 실시한 경우 ‘수정체수술 질병군’ 과 ‘서혜 및 대퇴부 탈장수술 질병군’은 양측 수술 질병군으로 분류되며, ‘편도 및 아데노이드절제술 질병군’과 ‘자궁부속기수술 질병군’은 편측ㆍ양측 수술에 불문하고 해당 질병군의 소정점수를 적용하여야 함
또한, 질병군 분류번호를 결정하는 주된 수술을 양측으로 실시한 경우 양측 수술 질병군 \textcolor{red}{분류여부에 관계없이 수술료를 추가 산정할 수 없음}}
\prezi{\clearpage}
\par
\medskip
\Que{질병군 분류번호를 결정하는 \textcolor{red}{주된 수술 이외에 대칭기관의 양측수술을 실시한 경우} 수술료 추가산정방법}
\Ans{질병군 분류번호를 결정하는 주된 수술 이외에 대칭기관의 양측수술을 실시한 경우 질병군의 소정점수 이외에 대칭기관의 양측 수술료를 추가 산정함. 다만 양측수술 불문하고 해당 소정점수를 산정토록 한 경우에는 해당하지 않음\par
\begin{description}\tightlist
\item[(예시1)] 편도전적출술(자 -230)과 양측 하비갑개절제술(자 -101)을 동시 시술한 경우 ⇒ 편도 및 아데노이드절제술 질병군 급여상대가치점수와 하비갑개절제술의 수술료 200\% 산정
\item[(예시2)] 충수절제술(자 -286)과 양측 *위축성비염수술(자 -98)을 동시 시술한 경우 ⇒ 충수절제술 질병군 급여상대가치점수와 위축성비염수술의 수술료 100\% 산정
\item[*] 자98 위축성비염수술(양측): 양측을 시술할지라도 소정점수만 
\end{description}
}
\prezi{\clearpage}
\begin{shaded}
기타진단의 중요도는 기타진단에 의해서 중등도가 결정되게 됩니다. 예를 들어보면, severe PIH로 제왕절개를 한경우에 주진단으로 severe PIH 한가지만 내 놓으면 기본적인 DRG금액만 받을수 있지만, CPD를 주진단으로 하고 severe PIH, hemorrhage등등의 기타진단을 기입하게 되면 severity등급에 따라서 최소 9만원이상의 금액을 더 받을수 있습니다. 실제로 따져봐도 severe PIH는 직접적으로 제왕절개를 해야할 주된 이유는 아닙니다. CPD나 failure to progression등이지요!!\\
to be continue--
\end{shaded}
\prezi{\clearpage}
