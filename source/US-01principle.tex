\section{초음파 급여화 원 칙}
초음파 검사는 다음과 같은 경우에 요양급여하며 , \highlight{이에 해당하지 않는 경우에는 비급여함} \par

\textsf{- 다 음 -}
\begin{enumerate}[1.]\tightlist
\item 급여대상 및 범위
	\begin{enumerate}[가.]\tightlist
	\item 기본 , 진단 , 특수 초음파
		\begin{enumerate}[1)]
		\item 암, 심장질환 , 뇌혈관질환 , 희귀난치성 질환
			\begin{enumerate}[가)]\tightlist
			\item 「본인일부부담금 산정특례에 관한 기준」 에 따른 산정특례 대상자 : 해당 산정특례 적용기간에 실시한 경우
			\item 산정특례 질환이 의심되는 환자: 해당 산정특례 질환이 의심되어 실시한 경우(1회 인정)%\footnote{산정특례 대상에서의 초음파 급여나 할인은 해당 질병과 해당질환에 의한 합병증에 대해서만 입니다. 즉 유방암환자는 유방에 대한 진료와 항암치료제인 tamoxifen의 사용에 의한 자궁내막암의증인 경우입니다. 여성생식기 초음파는 여성생식기 암 환자만 급여가 됩니다.}
			\end{enumerate}
		\item 신생아 중환자실 환자 : 신생아 중환자실 입원기간에 실시한 경우
 		\end{enumerate}
	\item 임산부 초음파
		\begin{enumerate}[1)]
		\item 산전진찰을 목적으로 아래와 같이 시행하는 경우에 인정하며, 다태아의 경우 
	제2태아부터는 소정점수의 50\%를 산정함.(나951나(1) ‘주’항 제외)\footnote{2,3분기 고위험초음파상 2태아부터는 고위험산정제외}
		\end{enumerate}
	\end{enumerate}
\end{enumerate}	
\prezi{\clearpage}
\begin{center}\emph{- 아 래 -}\end{center}
\begin{myshadowbox}
\begin{itemize}\tightlist
\item 4대 중증질환자) 진단 목적 초음파 횟수제한 삭제, 기본 / 유도초음파 급여전환
\item (4대 중증질환 의심자) 질환이 의심되어 실시한 경우 1회
\item (임산부) 산전진찰 목적 초음파 급여 전환
	\begin{itemize}\tightlist
	\item 정상산모: 임신 주수 별 해당횟수 급여(7회), 횟수 초과시 비급여
	\item 태아의 이상이 있거나, 이상이 예상될 경우 추가 급여 적용
	\end{itemize}
\item (신생아중환자실) 입원기간 동안 시행한 모든 초음파
\end{itemize}
\end{myshadowbox}

