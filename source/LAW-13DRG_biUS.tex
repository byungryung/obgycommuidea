\begin{myshadowbox}
\begin{enumerate}[15.]\tightlist
\item 질병군 진료 시 초음파검사는 「요양급여의 적용기준 및 방법에 관한 세부사항」제2장 검사료 초음파검사 세부인정기준을 적용하며, \highlight{인정기준에 의한 급여대상에 해당되는 경우에는} 제2부 각 장에 분류된 질병군 점수이외에 제1편 제2부 초음파검사료를 추가 산정한다. 
\end{enumerate}
\end{myshadowbox}
\prezi{\clearpage}
\begin{itemize}\tightlist
	\item EZ986 분만기간 초음파 : 분만을 위한 입원기간 동안 발생한 초음파 검사를 모두 의미함. 제왕절개를 위해 입원한 환자들의 경우는 옆의 비급여초음파를 최소한 2번 이상 실시하고 청구한다.
	\item EZ887 초음파를 이용한 태아 생물리학 계수( Biophysical Profile )
	\item 임신 유지목적으로 입원하여 6일 이내에 제왕절개분만이 이루어진 경우 : 분만기간 초음파(비급여)로 청구한다. 조산통으로 입원한 경우엔 2일에 한번씩은 초음파를 본다.
	\item 임신 유지목적으로 입원하여 입원일수가 6일을 초과한 시점에서 예상치 못하게 제왕절개분만이 이루어진 경우 
		\begin{itemize}\tightlist
		\item 입원(행위별 청구) : (정상임신부) 7회까지 급여, 그 외 비급여(태아의 이상이나 이상이 예측되는 경우) 급여
		\item 분리청구 시점 구분
		\item 제왕절개분만 입원(DRG 청구) : 분만기간 초음파(비급여)
         \end{itemize}                               
	\item 분만과 연결된 입원: 분만기간이 장기로 길어진 경우 분리청구 시점 기준으로 적용
		\begin{itemize}\tightlist
		\item 입원(행위별 청구) : (정상임신부) 7회까지 급여, 그 외 비급여(태아의 이상이나 이상이 예측되는 경우) 급여
		\item 분리청구 시점 구분
		\item 자연분만및 제왕절개분만 입원제왕절개분만 입원(DRG 청구) : 분만기간 초음파(비급여)
         \end{itemize} 	
%	\item 질병군 진료 이외의 목적으로 입원하여 입원일수가 6일을 초과한 시점에 예상치 못하게 질병군 수술이 이루어진 경우 입원일로부터 수술시행일 전일까지의 진료분을 제외한 경우의 보험 초음파등(6일전의 조기진통등으로 입원하여 제왕절개분만한 경우 횟수초과 급여초음파 청구 (해당 삼분기의 일반 또는 일반의 제한초음파 산정). 단, 1일 1회만 청구 가능함)
\end{itemize}
\prezi{\clearpage}
\par
\medskip
\Que{○○종합병원에 충수암 의증으로 입원한 환자가 2014년 2월 1일 초음파검사 등을 실시 후 충수암 진단으로 충수절제술을 시행한 경우 급여대상인 초음파검사의 특정내역 기재방법은?}
\Ans{
\begin{itemize}\tightlist
\item 초음파검사 비용의 추가 산정은 특정내역 MT007의 내역구분 ‘SON'으로 기재
\item 특정내역 기재형식 및 설명
	\begin{itemize}\tightlist
	\item X(3)/ccyymmdd/X/X(9)/9(10)/9(5).V9(2)/9(3)/9(10)/X(200)/X(1)/X(100)
	\item 내역구분/투여(실시)일자/코드구분/코드/단가/1일투여량(실시횟수)/총투여일수(실시횟수)/금액/준용명/면허종류/면허번호
	\end{itemize}
\item ※ 초음파검사가 급여대상이나 산정횟수를 초과하는 경우에는 특정내역 MT007 내역구분 ‘All'(보훈환자의 경우 ’100‘)에 기재
\end{itemize}}