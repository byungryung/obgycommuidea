\subsection{기타 QA}
\begin{commentbox}{‘요양급여비용열외군’이란?}
진료비가 예외적으로 많이 발생한 질병군 진료건에 대해 추가 보상을 하기 위한 제도로 질병군으로 산정한 요양급여비용 총액이 행위별로 산정한 금액보다 적고 그 차액이 100만원을 초과하는 경우 초과 금액을 추가로 지급함
\begin{itemize}\tightlist
\item 열외군보상금액 = (행위별총진료비 - DRG 총진료비)-100만원
\item 예시
	\begin{itemize}\tightlist
	\item  DRG 총진료비: 100만원
	\item 행위별 총진료비: 400만원
	\item DRG 요양급여비용 총액: 300만원(200만원 추가지급) 
	\end{itemize}
\end{itemize}	
\end{commentbox}
\prezi{\clearpage}
\Que{‘요양급여비용 열외군’ 환자인 경우 ‘행위별 총진료비’ 산정방법은?}
\Ans{‘요양급여비용 열외군’ 환자인 경우 ‘행위별 총진료비’를 산정방법은 다음과 같음\par
\begin{center}\emph{- 다 음 -}\end{center}
\begin{enumerate}[가)]\tightlist
\item 행위별수가제의 100/100본인부담 및 비급여사항을 포함하되, 질병군에서 환자에게 별도 징수가 가능한 이송처치료, PCA(통증자가조절법)와 비급여대상 등은 제외
\item 행위는 상대가치점수표에서 정한 기준에 의해 산정
\item 약제 및 치료재료는 약제급여목록 및 급여상한금액표 또는 치료재료급여ㆍ비급여목록 및 급여상한금액표의 상한금액을 초과하지 않는 범위내에서 실구입가로 산정 
\item 질병군의 급여범위에 해당되나 행위별에서의 비급여대상에 해당하는 행위는 해당 요양기관의 관행수가를 적용하여 산정하고 약제 및 치료재료는 실구입가로 산정 ⇒ 위 가~라에 의거 계산된 금액을 행위별 총진료비란에 기재\newline
☞ 요양급여비용 심사청구서·명세서 세부작성요령 (Ⅸ 질병군요양급여비용 작성요령)
\end{enumerate}
}
\prezi{\clearpage}
\begin{commentbox}{질병군 진료기간 중 질병군 수술과 전혀 다른 상병에 대해 협진 시 다른 상병으로 인한 진료비용의 별도산정 여부}
질병군 급여비용에는 동반상병 및 \textcolor{red}{합병증 진료에 대한 비용이 포함되어 있음. 
다만, 동반상병 및 합병증 진료에 따른 자원소모 등을 고려하여 기타진단을 진단명부여원칙에 따라 코딩}하는 경우 진단명에 따라 중증도 분류가 다르게 적용될 수 있음
\end{commentbox}
\prezi{\clearpage}
\begin{commentbox}{질병군 진료기간 중 수혈을 실시한 환자가 현혈증을 제시한 경우 청구방법은?}
질병군 진료기간 중 수혈을 실시한 경우 질병군 상대가치점수는 수혈비용 등을 포함하여 산출된 것이므로 환자에게 별도로 수혈비용을 부담시킬 수 없음
또한,「헌혈증서 소지자의 수혈비용에 대한 본인부담 산정방법」(급여 65720-1898호, '01.12.29.) 에 따라 질병군에 대한 본인부담액에서 헌혈증서에 의한 대한적십자사의 보상금액을 공제 후 환자에게 징수하고 행위별청구와 동일하게 수혈비용을 대한적십자사총재에게 청구함
\end{commentbox}