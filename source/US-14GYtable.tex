\begin{table*}
\setlength\tabulinesep{3pt}
\captionsetup{labelformat=empty}
\caption{여성생식기 초음파 및 유도초음파}
\begin{threeparttable}
\begin{tabu}{X|X|X[5.5]|X}
\tabucline[.5pt]{-}
\rowfont{\sffamily} 분류 번호 & 코 드 & 분 류 & 점수 \\\hline
		나-944 & & 라. 여성생식기 초음파 \newline Female Genital Ultrasonography & \\
	   & EB455 & (1) 일반 General\tnote{1} \highlight[red!70]{병원} 75,200(15,040)  \highlight[cyan!70]{의원}78,740 (7,870) & 866.76 \\
	   & EB455001 &  \hspace{.5cm}\highlight{일반/제한적)} \highlight[red!70]{병원}37,600(7,520)  \highlight[cyan!70]{의원}39,380 (3,940) & 433.38 \\
	   & EB455010 &  \hspace{.5cm}\highlight{일반/도플러} \highlight[red!70]{병원}82,720(16,540)  \highlight[cyan!70]{의원}86,620 (8,660) & 953.44 \\
	   & EB455011 &  \hspace{.5cm}\highlight{일반/도플러/제한적} \highlight[red!70]{병원}45,190(9,040)  \highlight[cyan!70]{의원}43,310 (4,330) & 476.72 \\
	   
	   & EB457 & (2) 정밀 Detailed \highlight[red!70]{병원}110,190(44,080)  \highlight[cyan!70]{의원}115,390 (34,610) &  1,270.03\\	
	   & EB457001 &  \hspace{.5cm}\highlight{정밀/제한적} \highlight[red!70]{병원}55,090(22,040) \highlight[cyan!70]{의원}57,690 (5,770) &  635.02\\
	   & EB457010 &  \hspace{.5cm}\highlight{정밀/도플러} \highlight[red!70]{병원}121,210(23,240)  \highlight[cyan!70]{의원}126,920 (12,690) &  1397.03\\
	   & EB457011 &  \hspace{.5cm}\highlight{정밀/도플러/제한적} \highlight[red!70]{병원}60,600(12,120)  \highlight[cyan!70]{의원}63,460 (6,350) & 698.52 \\
		나-956 & & 유도초음파 \newline Guiding Ultrasonography For Procedure & \\
%	   & EB561 & 가. 유도초음파(I) & 443.90 \\
	   & EB562 & 나. 유도초음파(II) \highlight[red!70]{병원}77,030(15,410) \highlight[cyan!70]{의원}80,660 (8,070) &   887.80\\	   
	   & EB563 & 다. 유도초음파(III) \highlight[red!70]{병원}92,430(18,490) \highlight[cyan!70]{의원}96,790 (9,680) & 1,065.36 \\
%	   & EB564 & 라. 유도초음파(IV) & 2,663.40 \\	

\tabucline[.5pt]{-}
\end{tabu}
{\small
\begin{tablenotes}
\item[1] EB456 :주: 자궁내 생리식염수를 주입하여 검사한 경우에는 1,108.73점을 산정한다.
\item[2] 4대 중증 질환자의 경우 2017년 1월부터 외래본인부담금 병원 40→20\%, 의원 30→10\%
\end{tablenotes}
}
\end{threeparttable}
\end{table*}

\subsection*{유도초음파 II}
\begin{itemize}\tightlist
\item C8572	자궁내막조직생검-구획소파생검
\item C8573	자궁내막조직생검-흡인생검
\item C8574	자궁내막조직생검-단순소파생검
\item C8575	자궁내막조직생검-자궁경내소파술
\item C8591	갑상선생검-침생검
\item M0031	피부 및 피하조직, 근육내 이물제거술[봉침, 파편 등]-근막절개하 이물제거술
\item M0032	피부 및 피하조직, 근육내 이물제거술[봉침, 파편 등]-기타
\item R4028	자궁내풍선카테터충전술[자궁용적측정포함]
\item R4103	질식배농술-질벽혈종제거
\item R4271	자궁내장치삽입술
\item R4277	자궁내장치제거료(실이보이지않는경우)-기타의경우
\item R4441	계류유산소파술-12주미만
\item R4442	계류유산소파술-12주이상
\item R4460	태아축소술
\item R4521	자궁소파수술
\end{itemize}

\subsection*{유도초음파 III}
\begin{itemize}\tightlist
\item R4016	양막내양수주입술
\item R4182	자궁내반증수술-용수정복
\item R4435	난소낭종 또는 난소농양 배액술[질부접근]
\end{itemize}
\clearpage
\section{인접부위 초음파}
인접된 부위에 초음파 검사를 동시에 시행하는 경우
\begin{enumerate}[①]\tightlist
\item 주된 검사는 소정점수의 100\%, 제2의 검사는 소정점수의 50\%를 산정하며, 최대 150\%까지 산정함(고시 제 2016-175호)
\item 복지부 \textsf{Q\&A} 32번 인접부위 기준에 (8) 나944라 여성생식기 초음파 /나951 임산부 초음파가포함되어 있음
\item 임산부 초음파 검사 중 자궁근종, 난소낭종 등 검사 시여성생식기 초음파 검사의 급여기준의 급여대상(4대 중증질환자 및 의심자) 및 범위에 해당하면 급여로 여성생식기 초음파50\% 청구 가능, 급여대상 및 범위에 해당하지 않으면 여성생식기 초음파는 비급여로 청구(고시 또는 \textsf{Q\&A}에 없는 내용)
\end{enumerate}