\subsection{제1장 기본진료료〔산정지침〕}
\begin{enumerate}[1.]\tightlist
\item 진찰료
	\begin{enumerate}[가.]\tightlist
	\item 진찰료는 외래에서 환자를 진찰한 경우에 처방전의 발행과는 관계없이 산정하며 초진환자를 진찰하였을 경우에는 초진진찰료, 재진환자를 진찰하였을 경우에는 재진진찰료를 산정한다.
		\begin{enumerate}[(1)]\tightlist
		\item 진찰료는 기본진찰료(초진의 경우 AA154-AA157은 155.57점, AA100, AA109는 152.11점, 10100은 152.06점, 재진의 경우 AA254-AA257은 98.03점, AA200, AA209, 10200은 95.98점)와
외래관리료(진찰료에서 기본진찰료를 제외한 점수)의 소정점수를 합하여 산정한다.
		\item \uline{초진환자란 해당 상병으로 동일 의료기관의 동일 진료과목 의사에게 진료받은 경험이 없는 환자를 말한다.}
		\item \uline{재진환자란 해당 상병으로 동일 의료기관의 동일 진료과목 의사에게 계속해서 진료받고 있는 환자를 말한다.}
		\item \uline{해당 상병의 치료가 종결되지 아니하여 계속 내원하는 경우에는 내원 간격에 상관없이 재진환자로 본다. 또한, 완치여부가 불분명하여 치료의 종결 여부가 명확하지 아니한 경우 90일 이내에 내원시 재진환자로 본다.}
		\item 해당 상병의 치료가 종결된 후 동일 상병이 재발하여 진료를 받기 위해서 내원한 경우에는 초진환자로 본다. 다만 치료종결 후 30일 이내에 내원한 경우에는 재진환자로 본다.
		\item 치료의 종결이라 함은 해당 상병의 치료를 위한 내원이 종결되었거나, 투약이 종결되었을 때로 본다.
		\item 진찰료 중 기본진찰료는 병원관리 및 진찰권발급 등, 외래관리료는 외래환자의 처방 등에 소요되는 비용을 포함한다.
		\end{enumerate}
	\item 다음 각 호의 1에 해당하는 경우에는 \uline{진찰료는 1회 산정한다.}
		\begin{enumerate}[(1)]\tightlist
		\item \uline{동일 의사가 동시에 2가지 이상의 상병에 대하여 진찰을 한 경우}
		\item 하나의 상병에 대한 진료를 계속 중에 다른 상병이 발생하여 동일 의사가 동시에 진찰을 한 경우(재진진찰료)
		\item \uline{동일한 상병에 대하여 2인 이상의 의사가 동일한 날에 진찰을 한 경우}
		\end{enumerate}
	\item \uline{2개 이상의 진료과목이 설치되어 있고 해당 과의 전문의가 상근하는 요양기관에서 동일환자의 다른 상병에 대하여 전문과목 또는 전문 분야가 다른 진료담당 의사가 각각 진찰한 경우에는 진찰료를 각각 산정할 수 있다.}
	\item \uline{진료담당의사가 검사\cntrdot{} 방사선 진단 등을 처방지시하였으나 요양기관의 사정에 의하여 진료 당일에 검사\cntrdot{} 방사선 진단 등을 실시하지 못한 경우에는 검사\cntrdot{} 방사선 진단을 실시한 당일의 진찰료는 산정하지 아니한다.}
	\item 의료법 제18조에 따라 요양기관인 의료기관의 의사 또는 치과의사가 작성\cntrdot{} 교부한 처방전에 따라 요양기관인 약국 또는 한국희귀의약품센터에서 \uline{조제받은 주사제를 투여받기 위해서 당해 요양기관에 당일에 재내원하는 경우에는 진찰료를 별도 산정하지 아니한다.}
	\end{enumerate}
\item 입원료등(입원료\cntrdot{} 무균치료실입원료\cntrdot{} 낮병동입원료\cntrdot{} 신생아입원료\cntrdot{} 중환자실입원료\cntrdot{} 격리실입원료\cntrdot{} 납차폐특수치료실입원료)
	\begin{enumerate}[가.]
	\item 입원료 등의 소정점수에는 입원환자 의학관리료(소정점수의 40\%),입원환자 간호관리료(소정점수의 25\%), 입원환자 병원관리료(소정점수의 35\%)가 포함되어 있으며 요양기관 종별에 따라 산정한다.
	\item 입원료 등을 산정하기 위해서는 국민건강보험법 제43조 및 동법 시행규칙 제12조에 따라 요양기관의 병실 및 병상 현황을 신고하여야 한다.
	\item 무균치료실입원료, \uline{낮병동입원료, 신생아입원료}, 중환자실입원료,격리실입원료, 납차폐특수치료실입원료 등 \uline{특수병실 입원료를 산정 할 수 있는 경우는 다음과 같으며} 특수병실 입원료를 산정하는 경우에는 입원료 등을 중복하여 산정하지 아니한다.
		\begin{enumerate}[(1)]\tightlist
		\item 무균치료실 입원료:조혈모세포이식환자를 조혈모세포이식의 요양급여에 관한 기준 제3조제2항제1호의 기준에 적합한 무균치료실에격리하여 치료한 경우
		\item \uline{낮병동 입원료}
			\begin{enumerate}[(가)]\tightlist
			\item 다음 각 호의 1에 해당하는 경우 \uline{1) 분만 후 당일 귀가 또는 이송하여 입원료를 산정하지 아니한 경우, 2) 응급실, 수술실 등에서 처치\cntrdot{} 수술 등을 받고 연속하여 6시간 이상,관찰 후 귀가 또는 이송하여 입원료를 산정하지 아니한 경우, 3) 정신건강의학과의 “낮병동”에서 6시간 이상 진료를 받고 당일 귀가한 경우}
			
			\item 낮병동 입원료를 산정하는 당일 외래 또는 응급실에서 진찰을 행한 경우에는 진찰료를 함께 산정할 수 있다. \uline{다만, 예정된 외래 수술을 위해 내원하는 경우 또는 정신건강의학과의 “낮병동”에서 매일 또는 반복하여 진료를 받는 경우에는 진찰료를 산정하지 아니한다.}
			\item \uline{낮병동 입원료를 산정하는 당일의 본인일부부담금은 입원진료 본인일부부담률에 따라 산정한다.}
			\end{enumerate}
		\item \uline{신생아 입원료:질병이 없는 신생아를 신생아실(신생아실 입원료) 또는 모자동실(모자동실입원료)에서 진료\cntrdot{} 간호한 경우}
		\item 중환자실 입원료 :「의료법」시행규칙 제34조 [별표4]에서 정한 중환자실의 시설\cntrdot{} 장비를 갖춘 중환자실(ICU)이 설치된 상급종합 병원, 종합병원, 병원에서 지극히 심각한 질환이나 손상을 입어 집중적인 치료 및 간호가 필요한 성인 또는 소아환자(\uline{일반중환자실 입원료}또는 소아 중환자실 입원료) 또는 신생아(신생아 중환자실 입원료)를 중환자실 에서 진료한 경우
		\item 격리실 입원료:다음 각 호의 1에 해당하는 경우. 다만, 당해 전염성 환자만을 수용하는 요양기관에서는 입원료로 산정한다.
			\begin{enumerate}[(가)]\tightlist
			\item 면역이 억제된 환자를 보호하기 위하여 일반 환자와 격리하여 치료한 경우			\item 일반 환자를 보호하기 위하여 전염력이 강한 전염성 환자를 일반 환자와 격리하여 치료한 경우
			\item 3도 이상으로 36\% 범위 이상의 화상환자를 진료에 반드시 필요하여 격리하여 치료한 경우
			\item 기타 보건복지부장관이 반드시 격리가 필요하다고 인정하여 고시하는 경우
			\end{enumerate}
		\end{enumerate}	
	\end{enumerate}
\end{enumerate}

\Que{HRT 3달치 처방받고 다음에도 계속 HRT처방을 받기위해 방문하시는분은 초진 or 재진 ?}
\Ans{\highlight{재진}. 해당 상병의 치료가 종결되지 아니하여 계속 내원하는 경우에는 내원 간격에 상관없이 재진환자임}
\Que{HRT시 페경기전후 장애 상병쓰는데, 이것도 만성질환으로 들어가나요? 환자분이 안젤릭 한달 처방받고 그 다음 달에 오셔서(한달 조금 넘어) 초진으로 되었은데 처방내리니 이번 청구에서 재진으로 삭감조정되더라구요.. 그래서 폐경치료도 만성질환으로 들어가는 지 궁금해서요?}
\Ans{재진}
\Que{HRT관리 받으시는 분이 혈액검사상 고지혈증이 관찰되어 치료하는 경우는? 초진 or 재진} \index{조\cntrdot 재진}
\Ans{\highlight{재진}
\begin{itemize}\tightlist
\item 하나의 상병에 대한 진료를 계속 중에 다른 상병이 발생하여 동일 의사가 동시에 진찰을 한경우
\item 즉 \emph{상병을 두개 넣은 경우네는 재진에 해당}
\item 또는 상병을 고지혈증만 넣는다 해도 30일 이내라면 재진
\end{itemize}
\highlight{초진}
\begin{itemize}\tightlist
\item 고지혈증 \emph{상병만 넣고} N951 상병으로 진찰 받은지 한달이 넘은 경우.
\end{itemize}}

\Que{2월 1일 질염환자가 3월1일에 다시 질염으로 재발되어 왔다면? 초진 or 재진}
\Ans{\highlight{초진}. 해당 상병의 \emph{치료가 종결된 후 동일 상병이 재발하여 진료를 받기 위해서 내원한 경우에는 초진후 30일 이내에 내원한 경우에는 재진환자}로 본다. 치료의 종결이라 함음 해당 상병의 치료를 위한 내우언이 종경되었거나, 투약이 종결되었을때를 말함}

\Que{한달 이내에 트리코모나스 질염과 캔디다 질염으로 각각 다른 상병으로 내원하면? 초진 or 재진}
\Ans{\highlight{재진.}하나의 상병에 대한 진료를 계속중에 다른 상병이 발생하여 동일 의사가 동시에 진찰을 한 경우(재진진찰료)}

\Que{엄청나 산부인과의 엄청난 선생님에게 진료를 본 환자가 한달이 채 안되어서 같은 산부인과 더엄청난 선생님이 다른 상병으로 진료 받는 경우?}
\Ans{\highlight{초진}
\begin{itemize}\tightlist
\item 재진의 조건 : 같은 진료과목, 같은 병원이지만 같은 상병이 아니므로 초진
\item 초재진 구분의 3대 기준 
	\begin{enumerate}\tightlist
	\item 상병이 같으냐? 다르냐?
	\item 같은 병원이냐? 아니냐?
	\item 같은 진료과목 의사이냐? 아니냐?[같은 선생님이냐? 아니냐가 아닙니다]
	\end{enumerate}
\end{itemize}}