\subsection{병실료등}
\begin{myshadowbox}
\begin{enumerate}[8.]\tightlist
\item 가입자 또는 피부양자가 제1호에 따른 요양기관(제3편을 적용받는 요양병원은 제외)에서 「국민건강보험법」제43조에 따라 신고한 일반입원실 및 정신과폐쇄병실의 4인실 또는 5인실을 이용한 경우에는 별표 2의3의 추가 비용 계산식에 따른 금액을 추가 산정하고, 상급종합병원의 일반입원실 및 정신과폐쇄병실의 1인실(보건복지부장관이 정하여 고시하는 불가피한 1인실 입원의 경우 제외)을 이용한 경우에는 제5호 본문에 따른 금액에서 1인실 이용일수에 해당하는 기본입원료(제1편(행위별 수가)제2부제1장 가-2-가)를 제외하고 산정한다.
\item 영 별표 2 제2호 나목의 “보건복지부장관이 정하여 고시하는 입원실을 이용한 경우”라 함은 가입자 등이 제1호에 따른 요양기관에서 국민건강보험법 제43조에 따라 신고한 일반입원실 및 정신과폐쇄병실의 4인실 또는 5인실을 이용한 경우를 말하며, 별표 2의3의 본인부담액 계산식에 따른 금액을 더 하여 본인부담액을 산정한다.
\end{enumerate}
\end{myshadowbox}
\prezi{\clearpage}
\Que{4인실 또는 5인실 이용에 따른 추가비용 산정 시 간호등급이나 입원료 체감제 등도 적용되는지?}
\Ans{4인실 또는 5인실입원료는 제1편(행위)의 종별에 따른 입원료(가-2)를 말하며, 입원료관련 가산 또는 감산은 적용하지 않으므로 간호등급이나 입원료체감제 등은 \textcolor{red}{적용하지 않음}\par
☞ (별표 2의3) ‘주1’ 참조}
\prezi{\clearpage}
\par
\medskip
\Que{본인부담경감대상자의 4인실 또는 5인실 이용에 따른 추가비용 산정 시 본인부담률 적용 방법은?}
\Ans{「국민건강보험법시행령」 별표2 제3호에 해당하는 대상자인 경우에는 그 각목에서 정한 본인부담률을 적용함 (중증질환 5\%, 희귀난치질환 10\%, 6세미만 10\%, 신생아 면제 등)\par
☞ (별표2의3) ‘주2’ 참조}
\prezi{\clearpage}
\par
\medskip
\Que{상급병실 입원료 개정 관련 신설된 심사불능 코드 항목은?}
\Ans{
\begin{description}\tightlist
\item[코드] 60-36 \emph{내용} 질병군(DRG)입원료 산정과 요양기관 현황신고내역 불일치 * '14.9.10 이후 접수분 적용
\item[코드] 60-37 \emph{내용} 질병군(DRG)입원료 산정착오 또는 기재착오 *'14.9.1 이후 접수분 적용
\end{description}
}