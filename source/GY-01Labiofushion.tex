\section{음순 유착 해리술}
\begin{paracol}{2}
\setlength{\columnseprule}{0.4pt}
\setlength{\columnsep}{2em}
\begin{leftcolumn}
\begin{commentbox}{}
\begin{itemize}\tightlist
\item[\dsjuridical] Q525 음순의 유착 
\item[\dsmedical] R4041 [\myexplfn{1273.17} 원] 
\end{itemize}
\end{commentbox}
\medskip
\centering

\includegraphics[width=0.75\linewidth]{labial-fusion}
\end{leftcolumn}

\begin{rightcolumn}
음순유착 해리술이란? \par
음순 유착(labial agglutination or adhesions)은 주로 사춘기전 낮은 에스트로겐 농도와 만성 염증에 의한 자극으로 소음순이 음핵 뒤에서 음순 소대 앞까지 붙어있는 경우로 에스트로겐제제의 크림 도포와 함께 외음부를 위생적으로 깨끗이 유지하면 대개 2-3주 내에 정상으로 돌아오는 경우가 많다. 그러나 음순 유착 환아를 경험해 보지 않으면 음순 유착을 내부 생식기 기형까지 동반된 선천성 기형으로 오인하고 여러 가지 검사를 시행할 우려가 있다. 대부분 에스트로겐 크림의 국소 도포와 도수리(manual separation)로 부작용 없이 호전을 보임
\end{rightcolumn}
\end{paracol}
\subsection{에스트로겐 크림 대체재}
현재 에스트로겐크림이 나오지 않음으로 인해서 여러 가지 방법이 시도 되고 있습니다.
\begin{enumerate}[①]\tightlist
\item 에스트로겐 질정을 녹여서 손으로 우깨서 사용한다.
\item 스테로이드 크림으로 쓴다.
\end{enumerate}