\section{KDRG (입원환자분류체계) : Korean Diagnosis Related Group}
\leftrod{질병군 분류란 ?}
\par
\medskip   
질병군 분류(진단명 기준 환자군, Diagnosis Related Group(DRG))는 입원 환자를 자원소모 유사성과 임상적 유사성에 기초하여 분류하는 입원환자 분류체계이다.
\prezi{\clearpage}
\leftrod{질병군 분류번호는 ?}
\par
\medskip
질병군 분류번호는 총 6자리이며, 첫 4자리는 질병군범주, 5번째 자리는 연령구분, 6번째 자리는 합병증 및 동반상병(기타진단)에 의한 분류 (중증도)이다. \par
\begin{center}
\includegraphics[width=.95\textwidth]{DRGprocess}
\end{center}
\par
\medskip
\prezi{\clearpage}
\subsection{주진단 범주(MDC) 분류}
\begin{description}\tightlist
\item[\cntrdots{}] A - M
\item[13 N] 여성 생식기계의 질환및 장애\index{severity}
\item[14 O] 임신,출산, 산욕
\end{description}
\prezi{\clearpage}
\subsection{질병군 분류번호 결정의 이해}
\begin{itemize}\tightlist
\item 환자가 수술을 받았는지 여부에 따라 외과계와 내과계 질병군으로 구분되며, 외과계 질병군은 환자가 받은 수술에 따라 질병군이 결정되고 내과계 질병군은 주진단명에 의해 결정된다.
\item 외과계 그룹은 시술명에 따라 세분화되며, 한 환자가 동일 입원기간 내에 여러 시술을 받은 경우 ‘외과적 우선순위(별표7 질병군범주 우선 순위)’에 따라 우선순위가 가장 높은 외과 질병군으로 배정된다.
\item 개복이나 내시경수술(복강경이나 흉강경)의 구분 및 단측과 양측 등 질병군 분류의 구분이 필요한 경우에는 부가코드(ADC)를 이용하여 질병군을 결정한다.
\item 주진단과 수술에 따라 ADRG(질병군 분류번호 4째자리)까지 분류한 다음, 필요시 연령에 따라서 ADRG를 추가로 세분화한다.
\item 기타진단을 이용한 중증도 분류 과정은 3가지로 구분된다. 
	\begin{itemize}\tightlist
	\item 첫번째 단계는 기타진단의 중증도 점수를 결정하는 것이고 외과환자의 경우 0점 에서 4범까지 중증도 점수를 부여하고 있다.(참고 별표4 기타진단의 중증도 점수)
	\item 두번째 단계는 한 환자가 2개 이상의 기타진단을 가질 경우 환자 단위의 중증도 점수를 결정하는 것이다. 개별 기타진단의 중증도 점수를 통합하는 공식이 있어서 이 공식을 이용해서 환자단위 중증도 점수 (PCCL, Patient Clinical Complexity Level)를 결정하게 된다.
	\item 환자단위 중증도 점수를 이용하여 ADRG별로 중증도 분류단계(최종 질병군 분류번호)를 결정하게 된다. 
	\end{itemize}
\item 이때 결정된 중증도 분류는 ADRG 별로 중증도 분류의 단계를 달리하기 때문에 환자단위 중증도 점수가 ADRG의 중증도 분류와 일치하지는 않는다.
\end{itemize}
\prezi{\clearpage}
\leftrod{질병군 분류번호 결정 요령II}
\par
\medskip
\begin{enumerate}[가.]\tightlist
\item 질병군 분류번호는 주진단, 외과계 시술, 연령 및 기타진단 등에 의하여 6자리로 구성하며, 앞의 4자리는 “질병군범주”를, 5번째 자리는 “연령구분”을, 6번째 자리는 “합병증 및 동반상병 분류”를 나타낸다.
	\begin{itemize}\tightlist
	\item 질병군범주는 ‘주진단’과 ‘외과계 시술’ 등에 의하여 결정되며, 질병군 범주의 결정 및 그 분류번호는 별표3과 같다. 단, 주진단과 첫 번째 기타진단이 「한국표준질병 사인분류」의 다중코딩 지침에 따라 ‘검표(†)와 별표(*) 체계’ 에 해당할 경우 첫 번째 기타진단에 의하여 질병군 범주가 결정된다.
	\item 연령구분은 ‘연령’에 따라 다음 질병군 범주에 한하여 아래와 같이 결정되며, (가)~(다) 이외의 질병군 범주는 연령에 관계없이 분류번호 “0”으로 결정된다.
	\item 합병증 및 동반상병 분류(이하 “합병증분류”라 한다)는 기타진단에 의하여 다음과 같이 결정된다.
		\begin{itemize}\tightlist
		\item 합병증분류에 이용되는 기타진단은 각각의 중증도 점수(별표4 참조)를 가지고 있으나, 주진단 및 기타진단 상호간에 관련성이 높은 경우에는 중증도 점수가 1점 이상이더라도 0점으로 결정된다. (별표5 참조)
		\item 위(가)에 의한 기타진단별 중증도 점수를 반영하여 환자단위 중증도 점수를 결정하며, 동 점수를 이용하여 질병군별로 합병증 분류를 0, 1, 2, 3으로 결정한다. (별표6 참조)
		\end{itemize}
	\end{itemize}	
\item 위 가-⑴ 중 별표3의 각 주진단범주(안과계, 이비인후과계, 소화기계, 여성생식기계, 임신․분만․산욕)에 명시된 질병군범주에 해당되는 경우 로서 질병군범주 우선순위(별표7 참조)에서 당해 질병군범주 보다 높은 범주에 분류된 시술을 함께 행한 경우는 질병군적용에서 제외한다
\end{enumerate}
\prezi{\clearpage}
\begin{center}
\includegraphics[width=.95\textwidth]{DRGcode}
\end{center}
\prezi{\clearpage}
\leftrod{별표 7 임신분만산욕기계 질병군}
\par
\medskip
\tabulinesep =_2mm^2mm
\begin {longtabu} to\linewidth {|X[1,l]|X[3,l]|X[7,l]|} \tabucline[.5pt]{-}
\rowcolor{ForestGreen!40}  순위 & 질병군범주 &	해당 주진단 시술코드 및 부가코드 \centering 총진료비 \\ \tabucline[.5pt]{-}
\rowcolor{Yellow!40} 1 &	자궁적출술을 동반한 \newline 제왕절개분만 &	(R4507, R4508, R4509, R4510, R5001, R5002) \newline
or \newline
(R4517, R4518, R4514, R4519, R4520, R4516 and R4143, R4144,R4145, R4146)  \\ \tabucline[.5pt]{-}
\rowcolor{Yellow!40} 2 & 제왕절개분만(다태아) &	R4519, R4520, R4516 \\ \tabucline[.5pt]{-}
\rowcolor{Yellow!40} 3 & 제왕절개분만(단태아) &	R4517, R4518, R4514  \\ \tabucline[.5pt]{-}
\rowcolor{Yellow!40} 4 &	질식분만(기타 복잡 수술 시행) &	R4351, R4353, R4356, R4358, R3131, R3133, R3136, R3138, R3141,R3143, R3146, R3148, RA431, RA432, RA433, RA434, RA311,RA312, RA313, RA314, RA315, RA316, RA317, RA318, R4361,R4362, RA361, RA362, R4380, RA380 \newline
and \newline
E7691, E7690, O2045, P2141, Q2440, Q2450, Q3012, Q3013, Q3014,Q3017, R4130, R4143, R4144, R4145, R4146, R4154, R4155, R4157,R4170, R4181, R4183, R4202, R4203, R4221, R4223, R4224, R4250,R4261, R4262, R4295, R4331, R4332, R4390, R4400, R4405, R4411,R4412, R4413, R4421, R4423, R4424, R4427, R4428, R4425, R4426,M6650
\\ \tabucline[.5pt]{-}
\end{longtabu}
\par
\medskip