\section{비급여원내약품관: 진통제} \label{analgesia}
생각보다 원외처방 환수와 삭감이 잘되는것이 NSAID 계열 약물인거 같습니다.
아래 보시면 NSAIDs 계열 약물의 효능 효능에 생리통에 되어 있는 것이 거의 없어 보입니다.
따라서 NSAIDs 를 사용할때는 M5496[요통]계열을 넣으시는 것이 좋겠습니다.\par
거기에 현재 생리통에 디클로페낙이 식약처 허가범위라서 삭감됨. 따라서 \highlight{비급여 디클로페낙 주사제를 원내 고시후 사용하면 좋을것임}

\subsection{비스테로이드성 진통 소염제}
\begin{enumerate}\tightlist
\item ketoprofen : 경동 케토프로펜 주 50mg/ml, 동국 케토프로펜 주사(프리필드) 100mg/2ml, 토푸렌 주 100mg, 토푸렌 주 50mg
\item diclofenac sodium : 동구 디클로페낙나트륨 주 25mg/ml, 디페클로 주 90mg/2ml, 디페클로 주 90mg/3ml, 유니페낙 주 3ml, 이지판 주 90mg/2ml, 크낙주 90mg/3ml, 하이페낙 주 90mg/2ml, 하이페낙 주 90mg/3ml
\item piroxicam : 로시덴 주 20mg/ml(PFS), 캄옥시 주 20mg/ml
\item naproxen sodium : 바이넥스 나프록센나트륨 주 275mg/2.5ml
\item ibuprofen : 칼도롤 주사액 400mg
\end{enumerate}
\subsection{비마약성 진통 소염제}
\begin{enumerate}\tightlist
\item nefopam hydrochloride : 네큐팜 주사액 20mg/2ml, 네큐팜 주사액 40mg/4ml, 아나포 주, 아큐판 주사액, 아큐페인 주
\item tramadol hydrochloride : 뉴돌핀 주(수출용), 코리돌 주 100mg/2ml
\item dextrose외  : 살소부로칼 주
\item dried honey bee venom : 아피톡신 주 0.1mg, 아피톡신 주 1mg, 아피톡신 주 2mg
\end{enumerate}
\tabulinesep =_2mm^2mm
\begin {tabu} to\linewidth {|X[2,l]|X[8,l]|} \tabucline[.5pt]{-}
\rowcolor{ForestGreen!40} 약물명 & \centering 적응증  \\ \tabucline[.5pt]{-}
\rowcolor{Yellow!40}  Aspirin &  \\ \tabucline[.5pt]{-}
\rowcolor{Yellow!40}  Diclofenac & 류마티양 관절염, 골관절염(퇴행성 관절질환), 강직성 척추염, 수술후ㆍ외상후 염증 및 동통, 급성통풍, 신 및 간산통  \\ \tabucline[.5pt]{-}
\rowcolor{Yellow!40} Etodolac  & 류마티양 관절염, 골관절염(퇴행성 관절질환), 강직성 척추염, 수술후ㆍ외상후ㆍ발치후 동통 \\ \tabucline[.5pt]{-}
\rowcolor{Yellow!40} Fenoprofen  & 경증 및 중등도의 통증, 류마티양 관절염 및 골관절염(퇴행성 관절질환) \\ \tabucline[.5pt]{-}
\rowcolor{Yellow!40} Flurbiprofen  & 변형성 관절증, 견관절 주위염, 건/건초염, 건주위염, 상완골 상과염(테니스엘보 등), 근육통, 외상 후의 종창, 동통 \\ \tabucline[.5pt]{-}
\rowcolor{Yellow!40} Ibuprofen  & 염증에 의한 경증 및 중등도의 통증 완화 \\ \tabucline[.5pt]{-}
\rowcolor{Yellow!40} Indomethacin  & 류마티양 관절염, 골관절염(퇴행성 관절질환), 급성통풍성 관절염, 강직성 척추염, 수술후\bullet 외상후 동통 \\ \tabucline[.5pt]{-}
\rowcolor{Yellow!40} Ketoprofen  & 주효능 효과: 류마티양 관절염, 골관절염(퇴행성 관절질환). 
다음 질환에 사용 가능: 강직성 척추염, 급성통풍, 견관절주위염, 외상후\bullet 수술후 염증 및 동통, 건염, 활액낭염 \\ \tabucline[.5pt]{-}
\rowcolor{Yellow!40} Nabumetone  & 골관절염(퇴행관절염), 류마티스 관절염의 소염\bullet 진통 \\ \tabucline[.5pt]{-}
\rowcolor{Yellow!40} Naproxin  & 류마티양 관절염, 골관절염(퇴행성 관절질환), 강직성 척추염, 건염, 급성통풍, 월경곤란증. 활액낭염, 골격근장애(염좌, 좌상, 외상, 요천통), 수술후 동통, 편두통, 발치후 동통 \\ \tabucline[.5pt]{-}
\rowcolor{Yellow!40} Oxaprazin  & 약없슴 \\ \tabucline[.5pt]{-}
\rowcolor{Yellow!40} Phenylbutazone  & 약없슴ㄴ \\ \tabucline[.5pt]{-}
\rowcolor{Yellow!40} Piroxicam  &  타 NSAIDs에 불응성이거나 효과가 불충분한 다음의 경우에만 투여: 
1) 류마티스 관절염, 골관절염(퇴행관절염), 강직척추염. 
2) 만성에 한하여 요통, 견관절주위염, 경견완증후군. \\ \tabucline[.5pt]{-}
\rowcolor{Yellow!40} Sulindac  & 골관절염(퇴행성 관절질환), 류마티양 관절염, 강직성척추염, 급성통풍성 관절염, 관절주위염(건염, 점액낭염, 건초염) \\ \tabucline[.5pt]{-}
\rowcolor{Yellow!40} Tolmetin  & 1. 류마티스성 관절염, 골관절염 증상 및 징후의 완화; 급성 발적, 만성질환의 장기 치료. 
2. 소아 류마티스성 관절염 증상 및 징후의 완화 \\ \tabucline[.5pt]{-}
\end{tabu}
\par
\medskip
\tabulinesep =_2mm^2mm
\begin {tabu} to\linewidth {|X[2.5,l]|X[8,l]|} \tabucline[.5pt]{-}
\rowcolor{ForestGreen!40} 약물명 & \centering 적응증  \\ \tabucline[.5pt]{-}
\rowcolor{Yellow!40} Acetaminophen  & 감기로 인한 발열 및 동통, 두통, 신경통, 근육통, 월경통, 염좌통, 치통, 관절통, 류마티양 동통 \\ \tabucline[.5pt]{-}
\rowcolor{Yellow!40} AAP + Tramadol  &  중등도-중증의 급\bullet 만성 통증 \\ \tabucline[.5pt]{-}
\end{tabu}
