\section{다태임신의 초음파급여}
%산전진찰을 목적으로 아래와 같이 시행하는 경우에 인정하며, 다태아는 태아 수에 따라 각각의 소정점수를 산정함.\par
요양급여의 적용기준및 방법에 관한 세부사항\par
\menu{I. 행위 > 제2장 검사료 > 1. 급여 대상 및 범위}%\par
\begin{enumerate}[나.]\tightlist
\item 임산부 초음파
	\begin{enumerate}[1)]\tightlist
	\item 산전진찰을 목적으로 아래와 같이 시행하는 경우에 인정하며, \textcolor{red}{다태아의 경우 제2태아부터는 소정점수의 50\%를 산정함.(나951나(1) ‘주'항 제외)}
	\item 임신 과정중 의학적 판단 하에 태아에게 이상이 있거나 이상이 예상되어 상기 산정횟수를 초과하여 시행해야 하는 경우에는 \textcolor{red}{해당 삼분기의 일반 또는 일반의 제한적 초음파로 산정하며(‘주'항 제외), 입원중 동일목적으로 1일 수회 시행하는 경우에도 1일 1회만 산정함.}
	\end{enumerate}
\end{enumerate}	
%\item \cntrdots{}	

\menu{> 3. 상기 1.의 규정 이외에 아래와 같은 경우에도 요양급여를 인정함.}%\par
\begin{enumerate}[다.]\tightlist
\item 나943다 태아정밀 심초음파는 산전진찰 결과 태아의 심장에 이상소견이 있어 정밀검사를 시행하는 경우 산정하며, 이 경우 \textcolor{red}{다태아는 1.1.1)의 적용을 받음.}
\end{enumerate}

%산전진찰을 목적으로 아래와 같이 시행하는 경우에 인정하며, \uline{다태아의 경우 제2태아부터는 소정점수의 %50\%를 산정함.}(나951나(1) ‘주’항제외) 
“나951나(1) ‘주’항” 해당 임산부이란? 가) 태아에게 문제를 초래하는 임부의 질환상태(임신성 당뇨병, 임신성 고혈압 등), 나) 태아에게 문제를 초래하는 임부 자궁의 이상(여성생식기종양, 자궁경관무력증, 자궁기형 등), 다) 정상 분만이 불가능한 태반의 이상(전치태반, 태반조기박리 등), 라) 양수과다증 또는 양수과소증, 마) 자궁내 태아 성장지연 즉 high risk인 경우는 \uline{1태아에서는 인정되지만, 2태아부터는 인정되지 않음}으로 인해서 480.36점을 추가로 산정 
\prezi{\clearpage}
\begin{commentbox}{다태아 초음파 산정 시 기재 가 필요한 사항은 ?}
태아 수에 따른 정확한 상병 ( 완전코드 ) 을 기재해야 함 .
\begin{itemize}[-]\tightlist
\item 2 태아 : 쌍둥이임신 (O300)
\item 3 태아 : 세쌍둥이임신 (O301)
\item 4 태아 : 네쌍둥이임신 (O302)
\item 5 태아 이상 : 기타 다태임신 (O308)
\item 다만 , 상세불명의 다태임신 (O309) 코드를 사용할 경우 사유 기재
\end{itemize}
\end{commentbox}
\prezi{\clearpage}
\subsection{산부인과학회 세부급여기준 Q\&A}
\begin{enumerate}\tightlist
\item고위험 가산에 해당하지 않는 쌍둥이 임산부가 임신 30주에 일반 초음파 검사를 한 경우 \par
제 1태아: 제2, 3삼분기 일반 (EB515) + 제 2태아: 제 2,3 삼분기 일반 (EB515)의 50\%
\item 임신성 당뇨가 있는 쌍둥이 임산부가 임신 30주에 일반 초음파 검사를 한 경우 \par
제 1태아: 제2, 3삼분기 일반 고위험 (EB516) + 제 2태아: 제 2,3 삼분기 일반 (EB515)의 50\%
\item 쌍태아 이상에서 하나의 태아가 유산 또는 사산된 경우 유산 또는 사산된 태아수도 포함하여 가산을 하나요?\par
유산 또는 사산된 태아에 대한 evaluation을 하는 경우에는 태아수에 포함하여 가산하고, 생존 태아만 초음파 검사를 한 경우에는 생존 태아 수에 따라 가산하면 됩니다.
\end{enumerate}
\prezi{\clearpage}
\subsection{다태아 초음파에서 물음}
\Que{Discordant Twin시의 초음파 F/U은 어떤 식으로 하실건지요? }
\Ans{일단 anomaly로 EB514 or EB518 산정 가능.\par
고위험이 아니므로 100\% EB518010과 50\% EB518010가능.\par

\highlight{지속적인 추적조사 초음파}\par
횟수초과 초음파의 경우는 일반 초음파 100\% EB515010와 50\% EB515010 가능}

