\section{STILL BIRTH}
\myde{}{\begin{itemize}\tightlist
\item[\dsjuridical] O021 : < 22주(Missed abortion), 
\item[\dsjuridical] O364 : ≥ 22 (Maternal Care For Intrauterine Death)
\item[\dsjuridical] P95 : 상세불명 원인의 태아사망
%\item[\dsmedical] [R4452] 인공임신중절수술,임신16주이상 - 20주 미만
%\item[\dsmedical] [R4458] 인공임신중절수술,임신16주이상 - 20주 미만
%\item[\dsmedical] [R4458] 인공임신중절수술,임신16주이상 - 20주 미만
\item[\dsmedical] 자445라 R4458 인공임신중절수술,임신16주이상 - 20주 미만 [\myexplfn{1843.44} 원]
\item[\dsmedical] 자445마 R4459 인공임신중절수술,임신20주이상 - 24주 미만 [\myexplfn{2589.98} 원]
\item[\dsmedical] 자-435 분만의 소정금액. 임신 24주 이상 
\item[\dsmedical] R4460 태아축소술 Embryotomy [\myexplfn{1826.24} 원]
\end{itemize}}
{\begin{enumerate}\tightlist
\item 임신 24주이내 태아를 사산한 경우에는 자445 인공임신중절수술에 준용하며, 임신 24주를 초과하여 태아를 사산한 경우에는 자435 분만의 소정금액과 자437 분만전처치, 자437-1 분만후 처치의 소정점수를 산정하고 태아축소술을 실시한 경우에는 자446 태아축소술의 소정점수만을 산정함.
\item 모자보건법시행령 제15조(인공임신수술의 허용한계) 28주에서 24주로 개정(2009.7.7)
\item 사산보고서 : \url{http://www.narastat.kr}에서 작성
\end{enumerate}
}
\prezi{\clearpage}

\begin{commentbox}{사산 제왕절개분만시 DRG여부}
28주에 태아를 사산한 경우 자연분만을 시도하다가 사유가 있어(출혈, 전치태반등) 자궁절개술(Hysterotomy)한 경우 자 451 제왕절개 만출술을 준용한다면 수술일부터는 DRG(포괄수가제)청구가 가능한가요?
\par
\medskip
자궁내 태아사망으로 자연분만을 시도하다가 제왕절개수술을 시행한 경우에는 임신 주수와 관계없이 ‘자-451 제왕절개만출술’의 수가를 산정하며 제왕절개분만 질병군에 해당합니다.
이 경우 상병기호는 임신기간이 22주 미만일 때 태아 사망이 발생했다면 ‘O021 계류유산’, 임신기간이 22주 이상일 때 태아 사망이 발생하여 사산아를 분만한 경우에는 ‘O364 자궁내 태아사망의 산모관리’ 상병기호를 부여하여야 합니다.\par

또한 질병군 포괄수가 요양급여비용은 입원부터 퇴원일까지 하나의 명세서로 청구하여야 하며, 질병군 진료 이외의 목적으로 입원하여 입원일수가 6일을 초과한 시점에 예상치 못하게 질병군 수술이 이루어진 경우 입원일로부터 수술시행일 전일까지의 진료분을 분리청구토록 하고 있으나, 질의하신 내용은 \textcolor{red}{분만을 목적으로 입원한 경우이므로 입원일부터 퇴원일까지의 요양급여비용은 질병군 포괄수가제를 적용}하여야 합니다. 따라서 \textcolor{blue}{문의하신 사례는 주진단 ‘O364 자궁내 태아사망의 산모관리’, 수술료 ‘자-451 제왕절개만출술’에 해당하여 입원일부터 퇴원일까지의 진료가 제왕절개분만 질병군 대상임}을 알려드립니다. 아울러 상세한 내역은 건강보험심사평가원 홈페이지(www.hira.or.kr)>요양기관업무포탈>심사정보>자료방>자료실>질병군별 포괄수가(DRG)의 질의응답을 참고하시기 바랍니다. 
\end{commentbox}
\prezi{\clearpage}
\subsection{인공임신중절수술 (쌍태아) 수가 산정방법}
16주 이상인 쌍태아의 인공임신중절수술은 자445라 또는 \textcolor{red}{인공임신중절수술 소정점수의 50\%를 가산}하여 산정함.
\prezi{\clearpage}
\subsection{자궁내 태아사망으로 질식분만을 시도하다가 제왕절개수술을 한 경우}
\begin{itemize}\tightlist
\item 주수와 관계없이 제왕절개 질병군 청구
\item < 22주 : O021(Missed abortion), ≥ 22 : O364 (Maternal Care For Intrauterine Death)
\item 입원일수와 관계없이 질병군 포괄수과제 적용
\end{itemize}
\prezi{\clearpage}
\subsection{24주 이전에 불가피한 분만}
\begin{itemize}\tightlist
\item IIOC or PROM시엔 사산이 아닌 정상적인 태아를 분만한 경우는 분만관련 수가를 적용
\item 제1편 제2부 제9장 `자435 분만'을 산정한다.
\end{itemize}
