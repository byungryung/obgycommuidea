\section{직원 월급}
\subsection{임금관리}
\emph{임금구분}
\begin{description}\tightlist
\item[통상임금] 정기적, 일률적, 고정적으로 소정근로에 대해 지급하는 
   시급, 일급, 주급, 월급, 도급액
\item[평균임금] 산정사유 발생일 이전 3개월간 지급된 임금총액
     산정사유 발생일 이전 3개월간 총일수
\end{description}
\emph{임금 유형}
\begin{description}\tightlist
\item[호봉제] 각 직급별 또는 직무별 호봉표에 근거
   (단, 연장․야간․휴일근로수당, 산전․후휴가급여, 육아휴직급여,
    연차수당 등은 근기법을 적용)  

\item[연봉제] 당해연도 예산에 따라(성과평가에 따라) 직종별 연봉산정 기준액에 따라
   회사가 정함.
   (연봉액에 연장․야간․휴일근로수당, 연차수당 및 퇴직금 미포함)
   ※ 회계연도 중 채용된 직원 : 일할기준액×연봉기준일수

\item[일급제] \{일급단가×(근무일+유급휴일)\} + 각종 수당
\item[시급제] \{시급단가×근무시간×(근무일+유급휴일)\} + 각종 수당
\end{description}

\subsection{최저임금}
\leftrod{최저임금의 적용}\par
\begin{itemize}\tightlist
\item 사용자는 최저임금의 적용을 받는 근로자에게 최저임금액 이상의 임금을 지급해야 합니다. (최저임금법 제6조제1항)
\item 최저임금액보다 적은 임금을 지급한 자에게는 3년 이하의 징역 또는 2천만원 이하의 벌금에 처합니다. 이 경우 징역과 벌금은 병과 할 수 있습니다. (최저임금법 제28조)
\item 근로자가 받는 임금이 최저 임금액에 미치는지를 판단하기 위해서는 최저임금의 적용을 위한 임금의 범위에 산입하는 임금의 총액을 최저임금액과 같은 기준으로 환산하여 비교합니다. (최저임금제도 업무처리지침)
\end{itemize}
\url{http://maitland.tistory.com/926}
\par
\medskip

\includegraphics{belowfee}
\par
\medskip
\subsection{일자리안정자금}
\leftrod{지원요건}
\par
\medskip
\tabulinesep =_2mm^2mm
\begin{tabu} to\linewidth {|X[1,l]|X[6,l]|} \tabucline[.5pt]{-}
\rowcolor{ForestGreen!40} 구분 & 주요 내용 \\ \tabucline[.5pt]{-}
\rowcolor{Yellow!40} 지원 대상 & 30인 미만 고용 사업주 \newline 공동주택 경비\cntrdot{}청소는 30인 이상 기업도 지원 \\ \tabucline[.5pt]{-}
\rowcolor{Yellow!40} 지원 요건 & 월 보수액 190만원 미만 노동자 \newline 지원금 신청 이전 1개월 이상 고용 유지 \newline 최저임금 준수 \newline 고용보험 가입 의무자는 고용보험 가입 \newline 기존 노동자는 최소한 전년도 보수수준 유지(인위적 임금삭감 방지) \\ \tabucline[.5pt]{-}
\rowcolor{Yellow!40} 지원 제외 & 고소득 사업주(과세소득 5억원 초과) \newline 임금체불 명단 공개중인 사업주 \newline 국가등 공공기관 \newline 인건비 재정지원을 받고 있는 사업주 \newline 30인 미만 요건을 맞추기 위한 인위적 고용조정 \newline 특수관계인(사업주 본인, 배우자, 직계존비속) \\ \tabucline[.5pt]{-}
\rowcolor{Yellow!40} 지원 금액 & 노동자 1인당 매월 13만원 \newline 주 40시간 미만 노동자는 시간비례로 3-12만원 \\ \tabucline[.5pt]{-}
\end{tabu}



