\subsection{행정조사기본법}
제1조(목적)
이 법은 행정조사에 관한 기본원칙·행정조사의 방법 및 절차 등에 관한 공통적인 사항을 규정함으로써 행정의 공정성·투명성 및 효율성을 높이고, \highlightR{국민의 권익을 보호함을 목적}으로 한다.
[[시행일 2007.8.18]]

제4조(행정조사의 기본원칙)
\begin{enumerate}[①]\tightlist
\item 행정조사는 \highlightY{조사목적을 달성하는데 필요한 최소한의 범위 안에서 실시하여야 하며}, 다른 목적 등을 위하여 \highlightY{조사권을 남용하여서는 아니 된다.}
\item 행정기관은 조사목적에 적합하도록 조사대상자를 선정하여 행정조사를 실시하여야 한다.
\item 행정기관은 \highlightY{유사하거나 동일한 사안에 대하여는 공동조사 등을 실시함으로써} \highlightR{행정조사가 중복되지 아니하도록} 하여야 한다.
\item 행정조사는 \highlightY{법령등의 위반에 대한 처벌보다는} \highlightR{법령등을 준수하도록 유도하는 데 중점}을 두어야 한다.
\item 다른 법률에 따르지 아니하고는 행정조사의 대상자 또는 행정조사의 내용을 공표하거나 \highlightY{직무상 알게 된 비밀을 누설하여서는 아니된다.}
\item 행정기관은 행정조사를 통하여 알게 된 정보를 다른 법률에 따라 내부에서 이용하거나 다른 기관에 제공하는 경우를 제외하고는 원래의 조사목적 이외의 용도로 이용하거나 타인에게 제공하여서는 아니 된다.
\end{enumerate}

제11조(현장조사)
\begin{enumerate}[①]\tightlist
\item 조사원이 가택·사무실 또는 사업장 등에 출입하여 현장조사를 실시하는 경우에는 행정기관의 장은 다음 각 호의 사항이 기재된 현장출입조사서 또는 법령등에서 현장조사시 제시하도록 규정하고 있는 문서를 조사대상자에게 발송하여야 한다.
	\begin{enumerate}[1.]\tightlist
	\item \highlight{조사목적}
	\item \highlight{조사기간과 장소}
	\item \highlight{조사원의 성명과 직위}
	\item \highlight{조사범위와 내용}
	\item \highlight{제출자료}
	\item \highlight{조사거부에 대한 제재(근거 법령 및 조항 포함)}
	\item \highlight{그 밖에 당해 행정조사와 관련하여 필요한 사항}
	\end{enumerate}
\item 제1항에 따른 \highlightR{현장조사는 해가 뜨기 전이나 해가 진 뒤에는 할 수 없다.} 다만, 다음 각 호의 어느 하나에 해당하는 경우에는 그러하지 아니하다.
  1. 조사대상자(대리인 및 관리책임이 있는 자를 포함한다)가 동의한 경우
  2. 사무실 또는 사업장 등의 업무시간에 행정조사를 실시하는 경우
  3. 해가 뜬 후부터 해가 지기 전까지 행정조사를 실시하는 경우에는 조사목적의 달성이 불가능하거나 증거인멸로 인하여 조사대상자의 법령등의 위반 여부를 확인할 수 없는 경우
\item 제1항 및 제2항에 따라 현장조사를 하는 조사원은 그 권한을 나타내는 증표를 지니고 이를 조사대상자에게 내보여야 한다.
\end{enumerate}

제14조(공동조사)
\begin{enumerate}[①]\tightlist
\item 행정기관의 장은 \highlightY{다음 각 호의 어느 하나에 해당하는 행정조사를 하는 경우에는 공동조사를 하여야 한다.}
	\begin{enumerate}[1.]\tightlist
	\item 당해 행정기관 내의 2 이상의 부서가 \highlightR{동일하거나 유사한 업무분야에 대하여 동일한 조사대상자에게 행정조사를 실시하는 경우}
	\item 서로 다른 행정기관이 대통령령으로 정하는 분야에 대하여 동일한 조사대상자에게 행정조사를 실시하는 경우
	\end{enumerate}
\item 제1항 각 호에 따른 사항에 대하여 \highlightY{행정조사의 사전통지를 받은 조사대상자는 관계 행정기관의 장에게 공동조사를 실시하여 줄 것을 신청할 수 있다.} 이 경우 조사대상자는 신청인의 성명·조사일시·신청이유 등이 기재된 공동조사신청서를 관계 행정기관의 장에게 제출하여야 한다.
\item 제2항에 따라 \highlightY{공동조사를 요청받은 행정기관의 장은 이에 응하여야 한다.}
\item 국무조정실장은 행정기관의 장이 제6조에 따라 제출한 행정조사운영계획의 내용을 검토한 후 관계 부처의 장에게 공동조사의 실시를 요청할 수 있다. [개정 2008.2.29 제8852호(정부조직법), 2013.3.23 제11690호(정부조직법)]
\item 그 밖에 공동조사에 관하여 필요한 사항은 대통령령으로 정한다.
\end{enumerate}

제15조(중복조사의 제한)
\begin{enumerate}[①]\tightlist
\item 제7조에 따라 정기조사 또는 수시조사를 실시한 행정기관의 장은 \highlightR{동일한 사안에 대하여 동일한 조사대상자를 재조사 하여서는 아니 된다.} 다만, 당해 행정기관이 이미 조사를 받은 조사대상자에 대하여 위법행위가 의심되는 새로운 증거를 확보한 경우에는 그러하지 아니하다.
\item 행정조사를 실시할 행정기관의 장은 행정조사를 실시하기 전에 \highlightR{다른 행정기관에서 동일한 조사대상자에게 동일하거나 유사한 사안에 대하여 행정조사를 실시하였는지 여부를 확인할 수 있다.}
\item 행정조사를 실시할 행정기관의 장이 제2항에 따른 사실을 확인하기 위하여 행정조사의 결과에 대한 자료를 요청하는 경우 요청받은 행정기관의 장은 특별한 사유가 없는 한 관련 자료를 제공하여야 한다.
\end{enumerate}

제17조(조사의 사전통지)
\begin{enumerate}[①]\tightlist
\item 행정조사를 실시하고자 하는 행정기관의 장은 제9조에 따른 출석요구서, 제10조에 따른 보고요구서·자료제출요구서 및 제11조에 따른 현장출입조사서(이하 "출석요구서등"이라 한다)를 \highlightR{조사개시 7일 전까지 조사대상자에게 서면으로 통지하여야 한다.} 다만, 다음 각 호의 어느 하나에 해당하는 경우에는 행정조사의 개시와 동시에 출석요구서등을 조사대상자에게 제시하거나 행정조사의 목적 등을 조사대상자에게 구두로 통지할 수 있다.
	\begin{enumerate}[1.]\tightlist
	\item 행정조사를 실시하기 전에 관련 사항을 미리 통지하는 때에는 증거인멸 등으로 행정조사의 목적을 달성할 수 없다고 판단되는 경우
	\item 「통계법」  제3조제2호에 따른 지정통계의 작성을 위하여 조사하는 경우
	\item 제5조 단서에 따라 조사대상자의 자발적인 협조를 얻어 실시하는 행정조사의 경우
	\end{enumerate}
\item 행정기관의 장이 출석요구서등을 조사대상자에게 발송하는 경우 출석요구서등의 내용이 외부에 공개되지 아니하도록 필요한 조치를 하여야 한다.
\end{enumerate}

제20조(자발적인 협조에 따라 실시하는 행정조사)
\begin{enumerate}[①]\tightlist
\item 행정기관의 장이 제5조 단서에 따라 조사대상자의 \highlight{자발적인 협조를 얻어 행정조사를 실시하고자 하는 경우 조사대상자는 문서·전화·구두 등의 방법으로 당해 행정조사를 거부할 수 있다.}
\item 제1항에 따른 행정조사에 대하여 조사대상자가 조사에 응할 것인지에 대한 \highlight{응답을 하지 아니하는 경우에는 법령등에 특별한 규정이 없는 한 그 조사를 거부한 것으로 본다.}
\item 행정기관의 장은 제1항 및 제2항에 따른 조사거부자의 인적 사항 등에 관한 기초자료는 특정 개인을 식별할 수 없는 형태로 통계를 작성하는 경우에 한하여 이를 이용할 수 있다.
\end{enumerate}

제22조(조사원 교체신청)
\begin{enumerate}[①]\tightlist
\item 조사대상자는 조사원에게 공정한 행정조사를 기대하기 어려운 사정이 있다고 판단되는 경우에는 행정기관의 장에게 당해 \highlightR{조사원의 교체를 신청할 수 있다.}
\item 제1항에 따른 교체신청은 그 이유를 명시한 서면으로 행정기관의 장에게 하여야 한다.
\item 제1항에 따른 \highlightR{교체신청을 받은 행정기관의 장은 즉시 이를 심사하여야 한다.}
\item 행정기관의 장은 제1항에 따른 \highlightR{교체신청이 타당하다고 인정되는 경우에는 다른 조사원으로 하여금 행정조사를 하게 하여야 한다.}
\item 행정기관의 장은 제1항에 따른 교체신청이 조사를 지연할 목적으로 한 것이거나 \highlightR{그 밖에 교체신청에 타당한 이유가 없다고 인정되는 때에는 그 신청을 기각하고 그 취지를 신청인에게 통지하여야 한다.}
\end{enumerate}

제23조(조사권 행사의 제한)
\begin{enumerate}[①]\tightlist
\item 조사원은 제9조부터 제11조까지에 따라 사전에 발송된 사항에 한하여 조사대상자를 조사하되, 사전통지한 사항과 관련된 추가적인 행정조사가 필요할 경우에는 조사대상자에게 추가조사의 필요성과 조사내용 등에 관한 사항을 서면이나 구두로 통보한 후 추가조사를 실시할 수 있다.
\item 조사대상자는 \highlightR{법률·회계 등에 대하여 전문지식이 있는 관계 전문가로 하여금 행정조사를 받는 과정에 입회하게 하거나} 의견을 진술하게 할 수 있다.
\item 조사대상자와 조사원은 \highlightR{조사과정을 방해하지 아니하는 범위 안에서 행정조사의 과정을 녹음하거나 녹화할 수 있다.} 이 경우 녹음·녹화의 범위 등은 상호 협의하여 정하여야 한다.
\item 조사대상자와 조사원이 제3항에 따라 녹음이나 녹화를 하는 경우에는 사전에 이를 당해 행정기관의 장에게 통지하여야 한다.
\end{enumerate}

제29조(행정조사의 점검과 평가)
\begin{enumerate}[①]\tightlist
\item 국무조정실장은 행정조사의 효율성·투명성 및 예측가능성을 제고하기 위하여 각급 행정기관의 \highlight{행정조사 실태, 공동조사 실시현황 및 중복조사 실시 여부 등을 확인·점검하여야 한다.} [개정 2008.2.29 제8852호(정부조직법), 2013.3.23 제11690호(정부조직법)]
\item 국무조정실장은 제1항에 따른 확인·점검결과를 평가하여 대통령령으로 정하는 절차와 방법에 따라 국무회의와 \highlightR{대통령에게 보고하여야 한다.} [개정 2008.2.29 제8852호(정부조직법), 2013.3.23 제11690호(정부조직법)]
\item 국무조정실장은 제1항에 따른 확인·점검을 위하여 각급 행정기관의 장에게 \highlightR{행정조사의 결과 및 공동조사의 현황 등에 관한 자료의 제출을 요구할 수 있다.} [개정 2008.2.29 제8852호(정부조직법), 2013.3.23 제11690호(정부조직법)]
\item 행정조사의 확인·점검 대상 행정기관과 행정조사의 확인·점검 및 평가절차에 관한 사항은 대통령령으로 정한다.
\end{enumerate}


