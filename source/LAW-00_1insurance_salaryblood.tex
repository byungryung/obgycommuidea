\begin{mdframed}[linecolor=blue,middlelinewidth=2]  
제1편 행위 급여 \cntrdot{}  비급여 목록 및 급여 상대가치점수 >> 제2부 행위 급여 목록\cntrdot{} 상대가치점수 및 산정지침 >> 제16장 전혈및 혈액성분 제제료 
\end{mdframed}
\subsection{제16장 \newindex{전혈및 혈액성분 제제료}}
\emph{〔산정지침〕}
\begin{enumerate}[(1)]\tightlist
\item “혈액관리법 제11조”의 규정에 의하여 장관이 별도 고시한 항목과 금액으로 산정한다.
\item \uline{수혈에 소요되는 약제 및 재료대(1회용 주사기, 1회용 주사침, 나비침, 정맥내유치침, 수액세트, 혈액 Bag 등)는 소정금액에 포함되므로 별도 산정하지 아니한다. 다만, 정맥내 유치침을 사용한 경우에는 「마-5-주1」에 따라 산정하며, 다음의 경우에는 “약제 및 치료재료의 비용에 대한 결정 기준”에 의하여 별도 산정한다.}
	\begin{enumerate}[(가)]\tightlist
	\item 백혈구여과제거적혈구 및 백혈구여과제거혈소판의 경우에 사용된 약제 및 재료대
	\item 혈액성분채집술(복합성분채집 혈장은 제외)에 사용된 약제 및 재료대 (요양기관이 대한적십자사혈액원 등으로부터 성분채집에 의한 혈액 성분제제를 구입한 경우 포함)
	\end{enumerate}
\item 혈액성분채집술에 의한 혈액성분채혈시 공혈자에 대한 공혈적합성 여부를
판정하기 위한 검사비용은 소정금액에 포함되므로 별도 산정하지 아니한다.
\end{enumerate}