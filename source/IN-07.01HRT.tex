\section{폐경 후 호르몬요법 }
\myde{}{%
\begin{itemize}\tightlist
\item[\dsjuridical] N951 폐경 및 여성의 갱년기상태
\end{itemize}
}
{
\begin{commentbox}{폐경기증후군에 투여하는 약제}  
갱년기장애(폐경기증후군) 증상의 발현 시 동 증상완화를 위하여 호르몬제나 칼슘제제 등의 약제를 투여하는 경우에 요양급여를 인정함.\par
* 시행일: 2013.9.1.\par
* 종전고시: 고시 제2001-28호(2001.6.8.)
\end{commentbox}
\emph{[일반원칙] 폐경 후 호르몬요법 인정기준}
폐경기증후군 및 골다공증에 사용하는 호르몬요법은 허가범위 내에서 아래와 같은 기준으로 투여 시 요양급여를 인정함.
\par
\begin{center}\emph{- 아     래 -}\end{center}
\begin{enumerate}[가.]\tightlist
\item 적응증 
	\begin{enumerate}[1)]\tightlist
	\item \textcolor{red}{폐경기증후군의 증상완화}와 \textcolor{blue}{골밀도검사에서 같은 성, 젊은 연령의 정상치보다 1표준편차 이상 감소된 경우에 골다공증의 예방 및 치료목적으로 투여 시 요양급여를 인정함.} 
	\item \textcolor{red}{심혈관계 질환의 예방 및 치료에는 인정하지 아니함.}
	\item 재평가 기간매 12개월마다 재평가를 실시하여야 함(환자의 전반적인 상태 및 필요성) 
	\item \textcolor{blue}{적정투여기간60세까지 투여하며, 60세를 초과하여 호르몬 요법을 지속하는 경우에는 동 치료의 효과를 평가하여 지속투여 여부를 결정하여야 함.}
	\end{enumerate}
\end{enumerate}
}
\subsection{폐경기장애를 동반한 골다공증치료에 호르몬제와 골다공증치료제 병용투여시 인정기준}
골다공증치료제에는 호르몬요법(Estrogen, Estrogen derivates 등)과 비호르몬요법(Bisphosphonate, Calcitonin, Vit D, Ipriflavon 등)이 있으며, \textcolor{red}{칼슘제제와의 병용투여를 제외하고는 병용투여에 대한 효과가 입증되지 아니하였으므로 호르몬대체요법(HRT)과 비호르몬요법제를 병용투여하거나 비호르몬요법간 병용투여는 인정하지 아니한다.}
%\clearpage
\subsection{폐경기 약제의 효능/효과(식약처 허가사항)}
\medskip
\tabulinesep =_2mm^2mm
\begin {tabu} to\linewidth {|X[1,l]|X[3,l]|} \tabucline[.5pt]{-}
\rowcolor{ForestGreen!40} \centering 제품명/성분명 & \centering 식약처 허가사항 \\ \tabucline[.5pt]{-}
\rowcolor{Yellow!40} 리브론 정 2.5mg &	자연적인 또는 수술에 의한 폐경 이후의 증상(홍조, 야간 발한), 골절되기 쉬운 폐경 이후 골다공증. \\ \tabucline[.5pt]{-}
\rowcolor{Yellow!40} 안젤릭 정 & 1. 폐경 후 일년이 지난 여성의 에스트로겐 결핍증에 대한 호르몬 대체 요법. 2. 골다공증 예방으로 허가 받은 다른 약제에 불내성이거나 금기이고 골절 가능의 위험성이 증가된 폐경 후 여성의 골다공증 예방.\\ \tabucline[.5pt]{-}
\rowcolor{Yellow!40} 크리멘 28 정 &	폐경 후 ( 마지막 생리 후 최소 1 년이 경과된 시점 ) 여성의 에스트로겐 결핍의 증상경감을 위한 호르몬 대체요법 (HRT) \\ \tabucline[.5pt]{-}
\rowcolor{Yellow!40} 프레미나 정 0.3mg 프레미나 정 0.625mg & 성선기능저하증·난소적출·난소기능부전으로 인한 저에스트로겐증, 위축성 질염·외음위축증, 갱년기장애, 기능성 자궁출혈, 골다공증, 폐경 후의 유방암(경감용), 수술불능의 진행성 전립선암(경감용). \\ \tabucline[.5pt]{-}
\rowcolor{Yellow!40} 디비나 정 &	폐경 후 ( 마지막 생리 후 최소 1 년이 경과된 시점 ) 여성의 에스트로겐 결핍의 증상경감을 위한 호르몬 대체요법 (HRT) \\ \tabucline[.5pt]{-}
\rowcolor{Yellow!40} 리비알 정 &	폐경 후 1년이 경과한 여성의 에스트로겐 결핍 증상, 골절위험성이 높은 폐경 이후 여성의 골다공증 예방. \\ \tabucline[.5pt]{-}
\rowcolor{Yellow!40} 에스디올 하프 정 &	폐경후 1년 이상된 여성의 에스트로겐 결핍증상에 대한 호르몬 대체요법, 폐경후 여성의 골다공증 예방. \\ \tabucline[.5pt]{-}
\rowcolor{Yellow!40} 크리안 정 &	자궁절제가 되지 않은 폐경 후(마지막 생리 후 최소 1년이 경과된 시점) 여성의 에스트로겐 결핍의 증상경감을 위한 호르몬대체요법. \\ \tabucline[.5pt]{-}
\rowcolor{Yellow!40} 클리오벨 정 &	폐경 후 1년 이상 된 여성의 에스트로겐 결핍증상에 대한 호르몬 대체요법(다른 골다공증 예방약이 금기이거나 효과가 없는 경우). 골절을 일으키기 쉬운 폐경 후 여성의 골다공증 예방. \\ \tabucline[.5pt]{-}
\rowcolor{Yellow!40} 페모스톤 정 1/10 &	손상되지 않은 자궁을 가진 여성의 자연적 혹은 의인성 폐경 후(마지막 생리 후 최소 1년이 경과된 시점) 에스트로겐 결핍의 증상경감을 위한 호르몬대체요법, 골절의 위험이 있는, 자궁을 가진 폐경 후 여성의 골다공증 예방. \\ \tabucline[.5pt]{-}
\rowcolor{Yellow!40} 페모스톤 정 2/10 &	손상되지 않은 자궁을 가진 여성의 자연적 혹은 의인성 폐경 후(마지막 생리 후 최소 1년이 경과된 시점) 에스트로겐 결핍의 증상경감을 위한 호르몬대체요법, 골절의 위험이 있는, 자궁을 가진 폐경 후 여성의 골다공증 예방. \\ \tabucline[.5pt]{-}
\rowcolor{Yellow!40} 페모스톤 콘티 정 &	손상되지 않은 자궁을 가진 여성의 자연적 혹은 의인성 폐경 후(마지막 생리 후 최소 1년이 경과된 시점) 에스트로겐 결핍의 증상경감을 위한 호르몬대체요법, 골절의 위험이 있는, 자궁을 가진 폐경 후 여성의 골다공증 예방. \\ \tabucline[.5pt]{-}
\rowcolor{Yellow!40} 프레다 정 1mg &	에스트로겐 결핍 증상(갱년기장애). 폐경 후 골다공증. \\ \tabucline[.5pt]{-}
\rowcolor{Yellow!40} 프로기노바 1mg 정	& 갱년기 증상경감을 위한 호르몬대체요법 \\ \tabucline[.5pt]{-}
\rowcolor{Yellow!40} 프로기노바 2mg 정	& 갱년기 증상 치료 위한 호르몬 대체 요법, 골절을 일으키기 쉬운 여성의 골다공증 예방. \\ \tabucline[.5pt]{-}
\end{tabu}

