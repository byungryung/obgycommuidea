\section{골다공증치료제 인정기준}
\Que{DXA 장비를 이용하여 central bone의 골밀도를 측정한 결과 L1-L4의 T-score가 -2.0 - -2.4로, 대퇴부위에서는 -2.2 ~ -2.4, 대퇴의 Ward’s triangle 부위는 -2.5로 결과가 나왔습니다. 이 경우에 골다공증치료제의 보험급여가 가능한가요?}
\Ans{골밀도검사 측정결과를 해석 시에는 요추 L1~L4 중 2부위 이상의 평균 골밀도 또는 Ward’s triangle 부위를 제외한 대퇴부 측정값 중 낮은 부위를 기준으로 적용합니다. 이 때 L1~L4 중 가장 낮은 값의 적용은 인정하지 아니합니다. 따라서 동 경우는 검사결과 상 골다공증치료제의 인정기준에 해당되지 않으므로 약값의 전액을 환자가 부담토록 하여야 합니다.}

\begin{commentbox}{(일반원칙)골다공증치료제}
\begin{enumerate}[1.]\tightlist
\item \textcolor{red}{허가사항 범위 내}에서 아래와 같은 기준으로 투여 시 요양급여를 인정하며, 동 인정기준 \textcolor{red}{이외에는 약값 전액을 환자가 부담토록 함}\newline
\begin{center}\emph{- 아 래 -}\end{center}
	\begin{enumerate}[가.]\tightlist
	\item 칼슘 및 Estrogen제제 등의 약제 골밀도검사에서 T-score가 -1 이하인 경우(T-score ≤ -1.0)
	\item Elcatonin제제, Raloxifene제제, Bazedoxifene제제, 활성형 Vit D3제제 및 Bisphosphonate 제제 등의 약제(결과지 등 첨부)
		\begin{enumerate}[1)]\tightlist
		\item 투여대상
			\begin{enumerate}[가)]\tightlist
			\item 중심골[Central bone; 요추, 대퇴(Ward's triangle 제외)〕: 이중 에너지 방사선 흡수계측 (\textcolor{red}{Dual-Energy X-ray Absorptiometry: DEXA)을 이용하여 골밀도 측정시 T-score가 -2.5 이하인 경우(T-score ≤ -2.5)}
			\item \textcolor{red}{정량적 전산화 단층 골밀도 검사(QCT) : 80㎎/㎤ 이하인 경우}
			\item 상기 \textcolor{red}{가), 나)항 이외: 골밀도 측정시 T-score가 -3.0 이하인 경우(T-score ≤ -3.0)}
			\item 방사선 촬영 등에서 골다공증성 골절이 확인된 경우
			\end{enumerate}
		\item 투여기간
			\begin{enumerate}[가)]\tightlist
			\item \textcolor{red}{투여대상 다)에 해당하는 경우에는 6개월 이내}
			\item \textcolor{red}{투여대상 가), 나)에 해당하는 경우에는 1년 이내, 라)에 해당하는 경우에는 3년 이내로 하며}, \textcolor{red}{추적검사에서 T-score가 -2.5 이하(QCT 80㎎/㎤ 이하)로 약제투여가 계속 필요한 경우는 급여}토록 함
			\end{enumerate}
		\end{enumerate}
	\item 단순 X-ray는 골다공증성 골절 확인 진단법으로만 사용할 수 있음
	\end{enumerate}
\item 골다공증 치료제에는 호르몬요법(Estrogen, Estrogen derivatives 등)과 비호르몬요법 (Bisphosphonate, Elcatonin, 활성형 Vit.D3, Raloxifene 및 Bazedoxifene제제 등)이 있으며, \textcolor{red}{호르몬요법과 비호르몬요법을 병용투여하거나 비호르몬요법 간 병용투여는 인정하지 아니함} 다만 아래의 경우는 인정 가능함 \newline
- 다 음 -
	\begin{enumerate}[가.]\tightlist
	\item 칼슘제제와 호르몬대체요법의 병용
	\item 칼슘제제와 그 외 비호르몬요법의 병용
	\item Bisphosphonate와 Vit. D 복합경구제(성분: Alendronate + Cholecalciferol 등)를 투여한 경우
	\end{enumerate}
\item 특정소견 없이 단순히 골다공증 예방목적으로 투여하는 경우에는 비급여 함  고시 제2015-68호 (2015.05.01. 시
\end{enumerate}
\end{commentbox}

\Que{골다공증치료제 복용 중 T-score>-2.5로 상승한 경우 급여 인정여부}
\Ans{골다공증치료제 급여기준에 해당되지 아니하므로 \textcolor{red}{급여 인정하지 아니함.}\par
   예시1) 2011.5월 DXA 또는 QCT로 골밀도 검사 후 골다공증 약제 복용하던 중 2012.5월 골밀도 검사 결과 T-score가 -2.0이 나왔다면 검사 이후의 약제 처방은 보험적용 대상이 아님.}

\Que{DXA 측정시 Central bone의 범위}
\Ans{
\begin{enumerate}[ ○ ]\tightlist
\item 요추와 대퇴 부위를 측정하되, \textcolor{red}{대퇴 중에서 Ward's triangle 부위는 제외함.}\par
\item Ward's triangle 측정시, T-score≤-3.0이더라도 급여 인정하지 아니함.
\item DXA를 이용하여 wrist, ankle 등 peripheral부위를 측정한 경우는 1)항에 포함되지 않고, 3)항에 준하여 급여 인정함.
\end{enumerate}
}

\Que{2011.10.1. 이전 시행한 골밀도 검사 결과로 치료제 급여적용이 가능한지 여부}
\Ans{2011.10.1. 이전 시행한 골밀도 검사 결과라도 \textcolor{red}{투약개시일 기준 1년 이내에 시행한 검사 결과라면 인정함.}}

\Que{골밀도검사 측정결과의 해석방법}
\Ans{\textcolor{red}{요추 L1~L4 중 2부위 이상의 평균 골밀도 또는 ward's triangle 부위를 제외한 대퇴부 측정값 중 낮은 부위를 기준으로 적용함.}\par
   \textcolor{red}{※ L1~L4 중 가장 낮은 값의 적용은 인정하지 아니함.}}

\Que{칼슘 및 estrogen 제제의 검사결과지 첨부여부 }
\Ans{기존대로 \textcolor{red}{검사결과지를 첨부하지 아니함.}}

\Que{칼시토닌, raloxifene제제, 활성형 Vit D3제제 및 bisphosphonate 제제 등의 약제 원외처방 및 원내조제시 명세서 작성 방법 }
\Ans{줄번호 단위 특정내역 JX999(기타내역)란에 ‘검사결과/장비명주1/검사기관주2로 기재
\begin{enumerate}[주1)]\tightlist
\item DXA, QCT로 central bone의 골밀도를 측정하여 DXA는 T-score≤-2.5, QCT는 80mg/㎤ 이하인 경우 기재함.
\item 타 요양기관에서 검사한 경우 해당 요양기관을 기재함.
\end{enumerate}
}
