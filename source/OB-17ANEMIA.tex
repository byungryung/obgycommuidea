\section{ANEMIA}
\myde{}{%
\begin{itemize}\tightlist
\item[\dsjuridical] D500 Iron deficiency anemia secondary to blood loss(chronic)
\item[\dsjuridical] D62 Acute posthemorrhagic anemia
\item[\dsjuridical] R42 Dizziness and giddiness
\item[\dsjuridical] O72 분만후 출혈
\item[\dsjuridical] O67 분만중 출혈
\item[\dschemical] B1010 혈색소(광전비색), B1020 헤마토크리트, B1040 적혈구수, B1050 백혈구수, B1060 혈소판수, B1091 백혈구 백분율(혈액), B1220 적혈구분포계수, B1230 혈소판분포계수, B1081 망상적혈구수(유세포분석법), B1100006 혈구형태(말초혈액도말), C2940 철, C2500 철결합능, C2520 훼리틴, C2532 비타민 B12, C2540 엽산 
\end{itemize}
}%
{
Definition>>\par
\begin{itemize}\tightlist
\item Women, non-pregnant(>15yrs) : 12.0
\item Women, pregnant : 11.0
\item Men (>15yrs) : 13.0
\end{itemize}
}
\prezi{\clearpage}
\subsection{저가의 철분제는 증상만으로도 보험처방 가능}
저가의 철분제의 경우 세부인정기준이 없으므로 진료담당의사가 환자진료에 반드시 필요하다고 판단하여 시행한 경우 허가사항범위내에서 필요, 적절하게 처방 투여된 의약품은 요양 급여가 가능하므로 \textcolor{red}{어지러움을 호소하는 환자나 임산부에게 빈혈 예방 차원에서 저가의 철분제를 처방할 경우 모두 보험 적용 가능}합니다.
\begin{itemize}\tightlist
\item 훼마톤 에이 정 (78원/1정) : 성분:ferric hydroxide-polymaltose complex 357㎎ (100㎎ as iron) + folic acid 350㎍ 
\item 훼럼포라 정 (78원/1정) : 성분:ferric hydroxide-polymaltose complex 357㎎ (100㎎ as iron) + folic acid 350㎍ 
\item 훼로바-유 서방정 (95원/1정) : 성분:dried ferrous sulfate 256㎎ (80㎎ as iron)
\item 헤모골드-에프 정 45mg (97원/1정) : 성분:carbonyl iron 45㎎
\item 헤모니아 캅셀 150mg (102원/1캅셀) : 성분:polysaccharide-iron comlplex 326.1㎎ (150㎎ as iron) 
\end{itemize}
\prezi{\clearpage}
\subsection{기타의 철분제}
\begin{itemize}\tightlist
\item Chondroitin sulfate-iron complex 경구제 (품명: 리코베론과립, 페리콘캡슐 등)
\item Iron acetyl-transferrin 200mg 경구제 (품명: 알부맥스캅셀 등) 
\item Iron proteinsuccinylate 400mg 경구제 (품명: 헤모큐츄어블정) 
\item 액제형 철분제제 (품명 : 헤모큐액 등) 
\end{itemize}
\prezi{\clearpage}
\subsection{Iron proteinsuccinylate 400mg 경구제(품명: 헤모큐츄어블정)의 급여기준}
허가사항 범위 내에서 아래와 같은 기준으로 투여 시 요양급여를 인정하며, 동 인정기준 이외에는 약값 전액을 환자가 부담토록 함.\par
-아 래-
\begin{enumerate}[가)]\tightlist
\item 일반적인 철결핍성 빈혈에는 혈액검사결과 다음에 해당되고 타 경구 철분제제 투여 시 위장장애가 있는 경우에 급여하며, 투여기간은 통상 4~6개월 급여함.
	\begin{enumerate}[1)]\tightlist
	\item 일반 환자 혈청페리틴(Serum ferritin) 12ng/㎖ 미만 또는 트란스페린산호포화도(Transferrin saturation rate) 15\% 미만인 경우 
	\item 만성신부전증 환자Serum ferritin 100ng/㎖ 미만 또는 Transferrin saturation rate 20\% 미만인 경우 
	\end{enumerate}
\item 임신으로 인한 철결핍성 빈혈혈액검사결과 Hb 10g/㎗ 이하이고 타 경구 철분제제 투여 시 위장장애가 있는 경우에 급여하며, 투여기간은 4~6개월로 함. 
\item 급성출혈 등으로 인한 산후 빈혈혈액검사결과 Hb 10g/㎗ 이하인 경우에 급여하며, 투여기간은 4주로 함.    
	\begin{itemize}[*]\tightlist
	\item 시행일: 2013.9.1.
	\item 종전고시: 고시 제2011-163호(2012.1.1.)
	\item 변경사유: 용어정비
	\end{itemize}
\end{enumerate}
\prezi{\clearpage}
\subsection{액제형 철분제제(품명 : 헤모큐액 등) 급여기준}
허가사항 범위 내에서 아래와 같은 기준으로 투여 시 요양급여를 인정하며, 동 인정기준 이외에는 약값 전액을 환자가 부담토록 함.\par
-아 래-
\begin{enumerate}[가.]\tightlist
\item 일반적인 철결핍성 빈혈에는 혈액검사결과 다음에 해당되고 타 경구 철분제제 투여 시 위장장애가 있는 경우에 급여하며, 투여기간은 통상 4~6개월 급여함.
	\begin{enumerate}[1)]\tightlist
	\item 일반 환자 혈청페리틴(Serum ferritin) 12ng/㎖ 미만 또는 트란스페린산호포화도(Transferrin saturation rate) 15\% 미만인 경우 
	\item 만성신부전증 환자Serum ferritin 100ng/㎖ 미만 또는 Transferrin saturation rate 20\% 미만인 경우 
	\end{enumerate}
\item 임신으로 인한 철결핍성 빈혈혈액검사결과 Hb 10g/㎗ 이하이고 타 경구 철분제제 투여 시 위장장애가 있는 경우에 급여하며, 투여기간은 4~6개월로 함. 
\item 급성출혈 등으로 인한 산후 빈혈혈액검사결과 Hb 10g/㎗ 이하인 경우에 급여하며, 투여기간은 4주로 함.    
	\begin{itemize}[*]\tightlist
	\item 시행일: 2013.9.1.
	\item 종전고시: 고시 제2005-57호(2005.9.1.)
	\item 변경사유: 용어정비
	\end{itemize}
\item 8세 미만의 소아는 철결핍성 빈혈이 확인된 경우 1차로 투여 시에도 요양급여하며, 미숙아의 경우는 예방 투여 시에도 인정함.
\end{enumerate}
\prezi{\clearpage}
\subsection{철분주사제(품명: 부루탈주 등)의 급여기준}
허가사항 범위 내에서 아래와 같은 기준으로 투여한 경우로서 요양급여비용 청구 시 매월 혈액검사 결과지, 철결핍을 확인할 수 있는 검사결과지, 투여소견서가 첨부된 경우에 요양급여를 인정하며, 동 인정기준 이외에는 약값 전액을 환자가 부담토록 함.\par
-아 래-
\begin{enumerate}[가)]\tightlist
\item 일반 환자헤모글로빈(Hb) 8g/dl이하이고 경구투여가 곤란한 경우로서 출혈 등이 있어 철분을 반드시 신속하게 투여할 필요성이 있는 철결핍성 빈혈환자로 혈청 페리틴(Serum ferritin) 12ng/㎖ 미만 또는 트란스페린 포화도(Transferrin saturation) 15\%미만인 경우(인페드주의 경우 투여용량은 8㎖ 이내) 
\item 투석중이 아닌 만성신부전증 환자Hb 10g/dl 이하인 경우에 투여하고, 목표(유지) 수치는 Hb 11g/dL까지 요양 급여를 인정하며, Serum ferritin 100ng/㎖ 미만 또는 Transferrin saturation 20\% 미만인 경우(다만, 경구투여가 곤란한 경우만 인정) 
\item 투석중인 만성신부전증 환자Hb 11g/dl 이하인 경우에 투여 시 인정하며,
	\begin{enumerate}[1)]\tightlist
	\item Serum ferritin 100ng/㎖ 미만 또는 Transferrin saturation 20\% 미만인 경우(다만, 복막투석환자는 경구투여가 곤란한 경우만 인정)
	\item 충분한 양의 Erythropoietin주사제를 투여함에도 빈혈이 개선되지 않는 Erythropoietin주사제 저항인 경우에는 Serum ferritin 300ng/㎖ 미만 또는 Transferrin saturation30\%미만인 경우
	\end{enumerate}
\item 항암화학요법을 받고 있는 비골수성 악성종양을 가진 환자 Hb 10g/dl 이하인 경우로서 
	\begin{enumerate}\tightlist
	\item 경구투여가 곤란한 환자로 Serum ferritin 100ng/㎖ 미만 또는 Transferrin saturation 20\%미만인 경우
	\item 충분한 양의 Erythropoietin주사제를 투여함에도 빈혈이 개선되지 않는 Erythropoietin주사제 저항인 경우에는 Serum ferritin 300ng/㎖ 미만 또는 Transferrin saturation 30\%미만인 경우
  	\end{enumerate}
	\begin{itemize}[*]\tightlist
	\item 시행일: 2013.9.1.
	\item 종전고시: 고시 제2013-34호(2013.3.1.)
	\item 변경사유: 용어정비
	\end{itemize}
\end{enumerate}