\section{제출 요구 서류 및 그 의미}
\begin{enumerate}[①]\tightlist
\item 인력 관련
	\begin{itemize}\tightlist
	\item 직원 명부(현재 근무자 및 \textcolor{red}{1년 이내 퇴사자 포함}) 
	\item 급여 대장 및 근로 계약서
	\item 근무일지(Duty 표)
	\item 의사, 간호사, 의료기사, 약사 면허증 사본
	\end{itemize}
\item 시설 관련
	\begin{itemize}\tightlist
	\item 보건소 등록된 평면도
	\item 식당위탁계약서 / 식사 대장
	\item 적출물 관리대장 / 배출시설 운영 일지
	\end{itemize}	
\item 물품 관련
	\begin{itemize}\tightlist
	\item 의약품 및 진료용 재료 구입 관련 증빙서류\newline
 (의약품 구입 목록 대장/의약품 수불 대장/거래 명세표)\newline
 (의료비품 대장/소모품 수불 대장/거래 명세표) 
	\item 마약관리대장
	\end{itemize}	
\item 검사 관련
	\begin{itemize}\tightlist
	\item 방사선과 : 촬영대장, PACS 데이터, 판독기록 
	\item 임상병리실 : 검사대장 / 수탁 검사 항목표 / 검사 결과지(원내 / 수탁)
	\end{itemize}	
\item 장비 관련
	\begin{itemize}\tightlist
	\item 의료 장비 보유 및 구입에 관한 증빙서류
	\item 방사선 촬영장치 정기검사 성적서(X-ray, Mammo), 설치등록증명서
	\item 임상 병리과 검사 장비 종류 확인해서 심평원 신고 된 내용과 일치하는 지 확인함 
	\item 수술 장비 등도 심평원 신고 내용과 일치하는 확인함(일부 병원에서는 실제 사용이 가능한 장비인지 작동해 보기도 한다고 함)
	\end{itemize}	
\item 입원 관련 서류
	\begin{itemize}\tightlist
	\item 공통 내용 : \textcolor{red}{입원 DB}/입퇴원 장부/일일 수납대장/\textcolor{red}{진료비 영수증(상세 내역)}/본인 부담금 수납대장 
	\item 비급여 항목 관련 : \textcolor{red}{비급여 리스트/수납대장}/수진자별 리스트
	\item 급여 항목 관련 : 요양급여비용 계산서 / 요양급여비용 심사 청구서 및 명세서
	\item \textcolor{red}{진료 기록부(의사 및 간호 차트/개인별 투약 기록/방사선 판독 결과 \& 임상병리 결과지)}
	\end{itemize}	
\item 외래 진료 관련 서류
	\begin{itemize}\tightlist
	\item 공통 내용 : \textcolor{red}{외래 DB}/수진자별 접수 대장/일일 수납대장/\textcolor{red}{진료비 영수증(상세 내역)}/본인 부담금 수납대장 
	\item 비급여 항목 관련 : \textcolor{red}{비급여 항목 리스트/수납대장}/수진자별 리스트
	\item 급여 항목 관련 : 요양급여비용 계산서 / 요양급여 비용 심사 청구서 및 명세서 
	\item \textcolor{red}{진료 기록부(의사 차트/오더 기록/방사선 판독 결과 \& 임상병리 결과지)}
	\end{itemize}	
\item 외래 및 입원 DB 관련 설명
	\begin{itemize}\tightlist
	\item DB에서 조사관들이 보는 자료는 우리가 보는 차트가 아님.
 ( 의사 차트는 보지 못하고 처방 기록만 볼 수 있는 것 같음.그래서 의사 차트 열람이 필요한 경우에는 해당 환자의 차트를 출력해 달라고 함)
	\item 처방이나 차팅 시간을 확인하는 것 같음.(추후에 수정한 내용들은 무시하고 실제 진료 시기에 기록한 내용들을 기준으로 판단함)
	\item DB는 반드시 줘야 하는지 ? (안주면 1년 이하의 업무 정지)
	\item 전자 차트 업체는 우리편이 아니라 심평원 편임
	\end{itemize}	
\end{enumerate}

