\section{DRG 기타진단의 진단기준}
\subsection{유의한 단백뇨를 동반하지 않은 임신성(임신-유발성)고혈압(O13)}\label{SUPPIH}
 : 정상혈압을 갖고 있던 여성에서 임신20주 이후에 수축기 혈압이 140 mmHg 이상이거나 확장기 혈압이 90mmHg 이상, 6시간 이상의 간격으로 최소한 2번 이상 증명되고 분만 후까지 단백뇨가 동반되지 않고 고혈압으로 남아 있는 경우

\subsection*{임신중독증(O14)}\label{PIH}
 : 아래의 혈압과 단백뇨의 조건이 모두 충족되는 경우
\begin{itemize}\tightlist
\item 혈압 : 정상혈압을 갖고 있던 여성에서 임신20주 이후에 수축기 혈압이 140mmHg 이상이거나 확장기 혈압이 90mmHg 이상, 6시간 이상의 간격으로 최소한 2번 이상 증명된 경우
\item 단백뇨 : 6시간 이상의 간격으로 2+이상(또는 100mg/dl이상) 2번 이상 증명된 경우 또는 24시간 요중에 단백질이 300mg 이상 존재가 확인된 경우
\end{itemize}

\subsection*{중증의 전자간 (O141)}\label{severePIH}
 - 전자간증이면서 다음의 기준 중 1개 이상 충족
\begin{itemize}\tightlist
\item 환자가 침상 안정 상태에서 적어도 6시간 간격으로 2회에 걸쳐 수축기 혈압 160mmHg 이상 또는 확장기 혈압 110mmHg 이상
\item 24시간 채뇨 소변에서 5gm이상의 단백뇨 또는 적어도 4시간 간격으로 2회 채뇨 점적뇨에서 3+이상
\item 24시간 500ml 이하의 핍뇨
\item 대뇌 장애 또는 시력 장애
\item 폐부종 또는 청색증
\item 상복부 또는 우상복부통증
\item 간기능 장애
\item 혈소판 감소증
\item 태아발육지연
\end{itemize}

\subsection*{헬프(HELLP) 증후군 (O142)}\label{HELLP}
 - 다음의 기준을 모두 충족
\begin{itemize}\tightlist
\item 용혈(hemolysis) : Abnormal peripheral blood smear (microangiopathic anemia), Increased bilirubin ≥ 1.2mg/dl, Increased LDH> 600 IU/L
\item 간효소치 상승 : Increased AST ≥ 72 IU/L, Increased LDH as above
\item 저혈소판혈증 Platelet count < 100×103/μl
\end{itemize}

\subsection*{자간증(O15)}\label{eclampsia}
 :임신중독증(O14)의 조건을 충족하면서 임신성 고혈압에 의해 경련(Convulsion)이 동반된 경우
\prezi{\clearpage}
\subsection*{분만 전 출혈(O46)}
 : 분만 전에(활발한 진통이 시작되기 전) 출혈이 있어 입원한 경우 또는 입원하여 분만 전에 출혈량에 관계없이 출혈이 있었던 경우(혈성이슬 제외)
\begin{itemize}\tightlist
\item 응고장애를 동반한 경우 응고장애를 동반한 분만 전 출혈(O460) 부여 가능
\end{itemize}

\subsection*{분만 중 출혈(O67)} : 분만 중 (활발한 진통이 시작된 후부터 태아의 만출까지 : 분만 제 1.2기)에 과다출혈이 있었던 경우로 분만전(입원당시)과 비교 Hct가 10\%이상 감소한 경우이거나 수혈이 필요하여 수혈을 실시한 경우
\begin{itemize}\tightlist
\item 응고장애를 동반한 경우에는 응고장애를 동반한 분만 중 출혈(O670) 부여 가능
\end{itemize}
\subsection*{분만 후 출혈(O72)} : 분만 제3기부터 분만 후 6주 이내(조기산후출혈과 지연산후출혈을 모두 포함)에 과다 출혈이 있었던 경우로 분만전(입원당시)과 비교 Hct가 10\%이상 감소한 경우이거나 수혈이 필요하여 수혈을 실시한 경우
\begin{itemize}\tightlist
\item 응고장애를 동반한 경우에는 O723(분만 후 응고 결여) 부여 가능
\end{itemize}
\prezi{\clearpage}
\subsection*{급성 출혈 후 빈혈(D62)} : 외과적 수술, 처치 후 다량의 출혈로 수술전(입원당시)과 비교 Hgb과 Hct 수치의 10\% 이상 감소 및 Hb 10g/dl 미만으로 저하되어 이에 대한 치료가 이루어진 경우(약제투여, 수혈 등)

\begin{shaded}
단지 출혈이 있었다고 해서 이러한 기타진단을 쓸수 있는것을 아니고, 이러한 출혈에 대한 어떠한 행위가 있어야 기타진단으로 인정됩니다. 분만전,중 출혈에서는 nalador와 merthergin사용, 분만후 출혈에서는 베노훼럼주 사용등이 있습니다. 
\end{shaded}

\subsection*{어느 개인병원에서는}
\begin{enumerate}[가.]\tightlist
\item 제왕절개시에 출혈이 많다고 생각되면 nalador와 merthergin사용하고, 분만중 출혈(o67)코드 넣자. 부인과 수술시 출혈이 많으면 수술중 적절한 처치후 D62 (급성출혈) 상병추가.
\item 3층 입원실에서는 수술후 Hgb이 8이하 이면(절개시 nalador사용과 관계없이) 
\item 3층 입원실에서는 수술전과 후 Hct비교하여서 10\%이상 떨어져 있고, 수술중 nalador사용을 하지 않을시는 
	\begin{itemize}\tightlist
	\item 분만후 출혈(o72)코드 집어 넣고(회진닥터)
	\item 베노훼럼주 1A + N/S 100cc (15분 이상 slowly IV)
	\item 그 이후 볼그래액 1pack를 저녁으로 복용하게 한다. (입원시에는)
	\item 5일째 CBC 추적조사하게 함.(Hct 10\%이상 감소변화 없거나 퇴원후에도 심한 빈혈로 철분제가 필요하면 훼로바-유서방정 아침저녁으로 7일간 복용처방) 
	\item 퇴원 1주후 CBC F/U
	\item 처방의가 필요에 따라서 베노훼럼주의 용량은 늘리수 있습니다. 아래 참조.
	\end{itemize}
\end{enumerate}
\subsection*{\newindex{베노훼럼주} 사용} 
다음의 \pageref{VenoferrumInj}를 참조하세요.

\prezi{\clearpage}

\subsection*{가진통(O47)}
 : 임신 만기 전에 자궁의 불규칙적인 수축으로 인한 통증으로 수축이 자연 소실되거나 자궁경관의 개대가 없는 상태로 분만으로 이어지지 않은 진통으로 확인된 경우
\prezi{\clearpage}
\subsection*{산후기 패혈증(O85)} : 분만 후 첫 24시간을 제외한 산후 10일 이내에 2일간 계속하여 38°C(100.4°F) 이상의 체온상승이 확인된 경우
\prezi{\clearpage}
\subsection*{고령 초임산부의 관리(Z355)} : 초임산부로서 만 35세 이상인 경우 \label{oldprimi} \index{진단코드!고령초임산부의 관리}
\subsection*{어린 초임산부의 관리(Z356)} : 초임산부로서 만 16세 미만인 경우 \label{youngprimi} \index{진단코드!어린 초임산부의 관리}
\subsection*{기타 고위험 임신의 관리(Z358)} : 경산으로 만 40세 이상인 경우와 만 35세 이상인 경산으로 전 출산과 만 5년 이상 Interval이 있는 경우 \label{otherhigh} \index{진단코드!기타 고위험 임신의 관리}
\prezi{\clearpage}
\subsection*{달리 분류되지 않은 처치에 의한 감염(T814)} : 수술부위의 통증, 국소 종창, 발적, 열감 등의 감염 징후를 동반하면서
\begin{itemize}\tightlist
\item 표재성 창상, 심부 절개 부위 및 기관/강 등 외과수술 부위에서 농성 분비물이 나오는 경우
\item 무균 처치시 획득된 체액이나 조직에서 미생물의 배양이 확인된 경우
\item 무균 처치시 획득된 체액이나 조직에서 미생물이 분리된 경우 등으로 외과의사나 주치의사의 판단에 의해 감염으로 진단한 경우
\end{itemize}
\prezi{\clearpage}
\subsection*{인슐린-의존 당뇨병(E10)} : 한국표준질병사인분류에 의하여 당뇨병이 불안정형(brittle), 연소성발병형(juvenile-onset), 케토증경향(ketosis-prone) (typeI)인 경우 또는I형
\begin{shaded}
GDM등도 기타진단으로 가능하다는 최근에 심평원의 대답을 듣었습니다. 물론 GDM때문에 추가적인 BST등을 했을때 인정됩니다.
\end{shaded}

\begin{tcolorbox}[frogbox,title=기타진단 기준tip]
\begin{itemize}\tightlist
\item Gestational DM : 입원하여 FBS, BS check
\item PPH : hysterotonic drug사용하고, postoperative CBC 체크상 Hct가 10이상 떨어져 있는 경우
\item 고위험산모에 대한 정확한 이해 및 숙지
	\begin{itemize}\tightlist
	\item 고령 초임산부의 관리 (Z355) : 초임산부로서 만 35세 이상
	\item 기타고위험 임신의 관리 (Z358) : 경산으로 만 40세 이상인 경우와 만 35세 이상인 경산으로 만 5년 이상 interval이 있는 경우
	\end{itemize}
\item Preterm입원후 6일내에 제왕절개시 부가진단 입력(O47) http://goo.gl/7YBexP
\item severe PIH등으로 제왕절개시 주진단으로 상병 넣지 말고 다른 CPD등 주진단으로 하고 부진단 http://goo.gl/q9p5G6
\item 무통분만으로 인한 leakage시 o295 임신중 척추및 경마괴 마취로 유발된 통증
\item 검사하는 것을 귀찮아하지 말고 무엇이든 근거를 남겨라. 
\end{itemize}
\end{tcolorbox}
