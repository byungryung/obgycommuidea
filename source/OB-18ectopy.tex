\section{Ectopic pregnancy:Surgical}
\myde{}{%
\begin{itemize}\tightlist
\item[\dsjuridical] O001 Tubal pregnancy
\item[\dsjuridical] O002 Ovarian pregnancy
\item[\dsjuridical] O008 Other site pregnancy, cervical, cornual 
\item[\dsmedical] R4531 자궁외임신수술-난관또는난소임신 \myexplfn{2757.20} 원
\item[\dsmedical] R4532 자궁외임신수술-자궁각임신 \myexplfn{3284.08} 원
\item[\dsmedical] R4533 자궁외임신수술-지긍경관임신 \myexplfn{4170.42} 원
\item[\dsmedical] C5916 Biopsy 파라핀6개 이하 \myexplfn{445.87} 원
\end{itemize}
}%
{\begin{enumerate}[1.]\tightlist
\item 자궁각 임신에 의한 자궁파열, 난소의 임신에 의한 난소부분절제술, 나853나 절개생검-심부-개복에 의한것 이외의 난관임신이 메인이 되는 복강경하수술은 포함되지 않습니다.
\end{enumerate}}
\prezi{\clearpage}
\subsection{자궁외임신 상병으로 파열된 곳은 자궁외임신수술을 하고 파열되지 아니한 쪽(대칭기관)에 난관결찰술시행시 수기료 산정방법}
자궁외임신으로 인하여 난소및 난관파열시 파열된 곳은 자궁외임신 수술을 하고 파열되지 아니한쪽(대칭기관)에 대하여 난관결찰술을 시행한 경우에는 자453 자궁외임신수술 100\%와 자434 난관결찰술 50\%로 산정함.
\begin{enumerate}\tightlist
\item Laparoscopic tubal ligation R4341 \myexplfn{1251.56}
\item Laparoscopic tubal fulguration R4342 \myexplfn{1313.58}
\item Laparotomic tubal ligation R4345 \myexplfn{1077.95}
\end{enumerate}

\prezi{\clearpage}
\section{Ectopic pregnancy:Medical}
\myde{}{%
\begin{itemize}\tightlist
\item[\dsjuridical] O001 Tubal pregnancy
\item[\dsjuridical] O002 Ovarian pregnancy
\item[\dsjuridical] O008 Other site pregnancy, cervical, cornual 
\item[\dsmedical] C3520 hCG \myexplfn{212.70} 
\end{itemize}
}%
{\begin{enumerate}[(1)]\tightlist
\item 자궁외임신 진단목적으로 S-HCG 인정(Serum), OP후는 자연 소실되므로 불인정, OP하지 않고 자연소실을 기대하여 추적 관찰시 첫회 실시후
     48시간 간격으로 2회 그후는 주1회 총 5-6회 인정함을 원칙으로 함.
\item 포상기태에 주1회 인정
\item 절박유산에 7-10일 간격으로 4-5회 인정
\item 자궁경부암에 Tumor Marker로서 의미 없음 (CEA만 인정) (1994/10/24)
\item 난소과자극증후군(OHSS : Ovarian Hyper Stimulation Syndrome)에 태아의 이상유무를 판단하기 위한 β-HCG 검사는 통상 주 1-2회 정도 시행함을 원칙으로 하며 그 이상 시행한 경우에는 의사소견서 참조하여 사례별로 인정한다.(2007.09.01)
\end{enumerate}
\begin{itemize}[○]\tightlist
\item 투여대상 : 혈류역학적(hemodynamic)으로 \dotemph{안정(stability)상태이고 복강 내에 난관 파열 등으로 인한 출혈소견이 없는 경우로서 다음 조건 중 하나이상을 충족하는 경우}에 인정함.(단, 동 약제 금기대상은 제외)
	\begin{enumerate}[①]\tightlist
	\item 임신 6주 이하
	\item 혈청 β-hCG가 15,000mIU/ml 이하
	\item 자궁외 임신으로 인한 난관의 종괴가 3.5㎝ 이하 
	\item 초음파검사에서 태아 심박동이 없는 경우
	\end{enumerate}
\item 투여요법  
	\begin{enumerate}[①]\tightlist
	\item 복합요법(Methotrexate 1mg/kg/day와 Leukovorin 0.1mg/kg/day를 교대로 근주, 각각 4일) 또는 
	\item 단일요법(Methotrexate 50mg/㎡ 근주)
	\end{enumerate}
\end{itemize}
}

\subsection{Body surface area}
\url{http://halls.md/body-surface-area/bsa.htm} 참조.
단일요법시엔 BSA * 50mg하면 됩니다.

\subsection{PROTOCOL}
\begin{enumerate}\tightlist
\item MTX contraindication
	\begin{itemize}\tightlist
	\item unstalbe V/S
	\item ruptured
	\item breastfeeding, Immunodeficiency, acoholism, peptic ulcer, hepatic ds, renal ds.
	\item G-sac size \textgreater 3.5cm, FHB(+)
	\end{itemize}
\item MTX Tx. Regimen
	\begin{itemize}\tightlist
	\item Single dose
		\begin{itemize}\tightlist
		\item administer MTX 50mg/m2 (BSA) on day 0
		\item measure hCG level on day 4 \& 7 
		\item \MVRightarrow if level drop by 15\% \MVRightarrow monitor hCG weakly until normal level
		\item if do not drop by 15\% \MVRightarrow repeat MTX
		\end{itemize}
	\item Two dose
		\begin{itemize}\tightlist
		\item administer MTX 50mg/m2 (BSA) on day 0 \& 4
		\item measurement hCG level on day 4 \& 7
		\item if level drop by 15\% \MVRightarrow monitor hCG weakly until normal level
		\item if do not drop by 15\% \MVRightarrow repeat MTX day 7 \& 11 and hCG on day 7 \& 11 or surgical intervention
		\end{itemize}
	\item Multi-dose
		\begin{itemize}\tightlist
		\item administer MTX 1mg/kg IM on day 1,3,5,7
		\item administer leucovorin 0.1mg/kg on day 2,4,6,8
		\item measure hCG level + LFT on days 1,3,4,7 until level drop by 15\%
		\item once hCG level drop by 15\% stop MTX and monitor hCG weakly until normal level
		
		\end{itemize}
	\end{itemize}
\end{enumerate}
