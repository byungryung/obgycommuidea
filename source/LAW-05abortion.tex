\section{낙태죄 Abortion}
\subsection{관계법령}
\paragraph{형법}\par
심각함gg
\emph{제269조] (낙태)}
	\begin{enumerate}[①]\tightlist
	\item 부녀가 약물 기타 방법으로 낙태한 때에는 1년 이하의 징역 또는 200만원 이하의 벌금에 처한다. [개정 1995.12.29]
	\item 부녀의 촉탁 또는 승낙을 받어 낙태하게 한 자도 제1항의 형과 같다. [개정 1995.12.29]
	\item 제2항의 죄를 범하여 부녀를 상해에 이르게 한 때에는 3년 이하의 징역에 처한다. 사망에 이르게 한 때에는 7년 이하의 징역에 처한다. [개정 1995.12.29]
	\end{enumerate}
\emph{[제270조] (의사 등의 낙태, 부동의낙태)}
	\begin{enumerate}[①]\tightlist
	\item 의사, 한의사, 조산사, 약제사 또는 약종상이 부녀의 촉탁 또는 승낙을 받어 낙태하게 한 때에는 2년 이하의 징역에 처한다.  <개정 1995.12.29.>
	\item 부녀의 촉탁 또는 승낙없이 낙태하게 한 자는 3년 이하의 징역에 처한다
	\item 제1항 또는 제2항의 죄를 범하여 부녀를 상해에 이르게 한때에는 5년 이하의 징역에 처한다. 사망에 이르게 한때에는 10년 이하의 징역에 처한다.  <개정 1995.12.29.>
	\item 전 3항의 경우에는 7년 이하의 자격정지를 병과한다
	\end{enumerate}
%\end{description}

\paragraph{모자보건법}\par
\emph{제14조(인공임신중절수술의 허용한계)}
	\begin{enumerate}[①]\tightlist
	\item 의사는 다음 각 호의 어느 하나에 해당되는 경우에만 본인과 배우자(사실상의 혼인관계에 있는 사람을 포함한다. 이하 같다)의 동의를 받아 인공임신중절수술을 할 수 있다.
		\begin{enumerate}[1.]\tightlist
		\item 본인이나 배우자가 대통령령으로 정하는 우생학적(優生學的) 또는 유전학적 정신장애나 신체질환이 있는 경우
		\item 본인이나 배우자가 대통령령으로 정하는 전염성 질환이 있는 경우
		\item 강간 또는 준강간(準强姦)에 의하여 임신된 경우
		\item 법률상 혼인할 수 없는 혈족 또는 인척 간에 임신된 경우
		\item 임신의 지속이 보건의학적 이유로 모체의 건강을 심각하게 해치고 있거나 해칠 우려가 있는 경우
		\end{enumerate}
	\item 제1항의 경우에 배우자의 사망\bullet실종\bullet행방불명, 그 밖에 부득이한 사유로 동의를 받을 수 없으면 본인의 동의만으로 그 수술을 할 수 있다.
	\item 제1항의 경우 본인이나 배우자가 심신장애로 의사표시를 할 수 없을 때에는 그 친권자나 후견인의 동의로, 친권자나 후견인이 없을 때에는 부
	양의무자의 동의로 각각 그 동의를 갈음할 수 있다.[전문개정 2009.1.7.]
	\end{enumerate}
\emph{제28조(「형법」의 적용 배제)}\par
이 법에 따른 인공임신중절수술을 받은 자와 수술을 한 자는 「형법」 제269조제1항\bullet제2항 및 제270조제1항에도 불구하고 처벌하지 아니한다.[전문개정 2009.1.7.]\par \medskip

\paragraph{의료법}\par
\emph{제8조(결격사유 등)} 다음 각 호의 어느 하나에 해당하는 자는 의료인이 될 수 없다.  <개정 2007.10.17.>
	\begin{enumerate}[1.]\tightlist
	\item 「정신보건법」 제3조제1호에 따른 정신질환자. 다만, 전문의가 의료인으로서 적합하다고 인정하는 사람은 그러하지 아니하다.
	\item 마약\bullet대마\bullet향정신성의약품 중독자
	\item 금치산자\bullet한정치산자
	\item 이 법 또는 「형법」 제233조, 제234조, 제269조, 제270조, 제317조제1항 및 제347조(허위로 진료비를 청구하여 환자나 진료비를 지급하는 기관이나 단체를 속인 경우만을 말한다), 「보건범죄단속에 관한 특별조치법」,「지역보건법」,「후천성면역결핍증 예방법」,「응급의료에 관한 법률」,「농어촌 등 보건의료를 위한 특별 조치법」,「시체해부 및 보존에 관한 법률」,「혈액관리법」,「마약류관리에 관한 법률」,「약사법」,「모자보건법」, 그 밖에 대통령령으로 정하는 의료 관련 법령을 위반하여 금고 이상의 형을 선고받고 그 형의 집행이 종료되지 아니하였거나 집행을 받지 아니하기로 확정되지 아니한 자
	\end{enumerate}
\subsection{헌법재판소 결정 및 낙태죄의 위헌성}
\paragraph{헌법재판소 결정}
\begin{itemize}\tightlist
\item 2012.8.23.자2010헌바402 결정: 결론은 합헌(정족수 부족)
\item 재판관 4인의 합헌의견
	\begin{itemize}\tightlist
	\item 태아의 독자적 생존 여부를 낙태죄 허용의 기준으로 삼을수 없음
	\item 임부의 자기결정권(사익)이 태아의 생명권(공익)보다 중하다고 볼 수없음.
	\end{itemize}
\item 재판관 4인의 위헌의견
	\begin{itemize}\tightlist
	\item 인간생명의 발달 단계에 따라 보호 정도나 보호수단을 달리 할 수 있으므로 임신 초기 낙태까지 전면적\bullet일률적으로 금지하는 것은 임부의 자기결정권 침해
	\item 형법상 낙태죄 규정이 사문화된 현실도 고려
	\item 임신 12주를 일응의 기준으로 제시
	\end{itemize}
\end{itemize}
\subsection{낙태죄의 현실}
\begin{itemize}\tightlist
\item 2005년 9월 보건복지부 발표 전국 인공임신중절 실태조사 결과
	\begin{itemize}\tightlist
	\item 한해 낙태시술 추정 건수 약 34만 2천여건 : 통계 주관기관마다 차이(최대 200만건까지 추정)
	\item 그중 14만 3천 여건이 미혼 여성의 낙태(약 42\%)
	\item 나머지 기혼 여성 낙태도 단산 혹은 터울 조절이 이유(약 75\%)
	\end{itemize}
\item 낙태시술 추정 건수 중 4.4\%정도만 관련 법령상 합법 조건 충족
\item 불법 낙태의 약 90\%는 사회 경제적 사유
\item 낙태시술 처정 건수 중 96.3\%가 임신 12주 미만에
\item 낙태죄를 처발해도 낙태를 감행할 것이라는 의견 84.6\%	
\end{itemize}
\Large \emph{낙태 실형 판결시대, 형법 및 모자보건법 개정을 휘한 입법적인 노력이 필요하고, 만약 소송에 걸리면 무조건 선고유예를 받아야 한다. 집행유예는 의사면회취소 사유가 됨}
\normalsize
