\section{Bacterial vaginosis}
\myde{}{%
\begin{itemize}\tightlist
\item[\dsjuridical] N760 급성질염
\item[\dschemical] B0042 요침사현미경검사(wetSmear(DirectSmear)) [\myexplfn{9.14} 원]
\item[\dschemical] C5956006 가드네렐라균 중합효소반응 [\myexplfn{415.49} 원]
\item[\dsmedical] 바지씨질정 Ascorbic Acid 250mg (일반약)
\end{itemize}
}{
Dx Amsel criteria (3/4)
\begin{itemize}[-]\tightlist
\item Homogeneous, thin, grayish-white discharge that smoothly coats the vaginal walls
\item Vaginal pH greater than 4.5
\item Positive whiff-amine test : Several drops of a potassium hydroxide (KOH) solution are added to a sample of vaginal discharge to see whether a strong fishy odor is produced. A fishy odor on the whiff test usually means bacterial vaginosis is present. defined as the presence of a fishy odor
\item Clue cells on saline wet mount
\end{itemize}
}
\subsection{Bacterial Vaginosis at pregnancy}
\begin{itemize}\tightlist
\item Treatment : pregnant women
	\begin{itemize}[-]\tightlist
	\item oral tx : no fetal or obstetrical effects1
	\item metronidazole 500mg bid for 7days
	\item metronidazole 250mg tid for 7 days
	\item clindamycin 300mg bid for 7 days
	\item topical tx : 추천되지 않음.2
	\end{itemize}
\item Treatment :  Asymptomatic in pregnancy
	\begin{itemize}[-]\tightlist
	\item chorioamnionitis를 유발하여 preterm delivery risk 증가
	\item 그러나 증상이 없을 때 screening은 권장하지 않음.
	\item 이전에 preterm delivery 한 경우는 screening 
	\end{itemize}
\end{itemize}

\subsection{질염관련 검사류:Microscopy}
\tabulinesep =_2mm^2mm
\begin {tabu} to\linewidth {|X[2,l]|X[5.5,l]|X[2,l]|} \tabucline[.5pt]{-}
\rowcolor{ForestGreen!40} \centering 검사코드(가격) & \centering 검사명 &	\centering 적합상병 \\ \tabucline[.5pt]{-}
\rowcolor{Yellow!40} B4107(\myexplfn{75.96} 원) & 미생물현미경검사(일반염색)-피부 진균 KOH도말 현미경검사 & B373(캔디다질염)  \\ \tabucline[.5pt]{-}
\rowcolor{Yellow!40} B4101(\myexplfn{20.76} 원) & 미생물현미경검사(일반염색. Gram stain) & N76(질염)  \\ \tabucline[.5pt]{-}
\rowcolor{Yellow!40} B4106(\myexplfn{18.37} 원) & 미생물현미경검사(기타일반염색) & N76(질염)  \\ \tabucline[.5pt]{-}
\rowcolor{Yellow!40} B0042(\myexplfn{9.14} 원) & 요침사현미경검사(WetSmear(DirectSmear)) & N300(급성방광염)/N76 \\ \tabucline[.5pt]{-}
\end{tabu}
\par
\medskip
\tabulinesep =_2mm^2mm
\begin {tabu} to .8\linewidth {|X[1,l]|X[4,l]|} \tabucline[.5pt]{-}
\rowcolor{ForestGreen!40} \centering 정 보 & \centering Rationale \\ \tabucline[.5pt]{-}
\rowcolor{Yellow!40} Bacteria & So what ? \\ \tabucline[.5pt]{-}
\rowcolor{Yellow!40} WBC & 염증의 중등도 판단 \\ \tabucline[.5pt]{-}
\rowcolor{Yellow!40} Yeast & Candida vaginalis   \\ \tabucline[.5pt]{-}
\rowcolor{Yellow!40} Trichomonas & Trichomonas vaginalis  \\ \tabucline[.5pt]{-}
\rowcolor{Yellow!40} Epithelial cell & So what ?  \\ \tabucline[.5pt]{-}
\rowcolor{Yellow!40} G+ rod & Lactobacillus  \\ \tabucline[.5pt]{-}
\rowcolor{Yellow!40} G- rod & So what ?  \\ \tabucline[.5pt]{-}
\rowcolor{Yellow!40} G- cocci & Neisseria gonorrhea[Diplococci] \\ \tabucline[.5pt]{-}
\end{tabu}
