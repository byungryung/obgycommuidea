\section{Hypertension}
갱년기 관리 환자들이 매번 혈압을 재고 있는데 높은 경우가 많습니다. 활용할 방법은 없을까요?
\myde{}{%
\begin{itemize}\tightlist
\item[\dsjuridical] I109 기타 및 상세불명의 원발성 고혈압
\item[\dsjuridical] E785 상세불명의 고지질혈증
\end{itemize}
\tabulinesep =_2mm^2mm
\begin{tabu} to \linewidth {|X[1,l]|X[3,l]|} \tabucline[.5pt]{-}
\rowcolor{ForestGreen!40} \centering 코드 & \centering 검사명 \\ \tabucline[.5pt]{-}
\rowcolor{Yellow!40} D2253 & 요일반검사10종까지  \\ \tabucline[.5pt]{-}
\rowcolor{Yellow!40} D2202 & 요침사검사[유세포분석법] \\ \tabucline[.5pt]{-}
\rowcolor{Yellow!40} D0002050 & 혈색소(광전비색) \\ \tabucline[.5pt]{-}
\rowcolor{Yellow!40} D0002040 & 헤마토크리트 \\ \tabucline[.5pt]{-}
\rowcolor{Yellow!40} D0002030 & 적혈구수 \\ \tabucline[.5pt]{-}
\rowcolor{Yellow!40} D0002010 & 백혈구수 \\ \tabucline[.5pt]{-}
\rowcolor{Yellow!40} D0002070 & 혈소판수 \\ \tabucline[.5pt]{-}
\rowcolor{Yellow!40} D0013 & 백혈구 백분율(혈액) \\ \tabucline[.5pt]{-}
\rowcolor{Yellow!40} D1860 & sGOT  \\ \tabucline[.5pt]{-}
\rowcolor{Yellow!40} D1850 & ALT \\ \tabucline[.5pt]{-}
\rowcolor{Yellow!40} D2263 & 지질(트리그리세라이드)  \\ \tabucline[.5pt]{-}
\rowcolor{Yellow!40} D2613 & HDL콜레스테롤  \\ \tabucline[.5pt]{-}
\rowcolor{Yellow!40} D2611 & 총콜레스테롤정량 \\ \tabucline[.5pt]{-}
\rowcolor{Yellow!40} D2300003 & 요소질소(NPN포함) \\ \tabucline[.5pt]{-}
\rowcolor{Yellow!40} D2280003 & 크레아티닌 \\ \tabucline[.5pt]{-}
\rowcolor{Yellow!40} D1840003 & 총단백정량 \\ \tabucline[.5pt]{-}
\rowcolor{Yellow!40} D2510053 & LDH \\ \tabucline[.5pt]{-}
\rowcolor{Yellow!40} D1880003 & 알부민 \\ \tabucline[.5pt]{-}
\rowcolor{Yellow!40} D1830003 & 빌리루빈 정량(총빌리루빈) \\ \tabucline[.5pt]{-}
\rowcolor{Yellow!40} D1870003 & 알칼리포스파타제 \\ \tabucline[.5pt]{-}
\end{tabu}
}
{
\begin{itemize}\tightlist
\item 50대 이상 여성의 30\% 이사이 고혈압입니다. 
\item HRT환자군에게 사용하시면 자연스럽게 유입된니다. 
\item 총진료비 50680원 본인부담금 15200원입니다.
\end{itemize}
}
