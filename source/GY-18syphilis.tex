\section{매독}
\myde{}{
\begin{itemize}\tightlist
\item[\dsjuridical] A51 조기매독
\item[\dsjuridical] A52 만기매독 
\item[\dsjuridical] A50 선천매독
\item[\dsjuridical] A530 조기인지 만기인지 상세불명의 잠복성 매독
\item[\dschemical] C4600 RPR(정성) 매독 반응검사 [VDRL, RPR, ART] Syphilis Reagin Test \myexplfn{18.35} 원
\item[\dschemical] C4601 RPR(정량) 역가검사를 실시한 경우에는 33.05점을 산정한다. \myexplfn{33.05} 원
\item[\dschemical] C4611 TPHA \myexplfn{166.74} 원
\item[\dschemical] C4620 FTA-ABS IgG \& IgM \myexplfn{144.00} 원
\item[\dschemical] C4621 FTA-ABS IgG \& IgM(역가검사)  \myexplfn{254.76} 원
\item[\dschemical] Treponema pallidum PCR 
\end{itemize}
}
{
매독은 직접 균체를 검출하기 어렵고 배양할 수 없으므로 매독의 진단과 치유판정은 대부분 혈청학적 검사에 의해 이루어 집니다. 매독에 대한 혈청학적 검사는 Nontreponemal test와 Treponemal test로 나눕니다.The nontreponemal tests are able to detect non-specific treponemal antibodies. There are two common tests under the nontreponemal test. They include \textcolor{red}{VDLR (Venereal Disease Research Laboratory) and RPR (Rapid Plasma Reagin)}\\
\uline{Nontreponemal test로는 RPR이 있으며 감염에 대한 첫 번째 선별검사로 사용됩니다. 또한 RPR 정량검사를 통해 치료 전 과 후의 역가를 측정하면 치료에 대한 반응을 확인할 수 있어 환자의 모니터링이 가능합니다. Treponemal test는 매독균을 항원으로 사용하여 매독균의 세포성분에 대한 항체를 검출하는 것으로 TPLA와 FTA-ABS가 있으며, 매독을 확진하고 RPR 검사의 위양성을 감별할 수 있습니다.}
}

\begin{table}
\tabulinesep =_2mm^2mm
\begin {tabu} to\linewidth {|X[3,c]|X[7,l]|X[1,l]|X[3,l]|} \tabucline[.5pt]{-}
\rowcolor{ForestGreen!40} \centering 처방코드 & \centering 처방명칭 &	\centering 수량 & \centering 용법 \\ \tabucline[.5pt]{-}
\rowcolor{Yellow!40} C4600 & 매독반응검사 & 1 &  \\ \tabucline[.5pt]{-}
\rowcolor{Yellow!40} C4601 & 매독반응검사(역가검사) & 1 &  \\ \tabucline[.5pt]{-}
\rowcolor{Yellow!40} C4611 & 매독감작혈구응집(역가검사) & 1 & 3,4중 택1 \\ \tabucline[.5pt]{-}
\rowcolor{Yellow!40} C4620006 & 형광트레포네마항체흡수검사 & 2 & 3,4중 택1 \\ \tabucline[.5pt]{-}
\end{tabu}
\caption{통상위의 1번검사를 시행 후 양성이 나오게 되면 2-4번 검사을 시행하게 됩니다.
3,4검사중 한가지만 보험이 되며 나머지는 비급여로 나가셔야 합니다.
3번은 TPHA, 4번은 FTA-ABS IgG\&IgM 이므로 수량에 2로 적어서 나가시면 됩니다.
}
\end{table}

\subsection{검사의 의미}
\hspace{-0.5cm}\includegraphics{VDRLDx}
\subsection{잠복매독}
잠복매독은 질환의 증거는 없이 혈청학적 활동성이 있는 경우로 정의되며 조기잠복매독은 다음과 같은 특성이 있습니다.
\begin{itemize}\tightlist
\item 혈청학적 정량검사에서 4배 이상 역가가 증가한 경우
\item 1기 혹은 2기 매독의 증상과 일치하지 않은 증상이 있는 경우
\item 1기, 2기, 조기잠복매독을 가진 성접촉자가 있는 경우
\item 12개월 내에 노출된 환자의 비트레포네마 혹은 트레포네마 검사에서 양성인 경우
\end{itemize}
보통 RPR 역가는 후기잠복매독 때보다는 조기잠복매독인 경우 높게 측정됩니다. 잠복매독환자는 주의 깊게 구강, 여성의 회음부, 항문 주위, 귀두표피 아래(포경수술 하지 않은 경우) 등의 점막표면을 검사해야 하며 모든 환자는 HIV 감염에 대해 검사를 시행해야 합니다.

\includegraphics[scale=.9]{algoVDRL}

\subsection{치료}
\emph{2011 KCDC guideline} 
\par
\medskip
\tabulinesep =_2mm^2mm
\begin {tabu} to\linewidth {|X[1,l]|X[1,c]|X[1,c]|} \tabucline[.5pt]{-}
\rowcolor{ForestGreen!40}  병 기 & 권장요법 & 대체요법 \\ \tabucline[.5pt]{-}
\rowcolor{Yellow!40} 조기매독 \newline (1기, 2기 조기잠복매독) & Benzathine penicilin G 240만 IU IM & Doxycycline 100mg 2T \# 2 or 200mg PO X 14 일 \newline Erythromycin 500mg 4T \# 4 X 14 일 \newline Azithromycin 2g PO \\ \tabucline[.5pt]{-}
\rowcolor{Yellow!40} 후기잠복매독,\newline 지속기간을 모르는 잠복매독, 심혈관매독 & Benzathine penicilin G 240만 IU IM ㅌ 3 일 & Doxycycline 100mg 2T \# 2 or 200mg PO X 28 일 \newline Erythromycin 500mg 4T \# 4 X 28 일  \\ \tabucline[.5pt]{-}
\rowcolor{Yellow!40} 신경매독 & Pencillin G potassium cyrstal 300-400만 IU 정맥주사 4시간 간격으로 18-21일 요법 (1일 투여량 1800-2400만 IU) & 페니실린 탈감작 후 페니실린 투여를 우선 고려 \newline Ceftriaxone 2g IM X 14 days \\ \tabucline[.5pt]{-}
\end{tabu}
\par
\medskip
\paragraph{매독 치료후 추적검사}
매독 치료를 시작할 경우는 치료 전 RPR 역가 검사를 시행하여 환자의 최초 역가를 확인하고 치료 후 추적 검사
를 통한 역가의 감소 추이를 보고 치료경과를 추적관찰합니다. \textcolor{red}{성공적인 치료의 기준은 RPR 정밀 검사 결과가
음성으로 전환되거나 RPR 정량검사의 역가가 4배 이상 감소한 경우입니다.} 만약 RPR이 음성인 1기 매독을 치료
할 경우에는 FTA-ABS IgM 검사를 통해 치료 반응을 추적할 수 있습니다.
