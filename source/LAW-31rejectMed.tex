\section{처방 의약품 반납 가능여부}
\Que{환자가 처방된 의약품을 복용하다가 위장장애가 심하다며 1일분만 복용 후 나머지 약제를 가지고 와 반납을 해달라고 합니다. 이와 같이 부작용으로 처방약제 복용이 어렵다면 약제 반납이 가능한가요?}
\Ans{가능하지 않습니다. 일단 조제\cntrdot{}투약된 의약품은 타 환자에게 재사용 할 수 없으며 여타의 오염에 의하여 심각한 부작용을 야기할 수 있으므로 반납이 불가능 합니다.}

\begin{commentbox}{처방 의약품 반납관련 처리방법}
\begin{enumerate}[1.]\tightlist
\item 의약분업의 시행 이전 의료기관에서는 환자에게 투약한 의약품에 대하여 부작용의 발현, 복용불편 등의 사유로 환자의 요구가 있는 경우에 잔여 의약품을 반납 처리하는 것이 일반화되어 있음
\item 그러나, 의약품은 그 특성상 보관 및 관리가 엄격하여야 하며, 여타의 오염에 의하여 심각한 부작용을 야기할 수 있으므로 일단 조제\cntrdot{}투약된 의약품을 반납 받아 다른 환자에게 재사용하는 것은 사실상 불가능하며, 의약품의 경우 정상적인 처방 및 조제\cntrdot{}투약이라 하더라도 필연적으로 부작용이 발현될 수 있으므로 부작용이 발현되었다하여 잔여 의약품을 반납처리하는 것은 정상적인 진료 및 투약 등을 저해하게 됨
\item 따라서, 요양기관에서는 여타의 이유로 환자가 복용중인 의약품을 반납받아 다른 환자에게 재사용하거나, 이를 보험으로 정산처리해서는 아니됨 고시 제2000-73호 (2000.12.30. 시행)
\end{enumerate}
\end{commentbox}
