\subsection{현지조사 후 처분 }
\includegraphics[width=\textwidth]{postsilsa}

허위 청구 4중 처분  
\begin{enumerate}[1)]\tightlist
\item 의료법에 근거 행정처분(면허정지)
\item 국민건강보험법에 의거 환수,업무정지,과징금
\item 의료법, 형법에 의해 형사 처벌
\item 허위 청구로 인한 면허정지는 의료업을 할 수 없다 (의료법 66조3항)
\item 하나의 행위에 대해 처벌은 사실상 4번 받음 – 헌법상의 평등권, 형평성, 비례의 원칙, 과잉금지의 원칙에 사실상 위배
\end{enumerate}

부당금액 정산 및 행정처분 산출\par 
조사대상기간 6개월 (진료비1억) → 총 부당금액 300만원 → 부당비율 3\% → 월평균부당금액 50만원 → 업무정지 일수는?\par
\includegraphics[width=\textwidth]{postsilsa2}\par

업무정지 가중 처분
\begin{enumerate}[1)]\tightlist
\item 5년 이내에 업무정지(과징금처분)을 받은 사실이 있는 경우
\item 당해 업무정지기간 또는 과징금의 2배에 해당하는 처분 부과 (업무정지1년, 과징금 5배 초과 불가)
\item 5년 기산 방법 : 행정처분통보문서 송달일자로 부터 부당청구가 다시 확인 된 날(현지조사 확인서 징구일)
\end{enumerate}

\begin{commentbox}{면허자격정지 개념}
면허자격 정지
\begin{itemize}\tightlist
\item 부정한 방법으로 진료비 또는 약제비를 거짓청구 → 1년 범위 내에서 면허자격정지
\item 의료법 제66조(자격 정지 등)제1항 제7호
\item  약사법 제79조(약사·한약사 면허의 취소 등)제2항
\end{itemize}
의료업 정지
\begin{itemize}\tightlist
\item 자격정지 기간 중 의료업 불가
\item 의료법 제66조(자격정지 등)제3항 → 관련 서류를 위조,변조하거나 속임수 등 부정한 방법으로 진료비를 거짓청구 한 때(자격정지)
\end{itemize}
\end{commentbox}
\includegraphics[width=\textwidth]{postsilsa3}\par

\paragraph{현지실사 후 처분선택 }\par
업무정지  VS  과징금 최대 5배 선택시 고려점\par
(1억 낼 것이냐?  VS  5개월 쉴것이냐?)\par
→  최종판결 전 업무정지가 시행된다\par
개선방향 – 면허정지, 업무정지 같은 처벌은 최종심 판결 이후 시행해야

\paragraph{행정처분 절차}
\begin{enumerate}\tightlist
\item 처분사전통지
	\begin{itemize}\tightlist
	\item 처분원인이 되는 사실
	\item 처분 내용 및 근거
	\item 의견 제출 방법
	\end{itemize}
\item 의견청취
	\begin{itemize}\tightlist
	\item 우편 또는 보건복지부를 방문하여 의견제출
	\end{itemize}
\item 의견검토
	\begin{itemize}\tightlist
	\item 제출된 의견에 대해 심평원에서 검토하여 복지부에 보고
	\end{itemize}
\item 행정처분
	\begin{itemize}\tightlist
	\item 업무정지처분 또는 과징금 처분
	\end{itemize}	
\item 관련근거: 행정절차법 제21조제1항, 제22조제3항	
\end{enumerate}
	
\paragraph{형사고발 기준}
\begin{enumerate}\tightlist
\item 업무정지기간 중 요양급여(건보법 제115조)
	\begin{itemize}\tightlist
	\item 1년 이하 징역 및 1천 만원 이하 벌금
	\end{itemize}
\item서류 미제출, 거짓보고, 거짓서류제출, 조사거부ㆍ방해ㆍ기피할 경우(건보법 제 116조)
	\begin{itemize}\tightlist
	\item 1천 만원 이하 벌금
	\end{itemize}
\item요양급여비용 거짓 청구는 형법상 사기죄로 고발
	\begin{itemize}\tightlist
	\item 10년 이하 징역 또는 2천 만원 이하 벌금 (형법 제347조)
	\end{itemize}
\item ※ 고발(내부기준) 거짓청구금액 750만원 또는 거짓청구비율 10\% 이상인 기관
\end{enumerate}	
\begin{commentbox}{거짓청구기관 명단공표(법 제 100조)}
행정처분을 받은 요양기관 중 관련 서류를 위·변조하여 요양급여비용을 거짓으로 청구한 요양기관은 위반행위, 처분내용, 명칭, 주소, 대표자의 성명 등을 공표 \par
건강보험공표심의위원회 구성 : 위원장 1인 포함 9명, 임기 2년(연임 가능)\par
\begin{enumerate}\tightlist
\item <공표대상> - 거짓청구금액 1,500만원 이상 - 거짓청구비율 100분의 20이상인 경우
\item <공표방법> - 보건복지부, 심평원, 공단, 시·군·구 등의 홈페이지에 공표 - 거짓청구가 중대한 위반에 해당하는 경우 신문 또는 방송에 추가 공표
\item <공표절차> - 건강보험 공표 심의위원회에서 공표 대상 선정 → 대상자에게 의견진술 기회부여 → 소명자료 검토 후 최종 대상확정 → 인터넷ㆍ언론 등에 공표	
\end{enumerate}
\end{commentbox}

\subsection{현지실사문제점 }
\begin{enumerate}[1)]\tightlist
\item 실적, 처벌 목적이 아니라 운용 취지에 맞게 올바른 청구문화의 정착위해 운용해야 
\item 행정조사기본법 미준수  
    최소한의 조사원칙,조사권 남용금지 (4조1항)
    공동조사원칙,중복조사금지(4조3항, 15조)
    처벌보다 계도원칙(4조4항)
    조사목적, 조사범위와 내용 명확화의 원칙(11조)
    7일전 사전통지원칙(17조)
\end{enumerate}


