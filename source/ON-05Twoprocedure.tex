\section{두가지 이상의 수술시 수기료 산정방법}
\tabulinesep =_2mm^2mm
\begin {tabu} to\linewidth {|X[2,l]|X[2,l]|X[1,l]|} \tabucline[.5pt]{-}
\rowcolor{ForestGreen!40} \centering 상황1 & \centering 상황2 &	\centering 청구방법 \\ \tabucline[.5pt]{-}
\rowcolor{Yellow!40} 자궁경부소작술(R4310) & 질강처치(R4106) & 둘중 하나만 청구 \\ \tabucline[.5pt]{-}
\rowcolor{Yellow!40} 루프제거술[R4275 or R4277] & 질강처치(R4106) & 둘중 하나만 \\ \tabucline[.5pt]{-}
\rowcolor{Yellow!40} STD Multiplex (RT) PCR 6종 & KOH검사[B4107],그람염색[B4101] & 두개다 청구 \\ \tabucline[.5pt]{-}
\rowcolor{Yellow!40} STD Multiplex (RT) PCR 6종 & Single PCR[C5956006] & 두개다 청구 \\ \tabucline[.5pt]{-}
\rowcolor{Yellow!40} 자궁경부착공생검[C8576] & 자궁경부소작술[R4310] & 두개다 청구 \\ \tabucline[.5pt]{-}
\rowcolor{Yellow!40} 자궁근종절제술-질부접근[R4123] & 자궁질부전기소작술[R4310] & 한개 50\% \\ \tabucline[.5pt]{-}
\rowcolor{Yellow!40} 자궁경관점막폴립절제술[R4240] & 자궁질부전기소작술[R4310] & 한개 50\% \\ \tabucline[.5pt]{-}
\rowcolor{Yellow!40} 요실금수술[R3565] & 방광류교정술[R3630] or 질벽봉합술[R0410] & 한개 50\% 3개 수술 다할시 두개 50\% \\ \tabucline[.5pt]{-}
\end{tabu}

\medskip
\begin{commentbox}{여러 가지 시술을 함께 하는경우의 보험적용에 대해 문의합니다}
\begin{enumerate}\tightlist
\item 자궁경부 상피내 종양 및 침윤암 의증시 자궁경부 착공 생검(C8576 ,펀치 생검)을 시행합니다. 착공 생검에 의한 출혈의 지혈위해 자궁경부 전기소작술 (R4310)시행하는 경우 두 개다 보험적용가능한가요? 주시술과 보조시술로 분류하여 하나를 50\% 청구해야하는건가요? 둘중 하나만 된다면, 하나는 비급여로 환자에게 청구할수있나요?
\item 자궁경부 외번(eversion), 만성 자궁경부염인 경우 자궁경부 전기소작술 (R4310)을 시행하고 자궁경부의 염증성 질환 치료를 위해 질강처치(R4106)를 한 경우 두 개다 보험적용가능한가요? 주시술과 보조시술로 분류하여 하나를 50\% 청구해야하는건가요? 둘중 하나만 된다면, 하나는 비급여로 환자에게 청구할수있나요?
\item 질벽에 이물질이 있는 경우 질이물 제거(R4105) 시행후 자궁경부의 염증성 질환 치료를 위해 질강처치(R4106)를 한 경우 두 개다 보험적용가능한가요? 주시술과 보조시술로 분류하여 하나를 50\% 청구해야하는건가요? 둘중 하나만 된다면, 하나는 비급여로 환자에게 청구할수있나요? 
\item 질벽탈출증, 방광류등이 있는 경우 질탈 교정술 (페서리 착용)(R4113) 시행후 자궁경부의 염증성 질환 치료를 위해 질강처치(R4106)를 한 경우 두 개다 보험적용가능한가요? 주시술과 보조시술로 분류하여 하나를 50\% 청구해야하는건가요? 둘중 하나만 된다면, 하나는 비급여로 환자에게 청구할수있나요
\end{enumerate}
%}
%{
요양기관에서 이루어진 수술 사례의 수가산정방법에 대하여는 현행 「건강보험행위 급여ㆍ비급여 목록표 및 급여 상대가치점수」제1편 제2부 제9장 처치 및 수술료 [산정지침](6)에 의거 '\textcolor{red}{동일 피부 절개 하에 2가지 이상 수술을 동시에 시술한 경우 주된 수술은 소정점수에 의하여 산정하고, 제2의 수술부터는 해당 수술 소정점수의 50\%(산정코드 세 번째 자리에 1로 기재)}, 상급종합병원\cntrdot{} 종합병원은 해당 수술 소정점수의 70\%(산정코드 세 번째 자리에 4로 기재)를 산정한다. 다만, 주된 수술 시에 부수적으로 동시에 실시하는 수술의 경우에는 주된 수술의 소정점수만 산정한다'고 명시하고 있습니다.\par 또한, 질강처치료(고시 제2013-36호(행위),'13.2.28)는 외음과 질의 칸디다증, 비뇨생식기의 편모충증, 자궁경부의 염증성 질환, 자궁경부의 미란 및 외반 상병에 실시한 경우에 치료기간 중 1회 인정한다고 명시되어 있습니다.
\par 따라서 상기 규정을 참고하여 시술과정에 대한 실제 기록에 근거하여 청구하시기 바라며, 환자의 상태 등 전체 진료내역을 고려한 심사가 이루어져야 정확한 판단이 가능한 바, 질의에 대한 상세 답변이 곤란한 점 이해있으시기 바랍니다. 
\end{commentbox}