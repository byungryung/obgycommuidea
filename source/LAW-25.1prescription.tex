\section{처방전}
\begin{shaded}
\begin{itemize}\tightlist
\item 자신이 직접 의약품을 조제할 수 있는 경우가 아니면 처방전을 작성하여 환자에게 내주거나 발송 (전자처방전만 해당된다)
\item 정당한 사유 없이 전자처방전에 저장된 개인정보를 탐지하거나 누출\cntrdot{}변조 또는 훼손 금지
\item 처방전을 발행한 의사는 처방전에 따라 의약품을 조제하는 약사가 문의한 때 즉시 응대
\item 사유가 종료된 때 응대할 수 있는 경우
	\begin{enumerate}\tightlist
	\item 응급환자를 진료 중인 경우
	\item 환자를 수술 또는 처치 중인 경우
	\item 그 밖에 약사의 문의에 응할 수 없는 정당한 사유가 있는 경
	\end{enumerate}
\end{itemize}
\end{shaded}
\subsection{처방전에 기재사항들}
\begin{enumerate}\tightlist
\item 환자의 성명 및 주민등록번호
\item 의료기관의 명칭, 전화번호 및 팩스번호
\item 한국표준질병\cntrdot{}사인 분류에 따른 질병분류기호; 환자가 요구한 경우에는 적지 않아도 됨
\item 의료인의 성명\cntrdot{}면허종류 및 번호
\item 처방 의약품의 명칭\cntrdot{}분량\cntrdot{}용법 및 용량
\item 처방전 발급 연월일 및 사용기간
\item 의약품 조제시 참고 사항
\end{enumerate}

\subsection{처방전에 대해 알아야할 내용들}
\begin{enumerate}\tightlist
\item 처방전 2부 발급의무 
\item 환자가 그 처방전을 추가로 발급하여 줄 것을 요구하는 경우에는 환자가 원하는 약국으로 팩스\cntrdot{}컴퓨터통신 등을 이용하여 송부
\item 환자를 치료하기 위하여 필요하다고 인정되면 다음 내원일에 사용할 의약품에 대하여 미리 처방전
\end{enumerate}