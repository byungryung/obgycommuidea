\section{요양기관 이전 시 재개설 여부}
\Que{요양기관의 장소 이전으로 서울시 강남구에서 서울시 종로구로 소재지를 변경하려고 합니다. 이 경우 보건소와 심사평가원에 폐업 후 재개설신고를 해야 하는지요?}
\Ans{요양기관 소재지를 이전하면서 관할 행정구역이 변경되는 경우 (예: 서울시 강남구 → 서울시 종로구) 요양기관 폐업 후 재개설해야 합니다.
폐업 및 개설은 보건소에 신고해야 하며, 보건소의료자원 신고일원화 관련하여 보건소 처리 내용은 심사평가원에 통보됩니다. 보건소 폐업 신고 시, 폐업 당일은 원칙적으로 진료가 불가하므로 최종 진료일의 익일을 폐업일로 보건소에 신고해야 함을 유의해야 합니다.
(예: 최종진료일이 6월 15일 경우 폐업일자는 6월 16일로 보건소에 신고해야 함)
폐업 후 재개설을 하면 시설, 인력, 장비를 새로 신고해야 합니다. 이 경우, 폐업 전에 “보건의료자원통합신고포털”에 접속하여 기존에 등록되어 있던 데이터를 조회해 보거나, 데이터를 엑셀로 다운받아 참고하면 편리하게 신고할 수 있습니다.
다만, 요양기관 소재지를 이전하면서 관할 행정구역이 변경되지 않는 경우 (예: 서울시 강남구 → 서울시 강남구) 재개설이 아닌 보건소에 요양기관 소재지 변경 신고만 하면 됩니다.(심사평가원에 통보됨)}

