\begin{mdframed}[linecolor=blue,middlelinewidth=2]  
제1편 행위 급여 \cntrdot{}  비급여 목록 및 급여 상대가치점수 >> 제2부 행위 급여 목록\cntrdot{} 상대가치점수 및 산정지침 >> 제5장 주사료
\end{mdframed}

\subsection{\newindex{주사료〔산정지침〕}}
\begin{enumerate}[(1)]\tightlist
\item \uline{주사시 사용된 주사재료대(1회용 주사기, 1회용 주사침, 나비침, 정맥내유치침, 수액세트, 혈액Bag 등)와 수혈에 소요된 약제 및 재료대는 소정점수에 포함되므로 별도 산정하지 아니한다. 다만, 정맥내유치침을 사용한 경우에는 「마-5-주1」및「마-15-다-주1」에 따라 산정하며,} 다음의 경우에는 “약제 및 치료재료의 비용에 대한 결정기준”에 의하여 별도 산정한다.
	\begin{enumerate}[(가)]\tightlist
	\item 치료적 성분채집술에 사용된 약제 및 재료대(요양기관이 대한적십자사 혈액원 등으로부터 성분채집에 의한 혈액성분제제를 구입한 경우 포함)
	\item 조혈모세포이식 시 사용된 골수, 말초혈액, CD34+ Collection Kit,Cryo Bag
	\item 적혈구수집기(Cell Salvage)를 이용한 자가수혈에 사용된 재료대
	\end{enumerate}
\item 제1절 주사료를 산정하는 경우 만8세 미만의 소아에 대하여 정맥내 점적 주사(마-5, 마-15-다)는 주사료 소정점수의 30\%를 가산하고, 기타 주사는 주사료 소정점수의 20\%를 가산한다.(산정코드 첫 번째 자리에 3으로 기재) 다만, 피하 또는 근육내주사(마-1), 생물학적제제주사(마-4), 수액제 주입로를 통한 주사(마-5-1), 항암제 피하내주사(마-15-가), 급속항온주입(마-16)은 그러하지 아니한다.
\end{enumerate}
\uline{정맥내유치침 KK059 마5주1 (400원)} \par
\uline{항암제정밀지속적점적주입위한InfusionPump사용료[기기당1일1회] 마15다주2 KK158(2010원)}\par