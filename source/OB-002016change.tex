\section{2016년 질식분만숫가 변화}
%\subsection{변화전의 숫가체계}
%\includegraphics{Flabor}
%\clearpage
\subsection{질식분만의 새로운 high Risk}
\myde{}{%
\begin{enumerate}[가.]\tightlist
\item Z355, Z358 출산 당시 나이가 만 35세 이상인 산모
\item E669, O260 임신 제1 삼분기 당시 BMI가 27.5 kg/㎡ 이상인 산모
\item O3419 임신 중 5㎝ 이상의 자궁근종 또는 [O3409] 자궁기형을 가진 산모
\item O6011 임신 34주 미만의 조산
\item O12 전자간증, 자간증 또는 가중합병전자간증 
\item O440 전치태반 또는 [O459] 태반 조기 박리
\item O40 양수과다증 또는 [O410] 양수과소증
\item 심혈관계 질환, 신장 질환, 당뇨병 [O249], 혈액응고장애, 백혈병, 매독 또는 HIV 양성 중 어느 하나 이상에 속하면서 분만에 직접적인 위험을 줄 수 있는 질환을, 임신 전 또는 임신 기간 중 진단 받고 지속 치료중인 산모
\item o359 출산과정에 영향을 미치거나, 분만 중 태아 또는 신생아의 생존 능력에 영향을 미치는 태아 기형   
\item (O366,P081)	임신기간에 비해 기타 체중과다인 영아] 출생당시 체중이 4kg 이상 또는 [O365,P071 기타 저체중출산아] 2.5Kg 미만의 신생아
\item O430 쌍태간 수혈 증후군 
\end{enumerate}
}%
{보건복지부 공고 제 2016 - 627 호 \par
고위험분만 인정기준 \par
다음의 요건 중 1개 이상을 충족한 경우에 고위험 분만에 해당되는 것으로 함 \par
- 다  음 - \par
고위험 분만에 해당되는 경우에는 \highlight{소정점수의 30\%를 추가 가산한다.(산정코드 첫 번째 자리에 S로 기재) 다만, ‘주2’에 의하여 가산을 적용받는 경우에는 그러하지 아니한다.}. 자-435, 자-436에만 해당된다. 
}
\par
\medskip
\prezi{\clearpage}
\includegraphics[scale=.75]{VDhighrisk}
\prezi{\clearpage}
\subsection{분만취약지 선정}
분만취약지 적용기준 \par
건강보험 행위 급여\bullet 비급여 목록표 및 급여 상대가치점수에 의한 분만취약지는 잠재적 분만취약지를 포함하여 아래와 같으며, 해당 취약지역 소재 요양기관(조산원 포함)에서 분만이 이루어진 경우에는 분만취약지 분만 가산 수가를 적용함. \par
분만취약지 소재 요양기관에서 분만(자-435, 자-436,자-438, 자-450, 자-451)한 경우 소정점수의 200\%를 가산한다.(산정코드 첫 번째 자리에 R로 기재) 다만, 고위험 분만과 가산이 동시 적용되는 경우에는 산정코드 첫 번째 자리에 T로 기재한다. 

\begin{commentbox}{잠재적 분만취약지를 포함한 분만취약지}
\begin{description}\tightlist
\item[인천] 강화군, 옹진군
\item[경기] 가평군, 양평군, 여주시, 연천군
\item[강원] 고성군, 삼척시, 양구군, 양양군, 영월군, 인제군,정선군, 철원군, 태백시, 평창군, 홍천군, 화천군,횡성군
\item[충북] 괴산군, 단양군, 보은군, 영동군, 옥천군, 음성군,제천시, 증평군, 진천군, 
\item[충남] 계룡시, 공주시, 금산군, 논산시, 보령시, 부여군,서천군, 예산군, 청양군, 태안군, 홍성군
\item[전북] 고창군, 김제시, 남원시, 무주군, 부안군, 순창군, 완주군, 임실군, 장수군, 정읍시, 진안군
\item[전남] 강진군, 고흥군, 곡성군, 구례군, 나주시, 담양군,무안군, 보성군, 신안군, 영광군, 영암군, 완도군,장성군, 장흥군, 진도군, 함평군, 해남군, 화순군
\item[경북] 고령군, 군위군, 김천시, 문경시, 봉화군, 상주시,성주군, 영덕군, 영양군, 영주시, 영천시, 예천군,울릉군, 울진군, 의성군, 청도군, 청송군
\item[경남] 거창군, 고성군, 남해군, 밀양시, 사천시, 산청군,의령군, 창녕군, 하동군, 함안군, 함양군, 합천군
\end{description}
\end{commentbox}
\prezi{\clearpage}
\subsection{심야가산}
공휴일시의 분만시 50\% 가산입니다.  \highlight{청구코드두번째자리에 50}\par
18:00 - 09:00 시의 분만(제왕절개와 질식분만)시 50\% 가산입니다. \highlight{청구코드두번째자리에 10}\par
22:00 - 06:00 시의 분만(제왕절개와 질식분만)시 100\% 가산입니다.\highlight{청구코드두번째자리에 60} %분만취약지에서 22시-06시에 제왕절개분만을 행한 경우에는 야간․공휴가산 소정점수를 2회 추가 산정한다.
다음날 수술하는 경우에는 원래대로 하고 있습니다. 
\prezi{\clearpage}
