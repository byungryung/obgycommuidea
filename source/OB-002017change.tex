\section{2017년 외래본인부담금 인하}
\subsection{임신부/ 조산아 및 저체중 출생아 외래본인 부담 인하}
2017년 1월 1일 부터 임신\cntrdot{}출산 환경 조성을 위한 임신부와 조산아의 외래 본인부담률을 10\%로 적용해야합니다. 
\begin{itemize}\tightlist
\item 임신부 : 임신 전 기간에 걸친 외래 본인부담률 10\% 적용
\item 조산아 및 저체중아 : 만3세까지 외래 본인부담률 10\% 적용 
\item 조산아 및 저체중 출생아가 외래진료를 받은 경우에 등록번호를 MT051에 기재 해야 됨
\item 진료실 프로그램에서 환자선택 후 환자정보특성에 임신부 및 조산아 체크를 하면 자동으로 해당 기간동안 본인부담율이 10\%로 적용됩니다 
\item \href{http://www.obgydoctor.co.kr/xe/index.php?document_srl=25045&mid=m_faq}{산부인과 희망제작소 home}
\end{itemize}
\prezi{\clearpage}
\begin{commentbox}{「의료급여수가의 기준 및 일반기준」개정}
\begin{enumerate}[1)]\tightlist
\item 의료급여법 시행령(대통령령 제27730호, ‘16.12.30일 공포, ’17.1.1일 시행)
\item 의료급여법 시행규칙(보건복지부령 제459호, ‘16.12.30일 공포, ’17.1.1일 시행 )
\item 의료급여수가의 기준 및 일반기준(고시 제2016-272호, ‘17.1.1일 시행)
\item 요양비의 의료급여기준 및 방법(고시 제2016-245호, ‘17.1.1일 시행)
\item 임신\cntrdot{}출산 진료비등의 의료급여기준 및 방법(고시 제2016-246호, ‘17.1.1일 시행)
\end{enumerate}

의료급여수가의 기준 및 일반기준 중 일부를 다음과 같이 개정한다.
\begin{description}\tightlist
\item[제4조]를 삭제한다.
\item[제5조제1항제1호] 중 “긴급수술을 요하는 경우”를 “분만 및 수술을 동반하는 경우”로 한다. 
\item[제5조제1항제2호] 를 삭제하고, 제3호부터 제5호까지를 각각 제2호부터 제4호까지로 한다.
\item[제17조의8]을 다음과 같이 신설한다.
\end{description}
\emph{제17조의8(임신부, 조산아 및 저체중 출생아에 대한 의료급여)}
\begin{enumerate}[①]\tightlist
\item 영 제13조제1항 별표 1 제2호 자목의 \textcolor{red}{“임신부”란 임신이 확인된 이후 임신이 유지되는 기간에 있는 사람(유산\cntrdot{}사산으로 인한 외래진료를 받는 사람을 포함한다)을 말하며}, 차목의 “만3세까지의 조산아(早産兒) 및 저체중 출생아”에 대한 의료급여의 대상과 기간은「요양급여의 적용기준 및 방법에 관한 세부사항」에서 정하는 바에 따른다.
\end{enumerate}
\end{commentbox}
\prezi{\clearpage}
\subsection{의료급여 산모의 특정내역 기입방법}
\begin{itemize}\tightlist
\item 의료급여 1종과 2종 산모는 입력할 필요 없습니다. 아니 입력하면 에러가 떨어집니다.
\item 의료급여 2종 산모 기입사항 : CT,MRI, PET와 관계있는 병의원은 코드기입시 
	\begin{itemize}\tightlist
	\item 특정기호 F015 외에
	\item 특정내역-본인부담 구분코드 B010
	\end{itemize}
\end{itemize}	

%\subsection{계류유산시 외래본인부담금}
%계류유산 확인되고 그 날 같이 수술하는 경우엔 본인부담금 10\%로 하고 있고, 삭감은 없는것 같습니다.\par
%다음날 수술하는 경우에는 원래대로 하고 있습니다. 
 
