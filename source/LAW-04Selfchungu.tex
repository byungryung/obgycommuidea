\section{의사가 본인을 진료하는 경우의 청구 방법}
[관련근거]
\begin{enumerate}[(1)]\tightlist
\item 보건복지부고시 제2007-139호 ( 2008.1.1 시행 ) 
[ 의사, 약사의 본인진료 및 조제시 요양급여비용 산정방법 ]
의사가 자신의 질병을 직접 진찰하거나 투약, 치료하는 등 본인 진료시에는 사용한 약제 및 치료재료만 실거래 가격으로 보상함. 또한, 약사 본인이 본인의 의약품을 조제한 경우에도 기술료를 제외한 의약품비만 실거래 가격으로 보상함.
\item 행정해석 급여 65720-10209 ( 2000.5.30 ) \par
~ 중 략 ~ \par
개원의 본인의 검사를 검사기관 (외부 검사기관:수탁기관)으로 위탁한 경우에는 수탁기관으로 검사료를 지불하여야 하므로 이 경우 검사에 소요된 비용은 의료보험 급여가 타당함.
(참고 : 의사 본인 진료시 위탁검사를 시행한 경우에는 검사기관(수탁기관)으로 지불한 검사료만 인정하고 위탁검사관리료(10\%)는 인정하지 아니함.)
\end{enumerate}