\section{질염관련 PCR검사류}
\myde{}{%
\begin{itemize}\tightlist
\item[\dsjuridical] N760-N768 질염 <<N761 아급성 및 만성질염>>
\item[\dsjuridical] N730-N739  골반염
\item[\dsjuridical] N710,N711,N719  자궁염
\item[\dsjuridical] N72 자궁경부의 염증성 질환
\item[\dsjuridical] A638 기타 명시된 주로 성행위로 전파되는 질환(유레아플라스마)
\item[\dsjuridical] A493 마이코플라즈마
\item[\dsjuridical] A542 임균
\item[\dsjuridical] A560 클라미디아
\item[\dsjuridical] A5900 편모충,트리코모나스
\item[\dschemical] \sout{C6014006 하부요로생식기 및 성매개 감염원인균(다중종합효소연쇄반응법) STD 6종, STD 12종}
\item[\dschemical] D6802016 누680나 핵산증폭-다종그룹2 하부요로생식기및성매개감염원인균[다중실시간중합효소연쇄반응법] STD7종Real-Time 76,640원
\item[\dschemical] \sout{C5896006 하부요로생식기 및 성매개 감염원인균(다중실시간 종합효소연쇄반응법) STD 6종(RT-PCR), STD 7종(RT-PCR)}
\item[\dschemical] D6802026 누680나 핵산증폭-다종그룹2 하부요로생식기및성매개감염병원체[다중중합효소연쇄반응법] STD12종multiplex 76,640원
\item[\dschemical] D591104C	핵산증폭-정성그룹1 Chlamydiatrachomatis 32,170	클라미디아 STD1종Single
\item[\dschemical] D591105C	핵산증폭-정성그룹1 Gardnerellavaginalis 32,170	가드네렐라 STD1종Single
\item[\dschemical] D591110C	핵산증폭-정성그룹1 Mycoplasmagenitalium 32,170	마이코.G STD1종Single
\item[\dschemical] D591111C	핵산증폭-정성그룹1 Mycoplasmahominis	32,170 STD1종Single
\item[\dschemical] D591112C	핵산증폭-정성그룹1 Neisseriagonorrhoeae	32,170 STD1종Single
\item[\dschemical] D591114C	핵산증폭-정성그룹1 Ureaplasmaurealyticum	32,170	유레아.U STD1종Single
\item[\dschemical] D642103C	핵산증폭-정성그룹1 Trichomonasvaginalis	32,170 STD1종Single
\item[\dschemical] D658104C	핵산증폭-정성그룹1 HerpesSimplexVirus(HSV)(Type1,2)	32,170 STD1종Single
\item[\dschemical] D693101C	핵산증폭-정성그룹1 Treponemapallidium	32,170 STD1종Single
\item[\dsjuridical] 칸디다 상병대신 (B373) 다른 상병 기입을 권유합니다 <<의견>>
\end{itemize}
청구메모>>
\begin{enumerate}\tightlist
\item 미생물 배양검사는 보험기준이 있어 이에 준한 상병과  심사참고란을  활용하면 됩니다. “CRP 상승 , WBC  상승 ,  복통 ,  발열 등 ” 골반염  증상 ( 있음 ), " 질 분비물이 현저히 증가하거나 악취가 나는 등 부인과적 감염이 의심되는 경우"
\item 염증과 균의  과거력 : "임질", "유레아플라스마","마이코플라스마","클라미디아","트리코모나스" 등
\item 기타 임상적으로 검사가  필요한경우 :  성매개감염질환  의심 ,  하부요로생식기 감염의심
\item 재발가능성있어   검사함 ( 손OO원장님 )
\item 치료후  재검확인 ( 장OO   원장님 )  치료 후 재발의심되어검사함
\item 검사하는   내용 기입 ( 김OO 원장님 )
\item 타 의원 반복치료후에도 재발되어 방문
\end{enumerate}
}
{\begin{enumerate}[1.]\tightlist
\item 요양기관들(2개 기관)은 하부요로생식기 및 성매개 감염 원인균검사가 '14.11.1. 신의료기술에서 급여 수가로 신설된 이후 검사료 및 수진자당 진료비가 급등한 기관으로 성매개 감염균 검사를 다빈도로 시행 하는 경향임.-----> 일률적인 검사보다는 선별검사를 권유합니다. 
\item 진료기록부 검토 결과, 진료 내용이 일률적으로 기재되어 있어 의학적 타당성 및 신빙성이 결여된다고 판단되고, 검사는 정도에 따라 순차적으로 시행함이 타당하므로 초진에 시행한 나589-1(가)하부 요로생식기 및 성매개 감염 원인균[다중 중합효소연쇄반응법]검사, 나595-5(라)하부 요로생식기 및 성매개 감염 원인균[다중 실시간 중합 효소연쇄반응법] 검사는 인정하지 아니함. ----> 진료기록부의 일률적인 기술보다는 다양한 폼의 상세 기술을 권유합니다 
\item 복통 및 열(37.9℃)이 있어 응급실에 내원한 24세의 환자로, CRP 및 WBC 상승, 하복부 압통 등의 소견으로 시행한 하부요로생식기 및 성매개감염원인균 검사 인정함--------->골반 감염(N730-N739) 상병 활용이 필요하고 골반감영관련된 증상위주로 (복통 및 열, CRP 및 WBC 상승, 하복부 압통 등)의 소견병력조사와 진료기록부 기입이 필요함 
\end{enumerate}
}

\Que{STD 6종 검사가 모든 검사 센터가 보험적용이 다 되나요?}\index{STD PCR QA!검사센타}
\Ans{아니요. 특정 검사센터만 보험적용되는 것이 아닌 식약처에서 요구하는 사항만 충족되면 multiplex 및 RT multiplex 검사가 모든 검사센터에서 가능합니다.\par
3가지 조건만 충족되면 됩니다. 하부 요로생식기 및 성매개 감염검사 보험 청구와 관련해서 식약처에서는 \emph{반드시 식약처 허가된 시약을 사용하여야 하며}, 그렇지 아니할 경우 보험산정이 되지 않으며, 허가되지 않는 시약을 진단 등의 목적으로 사용하였을 경우
 의료기기법 제 26조 (일반행위의 원칙)에 위배되어 의료기기법 제 52조 (벌칙)의 조항에 의거 징역 2년 또는 벌금 2000만원에 처해질 수 있습니다.  라고 합니다. STD 검사하는 산부인과에서는 식약처 허가된 제품임을 검사센터에서 요구하여 확인해야 합니다. 
	\begin{enumerate}\tightlist
	\item 식약처 허가된 제품 
	\item 6종을 다 포함할것 
	\item PCR or RTPCR 방법이어야 할것.
	\end{enumerate}
}

\Que{STD 6종과 더불어 가드네렐라 균을 보고 싶은데 가능할까요?}
\Ans{STD6종(C6014006 or C5956006)+가드네렐라균 (C5956006, 중합효소반응)으로 가능합니다. 충분한 charting과 위의 사항을 참고하세요.}
%\Que{요즘 청구 건수가 많아져서 약간 불안합니다.}
%\Ans{네 향후에는 진입장벽을 높일 것입니다. 하지만 그때그때 대책을 마련하면 됩니다. 	또한 그동안 검사하였던 것을 소급해서 뺏어갈까 걱정하시는 원장님들이 계시는데 절대 	그런일이 벌어지지 않습니다. 걱정안하셔도 됩니다.	과거 NST GDM 소급 사태와는 본질적으로 다른 사항입니다.}
\Que{STD6종 검사에서 양성이 있어 약 복용 후 재검 시 청구 가능 한가요?}
\Ans{재검에 대한 심사기준이나 지침은 없는 상태이나, 치료 후 추적관찰은 의학적 필요가 있다고 판단되어 지는 상황이므로 보험적용 할 수 있을 것으로 판단됩니다.}	
\Que{재검의 interval은 어느 정도가 적당한가요?}
\Ans{치료후 완치 판정을 위해 달을 바꾸어서 한번더 확인 해야 한다.}
\Que{STD6종 검사에서 일부의 균만 양성이 나왔는데 STD6종 검사를 다시 할 수 있나요?}
\Ans{네 가능합니다. 만약 마이코플라즈마 유레아 플라즈마 2종 균이 나와서 치료 후 추적관찰을 한다면 STD6종 검사를 다시하시거나 중합효소연쇄반응 [C5956006]으로 2종 하셔도 됩니다.}
%\Que{Real time 방식으로 하면 삭감이 될까요?}
%\Ans{STD  검사 방식에( Real time Multiplex PCR,Multiplex PCR) 따른 급여기준은 아직 없습니다. 급여기준이 없다는 것은 삭감기준도 없다는 것으로 임상적으로  필요한 경우 급여 적용 하시면 됩니다. 현재 일부의 지역에서는 초진 환자에서 검사를 시행한 경우엔 삭감하고 있는 실정입니다만..}
\Que{STD6종을 비급여로 받아도 되나요?}
\Ans{“하부 요로 생식기 및 성매개 감염 의심환자" 이외의 경우 원내고시 후 고시된 가격으로 받으시면 되십니다.	예) \textcolor{blue}{단순 검진하러 오신분 / 성파트너가 성감염성 질환에 걸린분 등}}
\Que{Single PCR 은 몇 개 까지 가능한가요?}
\Ans{두개 까지 가능하고 사례에 따라 그 이상도 가능합니다.}
\Que{HSV 같은 경우 Type I \& II 가 있는데 200\% 청구 가능한가요?}
\Ans{네 가능합니다.  청구차트에 2-1-1로 표시하시면 됩니다}

\subsection{2017년 STD검사에 대한 청구 상황}
초진시는 STD RT는 안 내고 있는 경향이고, 재진시도 청구메모를 잘 적어야 삭감이 없는듯 싶습니다.
\begin{itemize}\tightlist
\item 저의 경우 몇 개월 전에 사전 연락 없이 갑자기 초진 RT 검사 전액을 삭감했었습니다. 그 전까지 잘 나왔었는데요 \par
그래서 사유를 일일이 다 적어서 재심 청구했는데 몇 개월 지나서 반 정도는 들어왔습니다. 그 다음부터는 초진이고 재진이고 그냥 다 STD 12 종을 냅니다\par
여기에서 보면 지역에 따라 초진 재진 구분 없이 잘 주는 곳도 있는 것 같기도 하지만 한번 대량 삭감이 되고 나니 그렇게 따를 수밖에요. 사유를 잘 적고 못 적고의 차이는 아닌 것 같습니다. 그리고 저는 STD 12종 검사와  RT와의 정밀도라든지 하는 차이를 잘 모르겠더라고요.\par
그래서 그냥 \emph{초진이든 재진이든 마음 편히 먹고 STD 12종으로 하고 있습니다}
\item 심평원에서 multi 12종은초재진 상관 없이 인정,RT7종은 재진만 인정 한다는 내부적인 선을 정한것 같습니다. 12종이나 7종 하면서 1종 (어떤균이든) 추가시 참고사항을 잘 적으면 넘어가는듯 싶습니다.
\item 검사실 비용은 m6 < m12 < m7 입니다. 참 m12는 다이오진 제품만 있습니다.
\item 저는 20여건 지급불능 ㅡ 모두 인정되었습니다. 한 달 정도 걸렷구요 ㅡ 앞으로 rt 는 초재진을 불문하고 자제해달라는 전화는 받았습니다.
\end{itemize}