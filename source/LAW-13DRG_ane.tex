\subsection{질병군 급여 일반원칙(마취 초빙료관련)}
\begin{myshadowbox}
\begin{enumerate}[18.]\tightlist
\item 질병군 진료시 마취통증의학과 전문의를 초빙하여 마취를 실시한 경우에는 제1편(행위별 수가)제2부제6장 바-2 마취통증의학과 전문의 초빙료를 추가 산정하며, 제1편(행위별 수가)제2부제6장 및 「요양급여의 적용기준 및 방법에 관한 세부사항」의 마취통증의학과 전문의 초빙료 산정 관련 규정을 적용한다.
\end{enumerate}
\end{myshadowbox}

마취통증의학과 전문의를 초빙하여 마취를 실시한 경우 특정내역 MT007(DRG 세부내역)의 내역구분 ‘ANE' 에 마취통증의학과 전문의 초빙료(바 2, L7990)를 질병군 요양급여비용 이외에 추가 산정함\par
이 경우 마취통증의학과 전문의 초빙료 및 면허종류, 면허번호를 기재하여야 함 \par
(작성요령) ‘15년 의원 단가 기준
\prezi{\clearpage}
\par
\medskip
\Que{마취통증의학과 전문의가 상근하는 요양기관에서 마취통증의학과 전문의 초빙료를 산정할 수 있는지?}
\Ans{마취통증의학과 전문의가 상근하는 요양기관은 마취통증의학과 전문의 초빙료를 산정할 수 없음 \par
다만, 고시 제2012-153호(‘12.11.27.)에 따라 일부 예외적인 상황에 한하여 마취통증의학과 전문의 초빙료를 산정할 수 있으며 관련 법령에 의거 인력 등에 대한 변경신고(유선신고 포함)가 이루어져야함\par
☞「마취통증의학과 전문의가 상근하는 요양기관에서 마취통증의학과 전문의 초빙시 인정여부」고시 제2012-153호, 12.11.27.) \par
마취통증의학과 전문의가 상근하는 요양기관에서 마취통증의학과 전문의 초빙료를 산정할 수 있는 경우는 다음과 같으며, 요양급여비용 청구 시 부득이한 사유 또는 신고사실을 확인할 수 있도록 마취기록부, 변경신고서 등 객관적인 증빙자료를 첨부하여야 함\par
\begin{center}\emph{- 다 음 -}\end{center}
\begin{enumerate}[가.]\tightlist
\item 상근하는 마취통증의학과 전문의가 예비군 훈련 등 부득이한 사유로 부재중인 경우 수술이 가능한 다른 요양기관으로 환자를 이송 조치함이 원칙이나 이송할 수 없는 상황에서 마취통증의학과 전문의를 초빙하는 경우. 다만, 이 경우 관련 법령에 의거 인력 등에 대한 변경신고(유선신고 포함)가 이루어져야함
\item  천재지변, 기타 예기치 못한 구급사태 등으로 인하여 동일 시간대에 2인 이상의 수술이 동시에 이루어져야 할 부득이한 사유로 마취통증의학과 전문의를 초빙하는 경우
\item  마취통증의학과 전문의가 상근하는 산부인과 병ㆍ의원에서 야간 또는 공휴일에 임신 또는 분만관련 응급수술을 시행하게 되어 부득이하게 마취통증의학과 전문의를 초빙하는 경우
\end{enumerate}
}
\prezi{\clearpage}
\par
\medskip
\Que{7개 질병군 수술 후 또는 통증자가조절법(PCA) 실시로 인한 구역ㆍ구토 치료를 위해 항구토제를 사용한 경우 별도 산정  여부}
\Ans{수술 후 발생한 구역 및 구토의 치료를 위해사용한 항구토제는 요양급여 대상이며 포괄수가 급여비용에 포함되어 \textcolor{red}{별도로 산정할 수 없음.}  또한, 7개질병군 수술 후 통증관리를 위한 통증자가조절법(PCA)은 전액본인부담(100분의100) 항목이며, 이로 인한 구역 및 구토 방지ㆍ치료를 위해 항구토제를 사용한 경우도 별도 산정할 수 없음}
 \prezi{\clearpage}
\par
\medskip
\Que{질식분만 전 통증조절 목적으로 “무통분만 경막외 마취”를 실시하였으나, 질식분만에 실패하여 제왕절개분만을 실시한 경우 “무통분만 경막외 마취” 비용을 환자에게 별도 부담시킬 수 있는지 여부}
\Ans{질식분만 전 통증조절 목적으로 실시한 “무통분만 경막외 마취”는 질식분만에 실패하여 제왕절개분만을 실시한 경우에도 질병군 급여상대가치점수에 포함되어 \textcolor{red}{별도로 비용을 부담시킬 수 없음}\par
다만, 무통분만 경막외 마취를 실시하여 무통분만을 시도하다가 실패하여 제왕절개분만 후, 이미 가지고 있는 경막외 마취 카테터를 통해  통증자가조절법(PCA)을 시행한 경우에는 수기료(바22나(3)(나)LA227) 및 재충전 약제비 등은 100/100본인부담이 가능함}
