\section{비급여등 산부인과의사가 주의할 사항}
\begin{enumerate}[1.]\tightlist
\item 다음 각목의 질환으로서 \highlightR{업무 또는 일상생활에 지장이 없는 경우에 실시 또는 사용되는 행위$\cdot$약제 및 치료재료}
	\begin{enumerate}[가.]\tightlist
	\item 단순한 피로 또는 권태
	\item 주근깨$\cdot$다모(多毛)$\cdot$무모(無毛)$\cdot$백모증(白毛症)$\cdot$딸기코(주사비)$\cdot$점(모반)$\cdot$사마귀$\cdot$여드름$\cdot$노화현상으로 인한 탈모 등 피부질환
	\item 발기부전(impotence)$\cdot$불감증 또는 생식기 선천성기형 등의 비뇨생식기 질환
	\item 단순 코골음
	\item 질병을 동반하지 아니한 단순포경(phimosis)
	\item 검열반 등 안과질환
	\item 기타 가목 내지 바목에 상당하는 질환으로서 보건복지부장관이 정하여 고시하는 질환
	\end{enumerate}
\item 다음 각목의 진료로서 \highlightY{신체의 필수 기능개선 목적이 아닌 경우에 실시 또는 사용되는 행위$\cdot$약제 및 치료재료}
	\begin{enumerate}[가.]\tightlist
	\item 쌍꺼풀수술(이중검수술), 코성형수술(융비술), 유방확대$\cdot$축소술, 지방흡인술, 주름살제거술 등 \uline{미용목적의 성형수술과 그로 인한 후유증치료}
	\item 사시교정, 안와격리증의 교정 등 시각계 수술로써 시력개선의 목적이 아닌 외모개선 목적의 수술
	\item <삭제>
	\item 저작 또는 발음기능개선의 목적이 아닌 외모개선 목적의 악안면 교정술 및 교정치료
	\item 관절운동 제한이 없는 반흔구축성형술 등 외모개선 목적의 반흔제거술
	\item 안경, 콘텍트렌즈 등을 대체하기 위한 시력교정술
	\item 기타 가목 내지 바목에 상당하는 \uline{외모개선 목적의 진료로서 보건복지부장관이 정하여 고시하는 진료}
	\end{enumerate}
\item \highlight{다음 각목의 예방진료로서 질병$\cdot$부상의 진료를 직접목적으로 하지 아니하는 경우에 실시 또는 사용되는 행위$\cdot$약제 및 치료재료}
	\begin{enumerate}[가.]\tightlist
	\item \uline{본인의 희망에 의한 건강검진}(법 제52조의 규정에 의하여 공단이 가입자등에게 실시하는 건강검진 제외)
	\item \uline{예방접종(파상풍 혈청주사 등 치료목적으로 사용하는 예방주사 제외)}
	\item 구취제거, 치아 착색물질 제거, 치아 교정 및 보철을 위한 치석제거 및 구강보건증진 차원에서 정기적으로 실시하는 치석제거. 다만, 치석제거만으로 치료가 종료되는 전악(全顎) 치석제거로서 보건복지부장관이 정하여 고시하는 경우는 제외한다.
	\item 불소국소도포, 치면열구전색(치아홈메우기) 등 치아우식증 예방을 위한 진료. 다만, 18세 이하의 치아우식증에 이환되지 않은 순수 건전치아인 제1큰어금니 또는 제2큰어금니에 대한 치면열구전색(치아홈메우기)은 제외한다.
	\item 멀미 예방, 금연 등을 위한 진료
	\item \uline{유전성질환 등 태아의 이상유무를 진단하기 위한 세포유전학적검사}
	\item 장애인 진단서 등 각종 증명서 발급을 목적으로 하는 진료
	\item 기타 가목 내지 마목에 상당하는 예방진료로서 보건복지부장관이 정하여 고시하는 예방진료
	\end{enumerate}
\item 보험급여시책상 \highlightR{요양급여로 인정하기 어려운 경우 및 그 밖에 건강보험급여원리에 부합하지 아니하는 경우로서 다음 각목에서 정하는 비용$\cdot$행위$\cdot$약제 및 치료재료}
	\begin{enumerate}[가.]\tightlist
	\item 가입자 등이 다음 각 항목 중 어느 하나의 요건을 갖춘 요양기관에서 \uline{1개의 입원실에 3인 이하가 입원할 수 있는 병상(이하 "상급병상"이라 한다)}을 이용함에 따라 제8조에 따라 고시한 요양급여대상인 입원료(이하 "입원료"라 한다) 외에 추가로 부담하는 입원실 이용 비용. 다만, 상급종합병원의 상급병상 중 1인실 병상을 이용하는 경우에는 입원료를 포함한 입원실 이용비용 전액(다만, 격리치료 대상인 환자가 1인실에 입원하는 경우 등 보건복지부장관이 정하여 고시하는 불가피한 경우는 제외한다)
    		\begin{enumerate}[(1)]\tightlist
    		\item \uline{의료법령에 따라 허가를 받거나 신고한 병상 중 입원실 이용비용을 입원료만으로 산정하는 일반병상(이하 "일반병상"이라 한다)}을 다음의 구분에 따라 운영하는 경우. 다만, 규칙 제12조제1항 또는 제2항에 따라 제출한 요양기관 현황신고서 또는 요양기관 현황 변경신고서 상의 격리병실, 무균치료실, 특수진료실 및 중환자실과 「의료법」 제27조제3항제2호에 따른 외국인환자를 위한 전용 병실 및 병동의 병상은 일반병상 및 상급병상의 계산에서 제외한다.
     		\begin{enumerate}[(가)]\tightlist
     		\item 의료법령에 따라 신고한 \uline{병상이 10병상을 초과하는}「의료법」 제3조제2항제1호에 따른 \uline{의원급 의료기관과} 같은 항 제3호에 따른 \uline{병원급 의료기관(종합병원 및 상급종합병원은 제외하되, 「의료법」 제3조의5에 따라 지정된 산부인과 전문병원은 포함한다)}: \uline{일반병상을 총 병상의 50퍼센트 이상 확보할 것}
     		\item 「의료법」 제3조제2항제3호마목에 따른 종합병원(상급종합병원을 포함하되, 「의료법」 제3조의5에 따라 지정된 산부인과 전문병원은 제외한다): 일반병상을 총 병상의 70퍼센트 이상 확보할 것
     		\end{enumerate}
   		\item 의료법령에 의하여 신고한 병상이 10병상 이하인 경우
   		\end{enumerate}
	\item 가목에도 불구하고 다음 각 항목에 해당하는 경우에는 다음의 구분에 따른 비용 
    		\begin{enumerate}[(1)]\tightlist
    		\item 가입자등이 「의료법」 제3조제2항제3호라목에 따른 요양병원(「정신보건법」 제3조제3호에 따른 정신의료기관 중 정신병원, 「장애인복지법」 제58조제1항제4호에 따른 장애인 의료재활시설로서 「의료법」 제3조의2의 요건을 갖춘 의료기관은 제외한다. 이하 같다) 중 입원실 이용비용을 입원료만으로 산정하는 일반병상(규칙 제12조제1항 또는 제2항에 따라 제출한 요양기관 현황신고서 또는 요양기관 현황 변경신고서 상의 격리병실, 무균치료실, 특수진료실 및 중환자실과 「의료법」 제27조제3항제2호에 따른 외국인환자를 위한 전용 병실 및 병동의 병상은 제외한다)을 50퍼센트 이상 확보하여 운영하는 요양병원에서 1개의 입원실에 5인 이하가 입원할 수 있는 병상을 이용하는 경우: 제8조제4항 전단에 따라 고시한 입원료 외에 추가로 부담하는 입원실 이용 비용
    		\item 가입자등이 가목(1)에서 정한 요건을 갖춘 상급종합병원, 종합병원, 병원 중 「암관리법」 제22조에 따라 완화의료전문기관으로 지정된 요양기관에서 1인실 병상을 이용하여 같은 법 제24조에 따라 완화의료 입원진료를 받는 경우(격리치료 대상인 환자가 1인실에 입원하는 경우, 임종실을 이용하는 경우 등 보건복지부장관이 정하여 고시하는 불가피한 경우는 제외한다): 제8조제4항 전단에 따라 고시한 완화의료 입원실의 입원료 중 5인실 입원료 외에 추가로 부담하는 입원실 이용 비용
    		\end{enumerate}
	\item 법 제51조에 따라 장애인에게 보험급여를 실시하는 보장구를 제외한 보조기$\cdot$보청기$\cdot$안경 또는 콘택트렌즈 등 보장구. 다만, 보청기 중 보험급여의 적용을 받게 될 수술과 관련된 치료재료인 보건복지부장관이 정하여 고시하는 보청기는 제외한다.
	\item \uline{보조생식술(체내$\cdot$체외인공수정 포함)시 소요된 비용}
	\item \uline{친자확인을 위한 진단} 
	\item 치과의 보철(보철재료 및 기공료 등을 포함한다) 및 치과임플란트를 목적으로 실시한 부가수술(골이식수술 등을 포함한다). 다만, 보건복지부장관이 정하여 고시하는 65세 이상인 사람의 틀니 및 치과임플란트는 제외한다.
	\item 및 \item 삭제 <2002.10.24>
	\item 이 규칙 제8조의 규정에 의하여 보건복지부장관이 고시한 약제에 관한 급여목록표에서 정한 일반의약품으로서 「약사법」 제23조에 따른 조제에 의하지 아니하고 지급하는 약제
	\item 삭제 <2006.12.29>
	\item 「의료법」 제46조에 따른 선택진료를 받는 경우에 선택진료에 관한 규칙에 따라 추가되는 비용
	\item 「장기등 이식에 관한 법률」에 따른 장기이식을 위하여 다른 의료기관에서 채취한 골수 등 장기의 운반에 소요되는 비용
	\item 「마약류 관리에 관한 법률」 제40조에 따른 마약류중독자의 치료보호에 소요되는 비용
	\item 이 규칙 제11조제1항 또는 제13조제1항의 규정에 따라 \uline{요양급여대상 또는 비급여대상으로 결정$cdot$고시되기 전까지의 행위$cdot$치료재료}(「신의료기술평가에 관한 규칙」 제2조제2항에 따른 평가 유예 신의료기술을 포함하되, 같은 규칙 제3조의4에 따른 신의료기술평가 결과 안전성$cdot$유효성을 인정받지 못한 경우에는 제외한다). 다만, 제11조제8항 또는 제13조제1항 후단의 규정에 따라 소급하여 요양급여대상으로 적용되는 행위$cdot$치료재료(「신의료기술평가에 관한 규칙」 제2조제2항에 따른 평가 유예 신의료기술을 포함한다)는 제외한다.
	\end{enumerate}
	\begin{enumerate}[거.]\tightlist
	\item 「신의료기술평가에 관한 규칙」 제3조제8항제2호에 따른 제한적 의료기술
	\end{enumerate}
	\begin{enumerate}[너.]\tightlist
	\item 「의료기기법 시행규칙」 제32조제1항제6호에 따른 의료기기를 장기이식 또는 조직이식에 사용하는 의료행위
	\end{enumerate}
	\begin{enumerate}[더.]\tightlist
	\item  그 밖에 요양급여를 함에 있어서 비용효과성 등 진료상의 경제성이 불분명하여 보건복지부장관이 정하여 고시하는 검사$\cdot$처치$\cdot$수술 기타의 치료 또는 치료재료
	\end{enumerate}
\item 삭제 <2006.12.29>
\item 영 제21조제3항제2호에 따라 \uline{보건복지부장관이 정하여 고시하는 질병군에 대한 입원진료의 경우에는 제1호 내지 제4호(제4호 하목을 제외한다), 제7호에 해당되는 행위}$\cdot$\uline{약제 및 치료재료}. 다만, 제2호 사목, 제3호 아목, 제4호더목은 다음 각목에서 정하는 경우에 한한다.
	\begin{enumerate}[가.]\tightlist
	\item 보건복지부장관이 정하여 고시하는 행위 및 치료재료
	\item 질병군 진료 외의 목적으로 투여된 약제 6의2. 영 제21조제3항제3호에 따른 완화의료 입원진료의 경우에는 제1호부터 제3호까지, 제4호나목(2)$cdot$더목에 해당되는 행위$\cdot$약제 및 치료재료. 다만, 제2호사목, 제3호아목 및 제4호더목은 보건복지부장관이 정하여 고시하는 행위 및 치료재료에 한정한다.
	\end{enumerate}
\item 건강보험제도의 여건상 요양급여로 인정하기 어려운 경우
	\begin{enumerate}[가.]\tightlist
	\item 보건복지부장관이 정하여 고시하는 한방물리요법
	\item 한약첩약 및 기상한의서의 처방 등을 근거로 한 한방생약제제
	\end{enumerate}
\item 약사법령에 따라 허가를 받거나 신고한 범위를 벗어나 약제를 처방$\cdot$투여하려는 자가 보건복지부장관이 정하여 고시하는 절차에 따라 의학적 근거 등을 입증하여 비급여로 사용할 수 있는 경우. 다만, 제5조제3항에 따라 중증환자에게 처방$\cdot$투여하는 약제 중 보건복지부장관이 정하여 고시하는 약제는 건강보험심사평가원장의 공고에 따른다.
\end{enumerate}

\subsection{비급여의 분류}
\medskip
\tabulinesep =_2mm^2mm
\begin {tabu} to\linewidth {|X[2,c]|X[4,l]|X[4,l]|} \tabucline[.5pt]{-}
\rowcolor{ForestGreen!40} \centering 유 형 & \centering 정 의 & \centering 발생기전 \\ \tabucline[.5pt]{-}
\rowcolor{Yellow!40} 항목외 임의비급여 & 급여 비급여 목록에 등재 되어 있지 않은 행위 및 치료재표에 대한 비용 징수 & 주로 신기술 도입과정에서 발생  \\ \tabucline[.5pt]{-}
\rowcolor{Yellow!40} 급여기준 초과 & 급여기준초과시 별도로 비급여가 명시되지 않은 부분을 비급여로 징수 & 의학적으로 급여기준이 마련되지 못하거나 재정의 한계로 급여 기준이 이에 미치지 못해 발생  \\ \tabucline[.5pt]{-}
\rowcolor{Yellow!40} 별도산정 불가 & 약제 치료재료비용이 행위료에 포함되어 있는 것을 별도로 비용징수 & 치료재료 및 약제, 의료행위에 수반되는 각종 기자재들의 발전과 고가화에 따라 발생  \\ \tabucline[.5pt]{-}
\rowcolor{Yellow!40} 허가사항 초과 & 식약처의 허가범위를 초과하여 사용한 약제 치료재료의 비용징수 & 신규약제 및 치료재료에 대한 축적속에서 적용대상의 확대를 급여기준이 따라가지 못해 발생  \\ \tabucline[.5pt]{-}
\rowcolor{Yellow!40} 심사삭감 & 심사삭감을 우려하여 급여 기준에 해당하는 항목에 대한 비용징수 & 과도한 삭감에 대한 의료공급자의 자기방어적 기전으로 발생  \\ \tabucline[.5pt]{-}
\end{tabu}

\par
\medskip
\begin{commentbox}{임의 비급여란?}
건강보험법상 급여나 비급여 항목에 포함되지 않는 진료행위. 현행 국민건강보험법상 \textcolor{red}{명시된 비급여 항목 외의 비급여 의료 환자에게 청구하는 것은 불법(Negative 방식)}
\end{commentbox}
\emph{임의 비급여 문제점}\par
현행 제도는 급여로 적용되는 기준을 초과하거나 급여·비급여 대상에 포함되어 있지 않지만 의학적으로 환자에게 필요한 의료행위, 즉 '의학적 기준에 근거한 불가피한 비급여'의 경우에도 법령상 금지되어 있다는 이유로 환불처분 및 행정처분을 부과\par
\textcolor{blue}{환자의 건강회복권, 생명권 및 진료선택권 등 자기결정권이 심각하게 제한을 받음}

\leftrod{급여기준을 초과한 항목의 임의비급여 처리(여의도성모병원의 백혈병 진료비 사건 )}
일회용 골수검사용 주사기:2008년 1월부터 보험적용\par
글리벡(백혈병 치료약물)

\begin{itemize}\tightlist
\item 2006년 12월 백혈병환우회는 성모병원이 백혈병환자로부터 진료비를 과다본인부담시켰다며 폭로 
\item 2008년 복지부는 실사를 거쳐 의료급여분과 건강보험분을 포함해 총 169억원에 달하는 임의비급여 진료비 환수 및 과징금 처분
\item 2009년 10월 성모병원 행정소송 1심 승소
\item 2010년 11월 2심 승소
\item 2010년 12월 복지부와 건강보험공단이 대법원에 상고
\item 2012년 6월 대법원에서 고등법원으로 파기 환송
:   대법원은 여의도성모병원이 ▲의학적 안전성과 유효성을 갖추고 ▲시급성이 있으며 ▲환자에게 미리 그 내용과 비용을 설명해 동의를 받은 임의비급여인지를 증명하라고 함
\item 2012년 11월 서울 고등법원은 심평원과 성모병원의 항소 모두 기각
\end{itemize}

\leftrod{별도산정 불가에 따른 임의비급여 처리(NST 환수사태)}
\begin{itemize}\tightlist
\item 2002년 고시를 통해 분만 전 감시와 태동검사는 같은 것이라며 의사들의 요구를 거절(신의료기술 보류)
\item 2003년 의료행위전문평가위원회가 수가신설이 필요하다고 결정
\item 2009년 3월 심평원은 태동검사 1회에 한하여 보험급여 인정, 2002년부터 2009년 사이에 한 태동검사 환수 결정
\item 2010년 복지부 건강보험분쟁조정위원회가 진료비 환불처분이 부당하다며 산부인과 의사가 제기한 취소청구가 기각
\item 2011년 1월 태동검사가 보험 대상으로 바뀌기 이전 부과됐던 검사비까지 산모들에게 환급하라고 한 처분은 부당하다며 산부인과 의사 최 모 씨 등 15명이 건강보험심사평가원을 상대로 낸 행정소송에서 1심패소 고등법원에서도 원고들의 항소가 기각되었으며,  원고들이 대법원에 상고
\item 2012년 8월 대법원이 산부인과의 산전 비자극검사(NST) 비용을 환자들에게 환급하는게 정당하다는 원심 판결을 파기하고 서울고법으로 환송
\item 2013년 4월 고법 소송  \textcolor{red}{심평원 '승' '산전 비자극검사, 임의비급여 예외적 인정 요건 갖추지 못했다‘}
\item 2014년 9월  서울고등법원 제7행정부는 산부인과 의사들이 심평원을 상대로 제기한 과다본인부담금 확인처분 등 취소청구 파기환송심에서 의사들의 청구를 기각
\end{itemize}
\begin{center}
\parbox[t]{.8\textwidth}{재판부는 “어떤 진료행위를 요양급여대상에 편입시킬 것인지는 여러 상황을 고려해 정책적으로 결정해야 할 문제로 설령  의학적 필요성이 인정된다고 해도 해당 의료행위를 요양급여대상으로 인정할 수는 없다”며 \textcolor{red}{“임의비급여 진료를 광범위하게 인정할 경우 건강보험제도를 형식뿐인 제도로 만들 수 있는 우려가 있고 그 인정은 엄격한 요건 하에 이뤄져야 한다고 볼 때 수진자 동의의 요건을 의사들이 주장하는 추정적 동의의 경우에도 적용하는 것이 적절하다고 보기 어렵다”}고 판결했다.}
\end{center}

\leftrod{식약청 허가범위를 초과하여 사용한 경우 임의비급여( 서울대학교병원 진료비 환불 사건)}
담도용 스텐트를 기도에 사용(허가 사항을 초과)\par
펜타닐 수술실이 아닌 중환자실에서 사용했을 경우 (허가 사항을 초과) \par
마이토마이신 원래 항암제였지만, 수년 전부터 기도협착 방지에 효과가 있다는 연구 결과가 있어서 사용(허가 사항을 초과)\par

\begin{itemize}\tightlist
\item 2003년 선천성 기관지 질병으로 11차례에 걸쳐 입원하여 총102차례에 걸쳐 치료를 받던 신생아 이모군이 사망하자 유가족에게 8,000여만원의 진료비를 받았다.
이군의 어머니는 건강보험심사평가원에 요양급여대상여부확인 신청을 했다. 
\item 2004년 서울대병원은 건강보험심사평가원으로부터 “환자 유가족에게 5,000여만원을 환불하라”는 통보를 받자 소송을 냈다.
\item 2007년 9월 1심 승소(환불 처분을 받은 5,089만원 중 286만원에 대한 취소)
\item 2008년 9월 2심 일부승소(환불 처분을 받은 5,089만원 중 152만원에 대한 취소:사실상 패소)
\item 2013년 3월 대법원은 원심을 파기하고 사건을 서울고등법원으로 환송
\end{itemize}
