\maxtocdepth{subsubsection}
\maxsecnumdepth{subsubsection}
\chapter{보험청구}

\section{국민건강보험의 이해: 각 직역별}
\subsection{보건복지부}
건강보험사업의 주관 (법 제2조)
•주관 내용
–국민건강보험종합계획 수립
–‘건강보험정책심의위원회’ 운영
–공단ㆍ심평원 및 요양기관 관리 감독
–요양급여대상 및 그 인정기준 결정
–행위별 상대가치점수, 약제ㆍ치료재료별 상한금액 결정
–요양기관 현지조사 및 업무정지(과징금) 처분
\begin{center}
\includegraphics[width=.9\textwidth]{NGW}
\end{center}
건강보험정책심의위원회 (국민건강보험법 제 4조)
–보건복지부장관 소속
–기능
•요양급여의 기준
•요양급여비용에 관한 사항
•직장가입자의 보험료율
•지역가입자의 보험료부과 점수당 금액
–구성: 위원장 1명과 부위원장 1명을 포함하여 25명의 위원
•위원장 : 보건복지부차관
•부위원장 : 공익위원 증에서 위원장이 지명
•위원
8명 근로자단체 및 사용자단체가 추천 각 2명,
시민단체, 소비자단체, 농어업인단체 및 자영업자단체가 추천하는 각 1명
8명 의료계를 대표하는 단체 및 약업계를 대표하는 단체가 추천하는 8명
8명 공무원 2명, 공단의 이사장 및 심평원의 원장이 추천하는 각 1명,
건강보험에 관한 학식과 경험이 풍부한 4명


\subsection{국민건강보험공단(NHIS)}
\begin{center}
\includegraphics[width=.9\textwidth]{NHIS}
\end{center}
건강보험의 보험자 (법 제13조)
•업무
–가입자 및 피부양자의 자격 관리
–보험료와 그 밖에 이법에 다른 징수금의 부과ㆍ징수
–보험급여의 관리
–가입자 및 피부양자의 건강 유지와 증진을 위하여 필요한 예방사업
–보험급여 비용의 지급
–자산의 관리ㆍ운영 및 증식사업
–의료시설의 운영
–건강보험에 관한 교육훈련 및 홍보
–건강보험에 관한 조사연구 및 국제협력
•이사회(15명)
–이사장
•보건복지부장관의 제청에 의하여 대통령이 임명
–이사
•6명: 노동조합, 사용자단체, 시민단체, 소비자단체, 농어업인단체, 노인단체에서 각 1명
•5명: 이사장 추천
•3명: 관계 공무원
•재정운영위원회(30명)
–요양급여비용의 계약체계, 직장가입자의 보수월액 산정방법 및 지역가입자의 보험료 부과점수 산정방법, 보험료 등의 결손처분 등을 심의 의결
–10명 : 직장가입자를 대표
–10명 : 지역가입자를 대표
–10명 : 공익을 대표

\subsection{건강보험심사평가원(HIRA)}
•심사 및 평가기관 (법 제62조)
•업무
–요양급여비용의 심사
–요양급여의 적정성 평가
–심사기준 및 평가기준의 개발
–관련 조사연구 및 국제협력
–다른 법률에 따른 급여비용 심사 및 적정성평가 위탁 업무
\begin{center}
\includegraphics[width=.9\textwidth]{HIRA}
\end{center}
•이사회(15명)
–원장
•대통령이 임면
–이사
•5명 의약관계단체 추천자
•1명 국민건강보험공단 추천자
•3명 건강보험심사평가원 추천자
•4명 노동조합, 사용자단체, 소비자단체, 농어업인 단체 각 1명
•1명 관계 공무원
•진료심사평가위원회
–1,090명 이내(상근 90명, 비상근 1000명)

\subsection{요양기관}
•요양기관 당연지정제 채택 (법 제42조)
–관련 법률에 의거 개설ㆍ등록된 의료기관 및 약국 등은 별도의 신청 또는 지정 절차 없이 당연 요양기관으로 지정되어 가입자 등의 질병, 부상, 분만에 대한 요양급여를 실시함
–민간의료기관이 대부분 (90\%이상) 인 우리나라 특수성 반영
•종별 요양기관 현황(2016년 기준)

\subsection{가입자}
•전국민 건강보험 가입 (법 제5조, 제6조)
–의료급여 수급권자, 유공자등 의료보호대상자를 제외한 국내에 거주하는 국민 모두가 가입자 또는 피부양자가 됨
–직장가입자(피부양자)와 지역가입자로 구분
•가입자별 대상자 현황(2016년 6월 기준)

\section{건강보험 재정 및 보험료}
건강보험 재원
•보험료 (법 제69조)
•국고지원금 (법 제108조)
•국민건강증진기금에서의 지원금
(국민건강증진법 부칙)
•기타 수입금
\begin{center}
\includegraphics[width=.9\textwidth]{MIincome}
\end{center}

\subsection{건강보험료 부과체계}

\subsubsection{가입자별 상이한 부과체계 (2016년)}
\tabulinesep =_2mm^2mm
\begin{tabu} to\linewidth {|X[1,l]|X[3,l]|X[5,l]|} \tabucline[.5pt]{-}
\rowcolor{ForestGreen!40}  구 분 & 직장가입자 & 지역가입자 \\ \tabucline[.5pt]{-}
\rowcolor{Yellow!40} 부과기준 & \begin{itemize}\tightlist \item 보수 * 사용자 : 사업소득 \item 보수 外 소득 \end{itemize} & \begin{itemize}\tightlist \item 연소득 > 500만원 초과 세대: 소득, 재산, 자동차 점수 합산 \item 연소득 ≤ 500만원 이하 세대: 평가소득*, 재산, 자동차 점수 합산 \item * 성․연령, 소득, 재산, 자동차로 산정\end{itemize} \\ \tabucline[.5pt]{-}
\rowcolor{Yellow!40} 산정방식 & \begin{itemize}\tightlist \item 보수월액 × 정률(6.12\%, ‘17) \item 보수 外 소득 연 7,200만원 초과 : 보수 外 소득의 3.06\% 추가로 부과 \end{itemize} & \begin{itemize}\tightlist \item 보험료 부과점수 × 점수당 금액(179.6원, ‘17) \end{itemize} \\ \tabucline[.5pt]{-}
\rowcolor{Yellow!40} 납부자 & ▪사용자 50\%, 근로자 50\% & ▪지역가입자 세대 100\% \\ \tabucline[.5pt]{-}
\rowcolor{Yellow!40} 최저‧최고 보험료(월) & ▪(본인부담) 8,560원~239만원\newline * 사용자부담 포함시 17,120원~478만원 & ▪3,590원~228만원\newline * 전액 본인부담 \\ \tabucline[.5pt]{-}
\end{tabu}
\par
\medskip






































\section{갑자기 청구 금액이 많아지면 실사 나오지 않을까요?}
\begin{tcolorbox}[enhanced jigsaw,breakable,pad at break*=1mm,
  colback=blue!5!white,colframe=blue!75!black,title= 실사 나오지 않을까요?,
  watermark color=white,watermark text=실사]
\begin{description}\tightlist
\item[K :] 상기 내용은 일년에 50-150군데 정도 이루어지는 기획실사 안내입니다. 대분분의 현지조사(실사)는 민원과 부당청구 개연성에의해 의심되는 기관에 의해 이루어집니다.
\item[A :] 네- 공단 메세지 신청해놓았더니 저런 내용까지 메세지가 오니까 괜히 긴장되네요- 
\item[KK :] 이게 그들의 레이다 망인데. 결국 진료지표중 가장 중요한 축인 고가도 지표와 내원 입원일수 지표인데..CI \& VI.
\item[KK :] \textcolor{red}{고가도 지표를 확인하는 항목}이 \uline{항생제 처방률, 주사제 처방률, 약품목수, 처방의약품비}만 있습니다. 즉 약에만 타겟되어서 산부인과는 별로 떨 필요가 없습니다.
\item[KK :] 다행이 검사는 이 항목에 없습니다. 
\item[A :] 다행이네요. 감사합니다!! 
\item[KK :] 그리고 모든 지표가 질병군별 동종업종 과 비교해서 나옵니다. 중요한 얘기인데..
\item[KK :] 질병군별 이라는 것이 N761이 주상병이면 다른 산부인과와 얼마나 차이 나느냐의 문제 인데.
\item[KK :] 전 늘 1번 상병이 K30 / K297이라..통계에서 빠질수 있습니다.[이거 너무 꿀팁인데 ㅎㅎ]
\item[A :] 제일 첫 상병이 주 상병인가요? 1번 상병이 제일 중요한가요? 저는 그런거 상관없이 마구잡이로 넣고 있는데..
\item[KK :] 그래서 화개 김옹이 맨날 하시는 말씀...이왕이면 엄청날거 같은 상병을 위로 / 타과 상병을 위로 넣으라고 말씀하시는 거 입니다. 
\item[KK :] 첫번째 상병을 주상병으로 잡죠..그렇죠?김옹,,
\item[A :] 네... 저도 오늘 이전 강의 파일을 리뷰했네요.. 
\item[J :] 해결된 상병은 담번에 다 삭제하구여
\item[K :] 예,,모든 지표와 통계는 첫번째 상병으로(주상병),,이루어집니다...
\item[C :] 그럼 주진단명은 아니지만  K599같은것을 먼저입력해도 전혀 상관없나요?
\item[J :] 상병당 평균진료비가 낮은건...
\item[J :] 많은걸로다...해야겠지요
\item[K :] 가급적이면,,평균진료비가 많을것 같은것,,아니면 타과 상병(모든 지표가 질병군별 동종업종 과 비교)을 넣는게 좋습니다...
\item[A :] 강의 파일을 보니 상병은 많을 수록 좋다고 되어 있던데,, 해결된 상병은 그때그때 지워주는게 좋을까요? 
\item[K :] 그때그때 지워주는게 제일 좋죠,,,하지만,, 진료당일 진료나 약제와 관련된 상병이 없는것 보다는 관련이 될것같은  상병이 많은게 좋습니다,,,(원외처방 환수나 청구 삭감방지를 위해서죠)
\item[A :] 네.. 조언감사드립니다!! 
\end{description}
\end{tcolorbox}
원장님들이 보험청구가 많아지면 진료지표가 높아지고,실사대상타겟이 된다고 알고있고,그래서 보험으로하는걸 많이 걱정하십니다.\par
진료지표에는 자율시정통보제도[진료지표], 지표연동관리제가 있었습니다.. 자율시정통보제도[진료지표]는 요양기관의 상병별 지표(건당 내원일수, 내원일당 진료비)로 진료비 청구수준의 높낮이를 알려주는 지표이고,지표연동관리제는 관리대상 항목으로는 \uline{1.내원 일수 2. 항생제  처방률 3. 주사제  처방률 4. 약품목수 5.외래처방 약품비} 입니다. \emph{건강보험재정에서 원외처방 약제비 지출이 50\%가 넘습니다} \par
이중 규제 논란으로 의료계의 반발을 샀던 자율시정통보제와 지표연동관리제가 `지표연동자율개선제'라는 하나의 제도로 통합 운영되게 되었습니다.\par

기존 두 제도의 취지와 운영 방식이 유사하고 현지조사와 연계된 이중 규제라는 논란이 있어왔던 까닭에 의료계도 이번 통합 운영 방침에 환영의 뜻을 밝히고 나섰다.

26일 대한의사협회 관계자에 따르면 최근 복지부는 의협에 복지부와 건강보험심사평가원에서 각각 시행하고 있던 자율시정통보제 및 지표연동관리제가를 `지표연동자율개선제'로 일원화하겠다는 뜻을 전달했다.

그간 복지부는 요양급여비용 부당청구 사전예방 및 기관단위의 총량적 심사를 위해 `자율시정통보제'를, 건강보험심사평가원은 `지표연동관리제'를 각각 시행해 왔다.

두 제도는 관리지표가 상대적으로 높은 기관에 그 내용을 통보해 자율적 진료행태 개선을 유도하고 시정하지 않는 기관에 대해서는 현지조사를 실시해 왔다.

두 제도 모두 핵심 관리지표로 내세운 것이 `\newindex{내원일수 지표(VI)}'와 `\newindex{건당 진료비고가도 지표(CI)}'로 사실상 동일해 관리대상 기관 역시 대부분 중복되고 계도 내용마저 비슷해 의료계는 중복 처벌이라는 비판을 지속적으로 제기해 왔다.
\begin{figure}
\includegraphics[scale=.45]{VI}
\includegraphics[scale=.45]{ECI}
%\includegraphics[scale=.45]{DCI}
\caption{수학적 공식이 머리를 아프게 합니다만 사실 간단한 산술식입니다.
다른 동종산부인과와 비교해서 내 병의원이 얼마나 고가의 치료를 하고 있는지를 보여 주는 것이죠.
건당 진료비 고가도 지표[ECI]를 ``1"이되면 아무 문제가 없습니다.
결국 정보의 공유와 실행으로 많은 산부인과가 많은 보험청구를 같이 실행하면 수치는 점점 1에 가까와 질 것입니다.
청구액을 줄여서 1에 맞추는 것보다 모두 청구액을 늘려서 1에 맞추는 방향이 우리가 가야할 방향입니다.
\dotemph{VI나 CI가 1.1이상일때 실사 가능성이 있다고 합니다.} }
%하지만 광주권에 있는 저희 병원에서는 VI는 당연히 현저하게 낮을거라고 확신하고 있으며 또한 CI도 현저하게 낮을거라고 생각합니다.}
\end{figure}

이에 복지부는 유사한 지표 점검 제도를 일원화해 지표연동자율개선제를 운영하기로 합의점을 찾았다.

복지부는 ``자율시정통보제는 폐지하고 제도의 취지와 핵심 내용은 지표연동관리제에 포괄적으로 승계해 지표점검제도를 일원화하겠다"고 밝혔다.

내용을 살펴보면 진료형태별 통합 방안으로 외래진료비는 지표연동관리제의 내원일수 관리항목으로 통합하게 된다.

입원진료비는 지표연동관리제의 관리항목으로 추가하되, 입원관리지표 재설정 전까지는 현행 자율시정통보제의 지표산출체계를 적용키로 했다.

다만 심사과정에서 거짓·부당청구가 의심·확인되는 요양기관에 대해 현지조사 대상으로 선정하는 지표연동관리-현지조사 연계 방침은 현행 방식을 그대로 유지하기로 했다.

복지부는 ``이번 제도를 통해 기존 제도의 목적과 내용, 대상기관이 서로 유사·중복됨에 따른 요양기관의 업무부담을 해소할 수 있다"면서 ``중복적 성격의 규제를 완화함으로써 정부의 불필요한 규제완화 정책기조에 부합하고, 지표점검제도의 효율성을 향상시킬 수 있다"고 강조했다.

이중규제에 대한 문제를 지속적으로 제기하던 의협도 환영의 뜻을 밝혔다.

의협은 ``제2차 의정협의에서 일원화하기로 합의한 사실을 강조하며 두 제도를 일원화하도록 지속적으로 건의해 왔다"면서 ``향후에도 의료기관 업무 부담을 최소화하는 방안으로 지표 항목과 기준 완화 등 개선을 지속적으로 요청할 계획"이라고 반겼다.

\subsection{건당진료비(CI)의 이해}
\begin{itemize}\tightlist
\item 우리병원이 다른 병원(비교군) 평균과 비교해서 진료군별 건당 얼마나 높은지 비교
\item 지표의 헛점
	\begin{enumerate}\tightlist
	\item 질병군별 건당 진료비라는 점!!!!
		N760 K297 로  STD 6 종 검사를 하였다면 K297을 1번 상병(주상병) 으로 넣는다
	\item 이왕이면 건당 진료비가 높을것 같은 상병 입력 생리통 보다는 자궁근종을 1번 상병으로
	\item 타과상병을 위로위로..	.
	\end{enumerate}
\item 모든 지표는 모두 비교군과(남들과의) 비교로 나오게 되므로 모든 병원이 전체적으로 올라가면 만사형통
\end{itemize}
\subsection{관리항목별 선정기준 예시}
\tabulinesep =_2mm^2mm
\begin {tabu} to\linewidth {|X[2,l]|X[3,l]|} \tabucline[.5pt]{-}
\rowcolor{ForestGreen!40}  관리항목 & \centering 선정기준 \\ \tabucline[.5pt]{-}
\rowcolor{Yellow!40} 내원일수 & 내원일수지표(VI) 1.1 이상 \& 건강진료비고가도지표(CI) 1.0 이상 \&  개설기관 상위 15\%  \\ \tabucline[.5pt]{-}
\rowcolor{Yellow!40} 급성상기도감염 항생제 처방률 & 항생체 처방률 80\% 이상 기관 \\ \tabucline[.5pt]{-}
\rowcolor{Yellow!40} 주사제 처방률 & 주사제 처방률 60\% 이상 기관  \\ \tabucline[.5pt]{-}
\rowcolor{Yellow!40} 약품목수 & 6품목 이상 처방비율 40\% 이상 기관  \\ \tabucline[.5pt]{-}
\rowcolor{Yellow!40} 외래처방약품비 & 외래처방약품비고가도지표(OPCI) 1.3 이상 기관  \\ \tabucline[.5pt]{-}
\end{tabu}

\subsection{실사조사 유형의 분류}
\begin{itemize}\tightlist
\item 민원제보기관 (검.경찰 권익위, 국민신고마당) : 잦은 민원은 현지실사의 가장 많은 이유입니다.
\item 내부종사자 신고,진료내역통보 : 내부종사자에 의한 실사도 두번째이유입니다.
\item 자율시정 불응기관 :일년에 3-4000개 통보 기관중 40-50개 현지조사 : 자율시정 여러번 미시정시도 실사의 이유입니다.
\item 심평원 의뢰기관 
\item 공단 의뢰기기관 
\item 데이터 마이닝(부당감지지표)에 의한 기관 
\end{itemize}

\begin{shaded}
상대적으로 저희과(산부인과)는 원외처방 고가약제가 적습니다. 그래서 진료지표가 낮습니다(청구액이 너무 적어요)
또한 실사의 가장흔한 원인은 \mycoloredbox{퇴직직원의 내부고발과 다발성 민원입니다}.청구가 많아 진료지표가 많이 높아도 타과에 비해 많이 높아지기 힘들고, 실사까지 이어지기는 힘듭니다.  그래도 걱정이 많이 되시면
\begin{description}\tightlist
\item[해결책1.] 복합 상병시 건당진료비가 높을것 같은 상병을 맨 위에 기재 :질염(N761-N768)보다는 골반염(N730-N739)또는 자궁근종(D250-D259) 
\item[해결책2.] 다빈도 상병을 보조상병으로 입력: K30, K297, M5496, R1039등등
\item[해결책3.] 진료비가 높은 경우 타과 상병을 입력 :동일 과에 대한 다빈도 상병을 기준으로 통계를 냄: 질염(N761-N768)보다는 고혈압(I10), 당뇨(E149)을 위의 상병으로 넣을것.
\item[해결책4.] 이왕이면 엄청난 상병을 입력: 정확한 코딩을 위하여서는 환자의 상태에 대한 상병(코드)을 적어야 하지만 실상 심평원에서는 환자의 상태는 별개로 상병(코드)과 
  처치 내역을 보고 심사를 하고 있으므로 광역의  상병 또는 중환 상병이 필요(난소의 양성 신생물(D279)보다는 난소의 악성신생물(C569))
\end{description}
결론적으로 소신진료후 진료기록포함 증빙자료 작성하여 보존 하시면 됩니다.\\
또 다른 \dotemph{근본적인 해결책은 VI와 CI를 동일하게 모든 병원에서 늘리면 됩니다.}
\end{shaded}
\clearpage
\section{진료비 확인민원과 실비보험}
결국은 영수증이 중요합니다. 영수증은 구분 별로 (검사. 치료재료, 기타)와 급여, 비급여 구분을 잘 해줘야 실비보험, 진료비 확인민원에 시달리지 않아요.\\
비용을 받을 수있는 부분(합법 비급여)은 비싸게, 나머지는 원칙대로 급여,비급여 구분하고, 영수증도 환자에게 설명하기 편하게, 구분별로 (검사. 치료재료, 기타)급여, 비급여 구분을 잘 해서 설명 해줘야합니다.
\clearpage

\section{심평원의 무엇을 기준으로 심사를 할까요?}
\begin{center}
\includegraphics[scale=.75]{chungubrain.png}
\end{center}

\begin{itemize}
\item 진단명
\item 진단에 대한 지침, 규정에 맞느냐? 
\end{itemize}
그래서 우리가 준비를 잘 해야 하는것은

\begin{itemize}
\item 적합상명 (배제진단명등)
\item 청구메모등을 잘쓴다.\index{청구메모예시들}
	\begin{enumerate}\tightlist
	\item N882  자궁목의 협착 : 청구메모에 ``자궁목의 협착이 있어 심사평가원의 답변에 근거하여 요도확장술을 준용하여 사용함"
	\item 3도 열상 : 항문에 달하는 회음부열창있고, 항문괄약근의 분리있어서 vicryl 1-0로 3-4번의 단면을 맞춰서 interrupted suture  해줌.
회음부통증과 변비로 인한 상처감염과 tearing우려로 항생제와 설사제, 진통제처방함.
	\item 4도 열상 : 직장과 항문괄약근의 열창이 있어서 직장은 vicryl 3-0로 1cm 간격으로 layer by layer로 꼬매주고, 항문괄약근은 vicryl 1-0로 3-4번 단면을 맞춰서 interrupted suture해줌
회음부 통증과 변비로 인한 감염위험으로 인해 항생제와 설사제 진통제치료함
	\item 경부열상 : 경부열상이 있어서 ovum forcep으로 잡고당기면서,vicryl 1-0로  interrupted꼬매줌.
	\item STI6종검사 : 1. 질내분비물이 현저하게 증가하고 악취가 나는등 부인과적 질내감염이 의심되어 시행함.
2. 곤지름이나 트리코모나스등 다른 성병이 있음
	\item Ovarian Tumor marker : 수술이 필요할 정도로 큰 난소의 낭종으로 골반초음파상 난소암의심으로 검사함.
	\item 액상세포진검사 : 1. 자궁경부세포진검사상 이상소견보여 추후관찰이 필요한경우
2. 인유두종바이러스 검사에서 이상이 있어 추후관찰이 필요함.
3. 자궁경부암 전단계또는 자궁경부암으로 진단되어 치료를 받은후 재발여부 평가위해서 시행함
4. 자궁경부출혈이나 폴립이 있어 시행함.
	\item HPV 검사 : 1.  자궁경부세포진 검사상 이상소견이 있는 경우임.
	\item MTX : 혈류역학적으로 안정상태이고 초음파상 복강내출혈소견없으며 다음것중에 번을 충족함
1. 임신 6주이하, 2. 혈청  hCG < 15,000, 3. 자궁외임신으로인한 난관의 종괴가 3.5cm이하 4. 초음파검사에서 태아 심박동이 없는경우
	\end{enumerate}
\end{itemize}

\section{BREAK YOUR FRAME}
\begin{itemize}
\item 심평원과 우리가 사용되는 언어가 다름을 인정하라
\item 환자의 진료비에 대한 거부감을 충분한 설명과 영수증을 통해서 거부감을 제거하라
\item 직원에게 청구를 시키면 청구하면 안된다는 관점으로 보고 일하기 때문에 직접 청구를 하는게 좋습니다.
\item 삭감(상병과 청구메모), 실사(민원과 직원관리), 세금(압도적 매출상승 조절)등을 통해서 걱정에서 벗어나세요.
\item 현상이 아닌 본질을 추구 하라
	\begin{enumerate}\tightlist
	\item Hematuria\index{hematuria} → RUA MICRO ← CZ521/NMP22/ASO/C3/C4/BUN/Cr/ANAb/ PT/PTT/ BL SONO등 13가지 보험청구
	\item \newindex{GALACTORREHA} → PRL ← TSH/T3/Free T4/E2/LHp/FSH/Cathecolamine3 등 12가지 보험청구
	\item HBsAg [+]  or OT/PT증가 → HBsAg/HBsAb ← HBsAg/HBsAb/HCAb/HBeAg/HBV DNA/STOT/SGPT/CPK/HBcAB IgG IgM등 12가지
	\item 갑산선 기능이상 → TSH/Free T4 ← TSH/Ft4/T3/TBIG/ATAb/Antimicosomal Ab Thyroglobulin등 10가지  보험청구
	\end{enumerate}
\item 가치부여 : N760, N86, N30, N92, Hormone, Tumor marker, Mirena, culture, N95, N73, N97, N73, CZ521, HPV, 비급여, SONO, N91, CX541, B373	
\item TWO STYLE(진료는 My style, 청구는 BRROO style) 예를 들면 생리\index{amenorrhea}가 없을때
	\begin{enumerate}\tightlist
	\item N915 상세불명의 희발월경 : FSH/LH/E2/Progesterone
	\item E079 상세불명의 갑상선의 장애 : TSH/FT4/T3
	\item E282 다낭성난소증후군 : Testosterone/DHEAs
	\item N643 출산과 관련되지 않는 젖 흐름증 : PRL
	\item O089 상세불명의 유산,자궁외임신 및 기타 임신에 따른 합병증 :	bHCG
	\end{enumerate}
\item 행위수가가 최대인것을 선택하라 : 외음의 농양\index{외음의 농양} 처치후 다음것중에 어떤것?
	\begin{enumerate}\tightlist
	\item N764 외음의 고름집[농양]   - M0137 흡입배농 및 배액처치 [8,200원]
	\item \emph{D289 상세불명의 여성생식기의 양성신생물 - R4066 외음부 양성종양 적출술[105,400원]}
	\item D281 질의 양성신생물 - R4070 질종양적출술 [99,800원]
	\item D239 상세불명 피부의 기타 양성 신생물 -    N0141 피부양성종양적출술(간단한표재성) [31100원] ,   N0142  피부양성종양적출술(기타근육층에 달하는것) [52300원]
	\item A630 항문 성기[성병의]사마귀[콘딜로마, 곤지름] –    R4305 음부콘딜로마의 수술적 치료[절제술, 전기소작술, 냉동치료] 
   [35400원]
	\item N950/N951 바르톨린샘의 낭종/농양 - R4065 바돌린선 낭종 조대술 [79,800원]
	\end{enumerate}
\item 인정비급여를 공략하라	
\end{itemize}

\subsection{심화가 필요한 이유}
\begin{itemize}\tightlist
\item 자기납득 : 내 스스로 납득이 안되면 누구에게도 권할 수 없다.
\item 설명의 진정성 : 가치가 높은 진료를 제공하게 된다.
\item SAFE : 마지막 안정망은 차팅입니다.
\end{itemize}
\clearpage

\section{건강보험제도의 이해 순서 (지피지기知彼知己)}
이해 순서
\begin{enumerate}[1)]\tightlist
\item 국민건강보험법 제41조(요양급여)
\item 국민건강보험 요양급여의 기준에 관한 규칙 
\item 국민건강보험 요양급여의 기준에 관한 규칙 [별표1]요양급여의 적용기준 및 방법(제5조제1항관련)
\item 국민건강보험 요양급여의 기준에 관한 규칙[별표2]비급여대상(제9조제1항관련)
\item 건강보험요양급여비용
\end{enumerate}

\subsection{\newindex{제41조(요양급여)}}
\begin{enumerate}[①]\tightlist
\item 가입자와 피부양자의 질병, 부상, 출산 등에 대하여 다음 각 호의 요양급여를 실시한다.
	\begin{enumerate}[	1.]\tightlist
	\item 진찰·검사
	\item 약제(藥劑)·치료재료의 지급
	\item 처치·수술 및 그 밖의 치료
	\item 예방·재활
	\item 입원
	\item 간호
	\item 이송(移送)
	\end{enumerate}
\item 제1항에 따른 \uline{요양급여(이하 ``요양급여"라 한다)의 방법·절차·범위·상한 등의 기준은 보건복지부령으로 정한다.}
\item 보건복지부장관은 제2항에 따라 요양급여의 기준을 정할 때 \uline{업무나 일상생활에 지장이 없는 질환, 그 밖에 보건복지부령으로 정하는 사항은 요양급여의 대상에서 제외}할 수 있다.
\end{enumerate}

\subsection{\newindex{제5조(요양급여의 적용기준 및 방법)}}
\emph{국민건강보험 요양급여의 기준에 관한 규칙  제5조(요양급여의 적용기준 및 방법)}
\begin{enumerate}[①]\tightlist
\item 요양기관은 가입자등에 대한 요양급여를 \uline{별표 1의요양급여의 적용기준 및 방법에 의하여 실시}하여야 한다.
\item 제1항의 규정에 의한 요양급여의 적용기준 및 방법에 관한 세부사항과 조혈모세포이식 \uline{요양급여의 적용기준 및 방법에 관한 세부사항은} 의약계·공단 및 건강보험심사평가원의 의견을 들어 \uline{보건복지부장관이 정하여 고시}한다.<개정 2008·3·3, 2010·3·19, 2010·12·23>
\item 제2항의 규정에 불구하고 「국민건강보험법 시행령」(이하 ``영"이라 한다) 별표 2 제3호마목에 따른 중증질환자(이하 ``중증환자"라 한다)에게 처방·투여하는 약제중 보건복지부장관이 정하여 고시하는 약제에 대한 요양급여의 적용기준 및 방법에 관한 세부사항은 제5조의2의 규정에 의한 중증질환심의위원회의 심의를 거쳐 건강보험심사평가원장이 정하여 공고한다. 이 경우 건강보험심사평가원장은 요양기관 및 가입자등이 해당 공고의 내용을 언제든지 열람할 수 있도록 관리하여야 한다. <신설 2005·10·11, 2008·3·3, 2010·3·19, 2012ㆍ8ㆍ31>
\end{enumerate}

\paragraph{[별표 1]요양급여의 적용기준 및 방법(제5조제1항관련)}
\begin{enumerate}[1.]\tightlist
\item \mycoloredbox{요양급여의 일반원칙}
	\begin{enumerate}[가.]\tightlist
	\item 요양급여는 가입자 등의 \uline{연령·성별·직업 및 심신상태 등의 특성을 고려하여 진료의 필요가 있다고 인정되는 경우에 정확한 진단을 토대로 하여 환자의 건강증진을 위하여 의학적으로 인정되는 범위 안에서 최적의 방법으로 실시하여야 한다.}
	\item 요양급여를 담당하는 의료인은 의학적 윤리를 견지하여 환자에게 심리적 건강효과를 주도록 노력하여야 하며, 요양상 필요한 사항이나 예방의학 및 공중보건에 관한 지식을 환자 또는 보호자에게 이해하기 쉽도록 적절하게 설명하고 지도하여야 한다.
	\item 요양급여는 \uline{경제적으로 비용 효과적인 방법으로 행하여야 한다.}
	\item 요양기관은 \uline{가입자 등의 요양급여에 필요한 적정한 인력·시설 및 장비를 유지하여야 한다. 이 경우 보건복지부장관은 인력·시설 및 장비의 적정기준을 정하여 고시할 수 있다.}
	\item 라목의 규정에 불구하고 가입자 등에 대한 최적의 요양급여를 실시하기 위하여 필요한 경우, 보건복지부장관이 정하여 고시하는 바에 따라 다른 기관에 검사를 위탁하거나, 당해 요양기관에 소속되지 아니한 전문성이 뛰어난 의료인을 초빙하거나, 다른 요양기관에서 보유하고 있는 양질의 시설·인력 및 장비를 공동 활용할 수 있다.
	\end{enumerate}
\item \mycoloredbox{진찰·검사, 처치·수술 기타의 치료}
	\begin{enumerate}[가.]\tightlist
	\item \uline{각종 검사를 포함한 진단 및 치료행위는 진료상 필요하다고 인정되는 경우에 한하여야 하며 연구의 목적으로 하여서는 아니된다.}
	\item 영 제21조제3항제2호에 따라 보건복지부장관이 정하여 고시하는 질병군에 대한 입원진료의 경우 그 입원진료 기간동안 행하는 것이 의학적으로 타당한 검사·처치 등의 진료행위는 당해 입원진료에 포함하여 행하여야 한다.
	\end{enumerate}
\item 약제의 지급 
	\begin{enumerate}[가.]\tightlist
	\item 처방·조제
		\begin{enumerate}[(1)]\tightlist
		\item 영양공급·안정·운동 그 밖에 요양상 주의를 함으로써 치료효과를 얻을 수 있다고 인정되는 경우에는 의약품을 처방·투여하여서는 아니되며, 이에 관하여 적절하게 설명하고 지도하여야 한다.
		\item 의약품은 약사법령에 의하여 \large{허가 또는 신고된 사항(효능·효과 및 용법·용량 등)의 범위 안에서 환자의 증상 등에 따라 필요·적절하게 처방·투여하여야 한다. (약제 전산심사)}다만, \uline{안전성·유효성 등에 관한 사항이 정하여져 있는 의약품 중 진료상 반드시 필요하다고 보건복지부장관이 정하여 고시하는 의약품의 경우에는 허가 또는 신고된 사항의 범위를 초과하여 처방·투여할 수 있으며}, 중증환자에게 처방·투여하는 약제로서 보건복지부장관이 정하여 고시하는 약제의 경우에는 건강보험심사평가원장이 공고한 범위 안에서 처방·투여할 수 있다.
		\item \uline{요양기관은 중증환자에 대한 약제의 처방·투여시 해당약제 및 처방·투여의 범위가 (2)의 허용범위에는 해당하지 아니하나 해당환자의 치료를 위하여 특히 필요한 경우에는 건강보험심사평가원장에게 해당약제의 품목명 및 처방·투여의 범위 등에 관한 자료를 제출한 후 건강보험심사평가원장이 중증질환심의위원회의 심의를 거쳐 인정하는 범위 안에서 처방·투여할 수 있다.}
		\item 제10조의2제2항에 따라 식품의약품안전처장이 긴급한 도입이 필요하다고 인정한 품목의 경우에는 식품의약품안전처장이 인정한 범위 안에서 처방·투여하여야 한다.
		\item 항생제·스테로이드제제 등 오남용의 폐해가 우려되는 의약품은 환자의 병력·투약력 등을 고려하여 신중하게 처방·투여하여야 한다.
		\item \uline{진료상 2품목 이상의 의약품을 병용하여 처방·투여하는 경우에는 1품목의 처방·투여로는 치료효과를 기대하기 어렵다고 의학적으로 인정되는 경우에 한한다.}
		\end{enumerate}  
	\item 주사
		\begin{enumerate}[(1)]\tightlist
		\item \uline{주사는 경구투약을 할 수 없는 경우, 경구투약시 위장장애 등의 부작용을 일으킬 염려가 있는 경우, 경구투약으로 치료효과를 기대할 수 없는 경우 또는 응급환자에게 신속한 치료효과를 기대할 필요가 있는 경우에 한한다.}
		\item \uline{동일 효능의 내복약과 주사제는 병용하여 처방·투여하여서는 아니된다. 다만, 경구투약만으로는 치료효과를 기대할 수 없는 불가피한 경우에 한하여 병용하여 처방·투여할 수 있다.}
		\item 혼합주사는 치료효과를 높일 수 있다고 의학적으로 인정되는 경우에 한한다. 
		\item \uline{당류제제·전해질제제·복합아미노산제제·혈액대용제·혈액 및 혈액성분제제의 주사는 의학적으로 특히 필요하다고 인정되는 경우에 한한다.}
		\end{enumerate}
	\end{enumerate}
\item 치료재료의 지급 : 
\uline{치료재료는} 약사법 기타 다른 관계법령에 의하여 \uline{허가·신고 또는 인정된 사항(효능·효과 및 사용방법)의 범위 안에서 환자의 증상에 따라 의학적 판단에 의하여 필요·적절하게 사용한다. 다만, 안전성·유효성 등에 관한 사항이 정하여져 있는 치료재료 중 진료에 반드시 필요하다고 보건복지부장관이 정하여 고시하는 치료재료의 경우에는 허가·신고 또는 인정된 사항(효능·효과 및 사용방법)의 범위를 초과하여 사용할 수 있다.}
\item 예방·재활 : 
재활 및 물리치료(이학요법)는 약물투여 또는 처치 및 수술 등에 의하여 치료효과를 얻기 곤란한 경우로서 재활 및 물리치료(이학요법)가 보다 효과가 있다고 인정되는 경우에 행한다.
\item 입원
	\begin{enumerate}[가.]\tightlist
	\item \uline{입원은 진료상 필요하다고 인정되는 경우에 한하며 단순한 피로회복·통원불편 등을 이유로 입원지시를 하여서는 아니된다.}
	\item 퇴원은 의학적 타당성과 퇴원계획의 충분성 등을 신중하게 고려하여 적절한 시기에 행하여져야 한다.
	\item 입원환자에 대한 식사는 환자의 치료에 적합한 수준에서 의료법령 및 식품위생법령에서 정하는 기준에 맞게 위생적인 방법으로 제공하여야 한다
	\end{enumerate}
\end{enumerate}

\subsection{\newindex{제8조(요양급여의 범위 등)}}
\begin{enumerate}[①]\tightlist
\item 법 제41조제2항에 따른 \large{요양급여의 범위(이하 ``요양급여대상"이라 한다)는 다음 각 호와 같다.} <개정 2006·12·29, 2012ㆍ8ㆍ31>
	\begin{enumerate}[1.]\tightlist
	\item \uline{법 제41조제1항 각 호의 요양급여(약제를 제외한다) : 제9조에 따른 비급여대상을 제외한 일체의 것}
	\item 법 제41조제1항제2호의 요양급여(약제에 한한다) : 제11조의2, 제12조 및 제13조에 따라 요양급여대상으로 결정 또는 조정되어 고시된 것
	\end{enumerate}
\item 보건복지부장관은 제1항의 규정에 의한 요양급여대상을 급여목록표로 정하여 고시하되, 법 제41조제1항의 각호에 규정된 요양급여행위(이하 ``행위"라 한다), 약제 및 치료재료(법 제41조제1항제2호의 규정에 의하여 지급되는 약제 및 치료재료를 말한다. 이하 같다)로 구분하여 고시한다. 다만, 보건복지부장관이 정하여 고시하는 요양기관의 진료에 대하여는 행위·약제 및 치료재료를 묶어 1회 방문에 따른 행위로 정하여 고시할 수 있다. <개정 2001·12·31, 2008·3·3, 2010·3·19, 2012ㆍ8ㆍ31>
\item 보건복지부장관은 제2항에도 불구하고 영 제21조제3항제2호에 따라 보건복지가족부장관이 정하여 고시하는 질병군에 대한 입원진료의 경우에는 해당 질병군별로 별표 2 제6호에 따른 비급여대상, 규칙 별표 6 제1호사목에 따른 이송처치료 및 같은 호 아목1)에 따른 요양급여비용의 본인부담 항목을 제외한 모든 행위·약제 및 치료재료를 묶어 하나의 포괄적인 행위로 정하여 고시할 수 있다. 이 경우 하나의 포괄적인 행위에서 제외되는 항목은 보건복지부장관이 정하여 고시할 수 있다. <개정 2001·12·31,2005·10·11, 2008·3·3 , 2010·3·19, 2012ㆍ8ㆍ31, 2014ㆍ9ㆍ1, 2015ㆍ5ㆍ29, 2015ㆍ6ㆍ30>
\item 보건복지부장관은 제2항에도 불구하고 영 제21조제3항제1호에 따른 요양병원의 입원진료나 같은 항 제3호에 따른 완화의료의 입원진료의 경우에는 제2항의 행위·약제 및 치료재료를 묶어 1일당 행위로 정하여 고시할 수 있다. 이 경우 1일당 행위에서 제외되는 항목은 보건복지부장관이 정하여 고시할 수 있다. <신설 2007·12·28 보건복지부령428, 2008·3·3 , 2010·3·19, 2012ㆍ8ㆍ31, 2015ㆍ6ㆍ30>
\item 보건복지부장관은 제2항부터 제4항까지의 규정에 따라 요양급여대상을 고시함에 있어 행위 또는 하나의 포괄적인 행위의 경우에는 영 제21조제2항에 따른 요양급여의 상대가치점수를 함께 정하여 고시하여야 한다. <2007·12·28 보건복지부령428, 2008·3·3, 2010·3·19, 2012ㆍ8ㆍ31>
\end{enumerate}  
\subsection{\newindex{제9조(비급여대상)}}
\begin{enumerate}[①]\tightlist
\item 법 제41조제3항에 따라 요양급여의대상에서 제외되는 사항(이하 \uline{``비급여대상"이라 한다)은 별표 2와 같다.} <개정 2012ㆍ8ㆍ31>
\end{enumerate}

\paragraph{[별표 2]\newindex{비급여대상(제9조제1항관련)}}
\begin{enumerate}[1.]\tightlist
\item 다음 각목의 질환으로서 \uline{업무 또는 일상생활에 지장이 없는 경우에 실시 또는 사용되는 행위·약제 및 치료재료}
	\begin{enumerate}[가.]\tightlist
	\item 단순한 피로 또는 권태
	\item 주근깨·다모(多毛)·무모(無毛)·백모증(白毛症)·딸기코(주사비)·점(모반)·사마귀·여드름·노화현상으로 인한 탈모 등 피부질환
	\item 발기부전(impotence)·불감증 또는 생식기 선천성기형 등의 비뇨생식기 질환
	\item 단순 코골음
	\item 질병을 동반하지 아니한 단순포경(phimosis)
	\item 검열반 등 안과질환
	\item 기타 가목 내지 바목에 상당하는 질환으로서 보건복지부장관이 정하여 고시하는 질환
	\end{enumerate}
\item 다음 각목의 진료로서 \uline{신체의 필수 기능개선 목적이 아닌 경우에 실시 또는 사용되는 행위·약제 및 치료재료}
	\begin{enumerate}[가.]\tightlist
	\item 쌍꺼풀수술(이중검수술), 코성형수술(융비술), 유방확대·축소술, 지방흡인술, 주름살제거술 등 \uline{미용목적의 성형수술과 그로 인한 후유증치료}
	\item 사시교정, 안와격리증의 교정 등 시각계 수술로써 시력개선의 목적이 아닌 외모개선 목적의 수술
	\item <삭제>
	\item 저작 또는 발음기능개선의 목적이 아닌 외모개선 목적의 악안면 교정술 및 교정치료
	\item 관절운동 제한이 없는 반흔구축성형술 등 외모개선 목적의 반흔제거술
	\item 안경, 콘텍트렌즈 등을 대체하기 위한 시력교정술
	\item 기타 가목 내지 바목에 상당하는 \uline{외모개선 목적의 진료로서 보건복지부장관이 정하여 고시하는 진료}
	\end{enumerate}
\item \uline{다음 각목의 예방진료로서 질병·부상의 진료를 직접 목적으로 하지 아니하는 경우에 실시 또는 사용되는 행위·약제 및 치료재료}
	\begin{enumerate}[가.]\tightlist
	\item \uline{본인의 희망에 의한 건강검진}(법 제47조의 규정에 의하여 공단이 가입자등에게 실시하는 건강검진 제외)
	\item \uline{예방접종(파상풍 혈청주사 등 치료목적으로 사용하는 예방주사 제외)}
	\item 구취제거, 치아 착색물질 제거, 치아 교정 및 보철을 위한 치석제거 및 구강보건증진 차원에서 정기적으로 실시하는 치석제거
	\item 불소국소도포, 치면열구전색 등 치아우식증 예방을 위한 진료(치아우식증에 이환되지 않은 순수 건전치아를 가진 만 6세 이상 14세 이하 소아의 제1대구치에 대한 치면열구전색 제외)
	\item 멀미 예방, 금연 등을 위한 진료
	\item \uline{유전성질환 등 태아의 이상유무를 진단하기 위한 세포유전학적검사}
	\item 기타 가목 내지 마목에 상당하는 예방진료로서 보건복지부장관이 정하여 고시하는 예방진료
	\end{enumerate}
\item 보험급여시책상 \uline{요양급여로 인정하기 어려운 경우} 및 그 밖에 \uline{건강보험급여원리에 부합하지 아니하는 경우로서} 다음 각목에서 정하는 \uline{비용·행위·약제 및 치료재료}
	\begin{enumerate}[가.]\tightlist
	\item 가입자 등이 다음 각 항목 중 어느 하나의 요건을 갖춘 요양기관에서 \uline{1개의 입원실에 3인 이하가 입원할 수 있는 병상(이하 ``상급병상"이라 한다)}을 이용함에 따라 영 제24조제2항 및 제8조제4항의 규정에 의하여 고시한 상대가치점수로 산정한 입원료(이하 ``기본입원료"라 한다) 외에 추가로 부담하는 입원실 이용 비용
    		\begin{enumerate}[(1)]\tightlist
			\item \uline{의료법령에 따라 허가를 받거나 신고한 병상 중 기본입원료만 산정하는 일반병상(이하 “일반병상”이라 한다)을 다음의 구분에 따른 비율 이상을 확보하여 운영하는 경우.} 다만, 규칙 제12조제3항 또는 제4항에 따라 제출한 요양기관현황통보서 또는 요양기관변경통보서 상의 격리병실, 무균치료실, 특수진료실 및 중환자실과 「의료법」 제27조제3항제2호에 따른 외국인환자를 위한 전용 병실 및 병동의 병상은 일반병상 및 상급병상의 계산에서 제외한다.
     			\begin{enumerate}[(가)]\tightlist
				\item 의료법령에 따라 신고한 \uline{병상이 10병상을 초과하는} 「의료법」 제3조제2항제1호에 따른 \uline{의원급 의료기관과} 같은 항 제2호에 따른 \uline{병원급 의료기관(종합병원 및 상급종합병원은 제외한다)}: \uline{환자 6인 이상이} 입원할 수 있는 일반병상의 입원료(이하 “최저입원료"라 한다)만으로 산정하는 일반병상을 \uline{50퍼센트 이상 확보할 것}. 이 경우 「암관리법」 제22조에 따라 완화의료전문기관으로 지정된 요양기관에서 같은 법 제24조에 따라 완화의료 입원진료를 받는 경우에는 환자 5인이 입원할 수 있는 일반병상의 입원료(이하 ``5인실 입원료"라 한다)를 최저입원료로 한다. 
     			\item 「의료법」 제3조제2항제3호마목에 따른 종합병원(상급종합병원을 포함한다): 70퍼센트
				\end{enumerate}
   			\item \uline{의료법령에 의하여 신고한 병상이 10병상 이하인 경우}
			\end{enumerate}
	\item 「의료법」 제3조제2항제3호마목에 따른 종합병원(상급종합병원을 포함한다): 다음 표에 따른 비율 
	
	\medskip%par
	\tabulinesep =_2mm^2mm
	\begin{tabu} to\linewidth {|X[2,l]|X[2,l]|} \tabucline[.5pt]{-}
	\rowcolor{ForestGreen!40}  \centering 구 분 &	\centering 확보 비율 \\ \tabucline[.5pt]{-}
	\rowcolor{Yellow!40} 보건복지부령 제30호 국민건강보험 요양급여의 기준에 관한 규칙 일부개정령 부칙 제3조제1항 각 호에 해당하는 종합병원
 & 최저입원료만으로 산정하는 일반병상을 50퍼센트 이상 확보할 것 \\ \tabucline[.5pt]{-}
	\rowcolor{Yellow!40} 보건복지부령 제30호 국민건강보험 요양급여의 기준에 관한 규칙 일부개정령 부칙 제3조제2항 각 호에 해당하는 종합병원
 & 최저입원료만으로 산정하는 일반병상을 50퍼센트 이상 확보하고, 2010. 12. 23. 이후 신설 병상 중 일반병상을 70퍼센트 이상 확보할 것 \\ \tabucline[.5pt]{-}
	\rowcolor{Yellow!40} 그 외의 모든 종합병원  & 일반병상을 70퍼센트 이상 확보할 것. 이 경우 최저입원료만으로 산정하는 일반병상이 총 병상의 50퍼센트 이상이어야 한다. \\ \tabucline[.5pt]{-}
	\end{tabu}
	\item 법 제46조에 의하여 장애인에게 보험급여를 실시하는 보장구를 제외한 보조기·보청기·안경 또는 콘택트렌즈 등 보장구
	\item \uline{보조생식술(체내·체외인공수정 포함)시 소요된 비용}
	\item \uline{친자확인을 위한 진단  }
	\item 치과의 보철(보철재료 및 기공료 등 포함)
	\item 및 아. 삭제 <2002.10.24>
	\item 이 규칙 제8조의 규정에 의하여 보건복지부장관이 고시한 약제에 관한 급여목록표에서 정한 일반의약품으로서 「약사법」 제23조에 따른 조제에 의하지 아니하고 지급하는 약제
	\item 삭제 <2006.12.29>
	\item 「의료법」 제46조에 따른 선택진료를 받는 경우에 선택진료에 관한 규칙에 따라 추가되는 비용
	\item 「장기등 이식에 관한 법률」에 따른 장기이식을 위하여 다른 의료기관에서 채취한 골수 등 장기의 운반에 소요되는 비용
	\item 「마약류 관리에 관한 법률」 제40조에 따른 마약류중독자의 치료보호에 소요되는 비용
	\item 이 규칙 제11조제1항 또는 제13조제1항의 규정에 의하여 \uline{요양급여대상 또는 비급여대상으로 결정·고시되기 전까지의 신의료기술} 등. 다만, 제11조제4항 또는 제13조제1항 후단의 규정에 의하여 소급하여 요양급여대상으로 적용되는 신의료기술 등을 제외한다. 
	\item 그 밖에 요양급여를 함에 있어서 비용효과성 등 \uline{진료상의 경제성이 불분명하여 보건복지부장관이 정하여 고시하는 검사·처치·수술 기타의 치료 또는 치료재료}
	\end{enumerate}
\item 삭제 <2006.12.29>
\item 영 별표 2 제2호의 규정에 의하여 보건복지부장관이 정하여 고시하는 질병군에 대한 입원진료의 경우에는 제1호 내지 제4호(제4호 하목을 제외한다), 제7호에 해당되는 행위·약제 및 치료재료. 다만, 제2호 사목, 제3호 사목, 제4호거목은 다음 각목에서 정하는 경우에 한한다.
	\begin{enumerate}[가.]\tightlist
	\item \uline{보건복지부장관이 정하여 고시하는 행위 및 치료재료}
  	\item \uline{질병군 진료 외의 목적으로 투여된 약제}
	\end{enumerate}
\item 건강보험제도의 여건상 요양급여로 인정하기 어려운 경우
	\begin{enumerate}[가.]\tightlist
	\item 보건복지부장관이 정하여 고시하는 한방물리요법
  	\item 한약첩약 및 기상한의서의 처방 등을 근거로 한 한방생약제제
	\end{enumerate}
\item 약사법령에 따라 허가를 받거나 신고한 범위를 벗어나 약제를 처방·투여하려는 자가 보건복지부장관이 정하여 고시하는 절차에 따라 의학적 근거 등을 입증하여 비급여로 사용할 수 있는 경우. 다만, 제5조제3항에 따라 중증환자에게 처방·투여하는 약제 중 보건복지부장관이 정하여 고시하는 약제는 건강보험심사평가원장의 공고에 따른다.
\end{enumerate}

\subsection{\newindex{건강보험요양급여비용}}
\begin{mdframed}[linecolor=blue,middlelinewidth=2]
제1편 행위 급여 \cntrdot{}  비급여 목록 및 급여 상대가치점수 >> 제1부 행위 급여 일반원칙 >> Ⅰ. 일반기준
\end{mdframed}

\paragraph{I.일반기준}
\begin{enumerate}[1.]\tightlist
\item 요양기관이 국민건강보험법령의 규정에 의한 \uline{요양급여를 실시하고 행위에 대한 비용을 산정할 때에는} 제2부 각 장에 분류된 분류항목의 \mycoloredbox{상대가치 점수}(이하 “점수”라 한다)에 국민건강보험법 제45조제3항과 같은 법 시행령 제21조제1항에 따라 \uline{정하여진 점수당 단가}(제16장에 분류된 항목은 「지역보건법」에 따른 보건소\cntrdot{} 보건의료원 및 보건지소와 「농어촌등 보건의료를 위한 특별조치법」에 따라 설치된 보건진료소의 점수당 단가)\uline{를 곱하여 10원 미만은 4사5입한 금액으로 산정한다.} 다만, 요양기관 종별가산율에 의하여 산출된 금액에 대하여는 원미만을 4사5입한다.
\item 각종 가감률에 의하여 산출된 금액에 대하여는 1호와 동일한 방법으로 산정하되 상대가치점수에 가감률을 곱하여 총 점수(소수점 이하 셋째 자리에서 4사5입)를 산출하고, 각종 가감률이 복합 적용될 경우에는 가감률을 모두 합한 총 가감률을 상대가치점수에 곱하여 총 점수(소수점 이하 셋째 자리에서 4사5입)를 산출한다. 이 경우 가감률이 중복 적용될 경우에는 중복 가산하지 아니한다.
\item 제2부 각 장에 \uline{분류되지 아니한 진찰\cntrdot{} 처치\cntrdot{} 수술 및 기타의 치료에 대한 요양급여를 실시한 경우에 우선적으로 행위의 내용\cntrdot{} 성격과 상대 가치점수가 가장 유사한 분류항목에 준용하여 산정하여야 한다.(준용산정)}
\item 상급종합병원, 종합병원, 병원, 요양병원(의과), 의원, 보건의료원(의과), 의과 진료과목이 있는 한방병원\cntrdot{}치과병원인 요양기관은 제2부 제1장 내지 제9장 및 제16장, 제17장에 분류된 분류항목과 제10장에 분류된 분류항목 중 고정장치의 제거, 악간고정술, 치간고정술, 순열수술후
보호장치, 상고정장치술, 구강내소염수술, 구강외소염수술, 구강내열상 봉합술, 구강외열상봉합술, 협순소대성형술, 악골수염수술, 악골내고정용 금속제거술에 한하여 산정한다
\item 치과병원, 치과의원, 보건의료원(치과), 치과 진료과목이 있는 상급 종합병원 ㆍ종합병원ㆍ 병원
ㆍ한방병원인 요양기관은 제2부 제1장 내지 제10장과 제16장 내지 제18장에 분류된 항목에 한하여 산정한다.
\item 국립병원 한방진료부, 한방병원, 한의원, 보건의료원(한방과), 한방 진료 과목이 있는 상급종합병원\cntrdot{} 종합병원\cntrdot{} 병원\cntrdot{} 요양병원\cntrdot{}치과병원인 요양기관은 제2부 제1장, 제4장, 제13장, 제14장 및 제17장에 분류된 분류 항목에 한하여 산정한다.
\item 약국 및 한국희귀의약품센터인 요양기관은 제2부 제15장에 분류된 분류항목에 한하여 산정한다.
\item 조산원인 요양기관은 다음 분류항목에 한하여 산정한다.
	\begin{enumerate}[가.]\tightlist
	\item 제2부 제11장 및 제17장에 분류된 분류항목
	\item 제2부 제9장에 분류된 분류항목 중 자궁내장치삽입술 및 자궁내장치 제거료
	\item 기타 보건복지부장관(이하 “장관”이라 한다)이 불가피하다고 인정하는 경우
	\end{enumerate}
\item 보건소, 보건지소, 보건진료소인 요양기관은 다음 분류항목에 한하여 산정한다.
	\begin{enumerate}[가.]\tightlist
	\item 제2부 제12장에 분류된 분류항목
	\item 제2부 제9장에 분류된 정관절제술 또는 결찰술, 난관결찰술, 자궁내장 치삽입술,자궁내장치제거료
	\item 기타 장관이 불가피하다고 인정하는 경우
	\end{enumerate}
\item 의료법 제35조에 의한 부속 의료기관은 다음 분류항목에 한하여 산정한다.
	\begin{enumerate}[가.]\tightlist
	\item 제2부 제1장 재진진찰료, 의약품관리료, 혈액관리료
	\item 제2부 제4장 퇴장방지의약품 사용장려비
	\item 제2부 제5장, 제9장, 제10장, 제13장, 제14장 및 제16장에 분류된 분류항목
	\end{enumerate}	
\end{enumerate}

\paragraph{II. \newindex{요양기관 종별가산율}}
\begin{enumerate}[1.]\tightlist
\item 제2부 제2장 내지 제10장, 제13장 및 제14장에 분류된 분류항목에 \uline{대하여는 소정점수에 점수당 단가를 곱한 금액을 모두 합산한 금액에 요양기관의 종별에 따라 다음 각 호의 비율을 가산한다.}
	\begin{enumerate}[가.]\tightlist
	\item \uline{다음 각 항의 요양기관은 30\%}
		\begin{enumerate}[(1)]\tightlist
		\item 상급종합병원으로 인정받은 종합병원
		\item 상급종합병원에 설치된 치과대학 부속 치과병원
		\item 상급종합병원에 설치된 특수전문병원
		\item 의료법 제35조에 의한 부속 의료기관
		\end{enumerate}
	\item \uline{다음 각 항의 요양기관은 25\%}
		\begin{enumerate}[(1)]\tightlist
		\item 상급종합병원을 제외한 종합병원
		\item 상급종합병원에 설치된 경우를 제외한 치과대학 부속 치과병원
		\item 허가 병상 수가 30병상 이상이고, 한방 6개 과가 설치되어 있는 한의과대학 부속 한방병원
		\item 국립병원 한방진료부
		\item 의료법 제35조에 의한 부속 의료기관
		\end{enumerate}
	\item \uline{다음 각 항의 요양기관은 20\%}
		\begin{enumerate}[(1)]\tightlist
		\item 병원
		\item 위 “가-⑵” 또는 “나-⑵”에 해당되지 아니하는 치과병원
		\item 위 “나-⑶"에 해당되지 아니하는 한방병원
		\item 요양병원
		\item 의료법 제35조에 의한 부속 의료기관
		\end{enumerate}
	\item \uline{다음 각 항의 요양기관은 15\%}
		\begin{enumerate}[(1)]\tightlist
		\item \uline{의원}
		\item 치과의원
		\item 한의원
		\item 보건의료원
		\item 의료법 제35조에 의한 부속 의료기관
		\end{enumerate}		
	\item 다음 각 항의 요양기관은 종별가산율을 적용하지 아니한다.
		\begin{enumerate}[(1)]\tightlist
		\item 약국 및 한국희귀의약품센터 
		\item 조산원, 보건소, 보건지소, 보건진료소
		\item 의료법 제35조에 의한 부속 의료기관
		\end{enumerate}
	\end{enumerate}
\item 위 “1"의 규정에도 불구하고 \uline{아래 항목에 대해서는 요양기관 종별 가산율을 적용하지 아니한다.}
	\begin{enumerate}[가.]\tightlist
	\item 바이러스 혈청검사(나-476, C4760)
	\item 각 장의 산정지침 또는 분류항목의 “주"에서 별도로 산정할 수 있도록 규정한 약제비, 치료재료대 등
	\item 영상저장 및 전송시스템 (Full PACS)(GB011-GB045, HB011-HB041, HG011-HG045, HG111 -HG141)을 이용한 처리비용, C-Arm형 영상 증폭장치 이용료(다-101, G0400)
	\item 생혈(마-103, X3010), 교환(마-104, X4000), 조혈모세포의 이식 준비 -냉동 처리 및 보관(마-105-다-(1), X5020), 기증제대혈제제 비용 (마-105-라-(3)-가, X5137), 자가수혈채혈료(마-106-가, X6001 내지
X6008), 연성신요관경하 요관협착확장술 “주"(자-319-3 “주", R3196),연성신요관경하 결석제거술 “주3"(자-321-3 “주3", R3429)
	\item 퇴장방지의약품 사용장려비
	\item 검체검사 위탁에 관한 기준에서 정한 수탁기관으로 위탁하는 경우의
검사료 및 위탁검사관리료
	\item \uline{Infusion Pump 사용료(KK058, KK158)}
	\item \uline{마취통증의학과 전문의 초빙료(L7990)}
	\end{enumerate}
\item 위 “1-나" 항의 종별가산율을 적용받은 종합병원이 의료법 제3조의3 기준에 부적합한 경우에는 3월 이내의 범위 내에서 기간을 정하여 시정 하도록 하고 동 시정기간 내에 시정하지 아니한 때에는 시정기간 종료 익일부터는 위 “1-다"항의 종별가산율을 적용한다
\end{enumerate}

\paragraph{Ⅲ. \newindex{차등수가}}
의과의원, 치과의원, 한의원, 보건의료원, 약국 및 한국희귀의약품센터의 경우에는 \uline{의사, 치과의사, 한의사, 약사 1인당 1일 진찰횟수, 약국 및 한국 희귀의약품센터의 경우에는 조제건수(처방전 매수를 말한다.} 이하 같다)에 따라서 요양기관에 진찰료와 조제료 등(조제료, 약국관리료, 조제기본료, 복약지도료를 말한다. 이하 같다)을 아래와 같이 차등지급한다. 다만, 의료 급여 환자, 장관이 별도로 정한 평일 18시(토요일은 13시)-익일 09시의 진찰료와 조제료 등, 기타 장관이 별도로 정하는 경우에는 차등수가 적용 대상에서 제외할 수 있다.\par
\begin{enumerate}[가.]\tightlist
\item \uline{의과의원, 치과의원, 한의원, 보건의료원의 의사, 치과의사, 한의사 1인당 1일 진찰횟수를 기준으로 진찰료에 대하여 다음과 같이 차등지급한다.}
	\begin{enumerate}[(1)]
	\item \uline{75건 이하:100\%}
	\item \uline{75건을 초과하여 100건까지:90\%}
	\item \uline{100건을 초과하여 150건까지:75\%}
	\item \uline{150건을 초과한 건:50\%}
	\end{enumerate}
\item 약국 및 한국희귀의약품센터의 약사 1인당 1일 조제건수(의약분업 예외 지역에서는 직접조제건수 포함)를 기준으로 조제료 등에 대하여 다음과 같이 차등지급한다.
	\begin{enumerate}[(1)]
	\item 75건 이하:100\%
	\item 75건을 초과하여 100건까지:90\%
	\item 100건을 초과하여 150건까지:75\%
	\item 150건을 초과한 건:50\%
	\end{enumerate}
\begin{mdframed}[linecolor=blue,middlelinewidth=2]
\Large{2015년 12월 차등수가제 폐지(의원급만)}\par
\bigskip

\Large{2016년 전문병원 의료 질 지원금(입원일당 1820원, 29억원 규모)과 전문병원 관리료(3개 분야 차등지원, 70억원 규모) 신설}\par
전문병원 관리료의 경우, ▲척추(한방 포함)와 관절, 대장항문은 790원(입원 1일당) ▲화상과 수지접합, 심장, 알코올, 유방, 주산기, 뇌혈관, 산부인과, 신경과, 안과, 외과, 이비인후과,재활의학과, 한방 중풍은 1980원 등이다.\par
\uline{암환자 교육상담료 신설과} 위험분담제 `피레스파정'(특발성 폐섬유증 치료제) 급여 적용, 당뇨병 환자 소모품 확대(인슐린 투여 환자, 채혈침, 인슐린 주사기, 펜인슐린바늘)
\end{mdframed}

\medskip
\tabulinesep =_2mm^2mm
\begin {tabu} to\linewidth {|X[2,l]|X[1,l]|X[2,l]|} \tabucline[.5pt]{-}
\rowcolor{ForestGreen!40}  \centering 전문과목\cntrdot{}질환 & \centering 금액(입원1일당) & \centering 비고 \\ \tabucline[.5pt]{-}
\rowcolor{Yellow!40} 척추(한방포함), 관절, 대장 항문 분야 & 790원 & 비급여 모니터링 결과 연계 조정 \\ \tabucline[.5pt]{-}
\rowcolor{Yellow!40} 화상, 수지접합, 심장, 알코올, 유방, 주산기, 뇌혈관, 산부인과, 신경과 안과, 외과, 이비인후과, 재활의학과, 한방중풍 분야 & 1,980원 & 평균 재원일수가 짧은 분야(안과, 이비인후과)는 외래환자수가(390원)신설 \\ \tabucline[.5pt]{-}
\rowcolor{Yellow!40} 수지접합, 알코올, 화상, 재활의학, 뇌혈관, 주산기, 유방, 심장 & 390원 가산(20\%) & 사회적 필요 서비스 분 \\ \tabucline[.5pt]{-}
\end{tabu}

\medskip
 
\item 차등지급되는 진찰료(약국 및 한국희귀의약품센터의 경우에는 조제료 등을 말한다)는 차등지수에 1개월(또는 1주일)간 총 진찰료를 승하여 산출하되 10원 미만은 4사5입한 금액으로 산출하며 차등지수는 의사, 치과의사, 한의사, 약사 1인당 1일평균 진찰횟수(약사의 경우에는 조제 건수)를 n으로 할 때에 다음과 같이 산정하되 소수점 여덟째 자리에서 4사5입 한다.
	\begin{enumerate}[(1)]
	\item n이 75 이하일 경우에는 차등지수를 1로 한다.
	\item n이 75를 초과하여 100 이하일 경우에는 \{75×1.00 + (n-75)×0.90\}/n
	\item n이 100을 초과하여 150 이하일 경우에는 \{75×1.00 + 25×0.90 + (n-100)×0.75 \}/n
	\item n이 150을 초과하는 경우에는 \{75×1.00 + 25×0.90 + 50×0.75 + (n-150)×0.50 \}/n
	\end{enumerate}
\item 의사, 치과의사, 한의사 1인당 1일 평균 진찰횟수, 약사 \uline{1인당 1일 평균 조제건수는 내원환자의 순서 및 초\cntrdot{} 재진을 구분하지 아니하고 1개월 (또는 1주일)간 총 진찰(조제)횟수의 합을 구하고} 이를 해당 요양기관이
국민건강보험법 시행규칙 제12조제1항 및 제2항의 규정에 의하여통보한 의사, 치과의사, 한의사가 진료한 총일수, 약국 및 한국희귀 의약품센터의 약사가 조제한 \uline{총일수로 나누어서 계산하되} 소수점 첫째 자리에서 절사하여 산정한다.
\item \uline{진료(조제)일수는 1개월(또는 1주일) 동안 의사(약사)가 실제 진료(조제)한 날수를 말한다.}
\end{enumerate}

\paragraph{Ⅳ. 예외규정}
1. 의료법 제35조에 의한 부속 의료기관은 해당 산정항목에 대하여 공휴\cntrdot{} 야간 가산 등 각종 가산을 산정하지 아니한다.
2. 공무원 및 교직원의 공무상 질병 또는 부상에 대한 요양급여에 소요된 비용의 산정은 산업재해보상보험법 제40조제5항의 규정에 의한 기준에 의한다.
  
\begin{mdframed}[linecolor=blue,middlelinewidth=2]
건강보험요양급여비용제1편 행위 급여 \cntrdot{}  비급여 목록 및 급여 상대가치점수 >> 제2부 행위 급여 목록 상대가치점수및  산정지침 >>  제1장 기본진료료
\end{mdframed}  
  
\subsection{제1장 기본진료료〔산정지침〕}
\begin{enumerate}[1.]\tightlist
\item 진찰료
	\begin{enumerate}[가.]\tightlist
	\item 진찰료는 외래에서 환자를 진찰한 경우에 처방전의 발행과는 관계없이 산정하며 초진환자를 진찰하였을 경우에는 초진진찰료, 재진환자를 진찰하였을 경우에는 재진진찰료를 산정한다.
		\begin{enumerate}[(1)]\tightlist
		\item 진찰료는 기본진찰료(초진의 경우 AA154-AA157은 155.57점, AA100, AA109는 152.11점, 10100은 152.06점, 재진의 경우 AA254-AA257은 98.03점, AA200, AA209, 10200은 95.98점)와
외래관리료(진찰료에서 기본진찰료를 제외한 점수)의 소정점수를 합하여 산정한다.
		\item \uline{초진환자란 해당 상병으로 동일 의료기관의 동일 진료과목 의사에게 진료받은 경험이 없는 환자를 말한다.}
		\item \uline{재진환자란 해당 상병으로 동일 의료기관의 동일 진료과목 의사에게 계속해서 진료받고 있는 환자를 말한다.}
		\item \uline{해당 상병의 치료가 종결되지 아니하여 계속 내원하는 경우에는 내원 간격에 상관없이 재진환자로 본다. 또한, 완치여부가 불분명하여 치료의 종결 여부가 명확하지 아니한 경우 90일 이내에 내원시 재진환자로 본다.}
		\item 해당 상병의 치료가 종결된 후 동일 상병이 재발하여 진료를 받기 위해서 내원한 경우에는 초진환자로 본다. 다만 치료종결 후 30일 이내에 내원한 경우에는 재진환자로 본다.
		\item 치료의 종결이라 함은 해당 상병의 치료를 위한 내원이 종결되었거나, 투약이 종결되었을 때로 본다.
		\item 진찰료 중 기본진찰료는 병원관리 및 진찰권발급 등, 외래관리료는 외래환자의 처방 등에 소요되는 비용을 포함한다.
		\end{enumerate}
	\item 다음 각 호의 1에 해당하는 경우에는 \uline{진찰료는 1회 산정한다.}
		\begin{enumerate}[(1)]\tightlist
		\item \uline{동일 의사가 동시에 2가지 이상의 상병에 대하여 진찰을 한 경우}
		\item 하나의 상병에 대한 진료를 계속 중에 다른 상병이 발생하여 동일 의사가 동시에 진찰을 한 경우(재진진찰료)
		\item \uline{동일한 상병에 대하여 2인 이상의 의사가 동일한 날에 진찰을 한 경우}
		\end{enumerate}
	\item \uline{2개 이상의 진료과목이 설치되어 있고 해당 과의 전문의가 상근하는 요양기관에서 동일환자의 다른 상병에 대하여 전문과목 또는 전문 분야가 다른 진료담당 의사가 각각 진찰한 경우에는 진찰료를 각각 산정할 수 있다.}
	\item \uline{진료담당의사가 검사\cntrdot{} 방사선 진단 등을 처방지시하였으나 요양기관의 사정에 의하여 진료 당일에 검사\cntrdot{} 방사선 진단 등을 실시하지 못한 경우에는 검사\cntrdot{} 방사선 진단을 실시한 당일의 진찰료는 산정하지 아니한다.}
	\item 의료법 제18조에 따라 요양기관인 의료기관의 의사 또는 치과의사가 작성\cntrdot{} 교부한 처방전에 따라 요양기관인 약국 또는 한국희귀의약품센터에서 \uline{조제받은 주사제를 투여받기 위해서 당해 요양기관에 당일에 재내원하는 경우에는 진찰료를 별도 산정하지 아니한다.}
	\end{enumerate}
\item 입원료등(입원료\cntrdot{} 무균치료실입원료\cntrdot{} 낮병동입원료\cntrdot{} 신생아입원료\cntrdot{} 중환자실입원료\cntrdot{} 격리실입원료\cntrdot{} 납차폐특수치료실입원료)
	\begin{enumerate}[가.]
	\item 입원료 등의 소정점수에는 입원환자 의학관리료(소정점수의 40\%),입원환자 간호관리료(소정점수의 25\%), 입원환자 병원관리료(소정점수의 35\%)가 포함되어 있으며 요양기관 종별에 따라 산정한다.
	\item 입원료 등을 산정하기 위해서는 국민건강보험법 제43조 및 동법 시행규칙 제12조에 따라 요양기관의 병실 및 병상 현황을 신고하여야 한다.
	\item 무균치료실입원료, \uline{낮병동입원료, 신생아입원료}, 중환자실입원료,격리실입원료, 납차폐특수치료실입원료 등 \uline{특수병실 입원료를 산정 할 수 있는 경우는 다음과 같으며} 특수병실 입원료를 산정하는 경우에는 입원료 등을 중복하여 산정하지 아니한다.
		\begin{enumerate}[(1)]\tightlist
		\item 무균치료실 입원료:조혈모세포이식환자를 조혈모세포이식의 요양급여에 관한 기준 제3조제2항제1호의 기준에 적합한 무균치료실에격리하여 치료한 경우
		\item \uline{낮병동 입원료}
			\begin{enumerate}[(가)]\tightlist
			\item 다음 각 호의 1에 해당하는 경우 \uline{1) 분만 후 당일 귀가 또는 이송하여 입원료를 산정하지 아니한 경우, 2) 응급실, 수술실 등에서 처치\cntrdot{} 수술 등을 받고 연속하여 6시간 이상,관찰 후 귀가 또는 이송하여 입원료를 산정하지 아니한 경우, 3) 정신건강의학과의 “낮병동”에서 6시간 이상 진료를 받고 당일 귀가한 경우}
			
			\item 낮병동 입원료를 산정하는 당일 외래 또는 응급실에서 진찰을 행한 경우에는 진찰료를 함께 산정할 수 있다. \uline{다만, 예정된 외래 수술을 위해 내원하는 경우 또는 정신건강의학과의 “낮병동”에서 매일 또는 반복하여 진료를 받는 경우에는 진찰료를 산정하지 아니한다.}
			\item \uline{낮병동 입원료를 산정하는 당일의 본인일부부담금은 입원진료 본인일부부담률에 따라 산정한다.}
			\end{enumerate}
		\item \uline{신생아 입원료:질병이 없는 신생아를 신생아실(신생아실 입원료) 또는 모자동실(모자동실입원료)에서 진료\cntrdot{} 간호한 경우}
		\item 중환자실 입원료 :「의료법」시행규칙 제34조 [별표4]에서 정한 중환자실의 시설\cntrdot{} 장비를 갖춘 중환자실(ICU)이 설치된 상급종합 병원, 종합병원, 병원에서 지극히 심각한 질환이나 손상을 입어 집중적인 치료 및 간호가 필요한 성인 또는 소아환자(\uline{일반중환자실 입원료}또는 소아 중환자실 입원료) 또는 신생아(신생아 중환자실 입원료)를 중환자실 에서 진료한 경우
		\item 격리실 입원료:다음 각 호의 1에 해당하는 경우. 다만, 당해 전염성 환자만을 수용하는 요양기관에서는 입원료로 산정한다.
			\begin{enumerate}[(가)]\tightlist
			\item 면역이 억제된 환자를 보호하기 위하여 일반 환자와 격리하여 치료한 경우			\item 일반 환자를 보호하기 위하여 전염력이 강한 전염성 환자를 일반 환자와 격리하여 치료한 경우
			\item 3도 이상으로 36\% 범위 이상의 화상환자를 진료에 반드시 필요하여 격리하여 치료한 경우
			\item 기타 보건복지부장관이 반드시 격리가 필요하다고 인정하여 고시하는 경우
			\end{enumerate}
		\end{enumerate}	
	\end{enumerate}
\end{enumerate}

\Que{HRT 3달치 처방받고 다음에도 계속 HRT처방을 받기위해 방문하시는분은 초진 or 재진 ?}
\Ans{\mycoloredbox{재진}. 해당 상병의 치료가 종결되지 아니하여 계속 내원하는 경우에는 내원 간격에 상관없이 재진환자임}
\Que{HRT시 페경기전후 장애 상병쓰는데, 이것도 만성질환으로 들어가나요? 환자분이 안젤릭 한달 처방받고 그 다음 달에 오셔서(한달 조금 넘어) 초진으로 되었은데 처방내리니 이번 청구에서 재진으로 삭감조정되더라구요.. 그래서 폐경치료도 만성질환으로 들어가는 지 궁금해서요?}
\Ans{재진}
\Que{HRT관리 받으시는 분이 혈액검사상 고지혈증이 관찰되어 치료하는 경우는? 초진 or 재진} \index{조\cntrdot 재진}
\Ans{\mycoloredbox{재진}
\begin{itemize}\tightlist
\item 하나의 상병에 대한 진료를 계속 중에 다른 상병이 발생하여 동일 의사가 동시에 진찰을 한경우
\item 즉 \emph{상병을 두개 넣은 경우네는 재진에 해당}
\item 또는 상병을 고지혈증만 넣는다 해도 30일 이내라면 재진
\end{itemize}
\mycoloredbox{초진}
\begin{itemize}\tightlist
\item 고지혈증 \emph{상병만 넣고} N951 상병으로 진찰 받은지 한달이 넘은 경우.
\end{itemize}}

\Que{2월 1일 질염환자가 3월1일에 다시 질염으로 재발되어 왔다면? 초진 or 재진}
\Ans{\mycoloredbox{초진}. 해당 상병의 \emph{치료가 종결된 후 동일 상병이 재발하여 진료를 받기 위해서 내원한 경우에는 초진후 30일 이내에 내원한 경우에는 재진환자}로 본다. 치료의 종결이라 함음 해당 상병의 치료를 위한 내우언이 종경되었거나, 투약이 종결되었을때를 말함}

\Que{한달 이내에 트리코모나스 질염과 캔디다 질염으로 각각 다른 상병으로 내원하면? 초진 or 재진}
\Ans{\mycoloredbox{재진.}하나의 상병에 대한 진료를 계속중에 다른 상병이 발생하여 동일 의사가 동시에 진찰을 한 경우(재진진찰료)}

\Que{엄청나 산부인과의 엄청난 선생님에게 진료를 본 환자가 한달이 채 안되어서 같은 산부인과 더엄청난 선생님이 다른 상병으로 진료 받는 경우?}
\Ans{\mycoloredbox{초진}
\begin{itemize}\tightlist
\item 재진의 조건 : 같은 진료과목, 같은 병원이지만 같은 상병이 아니므로 초진
\item 초재진 구분의 3대 기준 
	\begin{enumerate}\tightlist
	\item 상병이 같으냐? 다르냐?
	\item 같은 병원이냐? 아니냐?
	\item 같은 진료과목 의사이냐? 아니냐?[같은 선생님이냐? 아니냐가 아닙니다]
	\end{enumerate}
\end{itemize}}

\begin{description}\tightlist	
\item[가-2] 입원료 Inpatient Care\footnote{주 : 1. 내과질환자, 정신질환자, 만8세 미만의 소아환자에 대하여는 소정점수의 30\%를 가산(산정코드 세 번째 자리에 4로 기재)한다.(주2에 해당하는 경우 제외) 2. 강내치료를 위하여 밀봉소선원치료실에 입원한 경우에는 3일 이내의 기간 동안 소정점수의 100\%를 가산한다.(산정코드 세 번째 자리에 3으로 기재)}
	\begin{enumerate}[가.]\tightlist
	\item 기본입원료 
	
	\medskip
	\tabulinesep =_2mm^2mm
	\begin{tabu} to\linewidth {|X[2,l]|X[6,l]|X[1,l]|X[1,l]|} \tabucline[.5pt]{-}
	\rowcolor{ForestGreen!40}  코드 &	\centering 분 류 & 점수 & 금액 \\ \tabucline[.5pt]{-}
	\rowcolor{Yellow!40} AB100(15100) & (1) 상급종합병원 & 522.27 & \myexplfng{522.27} \\ \tabucline[.5pt]{-}
	\rowcolor{Yellow!40} AB200(15200)  & (2) 종합병원 & 480.64 & \myexplfng{480.64} \\ \tabucline[.5pt]{-}
	\rowcolor{Yellow!40} AB300 & (3) 병원, 치과병원, 한방병원 내 치\cntrdot{}의과 & 421.01 & \myexplfng{421.01} \\ \tabucline[.5pt]{-}
	\rowcolor{Yellow!40} 15300 & (4) 한방병원, 병원\cntrdot{}치과병원 내 한의과 & 417.36 &  \myexplfng{417.36} \\ \tabucline[.5pt]{-}
	\rowcolor{Yellow!40} AB400 & (5) 의원, 치과의원, 보건의료원 치\cntrdot{}의과 & 358.86 & \myexplfng{358.86} \\ \tabucline[.5pt]{-}
	\rowcolor{Yellow!40} 15400 & (6) 한의원, 보건의료원 한방과 & 355.29 & \myexplfng{355.29} \\ \tabucline[.5pt]{-}
	\end{tabu}
	
	\item 5인실 입원료 
	
	\medskip
	\tabulinesep =_2mm^2mm
	\begin{tabu} to\linewidth {|X[2,l]|X[6,l]|X[1,l]|X[1,l]|} \tabucline[.5pt]{-}
	\rowcolor{ForestGreen!40}  코드 &	\centering 분 류 & 점수 & 금액 \\ \tabucline[.5pt]{-}
	\rowcolor{Yellow!40} AB120(15120) & (1) 상급종합병원 & 678.95 & \myexplfng{678.95}  \\ \tabucline[.5pt]{-}
	\rowcolor{Yellow!40} AB220(15220) & (2) 종합병원 & 624.83 & \myexplfng{624.83} \\ \tabucline[.5pt]{-}
	\rowcolor{Yellow!40} AB320 & (3) 병원, 치과병원, 한방병원 내 치\cntrdot{}의과 & 547.31 & \myexplfng{547.31} \\ \tabucline[.5pt]{-}
	\rowcolor{Yellow!40} 15320 & (4) 한방병원, 병원\cntrdot{}치과병원 내 한의원 & 542.57 & \myexplfng{542.57} \\ \tabucline[.5pt]{-}
	\rowcolor{Yellow!40} AB420 & (5) 의원, 치과의원, 보건의료원 치\cntrdot{}의과 & 466.52 & \myexplfng{466.52} \\ \tabucline[.5pt]{-}
	\rowcolor{Yellow!40} 15420 & (6) 한의원, 보건의료원 한방과 & 61.88 & \myexplfng{61.88} \\ \tabucline[.5pt]{-}
	\end{tabu}
	
	\item 4인실 입원료 
	
	\medskip
	\tabulinesep =_2mm^2mm
	\begin{tabu} to\linewidth {|X[2,l]|X[6,l]|X[1,l]|X[1,l]|} \tabucline[.5pt]{-}
	\rowcolor{ForestGreen!40}  코드 &	\centering 분 류 & 점수 & 금액 \\ \tabucline[.5pt]{-}	
	\rowcolor{Yellow!40} AB140(15140) & (1) 상급종합병원 & 835.63 & \myexplfng{835.63} \\ \tabucline[.5pt]{-}
	\rowcolor{Yellow!40} AB240(15240) & (2) 종합병원 & 769.02 & \myexplfng{769.02} \\ \tabucline[.5pt]{-}
	\rowcolor{Yellow!40} AB340 & (3) 병원, 치과병원, 한방병원 내 치\cntrdot{}의과 & 673.62 & \myexplfng{673.62} \\ \tabucline[.5pt]{-}
	\rowcolor{Yellow!40} 15340 & (4) 한방병원, 병원\cntrdot{}치과병원 내 한의과 & 667.78 & \myexplfng{667.78} \\ \tabucline[.5pt]{-}
	\rowcolor{Yellow!40} AB440 & (5) 의원, 치과의원, 보건의료원 치\cntrdot{}의과 & 574.18 & \myexplfng{574.18} \\ \tabucline[.5pt]{-}
	\rowcolor{Yellow!40} 15440 & (6) 한의원, 보건의료원 한방과 & 568.46 & \myexplfng{568.46} \\ \tabucline[.5pt]{-}
	\end{tabu}
	\end{enumerate}

\item[가-6] 낮병동 입원료 Day Care \par
\tabulinesep =_2mm^2mm
\begin{tabu} to\linewidth {|X[2,l]|X[6,l]|X[1,l]|X[1,l]|} \tabucline[.5pt]{-}
\rowcolor{ForestGreen!40}  코드 &	\centering 분 류 & 점수 & 금액 \\ \tabucline[.5pt]{-}
\rowcolor{Yellow!40} AF100(18100) & 가. 상급종합병원 & 522.27 & \myexplfng{522.27} \\ \tabucline[.5pt]{-}
\rowcolor{Yellow!40} AF200(18200) & 나. 종합병원 & 480.64 & \myexplfng{480.64} \\ \tabucline[.5pt]{-}
\rowcolor{Yellow!40} AF300 & 다. 병원, 치과병원, 한방병원 내 의·치과 & 421.01 & \myexplfng{421.01} \\ \tabucline[.5pt]{-}
\rowcolor{Yellow!40} 18300 & 라. 한방병원, 병원·치과병원 내 한의과 & 417.36 & \myexplfng{417.36} \\ \tabucline[.5pt]{-}
\rowcolor{Yellow!40} AF400 & 마. 의원, 치과의원, 보건의료원 의·치과 & 358.86 & \myexplfng{358.86} \\ \tabucline[.5pt]{-}
\rowcolor{Yellow!40} 18400 & 바. 한의원, 보건의료원 내 한의원 & 355.29 & \myexplfng{355.29} \\ \tabucline[.5pt]{-}
\end{tabu}
%\myexplfng{ } \\ \tabucline[.5pt]{-} %
\item[가-7] 신생아 입원료 Neonatal Care \footnote{주:신생아제대처치, 기저귀 교환, 혈압, 맥박, 호흡 측정, 목욕 등의 비용과 기저귀 비용이 포함되어 있으므로 그 비용을 별도 산정하지 아니한다.}
	\begin{enumerate}[가.]\tightlist
	\item 신생아실 입원료\footnote{주:질병이 없는 신생아를 신생아실에서 진료\cntrdot{} 간호한 경우에 산정한다.} 
	
	\medskip
	\tabulinesep =_2mm^2mm
	\begin{tabu} to\linewidth {|X[2,l]|X[6,l]|X[1,l]|X[1,l]|} \tabucline[.5pt]{-}
	\rowcolor{ForestGreen!40}  코드 &	\centering 분 류 & 점수 & 금액 \\ \tabucline[.5pt]{-}
	\rowcolor{Yellow!40} AG111 & (1) 상급종합병원 & 856.41 & \myexplfng{856.41} \\ \tabucline[.5pt]{-} %59,950 \\ \tabucline[.5pt]{-}	
	\rowcolor{Yellow!40} AG211 & (2) 종합병원 & 789.87 & \myexplfng{789.87} \\ \tabucline[.5pt]{-} %55,290 \\ \tabucline[.5pt]{-}
	\rowcolor{Yellow!40} AG311 & (3) 병원, 치과병원\cntrdot{} 한방병원 내 의과 & 481.46 & \myexplfng{481.46} \\ \tabucline[.5pt]{-} %33,700 \\ \tabucline[.5pt]{-}
	\rowcolor{Yellow!40} AG411 & (4) 의원, 보건의료원 의과 & 448.94 & \myexplfng{448.94} \\ \tabucline[.5pt]{-} %33,400 \\ \tabucline[.5pt]{-}
	\end{tabu}
	
	\item 모자동실 입원료\footnote{주:질병이 없는 신생아를 모자동실에서 진료·간호한 경우에 산정한다.} 
	
	\medskip
	\tabulinesep =_2mm^2mm
	\begin{tabu} to\linewidth {|X[2,l]|X[6,l]|X[1,l]|X[1,l]|} \tabucline[.5pt]{-}
	\rowcolor{ForestGreen!40}  코드 &	\centering 분 류 & 점수 & 금액 \\ \tabucline[.5pt]{-}	
	\rowcolor{Yellow!40} AG112 & (1) 상급종합병원 & 1,145.19 & \myexplfng{1145.19} \\ \tabucline[.5pt]{-} %80,160 \\ \tabucline[.5pt]{-}
	\rowcolor{Yellow!40} AG212 & (2) 종합병원 & 1,061.00 & \myexplfng{1061.00} \\ \tabucline[.5pt]{-} %74,270 \\ \tabucline[.5pt]{-}
	\rowcolor{Yellow!40} AG312 & (3) 병원, 치과병원\cntrdot{} 한방병원 내 의과 & 625.94 & \myexplfng{625.94} \\ \tabucline[.5pt]{-} %43,820 \\ \tabucline[.5pt]{-}
	\rowcolor{Yellow!40} AG412 & (4) 의원, 보건의료원 의과 & 572.60 & \myexplfng{572.60} \\ \tabucline[.5pt]{-} %42,600 \\ \tabucline[.5pt]{-}
	\end{tabu}
	
	\item 신생아 모유수유간호관리료\footnote{주:「가」 또는 「나」를 산정하는 신생아에게 모유수유를 한 경우에 산정한다.} 
	
	\medskip
	\tabulinesep =_2mm^2mm
	\begin{tabu} to\linewidth {|X[2,l]|X[6,l]|X[1,l]|X[1,l]|} \tabucline[.5pt]{-}
	\rowcolor{ForestGreen!40}  코드 &	\centering 분 류 & 점수 & 금액 \\ \tabucline[.5pt]{-}	
	\rowcolor{Yellow!40} AG113 & (1) 상급종합병원 & 425.57 & \myexplfng{425.57} \\ \tabucline[.5pt]{-} %29,790 \\ \tabucline[.5pt]{-}
	\rowcolor{Yellow!40} AG213 & (2) 종합병원 & 374.84 & \myexplfng{374.84} \\ \tabucline[.5pt]{-} %26,240 \\ \tabucline[.5pt]{-}
	\rowcolor{Yellow!40} AG313 & (3) 병원, 치과병원\cntrdot{} 한방병원 내 의과 & 213.46 & \myexplfng{213.46} \\ \tabucline[.5pt]{-} %14,940 \\ \tabucline[.5pt]{-}
	\rowcolor{Yellow!40} AG413 & (4) 의원, 보건의료원 의과 & 187.53 & \myexplfng{187.53} \\ \tabucline[.5pt]{-} %13,950 \\ \tabucline[.5pt]{-}
	\end{tabu}
	\end{enumerate}

\item[가-8] 협의진찰료 Consultation \footnote{주: 「의료법」 제47조에 의한 감염관리위원회 및 감염관리실을 설치·운영하는 요양기관에서 감염전문관리를 실시한 경우에도 소정점수를 산정한다. (기본코드 다섯 번째 자리에 2로 기재)} 
	\begin{enumerate}[가.]\tightlist
	\item 상급종합병원, 상급종합병원에 설치된 치과대학부속치과병원 
	
	\medskip
	\tabulinesep =_2mm^2mm
	\begin{tabu} to\linewidth {|X[2,l]|X[6,l]|X[1,l]|X[1,l]|} \tabucline[.5pt]{-}
	\rowcolor{ForestGreen!40}  코드 &	\centering 분 류 & 점수 & 금액 \\ \tabucline[.5pt]{-}		
	\rowcolor{Yellow!40} AH500 & (1) 의과, 치과 & 155.57 & \myexplfng{155.57} \\ \tabucline[.5pt]{-} %10,890 \\ \tabucline[.5pt]{-}
	\rowcolor{Yellow!40} 11500 & (2) 한의과 & 150.05 & \myexplfng{150.05} \\ \tabucline[.5pt]{-} % \\ \tabucline[.5pt]{-}
	\end{tabu}
	
	\item 종합병원, 상급종합병원에 설치된 경우를 제외한 치과대학부속치과병원 
	
	\medskip
	\tabulinesep =_2mm^2mm
	\begin{tabu} to\linewidth {|X[2,l]|X[6,l]|X[1,l]|X[1,l]|} \tabucline[.5pt]{-}
	\rowcolor{ForestGreen!40}  코드 &	\centering 분 류 & 점수 & 금액 \\ \tabucline[.5pt]{-}		
	\rowcolor{Yellow!40} AH600 & (1) 의과, 치과 & 141.25 & \myexplfng{141.25} \\ \tabucline[.5pt]{-} %9,890 \\ \tabucline[.5pt]{-}
	\rowcolor{Yellow!40} 11600 & (2) 한의과 & 136.24 &  \myexplfng{136.24} \\ \tabucline[.5pt]{-} %\\ \tabucline[.5pt]{-}
	\end{tabu}
	
	\item 병원, 한방병원, 치과병원 
	
	\medskip
	\tabulinesep =_2mm^2mm
	\begin{tabu} to\linewidth {|X[2,l]|X[6,l]|X[1,l]|X[1,l]|} \tabucline[.5pt]{-}
	\rowcolor{ForestGreen!40}  코드 &	\centering 분 류 & 점수 & 금액 \\ \tabucline[.5pt]{-}	
	\rowcolor{Yellow!40} AH700 & (1) 의과, 치과 & 127.02 & \myexplfng{127.02} \\ \tabucline[.5pt]{-} %8,890 \\ \tabucline[.5pt]{-}
	\rowcolor{Yellow!40} 11700 & (2) 한의과  & 122.51 &  \myexplfng{122.51} \\ \tabucline[.5pt]{-} %\\ \tabucline[.5pt]{-}
	\end{tabu}
	
	\item 요양병원, 보건의료원 
	
	\medskip
	\tabulinesep =_2mm^2mm
	\begin{tabu} to\linewidth {|X[2,l]|X[6,l]|X[1,l]|X[1,l]|} \tabucline[.5pt]{-}
	\rowcolor{ForestGreen!40}  코드 &	\centering 분 류 & 점수 & 금액 \\ \tabucline[.5pt]{-}	
	\rowcolor{Yellow!40} AH800 & (1) 의과, 치과 & 69.63 & \myexplfng{69.63} \\ \tabucline[.5pt]{-} %4,870원 \\ \tabucline[.5pt]{-}
	\rowcolor{Yellow!40} 11800 & (2) 한의과  & 67.16 & \myexplfng{67.16} \\ \tabucline[.5pt]{-} %  \\ \tabucline[.5pt]{-}
	\rowcolor{Yellow!40} AH900 & 마. 의원, 치과의원 & 69.63 & \myexplfng{69.63} \\ \tabucline[.5pt]{-} %5,180 \\ \tabucline[.5pt]{-}
	\rowcolor{Yellow!40} 11900 & 바. 한의원 & 67.16 & \myexplfng{67.16} \\ \tabucline[.5pt]{-} % \\ \tabucline[.5pt]{-}
	\end{tabu}
  	\end{enumerate}
  
\item[가-8-1] 집중영양치료료 Therapy by Nutrition Support Team 

\medskip
\tabulinesep =_2mm^2mm
\begin{tabu} to\linewidth {|X[2,l]|X[6,l]|X[1,l]|X[1,l]|} \tabucline[.5pt]{-}
\rowcolor{ForestGreen!40}  코드 &	\centering 분 류 & 점수 & 금액 \\ \tabucline[.5pt]{-}	
\rowcolor{Yellow!40} AI600 & 가. 상급종합병원 & 535.87 & \myexplfng{535.87} \\ \tabucline[.5pt]{-} %37,510 \\ \tabucline[.5pt]{-}
\rowcolor{Yellow!40} AI700 & 나. 종합병원 & 402.57 & \myexplfng{402.57} \\ \tabucline[.5pt]{-} %28,180 \\ \tabucline[.5pt]{-}
\end{tabu}

\item[가-9] AJ001(19001) 중환자실 입원료 ICU Patient Care
	\begin{enumerate}[가.]\tightlist
	\item 성인 또는 소아 중환자실 입원료 Adult or Pediatric \footnote{주:1.일반 중환자실에 전담의를 두는경우에는 272.06점을 별도 산정한다.} : 2015년 9월 신설 
	
	\medskip
	\tabulinesep =_2mm^2mm
	\begin{tabu} to\linewidth {|X[2,l]|X[6,l]|X[2,l]|} \tabucline[.5pt]{-}
	\rowcolor{ForestGreen!40}  코드 &	\centering 분 류 & 점수  \\ \tabucline[.5pt]{-}	
	\rowcolor{Yellow!40} AJ003 (19003) & 2. 중환자실 1Unit당 1인 이상의 전문의를 포함하여 전담의를 두는 경우에는  & 421.71점을 별도 산정한다.  \\ \tabucline[.5pt]{-}
	\rowcolor{Yellow!40} AJ100 (19400) & (1) 상급종합병원 & 2,648.30점   \\ \tabucline[.5pt]{-}
	\rowcolor{Yellow!40} AJ200 (19200) & (2) 종합병원 & 1,545.86점 \\ \tabucline[.5pt]{-}
	\rowcolor{Yellow!40} AJ300 & (3) 병원, 치과병원, 한방병원 내 의·치과 & 1,133.68점   \\ \tabucline[.5pt]{-}
	\rowcolor{Yellow!40} 19300 & (4) 한방병원, 병원·치과병원 내 한의과 & 1,128.39점  \\ \tabucline[.5pt]{-}
	\end{tabu}
	
	\item 신생아 중환자실 입원료 Neonatal \footnote{주:신생아 중환자실에는 전담전문의를 두어야 한다.}(2015년 9월 신설) 
	
	\medskip
	\tabulinesep =_2mm^2mm
	\begin{tabu} to\linewidth {|X[2,l]|X[6,l]|X[1,l]|X[1,l]|} \tabucline[.5pt]{-}
	\rowcolor{ForestGreen!40}  코드 &	\centering 분 류 & 점수 & 금액 \\ \tabucline[.5pt]{-}	
	\rowcolor{Yellow!40} AJ101 & (1) 상급종합병원 & 4,074.04 & \myexplfng{4074.04} \\ \tabucline[.5pt]{-} %285,180 \\ \tabucline[.5pt]{-}
	\rowcolor{Yellow!40} AJ201 & (2) 종합병원 & 3,755.24 & \myexplfng{3755.24} \\ \tabucline[.5pt]{-} %262,870 \\ \tabucline[.5pt]{-}
	\rowcolor{Yellow!40} AJ301 & (3) 병원, 치과병원\cntrdot{} 한방병원 내 의과 & 3,025.80 & \myexplfng{3025.80} \\ \tabucline[.5pt]{-} %211,810 \\ \tabucline[.5pt]{-}
	\end{tabu}
	
	\item 소아 중환자실 입원료 Pediatric \footnote{주 : 1. 별도의 Unit으로 운영하는 소아 중환자실에서 만18세 미만 소아청소년을 입원 치료한 경우에 산정한다. } 
	
	\medskip
	\tabulinesep =_2mm^2mm
	\begin{tabu} to\linewidth {|X[2,l]|X[6,l]|X[2,l]|} \tabucline[.5pt]{-}
	\rowcolor{ForestGreen!40}  코드 &	\centering 분 류 & 점수  \\ \tabucline[.5pt]{-}	
	\rowcolor{Yellow!40} AJ004 (19004) & 2. 중환자실 1Unit 당 1인 이상의 전담의를 두는 경우에는 & 272.06 점을 별도 산정한다.  \\ \tabucline[.5pt]{-}	
	\rowcolor{Yellow!40} AJ005 (19005) & 3. 중환자실 1Unit 당 1인 이상의 전문의를 포함하여 전담의를 두는 경우에는 & 421.71점을 별도산정한다.  \\ \tabucline[.5pt]{-}	
	\rowcolor{Yellow!40} AJ102 (19402) & (1) 상급종합병원 &  3,442.79점  \\ \tabucline[.5pt]{-}	
	\rowcolor{Yellow!40} AJ202 (19202) & (2) 종합병원 &  2,311.06점  \\ \tabucline[.5pt]{-}	
	\rowcolor{Yellow!40} AJ302 & (3) 병원, 치과병원, 한방병원 내 의.치과 &  1,694.85점 \\ \tabucline[.5pt]{-}	
	\rowcolor{Yellow!40} 19302 & (4) 한방병원, 병원·치과병원 내 한의과 &  1,686.95점   \\ \tabucline[.5pt]{-}	
	\end{tabu}
	\end{enumerate}

\item[가-10] 격리실 입원료 Isolation Room Patient Care
	\begin{enumerate}[가.]\tightlist
	\item 일반 격리실 입원료
		\begin{enumerate}[(1)]\tightlist
		\item 상급종합병원 
		
		\medskip
		\tabulinesep =_2mm^2mm
		\begin{tabu} to\linewidth {|X[2,l]|X[6,l]|X[1,l]|X[1,l]|} \tabucline[.5pt]{-}
		\rowcolor{ForestGreen!40}  코드 &	\centering 분 류 & 점수 & 금액 \\ \tabucline[.5pt]{-}	
		\rowcolor{Yellow!40} AK100 & (가) 1인용 & 3,048.78 & \myexplfng{3048.78} \\ \tabucline[.5pt]{-} %213,410 \\ \tabucline[.5pt]{-}
		\rowcolor{Yellow!40} AK101 & (나) 다인용 & 1,219.51 & \myexplfng{1219.51} \\ \tabucline[.5pt]{-} %85,370 \\ \tabucline[.5pt]{-}
		\end{tabu}
		
		\item 종합병원 
		
		\medskip
		\tabulinesep =_2mm^2mm
		\begin{tabu} to\linewidth {|X[2,l]|X[6,l]|X[1,l]|X[1,l]|} \tabucline[.5pt]{-}
		\rowcolor{ForestGreen!40}  코드 &	\centering 분 류 & 점수 & 금액 \\ \tabucline[.5pt]{-}	
		\rowcolor{Yellow!40} AK200 & (가) 1인용 & 2,236.00 & \myexplfng{2236.00} \\ \tabucline[.5pt]{-} %156,520 \\ \tabucline[.5pt]{-}
		\rowcolor{Yellow!40} AK201 & (나) 다인용 & 1,118.00 & \myexplfng{1118.00 } \\ \tabucline[.5pt]{-} %78,260 \\ \tabucline[.5pt]{-}
		\end{tabu}
		
		\item 병원, 치과병원\cntrdot{} 한방병원 내 치\cntrdot{}의과 
		
		\medskip
		\tabulinesep =_2mm^2mm
		\begin{tabu} to\linewidth {|X[2,l]|X[6,l]|X[1,l]|X[1,l]|} \tabucline[.5pt]{-}
		\rowcolor{ForestGreen!40}  코드 &	\centering 분 류 & 점수 & 금액 \\ \tabucline[.5pt]{-}	
		\rowcolor{Yellow!40} AK300 & (가) 1인용 & 1,350.72 & \myexplfng{1350.72} \\ \tabucline[.5pt]{-} %94,550 \\ \tabucline[.5pt]{-}
		\rowcolor{Yellow!40} AK301 & (나) 다인용 & 900.48 & \myexplfng{900.48} \\ \tabucline[.5pt]{-} %63,030 \\ \tabucline[.5pt]{-}
		\end{tabu}
		
		\item 의원, 치과의원, 보건의료원 치\cntrdot{}의과 
		
		\medskip
		\tabulinesep =_2mm^2mm
		\begin{tabu} to\linewidth {|X[2,l]|X[6,l]|X[1,l]|X[1,l]|} \tabucline[.5pt]{-}
		\rowcolor{ForestGreen!40}  코드 &	\centering 분 류 & 점수 & 금액 \\ \tabucline[.5pt]{-}	
		\rowcolor{Yellow!40} AK400 & (가) 1인용 & 1,169.45 & \myexplfng{1169.45} \\ \tabucline[.5pt]{-} %87,010 \\ \tabucline[.5pt]{-}
		\rowcolor{Yellow!40} AK401 & (나) 다인용 & 779.63 & \myexplfng{779.63} \\ \tabucline[.5pt]{-} %58,000 \\ \tabucline[.5pt]{-}
		\end{tabu}
  		\end{enumerate}
  		
	\item 음압 격리실 입원료 \par
		\begin{enumerate}[(1)]\tightlist
		\item 상급종합병원 
		
		\medskip
		\tabulinesep =_2mm^2mm
		\begin{tabu} to\linewidth {|X[2,l]|X[6,l]|X[1,l]|X[1,l]|} \tabucline[.5pt]{-}
		\rowcolor{ForestGreen!40}  코드 &	\centering 분 류 & 점수 & 금액 \\ \tabucline[.5pt]{-}	
		\rowcolor{Yellow!40} AK110 & (가) 1인용 & 4,573.18 & \myexplfng{4573.18} \\ \tabucline[.5pt]{-} %320,120 \\ \tabucline[.5pt]{-}
		\rowcolor{Yellow!40} AK111 & (나) 다인용 & 1,829.27 & \myexplfng{1829.27} \\ \tabucline[.5pt]{-} %128,050 \\ \tabucline[.5pt]{-}
		\end{tabu}
		
		\item 종합병원 
		
		\medskip
		\tabulinesep =_2mm^2mm
		\begin{tabu} to\linewidth {|X[2,l]|X[6,l]|X[1,l]|X[1,l]|} \tabucline[.5pt]{-}
		\rowcolor{ForestGreen!40}  코드 &	\centering 분 류 & 점수 & 금액 \\ \tabucline[.5pt]{-}			
		\rowcolor{Yellow!40} AK210 & (가) 1인용 & 2,683.20 & \myexplfng{2683.20} \\ \tabucline[.5pt]{-} %187,820 \\ \tabucline[.5pt]{-}
		\rowcolor{Yellow!40} AK211 & (나) 다인용 & 1,341.60 & \myexplfng{1341.60} \\ \tabucline[.5pt]{-} %93,910 \\ \tabucline[.5pt]{-}
		\end{tabu}
		
		\item 병원, 치과병원\cntrdot{} 한방병원 내 치\cntrdot{}의과 
		
		\medskip
		\tabulinesep =_2mm^2mm
		\begin{tabu} to\linewidth {|X[2,l]|X[6,l]|X[1,l]|X[1,l]|} \tabucline[.5pt]{-}
		\rowcolor{ForestGreen!40}  코드 &	\centering 분 류 & 점수 & 금액 \\ \tabucline[.5pt]{-}			
		\rowcolor{Yellow!40} AK310 & (가) 1인용 & 1,485.80 & \myexplfng{1485.80} \\ \tabucline[.5pt]{-} %104,010 \\ \tabucline[.5pt]{-}
		\rowcolor{Yellow!40} AK311 & (나) 다인용 & 990.53 & \myexplfng{990.53} \\ \tabucline[.5pt]{-} %69,340 \\ \tabucline[.5pt]{-}
		\end{tabu}
		
		\item 의원, 치과의원, 보건의료원 치\cntrdot{}의과 
		
		\medskip
		\tabulinesep =_2mm^2mm
		\begin{tabu} to\linewidth {|X[2,l]|X[6,l]|X[1,l]|X[1,l]|} \tabucline[.5pt]{-}
		\rowcolor{ForestGreen!40}  코드 &	\centering 분 류 & 점수 & 금액 \\ \tabucline[.5pt]{-}			
		\rowcolor{Yellow!40} AK410 & (가) 1인용 & 1,286.39 & \myexplfng{1286.39 } \\ \tabucline[.5pt]{-} %95,710 \\ \tabucline[.5pt]{-}
		\rowcolor{Yellow!40} AK411 & (나) 다인용 & 857.59 & \myexplfng{857.59} \\ \tabucline[.5pt]{-} %63,800 \\ \tabucline[.5pt]{-}
		\end{tabu}
		\end{enumerate}
  	\end{enumerate}
  	
\item[가-10-1] 납차폐특수치료실 입원료(Lead-Shielded Room Patient Care) \footnote{주:방사성옥소를 이용한 개봉선원치료를 위하여 원자력 안전법령에 의한 시설을 갖춘 요양기관에서 납으로 차폐된 특수치료실에서 관리하는 경우 산정한다. (2015년 9월 수가변경)} 

\medskip
\tabulinesep =_2mm^2mm
\begin{tabu} to\linewidth {|X[2,l]|X[6,l]|X[2,l]|} \tabucline[.5pt]{-}
\rowcolor{ForestGreen!40}  코드 &	\centering 분 류 & 점수  \\ \tabucline[.5pt]{-}	
\rowcolor{Yellow!40} AQ600 & 가. 상급종합병원 & 3,895.58점   \\ \tabucline[.5pt]{-}
\rowcolor{Yellow!40} AQ700 & 나. 종합병원 & 3,561.64점   \\ \tabucline[.5pt]{-}
\rowcolor{Yellow!40} AQ800 & 다. 병원, 치과병원, 한방병원 내 의·치과 &  1,833.17점   \\ \tabucline[.5pt]{-}
\rowcolor{Yellow!40} AQ900 & 라. 의원, 치과의원, 보건의료원 내 의·치과 & 1,571.26점   \\ \tabucline[.5pt]{-}
\end{tabu}

\item[가-17] 회복관리료 Fee of Postanesthesia Care(2015년 9월 신설) 

\medskip
\tabulinesep =_2mm^2mm
\begin{tabu} to\linewidth {|X[2,l]|X[6,l]|X[2,l]|} \tabucline[.5pt]{-}
\rowcolor{ForestGreen!40}  코드 &	\centering 분 류 & 점수  \\ \tabucline[.5pt]{-}	
\rowcolor{Yellow!40} AP501 & 가. 상급종합병원 & 313.36점   \\ \tabucline[.5pt]{-}
\rowcolor{Yellow!40} AP601 & 나. 종합병원 & 288.38점  \\ \tabucline[.5pt]{-}
\rowcolor{Yellow!40} AP701 & 다. 병원, 치과병원, 요양병원, 한방병원 & 252.61점  \\ \tabucline[.5pt]{-}
\rowcolor{Yellow!40} AP801 & 라. 의원, 치과의원, 보건의료원 & 215.32점   \\ \tabucline[.5pt]{-}
\end{tabu}

	\begin{Cdoing}{가17 회복관리료(Fee of Postanesthesia Care) 인정기준(2015년 9월 신설)}
회복관리료는 아래와 같은 요건을 모두 충족한 회복실에서 회복관리를 시행한 경우 인정함
	\begin{enumerate}[가.]\tightlist
	\item 산정기준
		\begin{enumerate}[(1)]\tightlist
		\item 인력
			\begin{enumerate}[(가)]\tightlist
			\item 회복실의 회복관찰 업무를 총괄하는 상근하는 마취통증의학과 전문의가 1인 이상
			\item 회복실 내 환자 회복관리 업무만을 전담하는 간호사가 2인 이상 (정규직 전일제 근무 간호사로 1주간의 근로시간이 월평균 40시간인 근무자를 말함)
			\end{enumerate}
		\item 장비
			\begin{enumerate}[(가)]\tightlist
			\item 회복실내에 반드시 갖추어야 하는 장비 - 병상당 기본시설(산소공급장치, 흡인기), - 모니터링 장비: 말초산소포화도측정기, 심전도감시기, 비침습적 혈압측정기, 호기말이산화탄소분압감시기, - 체온조절기, - 호흡보조 장비 등(Nasal prong, Facial Mask, Ambu bag set), - 응급장비(기도삽관 장비 일체) 
			\item 필요시 즉시 사용가능하도록 수술실 또는 회복실에 갖추어야 하는 장비 - Emergency Cart, - 인공호흡기, - 제세동기
			\end{enumerate}
		\end{enumerate}
	\item 산정대상 : 바2가(1) 기관내 삽관에 의한 폐쇄순환식 전신마취 또는 바2가(2) 마스크에 의한 폐쇄순환식 전신마취 후 회복관리만을 목적으로 별도로 설치된 회복실에서 15분 이상 집중 회복관리를 한 경우
	\item 기타 : 회복관리가 종료되기 전에 출혈 등의 이유로 재수술 후 회복실에 다시 입실하여 회복관리가 이루어진 경우에는 회복관리료는 1회만 산정함
	\end{enumerate}
	\end{Cdoing}
	
\end{description}
\begin{hemphsentense}{3대 비급여 보장성 강화 (비급여 축소)에 따른 풍선 수가신설(보존책)}
\begin{enumerate}[1)]\tightlist
\item 2015년 9월부터 선택의사 지정비율을 80\% \MVRightarrow 67\%(2015년) \MVRightarrow 33\%(2016년) \MVRightarrow 비급여 선택진료 폐지, 건강보험적용(2017년)
\item 2014년 9월부터 4인실과 5인실 건강보험 적용 : 2015년부터 대형병원 의무 병상을 50\% \MVRightarrow 70\%(다만, 종합병원중 산부인과 전문병원은 현행 50\% 유지)
\end{enumerate}
\end{hemphsentense}

\begin{mdframed}[linecolor=blue,middlelinewidth=2]  
제1편 행위 급여 ․ 비급여 목록 및 급여 상대가치점수 >> 제1부 행위 급여 일반원칙 >>  5.의료질평가지원금(2015년 9월 신설)
\end{mdframed}
\begin{enumerate}[5.]\tightlist
\item  \uline{의료질평가지원금(2015년 9월 신설)}
	\begin{enumerate}[가.]\tightlist
	\item \uline{「의료질평가지원금 산정을 위한 기준」의 평가결과에 따라 상급종합병원, 종합병원에 한하여} 3개 분야(의료 질과 환자안전․공공성․의료전달체계 분야, 교육수련 분야, 연구개발 분야)별 \uline{최종등급에 해당하는 소정점수를 산정한다. }
	\item 입원진료의 경우에는 각 분야의 등급별 의료질평가지원금을 아래 항목의 산정횟수와 동일하게 산정한다. 다만, 입원료 중 병원관리료만을 산정하는 경우에는 제외한다.
		\begin{enumerate}[(1)]\tightlist
		\item 입원료(가-2)		\item 무균치료실 입원료(가-4)
		\item 낮병동 입원료(가-6)
		\item 신생아 입원료(가-7가, 가-7나)
		\item 중환자실 입원료(가-9), 다만 AJ001, AJ003, AJ004, AJ005, 19001, 19003, 19004, 19005는 제외
		\item 격리실 입원료(가-10)
		\item 납차폐특수치료실 입원료(가-10-1)
		\end{enumerate}
	\item 외래진료의 경우에는 각 분야의 등급별 의료질평가지원금을 외래환자 진찰료(가-1)의 산정횟수와 동일하게 산정하되, 재진진찰료(가-1나)의 “주6” 및 “주8”은 제외한다.
	\end{enumerate}  
\end{enumerate}

\begin{mdframed}[linecolor=blue,middlelinewidth=2]  
제1편 행위 급여 \cntrdot{}  비급여 목록 및 급여 상대가치점수 >> 제2부 행위 급여 목록\cntrdot{} 상대가치점수 및 산정지침 >> 제2장 검사료
\end{mdframed}
\subsection{\newindex{검사료〔산정지침〕}}
\begin{enumerate}[(1)]\tightlist
\item 제2장에 기재되지 아니한 검사로서 외관, 취기, 색도 등의 간단한 검사 또는 계산방법에 의하여 검사치를 얻는 경우에는 검사료를 산정하지 아니한다.
\item \uline{대칭기관에 대한 양측검사를 하였을 때에도 “편측”이라는 표기가 없는 한 소정점수만 산정한다.}
\item \uline{검사에 사용된 약제 및 재료대(1회용 주사침 및 주사기 포함)는 소정점수에 포함되므로 별도 산정하지 아니한다. 다만, 다음의 경우에는 “약제 및 치료 재료의 비용에 대한 결정기준”에 의하여 별도 산정한다.}
	\begin{enumerate}[(가)]\tightlist
	\item 인체에 주입된 약제
	\item 부하시험시 사용된 약제
	\item 안기능검사시 사용된 필름, 형광물질, 사진현상 및 인화료
	\item 내시경검사시 사용된 슬라이드 필름 및 사진현상료, 포라로이드필름 또는 칼라프린터 인화지
	\item 핵의학 기능검사시 사용된 방사성 동위원소 및 약제
	\item 제2장 분류항목에 별도로 규정한 약제 및 재료대
	\item 기타 장관이 별도로 인정한 약제 및 재료대
	\end{enumerate}
\item \uline{인체에서 채취한 가검물에 대한 검사를} “(부록) 검체검사 위탁에 관한 기준"에서 정한 \uline{수탁기관으로 위탁하는 경우에는} 제2장 제1절 및 제2절 분류항목 소정점수(가감률 적용 포함)에 \uline{수탁기관의 점수당 단가를 곱하여 계산한 금액의 10\%를 “위탁검사관리료"로 산정한다}
\item (별표)에 열거한 항목은 다음 중 어느 하나에 해당하는 경우에만 산정하되,\uline{㈎, ㈏ 및 ㈐의 경우에는 소정점수의 10\%를 가산하여 산정}한다.(산정코드 세 번째 자리에 6으로 기재)
	\begin{enumerate}[(가)]\tightlist
	\item \uline{진단검사의학과 전문의가 판독하고 판독소견서를 작성\cntrdot{} 비치한 경우}
	\item B세포 표면면역글로불린, 세포표지검사, 면역조직(세포)화학검사, 세포 주기 및 핵산분석검사(유세포측정법), 분자병리검사에 대하여 병리과 전문의가 판독하고 판독소견서를 작성\cntrdot{} 비치한 경우
	\item \uline{분자병리검사에 대하여 관련분야에 대하여 인증 받은 전문의가 판독하고 판독소견서를 작성\cntrdot{} 비치한 경우}
	\item 면역조직(세포)화학검사에 대하여 구강병리과가 설치된 요양기관의 치과의사가 판독하고 판독소견서를 작성\cntrdot{} 비치한 경우
	\end{enumerate}
\end{enumerate}
	
\begin{mdframed}[linecolor=blue,middlelinewidth=2]  
제1편 행위 급여 \cntrdot{}  비급여 목록 및 급여 상대가치점수 >> 제2부 행위 급여 목록\cntrdot{} 상대가치점수 및 산정지침 >> 제5장 주사료
\end{mdframed}
\subsection{\newindex{주사료〔산정지침〕}}
\begin{enumerate}[(1)]\tightlist
\item \uline{주사시 사용된 주사재료대(1회용 주사기, 1회용 주사침, 나비침, 정맥내유치침, 수액세트, 혈액Bag 등)와 수혈에 소요된 약제 및 재료대는 소정점수에 포함되므로 별도 산정하지 아니한다. 다만, 정맥내유치침을 사용한 경우에는 「마-5-주1」및「마-15-다-주1」에 따라 산정하며,} 다음의 경우에는 “약제 및 치료재료의 비용에 대한 결정기준”에 의하여 별도 산정한다.
	\begin{enumerate}[(가)]\tightlist
	\item 치료적 성분채집술에 사용된 약제 및 재료대(요양기관이 대한적십자사 혈액원 등으로부터 성분채집에 의한 혈액성분제제를 구입한 경우 포함)
	\item 조혈모세포이식 시 사용된 골수, 말초혈액, CD34+ Collection Kit,Cryo Bag
	\item 적혈구수집기(Cell Salvage)를 이용한 자가수혈에 사용된 재료대
	\end{enumerate}
\item 제1절 주사료를 산정하는 경우 만8세 미만의 소아에 대하여 정맥내 점적 주사(마-5, 마-15-다)는 주사료 소정점수의 30\%를 가산하고, 기타 주사는 주사료 소정점수의 20\%를 가산한다.(산정코드 첫 번째 자리에 3으로 기재) 다만, 피하 또는 근육내주사(마-1), 생물학적제제주사(마-4), 수액제 주입로를 통한 주사(마-5-1), 항암제 피하내주사(마-15-가), 급속항온주입(마-16)은 그러하지 아니한다.
\end{enumerate}
\uline{정맥내유치침 KK059 마5주1 (400원)} \par
\uline{항암제정밀지속적점적주입위한InfusionPump사용료[기기당1일1회] 마15다주2 KK158(2010원)}\par

\begin{mdframed}[linecolor=blue,middlelinewidth=2]  
제1편 행위 급여 \cntrdot{}  비급여 목록 및 급여 상대가치점수 >> 제2부 행위 급여 목록\cntrdot{} 상대가치점수 및 산정지침 >> 제5장 마취료
\end{mdframed}
\subsection{\newindex{마취료〔산정지침〕}}
\begin{enumerate}[(1)]\tightlist
\item 마취약제 주사 시 사용한 1회용 주사기 및 주사침 등의 재료대는 마취료 소정점수에 포함되므로 별도 산정하지 아니한다.
\item 신생아 마취시에는 마취료 소정점수의 60\%를 가산하며, 만8세 미만의 소아 또는 만70세 이상의 노인의 경우에는 마취료 소정점수의 30\%를 가산한다.(산정코드 첫 번째 자리에 신생아는 1, 만8세 미만은 3, 만70세 이상은 4로 기재)
\item 장기이식수술마취2), 심폐체외순환법마취5), 일측폐환기법마취6), 고빈도제트 환기법마취7), 개흉적 심장수술마취8), 뇌종양, 뇌혈관질환에 대한 개두술마취9)시에는 마취료 소정점수의 50\%를 가산한다.(산정코드 첫 번째 자리에 각각 2, 5, 6, 7, 8, 9로 기재)
\item 18시-09시 또는 공휴일에 응급진료가 불가피하여 마취를 행한 경우에는 소정점수의 50\%를 가산한다.(산정코드 두 번째 자리에 18시-09시는 1, 공휴일은 5로 기재) 이 경우 해당 마취를 시작한 시각을 기준으로 산정한다.
\item 수술 중에 발생하는 우발사고에 대한 처치(산소흡입, 응급적 인공호흡) 또는 주사(강심제) 등의 비용은 별도 산정할 수 있으나, 그 밖의 경우에는 산소 흡입, 응급적 인공호흡비용 및 EKG monitoring료는 산정하지 아니한다.
\item 동일 목적을 위하여 2 이상의 마취를 병용한 경우 또는 마취 중에 다른 마취법으로 변경한 경우에는 주된 마취의 소정점수만 산정한다.
\item 제6장에 분류되지 아니한 표면마취, 침윤마취 및 간단한 전달마취의 비용은 제2장, 제9장 또는 제10장에 분류한 소정 시술료에 포함되므로 별도 산정 하지 아니한다.
\item 마취통증의학과 전문의 초빙료를 산정하는 경우에는 초빙된 마취통증의학과 전문의의 면허종류, 면허번호를 요양급여비용 청구명세서에 기재하고, 마취 통증의학과 전문의가 서명 또는 날인한 마취기록지를 비치하여야 한다.
\end{enumerate}
  
\begin{mdframed}[linecolor=blue,middlelinewidth=2]  
제1편 행위 급여 \cntrdot{}  비급여 목록 및 급여 상대가치점수 >> 제2부 행위 급여 목록\cntrdot{} 상대가치점수 및 산정지침 >> 제9장 처치 및 수술료 등 
\end{mdframed}
\subsection{\newindex{제9장 처치 및 수술료 등}}
\paragraph{제1절 처치 및 수술료〔산정지침〕}
\begin{enumerate}[(1)]\tightlist
\item \uline{18시-09시 또는 공휴일에 응급진료가 불가피하여 처치 및 수술을 행한 경우에는 소정점수의 50\%를 가산한다.(산정코드 두 번째 자리에 18시-09시는 1, 공휴일은 5로 기재)} 이 경우 해당 처치 및 수술을 시작한 시각을 기준하여 산정한다.
\item 「응급의료에 관한 법률」에 의한 응급환자에게 응급의료기관이 응급실에서 응급의료수가기준 “(별표1) 응급의료수가기준액표 나. 응급처치료”의 해당 항목을 실시한 경우에는 소정점수의 50\%를 가산한다.(산정코드 두 번째 자리에 2로 기재)
\item \uline{제1절에 기재되지 아니한 처치 및 수술로서 간단한 처치 및 수술의 비용은 기본진료료에 포함되므로 산정하지 아니한다.}
\item 제1절에 기재되지 아니한 처치 및 수술로서 위 “⑶”에 해당되지 아니하는 처치 및 수술료는 \uline{제1절에 기재되어 있는 처치 및 수술 중에서 가장 비슷한 처치 및 수술 분류항목의 소정점수에 의하여 산정한다.(준용산정)}
\item \uline{대칭기관에 관한 처치 및 수술 중 “양측”이라고 표기한 것은 “양측”을 시술할지라도 소정점수만 산정한다.}
\item \uline{동일 피부 절개 하에 2가지 이상 수술을 동시에 시술한 경우 주된 수술은 소정점수에 의하여 산정하고, 제2의 수술부터는 해당 수술 소정점수의 50\%}(산정코드 세 번째 자리에 1로 기재), 상급종합병원·종합병원은 해당 수술 소정점수의 70\%(산정코드 세 번째 자리에 4로 기재)를 산정한다. \uline{다만, 주된 수술 시에 부수적으로 동시에 실시하는 수술의 경우에는 주된 수술의 소정점수만 산정한다.}
\item 제1절에 기재된 분류항목 중 상\cntrdot{} 하악골 악성종양 절제술(자-40-나,
자-43-나), 비강, 부비동악성종양적출술(자-96), 비인강 악성종양적출술,(자-104-1), 후두 전적출술(자-122-1-다), 후두 및 하인두 전적출술(자-125), 후두 전적출 및 하인두 부분적출술(자-125-1), 구순암적출술(자-215), 설암 수술(자-218),구강내악성종양적출술(자-220-다), 이하선악성종양적출술(자-223-나), 인두악성종양수술(자-229-1), 부갑상선악성종양절제술(자-454-나), 갑상선 악성종양근치수술(자-456) 시행시 경부의 림프절 청소술을 병행한 경우에는 위 “(6)"에도 불구하고 경부림프절청소술(자-211) “주"의 소정점수를 별도 산정한다.
\item \uline{근접하고 있는 다발성 절종을 수개 처에서 절개한 경우나 동일 검내에 존재하는 맥립종, 산립종의 수술 등은 1회 절개로 간주한다.}
\item 수술은 개시하였으나 병상의 급변 등 부득이한 사유로 인하여 그 수술을 중도에서 중단하여야 할 경우에는 수술의 중단까지와 시술상태가 가장 비슷한 항목의 수술료를 산정한다.
\item 각 분류항목의 \uline{처치 및 수술 등에 레이저를 이용한 경우에도 각 분류항목의 소정점수만을 산정한다.}
\item 각 분류항목의 처치 및 수술 등에\uline{ 내시경을 이용한 경우 내시경료는 소정 시술료에 포함되므로 별도 산정하지 아니한다.}
\item 처치 및 수술시에 사용된 약제 및 치료재료대는 소정점수에 포함되므로 별도 산정하지 아니한다. 다만, 다음에 \uline{열거한 약제 및 치료재료대는 “약제 및 치료재료의 비용에 대한 결정기준”에 의하여 별도 산정한다.}
	\begin{enumerate}[①]\tightlist
	\item 인공식도
	\item 인공심장판막
	\item 인공심폐회로
	\item 인공심박기
	\item ....
	\item 체내고정용 나사, 고정용 금속핀, 고정용 금속선, 고정용 못
	\item 지속적주입, 지속적배액 및 지속적 배기용도관 [체내유치]
	\item 폴리비니루, 호루말 등 충전술 사용재료
		\begin{itemize}\tightlist
		\item 고주파신경자극기 [수술삽입시만 산정]
		\item 고정용 신축성 붕대
		\item 개심술, 안면수술 등 장관이 별도로 정한 처치 및 수술시 사용된 봉합사
		\item \uline{일반처치 또는 수술후처치(자-2-1), 피부과처치(자-18), 화상처치(자-18-1),위세척(자-590)에 사용된 생리식염수 [단, 총사용량이 500ml 이상인 경우에 한함] }
		\item \uline{피부과처치(자-18) 또는 화상처치(자-18-1)시 사용된 연고, 처치 및 수술시 사용된 인체주입용 약제}(단, KMnO4 등의 소독약제는 소정 처치 및 수술료에 포함되므로 별도 산정하지 아니한다.)
		\item 산정지침 ⑽에 해당되는 레이저시술 중 장관이 별도로 인정한 “레이저시술”에 소요된 레이저 재료대
		\item 제1절 및 제2절 분류항목에 별도로 표기한 경우
		\item 기타 장관이 별도로 인정한 약제 및 치료재료(인체조직 포함)
  		\end{itemize}
	\end{enumerate}  
\item (별표 1) 및 (별표 2)에 열거한 항목을 외과 전문의가 시행한 경우에는 해당 항목 소정점수의 (별표 1)은 20\%, (별표 2)는 30\%를 가산한다.(산정코드 첫 번째 자리에 1로 기재)
\end{enumerate}

\subsection{일반처치 또는 수술후처치(자-2-1)} 
\paragraph{자-2-1 일반처치 또는 수술후처치 등 [1일당] Dressing or Post Operative Dressing etc.}
주:
\begin{enumerate}[1.]\tightlist
\item \uline{수술후 처치료는 수술 익일부터 산정한다.}
\item \uline{사용된 거즈, 탈지면, 붕대, 반창고의 비용은 소정 점수에 포함되므로 별도 산정하지 아니한다.}
\item 같은 날에\uline{「다」와「라」, 「마」와「사」,「바」와「자」 또는 「아」와「자」를 실시한 경우에는 둘 중 한 항목의 소정점수만을 산정한다.}
\item \uline{같은 날에 「가」의 (1) 또는 (2)를 여러 부위에 실시한 경우에는 두부, 복부, 배부, 좌·우·상·하지 7부위로 구분하여 각 부위별로 소정점수를 1회만 산정한다.}
\item 다만, 상급종합병원 중환자실에 입원중인 경우에는 [1일당], ’주3‘ 및 ’주4‘에도 불구하고 1일에
「가」는 2회 이내, 「라」와「바」는 3회 이내로 산정한다. (기본코드 5번째 자리에 5로 기재)
\end{enumerate}
%\myexplfn{ } \\ \tabucline[.5pt]{-} %
\tabulinesep =_2mm^2mm
\begin{tabu} to\linewidth {|X[2,l]|X[6,l]|X[1,l]|X[1,l]|} \tabucline[.5pt]{-}
\rowcolor{ForestGreen!40}  코드 &	\centering 분 류 & 점수 & 금액 \\ \tabucline[.5pt]{-}	
\rowcolor{Yellow!40} & 가. 창상처치 Wound Dressing && \\ \tabucline[.5pt]{-}
\rowcolor{Yellow!40} M0111 & (1) 단순처치 Simple Dressing & 58.04 & \myexplfn{58.04} \\ \tabucline[.5pt]{-} %4,320 \\ \tabucline[.5pt]{-}
\rowcolor{Yellow!40} M0121 & (2) 염증성 처치\footnote{주:수술창의 심한 염증 처치, 심한 욕창, 염증이 심한 상처의 처치에 산정한다.} Infectious Wound Dressing & 112.10 & \myexplfn{112.10} \\ \tabucline[.5pt]{-} %8,340 \\ \tabucline[.5pt]{-}
\rowcolor{Yellow!40} M0131 & 나. 장루처치 Stoma Care & 86.75 & \myexplfn{86.75} \\ \tabucline[.5pt]{-} %6,450 \\ \tabucline[.5pt]{-}
\rowcolor{Yellow!40} M0134 & 다. 수술후 튜브삽입에 의한 자연 배액감시 및 처치 Natural Drainage and Care after Operation & 47.57 & \myexplfn{47.57} \\ \tabucline[.5pt]{-} %3,540 \\ \tabucline[.5pt]{-}
\rowcolor{Yellow!40} M0137 & 라. 흡입배농 및 배액처치 Suction Drainage or Tracheostomy Suction etc. & 113.96 &  \myexplfn{113.96} \\ \tabucline[.5pt]{-} %8,480 \\ \tabucline[.5pt]{-}
\rowcolor{Yellow!40} M0141 & 마. 좌욕 Sitz Bath & 19.40 & \myexplfn{19.40} \\ \tabucline[.5pt]{-} %1,440 \\ \tabucline[.5pt]{-}
\rowcolor{Yellow!40} M0143 & 바. 체위변경처치\footnote{주:척수손상, 뇌졸중 환자 등에서 혈액순환 도모 및 욕창방지 등을 위해 피부마사지를 포함한 체위
변경 시에 산정한다.} Position Change & 86.96 & \myexplfn{86.96} \\ \tabucline[.5pt]{-} %6,470 \\ \tabucline[.5pt]{-}
\rowcolor{Yellow!40} M0151 & 사. 회음부 간호 Perineal Care & 56.47 & \myexplfn{56.47} \\ \tabucline[.5pt]{-} %4,200 \\ \tabucline[.5pt]{-}
\rowcolor{Yellow!40} M0153 & 아. 통목욕 간호 Tub Bath & 120.36 & \myexplfn{120.36} \\ \tabucline[.5pt]{-} %8,950 \\ \tabucline[.5pt]{-}
\rowcolor{Yellow!40} M0155 & 자. 침상목욕 간호 Bed Bath & 157.91 & \myexplfn{157.91} \\ \tabucline[.5pt]{-} %11,750 \\ \tabucline[.5pt]{-}
\end{tabu}
%\par

%\paragraph{피부과처치(자-18), 화상처치(자-18-1)}
\paragraph{자-18 피부과처치 [1일당] Dermatologic Dressing}
주:
\begin{enumerate}[1.]\tightlist
\item \uline{피부연고 도포 등 단순한 피부 처치는 기본진료료에포함되므로 별도 산정하지 아니한다.}
\item \uline{사용된 거즈, 탈지면, 붕대, 반창고의 비용은 소정 점수에 포함되므로 별도 산정하지 아니한다.}
\end{enumerate}
\tabulinesep =_2mm^2mm
\begin{tabu} to\linewidth {|X[2,l]|X[6,l]|X[1,l]|X[1,l]|} \tabucline[.5pt]{-}
\rowcolor{ForestGreen!40}  코드 &	\centering 분 류 & 점수 & 금액 \\ \tabucline[.5pt]{-}	
\rowcolor{Yellow!40} & 가. 농가진, 감염성피부질환 등에 Wet Dressing 또는 Soaking을 행한 경우 && \\ \tabucline[.5pt]{-}
\rowcolor{Yellow!40} N0181 & (1) 9\% 이하의 범위 & 72.48 & \myexplfn{72.48} \\ \tabucline[.5pt]{-} %5,390 \\ \tabucline[.5pt]{-}
\rowcolor{Yellow!40} N0182 & (2) 10\%∼18\%의 범위 & 85.26 & \myexplfn{85.26} \\ \tabucline[.5pt]{-} %6,340 \\ \tabucline[.5pt]{-}
\rowcolor{Yellow!40} N0183 & (3) 19\%∼36\%의 범위 & 96.43 & \myexplfn{96.43} \\ \tabucline[.5pt]{-} %7,170 \\ \tabucline[.5pt]{-}
\rowcolor{Yellow!40} N0184 & (4) 37\% 이상의 범위 & 135.14 & \myexplfn{135.14} \\ \tabucline[.5pt]{-} %10,050 \\ \tabucline[.5pt]{-}
\rowcolor{Yellow!40} & 나. 대상포진에 실시한 경우 In Herpes Zoster& &  \\ \tabucline[.5pt]{-}
\rowcolor{Yellow!40} N0061 & (1) 9\% 이하의 범위 & 76.60 & \myexplfn{76.60} \\ \tabucline[.5pt]{-} %5,700 \\ \tabucline[.5pt]{-}
\rowcolor{Yellow!40} N0062 & (2) 10\%∼18\%의 범위 & 88.45 & \myexplfn{88.45} \\ \tabucline[.5pt]{-} %6,580 \\ \tabucline[.5pt]{-}
\rowcolor{Yellow!40} N0063 & (3) 19\%∼36\%의 범위 & 100.99 & \myexplfn{100.99} \\ \tabucline[.5pt]{-} %7,510 \\ \tabucline[.5pt]{-}
\rowcolor{Yellow!40} N0064 & (4) 37\% 이상의 범위 & 139.47 & \myexplfn{139.47} \\ \tabucline[.5pt]{-} %10,380 \\ \tabucline[.5pt]{-}
\end{tabu}
%\par

  
\paragraph{자-18-1 화상처치 Burn Dressing}
주:
\begin{enumerate}[1.]\tightlist
\item 화상부위가 수개 부위일 경우에는 수개 부위의 화상범위를 합하여 아래 항목에 의거하여 산정하되 \uline{화상범위 산정시 1도 화상 범위는 제외한다.}
\item \uline{사용된 거즈, 붕대의 재료대는 별도 산정하되 탈지면, 반창고 등의 비용은 소정점수에 포함되므로 별도 산정하지 아니한다.}
\end{enumerate}
\tabulinesep =_2mm^2mm
\begin{tabu} to\linewidth {|X[2,l]|X[6,l]|X[1,l]|X[1,l]|} \tabucline[.5pt]{-}
\rowcolor{ForestGreen!40}  코드 &	\centering 분 류 & 점수 & 금액 \\ \tabucline[.5pt]{-}	
\rowcolor{Yellow!40} &가. 열탕, 화염, 동상, 화학화상 등의 경우 Scald, Flame, Frostbite, Chemical Burn etc. && \\ \tabucline[.5pt]{-}
\rowcolor{Yellow!40} & (1) 9\% 이하의 범위 & & \\ \tabucline[.5pt]{-}
\rowcolor{Yellow!40} N0011 & (가) 수, 족, 지, 안면, 경부, 성기를 포함하는 경우 Including Hand, Foot, Finger or Toe, Face,Neck.Genitalia & 341.06 & \myexplfn{341.06} \\ \tabucline[.5pt]{-} %25,370 \\ \tabucline[.5pt]{-}
\rowcolor{Yellow!40} N0012 & (나) 수, 족, 지, 안면, 경부, 성기를 포함하지 아니한 경우 Excluding Hand, Foot, Finger or Toe,Face, Neck, Genitalia & 198.46 & \myexplfn{198.46} \\ \tabucline[.5pt]{-} %14,770 \\ \tabucline[.5pt]{-}
\rowcolor{Yellow!40} N0053 & (2) 하지의 1지, 복부 또는 배부에 준하는 범위[10\%-18\%] One Lower Extremity, Abdomen or Back & 913.15 & \myexplfn{913.15} \\ \tabucline[.5pt]{-} %67,940 \\ \tabucline[.5pt]{-}
\rowcolor{Yellow!40} N0054 & (3) 양하지 또는 동체(복부 및 배부)에 준하는 범위 [19\%-36\%] Both Lower Extremities or Trunk & 1,682.54 & \myexplfn{1682.54} \\ \tabucline[.5pt]{-} %125,180 \\ \tabucline[.5pt]{-}
\rowcolor{Yellow!40} NA055 & (4) 상\cntrdot{} 하지 대부분, 양하지와 복부 또는 배부에 준하는범위 [37\%-54\%]
Upper\cntrdot{} Lower Extremities, Both Lower Extremities and Abdomen or Back &  2,633.56 & \myexplfn{2633.56} \\ \tabucline[.5pt]{-} %195,940 \\ \tabucline[.5pt]{-}
\rowcolor{Yellow!40} NA056 & (5) 전신대부분의 범위 [55\% 이상의 범위] more than 55\% of Body Surface Area
& 3,950.32 & \myexplfn{3950.32} \\ \tabucline[.5pt]{-} %293,900 \\ \tabucline[.5pt]{-}
\rowcolor{Yellow!40} & 나. 전기화상의 경우 Electrical Burn N0057 (1) 근육, 골격, 인대의 손상이 포함된 경우
with Injury of Muscle, Skeletal or Tendon & 1,682.54 & 125,180 \\ \tabucline[.5pt]{-}
\rowcolor{Yellow!40} NA057 & 주:섬광 또는 화염이 동반된 경우에는 & 3,365.08 &  \\ \tabucline[.5pt]{-}
\rowcolor{Yellow!40} N0058 & (2) 기타의 경우 Others & 992.39 & \myexplfn{992.39} \\ \tabucline[.5pt]{-} %73,830 \\ \tabucline[.5pt]{-}
\rowcolor{Yellow!40} NA058 & 주:섬광 또는 화염이 동반된 경우에는 & 1519.60점을 산정 & \\ \tabucline[.5pt]{-}
\end{tabu}
%\par


\begin{mdframed}[linecolor=blue,middlelinewidth=2]  
제1편 행위 급여 \cntrdot{}  비급여 목록 및 급여 상대가치점수 >> 제2부 행위 급여 목록\cntrdot{} 상대가치점수 및 산정지침 >> 제16장 전혈및 혈액성분 제제료 
\end{mdframed}
\subsection{제16장 \newindex{전혈및 혈액성분 제제료}}
\emph{〔산정지침〕}
\begin{enumerate}[(1)]\tightlist
\item “혈액관리법 제11조”의 규정에 의하여 장관이 별도 고시한 항목과 금액으로 산정한다.
\item \uline{수혈에 소요되는 약제 및 재료대(1회용 주사기, 1회용 주사침, 나비침, 정맥내유치침, 수액세트, 혈액 Bag 등)는 소정금액에 포함되므로 별도 산정하지 아니한다. 다만, 정맥내 유치침을 사용한 경우에는 「마-5-주1」에 따라 산정하며, 다음의 경우에는 “약제 및 치료재료의 비용에 대한 결정 기준”에 의하여 별도 산정한다.}
	\begin{enumerate}[(가)]\tightlist
	\item 백혈구여과제거적혈구 및 백혈구여과제거혈소판의 경우에 사용된 약제 및 재료대
	\item 혈액성분채집술(복합성분채집 혈장은 제외)에 사용된 약제 및 재료대 (요양기관이 대한적십자사혈액원 등으로부터 성분채집에 의한 혈액 성분제제를 구입한 경우 포함)
	\end{enumerate}
\item 혈액성분채집술에 의한 혈액성분채혈시 공혈자에 대한 공혈적합성 여부를
판정하기 위한 검사비용은 소정금액에 포함되므로 별도 산정하지 아니한다.
\end{enumerate}
  
\section{실손보험}
\includegraphics{silson}\\
\includegraphics[scale=.7]{silson2}\\
\begin{hemphsentense}{보장되지 않는 진단명들}
\begin{enumerate}\tightlist
\item 정신과질환 및 행동장애(F04-F99)
\item 여성생식기의 비염증성 장애로 인한 습관성 유산, 불임 및 인공수정관련 합병증(N96-N98)
\item 피보험자의 임신, 출산(제왕절개를 포함합니다), 산후기로 통원한 경우(O00-O99)
\item 선천성 뇌질환(Q00-Q04)
\item 비만(E66)
\item 비뇨기계 장애(N39, R32)
\item 직장 또는 항문질환 중 국민건강보험법상 요양급여에 해당하지 않는 부분(I84, K60-K62)
\item 주로 성행위로 전파되는 감염
\end{enumerate}
\end{hemphsentense}

\subsection{필요서류}
\includegraphics[scale=.75]{silsonrequest}\\


\begin{mdframed}[linecolor=blue,middlelinewidth=2]
\begin{itemize}\tightlist
\item 3만원이하 \MVRightarrow 영수증
\item 3-10만원 \MVRightarrow 병명 포함 처방전 +영수증
\item 10만원 초과 \MVRightarrow 영수증+진단명 및 통원일자 기간이 포함된 서류[진단서, 소견서, 
		통원확인서, 진료차트, 통원일자별 처방전]
\item 비급여 항목 발생시 진료비 세부내역서 필요
\end{itemize}
\end{mdframed}

\subsection{산부인과 응용}
\begin{mdframed}[linecolor=blue,middlelinewidth=2]
\begin{itemize}\tightlist
\item 질염환자 방문 \MVRightarrow STD6 등 실손처리
\item 혈뇨가 유독 심한 방광염 환자 \MVRightarrow full work up 후 실손처리
\item 아랫배 통증 환자 \MVRightarrow 골반초음파 후 실손처리
\item 생리통과 생리과다로 미레나 삽입한 환자 \MVRightarrow 실손처리
\item 캔디다 질염 환자 \MVRightarrow PCR or Culture 후 실손처리
\item 희발월경 환자 \MVRightarrow Hormone full work up 후 실손처리
\item 골반염 환자 \MVRightarrow full work up 후 실손처리
\end{itemize}
\end{mdframed}
이렇게 함으로써 병원 비싸다는 불만 감소, 경쟁의원간 가격 스트레스 감소될수 있음.
\clearpage

\section{우리나라의 급여system의 이해와 독소조항}
\subsection{요양기관에서 제공되는 행위,약제,치료재료를 요양급여대상과 비급여 대상으로 구분하고 비급여 대상으로 고시되지 않은 항목은 모두 요양급여 대상으로 적용하는 negative system입니다.}
위의 이유로 의해 아무리 좋은치료라 해도 \uline{인정되지않는 비급여치료는 불법입니다.}
\subsection{요양급여의 대상여부 확인 (국민건강보험법 제43조의 2)}
가입자 또는 피부양자는 본인일부부담금
   외에 부당한 비용이 제39조 3항의 규정에 의하여 요양급여의 대상에서 제외되는 것인지를 건강보험심사평가원에 확인을 요청할수 있다
 확인요청을 받은 심평원은 그 결과를 확인 요청자에게 통보하여야하며,확인 요청한 비용이 요양급여의 대상에 해당되는 비용으로 확인되면 그내용을 공단및 관련 요양기관에 통보하여야한다 
통보받은 요양기관은 과다 징수한 금액을 지체없이 확인요청자에게 지급하여야한다
만일 당해 요양기관이 과다본인부담금을 지급하지 아니한경우에는 공단은 당해 요양기관에 지급할 요양급여비용에서 그 과다본인부담금을 공제하여 이를 확인 요청자에게 지급할수있다
\subsection{\newindex{100대100}이란?}
100분의 100 본인부담은 보건복지부장관이 정하여 고시한 상한금액을 환자가 모두 부담하는 것을 말합니다. 
비급여는 보건복지부장관이 정하여 고시한 진료항목에 대하여 해당 진료를 실시하는 병의원에서 정한 금액을 환자가 모두 부담하는 것입니다.
따라서, 100분의 100 본인부담과 비급여의 차이는 100분의 100 본인부담은 법령 등으로 정하여진 상한금액이 있어 어느 병\cntrdot{}의원에서든 동일한 금액을 환자에
게 징수하여야하나, 비급여는 정하여진 금액이 없어 동일한 진료행위인 경우라도병\cntrdot{}의원별로 금액이 상이할 수 있습니다. 예를 들어 쌍꺼풀수술, 점제거술 등은 비
급여대상 진료로 병\cntrdot{}의원별로 금액이 상이합니다.
 

\section{비급여 원내고시의무, 보건소보고의무X}
원내고시 내용만 변경하면 됩니다..비급여 항목,,,보건소 제출은 5년전에 (2010년 1월 의료법 제45조가 개정·시행).....의무사항에서 제외되었습니다.

\subsection{비급여 대상의 항목(행위․약제․치료재료)과 그 가격을 적은 책자 등을 접수창구 등 환자 또는 환자의 보호자가 쉽게 볼 수 있는 장소에 비치}
\begin{itemize}\tightlist
\item `책자 등'이라 함은 비급여 진료비용이 모두 기재되어 환자들이 쉽게 열람할 수 있도록 의료기관 구내에 비치된 매체라면 폭넓게 인정됨. 
     * 제본된 책자, 제본되지 않은 인쇄물, 메뉴판, 벽보, 비용검색 전용 컴퓨터 등
\item `쉽게 볼 수 있는 장소의 범위'는 일반적으로 환자대기실·접수창구 및 진료받은 비용을 정산할 수납창구 등이 될 수 있음.
\end{itemize}

\subsection{비급여 대상의 항목(행위․약제․치료재료)을 묶어 1회 비용을 정하여 총액을 표기 가능}
\begin{itemize}\tightlist
\item 건강보험 급여비용까지 포함하여 표기 가능(다만, 급여비용 포함된 가격임을 알 수 있도록 비고란 등에 표기)
\item 비급여 비용은 원칙적으로 단일 가격으로 고지해야 하나, 환자 상태에 따른 행위의 난이도 차이가 발생할 수 있으므로 범위를 설정하여 표기 가능(가급적 항목 분류 세분화, 가격범위 설정 이유 등 표기)
\item 비급여 진료비용이나 항목이 변경된 경우, 변경된 날짜를 기재하고, 변경된 내용을 표기
\item 고지된 가격 이하로 비용을 받는 것은 가능하지만, 이를 초과하여 징수하지 못하며, 위반시 시정명령 처분(「의료법」 제45조제3항 및 제63조)
\end{itemize}
\clearpage

%\clearpage
%\section{\newindex{비급여}란?}
%\subsection{건강보험법 내 비급여대상(제9조제1항관련) 에 대한 규정}
%\subsection{비급여대상(제9조제1항관련)}

\section{\newindex{보험에 관련된 흔한 질문들}}
%\noindent
%\begin{minipage}[t]{0.45\linewidth}% Question goes here
%    \textbf{Q.}
%    성주체성장애 환자는 보험으로 진료비는 청구 가능한지요?\index{보험QA!성주체성장애}
%\end{minipage}
%\hfill% Separate the Question and Answer
%\begin{minipage}{0.45\linewidth}% Answer goes here
%    \begin{mdframed}[linecolor=blue,middlelinewidth=2]
%    건강검진, 예방접종 등 예방진료로서 질병·부상의 진료를 직접목적으로 하지 아니하는 경우에 실시 또는 사용되는 행위·약제 및 치료재료 등에 해당되지 않으므로 성주체성장애는 %\colorbox{Yellow!80}{급여 대상}으로 진찰후 진찰료 청구가 가능합니다
%    \end{mdframed}
%\end{minipage}\par
%\bigskip
\Que{성주체성장애 환자는 보험으로 진료비는 청구 가능한지요?}\index{보험QA!성주체성장애}
\Ans{건강검진, 예방접종 등 예방진료로서 질병·부상의 진료를 직접목적으로 하지 아니하는 경우에 실시 또는 사용되는 행위·약제 및 치료재료 등에 해당되지 않으므로 성주체성장애는 \mycoloredbox{급여 대상}으로 진찰후 진찰료 청구가 가능합니다}

\Que{응급피임약 (또는 비아그라)만 처방하는 경우 진찰료 청구가 가능한가요?}\index{보험QA!응급피임약}
\Ans{응급피임약 (또는 비아그라)만 처방하는 경우는 \mycoloredbox{비급여 대상}으로 진찰료 청구가 불가합니다.}

\Que{팔에 심는 피임약(임플라논) 제거시 수가 산정 방법 및 급여 여부} \index{보험QA!임플라논제거}\par
\Ans{피임시술은 고시 제2010-45호(‘10.7.1 시행)에 의거 본인이 원하여 실시한 경우는 \mycoloredbox{비급여대상}입니다. 현재 임플라논의 시술 및 제거 관련하여는 사례별로 목적에 따라 급여여부가 결정되어야 할 것으로 사료되오니 이해있으시길바랍니다.\par
다만 시술 및 제거 관련 수기료에 대하여는 현재 별도 정하고 있지는 않으나「건강보험요양급여비용」 제1편2부9장 처치 및 수술료 산정지침 (3) 및 (4)에서는 다음과 같이 언급하고 있으니 업무에 참고하시기 바랍니다.\par
- 다 음 -\par
(3) 제1절에 기재되지 아니한 처치 및 수술로서 간단한 처치 및 수술의 비용은 기본진료료에 포함되므로 산정하지 아니한다.\par
(4) 제1절에 기재되지 아니한 처치 및 수술로서 위 “(3)”에 해당되지 아니하는 처치 및 수술료는 제1절에 기재되어 있는 처치 및 수술 중에서 가장 비슷한 처치 및 수술 분류항목의 소정점수에 의하여 산정한다
}

\Que{ 술전검사중 HCV Ab에 대한 보험적용여부} \index{보험QA!술전검사중 HCV Ab보험적용}
\Ans{국민건강보험 요양급여의 기준에 관한 규칙」[별표1] 요양급여의 적용기준 및 방법 1항 다목에 의하면 요양급여는 경제적으로 비용효과적인 방법으로 행하여야 하며 2항 가목에서는 각종 검사를 포함한 진단 및 치료행위는 진료상 필요하다고 인정되는 경우에 한하여야 하며 연구의 목적으로 하여서는 아니된다고 정하고 있습니다.
  
나487 C형간염항체검사(HCV Ab)는 인정기준(고시 제2009-250호)에 의거, 급여대상 질환이나 타 검사소견등 의학적 타당성이 확인(진료기록 등)되는 경우에는 급여대상이지만 아닌 경우에는 \mycoloredbox{비급여}임을 알려드립니다.
}
%\begin{Cdoing}{
\clearpage

\section{\newindex{인정비급여항목}들}
\begin{itemize}\tightlist
\item 초음파
\item 자궁경부확대촬영검사 Cervicography (비급여목록 노886
\item 상급병실료
\item 고주파 자궁근종 융해술
\item 자기공명영상유도하 고강도 초음파 집속술(자궁근종)
\item HAL\&RAR(Hemorrhoidal artery ligation and Rectoanal repair)
\item Rubella IgG avidity test
\item 성기능상담[NZ002]
\item 성기능 장애 평가[FZ684]
\item AMH
\item fragile X test
\item FISH 등입니다.
\end{itemize}
\clearpage

\section{제6차 개정 한국표준질병사인분류 코딩지침서에서의 주진단명 부여방법}
주된병태란 말은 주진단과 같이 쓴다. 다음의 내용은\\ \href{http://kostat.go.kr/kssc/common/CommonAction.do?method=download&attachDir=bm90aWNl&attachName=JUVEJTk1JTlDJUVBJUI1JUFEJUVEJTkxJTlDJUVDJUE0JTgwJUVDJUE3JTg4JUVCJUIzJTkxJUVDJTgyJUFDJUVDJTlEJUI4JUVCJUI2JTg0JUVCJUE1JTk4XyVFQyVBNyU4OCVFQiVCMyU5MSVFQyVCRCU5NCVFQiU5NCVBOSVFQyVBNyU4MCVFQyVCOSVBOCVFQyU4NCU5QyUyODIwMTIuMDMlMjkucGRm} {제6차 개정 한국표준질병사인분류 코딩지침서}에 따릅니다.\\
유용한 KCD code를 찾을수 있는 Link입니다. \url{http://www.kcdcode.co.kr/}
\begin{itemize}[▷]\tightlist
\item 산과진단코드 부여 순서
	\begin{enumerate}\tightlist
	\item \uline{제왕절개나 기구를 사용하여 분만한 경우} \textcolor{red}{중재술을 하게 된 원인이 되는 병태}를 주된 병태로 부여한다.
		\begin{mdframed}[linecolor=blue,middlelinewidth=2]
			\begin{itemize}\tightlist
			\item 임신성당뇨로 유도분만하였으나 7시간 진통후 CPD로 제왕절개하여 건강한 아이 분만한 경우 :  주된병태 : O65.4 (CPD) 기타병태 : O24.4 (GDM), Z370 (Single live birth) 
			\item Full dilatation후 태아의 P position으로 vaccume extraction시도 했으나 태아의 하강이 없어 흡입분만 포기하고 제왕절개분만한 경우 :  주된병태 : O64.0 (Obstructed labour d/t incomplete rotation of fetal head) 기타병태 : O66.5 (상세불명의 집게및 진공흡착기 적용실패), Z370 (Single live birth) .\index{진단코드!POPP}
			
			\end{itemize}
		\end{mdframed}
	\item \uline{기구의 도움을 받지 않고 질식분만을 하였으나 산모가 출산전 산전병태로 입원}하였다면 산전병태를 주된병태로 분류한다. \textcolor{red}{유도분만의 이유가 주 진단이다}
		\begin{mdframed}[linecolor=blue,middlelinewidth=2]
		\begin{itemize}\tightlist
		\item 임신성고혈압으로 induction delivery하여 1st degree laceration있는 경우는  : 주된병태 : O13 (PIH) 기타병태 : O70.0 (1st degree laceration), Z370 (Single live birth) 
		\end{itemize}
		\end{mdframed}
		
	\item \uline{정상 분만진통으로 입원하여 정상질식분만을 한경우}에는 \textcolor{red}{분만을 주진단으로} 선정한다.
	\begin{mdframed}[linecolor=blue,middlelinewidth=2]
	\begin{description}\tightlist
	\item[O800] 자연두정태위분만
	\item[O801] 자연둔부태위분만
	\item[O814] 진공흡착기분만
	\item[O840] 모두질식분만에의한 다태분만
	\item[O842*] 모두제왕절개에의한 다태분만
	\end{description}
	 \end{mdframed}
	 
	\item 즉 \uline{쌍둥이로 제왕절개분만을 한 경우}는 주된병태는 \textcolor{red}{제왕절개를 한 이유}
	\begin{mdframed}[linecolor=blue,middlelinewidth=2]
	즉. 하나 이상의 태아의 태위장애를 동반한 다태임신의 산모관리(O32.5)이고, 기타병태로 O32.1(Bx 산모관리), O30.0(쌍둥이임신), Z37.2(쌍둥이, 둘 다 생존 출생)등이 있게 된다..\index{진단코드!쌍둥이 제왕절개}
	\end{mdframed}
	\item 분만문제가 진통 전에 발견되었는지, 아니면 진통후에 발견되었는지의 여부에 따라서 `O32-O34'또는 `O64-O66'으로 코딩한다.
		\begin{mdframed}[linecolor=blue,middlelinewidth=2]
		\begin{itemize}\tightlist
		\item 둔위로 Elective c-sec를 하게된경우는 ? 주 진단명이 O32.1 (둔부태위의 산모관리)이지, O64.1 (둔부태위로 인한 난산)이 아니다.
		\item 1분간 지속된 견갑난산을 가진 여아를 질식 분만하였다. 주진단명은 \dotemph{O66.0 (어깨난산으로 인한 난산)} 이다. 
		\end{itemize}
		\end{mdframed}
		
	\item 이전 제왕절개에 따른 분만의 경우에 넣는 코드는 다음과 같다.\index{진단코드!선행제왕절개}
		\begin{mdframed}[linecolor=blue,middlelinewidth=2]
		\begin{description}\tightlist
		\item[O75.7] 이전 제왕절개후 질분만
		\item[O66.4] 상세불명의 분만 시도의 실패 : 위의 두 경우는 TOL(Trial of Labor)를 시도하다가 성공하거나 실패한 경우이고 
		\item[O34.20] 이전의 제왕절개로 인한 흉터의 산모관리
		\item[O34.28] 이전의 기타 외과수술로 인한 자궁흉터의 산모관리 : 위의 두 경우는 Elective로 repeate c-sec를 한 경우로 \dotemph{분만문제가 진통전에 발견되었기 때문에 O32-O34를 쓴다는 원칙을 따른것임.}
		\end{description}
		\end{mdframed}
		
	\item O80-O82의 분류 : 이 코드는 기록되어 있는 정보가 분만이거나 분만 방법에 대해서만 국한되어 있을때 제한적으로 주된병태의 코드로 사용할 수 있다.또한 아무런 문제 없이 정상적인 분만을 하였을때 O80으로 할수 있다.
		\begin{mdframed}[linecolor=blue,middlelinewidth=2]
		\begin{itemize}\tightlist
		\item 산모가 만삭 정상분마을 위해 입원하여 산전, 분만중, 산후에 아무런 합병증이 없고, 기구나 기술을 필요로 하지 않고 정상분만을 한 경우 O80코드를 주진단명으로 할 수 있다.
		\item 특별한 합병증없이 다태분만을 한 경우 주된병태는 ``쌍둥이임신(O30.0)"으로 코드를 부여한다. ``O840" 모두 자연적인 다태분만은 분만의 방법을 나타내 주기 위하여 임의적인 추가코도로 부여 할 수 있다.
		\item 제왕절개술을 받은 경우 선택적이던 응급이던 상관없이 제왕절개술을 받은 이유를 주된병태로 우선 부여한다. 하지만 제왕절개술을 받은 이유가 불명확할 경우 ``O82.-제왕절개에 의한 단일 분만" 코드를 주된 병태로 부여 할 수 있다. \emph{현재와 같이 Repeated c-sec의 이유가 되지는 않는다.}
		\end{itemize}
		\end{mdframed}
	\item 만약 유도분만중 수술한 경우에는 수술한 원인이 주된 병태입니다. 그러나 induction failure는 주된 병태가 될수 없고, induction failure가 생긴 이유가 주된병태입니다. (예로 CPD나 first stage prolongation등)
	\item 모든 분만산모의 경우에는 신생아의 상태를 부진단으로 한다.
		\begin{mdframed}[linecolor=blue,middlelinewidth=2]
		\begin{description}\tightlist
		\item[Z370] 단일생산아 (single liveborn)
		\item[Z371] 단일사산아 (Single stillbirth)
		\item[Z372] 쌍둥이, 둘다 생존 출생 (Twins, both liveborn)
		\end{description}
		\end{mdframed}
	\end{enumerate}
%\item 제왕절개는 \dotemph{제왕절개의 적응증}을 주진단으로 한다. 
\item 입원환자 치료 중 기저질환이 밝혀지면 기저질환을 주진단으로 선정한다. (ex : 병명 - O001 난관임신)
\item 기저질환이 입원 시 알려져 있고, 문제에 대해서만 치료가 이루어지면 그 문제를 주진단으로 선정한다. (ex : 병명 - N833 자궁경부무력증)
\item 급만성이 동시에 발생한 경우 \emph{급성질환}을 주진단으로 선정한다. (ex : 병명 – O140 중등도의 전자간(급성), O249 임신성 당뇨(만성))
\end{itemize}

\section{DRG에서 주진단에 대한 논란}
\begin{hemphsentense}{산부인과 ``DRG 손실 커" 복지부 ``새 패러다임 협조"}
\href{https://dailymedi.com/news/view.html?section=1&category=5&no=771918}{주진단에 대한 복지부의 입장}
산부인과가 포괄수가제 병원급 확대 적용 이후 제왕절개술 주진단명 코딩 방법 변경에 따라 막대한 손실을 입고 있다는 우려가 제기됐다.
대한산부인과학회는 27일 제99차 학술대회를 개최하고 포괄수가제 등 ‘산부인과 건강보험의 과제’에 대해 논의하는 시간을 가졌다. 관동의대 산부인과 민응기 교수는 “우여곡절을 겪으면서 지난 7월부터 모든 의료기관을 대상으로 7개 질병군에 대한 포괄수가제 강제적용이 시작됐다”며 “건강보험심사평가원에 제왕절개술의 보험급여를 청구하면서 심각한 문제가 발생했다”고 전했다.\\
\textcolor{blue}{1997년 2월 포괄수가제 시범사업을 시작하면서부터 올 6월까지 제왕절개의 ‘주진단’명은 O820-O829로 코딩을 하고 제왕절개술을 한 주 사유를 `기타진단’으로 코딩해 중증도 보정을 받아왔다.} 또한 2007년도 포괄수가제 실무지침서에서도 똑같이 주진단을 O820-O829로 코딩하는 청구방법을 공지했고 그렇게 시행해왔다.\\
포괄수가제를 전면 시행하기로 한 지난 7월 1일 직전인 6월 15일 심평원 교육자료에서도 마찬가지였다.민 교수는 “그러나 7월 포괄수가제 전면실시 후 심평원에서는 \textcolor{red}{갑자기 한국표준질병사인분류 질병코딩지침서에 의거해 O820-O829를 주진단이 아닌 ‘기타진단’으로 코딩해 제왕절개술을 시행한 주 사유를 ‘주진단’명으로 코딩해야 한다고 불과 보름 만에 말을 바꿨다”고 지적했다.} 이로 인해 많은 환자에서 중증도가 반영되지 않아 제왕절개술을 시행한 모든 의료기관은 큰 손실을 보게 됐다는 것이다. 학회에 따르면 실제 의료기관이 똑같은 진단명으로 제왕절개술을 하더라도 평균 입원일수를 7일로 산정할 때 단태아는 10만1810원-26만1350원, 다태아의 경우 25만7170원-41만3510원의 금액을 손해 볼 수밖에 없다는 분석이다. 그는 “분만 전문 병의원의 경우 어림잡아 연간 수천만원에서 억대에 이르는 순익 손실을 보게 되는 금액”이라면서 “심평원은 갑작스런 코딩 방법 변경에 대한 납득할만한 해명을 내놓지 못하고 있다”고 강조했다.\\
민 교수는 이어 \uline{“오랜 기간 동안 시행해 온 대로 기존 코딩방법을 유지해야 한다”면서도 “기존 방법에 문제가 발견됐다면 그 틀을 벗어나지 않는 범위에서 문제점 해결 방안을 찾거나 중증도 반영이 달라지지 않도록 서둘러서 보완을 해야 할 것”이라고 피력했다.}\\
복지부 ``과거 방식 고치는데 있어 나타난 전환기적 불편함" 양해 구해\\
보건복지부는 이에 대해 과거 문제가 있었던 부분을 고치는 과정에서 발생한 문제점이라면서 바람직한 정착에 노력하겠다는 입장이다. 보건복지부 보험급여과 배경택 과장은 “과거 방식에 문제가 있어 고치는 과정에서 나타난 전환기적 불편함이라 생각한다. 심평원 업무처리가 더디게 느껴질 수 있으나 이는 복지부에서 의사결정 하는데 시간이 걸렸기 때문”이라고 양해를 구했다. 배 과장은 이어 “바람직하게 개선하는데 심평원, 복지부 모두가 노력할 것”이라고 말했다. 이와 함께 산부인과가 포괄수가제라는 새로운 패러다임을 구축하는데 동행자가 돼 줄 것을 당부했다.\\
그는 “포괄수가제는 다른 패러다임”이라며 “산부인과가 기존 행위별수가제에 안주할 수 있을까에 대해 고민을 해봐야 한다. 새로운 패러다임을 만드는데 적극 참여해 긍정적으로 구축할 수 있도록 노력할 필요가 있다”고 덧붙였다.
\end{hemphsentense}
예를 들면 한국표준질병사인분류 질병코딩지침서에 의거해 제왕절개를 하게된 주된 사유인 Placenta previa를 주진단으로 하게되면, 중등도가 반영되지 않아서 손해를 보게된다고 하는데 이게 어떤 의미일까? \\ DRG에서 마지막 단계로 합병증 및 동반상병 분류로 각 질병군 범주의 특성에 따라 구분된 환자단위 중증도 점수별로 \uline{최종 질병군 분류번호를 결정 하게 된다}. 여기에 쓰이는 것이 기타진단(부진단)입니다. 다시말하면, \textcolor{red}{기타진단의 중요도는 기타진단(부진단)에 의해서 중등도가 결정되게 됩니다.}\\
이와 같은 이류로 \textcolor{red}{중등도(=부진단 NOT주진단)}는 보험급여금에 차이를 주는데요. 여태까지는 o820등의 선택적제왕절개등을 주진단으로 하고 주된이유를 부진단으로 해서 어려운수술에 대한 보전을 받았는데, placenta previa같이 어려운 수술을 해도 현 DRG 주진단 system에서는 보전받을수 없게 되었다는 것입니다. \\
물론 severe PIH로 제왕절개를 한경우에 주진단으로 severe PIH 한가지만 내 놓으면 기본적인 DRG금액만 받을수 있지만, CPD를 주진단으로 하고 severe PIH, hemorrhage등등의 기타진단을 기입하게 되면 severity등급에 따라서 최소 9만원이상의 금액을 더 받을수 있습니다. 실제로 따져봐도 severe PIH는 직접적으로 제왕절개를 해야할 주된 이유는 아닙니다. CPD나 failure to progression등이지요!!\\
\clearpage

\section{DRG의 이해}
\subsection{7개 질병군 포괄수가제란? (2013년 7월부터)}
환자가 입원해서 퇴원할 때까지 발생하는 진료에 대하여 질병마다 미리 정해진 금액을 내는 제도입니다. 입원비가 하나로 묶여있다고 생각하시면 됩니다. (\textcolor{red}{같은 질병이라도 환자의 합병증이나 타상병 동반여부에 따라 가격은 달라질 수} 있습니다.)\\

\emph{적용대상질병군}\\ 
현재는 4개 진료과 7개 질병군을 대상으로 시행중
\begin{itemize}\tightlist
\item 안과 : 백내장수술(수정체 수술) 
\item 이비인후과 : 편도수술 및 아데노이드 수술 
\item 외과 : 항문수술(치질 등), 탈장수술(서혜 및 대퇴부), 맹장수술(충수절제술) 
\item 산부인과 : \textcolor{blue}{제왕절개분만, 자궁 및 자궁부속기(난소, 난관 등)수술(악성종양 제외), 자궁외임신제외}
\end{itemize}
※ 수정체수술(백내장수술), 서혜 및 대퇴부 탈장수술(장관절제 미동반) 등 간단한 항문수술의 경우에는 6시간 미만 관찰 후 당일 귀가 또는 이송시에도 포괄수가제(DRG설명보기)가 적용되어 본인부담금은 입원부담률인 20\%로 적용받게 됩니다. 다만 7개 질병군에 해당되는 수술을 받았어도 \textcolor{red}{의료급여 대상자 및 혈우병 환자와 HIV감염자(인체면역결핍바이러스병)}는 포괄수가제(DRG) 적용에서 제외됩니다.

\begin{Cdoing}{우리나라 포괄수가제}
질병군(DRG) 포괄수가는 국민건강보험법시행령 제21조제3항제2호에 따라 복지부장관이 별도 고시하는 7개 질병군으로 입원진료를 받은 경우에 적용하며, 질병군 입원진료는 질병군 급여 일반원칙에 따라 다음의 항목을 포함하고 있습니다.\\
- 다 음 -
\begin{itemize}\tightlist
\item 7개 질병군으로 응급실·수술실 등에서 수술을 받고 연속하여 6시간 이상 관찰 후 귀가 또는 이송한 경우 
\item 7개 질병군 중 수정체수술(대절개 단안 및 양안, 소절개 단안 및 양안), 기타항문수술, 서혜 및 대퇴부탈장수술 단측 및 양측(복강경 이용 포함)의 수술을 받고 6시간 이상 관찰 후 당일 귀가 또는 이송한 경우
\end{itemize}
\end{Cdoing}

\subsection{질병군 급여 일반원칙}
\begin{enumerate}[1.]\tightlist
\item 상급종합병원, 종합병원, 병원(요양병원을 포함한다), 의원(보건의료원을 포함 한다)인 요양기관이 국민건강보험법 시행령(이하 “영”이라 한다) 제21조 제3항제2호 및 국민건강보험 요양급여의 기준에 관한 규칙(이하 “요양급여 기준”이라 한다) 제8조제3항에 따라 포괄적인 행위가 적용되는 질병군에 대한 입원진료를 하는 경우에 적용한다.
\item 가입자 또는 피부양자(이하 “가입자 등”이라한다)가 질병군으로 입원진료를 받은 경우에 적용하되, 다음의 각 항목은 질병군 적용에서 제외하고 제1편을 적용한다.
	\begin{enumerate}[가.]\tightlist
	\item 혈우병환자, HIV감염자
	\item 입원일수가 30일을 초과할 경우 31일째부터 발생하는 진료분
	\item 차상위 본인부담경감대상자로서 제3호 나목에 해당하는 경우
	\item 질병군 진료 이외의 목적으로 입원하여 입원일수가 6일을 초과한 시점에 예상치 못하게 질병군 수술이 이루어진 경우 입원일로부터 수술시행일 전일까지의 진료분
	\end{enumerate}
\item 제2호 규정에 따른 질병군 입원진료에는 다음의 각 항목을 포함한다.
	\begin{enumerate}[가.]\tightlist
	\item 제2부 각 장에 분류된 질병군으로 응급실\cntrdot{} 수술실 등에서 수술을 받고 연속하여 6시간 이상 관찰 후 귀가 또는 이송한 경우
	\item 제2부 각 장에 분류된 질병군 중 수정체 소절개 수술 단안, 수정체 소절개 수술 양안, 수정체 대절개 수술 단안, 수정체 대절개 수술 양안, 기타항문 수술, 서혜 및 대퇴부 탈장수술(장관절제 미동반) 단측, 서혜 및 대퇴부 탈장수술(장관절제 미동반) 양측, 복강경을 이용한 서혜 및 대퇴부 탈장수술(장관절제 미동반) 단측, 복강경을 이용한 서혜 및 대퇴부 탈장수술(장관절제 미동반) 양측 질병군으로수술을 받고 6시간 미만 관찰 후 당일 귀가 또는 이송하는 경우
	\end{enumerate}
\item 제2부 각 장에 분류된 질병군 상대가치점수(이하 “점수”라 한다)는 다음 각목의 행위\cntrdot{} 약제 및 치료재료를 포함한다.
	\begin{enumerate}[가.]\tightlist
	\item 제1편 행위 급여ᆞ비급여 목록 및 급여 상대가치점수에서 정한 행위 급여목록표에 고시된 행위
	\item 요양급여기준 제8조제2항의 규정에 의하여 고시된 약제 급여 목록 및 급여 상한금액표의 약제와 치료재료 급여.비급여 목록 및 급여 상한 금액표의 치료재료
	\item 요양급여기준 별표 2의 비급여대상 중 제6호의 비급여대상을 제외한 행위.약제 및 치료재료
	\item 국민건강보험법 시행규칙 별표 6의 본인이 요양급여비용의 100분의 100을 부담하는 항목 중 제1호 자목에 해당하는 항목을 제외한 행위\cntrdot{}약제 및 치료재료
	\item 다음 항목 중 위 가목 내지 라목에 해당하는 경우
		\begin{enumerate}[(1)]\tightlist
		\item 요양급여기준 별표 1 제1호 마목에서 장관이 정하는 바에 따라
다른 기관에 검사를 위탁하거나 당해 요양기관에 소속되지 아니한
전문성이 뛰어난 의료인을 초빙하거나, 또는 다른 요양기관에서
보유하고 있는 양질의 시설.인력 및 장비를 공동 사용하는 경우
소요되는 행위.약제 및 치료재료
		\item 입.퇴원 당일에 발생한 행위.약제 및 치료재료로써 외래진료 및 퇴원약제 등을 포함하되 다음 항목은 제외한다.
			\begin{enumerate}[(가)]\tightlist
			\item  질병군 입원을 예견하지 못한 상태에서 입원 당일 외래진료를 받은 경우의 원외처방 약제비
			\item 질병군으로 퇴원 후 질병군과 관계없는 상병으로 퇴원 당일 외래진료를 받은 경우의 원외처방 약제비
			\item 질병군으로 퇴원 후 질병군 질환과 관계없는 상병으로 퇴원 당일 재입원하는 경우의 요양급여비용
			\end{enumerate}
		\item 요양기관의 요구에 의하여 가입자 등이 외부에서 직접 구입한 약제 및 치료재료
		\end{enumerate}
	\end{enumerate}
\item 질병군에 대한 요양급여비용을 산정할 때에는 제2부 각 장에 분류된 질병군 점수를 기준으로 별표 1의 질병군별 점수 산정요령에 의하여 산정된 점수 총합에 국민건강보험법 제45조제3항과 영 제21조제1항에 따른 점수당 단가를 곱하여 10원 미만을 절사한 금액을 요양급여비용 총액으로 산정한다. 이 경우 위 금액 외에 별도로 산정하는 비용이 있는 경우에는 각각의 산정 방식에 의하여 산정된 금액을 합산한다.
\item 제5호 본문에도 불구하고 질병군별 금액 산정시 점수당 단가는 별표 2의 질병군 행위 및 약제.치료재료 구성비율에 따른 행위부분 점수와 매년 상한금액 변화를 적용한 약제.치료재료 금액을 점수당 단가로 나눈 점수를 합한 점수(소수점 이하 셋째 자리에서 4사5입)에 적용한다.\\
<산식>\\
질병군별 금액 = \{질병군별 행위 점수 + (약제.치료재료 금액 ÷ 점수당 단가)\} × 점수당 단가
\item 제5호에 따라 산정한 요양급여비용의 총액이 영 제21조제1항 내지 제3항 및 요양급여기준(별표 2 제6호를 제외한다)에 의하여 산정한 총액보다 적고 그 차액이 100만원을 초과하는 경우(이 경우를 요양급여비용열외군이라 한다)에는 위 제5호에 따른 금액에 100만원을 초과하는 금액(10원 미만 절사)을 합한 금액을 요양급여비용 총액으로 산정한다.
\item 가입자 또는 피부양자가 제1호에 따른 요양기관(제3편을 적용받는 요양병원은 제외)에서 「국민건강보험법」 제43조에 따라 신고한 일반입원실 및 정신과폐쇄병실의 4인실 또는 5인실을 이용한 경우에는 별표 2의3의 추가비용 계산식에 따른 금액을 추가 산정하고, 상급종합병원의 일반입원실 및 정신 과폐쇄병실의 1인실(보건복지부장관이 정하여 고시하는 불가피한
1인실 입원의 경우 제외)을 이용한 경우에는 제5호 본문에 따른 금액에서
1인실 이용일수에 해당하는 기본입원료(제1편제2부제1장 가-2-가)를
제외하고 산정한다.
\item 영 별표 2 제2호 나목의 “보건복지부장관이 정하여 고시하는 입원실을
이용한 경우”라 함은 가입자 등이 제1호에 따른 요양기관에서 국민건강보험법
제43조에 따라 신고한 일반입원실 및 정신과폐쇄병실의 4인실 또는 5인실을
이용한 경우를 말하며, 별표 2의3의 본인부담액 계산식에 따른 금액을 더하여
본인부담액을 산정한다.
\item 영 별표 2 제2호 다목의 “그 고시에서 정한 금액”이라 함은 제7호 중
100만원 초과분에 해당하는 금액을 말한다.
\item (별표 2의1)에 열거한 항목을 외과 전문의가 시행한 경우에는 소정점수의
30\%에 대한 각 요양기관별 종별가산율을 적용한 금액을 추가 산정한다.
\item 18시-09시 또는 공휴일에 응급진료가 불가피하여 수술을 행한 경우에는
해당 질병군의 야간.공휴 소정점수를 추가 산정한다. 이 경우 수술 또는
마취를 시작한 시간을 기준으로 산정한다.
\item 질병군 요양급여를 실시하는 요양기관은 질병군 입원환자의 질병군
분류번호와 관련한 주진단 및 기타진단, 수술명 등은 진료기록부에 근거하여
정확한 코드를 부여하여야 하며, 진단명이 입원시부터 존재하였는지 여부를
확인할 수 있도록 진료기록부에 기록하고, 의료의 질 향상을 위한
점검표를 별지 서식에 따라 작성하여야 한다.
\item 입원 중인 환자를 제2부 각 장에 분류된 질병군 중 수정체 소절개 수술 단안,
수정체 소절개 수술 양안, 수정체 대절개 수술 단안, 수정체 대절개 수술
양안의 진료를 위해 다른 요양기관으로 의뢰하여 질병군 진료를 실시한
경우 해당 요양급여비용은 의뢰받은 요양기관에서 질병군으로 적용한다.
\item 질병군 진료 시 초음파검사는 「요양급여의 적용기준 및 방법에 관한
세부사항」 제2장 검사료 초음파검사 세부인정기준을 적용하며, 인정기준에
의한 급여대상에 해당되는 경우에는 제2부 각 장에 분류된 질병군 점수
이외에 제1편 제2부 초음파검사료를 추가 산정한다.
\item 별표 2의4에 열거한 항목에 해당하는 행위 및 치료재료는 제1편 제2부
행위 급여 상대가치점수와 「약제 및 치료재료의 비용에 대한 결정기준」에
의한 금액을 추가 산정한다.
\item 영 별표 2 제4호에 따른 요양급여 항목 및 본인부담률은 별표 2의5와 같다.
이 경우 별표 2의5에 열거한 항목에 해당하는 행위 및 치료재료는
「요양급여의 적용기준 및 방법에 관한 세부사항」을 적용하며, 인정기준에
의한 급여대상에 해당되는 경우에는 제1편 제2부 행위 급여 상대가치점수와
「약제 및 치료재료의 비용에 대한 결정기준」에 의한 금액을 추가 산정한다.	
\item 질병군 진료시 마취통증의학과 전문의를 초빙하여 마취를 실시한
경우에는 제1편제2부제6장 바-2의 마취통증의학과 전문의 초빙료를
추가 산정하며, 제1편제2부제6장 및「요양급여의 적용기준 및 방법에
관한 세부사항」의 마취통증의학과 전문의 초빙료 산정 관련 규정을
적용한다.
\item 질병군 진료시 질병군 분류번호를 결정하는 주된 수술 이외에 제1편
제2부제9장제1절(기본처치 제외) 또는 제10장제3절.제4절의 수술을
실시한 경우에는 해당 수술 소정점수를 추가 산정한다. 다만, \textcolor{red}{주된 수술과
동일 피부 절개 하에 실시되는 수술은 해당 수술 소정점수의 70\%를 산정
한다.}
\end{enumerate}
\subsection{질병군 분류번호를 결정하는 주된 수술 이외에 수술을 실시한 경우 수기료 추가 산정방법}
(보건복지부 고시 제2015 – 26호(2015년 1월 30일 )\par
질병군 분류번호를 결정하는 주된 수술 이외에 제1편제2부제9장제1절(기본처치 제외) 및 제10장제3절, 제4절의 수술을 실시한 경우의 추가 산정 방법은 다음과 같이 한다.\par
- 다 음 -
\begin{enumerate}[1.]\tightlist
\item 질병군 진료 중 질병군 분류번호를 결정하는 주된 수술과 날을 달리하여 실시하는 수술도 포함함
\item \textcolor{red}{해당 수술 항목의 소정점수만을 산정하고, 야간·공휴 가산 등을 포함한 모든 가산은 적용하지 아니함}
\item 아래의 경우는 추가 산정하지 아니함
	\begin{enumerate}\tightlist
	\item 합병증 혹은 처치 중의 우발적 천자 및 열상 등으로 실시한 수술
	\item 수정체수술 질병군과 동시에 실시한 유리체흡인술(자505), 유리체내주입술(자507), 유리체절제술-부분절제(자512-나)
	\item 편도절제술과 동시에 실시한 아데노이드절제술(내시경하에서 실시한 경우 포함)
	\item 기타 또는 주요 항문수술 질병군에 해당하는 수술을 2개 이상 실시한 경우
	\end{enumerate}
\item  위 1부터 3까지에서 정하고 있지 않은 내용은 「건강보험 행위 급여\cntrdot{} 비급여 목록표 및 급여 상대가치점수」 제1편 제2부 제9장, 제10장 및 「요양급여의 적용기준 및 방법에 관한 세부사항」Ⅰ. 행위 제9장, 제10장을 적용한다.
\end{enumerate}

\subsection{질병군 적용지침(산부인과)}
제4장 산부인과
\begin{enumerate}[(1)]\tightlist
\item 요양기관종별로 「복강경을 이용한 자궁적출술(악성종양제외)」,「기타 자궁적출술(악성종양제외)」,「복강경을 이용한 기타 자궁 수술(악성종양제외)」,「기타 자궁 수술(악성종양제외)」,「복강경을 이용한 자궁부속기 수술(악성 종양제외)」,「자궁부속기 수술(악성종양제외)」,「제왕절개분만(단태아)」,
「제왕절개분만(다태아)」의 각 질병군 소정점수를 적용한다.
\item 위 “1”의 규정에도 불구하고 「복강경을 이용한 기타 자궁 수술(악성종양 제외)」,「기타 자궁 수술(악성종양제외)」,「복강경을 이용한 자궁부속기
수술(악성종양제외)」,「자궁부속기 수술(악성종양제외)」의 각 질병군에 해당하는 수술을 실시한 경우 해당 질병군의 가산점수를 산정한다. 다만, 절개생검(심부[장기절개생검]-개복에 의한 것, 나-853-나-2), 유착성자궁부속기 절제술(자-433)과 난소를 전적출하는 부속기종양적출술([양측]-양성, 자-442-가)은
가산점수를 산정하지 아니한다.
\item 제왕절개분만(단태아)」,「제왕절개분만(다태아)」질병군 대상 중 출혈로 인해 혈관색전술(기타혈관, 자-664-나), 자궁내 풍선카테터 충전술[자궁용적
측정 포함](자-402-3)을 실시한 경우 질병군 점수를 적용하지 아니하며 제1편을 적용한다.
\item 각 질병군은 동 질병군에 해당하는 수술의 종목수 및 편ㆍ양측 수술에 불문하고 해당 소정점수를 적용한다.
\item 복강경을 이용한 수술 중 부득이한 사유로 중도에 개복술로 전환하여 수술을 종결한 경우에는 복강경을 이용하지 아니한 질병군에 해당하는 소정점수를 적용하고 복강경 등 내시경하 수술시 보상하는 239,000원(100분의 20에 해당하는 47,800원은 본인부담)의 금액을 추가 산정한다.
\item 제4부 비급여 목록 2. 신의료기술등의 비급여 제9장 처치 및 수술료 등의 (1) 다빈치 로봇 수술을 실시한 경우에는, 제2편제1부제5호에 따라 복강경을 이용한 자궁 및 자궁부속기 수술 질병군 요양급여비용의 총액에서 별표2-2의 질병군별 다빈치 로봇 수술시 제외금액표의 금액을 제외하고
산정한다. 다만, 야간\cntrdot{} 공휴 및 “2.” 등의 가산은 적용하지 아니한다.
\item 자궁근종, 자궁선근증에 초음파 유도하 고강도초음파집속술(조-566)을 실시한 경우 질병군 점수를 적용하지 아니하며 제1편을 적용한다.(2015년 9월 신설)
\end{enumerate}

\begin{mdframed}[linecolor=blue,middlelinewidth=2]
마취통증의학과전문의초빙료는 2015년 1월부터 산정방법이 변경되었으므로 마취과전문의를 초빙하여 7개 질병군을 수술한 병ㆍ의원은 해당 수가의 100\%를 별도 산정할 수 있음을 알려드립니다.\par
※ 마취통증의학과전문의초빙료 : 의원(105,290원), 병원(99,060원)
\end{mdframed}
\subsection{자414 \newindex{전자궁적출술과 난소 또는 난관 종양적출술 동시 시술시 별도 인정여부}}
\begin{mdframed}[linecolor=blue,middlelinewidth=2]
난소 또는 난관에 종양이 있어 전자궁적출술과 동시에 이를 적출하였을 경우에 자414 전자궁적출술과 자442가 부속기종양적출술을 제9장 처치 및 수술료 등[산정지침] (6)항에 의거 주된 수술 100\%, 그외 수술 50\%[종합병원(상급종합병원 포함)은 70\%]를 산정함.(2014년 8월)
\end{mdframed}

\begin{Cdoing}{임신력보존에 가산에 대한 새로운 고시, 고시제2014-240호(2014.12.30)}
복강경을 이용한 기타 자궁 수술(악성종양제외)」,「기타 자궁 수술(악성종양제외)」,「복강경을 이용한 자궁부속기 수술(악성종양제외)」,「자궁부속기 수술(악성종양제외)」질병군의 가산점수는 진료담당의사의 의학적 판단 하에 임신·출산능력을 보존하는 수술을 시행한 경우 산정함을 원칙으로 하며 인정기준은 다음과 같이 함 \\
- 다  음 - 
\begin{enumerate}[ 가.]\tightlist
\item  임신·출산을 담당하는 장기의 병변 부위만을 제거·교정하는 수술을 하여 임신·출산능력을 보존한 경우  다만, \textcolor{red}{자궁내막증이 있거나 불임(또는 난임) 등으로 임신가능성을 높이기 위해 난소 또는 난관 전절제술을 실시한 경우는 사례별로 인정}
\item 임신·출산을 담당하는 장기의 수술을 동시에 실시하여 그 수술결과로 임신·출산능력이 보존된 경우
\item 아래의 경우는 가산점수를 산정하지 아니함
	\begin{enumerate}[(1)]\tightlist
	\item  폐경 또는 55세 이상 여성(55세 이상이나 폐경이 아닌 경우 관련자료 첨부시 이를 참조하여 인정)
	\item  기존에 시행한 수술로 임신·출산 능력을 상실한 경우
	\end{enumerate}
\end{enumerate}	
\end{Cdoing}

\begin{hemphsentense}{환자가 전액 부담하는 항목}
단순피로 등 일상생활에 지장이 없는 질환, 미용목적, 본인희망의 건강검진 등 예방진료, 상급병실료 차액, 선택진료료, 초음파 등 
응급진료를 위하여 앰블런스를 이용하면서 받는 응급의료 이송처치료, 각종 수술 후 통증관리를 위한 자가통증조절법(PCA, 무통주사)에소용된 비용\\

비급여
  
제2장 검사료
  
제1절 검체검사료
  
(1)양수 아세틸콜린에스터라제 Amniotic Fluid Acetylcholinesterase
  
(2) 성호르몬결합글로불린 Sex Hormone Binding Globulin
\end{hemphsentense}  

\subsection{질병군 분류과정}
\includegraphics[scale=.425]{DRGDisease}
\subsubsection{제왕절개분만}
\begin{enumerate}[가.]\tightlist
\item 임신,분만,산욕 주진단범주
	\begin{itemize}\tightlist
	\item O016 제왕절개분만(단태아) Cesarean Delivery(First Fetus)
	\item O017 제왕절개분만(다태아) Cesarean Delivery(Multiple)
	\end{itemize}
\item 주진단 : 「임신, 분만, 산욕 주진단범주」에 분류된 주진단
	
\item 외과계 시술
	\begin{itemize}\tightlist
	\item O016 제왕절개분만(단태아) Cesarean Delivery(First Fetus)\\
	 자451가1가 R4517 제절개만출술(1태아임신의 경우)-초회(초산)\\
	 자451가1나 R4518 제왕절개만출술(1태아임신의 경우)-초회(경산)\\
	 자451가2 R4514 제왕절개만출술(1태아임신의 경우)-반복
	\item O017 제왕절개분만(다태아) Cesarean Delivery(Multiple)\\
	 자451나1가 R4519 제왕절개만출술(다태아임신의 경우)-초회(초산)\\
	 자451나1나 R4520 제왕절개만출술(다태아임신의 경우)-초회(경산)\\
	 자451나2 R4516 제왕절개만출술(다태아임신의 경우)-반복
	\end{itemize}
\end{enumerate}

\subsection{주진단과 기타진단에 대한이해}
\begin{enumerate}[가.]\tightlist
\item 주진단
	\begin{enumerate}[(1)]\tightlist
	\item 한번 입원한 건에 대하여는 하나의 주진단을 부여한다. 둘이상의 병태가 주진단 정의에 똑같이 부합될 때는 둘 중 어느 진단을 선택하여도 무방하나 하나의 진단만을 주진단으로 부여한다.
	\item 비급여대상 질환(「국민건강보험 요양급여의 기준에 관한 규칙」별표2 제6호에 해당하는 질환)이 주진단에 해당될 경우는 기타진단 중 가장 주된 진료를 받은 진단을 주진단으로 선정한다.
	\item 진단이 확립되지 아니한 경우 \dotemph{의심되는 진단(의증)을 주진단으로 부여할 수 있다.} 입원기간 중 생성된 진단 정보가 없어서 진료 후에도 주진단이 여전히 ‘의심되는’, ‘의문나는’ 등으로 기록되어 있는 경우 의심되는 진단을 확진된 것처럼 부여할 수 있다.
	\end{enumerate}

\item \newindex{기타진단}
\begin{enumerate}[(1)]\tightlist
\item \dotemph{확립된 진단만 부여하고 의심되는 진단(의증)은 기타진단으로 부여 하지 아니한다.} 기타진단은 확진된 경우만 부여할 수 있으며, 의심되는 진단(의증)은 부여하지 아니한다. 의심되는 진단(의증)의 경우는 그 진단과 관련되는 증상 및 증후〔ⅩⅧ장. 달리 분류되지 않은 증상, 징후와 임상 및 검사의 이상 소견에 해당되는 분류기호로 부여하여야 한다.
\item 비급여 대상 질환은 기타진단으로 부여하지 아니한다.
\item 이번 입원과 관련 없는 이전 병태는 기타진단으로 부여하지 아니한다. 진료기록부의 최종진단명란에 기재되어 있는 진단명은 주진단 이외 에는 일반적으로 모두 기타진단으로 간주할 수 있으나, 그 중 과거의 진료 또는 병력에 해당되는 병태로서 이번 입원과 관련 없는 경우는 기타진단으로 부여하지 아니한다.
\item 전신적인 만성질환은 기타진단으로 부여할 수 있다. 고혈압, 파킨슨병, 당뇨병\footnote{하지만 고혈압,당뇨등도 합병증이 없는 경우에는 기타진단.즉 중등도에 올라가기는 힘들다. severity점수가 없다} 등과 같은 만성질환은 지속적인 임상적 평가, 추가적인 간호 및 관찰이 요구될 수 있으므로 기타진단으로 부여할 수 있다.
\item 질병진행 과정중의 한 부분으로의 병태는 기타진단으로 부여하지 아니한다. 질병의 진행과정에 반드시 수반되는 병태는 기타진단으로 별도 부여하지 아니한다.
\item \uline{비정상적인 검사결과만으로(진료의가 임상적인 의미를 부여하지 않은 경우) 기타진단으로 부여하지 아니한다.}
\end{enumerate}
\end{enumerate}
\begin{shaded}
기타진단의 중요도는 기타진단에 의해서 중등도가 결정되게 됩니다. 예를 들어보면, severe PIH로 제왕절개를 한경우에 주진단으로 severe PIH 한가지만 내 놓으면 기본적인 DRG금액만 받을수 있지만, CPD를 주진단으로 하고 severe PIH, hemorrhage등등의 기타진단을 기입하게 되면 severity등급에 따라서 최소 9만원이상의 금액을 더 받을수 있습니다. 실제로 따져봐도 severe PIH는 직접적으로 제왕절개를 해야할 주된 이유는 아닙니다. CPD나 failure to progression등이지요!!\\
to be continue--
\end{shaded}
\subsection{고위험 기타진단 산정전/산정후 비교}
-조건 : 8박 9일/6인실/식사제외된 사항입니다.\\
\noindent

\tabulinesep =_2mm^2mm
\begin {tabu} to\linewidth {|X[4,c]|X[3,c]|X[3,c]|X[3,c]|} \tabucline[.5pt]{-}
\rowcolor{ForestGreen!40}  & \centering 본인부담 & \centering 공단부담 & \centering 총진료비 \\ \tabucline[.5pt]{-}
\rowcolor{Yellow!40} 산정전(O01600) & 394,834 & 1,388,528 & 1,783,362  \\ \tabucline[.5pt]{-}
\rowcolor{Yellow!40} 산정후(O01601-3) & 408,219 & 1,465,144 & 1,873,363 \\ \tabucline[.5pt]{-}
\rowcolor{Yellow!40} 차 액 & 13,385 & 76,616 & 90,001  \\ \tabucline[.5pt]{-}
\end{tabu}

\begin{shaded}
분만전 hct 37.9\%, 제왕절개분만 후 Hct 34.0\%로 10\%이상 Hct감소한 경우이나 환자의 심신상태 등이 양호하여 특별한 처치\cntrdot{}치료를 필요로 하지 않는 경우 O72 분만 후 출혈을 기타진단으로 코딩함음 오류임(복지부고시 \snm{기타진단부여기준(별표8)}에 맞지 않음)\\

현재 까지 알아낸 특별한 처치들이란?
Nalador usage, 부르탈 usage, GDM에서 BST check등...
\end{shaded}

\subsection{합병증 및 동반상병 분류 결정 단계}
\begin{enumerate}[(1)]\tightlist 
\item 기타진단의 중증도 점수
	\begin{itemize}\tightlist
	\item 합병증 분류에 이용되는 기타진단은 진단별로 2∼4까지의 중증도 점수를 갖는다.(「4.기타진단의 중증도 점수」 참조)
	\item 주진단 및 기타진단간 상호 연관성이 높은 기타진단은 중증도 점수가 2점 이상이더라도 0점이 된다.(「5.기타진단의 중증도 점수를 0으로 결정되게 하는 주진단」 참조)
	\end{itemize}
\item 환자단위 중증도 점수
	\begin{itemize}\tightlist
	\item 최종적으로 중증도 점수를 갖는 여러 개의 기타 진단들이 있을 경우 이를 통합하여 환자단위 중증도 점수를 결정하게 된다. 환자단위 중증도 점수는 아래와 같은 공식을 이용해서 계산된다.
	\item 환자단위의 중증도점수는  = 0 if there is no 기타진단, = 4 if x >4 , = x otherwise
	\item 점수의 의미는 0 : no CC effect, 1 : minor CC, 2 : moderate CC, 3 : severe CC, 4 : catastrophic CC 입니다. ※ CC(Complication and Comorbidity) : 합병증 및 동반상병
	\end{itemize}

\item 질병군범주별 합병증 및 동반상병 분류
	\begin{itemize}\tightlist
	\item 합병증 및 동반상병 분류의 마지막 단계로 각 질병군 범주의 특성에
따라 구분된 환자단위 중증도 점수별로 최종 질병군 분류번호를 결정
하게 된다. [표1 참조]
	\end{itemize}
\end{enumerate}

\begin{figure}
\centering
\includegraphics{severity2}	
\caption{다른 부인과나 산과의 질환의 경우는 severity에 따라서 2-3단계만 있는데에 비해서, 단태아제왕절개를 보면 severity에 따라서 4단계나 나누어 지는것을 볼수 있습니다. 중증도점수가 올라가면 갈수록 급여받을수 있는 금액이 올라갑니다. 한단계마다 거의 10만원가까이 올라갑니다.}
\end{figure}

\subsection{DRG에서 \newindex{임신력보존 수술에 대한 가산}인정(신설)}
\begin{description}\tightlist
\item[구분코드] MT041
\item[특정내역] 산부인과 가산점수 산정(**)
\item[특정내역기재형식] X(1)
\item[설명] 건강보험 행위급여\cntrdot{}비급여 목록표 및 급여 상대가치점수 제2편 질병군 급여\cntrdot{}비급여 목록 및 급여 상대가치점수 제2부 제4장 산부이과 적용지침2. 에 땨라 산부인과 가산점수를 산정한 경우 `Y'를 기재
\end{description}

\subsection{\newindex{임신력보존에 가산에 대한 새로운 고시}}, \mycoloredbox{고시제2014-240호(2014.12.30)}
복강경을 이용한 기타 자궁 수술(악성종양제외)」,「기타 자궁 수술(악성종양제외)」,「복강경을 이용한 자궁부속기 수술(악성종양제외)」,「자궁부속기 수술(악성종양제외)」질병군의 가산점수는 진료담당의사의 의학적 판단 하에 \uline{임신·출산능력을 보존하는 수술을 시행한 경우 산정함을 원칙으로 하며 인정기준은 다음과 같이 함}\par
- 다 음 -
\begin{enumerate}[가.]\tightlist
\item 임신·출산을 담당하는 장기의 병변 부위만을 제거·교정하는 수술을 하여 임신·출산능력을보존한 경우 다만, 자궁내막증이 있거나 불임(또는 난임) 등으로 임신가능성을 높이기 위해 난소 또는 난관 전절제술을 실시한 경우는 사례별로 인정
\item 임신·출산을 담당하는 장기의 수술을 동시에 실시하여 그 수술결과로 임신·출산능력이 보존된 경우
\item  아래의 경우는 가산점수를 산정하지 아니함
	\begin{enumerate}[(1)]\tightlist
	\item 폐경 또는 55세 이상 여성(55세 이상이나 폐경이 아닌 경우 관련자료 첨부시 이를 참조
하여 인정)
	\item 기존에 시행한 수술로 임신·출산 능력을 상실한 경우
	\end{enumerate}
\end{enumerate}	
\clearpage


\section{DRG 기타진단의 진단기준}
\subsection{유의한 단백뇨를 동반하지 않은 임신성(임신-유발성)고혈압(O13)}\label{SUPPIH}
 : 정상혈압을 갖고 있던 여성에서 임신20주 이후에 수축기 혈압이 140 mmHg 이상이거나 확장기 혈압이 90mmHg 이상, 6시간 이상의 간격으로 최소한 2번 이상 증명되고 분만 후까지 단백뇨가 동반되지 않고 고혈압으로 남아 있는 경우

\subsection*{임신중독증(O14)}\label{PIH}
 : 아래의 혈압과 단백뇨의 조건이 모두 충족되는 경우
\begin{itemize}\tightlist
\item 혈압 : 정상혈압을 갖고 있던 여성에서 임신20주 이후에 수축기 혈압이 140mmHg 이상이거나 확장기 혈압이 90mmHg 이상, 6시간 이상의 간격으로 최소한 2번 이상 증명된 경우
\item 단백뇨 : 6시간 이상의 간격으로 2+이상(또는 100mg/dl이상) 2번 이상 증명된 경우 또는 24시간 요중에 단백질이 300mg 이상 존재가 확인된 경우
\end{itemize}

\subsection*{중증의 전자간 (O141)}\label{severePIH}
 - 전자간증이면서 다음의 기준 중 1개 이상 충족
\begin{itemize}\tightlist
\item 환자가 침상 안정 상태에서 적어도 6시간 간격으로 2회에 걸쳐 수축기 혈압 160mmHg 이상 또는 확장기 혈압 110mmHg 이상
\item 24시간 채뇨 소변에서 5gm이상의 단백뇨 또는 적어도 4시간 간격으로 2회 채뇨 점적뇨에서 3+이상
\item 24시간 500ml 이하의 핍뇨
\item 대뇌 장애 또는 시력 장애
\item 폐부종 또는 청색증
\item 상복부 또는 우상복부통증
\item 간기능 장애
\item 혈소판 감소증
\item 태아발육지연
\end{itemize}

\subsection*{헬프(HELLP) 증후군 (O142)}\label{HELLP}
 - 다음의 기준을 모두 충족
\begin{itemize}\tightlist
\item 용혈(hemolysis) : Abnormal peripheral blood smear (microangiopathic anemia), Increased bilirubin ≥ 1.2mg/dl, Increased LDH> 600 IU/L
\item 간효소치 상승 : Increased AST ≥ 72 IU/L, Increased LDH as above
\item 저혈소판혈증 Platelet count < 100×103/μl
\end{itemize}

\subsection*{자간증(O15)}\label{eclampsia}
 :임신중독증(O14)의 조건을 충족하면서 임신성 고혈압에 의해 경련(Convulsion)이 동반된 경우

\subsection*{분만 전 출혈(O46)}
 : 분만 전에(활발한 진통이 시작되기 전) 출혈이 있어 입원한 경우 또는 입원하여 분만 전에 출혈량에 관계없이 출혈이 있었던 경우(혈성이슬 제외)
\begin{itemize}\tightlist
\item 응고장애를 동반한 경우 응고장애를 동반한 분만 전 출혈(O460) 부여 가능
\end{itemize}

\subsection*{분만 중 출혈(O67)} : 분만 중 (활발한 진통이 시작된 후부터 태아의 만출까지 : 분만 제 1.2기)에 과다출혈이 있었던 경우로 분만전(입원당시)과 비교 Hct가 10\%이상 감소한 경우이거나 수혈이 필요하여 수혈을 실시한 경우
\begin{itemize}\tightlist
\item 응고장애를 동반한 경우에는 응고장애를 동반한 분만 중 출혈(O670) 부여 가능
\end{itemize}
\subsection*{분만 후 출혈(O72)} : 분만 제3기부터 분만 후 6주 이내(조기산후출혈과 지연산후출혈을 모두 포함)에 과다 출혈이 있었던 경우로 분만전(입원당시)과 비교 Hct가 10\%이상 감소한 경우이거나 수혈이 필요하여 수혈을 실시한 경우
\begin{itemize}\tightlist
\item 응고장애를 동반한 경우에는 O723(분만 후 응고 결여) 부여 가능
\end{itemize}

\subsection*{급성 출혈 후 빈혈(D62)} : 외과적 수술, 처치 후 다량의 출혈로 수술전(입원당시)과 비교 Hgb과 Hct 수치의 10\% 이상 감소 및 Hb 10g/dl 미만으로 저하되어 이에 대한 치료가 이루어진 경우(약제투여, 수혈 등)

\begin{shaded}
단지 출혈이 있었다고 해서 이러한 기타진단을 쓸수 있는것을 아니고, 이러한 출혈에 대한 어떠한 행위가 있어야 기타진단으로 인정됩니다. 분만전,중 출혈에서는 nalador와 merthergin사용, 분만후 출혈에서는 베노훼럼주 사용등이 있습니다. 
\end{shaded}
\subsection*{어느 개인병원에서는}
\begin{enumerate}[가.]\tightlist
\item 제왕절개시에 출혈이 많다고 생각되면 nalador와 merthergin사용하고, 분만중 출혈(o67)코드 넣자. 부인과 수술시 출혈이 많으면 수술중 적절한 처치후 D62 (급성출혈) 상병추가.
\item 3층 입원실에서는 수술후 Hgb이 8이하 이면(절개시 nalador사용과 관계없이) 
\item 3층 입원실에서는 수술전과 후 Hct비교하여서 10\%이상 떨어져 있고, 수술중 nalador사용을 하지 않을시는 
	\begin{itemize}\tightlist
	\item 분만후 출혈(o72)코드 집어 넣고(회진닥터)
	\item 베노훼럼주 1A + N/S 100cc (15분 이상 slowly IV)
	\item 그 이후 볼그래액 1pack를 저녁으로 복용하게 한다. (입원시에는)
	\item 5일째 CBC 추적조사하게 함.(Hct 10\%이상 감소변화 없거나 퇴원후에도 심한 빈혈로 철분제가 필요하면 훼로바-유서방정 아침저녁으로 7일간 복용처방) 
	\item 퇴원 1주후 CBC F/U
	\item 처방의가 필요에 따라서 베노훼럼주의 용량은 늘리수 있습니다. 아래 참조.
	\end{itemize}
\end{enumerate}
\subsection*{\newindex{베노훼럼주} 사용} 
다음의 \pageref{VenoferrumInj}를 참조하세요.

%\clearpage

\subsection*{가진통(O47)}
 : 임신 만기 전에 자궁의 불규칙적인 수축으로 인한 통증으로 수축이 자연 소실되거나 자궁경관의 개대가 없는 상태로 분만으로 이어지지 않은 진통으로 확인된 경우
\subsection*{산후기 패혈증(O85)} : 분만 후 첫 24시간을 제외한 산후 10일 이내에 2일간 계속하여 38°C(100.4°F) 이상의 체온상승이 확인된 경우
\subsection*{고령 초임산부의 관리(Z355)} : 초임산부로서 만 35세 이상인 경우 \label{oldprimi} \index{진단코드!고령초임산부의 관리}
\subsection*{어린 초임산부의 관리(Z356)} : 초임산부로서 만 16세 미만인 경우 \label{youngprimi} \index{진단코드!어린 초임산부의 관리}
\subsection*{기타 고위험 임신의 관리(Z358)} : 경산으로 만 40세 이상인 경우와 만 35세 이상인 경산으로 전 출산과 만 5년 이상 Interval이 있는 경우 \label{otherhigh} \index{진단코드!기타 고위험 임신의 관리}
\subsection*{달리 분류되지 않은 처치에 의한 감염(T814)} : 수술부위의 통증, 국소 종창, 발적, 열감 등의 감염 징후를 동반하면서
\begin{itemize}\tightlist
\item 표재성 창상, 심부 절개 부위 및 기관/강 등 외과수술 부위에서 농성 분비물이 나오는 경우
\item 무균 처치시 획득된 체액이나 조직에서 미생물의 배양이 확인된 경우
\item 무균 처치시 획득된 체액이나 조직에서 미생물이 분리된 경우 등으로 외과의사나 주치의사의 판단에 의해 감염으로 진단한 경우
\end{itemize}

\subsection*{인슐린-의존 당뇨병(E10)} : 한국표준질병사인분류에 의하여 당뇨병이 불안정형(brittle), 연소성발병형(juvenile-onset), 케토증경향(ketosis-prone) (typeI)인 경우 또는I형
\begin{shaded}
GDM등도 기타진단으로 가능하다는 최근에 심평원의 대답을 듣었습니다. 물론 GDM때문에 추가적인 BST등을 했을때 인정됩니다.
\end{shaded}
\clearpage

\subsection{기타진단에 따른 중증도점수}
\subsubsection*{산과}
\tabulinesep =_2mm^2mm
\begin {longtabu} to\linewidth {|X[1.5,l]|X[6.5,l]|X[1.5,l]|X[.7,l]|} \tabucline[.5pt]{-} 
\rowcolor{ForestGreen!40} 상병코드 & 상병명 & 심사기준 & 점수 \\ \tabucline[.5pt]{-} \endhead
\rowcolor{Yellow!40} O11 & 동반된 단백뇨를 동반한 전에 있던 고혈압성 장애 \index{severity!동반된 단백뇨를 동반한 전에 있던 고혈압성 장애} &  & 2 \\ \tabucline[.5pt]{-}
\rowcolor{Yellow!40} O121 & 임신단백뇨\index{severity!임신단백뇨}  & & 2 \\ \tabucline[.5pt]{-}
\rowcolor{Yellow!40} O13 & 유의한 단백뇨를 동반하지 않은 임신성[임신-유발성]고혈압\index{severity!유의한 단백뇨를 동반하지 않은 임신성[임신-유발성]고혈압} & \pageref{SUPPIH} & 2 \\ \tabucline[.5pt]{-}
\rowcolor{Yellow!40} O140 & 중증도의 전자간\index{severity!중증도의 전자간} & \pageref{severePIH} & 2 \\ \tabucline[.5pt]{-}
\rowcolor{Yellow!40} O149 & 상세불명의 전자간\index{severity!상세불명의 전자간} & \pageref{PIH} & 2 \\ \tabucline[.5pt]{-}
\rowcolor{Yellow!40} O230 & 임신중 신장의 감염\index{severity!임신중 신장의 감염} & & 2 \\ \tabucline[.5pt]{-}
\rowcolor{Yellow!40} O231 & 임신중 방광의 감염\index{severity!임신중 방광의 감염} & & 2 \\ \tabucline[.5pt]{-}
\rowcolor{Yellow!40} O232 & 임신중 요도의 감염\index{severity!임신중 요도의 감염} & & 2 \\ \tabucline[.5pt]{-}
\rowcolor{Yellow!40} O233 & 임신중 요로의 기타 부분 감염\index{severity!임신중 요로의 기타 부분 감염} & & 2 \\ \tabucline[.5pt]{-}
\rowcolor{Yellow!40} O234 & 임신중 요로의 상세불명의 감염\index{severity!임신중 요로의 상세불명의 감염} & & 2 \\ \tabucline[.5pt]{-}
\rowcolor{Yellow!40} O235 & 임신중 생식관의 감염\index{severity!임신중 생식관의 감염} & & 2 \\ \tabucline[.5pt]{-}
\rowcolor{Yellow!40} O239 & 기타 및 상세불명의 임신중 비뇨생식관 감염\index{severity!기타 및 상세불명의 임신중 비뇨생식관 감염} & & 2 \\ \tabucline[.5pt]{-}
\rowcolor{Yellow!40} O240 & 전에 있던 인슐린-의존성 당뇨병\index{severity!전에 있던 인슐린-의존성 당뇨병} & & 2 \\ \tabucline[.5pt]{-}
\rowcolor{Yellow!40} O241 & 전에 있던 인슐린-비의존성 당뇨병\index{severity!전에 있던 인슐린-비의존성 당뇨병} & & 2 \\ \tabucline[.5pt]{-}
\rowcolor{Yellow!40} O242 & 전에 있던 영양실조-관련 당뇨병\index{severity!전에 있던 영양실조-관련 당뇨병} & & 2 \\ \tabucline[.5pt]{-}
\rowcolor{Yellow!40} O243 & 상세불명의 전에 있던 당뇨병\index{severity!상세불명의 전에 있던 당뇨병} & & 2 \\ \tabucline[.5pt]{-}
\rowcolor{Yellow!40} O244 & 임신중 생긴 당뇨병\index{severity!임신중 생긴 당뇨병} & & 2 \\ \tabucline[.5pt]{-}
\rowcolor{Yellow!40} O249 & 상세불명의 임신중 당뇨병\index{severity!상세불명의 임신중 당뇨병} & & 2 \\ \tabucline[.5pt]{-}
\rowcolor{Yellow!40} O266 & 임신, 출산 및 산후기중 간 장애\index{severity!임신, 출산 및 산후기중 간 장애} & & 2 \\ \tabucline[.5pt]{-}
\rowcolor{Yellow!40} O294 & 임신중 척추 및 경막외 마취로 유발된 두통\index{severity!임신중 척추 및 경막외 마취로 유발된 두통} & & 2 \\ \tabucline[.5pt]{-}
\rowcolor{Yellow!40} O295 & 임신중 척추 및 경막외 마취의 기타 합병증\index{severity!임신중 척추 및 경막외 마취의 기타 합병증} & & 2 \\ \tabucline[.5pt]{-}
\rowcolor{Yellow!40} O296 & 임신중 실패한 또는 어려운 삽관\index{severity!임신중 실패한 또는 어려운 삽관} & & 2 \\ \tabucline[.5pt]{-}
\rowcolor{Yellow!40} O298 & 임신중 마취의 기타 합병증\index{severity!임신중 마취의 기타 합병증} & & 2 \\ \tabucline[.5pt]{-}
\rowcolor{Yellow!40} O299 & 상세불명의 임신중 마취의 합병증\index{severity!상세불명의 임신중 마취의 합병증} & & 2 \\ \tabucline[.5pt]{-}
\rowcolor{Yellow!40} O360 & 리서스 동종면역의 산모관리\index{severity!리서스 동종면역의 산모관리} & & 2 \\ \tabucline[.5pt]{-}
\rowcolor{Yellow!40} O361 & 기타 동종면역의 산모관리\index{severity!기타 동종면역의 산모관리} & & 2 \\ \tabucline[.5pt]{-}
\rowcolor{Yellow!40} O411 & 양막낭 및 양막의 감염\index{severity!양막낭 및 양막의 감염} & & 2 \\ \tabucline[.5pt]{-}
\rowcolor{Yellow!40} O440 & 출혈이 없다고 명시된 전치태반\index{severity!출혈이 없다고 명시된 전치태반} & & 2 \\ \tabucline[.5pt]{-}
\rowcolor{Yellow!40} O441 & 출혈을 동반한 전치태반\index{severity!출혈을 동반한 전치태반} & & 2 \\ \tabucline[.5pt]{-}
\rowcolor{Yellow!40} O468 & 기타 분만전 출혈\index{severity!기타 분만전 출혈} & & 2 \\ \tabucline[.5pt]{-}
\rowcolor{Yellow!40} O469 & 상세불명의 분만전 출혈\index{severity!상세불명의 분만전 출혈} & & 2 \\ \tabucline[.5pt]{-}
\rowcolor{Yellow!40} O4700 & 임신 37주 전 제2 삼분기의 가진통\index{severity!임신 37주 전 제2 삼분기의 가진통} & & 2 \\ \tabucline[.5pt]{-}
\rowcolor{Yellow!40} O4701 & 임신 37주 전 제3 삼분기의 가진통\index{severity!임신 37주 전 제3 삼분기의 가진통} & & 2 \\ \tabucline[.5pt]{-}
\rowcolor{Yellow!40} O4709 & 임신 37주 전 상세불명의 삼분기의 가진통\index{severity!임신 37주 전 상세불명의 삼분기의 가진통} & & 2 \\ \tabucline[.5pt]{-}
\rowcolor{Yellow!40} O471 & 임신 37주 후의 가진통\index{severity!임신 37주 후의 가진통} & & 2 \\ \tabucline[.5pt]{-}
\rowcolor{Yellow!40} O479 & 상세불명의 가진통\index{severity!상세불명의 가진통} & & 2 \\ \tabucline[.5pt]{-}
\rowcolor{Yellow!40} O678 & 기타 분만중 출혈\index{severity!기타 분만중 출혈} & & 2 \\ \tabucline[.5pt]{-}
\rowcolor{Yellow!40} O679 & 상세불명의 분만중 출혈\index{severity!상세불명의 분만중 출혈} & & 2 \\ \tabucline[.5pt]{-}
\rowcolor{Yellow!40} O720 & 제3기 출혈\index{severity!제3기 출혈} & & 2 \\ \tabucline[.5pt]{-}
\rowcolor{Yellow!40} O721 & 기타 분만직후 출혈\index{severity!기타 분만직후 출혈} & & 2 \\ \tabucline[.5pt]{-}
\rowcolor{Yellow!40} O722 & 지연성 및 이차성 분만후 출혈\index{severity!지연성 및 이차성 분만후 출혈} & & 2 \\ \tabucline[.5pt]{-}
\rowcolor{Yellow!40} O723 & 분만후 응고결손\index{severity!분만후 응고결손} & & 2 \\ \tabucline[.5pt]{-}
\rowcolor{Yellow!40} O745 & 진통 및 분만중 척수 및 경막외마취-유발성 두통\index{severity!진통 및 분만중 척수 및 경막외마취-유발성 두통} & & 2 \\ \tabucline[.5pt]{-}
\rowcolor{Yellow!40} O746 & 진통 및 분만중 척수 또는 경막외마취의 기타 합병증\index{severity!진통 및 분만중 척수 또는 경막외마취의 기타 합병증} & & 2 \\ \tabucline[.5pt]{-}
\rowcolor{Yellow!40} O747 & 진통 및 분만중 실패한 또는 어려운 삽관\index{severity!진통 및 분만중 실패한 또는 어려운 삽관} & & 2 \\ \tabucline[.5pt]{-}
\rowcolor{Yellow!40} O748 & 기타 진통 및 분만중 마취의 합병증\index{severity!기타 진통 및 분만중 마취의 합병증} & & 2 \\ \tabucline[.5pt]{-}
\rowcolor{Yellow!40} O749 & 상세불명의 진통 및 분만중 마취의 합병증\index{severity!상세불명의 진통 및 분만중 마취의 합병증} & & 2 \\ \tabucline[.5pt]{-}
\rowcolor{Yellow!40} O753 & 진통중 기타 감염\index{severity!진통중 기타 감염} & & 2 \\ \tabucline[.5pt]{-}
\rowcolor{Yellow!40} O860 & 산과수술 상처의 감염\index{severity!산과수술 상처의 감염} & & 2 \\ \tabucline[.5pt]{-}
\rowcolor{Yellow!40} O861 & 분만에 따른 생식관의 기타감염\index{severity!분만에 따른 생식관의 기타감염} & & 2 \\ \tabucline[.5pt]{-}
\rowcolor{Yellow!40} O862 & 분만에 따른 요로감염\index{severity!분만에 따른 요로감염} & & 2 \\ \tabucline[.5pt]{-}
\rowcolor{Yellow!40} O863 & 분만에 따른 기타 비뇨생식관감염\index{severity!분만에 따른 기타 비뇨생식관감염} & & 2 \\ \tabucline[.5pt]{-}
\rowcolor{Yellow!40} O864 & 분만후 원인 불명 열\index{severity!분만후 원인 불명 열} & & 2 \\ \tabucline[.5pt]{-}
\rowcolor{Yellow!40} O868 & 기타 명시된 산후기 감염\index{severity!기타 명시된 산후기 감염} & & 2 \\ \tabucline[.5pt]{-}
\rowcolor{Yellow!40} O870 & 산후기중 표재성 혈전정맥염\index{severity!산후기중 표재성 혈전정맥염} & & 2 \\ \tabucline[.5pt]{-}
\rowcolor{Yellow!40} O871 & 산후기중 심부정맥혈전증\index{severity!산후기중 심부정맥혈전증} & & 2 \\ \tabucline[.5pt]{-}
\rowcolor{Yellow!40} O894 & 산후기중 척수 및 경막외마취-유발두통\index{severity!산후기중 척수 및 경막외마취-유발두통} & & 2  \\ \tabucline[.5pt]{-}
\rowcolor{Yellow!40} O895 & 산후기중 척수 및 경막외마취의 기타 합병증\index{severity!산후기중 척수 및 경막외마취의 기타 합병증} & & 2 \\ \tabucline[.5pt]{-}
\rowcolor{Yellow!40} O896 & 산후기중 실패한 또는 어려운 삽관\index{severity!산후기중 실패한 또는 어려운 삽관} & & 2 \\ \tabucline[.5pt]{-}
\rowcolor{Yellow!40} O898 & 기타 산후기중 마취의 합병증\index{severity!기타 산후기중 마취의 합병증} & & 2 \\ \tabucline[.5pt]{-}
\rowcolor{Yellow!40} O899 & 상세불명의 산후기중 마취의 합병증\index{severity!상세불명의 산후기중 마취의 합병증} & & 2 \\ \tabucline[.5pt]{-}
\rowcolor{Yellow!40} O900 & 제왕절개상처의 파열\index{severity!제왕절개상처의 파열} & & 2 \\ \tabucline[.5pt]{-}
\rowcolor{Yellow!40} O901 & 산과적 회음상처의 파열\index{severity!산과적 회음상처의 파열}& & 2 \\ \tabucline[.5pt]{-}
\rowcolor{Yellow!40} O902 & 산과적 상처의 혈종\index{severity!산과적 상처의 혈종} & & 2 \\ \tabucline[.5pt]{-}
\rowcolor{Yellow!40} O903 & 산후기 심근병증\index{severity!산후기 심근병증} & & 2 \\ \tabucline[.5pt]{-}
\rowcolor{Yellow!40} O904 & 분만후 급성 신부전\index{severity!분만후 급성 신부전} & & 2 \\ \tabucline[.5pt]{-}
\rowcolor{Yellow!40} O910 & 출산과 관련된 유두의 감염\index{severity!출산과 관련된 유두의 감염} & & 2 \\ \tabucline[.5pt]{-}
\rowcolor{Yellow!40} O911 & 출산과 관련된 유방의 농양\index{severity!출산과 관련된 유방의 농양} & & 2 \\ \tabucline[.5pt]{-}
\rowcolor{Yellow!40} Z355 & 고령 초임산부의 관리\index{severity!고령 초임산부의 관리} & \pageref{oldprimi} & 2 \\ \tabucline[.5pt]{-}
\rowcolor{Yellow!40} Z356 & 어린 초임산부의 관리\index{severity!어린 초임산부의 관리} & \pageref{youngprimi} & 2 \\ \tabucline[.5pt]{-}
\rowcolor{Yellow!40} Z358 & 기타 고위험 임신의 관리\index{severity!기타 고위험 임신의 관리} & \pageref{otherhigh} & 2 \\ \tabucline[.5pt]{-}
\rowcolor{Yellow!40} O141 & 중증의 전자간\index{severity!중증의 전자간} & & 3 \\ \tabucline[.5pt]{-}
\rowcolor{Yellow!40} O142 & 헬프증후군\index{severity!헬프증후군} & & 3 \\ \tabucline[.5pt]{-}
\rowcolor{Yellow!40} O150 & 임신중 자간\index{severity!임신중 자간} & & 3 \\ \tabucline[.5pt]{-}
\rowcolor{Yellow!40} O151 & 분만중 자간\index{severity!분만중 자간} & & 3 \\ \tabucline[.5pt]{-}
\rowcolor{Yellow!40} O152 & 산후기 자간\index{severity!산후기 자간} & & 3 \\ \tabucline[.5pt]{-}
\rowcolor{Yellow!40} O159 & 시기 상세불명의 자간\index{severity!시기 상세불명의 자간} & & 3 \\ \tabucline[.5pt]{-}
\rowcolor{Yellow!40} O293 & 임신중 국소 마취에 대한 독성 반응\index{severity!임신중 국소 마취에 대한 독성 반응} & & 3 \\ \tabucline[.5pt]{-}
\rowcolor{Yellow!40} O744 & 진통 및 분만중 국소마취에 대한 독성 반응\index{severity!진통 및 분만중 국소마취에 대한 독성 반응} & & 3 \\ \tabucline[.5pt]{-}
\rowcolor{Yellow!40} O85 & 산후기 패혈증\index{severity!산후기 패혈증} & & 3 \\ \tabucline[.5pt]{-}
\rowcolor{Yellow!40} O893 & 산후기중 국소마취에 대한 독성 반응\index{severity!산후기중 국소마취에 대한 독성 반응} & & 3 \\ \tabucline[.5pt]{-}
\rowcolor{Yellow!40} O290 & 임신중 마취의 폐 합병증\index{severity!임신중 마취의 폐 합병증} & & 4 \\ \tabucline[.5pt]{-}
\rowcolor{Yellow!40} O291 & 임신중 마취의 심장 합병증\index{severity!임신중 마취의 심장 합병증} & & 4 \\ \tabucline[.5pt]{-}
\rowcolor{Yellow!40} O292 & 임신중 마취의 중추신경계통 합병증\index{severity!임신중 마취의 중추신경계통 합병증} & & 4 \\ \tabucline[.5pt]{-}
\rowcolor{Yellow!40} O450 & 응고장애를 동반한 태반조기분리\index{severity!응고장애를 동반한 태반조기분리} & & 4 \\ \tabucline[.5pt]{-}
\rowcolor{Yellow!40} O458 & 기타 태반조기분리\index{severity!기타 태반조기분리} & & 4 \\ \tabucline[.5pt]{-}
\rowcolor{Yellow!40} O459 & 상세불명의 태반조기분리\index{severity!상세불명의 태반조기분리} & & 4 \\ \tabucline[.5pt]{-}
\rowcolor{Yellow!40} O460 & 응고 장애를 동반한 분만전출혈\index{severity!응고 장애를 동반한 분만전출혈} & & 4 \\ \tabucline[.5pt]{-}
\rowcolor{Yellow!40} O670 & 응고 장애를 동반한 분만중 출혈\index{severity!응고 장애를 동반한 분만중 출혈} & & 4 \\ \tabucline[.5pt]{-}
\rowcolor{Yellow!40} O710 & 진통 시작 전의 자궁 파열\index{severity!진통 시작 전의 자궁 파열} & & 4 \\ \tabucline[.5pt]{-}
\rowcolor{Yellow!40} O711 & 분만중 자궁파열\index{severity!분만중 자궁파열} & & 4 \\ \tabucline[.5pt]{-}
\rowcolor{Yellow!40} O740 & 진통 및 분만중 마취제로 인한 흡입폐렴\index{severity!진통 및 분만중 마취제로 인한 흡입폐렴} & & 4 \\ \tabucline[.5pt]{-}
\rowcolor{Yellow!40} O741 & 진통 및 분만중 마취로 인한 기타 폐합병증\index{severity!진통 및 분만중 마취로 인한 기타 폐합병증} & & 4 \\ \tabucline[.5pt]{-}
\rowcolor{Yellow!40} O742 & 진통 및 분만중 마취로 인한 심장합병증\index{severity!진통 및 분만중 마취로 인한 심장합병증} & & 4 \\ \tabucline[.5pt]{-}
\rowcolor{Yellow!40} O743 & 진통 및 분만중 마취로 인한 중추신경계통 합병증\index{severity!진통 및 분만중 마취로 인한 중추신경계통 합병증} & & 4 \\ \tabucline[.5pt]{-}
\rowcolor{Yellow!40} O751 & 진통 및 분만중 또는 그 후에 뒤따르는 쇼크\index{severity!진통 및 분만중 또는 그 후에 뒤따르는 쇼크} & & 4 \\ \tabucline[.5pt]{-}
\rowcolor{Yellow!40} O873 & 산후기중 대뇌정맥 혈전증\index{severity!산후기중 대뇌정맥 혈전증} & & 4 \\ \tabucline[.5pt]{-}
\rowcolor{Yellow!40} O880 & 산과적 공기 색전증\index{severity!산과적 공기 색전증} & & 4 \\ \tabucline[.5pt]{-}
\rowcolor{Yellow!40} O881 & 양수색전증\index{severity!양수색전증} & & 4 \\ \tabucline[.5pt]{-}
\rowcolor{Yellow!40} O882 & 산과적 피떡 색전증\index{severity!산과적 피떡 색전증} & & 4 \\ \tabucline[.5pt]{-}
\rowcolor{Yellow!40} O883 & 산과적 농혈성 및 패혈성 색전증\index{severity!산과적 농혈성 및 패혈성 색전증} & & 4 \\ \tabucline[.5pt]{-}
\rowcolor{Yellow!40} O888 & 기타 산과적 색전증\index{severity!기타 산과적 색전증} & & 4 \\ \tabucline[.5pt]{-}
\rowcolor{Yellow!40} O890 & 산후기중 마취의 폐 합병증\index{severity!산후기중 마취의 폐 합병증} & & 4 \\ \tabucline[.5pt]{-}
\rowcolor{Yellow!40} O891 & 산후기중 마취의 심장합병증\index{severity!산후기중 마취의 심장합병증} & & 4 \\ \tabucline[.5pt]{-}
\rowcolor{Yellow!40} O892 & 산후기중 마취의 중추신경계통 합병증\index{severity!산후기중 마취의 중추신경계통 합병증} & & 4 \\ \tabucline[.5pt]{-}
\end{longtabu}

\subsubsection*{부인과}
%\subsection{산과}
\tabulinesep =_2mm^2mm
\begin {longtabu} to\linewidth {|X[1.5,l]|X[6.5,l]|X[1.5,l]|X[.7,l]|} \tabucline[.5pt]{-} 
\rowcolor{ForestGreen!40} 상병코드 & 상병명 & 심사기준 & 점수 \\ \tabucline[.5pt]{-} \endhead 
\rowcolor{Yellow!40} D62 & 급성 출혈후 빈혈\index{severity!급성출혈후 빈혈} & CBC F/U other.. & 2 \\ \tabucline[.5pt]{-}
\rowcolor{Yellow!40} T814 & 처치후 봉합농양\index{severity!처치후 봉합농양} & & 2 \\ \tabucline[.5pt]{-}
\end{longtabu}
\clearpage

\section{질병군에 대한 질의응답}

\Que{진단명 부여기준은 ?}\index{DRGQA!진단명부여기준}

\Ans{기타진단은 입원기간 중 발생했거나 입원당시부터 주진단과 함께 가지고 있던 병태로서 임상적 평가, 치료적 요법, 진단적 처치, 재원기간 연장, 간호 및 관찰의 증가 측면에서 환자진료에 영향을 준 주진단 이외의 추가진단을 의미함
또한, 기타진단은 확진된 경우 부여하고 이번 입원과 관련이 없는 이전 병태, 비정상적인 검사결과 등은 기타진단으로 부여하지 않도록 하며 심사지침 「진단분류기호 부여기준(‘12.7.1)」에서 정한 일부 특정 진단명에 대하여는 해당 진단명의 조건도 함께 충족하여야 함\\
☞ 보건복지부 고시 진단의 정의 및 분류기호 부여기준(별표8), 「진단분류기호 부여기준」\par
 (예시)\par
분만전 Hct 37.9\%, 제왕절개분만 후 Hct 34.0\% 로 10\% 이상 Hct 감소한 경우이나 환자의 심신상태 등이 양호하여 특별한 처치·치료를 필요로 하지 않는 경우 O72 분만 후 출혈을 기타진단으로 코딩함은 오류임(복지부고시 「기타진단부여기준(별표8)」에 맞지 않음)
}

\Que{전치태반 또는 전자간증 등이 있는 임신부가 제왕절개술을 시행한 경우 주진단 부여원칙은?}\index{DRGQA!제왕절개시 주진단 부여원칙}
\Ans{제왕절개술을 받은 경우 선택적이던 응급이던 상관없이 제왕절개술을 받은 이유를 주된 병태로 우선 부여함 \par
 제왕절개술을 받은 이유가 불명확할 경우 O82  제왕절개에 의한 단일분만 코드를 주된 병태로 부여함 \par
 ☞ 관련근거 : 한국표준질병·사인분류 질병 코딩지침서 (P.113) O80 - O84의 분류
}
\Que{급성출혈 후 빈혈(D62), 분만 후 출혈(O72), 분만 중 출혈(O67)의 진단분류기호 부여기준 중 “분만(수술)전(입원당시)” 의 의미는?}\index{DRGQA!빈혈의 정의}
\Ans{입원하여 분만(수술) 전 시행한 혈액검사 및 통상 외래에서 분만(수술)전 시행한 검사를 의미함}

\Que{질병군 대상 수술 후 합병증으로 패혈증 등이 발생한 경우 합병증을 주진단으로 선정할 수 있는지?}\index{DRGQA!기타진단 부여원칙}
\Ans{진료 개시 후 주된 병태와 관련된 질환이나 합병증이 발생하였을 경우에는 이로 인한 자원소모가 많다고 할지라도 기타진단으로 부여함(기존 주된 병태를 주진단으로 적용) 
 ☞ 관련근거: 한국표준질병·사인분류 질병 코딩지침서 (P.2-4) 주된병태 선정원칙}

 
\subsection{청구방법}
\Que{\mycoloredbox{타법령}(산재, 자보 등)으로 입원 중 질병군 진료가 발생한 경우의 요양급여비용 청구방법은?}\index{DRGQA!산재자보시 DRG}
\Ans{타법령으로 입원 중 질병군 진료가 발생한 경우 전체 진료내역을 \mycoloredbox{행위별수가제}로 청구함\par
 ☞ 급여 65720-1898호(2001.12.29) 「타법령으로 입원진료 중 질병군 진료가 발생한 경우 요양급여비용 산정방법」
}

\Que{질병군 입원 진료기간 중에 건강보험 환자가 차상위 본인부담경감대상자 또는 의료급여 대상자로 자격 등이 변경된 경우 청구방법은?}\index{DRGQA!의료자격변동시}
\Ans{질병군 입원 진료 중에 건강보험 환자가 차상위 본인부담경감대상자 또는 의료급여 대상자로 자격 등이 변경된 경우 전체 진료내역을 \mycoloredbox{행위별수가제}로 청구함
}

\Que{복강경을 이용한 수술 중 부득이한 사유로 중도에 \mycoloredbox{개복술로 전환}하여 수술을 종결한 경우 “복강경 등 내시경하 수술시 보상하는 239,000원의 금액을 추가 산정함. 이 경우 본인일부부담금 산정특례 대상자의 본인부담률 적용 방법은?}\index{DRGQA!복강경 수술방법변경시}
\Ans{본인일부부담금 산정특례 대상자와 차상위 본인부담 경감대상자 등은 각각의 해당 본인부담률을 적용하며, 본인부담률이 5\%인 경우에는 239,000의 5\%에 해당하는 금액인 11,950원을 본인이 부담함
☞ 「국민건강보험법시행령」 별표2 제3호에 해당하는 대상자인 경우에는 그 각목에서 정한 본인부담률을 적용 }

\Que{질병군 부가코드란?}\index{DRGQA!부가코드란?}
\Ans{
질병군을 세분화하는 단·양측 여부와 수술방법 등의 구분을 위해 사용함\par
\tabulinesep =_2mm^2mm
\begin {tabu} to .85\linewidth {|X[2,l]|X[2,l]|} \tabucline[.5pt]{-} 
질병군 & 부가코드 \\  \tabucline[.5pt]{-}
수정체수술 & 소절개(ADC05), 양안(ADC04) \\  \tabucline[.5pt]{-}
서혜 및 대퇴부탈장수술 & 복강경(ADC03), 양측(ADC04) \\  \tabucline[.5pt]{-}
충수절제술 & 복강경(ADC03) \\  \tabucline[.5pt]{-}
자궁 및 자궁부속기수술 & 복강경(ADC03) \\  \tabucline[.5pt]{-}
\end{tabu}
}

\Que{‘12.7월 이후 질병군의 주요 심사불능 코드 항목은?}\index{DRGQA!심사불능코드}
\Ans{
\tabulinesep =_2mm^2mm
\begin {tabu} to .9\linewidth {|X[2,l]|X[10,l]|} \tabucline[.5pt]{-} 
65-10 & 행위별 진료내역 기재누락\\  \tabucline[.5pt]{-}
65-11 & 야간 및 공휴가산의 수술일과 시간 기재누락\\  \tabucline[.5pt]{-}
65-12 & 입원시 상병유무 기재누락\\  \tabucline[.5pt]{-}
65-13 & 행위\cntrdot{} 질병군 분리청구건의 6일 초과여부\\  \tabucline[.5pt]{-}
65-14 & 의료의 질 점검 기재 여부\\  \tabucline[.5pt]{-}
69-04 & 신생아(생후 28일까지) 탈장수술의 질병군 청구\\  \tabucline[.5pt]{-}
69-05 & 제왕절개분만 후 자궁동맥색전술 또는 자궁내 풍선확장술을 시행한 환자의 질병군 청구\\  \tabucline[.5pt]{-}
69-06 & 복강경 수술 중 부득이한 사유로 개복술 전환시 복강경 보상 비용 산정 오류 \\  \tabucline[.5pt]{-}
69-07 & 산부인과 가산점수 산정 오류\\  \tabucline[.5pt]{-}
69-08 & 원형자동문합기 치료재료 신고누락\\  \tabucline[.5pt]{-}
69-09 & 신의료기술 등 비급여(다빈치로봇을 이용한 수술)의 기재착오\\  \tabucline[.5pt]{-}
69-10 & 인공수정체 기재착오\\  \tabucline[.5pt]{-}
69-11 & 연령비교 진단분류기호 부여착오\\  \tabucline[.5pt]{-}
60-34 & 질병군(DRG) 초음파검사 수가 기재누락 또는 기재착오\\  \tabucline[.5pt]{-}
42-05 & 주상병에 의료인 면허정보 기재누락 또는 단순기재착오\\  \tabucline[.5pt]{-}
42-07 & 주상병에 기재된 의료인 면허정보와 인력신고현황 불일치\\  \tabucline[.5pt]{-}
42-09 & 주상병에 기재된 의료인 면허정보와 출입국 내역 비교 해당 의료인이 없는 기간 중 진료분 청구\\  \tabucline[.5pt]{-}
42-11 & 주상병에 기재된 의료인 면허정보와 휴가신고현황 비교 해당 의료인이 없는 기간 중 진료분 청구\\  \tabucline[.5pt]{-}
42-13 & 초음파 수가에 의료인 면허정보 기재누락 또는 단순기재착오\\  \tabucline[.5pt]{-}
42-14 & 초음파 수가에 기재된 의료인 면허정보와 인력신고현황 불일치\\  \tabucline[.5pt]{-}
\end{tabu}
}

\Que{○○종합병원에 충수암 의증으로 입원한 환자가 2014년 2월 1일 초음파검사 등을 실시 후 충수암 진단으로 충수절제술을 시행한 경우 급여대상인 초음파검사의 특정내역 기재방법은?}\index{DRGQA!특정내역 기재방법} \index{특정내역}
\Ans{
\begin{itemize}\tightlist
\item 초음파검사 비용의 추가 산정은 특정내역 MT007의 내역구분 ‘SON'으로 기재
\item  특정내역 기재형식 및 설명
	\begin{itemize}\tightlist
	\item X(3)/ccyymmdd/X/X(9)/9(10)/9(5).V9(2)/9(3)/9(10)/X(200)/X(1)/X(100)
	\item 내역구분/투여(실시)일자/코드구분/코드/단가/1일투여량(실시횟수)/총투여일수(실시횟수)/금액/준용명/면허종류/면허번호
	\end{itemize}
	
\item 발생단위구분
줄번호
특정내역구분
1
0000
MT007
특정내역
SON/20140201/1/E9443/0000040290/00001.00/001/0000040290//1/12345
\item ※ 초음파검사가 급여대상이나 산정횟수를 초과하는 경우에는 특정내역 MT007 내역구분 ‘All'(보훈환자의 경우 ’100‘)에 기재
\end{itemize}
}


\subsubsection*{자궁근종절제술-질부접근(R4123) 의 포괄수가제 포함여부}

\Que{자궁경부의 근종인경우 \mycoloredbox{외래에서 입원없이} 시행하는 자궁근종절제술-질부접근\mycoloredbox{(R4123)} 이 포괄수가제 포함 대상인가요?
포괄수가제 포함 대상인경우 행위별청구외에 추가로 다른 청구방법이 있나요?}{DRGQA!OPD based Vaginal myomectomy가 DRG인지?]
\Ans{질병군(DRG) 포괄수가는 국민건강보험법시행령 제21조제3항제2호에 따라 복지부장관이 별도 고시하는 7개 질병군으로 입원진료를 받은 경우에 적용하며, 질병군 입원진료는 질병군 급여 일반원칙에 따라 다음의 항목을 포함하고 있습니다.\par
- 다 음 -
\begin{itemize}\tightlist
\item 7개 질병군으로 응급실·수술실 등에서 수술을 받고 연속하여 6시간 이상 관찰 후 귀가 또는 이송한 경우 
\item 7개 질병군 중 수정체수술(대절개 단안 및 양안, 소절개 단안 및 양안), 기타항문수술, 서혜 및 대퇴부탈장수술 단측 및 양측(복강경 이용 포함)의 수술을 받고 6시간 이상 관찰 후 당일 귀가 또는 이송한 경우
\end{itemize}
따라서 자궁근종으로 자궁근종절제술-질부접근(R4123)을 받고 6시간 미만 관찰 후 당일 귀가 또는 이송한 경우(외래)는 \mycoloredbox{행위별청구대상}임을 알려드립니다.}

\Que{\mycoloredbox{유착방지제}의 포괄수가제 포함여부}\index{DRGQA!DRG에서 유착방지제}
\Ans{DRG 수가에는 급여진료비 뿐만 아니라 비급여 진료비도 포함되어 있습니다. 문의하신 유착방지제의 경우 현행DRG수가에 사용빈도에 따른 평균금액이 포함되어 있으므로 별도 환자에게 \mycoloredbox{비용을 받을수 없습}니다.}

\Que{현행 전액본인부담 약제의 수기료 산정방법에 대한 질의회신에 의하면 ``\mycoloredbox{단백아미노산제제}는 「요양급여의적용기준및방법에관한세부사항」에 의거 일부본인부담이 적용되는 급여기준을 정하고 이외 범위에 대해서는 건강보험의 비용 지원을 실시하기에는 비용효과성이 떨어진다고 판단하여 전액본인부담토록 급여기준을 운영하고 있으며, 전액본인부담하는 기준범위 이내에서는 주사 수기료도 비급여로 적용하는 것이 타당함을 알려 드립니다."라고 되어있습니다. 
2012년 7월 01일부터 실시되는 7개 질병군 포괄수가제에서 수술 후 의사가 진료군 진료에 해당되지 않는다는판단에도 불구하고 피로,권태등 환자가 요구에 의하여 단백아미노산제제를 투여받고자 한다면 국민건강보험 요양급여의 기준에 관한 규칙 제9조별표 비급여대상 4. 거. 그 밖에 요양급여를 함에 있어서 비용효과성 등 진료상의 경제성이 불분명하여 보건복지부장관이 정하여 고시하는 검사,처치, 수술, 기타의 치료 또는 치료재료. 의 항목 또는 6.나.질병군 진료 외 비급여 목적으로 투여된 약제에 해당되어 현행과 같이 주사 수기료도 포괄적인 행위(포괄수가)에서 제외하여 비급여로 환자 본인이 부담해야하는지 문의합니다}\index{DRGQA!DRG에서 아미노산제제}
\Ans{“요양급여는 진료의 필요가 있다고 인정되는 경우에 환자의 건강증진을 위하여 의학적으로 인정되는 범위 안에서 실시”(국민건강보험 요양급여의 기준에 관한 규칙 [별표1] 요양급여의 일반원칙)하여야 하며, “복합아미노산제제는 의학적으로 특히 필요하다고 인정되는 경우”(동 기준 「약제의 지급」)에 한하여야 한다고 규정하고 있습니다.

한편, 동 규칙 [별표2]비급여대상(제9조제①항관련) 제6호에서는 질병군에 대한 입원진료 시 비급여 대상을 규정하고 있으며, 동 내용에는 “제1호 업무 또는 일상생활에 지장이 없는 경우에 실시 또는 사용되는 행위ㆍ약제ㆍ치료재료 가. 단순한 피로 또는 권태”가 포함되어 있습니다.

따라서, 질병군으로 입원 진료 중 피로, 권태등 환자의 요구에 의해 단백아미노산제제를 투여받은 경우 발생된 주사수기료가 의학적으로 판단할 때 상기 2항의 규정에 부합되지 않는 경우라면 관련 비용은 질병군 \mycoloredbox{포괄수가에는 해당되지 않음}을 알려드립니다.}

\Que{쌍둥이 분만시 첫째는 자연분만, 둘째는 제왕절개로 했을 경우 포괄수가제 적용되는지 궁금합니다.
쌍둥이였고 첫째는 자연분만으로 낳았고, 둘째도 자연분만 시도하는 과정중에 자세가 바뀌어 불가피하게 제왕절개로 낳았습니다.이와 같은 경우에도 포괄수가제가 적용되나요??}\index{DRGQA!Twin다른 분만방법시?}
\Ans{자연분만과 제왕절개분만으로 분만방법을 달리하여 쌍둥이를 출산한 경우 전체 진료내역은 \mycoloredbox{행위별수가제가 적용}되오니 참고하시기 바랍니다}
\clearpage

\section{\newindex{낮병실료}}
\emph{낮병동 입원료는 실제 입원수속을 하지 않고 입원과 퇴원이 24시간 이내 어루어진 경우}, 1일의 입원료를 산정하고 진료비는 입원부담율에 의하여 산정하는 것을 말합니다.
낮병동 입원료는 응급실, 수술실 등에서 처치. 수술 등을 받고 연속하여 6시간 이상 관찰 후 당일 귀가 또는 퇴원하는 경우에 산정합니다.이때 낮병동 입원료의 산정 기산점은 의료기관에 내원하여 실제 진료가 시작된 시간을 기준으로 하며, \emph{의료기관의 진료기록부에 진료시간과 종료시간을 기재하여야 합니다.}
낮병동 입원료는 입원에 준하는 상태에서 항암제 투여, 정신과 진료, 응급진료, 처치 및 수술등을 받은 환자에 대한 관찰에만 최소한 6시간 이상 소요된 경우를 말하며 단순히 약제만을 투여하고 6시간 이상 의료기관에 지체하였다 하여 산정할 수는 없습니다.
\begin{enumerate}[(가)]\tightlist
\item 다음 각 호의 1에 해당하는 경우
	\begin{mdframed}[linecolor=blue,middlelinewidth=2]	
	\begin{enumerate}\tightlist
	\item 분만 후 당일 귀가 또는 이송하여 입원료를 산정하지 아니한 경우
	\item 응급실, 수술실 등에서 처치\cntrdot{}수술등을 받고 연속하여 6시간이상 관찰후 귀가 또는 이송하여 입원료를 산정하지 아니한 경우
	\item 정신건강의학과의 ``낮병동"에서6시간 이상 진료를 받고 당일 귀가한 경우
	\end{enumerate}
	\end{mdframed}
\item 낮 병동 입원료를 산정하는 당일 외래 또는 응급실에서 진찰을 행한 경우에는 진찰료를 함께 산정할 수 있다. 다만, 예정된 외래 수수을 위해 내원하는 경우 또는 정신건강의학과의 ``낮병동"에서 매일 또는 반복하여 진료를 받는 경우에는 진찰료를 산정하지 아니한다.
\item 낮 병동 입원료를 산저하는 당일의 본인일부부담금은 입원진료본인부담률에 따라 산정한다.
\end{enumerate}

\begin{enumerate}\tightlist
\item `입원료' 및 `낮병동 입원료' 산정시 기산점
	\begin{mdframed}[linecolor=blue,middlelinewidth=2]	
	\begin{itemize}\tightlist
	\item 입원과 퇴원이 같은 날에 이루어진 경우 1일의 입원료를 산정하는 기준은 입원실에 머무른 시간이 6시간 인상인 경우를 의미하는 것이며 이 경우 입원료 산정 기산점은 진료기록부 기재내역 및 환자가 실제로 입원실을 점유한 시점 등을 고려하여 입원실 입실시간을 기준으로 하는 것임
	\item `낮병동 입원료'의 경우는 응급실, 수술실 등에서 처치·수술 등을 받고 6시간 이상 관찰 후 당일 귀가 또는 퇴원하는 경우에 산정토록 되어 있는 바, 이 경우의 낮병동 입원료 산정 기산점은 의료기관 내원(도착)시간을 기준으로 함이 타당함.
	\end{itemize}
	\end{mdframed}
\item 입원실 또는 응급실 등의 체류시간이 6시간미만인 경우 수가산정방법 입원과 퇴원이 같은 날에 이루어진 경우로서 전체 입원시간이 6시간 미만인 경우와 응급실, 수술실 등에서 처치·수술 등을 받고 6시간 미만 머무른 경우에는 1일당 입원료 또는 낮병동 입원료를 산정하기 못하게 되므로 외래환자 본인부담률에 의해 진료비를 산정토록 함.
\item 낮병동 입원료를 적용받는 환자가 상급병실에 입원한 경우 상급병실 차액을 부담하여야 하는지 여부
병동 입원료를 적용받는 환자가 상급병실에 입원하여 동 병실에서 6시간이상 체류하였다면 이는 1일의 입원료를 산정하는 기준에 적합한 것이므로 환자에게 상급병실료 차액을 부담할 수 있음
개최일/시행일 : 2001-02-03관련근거: 급여65720-123호
\end{enumerate}

\tabulinesep =_2mm^2mm
\begin {tabu} to\linewidth {|X[1,l]|X[3,l]|X[6,l]|X[3,c]|X[3,c]|} \tabucline[.5pt]{-}
\rowcolor{ForestGreen!40} 가-6 & & 낮병동 입원료 Day Care & & \\ \tabucline[.5pt]{2-}
\rowcolor{Yellow!40}  & AF100 (18100) & 가. 상급종합병원 & 522.27 & \myexplfng{522.27} \\ \tabucline[.5pt]{-} %36,560 \\ \tabucline[.5pt]{2-}
\rowcolor{Yellow!40}  & AF200 (18200) & 나. 종합병원 & 480.64 & \myexplfng{480.64} \\ \tabucline[.5pt]{-} %33,640 \\ \tabucline[.5pt]{2-}
\rowcolor{Yellow!40}  & AF300 & 다. 병원, 치과병원, 한방병원 내 의\cntrdot{}치과 & 421.01 & \myexplfng{421.01} \\ \tabucline[.5pt]{-} %29,470 \\ \tabucline[.5pt]{2-}
\rowcolor{Yellow!40}  & 18300 & 라. 한방병원, 병원\cntrdot{}치과병원 내 한의과 & 417.36 & \myexplfng{417.36} \\ \tabucline[.5pt]{-} %29,220 \\ \tabucline[.5pt]{2-}
\rowcolor{Yellow!40}  & AF400 & 마. 의원, 치과의원, 보건의료원 의\cntrdot{}치과 & 358.86 & \myexplfng{358.86} \\ \tabucline[.5pt]{-} %25,120 \\ \tabucline[.5pt]{2-}
\rowcolor{Yellow!40}  & 18400 & 바. 한의원, 보건의료원 한방과 & 355.29 & \myexplfng{355.29} \\ \tabucline[.5pt]{-} %24,870 \\ \tabucline[.5pt]{-}
\end{tabu}
\clearpage

\section{ICD-9-CM Vol.3 소개}
ICD-9-CM Volume 3 is a system of procedural codes.\\
입원중 모든 procedure에 대한 code를 넣으면 좋겠지만, 현실적으로 불가능하므로 다음과 같이 정합니다.
%\begin{shaded}
\begin{itemize}\tightlist
\item 질식분만 산모의 경우는 분만코드(73.59)와 회음절개코드는 \dotemph{꼭} 넣어주라고 합니다.
\item 질식분만후 complication이 있는 경우 즉 hematoma(75.91)가 있는 경우에는 넣어주세요.(Optional)
\item 부인과 수술이 있는 경우에는 각각에 따른 ICD code를 넣습니다. TVH AP repair시에는 68.5와 70.5 두개의 code가 들어가야 합니다.
\end{itemize}
%\end{shaded}

\bigskip
\begin{tabularx}{1.1\linewidth}{lll}
    \toprule
    \footnotesize \textcolor{cyan}{\textsf{상병코드}} & \footnotesize \textcolor{red}{\textsf{명 칭}} & \footnotesize 처치코드(ICD)\tabularnewline
    \bottomrule
	\footnotesize \multirow{2}{10mm}{O800} & \footnotesize \multirow{2}{20mm}{자연두정태위분만} & \footnotesize 보조된 자연분만(73.59) \tabularnewline && \footnotesize 회음절개술(73.6) or 3도이상열상(75.62)  \tabularnewline \bottomrule
	\footnotesize O814 & \footnotesize 진공흡착기분만 & \footnotesize 회음절개가 동반된 흡입분만(72.71) \tabularnewline \bottomrule
	\footnotesize \multirow{2}{10mm}{O801} & \footnotesize \multirow{2}{20mm}{자연둔부태위분만} & \footnotesize 부분적둔위만출 (72.52) \tabularnewline && \footnotesize 회음절개술(73.6) or 3도이상열상(75.62) \tabularnewline 
    \bottomrule
\end{tabularx}

\subsection{OBSTETRICS}
\begin{itemize}[▷]\tightlist
\item (72) Forceps, vacuum, and breech delivery
	\begin{description}\tightlist
	\item[72.71] \dotemph{회음절개가 동반된 흡입분만}
	\item[72.79] 기타 흡입분만
	\item[72.52] \dotemph{부분적 둔위만출}
	\item[72.54] 전체적 둔위만출
	\end{description}
\item (73) Other procedures inducing or assisting delivery
	\begin{description}\tightlist
	\item[73.4] \textbf{\dotemph{유도분만}} 의학적유도분만
	\item[73.59] \textbf{\dotemph{정상두정태위분만}} Manually assisted delivery
	\item[73.6] \textbf{\dotemph{Episiotomy with repair}} Excludes: vacuum extraction (72.71)
	\end{description}
\item (74) Cesarean section and removal of fetus
	\begin{description}\tightlist
	\item[74.1] \textbf{\dotemph{csec}} Low cervical cesarean section
	\item[74.91] \textbf{Hysterotomy} for termination of pregnancy
	\item[74.99] unspecified cesarean section
	\end{description}
\item (75.4) Manual removal of retained placenta
\item (75.5) Repair of current obstetric laceration of uterus
\item (75.6) Repair of other current obstetric laceration
	\begin{description}\tightlist
	\item[75.62] \textbf{\dotemph{3도이상 열상의 수복}} \dotemph{병명코드: o702/o703}
	\end{description}
\item (75.9) Other obstetric operations
	\begin{description}\tightlist
	\item[75.91] \dotemph{Evacuation of obstetrical incision hematoma} of perineum
	\end{description}
\end{itemize}

\subsection{GYNECOLOGY}
\begin{itemize}[▷]\tightlist
\item Operation on ovary (65)
	\begin{description}\tightlist
	\item[65.2] \textbf{cystectomy}(Local excision or destruction of ovarian lesion or tissue)
	\item[65.25] \textbf{\dotemph{Laparoscopic cystectomy}} (Other laparoscopic local excision or destruction of ovary)
	\item[65.3] Unilateral oophorectomy
	\item[65.31] \textbf{Laparoscopic unilateral oophorectomy}
	\item[65.4] Unilateral salpingo-oophorectomy
	\item[65.41] \textbf{Laparoscopic unilateral salpingo-oophorectomy}
	\item[65.5] Bilateral oophorectomy
	\item[65.6] Bilateral salpingo-oophorectomy
	\item[65.8] Lysis of adhesions of ovary and fallopian tube
	\end{description}
\item (66) Operations on fallopian tubes
	\begin{description}\tightlist
	\item[66.0] Salpingotomy and salpingostomy
	\item[66.19] \textbf{\dotemph{D-lapa}} Other Diagnostic procedures on fallopian tubes
	\item[66.2] Bilateral endoscopic destruction or occlusion of fallopian tubes
	\item[66.3] Other Bilateral endoscopic destruction or occlusion of fallopian tubes \textbf{\dotemph{TL 66.31}} 
	\item[66.4] Total unilateral salpingectomy
	\item[66.62] \textbf{\dotemph{Ectopic salpingectomy}} 자궁관임신 제거가 동반된 자궁관절제술
	\end{description}
\item (67) Operations on cervix
	\begin{description}\tightlist
	\item[67.5] \textbf{\dotemph{cervical circlage}} Repair of internal cervical OS
	\end{description}
\item (68) Other incision and excision of uterus
	\begin{description}\tightlist
	\item[68.2] Excision or destruction of lesion or tissue of uterus
	\item[68.29] \textbf{\dotemph{uterine myomectomy}}
	\item[68.3] \textbf{\dotemph{Subtotal abdominal hysterectomy}}
	\item[68.4] \textbf{\dotemph{Total abdominal hysterectomy}}
	\item[68.5] \textbf{\dotemph{Vaginal hysterectomy}}
	\item[68.51] \textbf{\dotemph{LAVH}}
	\end{description}
\item (69) Other operations on uterus and supporting structures
	\begin{description}\tightlist
	\item[69.0] Dilation and curettage of uterus
	\item[69.01] 임신중절을 위한 DC
	\item[69.02] DC following delivery or abortion
	\item[69.09] Diagnostic DC
	\end{description}
\item (70) Operations on vagina and rectouterine pouch
	\begin{description}\tightlist
	\item[70.5] \textbf{\dotemph{AP 70.51, 70.52}} Repair of cystocele and rectocele
	\item[70.51] 전방질봉합술
	\item[70.52] 후방질봉합술
	\item[70.8] Obliteration of vaginal vault
	\end{description}
\end{itemize}

