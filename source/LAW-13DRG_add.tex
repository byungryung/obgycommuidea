\subsection{질병군 급여 일반원칙(가산)}
\begin{myshadowbox}
\begin{enumerate}[12.]\tightlist
\item 18시~09시 또는 공휴일에 응급진료가 불가피하여 수술을 행한 경우에는 해당 질병군의 야간 공휴 소정점수를 추가 산정한다. 이 경우 수술 또는 마취를 시작한 시간을 기준으로 한다.
\end{enumerate}
\end{myshadowbox}
\prezi{\clearpage}
\Que{통증자가조절법(PCA)의 야간ㆍ공휴 시술 가산 적용여부}
\Ans{100/100 본인부담항목인 통증자가조절법(PCA)은 환자에게 다른 통증관리방법(근육주사, 경구투여 등)과 PCA에 대해서 충분한 설명을 하고 환자가 동의한 경우에 실시하고 있어 응급진료에 인정하는 공휴 또는 야간가산을 적용하지 않음\par
☞ 급여65720-1787호(‘02.12.18) 「PCA관련 야간 공휴가산 인정여부에 대한 질의회신」}
\prezi{\clearpage}
\par
\medskip
\Que{마취와 수술시간이 모두 야간(또는 공휴)이어야 야간가산을 산정할 수 있는지 여부 }
\Ans{응급진료가 불가피하여 수술을 시행한 경우에는 마취나 수술 중 하나만 야간(또는 공휴)에 해당되어도 산정 가능함}
\prezi{\clearpage}
\par
\medskip
\Que{질병군 수술 후 수술부위의 출혈로 18시 이후에 응급으로 bleeding control을 시행한 경우 해당 질병군의 야간·공휴 소정점수를 추가 산정할 수 있는지 여부}
\Ans{질병군 수술에 따른 합병증으로 출혈이 발생하여 야간에 응급으로 bleeding control을 시행한 경우는 야간ㆍ공휴 소정점수 추가산정에 해당되지 않음
☞ “18~09시 또는 공휴일에 응급진료가 불가피하여 질병군 대상 수술을 행한 경우” 해당 질병군의 야간·공휴 소정점수를 추가 산정}
\prezi{\clearpage}
\begin{myshadowbox}
\begin{enumerate}[2.]\tightlist
\item 위 “1”의 규정에도 불구하고「복강경을 이용한 기타 자궁 수술(악성종양제외)」,「기타 자궁 수술(악성종양제외)」,「복강경을 이용한 자궁부속기 수술(악성종양제외)」,「자궁부속기 수술(악성종양제외)」의 각 질병군에 해당하는 수술을 실시한 경우 해당 질병군의 가산점수를 산정한다. 다만, 절개생검(심부[장기절개생검]-개복에 의한 것, 나-853-나-2), 유착성자궁부속기절제술(자-433)과 난소를 전적출하는 부속기종양적출술([양측]-양성, 자-442-가)은 가산점수를 산정하지 아니한다.
\end{enumerate}
\end{myshadowbox}
\prezi{\clearpage}
\Que{일측 난소의 전절제술을 시행한 기왕력이 있던 환자가 금번 질병군 진료기간 중에 자궁근종수술과 편측난소전절제술을 실시한 경우 가산점수 산정 가능여부}
\Ans{임신·출산을 담당하는 장기의 수술을 동시에 실시하여 그 수술결과로 임신·출산능력이 보존된 경우 「기타 자궁 수술」 및 「자궁부속기수술」 질병군의 가산점수를 산정할 수 있으므로 동 사례는 금번 수술로써 양측 난소를 절제한 경우에 해당되어 가산점수를 산정할 수 없음}
\prezi{\clearpage}
\par
\medskip
\Que{「복강경을 이용한 기타 자궁 수술(악성종양제외)」,「기타 자궁 수술(악성종양제외)」,「복강경을 이용한 자궁부속기 수술(악성종양제외)」,「자궁부속기 수술(악성종양제외)」의 각 질병군에 해당하는 수술을 실시한 경우 가산수가는 어떻게 계산하나요?}
\Ans{산부인과 질병군의 가산점수는 질병군별 소정점수와 해당 질병군의 고정비율을 이용하여 아래와 같이 산출합니다. 
{(질병군별 점수×고정비율)×1.3}+{질병군별 점수×(1-고정비율)}
위 식에서 구해진 가산점수를 다음의 질병군별 점수산정요령에 따라 질병군별 상대가치점수를 구하고 점수당 단가를 곱하여 급여비용을 산출합니다.
< 정상군의 경우 >
【{질병군별 점수×고정비율}+{질병군별 점수×(1-고정비율) × 가입자 등의 입원일수/질병군별 평균 입원일수}】× 20/100 +【질병군별 점수】× 80/100
(예시) 상급종합병원의 복강경을 이용하여 자궁근종절제술(복잡)을 실시한 경우, 질병군은 N04500(복강경을 이용한 기타 자궁 수술)이며, 30% 가산 수가적용시 가산점수 산출
※ 산부인과 가산점수 산정시에는 특정내역 ‘MT041 산부인과 가산점수 산정’에 Y'를 기재하여 청구합니다.\par
\begin{description}\tightlist
\item[질병군] N04500 복강경을 이용한 기타 자궁 수술(악성종양제외)
\item[점수] 37,504.24
\item[고정비율] 0.7
\item[가산점수] 45,380.13 {(37,504.24×0.7)×1.3} + {37,504.24 × (1-0.7)}
\end{description}
}
\prezi{\clearpage}
\par
\medskip
\Que{양측 난관전절제술을 실시한 경우에도 가산점수 산정이 가능한지 여부}
\Ans{자궁내막증이 있거나 불임(또는 난임) 등으로 임신가능성을 높이기 위해 난소 또는 난관 전절제술을 실시한 경우는 사례별로 인정할 수 있으며 환자의 연령, 출산력, 질환상태 등을 참고하여 진료담당의사의 의학적 판단 하에 임신·출산능력을 보존하는 수술을 시행한 경우 산정함을 원칙으로 함}
\prezi{\clearpage}
\par
\medskip
\Que{불임(또는 난임)으로 임신가능성을 높이기 위해 난소 또는 난관 전절제술을 실시한 경우는 사례별로 「기타 자궁 수술」 및 「자궁부속기수술」 질병군의 가산점수를 인정할 수 있음. 이 때 불임(또는 난임)의 정의는?}
\Ans{피임없이 정상적인 부부생활을 하면서 1년 내에 임신이 되지 않은 경우(일차성 불임)와 유산, 자궁 외 임신 및 분만 후 1년 이내에 임신이 되지 않은 경우(이차성 불임)를 의미함 }

