\section{수술전 검사}
\myde{}
{
\begin{itemize}\tightlist
\item[\dschemical] D1501 A,B,O혈구혈액형검사 \myexplfn{29.22} 원
\item[\dschemical] D1511 Rho혈액형검사 \myexplfn{13.82} 원
\item[\dschemical] D0002050 혈색소(광전비색) \myexplfn{16.96} 원
\item[\dschemical] D0002040 헤마토크리트 \myexplfn{10.94} 원
\item[\dschemical] D0002030 적혈구수 \myexplfn{10.97} 원
\item[\dschemical] D0002010 백혈구수 \myexplfn{10.99} 원
\item[\dschemical] D0002070 혈소판수 \myexplfn{12.99} 원
\item[\dschemical] D0013 백혈구 백분율(혈액) \myexplfn{25.35} 원
\item[\dschemical] D2253 요일반검사 10종까지 \myexplfn{28.27} 원
\item[\dschemical] D2202 요침사검사[유세포분석법] \myexplfn{16.74} 원
\item[\dschemical] D2300003 요소질소(NPN포함) \myexplfn{21.52} 원
\item[\dschemical] D2280003 크레아티닌 \myexplfn{16.75} 원
\item[\dschemical] D1860 SGOT \myexplfn{23.58} 원
\item[\dschemical] D1850 ALT[SGPT] \myexplfn{23.01} 원
\item[\dschemical] D3022003 당검사(정량) \myexplfn{17.32} 원
\item[\dschemical] D1003003 프로트롬빈시간 \myexplfn{24.83} 원
\item[\dschemical] D1004003 활성화부분트롬보플라스틴시간 \myexplfn{39.83} 원
\item[\dschemical] E6541 심전도검사-심저도기록및판독[표준12유도] \myexplfn{68.85} 원
\item[\dschemical] G2101 흉부[직접]1매 \myexplfn{79.67} 원
\end{itemize}
}
{
\Que{급여되는 수술은 하기전에 검사가 필요한데, 보험적용되는 검사항목은 어떤것이 있나요?}
\Ans{보험적용 가능한 수술전 검사항목에 대하여 \highlight{구체적으로 정하고 있지 않으며}, 환자의 상병및 상태등을 감안하여 \textcolor{red}{주치의의 판단에 따라 반드시 필요한 검사가 이루어져야 합니다}}
}
\prezi{\clearpage}
\subsection{QNA}
\Que{치질수술을 하기전에 환자의 상태를 알기위해서 혈액 검사를 합니다. B형간염 항원및 항체검사, 매독검사, AIDS 항체 검사를 하고자 하는데 보험급여가 가능한가요?}
\Ans{상기 검사는 건강보험요양급여비용 제1편 행위급여목록, 상대가치점수표 및 산정지침에 의거 요양급여항목에 해당되며, 가입자등의 연령, 성별, 직업및 심신상태 등의 특성을 고려하여 진료상 필요하다고 인정되는 경우에 정확한 진단을 토대로 의학적으로 인정되는 경우에 보험급여 가능합니다}

\Que{자동혈액분석기를 이용하여 일반혈액검사 6종(적혈구수, 백혈구수, 혈색소, 헤마토크리트, 적혈구분포계수, 혈소판분포계수)을 set화하여 실시하고 있는데요, 이 경우 적혈구분포계수 및 혈소판분포계수 검사는 관련 상병이 없어도 인정이 되나요?}
\Ans{현재 건강보험수가체계는 행위별로 되어 있으며, 각 검사 항목 당 상대가치점수를 산정하여 고시하고 있습니다. 따라서 환자에게 한번에 채취한 혈액으로 여러 가지 검사를 실시하였다 하더라도 검사 수기료는 각각의 검사 적응증에 따라 선별적으로 실시하여야 합니다. 또한 \textcolor{red}{적혈구분포계수 및 혈소판분포계수 검사의 적응증에 대한 고시 인정기준이 있으므로 관련 질환이 아닌 경우 일률적으로 set화하여 실시하여서는 안됩니다.}}
\begin{commentbox}{관련근거}
국민건강보험 요양급여의 기준에 관한 규칙 [별표1] 요양급여의 적용기준 및 방법(제5조 제1항 관련)
\begin{enumerate}[1.]\tightlist
\item 요양급여의 일반원칙  다. 요양급여는 경제적으로 비용효과적인 방법으로 행하여야 한다.
\item 진찰·검사·처치·수술 기타의 치료  가. 각종 검사를 포함한 진단 및 치료행위는 진료 상 필요하다고 인정되는 경우에 한하여야 하며 연구의 목적으로 하여서는 아니된다.
\end{enumerate}
□ 적혈구 분포계수 및 혈소판 분포계수의 적응증 : 나122 적혈구 분포계수는 빈혈의 감별진단에 실시하고, 나123 혈소판 분포계수는 혈소판질환 등 혈액질환의 감별진단에 실시하는 검사로 동 검사의 적응증이 아닌 질환에 기존 CBC항목에 추가하여 일률적으로 set화하여 산정할 수는 없음.
\end{commentbox}
\prezi{\clearpage}
\subsection{DRG포괄수가에서 보험적용 가능한 수술전 검사}
\myde{}
{
\textbf{1)전신마취 및 부위마취 (척추마취 및 기타마취 포함)}\par
\begin{itemize}\tightlist
\item[\dschemical] D0002050 혈색소(광전비색) \myexplfn{16.96} 원
\item[\dschemical] D0002040 헤마토크리트 \myexplfn{10.94} 원
\item[\dschemical] D0002030 적혈구수 \myexplfn{10.97} 원
\item[\dschemical] D0002010 백혈구수 \myexplfn{10.99} 원
\item[\dschemical] D0002070 혈소판수 \myexplfn{12.99} 원
\item[\dschemical] D1860 SGOT \myexplfn{23.58} 원
\item[\dschemical] D1850 ALT[SGPT] \myexplfn{23.01} 원
\item[\dschemical] D1880003 알부민 Albumin \myexplfn{22.45} 원
\item[\dschemical] D1840003 총단백정량 Total Protein Quantification \myexplfn{17.71} 원
\item[\dschemical] D2611 가. 총콜레스테롤 Total Cholesterol \myexplfn{22.02} 원
\item[\dschemical] D1870003 나. 알칼리 Alkaline Phosphatase  \myexplfn{20.09} 원
\item[\dschemical] D2800023 가. 소디움 Na \myexplfn{15.64} 원
\item[\dschemical] D2800063 나. 포타슘 K \myexplfn{16.87} 원
\item[\dschemical] D2800033 다. 염소 Cl \myexplfn{14.68} 원 
\item[\dschemical] D2253 요일반검사 10종까지 \myexplfn{28.27} 원
\item[\dschemical] D2202 요침사검사[유세포분석법] \myexplfn{16.74} 원
\item[\dschemical] D2300003 요소질소(NPN포함) \myexplfn{21.52} 원
\item[\dschemical] D2280003 크레아티닌 \myexplfn{16.75} 원
\item[\dschemical] D1003003 프로트롬빈시간 \myexplfn{27.73} 원
\item[\dschemical] D1004003 활성화부분트롬보플라스틴시간 \myexplfn{44.78} 원
\item[\dschemical] D1501 A,B,O혈구혈액형검사 \myexplfn{29.22} 원 \textcolor{red}{비급여}
\item[\dschemical] D1511 Rho혈액형검사 \myexplfn{13.82} 원 \textcolor{red}{비급여}
\item[\dschemical] E6541 심전도검사-심저도기록및판독[표준12유도] \myexplfn{68.85} 원
\item[\dschemical] G2101 흉부[직접]1매 \myexplfn{79.67} 원
\end{itemize}

\textbf{2)국소마취(마취없이 수술하는 경우 포함)} \par
\begin{itemize}\tightlist
\item[\dschemical] D0002050 혈색소(광전비색) \myexplfn{16.96} 원
\item[\dschemical] D0002040 헤마토크리트 \myexplfn{10.94} 원
\item[\dschemical] D0002030 적혈구수 \myexplfn{10.97} 원
\item[\dschemical] D0002010 백혈구수 \myexplfn{10.99} 원
\item[\dschemical] D0002070 혈소판수 \myexplfn{12.99} 원
\item[\dschemical] D1003003 프로트롬빈시간 \myexplfn{27.73} 원
\item[\dschemical] D1004003 활성화부분트롬보플라스틴시간 \myexplfn{44.78} 원
\end{itemize}
}
{
\Large \textbf{1)전신마취 및 부위마취 (척추마취 및 기타마취 포함)}\normalsize \par
CBC(5종), LFT(5종):SGOT/SGPT, Albumin,alk.phos,Protein,cholesterol
Electrolyte(3종)(Na, K, Cl)
BUN,Creatinine, U/A, 출혈 및 응고검사(2종), ABO/RH(비급여) : 수술 30일전 \par
CPA, EKG 수술 90일전 \par
\Large \textbf{2)국소마취(마취없이 수술하는 경우 포함)}\normalsize \par
CBC(5종), 출혈 및 응고검사(2종) 수술 30일전
}