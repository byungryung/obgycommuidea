\section{질강처치}
\myde{}{%
\begin{itemize}\tightlist
\item[\dsjuridical]  B373 외음과질의 칸디다증  
\item[\dsjuridical]  A590 비뇨생식기의 편모충증
\item[\dsjuridical]  N72 자궁경부의 염증성질환 
\item[\dsjuridical]  N86 자궁경부의 미란과 외반
\item[\dsjuridical]  R4106 질강처치료 [\myexplfn{58.04} 원]
\end{itemize}
}%
{한달에 한번만\par
응급 야간 가산 없음}

\subsection{질강처치 조정삭감}
전국적으로 2016 년 4 월 진료 내역부터 질강처치 (R4106) 와 같이 청구할때 질강처치 삭감 ( 조정 ) 된경우입니다\par
\begin{enumerate}\tightlist
\item 자궁내장치제거 .( R4275 실이 보이는 경우 , R4277 실이보이지않는경우 - 기타의 경우 )
\item R4240 자궁경관점막폴립절제술
\item R4105 질이물제거술
\item R4300 자궁경부 (질) 약물소작술 , R4310 자궁경부 (질) 전기소작술 , R4320 자궁경부 (질) 냉동 또는 열 응고술
\item R4113 질탈교정술 - 비수술적치료 [ 질페사리삽입술 ]
\item C8576 자궁경부 착공생검
 \end{enumerate} 
상기 와 같은 경우 또는 추가 시술 , 처치 행위시 동일날 질강처치 (R4106) 청구를 피해주시기 바랍니다 .\par
상기 사안에 대해서 직선제 산부인과의사회 보험위원회에서 대책을 논의중입니다 .\par
추가 내용있는 경우 추가 공지하겠습니다 .\par