\subsection{고위험 기타진단 산정전/산정후 비교}
-조건 : 8박 9일/6인실/식사제외된 사항입니다.\\
\noindent

\tabulinesep =_2mm^2mm
\begin {tabu} to\linewidth {|X[4,c]|X[3,c]|X[3,c]|X[3,c]|} \tabucline[.5pt]{-}
\rowcolor{ForestGreen!40}  & \centering 본인부담 & \centering 공단부담 & \centering 총진료비 \\ \tabucline[.5pt]{-}
\rowcolor{Yellow!40} 산정전(O01600) & 394,834 & 1,388,528 & 1,783,362  \\ \tabucline[.5pt]{-}
\rowcolor{Yellow!40} 산정후(O01601-3) & 408,219 & 1,465,144 & 1,873,363 \\ \tabucline[.5pt]{-}
\rowcolor{Yellow!40} 차 액 & 13,385 & 76,616 & 90,001  \\ \tabucline[.5pt]{-}
\end{tabu}
\prezi{\clearpage}
\begin{shaded}
분만전 hct 37.9\%, 제왕절개분만 후 Hct 34.0\%로 10\%이상 Hct감소한 경우이나 환자의 심신상태 등이 양호하여 특별한 처치\cntrdot{}치료를 필요로 하지 않는 경우 O72 분만 후 출혈을 기타진단으로 코딩함음 오류임(복지부고시 \snm{기타진단부여기준(별표8)}에 맞지 않음)\\

현재 까지 알아낸 특별한 처치들이란?
Nalador usage, 부르탈 usage, GDM에서 BST check등...
\end{shaded}

\prezi{\clearpage}
\tabulinesep =_2mm^2mm
\begin {longtabu} to\linewidth {|X[1,l]|X[6,l]|X[1,l]|X[1,l]|} \tabucline[.5pt]{-}
\rowcolor{ForestGreen!40}  기호 & \centering 한글명칭 & \centering \% & \centering 기준 \\ \tabucline[.5pt]{-}
\rowcolor{Yellow!40} D62 & 급성출혈후 빈혈 & 29.0 & O  \\ \tabucline[.5pt]{-}
\rowcolor{Yellow!40} Z355 & 고령 초임산부의 관리 & 17.7 & O  \\ \tabucline[.5pt]{-}
\rowcolor{Yellow!40} Z358 & 기타 고위험 임신의 관리 & 16.0 & O  \\ \tabucline[.5pt]{-}
\rowcolor{Yellow!40} O721 & 기타 분만직후 출혈 & 10.4 & O  \\ \tabucline[.5pt]{-}
\rowcolor{Yellow!40} O244 & 임신중 생긴 당뇨병 & 4.5 & X  \\ \tabucline[.5pt]{-}
\rowcolor{Yellow!40} O720 & 제3기 출혈 & 2.9 & O  \\ \tabucline[.5pt]{-}
\rowcolor{Yellow!40} O440 & 출혈이 없다고 명시된 전치태반 & 2.5 & X  \\ \tabucline[.5pt]{-}
\rowcolor{Yellow!40} O249 & 상세불명의 임신중 당뇨병 & 2.2 & X  \\ \tabucline[.5pt]{-}
\rowcolor{Yellow!40} O678 & 기타 분만중 출혈 & 2.1 & O  \\ \tabucline[.5pt]{-}
\rowcolor{Yellow!40} O679 & 상세불명의 분만중 출혈 & 2.0 & O  \\ \tabucline[.5pt]{-}
\rowcolor{Yellow!40} O441 & 출혈을 동반한 전치태반 & 1.9 & X  \\ \tabucline[.5pt]{-}
\rowcolor{Yellow!40} O459 & 상세불명의 태반조기분리 & 1.8 & X  \\ \tabucline[.5pt]{-}
\rowcolor{Yellow!40} O13 & 유의한 단백뇨를 동반하지 않은 임신성[임신-유발성]고혈압 & 1.8 & O  \\ \tabucline[.5pt]{-}
\rowcolor{Yellow!40} D500 & (만성)실혈에 이차성 분만후 철결핍빈혈 & 1.6 & X  \\ \tabucline[.5pt]{-}
\rowcolor{Yellow!40} O722 & 지연성 및 이차성 분만후 출혈 & 1.1 & O  \\ \tabucline[.5pt]{-}
\rowcolor{Yellow!40} O140 & 중증도의 전자간 & 0.7 & O  \\ \tabucline[.5pt]{-}
\rowcolor{Yellow!40} O149 & 상세불명의 전자간 & 0.7 & O  \\ \tabucline[.5pt]{-}
\rowcolor{Yellow!40} T810 & 달리 분류되지 않은 처치에 합병된 출혈 및 혈종 & 0.4 & X  \\ \tabucline[.5pt]{-}
\rowcolor{Yellow!40} O141 & 중증의 전자간 & 0.4 & O  \\ \tabucline[.5pt]{-}
\rowcolor{Yellow!40} K661 & 복강내 출혈 & 0.3 & X  \\ \tabucline[.5pt]{-}
\end{longtabu}
\par
\prezi{\clearpage}
\subsection{합병증 및 동반상병 분류 결정 단계}
\begin{enumerate}[(1)]\tightlist 
\item 기타진단의 중증도 점수
	\begin{itemize}\tightlist
	\item 합병증 분류에 이용되는 기타진단은 진단별로 2∼4까지의 중증도 점수를 갖는다.(「4.기타진단의 중증도 점수」 참조)
	\item 주진단 및 기타진단간 상호 연관성이 높은 기타진단은 중증도 점수가 2점 이상이더라도 0점이 된다.(「5.기타진단의 중증도 점수를 0으로 결정되게 하는 주진단」 참조)
	\end{itemize}
\item 환자단위 중증도 점수
	\begin{itemize}\tightlist
	\item 최종적으로 중증도 점수를 갖는 여러 개의 기타 진단들이 있을 경우 이를 통합하여 환자단위 중증도 점수를 결정하게 된다. 환자단위 중증도 점수는 아래와 같은 공식을 이용해서 계산된다.
	\item 환자단위의 중증도점수는  = 0 if there is no 기타진단, = 4 if x >4 , = x otherwise
	\item 점수의 의미는 0 : no CC effect, 1 : minor CC, 2 : moderate CC, 3 : severe CC, 4 : catastrophic CC 입니다. ※ CC(Complication and Comorbidity) : 합병증 및 동반상병
	\end{itemize}

\item 질병군범주별 합병증 및 동반상병 분류
	\begin{itemize}\tightlist
	\item 합병증 및 동반상병 분류의 마지막 단계로 각 질병군 범주의 특성에
따라 구분된 환자단위 중증도 점수별로 최종 질병군 분류번호를 결정
하게 된다. [표1 참조]
	\end{itemize}
\end{enumerate}
\prezi{\clearpage}
%\begin{figure}
%\centering
\includegraphics{severity2}	
\par
\medskip
%\caption{
다른 부인과나 산과의 질환의 경우는 severity에 따라서 2-3단계만 있는데에 비해서, 단태아제왕절개를 보면 severity에 따라서 4단계나 나누어 지는것을 볼수 있습니다. 중증도점수가 올라가면 갈수록 급여받을수 있는 금액이 올라갑니다. 한단계마다 거의 10만원가까이 올라갑니다.%}
%\end{figure}