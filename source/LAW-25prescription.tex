\section{처방전 관련}
\subsection{처방전 발급의무}
\leftrod{처방전의 기재 사항}
\begin{itemize}[-]\tightlist
\item 처방전에 다음 각 호의 사항을 적은 후 서명하거나 도장. 
	\begin{enumerate}[1.]\tightlist
	\item 환자의 성명 및 주민등록번호
	\item 의료기관의 명칭, 전화번호 및 팩스번호
	\item 한국표준질병\cntrdot{}사인 분류에 따른 질병분류기호; 환자가 요구한 경우에는 적지 않아도 됨
	\item 의료인의 성명\cntrdot{}면허종류 및 번호
	\item 처방 의약품의 명칭\cntrdot{}분량\cntrdot{}용법 및 용량
	\item 처방전 발급 연월일 및 사용기간
	\item 의약품 조제시 참고 사항
	\end{enumerate}
\item 처방전 2부를 발급의무. 
	\begin{enumerate}[1.]\tightlist
	\item 의약분업 합의할 때 발급의무가 있지만 한장으로 하는 것으로 양해가 되었다고 알고 있습니다.
	\item 처방전 병명 기입에 대해서는 환자가 원하는 경우 해줘야 합니다
	\end{enumerate}
\end{itemize}
	
\subsection{환자가 약제를 분실하여 처방전 재발급을 위해 내원한 경우}
\Que{환자가 약을 분실하여 동일한 처방전 재발급을 위해 내원하였습니다. 이 경우 발생한 진료비는 어떻게 청구해야 하나요?}
\Ans{환자가 이미 수령한 약제를 분실한 것은 환자에게 귀책사유가 있으므로, 동일한 처방전 발급을 위해 재내원한 경우의 진료비는 전액 본인부담 해야 합니다.}

\begin{commentbox}{약제 분실하여 다시 처방하는 경우에 대한 질의 회신}
환자가 약을 분실하여 다시 처방해야 하는 경우, 이미 수령한 약제를 분실한 것은 환자에게 귀책사유가 있으므로 진찰료 및 약국에서의 약제료, 조제료는 모두 전액 본인이 부담하도록 하고 요양급여비용은 청구할 수 없습니다. 이 경우 처방전양식 중 “기타”란에 “전액 본인부담”으로, “조제시 참고사항”란에 재처방 사유(예시 : 처방약 분실에 따른 재처방)를 표시하여야 합니다. 행정해석, 보험약제과-1070호 (2008.05.27.)
\end{commentbox}

\subsection{처방전 분실 후 재발급}
\Que{환자가 처방전을 분실하여 재발급 받기를 원할 때 재발행 방법이 있나요?}
\Ans{처방전 사용기간이내에 처방전을 단순히 분실하였을 경우는 분실된 처방전과 동일하게 재발급하고, 이때 처방전교부번호는 종전의 번호를 그대로 사용하고 재발급한 사실을 확인할 수 있도록 처방전에 표기하여야 합니다.}

\begin{commentbox}{처방전 재발급시 요양급여비용 산정방법}
처방전 재발급을 위해 의료기관에 내원시 요양급여비용의 산정은 다음과 같이 함\par
- 다 음 -
\begin{enumerate}[가.]\tightlist
\item 처방전 사용기간 경과후 재발급시
: 처방전에 기재된 ‘사용기간’은 환자가 동 처방전에 의하여 조제 받을 수 있는 유효기간 으로서, 이 기간이 경과한 때에는 그 사유와 관련 없이 종전 처방전에 의하여 조제 받을 수 없음 따라서, 처방전을 재발급 받기 위해서는 의료기관에 재차 내원하여야 하며, 처방전 발급여부는 의사 또는 치과의사의 판단하에 이루어지는바, 재발급 여부결정을 위해 진찰이 이루어진 경우 진찰료 등의 비용은 새로운 진료로 인해 발생되는 비용이므로 건강보험 법령에서 정한 부담률에 의하여 요양급여비용 중 일부를 본인이 부담하여야 함
\item 처방전 사용기간 이내에 처방전을 분실하여 재발급시
: 의사의 판단하에 재진찰 여부를 결정하되, 단순히 분실된 처방전과 동일하게 재발급하는 경우에는 진찰료를 별도 산정할 수 없으며, 이 때 처방전교부번호는 종전의 번호를 그대로 사용하고 재발급한 사실을 확인할 수 있도록 처방전에 표기함
\end{enumerate}
고시 제2003-65호 (2003.12.01. 시행)
\end{commentbox}