\subsection{제4장 산부인과 [적용지침]}
\begin{enumerate}[1.]\tightlist
\item 요양기관종별로 「복강경을 이용한 자궁적출술(악성종양제외)」,「기타 자궁적출술(악성종양제외)」,「복강경을 이용한 기타 자궁 수술(악성종양제외)」,「기타 자궁 수술(악성종양제외)」,「복강경을 이용한 자궁부속기 수술(악성종양제외)」,「자궁부속기 수술(악성종양제외)」,「제왕절개분만(단태아)」,「제왕절개분만(다태아)」의 각 질병군 소정점수를 적용한다.
\item 위 “1”의 규정에도 불구하고 「복강경을 이용한 기타 자궁 수술(악성종양 제외)」,「기타 자궁 수술(악성종양제외)」,「복강경을 이용한 자궁부속기 수술(악성종양제외)」,「자궁부속기 수술(악성종양제외)」의 각 질병군에 해당하는 수술을 실시한 경우 해당 질병군의 가산점수를 산정한다. 다만, 절개생검(심부[장기절개생검]-개복에 의한 것, 나-853-나-2), 유착성자궁부속기절제술(자-433)과 난소를 전적출하는 부속기종양적출술([양측]-양성, 자-442-가)은 가산점수를 산정하지 아니한다.
\item 「제왕절개분만(단태아)」,「제왕절개분만(다태아)」질병군 대상 중 출혈로 인해 혈관색전술(기타혈관, 자-664-나), 자궁내 풍선카테터 충전술[자궁용적측정 포함](자-402-3)을 실시한 경우 질병군 점수를 적용하지 아니하며 제1편을 적용한다.
\item 각 질병군은 동 질병군에 해당하는 수술의 종목수 및 편ㆍ양측 수술에 불문하고 해당 소정점수를 적용한다.
\item 복강경을 이용한 수술 중 부득이한 사유로 중도에 개복술로 전환하여 수술을 종결한 경우에는 복강경을 이용하지 아니한 질병군에 해당하는 소정점수를 적용하고 복강경 등 내시경하 수술시 보상하는 239,000원(100분의 20에 해당 하는 47,800원은 본인부담)의 금액을 추가 산정한다.
\item 제4부 비급여 목록 2. 신의료기술등의 비급여 제9장 처치 및 수술료 등의 (1) 다빈치 로봇 수술을 실시한 경우에는, 제2편제1부제5호에 따라 산정한 복강경을 이용한 자궁 및 자궁부속기 수술 질병군 요양급여비용의 총액에서 별표 2의2의 질병군별 다빈치 로봇 수술시 제외금액표의 금액을 제외하고 산정한다. 다만, 야간ㆍ공휴 및 “2.” 등의 가산은 적용하지 아니한다.
\item 자궁근종, 자궁선근증에 초음파 유도하 고강도초음파집속술(조-566)을 실시한 경우 질병군 점수를 적용하지 아니하며 제1편을 적용한다.
\item 22시- 06시에 제왕절개분만을 행한 경우에는 질병군 야간ㆍ공휴 소정점수를 2회 산정한다. 이 경우 수술 또는 마취를 시작한 시각을 기준으로 산정한다.
\item 분만취약지에서 제왕절개분만을 행한 경우에는 질병군 야간ㆍ공휴 소정점수를 4회 산정하고, 분만취약지는 제1편에서 정하고 있는 「요양급여의 적용기준 및 방법에 관한 세부사항」을 적용한다.
\end{enumerate}