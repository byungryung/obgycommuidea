\section{회음열창봉합술} 
\myde{}
{
\begin{itemize}\tightlist
\item[\dsjuridical] O702 분만중 제3도 회음열상, K5900 변비
\item[\dsjuridical] O703 분만중 제4도 회음열상, K5900 변비
\item[\dsjuridical] O713 자궁경부의 산과적 열상, K5900 변비
\item[\dsmedical] 처치 R4023 회음열창봉합술 – 항문에 달하는 것 \myexplfn{1020.33}
\item[\dsmedical] 처치 R4024 회음열창봉합술 – 질원개에 달하는 것 \myexplfn{1045.07}
\item[\dsmedical] 처치 R4025 회음열창봉합술 – 직장열창을 동반하는것 \myexplfn{1,149.70}
\item[\dsmedical] 처치 R4026 자궁경관열상봉합술 \myexplfn{834.96}
\item[\dsmedical] 재료 B0531009. Vicryl 1
\item[\dsmedical] 기타 항생제및 지사제
\end{itemize}
}
{
\begin{enumerate}\tightlist 
\item 처치 및 수술시 사용된 봉합사는 다음의 경우를 제외하고는 실사용량으로 산정할 수 있으며, ``치료재료급여목록및상한금액표" 범위내에서 실 구입가로 산정함.
\item 봉합사 제품명(Catalog No.),굵기(Gauge), 사용량 등을 진료기록부(수술기록지) 에 반드시 기재하여야 함. (적용일: 2008.1.1)
\item 청구참고사항 : 항문에 달하는 회음부열창있고, 항문괄약근의 분리있어서 vicryl 1-0로 3-4번의 단면을 맞춰서 interrupted suture  해줌. 회음부통증과 변비로 인한 상처감염과 tearing우려로 항생제와 설사제, 진통제 처방함.
\end{enumerate}
}

\par
\medskip
\prezi{\clearpage}
\Que{R4021 회음절개 및 봉합술과 R4023등 회음열창봉합술 동시 청구 가능여부?}
\Ans{처치 및 수술료의 일반사항중
\begin{enumerate}\tightlist	
\item 동일 절개하에서 2가지 이상 수술을 동시에 시술한 경우 주된 수술이란 2가지 이상 수술 중 소정금액이 높은 수술을 기준으로 함. 이 경우 '소정금액'이란 제9장 처치 및 수술료 등의 각 분류항목에 기재된 금액을 말함.     
\item 동일 피부 절개하에 해당과를 달리하여 각각 다른 병변을 수술한 경우, 진료전문과목이 다르더라도 동일 마취하에 연속하여 수술을 하는 것이므로 제9장 처치 및 수술료 등[산정지침] (5)항에 의거 주된 수술 100\%, 그외 수술 50\%[종합병원(상급종합병원 포함)은 70\%]를 산정함.(고시 제2016-204호, '16.11.1. 시행)
\end{enumerate}
  R4021 + R4026 은 한 절개부분이 아니므로 가능한걸로 알고 있지만, R4021 + R4023 은 한 절개부분이므로 숫가가 높은 한개만 청구가능}      
