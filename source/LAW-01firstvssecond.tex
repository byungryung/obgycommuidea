\section{초진과 재진의 구분}
\begin{enumerate}[1.]\tightlist
\item 진찰료
	\begin{enumerate}[가.]\tightlist
	\item 진찰료는 외래에서 환자를 진찰한 경우에 처방전의 발행과는 관계없이 산정하며 초진환자를 진찰하였을 경우에는 초진진찰료, 재진환자를 진찰하였을 경우에는 재진진찰료를 산정한다.
		\begin{enumerate}[(1)]\tightlist
		\item 진찰료는 기본진찰료(초진의 경우 AA154-AA157은 155.57점, AA100, AA109는 152.11점, 10100은 152.06점, 재진의 경우 AA254-AA257은 98.03점, AA200, AA209, 10200은 95.98점)와
외래관리료(진찰료에서 기본진찰료를 제외한 점수)의 소정점수를 합하여 산정한다.
		\item \uline{초진환자란 해당 상병으로 동일 의료기관의 동일 진료과목 의사에게 진료받은 경험이 없는 환자를 말한다.}
		\item \uline{재진환자란 해당 상병으로 동일 의료기관의 동일 진료과목 의사에게 계속해서 진료받고 있는 환자를 말한다.}
		\item \uline{해당 상병의 치료가 종결되지 아니하여 계속 내원하는 경우에는 내원 간격에 상관없이 재진환자로 본다. 또한, 완치여부가 불분명하여 치료의 종결 여부가 명확하지 아니한 경우 90일 이내에 내원시 재진환자로 본다.}
		\item 해당 상병의 치료가 종결된 후 동일 상병이 재발하여 진료를 받기 위해서 내원한 경우에는 초진환자로 본다. 다만 치료종결 후 30일 이내에 내원한 경우에는 재진환자로 본다.
		\item 치료의 종결이라 함은 해당 상병의 치료를 위한 내원이 종결되었거나, 투약이 종결되었을 때로 본다.
		\item 진찰료 중 기본진찰료는 병원관리 및 진찰권발급 등, 외래관리료는 외래환자의 처방 등에 소요되는 비용을 포함한다.
		\end{enumerate}
	\item 다음 각 호의 1에 해당하는 경우에는 \uline{진찰료는 1회 산정한다.}
		\begin{enumerate}[(1)]\tightlist
		\item \uline{동일 의사가 동시에 2가지 이상의 상병에 대하여 진찰을 한 경우}
		\item 하나의 상병에 대한 진료를 계속 중에 다른 상병이 발생하여 동일 의사가 동시에 진찰을 한 경우(재진진찰료)
		\item \uline{동일한 상병에 대하여 2인 이상의 의사가 동일한 날에 진찰을 한 경우}
		\end{enumerate}
	\item \uline{2개 이상의 진료과목이 설치되어 있고 해당 과의 전문의가 상근하는 요양기관에서 동일환자의 다른 상병에 대하여 전문과목 또는 전문 분야가 다른 진료담당 의사가 각각 진찰한 경우에는 진찰료를 각각 산정할 수 있다.}
	\item \uline{진료담당의사가 검사\bullet 방사선 진단 등을 처방지시하였으나 요양기관의 사정에 의하여 진료 당일에 검사\bullet 방사선 진단 등을 실시하지 못한 경우에는 검사\bullet 방사선 진단을 실시한 당일의 진찰료는 산정하지 아니한다.}
	\item 의료법 제18조에 따라 요양기관인 의료기관의 의사 또는 치과의사가 작성\bullet 교부한 처방전에 따라 요양기관인 약국 또는 한국희귀의약품센터에서 \uline{조제받은 주사제를 투여받기 위해서 당해 요양기관에 당일에 재내원하는 경우에는 진찰료를 별도 산정하지 아니한다.}
	\end{enumerate}
\end{enumerate}

HRT 3달치 처방받고 다음에도 계속 HRT처방을 받기위해 방문하시는분은 초진 or 재진 ?
\begin{quotebox}
재진. 해당 상병의 치료가 종결되지 아니하여 계속 내원하는 경우에는 내원 간격에 상관없이 재진환자임
\end{quotebox}
HRT시 페경기전후 장애 상병쓰는데, 이것도 만성질환으로 들어가나요? 환자분이 안젤릭 한달 처방받고 그 다음 달에 오셔서(한달 조금 넘어) 초진으로 되었은데 처방내리니 이번 청구에서 재진으로 삭감조정되더라구요.. 그래서 폐경치료도 만성질환으로 들어가는 지 궁금해서요?
\begin{quotebox}
재진
\end{quotebox}

HRT관리 받으시는 분이 혈액검사상 고지혈증이 관찰되어 치료하는 경우는? 

\begin{commentbox}{재진}
\begin{itemize}\tightlist
\item 하나의 상병에 대한 진료를 계속 중에 다른 상병이 발생하여 동일 의사가 동시에 진찰을 한경우
\item 즉 \emph{상병을 두개 넣은 경우네는 재진에 해당}
\item 또는 상병을 고지혈증만 넣는다 해도 30일 이내라면 재진
\end{itemize}
\highlight{초진}
\begin{itemize}\tightlist
\item 고지혈증 \emph{상병만 넣고} N951 상병으로 진찰 받은지 한달이 넘은 경우.
\end{itemize}
\end{commentbox}

2월 1일 질염환자가 3월1일에 다시 질염으로 재발되어 왔다면? 
\begin{quotebox}
초진. 해당 상병의 \emph{치료가 종결된 후 동일 상병이 재발하여 진료를 받기 위해서 내원한 경우에는 초진후 30일 이내에 내원한 경우에는 재진환자}로 본다. 치료의 종결이라 함음 해당 상병의 치료를 위한 내우언이 종경되었거나, 투약이 종결되었을때를 말함
\end{quotebox}

한달 이내에 트리코모나스 질염과 캔디다 질염으로 각각 다른 상병으로 내원하면? 
\begin{quotebox}
재진.하나의 상병에 대한 진료를 계속중에 다른 상병이 발생하여 동일 의사가 동시에 진찰을 한 경우(재진진찰료)
\end{quotebox}

엄청나 산부인과의 엄청난 선생님에게 진료를 본 환자가 한달이 채 안되어서 같은 산부인과 더엄청난 선생님이 다른 상병으로 진료 받는 경우?
\begin{commentbox}{\highlight{초진}}
\begin{itemize}\tightlist
\item 재진의 조건 : 같은 진료과목, 같은 병원이지만 같은 상병이 아니므로 초진
\item 초재진 구분의 3대 기준 
	\begin{enumerate}\tightlist
	\item 상병이 같으냐? 다르냐?
	\item 같은 병원이냐? 아니냐?
	\item 같은 진료과목 의사이냐? 아니냐?[같은 선생님이냐? 아니냐가 아닙니다]
	\end{enumerate}
\end{itemize}
\end{commentbox}