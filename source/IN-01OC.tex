\section{경구피임약}
\subsection{경구피임약 복용해도 될까요?}
청소년인 경우에는 ?
\begin{quotebox}
아이의 성장이나 reproductive system의 발달에 영향이 있다는 보고는 없다. 또한 bone mineral density의 변화를 주는 것 같지는 않다. 결론적으로 sexually active pubertal girl에서 사용은 문제가 없다.
\end{quotebox}
40세 이상인 경우에는?
\begin{quotebox}
Women aged over 40 years can be advised that no contraceptive method is contraindicated by age alone (Grade C). 사용에 문제가 없다.\par
경구피임약의 최대복용 가능 기간에 대한 한계는 정해져 있지 않으며, 여성 개개인의 임상적 상태에 따른 경구피임약 사용 이득과 위험성에 대한 평가가 이뤄진 후 사용되어야 한다.\par
건강상의 위험요인이 없는 비흡연 여성은 폐경때까지 복용가능하다.
\end{quotebox}
흡연자의 경우에는 ?
\begin{quotebox}
\begin{itemize}\tightlist
\item 35세 이상의 흡연여성은 경구피임약 사용이 금기이다.
\item 35세 미만에서는 흡연이 경구피임약 사용의 금기는 아니다.
\end{itemize}
\end{quotebox}
내과적 질환이 있는 여성에서 경구피임약 사용은?
\begin{quotebox}
\begin{itemize}\tightlist
\item 고혈압/고지혈증 : 일반적으로 사용이 권장되지 않으며, 다른 심혈관 질환의 위험 요인여부를 고려해 판단되어야 한다.
\item 당뇨 : 임신성 당뇨의 과거력이 있는여성, 합병증이 동반되지 않은 당뇨 환자에게서는 사용이 가능하다.
\item "It can be stated definitively that OC use does not produce an increase in DM"
\item 비만 : 경구피임약을 복용하는 비만 여성은 비사용 비만 여성에 비해 VTE가 더 많이 생길 수 있다. 그러나 다른 위험 요인이 없는 경우에는 경구피임약의 복용이 가능하다.
\end{itemize}
\end{quotebox}
현재 복용약물이 있는 여성에서 경구피임약 사용은?
\begin{quotebox}
경구피임약의 농도를 낮추는 약물 : Phenytoin, Carbamazepine, Phenobarbital과 같은 enzyme inducer와 Rifampin등의 약제는 경구피임약의 흡수를 감소시켜 파탄성 출혈및 피임실패를 유발할 수 있다.
\end{quotebox}
\begin{commentbox}{Afer EC OC}
\begin{itemize}\tightlist
\item Any regular contraceptive method can be started immediately after the use of ECPs :  \uline{ECP 복용 당일 또는 바로 다음날부터 경구피임약 시작가능}
\item LNG함유 응급피임약 복용 후에는 7일간, UPA함유 응급피임약 복용 후에는 14일간 콘돔과 같은  \uline{back-up 피임법을 사용}해야 한다.
\item 만약 환자가 임신이 되는 경우,  \uline{임신 초기의 응급피임약 및 경구피임약에 대한 노출은 태아의 기형을 유발하지 않는다}\footnote{The risk of birth defects with oral contraceptive ingestion during pregnancy concluded that there was no increase in risk for major malformation, congenital heart defects, or limb reduction defects.}는 사실을 주지시킨다.
\item 이 방법은 환자들의 complaiance를 증대시키는 반면, \uline{breakthrough bleeding등의 부작용은 증가시키지 않는다.}
\end{itemize}
\end{commentbox}

\subsection{Qlaira 클래라}
\myde{}{
\begin{itemize}\tightlist
\item[\dsjuridical] N921 불규칙적 주기를 가진 과다 및 빈발 월경 
\item[\dsmedical] 클래라 비급여 처방
\end{itemize}}
{[효능효과]
\begin{enumerate}[1.]\tightlist
\item 경구피임
\item 피임법으로 경구 피임제를 선택한 여성에 한하여 기질적 원인이 없는 월경과다
\end{enumerate}
}

\subsection{Yaz 야즈정}
\myde{}{
\begin{itemize}\tightlist
\item[\dsjuridical] N943 월경전긴장증후군 \footnote{FDA approves New indication for Yaz to treat emotional and physical symptoms of premenstrual dysphoric disorder}
\item[\dsjuridical] N944 원발성 월경통
\item[\dsjuridical] L709 상세불명의 여드름
\item[\dsmedical] 야즈 비급여 처방
\end{itemize}
}{
[효능효과]
\begin{itemize}[-]\tightlist
\item 여성에서의 피임
\item 피임법으로서 경구피임약을 사용하고자 하는 여성에서 월경전불쾌장애 증상의 치료
\item 피임을 위해 경구피임약을 사용하려는 피임에 금기가 아닌 14세 이상의 초경후 여성의 중등도 여드름(acne vulgaris) 치료
\item 피임법으로서 경구피임약을 사용하고자 하는 여성에서 월경곤란증의 치료
\end{itemize}
}
