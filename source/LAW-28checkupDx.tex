\section{건강검진 당일 진찰료 산정 관련}
\Que{공단건강검진 실시 당일에 환자의 기존 질환에 대한 진찰을 하고 처방전을 발급하였습니다. 이런 경우 진찰료는 별도로 산정이 가능한가요?}
\Ans{건강검진 당일 동일 요양기관에서 건강검진과는 별도로 질환에 대한 진찰이 이루어져 처방이나 별도의 진료행위가 발생한 경우는 진찰료의 50\% 산정이 가능합니다. 이 때 진찰료 산정코드 세 번째 자리에 일반건강검진은 3, 생애전환기 건강검진은 4로 기재하여야 하며, 줄번호 단위 특정내역 구분코드(JT018)란에 해당 사유를 기재하여 청구하여야 합니다.}

\Que{국가검진 암검진 진찰료도 공휴일코드로 넣음 삭감될까요? 국가검진 진찰료 토요일 가산청구가 9-오후1시까지만 되던데. 그이후 시간은 평일 진찰료만 넣나요? 아님 추가가능한 게 있는지요}
\Ans{자궁암검진이외에 부인과 질환으로 치료를 같이한경우 
\begin{itemize}\tightlist
\item 평일 9시부터 6시까지는 AA154005 초진 암검진 당일 진찰 로 진찰료 청구 
\item 평일 6시이후에 자궁암검진 시행시 야간가간코드인 A154015초진진찰료-의원,보건의료원내의과,, 산정이 가능합니다.
\item 토요일가산청구는 AA154035  초진 진찰료-의원,보건의료원내의과 (토요09-13, 암검진 당일 진찰) 추가 산정하면 되고 
\item 그이후 시간은 AA154055초진진찰료-의원,보건의료원내의과(공휴일, 암검진 당일 진찰)산정하면 됩니다.
\end{itemize}
}


\begin{commentbox}{건강검진 실시 당일 진료 시 진찰료 산정방법}
\begin{enumerate}[1.]\tightlist
\item 「국민건강보험법」 제52조에 의거 가입자 및 피부양자에게 실시하는 건강검진 당일 동일 요양기관에서 건강검진과는 별도로 질환에 대한 진찰이 이루어져 진찰 이외에 의사의 처방(약제 처방전 발급, 「건강보험 행위 급여·비급여 목록표 및 급여 상대가치점수」에 의하여 산정 가능한 진료행위)이 발생한 경우 해당 진찰료는 다음과 같이 산정함\newline
- 다 음 - \newline
「건강보험 행위 급여·비급여 목록표 및 급여 상대가치점수」 제1편 제2부 제1장 기본진료료[산정지침] 1. 진찰료 ‘가’ 에 의거 초진(또는 재진)진찰료의 50\%를 산정하며, 코드는 다음과 같이 기재함.진찰료 산정 사유에 대하여는 진료기록부에 기록하고, 「요양급여비용 청구방법, 심사청구서·명세서서식 및 작성요령」에 의하여 작성·청구토록 함
	\begin{enumerate}[(가)]\tightlist
	\item 일반건강검진(생애전환기 건강검진 포함) 시 질환에 대한 진찰이 이루어진 경우 : 산정코드 세 번째 자리에 일반건강검진은 3, 생애전환기 건강검진은 4로 기재
	\item 암검진 시 질환에 대한 진찰이 이루어진 경우 : 산정코드 세 번째 자리에 5로 기재
	\item 영유아 건강검진 시 질환에 대한 진찰이 이루어진 경우 : 산정코드 세 번째 자리에 2로 기재
	\end{enumerate}
\item 상기 ‘1’항에도 불구하고 건강보험 행위 급여·비급여 목록표 및 급여 상대가치점수 제1편 제2부 제1장 기본진료료[산정지침]에 의거 2개이상의 진료과목이 설치되어 있고 해당 과의 전문의가 상근하는 요양기관에서 건강검진 당일 검진실시 의사와 전문과목 및 전문분야가 다른 진료담당의사가 건강검진과는 별도로 질환에 대하여 진료한 경우에 한하여 초진(또는 재진) 진찰료를 산정할 수 있음
\item 또한, 건강검진을 실시한 요양기관에서 동일 의사에게 검진 결과에 대해 다른 날 설명하는 것은 검진결과 상담에 해당되어 진찰료를 별도 산정할 수 없으나, 검진결과 이상소견에 대해 단계적 정밀검사 또는 별도의 진료가 이루어진 경우에는 재진진찰료를 산정함. 고시 제2012-153호 (2012.12.01. 시행)
\end{enumerate}
\end{commentbox}


\begin{commentbox}{건강검진 당일 진찰료 산정방법에 관한 질의응답}
\begin{enumerate}[1.]\tightlist
\item 건강검진 실시 당일 진찰료 산정 시 공휴 야간 가산 산정여부
	\begin{itemize}\tightlist
	\item 건강검진 실시 당일 동일 요양기관에서 건강검진과는 별도로, 공휴일 또는 야간가산 적용시간에 질환에 대한 진찰이 이루어져 진찰 이외에 의사의 처방이 발생한 경우 공휴일·야간가산은 인정함 (2012.04.01. 진료분부터 적용)
	\item 다만, 현행 진찰료 야간가산의 적용기준(고시 제2006-09호, ’06.02.01 시행)에서 정한 야간가산 적용시간에 내원한 경우는 진료개시 시간을 기준으로 야간가산 적용하고,
	\item 야간가산 적용시간 외 시간에 내원한 경우는 환자가 요양기관에 도착한 시간을 기준으로 야간가산 적용
	\end{itemize}
\item 건강검진 실시 당일 별도의 질환에 대한 진료시 만성질환관리제에 의한 재진 진찰료 본인부담률 경감 대상적용환자인 경우 본인부담률은?
	\begin{itemize}\tightlist
	\item 건강검진 실시 당일 동일 요양기관에서 건강검진과는 별도로 질환에 대한 진찰이 이루어져 초진(또는 재진)진찰료의 50\%를 산정 시\newline
- 만성질환관리제에 의한 재진 진찰료 본인부담률 경감대상 환자인 경우에는 재진 진찰료 본인부담률을 20\%로 적용함
	\end{itemize}
\item 건강검진 실시 당일 위내시경 검사시 부스코판 등 전처치 약제가 처방된 경우 별도의 진찰료 50\%산정 가능여부
	\begin{itemize}\tightlist
	\item 건강검진 과정에 해당되어 전처치약제가 처방된다 하더라도 별도의 질환에 대한 진료로 볼 수 없으므로, 전처치 약제비 및 진찰료를 산정할 수 없음\newline - 공단에서 실시하는 건강검진 검사항목 중 위·대장 내시경 검사 검진비용에 주사약제 등 전처치 비용이 포함되어 있음
	\end{itemize}
\end{enumerate}
\end{commentbox}


