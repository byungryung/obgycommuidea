\section{행위별 수가제: 질식분만 총액}
\subsection{분만 본인부담 면제}
\begin{commentbox}{자연분만시 본인부담금 면제 대상 적용범주}
\begin{enumerate}\tightlist
\item 국민건강보험법시행령 [별표2]제3호의 규정에 의하여 본인이 부담할 비용을 면제하는 자연분만은 자435분만, 자436둔위분만, 자438 제왕절개술기왕력이 있는 질식분만, 카1 조산료, 타2 보건진료소 조산료, 타3다 보건지소 조산료, 타4다 보건소 조산료를 말함.
\item 다만, 자연분만을 시도하였으나 제왕절개술을 시행한 경우, 분만을 위해 입원하였으나 분만이 이루어지지 않은 경우는 해당되지 아니함.
\end{enumerate}  
\end{commentbox}
\prezi{\clearpage}
\Que{자연분만을 위해 내원하였으나 사산한 경우}
\Ans{사산한 태아를 자연분만으로 처치한 경우 인정}

\par  
\medskip
\prezi{\clearpage}
\Que{자연분만 이후 분만으로 인한 합병증 치료도 포함이 되는지? 적용 기간은?}
\Ans{합병증 치료시까지 인정}

\par  
\medskip 
\prezi{\clearpage}
\Que{자연분만 이후 입원기간 중 분만과 무관한 타상병(기왕증 포함)에 대한 진료가 이루어진 경우도 면제 대상이 되는지?}
\Ans{면제 대상이 아니므로 분리청구함}
  
\par  
\medskip 
\prezi{\clearpage}
\Que{임신 중 입원으로 인한 진료}
\Ans{임신으로 인한 합병증 치료 및 입원을 포함하여 임신 중 입원진료는 불인정}

\par  
\medskip 
\prezi{\clearpage}
\Que{분만전 임신유지를 위한 ABR 등의 원인으로 계속 입원 중에 분만이 이루어진 경우 면제 대상 시점은?}
\Ans{분만을 위한 처치가 이루어진 시점을 기준으로 인정함. (분만전처치 시점부터 인정)}

\par  
\medskip 
\prezi{\clearpage}
\Que{자연 분만시 무통분만도 모두 면제 대상에 포함인지?}
\Ans{분만의 과정으로 면제 대상임}
  
\par  
\medskip 
\prezi{\clearpage}
\Que{자연분만 후 출혈 등 합병증으로 당해 기관에서 치료가 불가하여 타 병원으로 전원하여 치료시 면제여부?}
\Ans{당해병원에서 치료가 불가하여 타병원으로 이송한 것은 당해병원에서 치료한 경우와 동일하게 면제대상임
\begin{itemize}\tightlist
\item 단, 타병원 이송사유가 본인의 원이나 필요에 의한 것은 면제대상이 아님
\item 요양기관을 달리하여 합병증치료후 분만을 실시한 처음 요양기관으로 다시 이동하여 치료하는 경우도 면제대상이 아님(2005.03.29.)
\end{itemize}}

\par  
\medskip 
\prezi{\clearpage}
\Que{자연분만하고 퇴원 후 출혈 등 합병증으로 재입원 치료시}
\Ans{면제대상이 아님}

\par  
\medskip
\prezi{\clearpage} 
\Que{요양기관 이외의 장소(요양기관으로 이동중 자동차내 등)에서 분만후, 분만후 처치 등을 위해 요양기관 내원시}
\Ans{긴급 부득이하게 분만의 일부 과정이 요양기관 이외의 장소에서 이루어지고 그 이후는 분만후 처치가 요양기관에서 이루어진 경우 면제대상임 (2005.03.29.)}

\par  
\medskip
\prezi{\clearpage} 
\Que{분만관련 약제(빈혈약, 소염제 등)를 분만입원기간 이후 기왕증 치료기간에 투여한 경우
예) 고혈압이 있는 환자가 정상 분만후 고혈압 검사등을 위해 10일간 입원하여 빈혈이 있어 빈혈약을 입원5일째부터 투여}
\Ans{분만입원기간이후 발생한 제반비용은 면제대상이 아님}

\par  
\medskip 
\prezi{\clearpage}
\Que{기왕증이 있는 산모가 분만을 전제로 입원하여 안전한 분만을 위하여 기왕증관련 약제투여 및 검사등을 시행한 경우
\begin{description}\tightlist
\item[예1)] Rh- 산모에게 분만전 검사 또는 약제 투여시
\item[예2)]진단된특발성혈소판감소성자반증(known ITP)인 환자로 분만시 예상되는 대량출혈에 대비키 위해 분만전 시행한 수혈
\end{description}}
\Ans{기왕증이 있더라도 분만을 위해 입원하여 합병증 등 위험요소를 줄여 안전분만을 유도하기 위한 약제투여 및 수혈등의 치료가 이루어진 경우는 면제대상임
(2005.03.29.)}
\prezi{\clearpage}
\subsection{추가 환자부담금 사항}
\begin{itemize}\tightlist
\item 4인실이상 상급병실료 (법정비급여)
\item 초음파 비용 및 검사료 (법정비급여)
\item 유도분만상급병실비
\item VBAC
\item 흡입/겸자분만
\item 야간/공휴일 가산
\item 고위험산모(만 35세이상등)
\item 산모식
\item 비급여항목 (임의비급여)
\item 신생아 케어(비타민 K주사, O2 inhalation 산소치료, O2 3L(1-10L) ,혈당검사)
\end{itemize}
\prezi{\clearpage}
\subsection{분만 가산 관련}
\Que{35세 이상인 등록 장애인이 자연분만 시 등록장애인 가산 및 35세 이상 산모의 분만 가산을 함께 적용받나요?}
\Ans{보건복지부 고시 제2013-19호(‘13.2.15. 일부 시행)에 따르면, 자연분만 수가에 신설된 주항 규정에 따르면, 2013년 2월 15일부터는 만 35세 이상 산모에 대하여 소정점수의 30\%를 추가 가산하며, 다만, ‘장애인으로 등록되어 있는 장애인에 대하여 소정점수의 50\%를 가산받는 경우’에는 35세 이상 산모의 30\%를 추가 가산 적용받지 아니하며, 50\% 가산 적용만 받습니다.} 

<장애 구분 및 연령 구분별 자연분만 가산 적용 분류> \par
비장애인 35세 미만 : 산정코드 (첫째자리) : 0 35세 이상 가산 미적용/ 장애가산 미적용\par
비장애인 35세 이상 : 산정코드 (첫째자리) :5 35세이상 30\% 가산 / 장애가산 미적용 \par
장애인 35세 미만 / 코드 신설* / 35세이상 가산 미적용 / 장애가산 50\% 가산 \par
장애인 35세 이상/ 상동 /상동/ 상동 \par
* 분만 전 처치 : RA370, 분만 후 처치 : RA376\par

\prezi{\clearpage}
\Que{35세 이상 산모 및 장애인 산모가 자연분만(분만 전\cntrdot{}후처치 포함)시 공휴 및 야간가산도 함께 적용받나요?}
\Ans{35세 이상 산모 및 장애인 산모가 야간 및 공휴일에 자연분만(분만 전\cntrdot{}후 처치) 시 기본 인상 및 금번 개정된 가산률 적용 외에도 공휴 및 야간가산을 적용받습니다. }

<야간 및 공휴일에 정상분만(초산 및 제1태아 기준) 시 적용코드 및 가산범위>\par
  
장애구분 /연령/기본코드/산정코드첫째자리/둘째자리/셋째자리\par
비장애인/ 35세 미만/ R4351/3/1(야간), 5(공휴)/0 /\par
비장애인 /35세 이상/ R4351/5/1(야간), 5(공휴)/0/\par
장애인 /35세 미만, 35세 이상/ RA431 3 /1(야간), 5(공휴)/0 \par
(2013-03-05)


