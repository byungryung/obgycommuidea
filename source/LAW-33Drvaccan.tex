\section{요양병원의 의사인력 휴가 시 청구 방법}
\Que{의사 1인, 한의사 2인이 근무하는 요양병원에서 의사가 해외 출국으로 부재중인 경우 한의사만 근무기간 동안 의과 입원환자 요양급여비용을 청구할 수 있나요?}
\Ans{「의료법」 제2조 및 제27조에 의거 의사는 의료와 보건지도를, 한의사는 한방의료와 한방보건지도를 임무로 하며, 의료인은 면허된 것 이외의 의료행위를 할 수 없도록 규정하고 있습니다. 따라서 의사 부재중 의과 입원환자에 대한 의과진료를 한의사가 대신 할 수 없으므로 기본적인 의사의 진료행위가 없는 상태에서 발생한 입원료, 식대 등 의과 요양급여비용은 인정하기 곤란하며, 단 동 상황에서 시행한 한의사의 한방 의료행위는 해당내역에 한해 인정할 수 있습니다.
 보험급여과-513호 (2009.02.12.)}

\begin{commentbox}{「의료법」 제2조(의료인)}
\begin{enumerate}[①]\tightlist
\item  이 법에서 “의료인”이란 보건복지부장관의 면허를 받은 의사·치과의사·한의사·조산사 및 간호사를 말한다. <개정 2008.2.29, 2010.1.18>
\item  의료인은 종별에 따라 다음 각 호의 임무를 수행하여 국민보건 향상을 이루고 국민의 건강한 생활 확보에 이바지할 사명을 가진다.
	\begin{enumerate}[1.]\tightlist
	\item  의사는 의료와 보건지도를 임무로 한다.
	\item  치과의사는 치과 의료와 구강 보건지도를 임무로 한다.
	\item  한의사는 한방 의료와 한방 보건지도를 임무로 한다.
	\end{enumerate}
\end{enumerate}	
\end{commentbox}


\begin{commentbox}{「의료법」 제27조(무면허 의료행위 등 금지)}
\begin{enumerate}[①]\tightlist
\item  의료인이 아니면 누구든지 의료행위를 할 수 없으며 의료인도 면허된 것 이외의 의료행위를 할 수 없다. 다만, 다음 각 호의 어느 하나에 해당하는 자는 보건복지부령으로 정하는 범위에서 의료행위를 할 수 있다. <개정 2008.2.29, 2009.1.30, 2010.1.18>
	\begin{enumerate}[1.]\tightlist
	\item  외국의 의료인 면허를 가진 자로서 일정 기간 국내에 체류하는 자
	\item  의과대학, 치과대학, 한의과대학, 의학전문대학원, 치의학전문대학원, 한의학전문대학원, 종합병원 또는 외국 의료원조기관의 의료봉사 또는 연구 및 시범사업을 위하여 의료행위를 하는 자
	\item  의학·치과의학·한방의학 또는 간호학을 전공하는 학교의 학생
	\end{enumerate}
\item  의료인이 아니면 의사·치과의사·한의사·조산사 또는 간호사 명칭이나 이와 비슷한 명칭을 사용하지 못한다.
\end{enumerate}
\end{commentbox}