\begin{myshadowbox}
\begin{enumerate}[3.]\tightlist
\item 제2호 규정에 따른 질병군 입원진료에는 다음의 각 항목을 포함한다.
	\begin{enumerate}[가.]\tightlist
	\item 제2부 각 장에 분류된 질병군으로 \textcolor{red}{응급실ㆍ수술실 등에서 수술을 받고 연속하여 6시간 이상 관찰} 후 귀가 또는 이송한 경우
	\item 제2부 각 장에 분류된 질병군 중 수정체 소절개 수술 단안, 수정체 소절개 수술 양안, 수정체 대절개 수술 단안, 수정체 대절개 수술 양안, 기타항문 수술, 서혜 및 대퇴부 탈장수술(장관절제 미동반) 단측, 서혜 및 대퇴부 탈장수술(장관절제 미동반) 양측, 복강경을 이용한 서혜 및 대퇴부 탈장수술(장관절제 미동반) 단측, 복강경을 이용한 서혜 및 대퇴부 탈장수술(장관절제 미동반) 양측 질병군으로 수술을 받고 6시간 미만 관찰 후 당일 귀가 또는 이송하는 경우
	\end{enumerate}
\end{enumerate}
\end{myshadowbox}
\prezi{\clearpage}		
\Que{자궁경부의 근종인경우 \highlight{외래에서 입원없이} 시행하는 자궁근종절제술-질부접근(R4123) 이 포괄수가제 포함 대상인가요?\par
포괄수가제 포함 대상인경우 행위별청구외에 추가로 다른 청구방법이 있나요?}

\begin{commentbox}{자궁근종절제술-질부접근(R4123) 의 포괄수가제 포함여부}
질병군(DRG) 포괄수가는 국민건강보험법시행령 제21조제3항제2호에 따라 복지부장관이 별도 고시하는 7개 질병군으로 입원진료를 받은 경우에 적용하며, 질병군 입원진료는 질병군 급여 일반원칙에 따라 다음의 항목을 포함하고 있습니다.\par
\begin{center}\emph{- 다 음 -}\end{center}
\begin{itemize}\tightlist
\item 7개 질병군으로 응급실ㆍ수술실 등에서 수술을 받고 연속하여 6시간 이상 관찰 후 귀가 또는 이송한 경우 
\item 7개 질병군 중 수정체수술(대절개 단안 및 양안, 소절개 단안 및 양안), 기타항문수술, 서혜 및 대퇴부탈장수술 단측 및 양측(복강경 이용 포함)의 수술을 받고 6시간 이상 관찰 후 당일 귀가 또는 이송한 경우
\end{itemize}
따라서 자궁근종으로 자궁근종절제술-질부접근(R4123)을 받고 \textcolor{red}{6시간 미만 관찰 후 당일 귀가 또는 이송한 경우(외래)}는 \highlight{행위별청구대상}임을 알려드립니다.
\end{commentbox}
\prezi{\clearpage}
\par
\medskip
\Que{제왕절개분만 후 산후출혈 등의 합병증으로 당일 이송한 경우(6시간 미만 진료) 질병군 적용여부}
\Ans{제왕절개분만 후 당일 귀가 또는 이송한 경우 \textcolor{blue}{6시간미만 진료도 질병군 적용 대상}임}
\prezi{\clearpage}
\par
\medskip
\Que{쌍둥이 임산부가 자연분만과 제왕절개분만으로 분만방법을 달리하여 출산한 경우 청구방법은?}
\Ans{전체 진료내역을 \textcolor{red}{행위별수가제로 청구함.} 자연분만과 제왕절개분만은 본인부담률이 다르므로 명세서를 분리하여 청구하고 제왕절개분만명세서에는 상해외인에 “D"를 기재하여 청구함}
\prezi{\clearpage}
\par
\medskip
\Que{제왕절개분만의 질병군요양급여비용 청구시 신생아진료비용 청구방법은?}
\Ans{제왕절개분만의 질병군 요양급여비용 청구와 별도로 신생아 진료비용은 \textcolor{red}{행위별로 청구}함(2010년 7월부터 적용)}
