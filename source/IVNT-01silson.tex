\section*{IVNT와 실손보험}
\subsection*{IVNT 시장이 급성장 하는 이유는?}
\begin{enumerate}[첫번째]\tightlist
\item 실손보험이 되서 입니다. IVNT 맞는 환자들의 심리는 맞으면 좋을것 같기는 한데 비싼 비용을 지불하기는 싫고 실손이 된다고 하니 한번 맞아 볼까? 입니다.
\item RISK < BENEFIT \newline 실손 보험사에서 문제 제기 할 수도 있고 귀찮아 질수도 있습니다. 하지만 급여 영영의 진료가 아니기 때문에 문제된 환자의 문제된 제품만 문제가 됩니다. 또한 미리 대비하시면 이런 일도 원초적으로 막을수 있으며 문제의 해결을 위해 소요되는 손실에 비해 이득이 크기 때문에 시장이 계속 넓어질 수 밖에 없습니다.
\end{enumerate}\tightlist

\subsection*{태반주사 어떻게 활용할 것인가?}
\begin{commentbox}{멜스몬}
식약처 허가 사항[효능/효과] \emph{갱년기 장애 증상 개선 → 실손보험} 적용가능 \par
\emph{Protocol} 주1회 1회에 3Ample SC, 진료소견서/영수증 발급
\begin{description}\tightlist
\item[적합상병] N951 폐경기 및 여성의 갱년기 상태
\item[소견서 내용] 상기분은 안면홍조, 불면, 불안증, 답답증 등 갱년기 장애 증상이 심하여 태반주사(멜스몬) 요법 시행하심.
\end{description}
\end{commentbox}

\begin{commentbox}{라이넥}
식약처 허가 사항[효능/효과] \emph{만성 간질환에 있어서의 간기능 개선 → 실손보험} 적용가능 \par
\emph{Protocol} 보통 1일 2ml 피하 또는 근육주사, 증상에 따라 1일 2-3회 주사, 진료소견서/영수증 발급\par
\begin{description}\tightlist
\item[적합상병] K796 상세불명의 간질환/간염 관련 상병
\item[소견서 내용] 상기 환자분 평소 과한 음주와 피로가 심하신 분으로 OT/PT 검사상 정상 수치보다 높아 태반주사(라이넥) 시행하심
\end{description}
\end{commentbox}

