%\kswrapfig{edenlogo}{\HUGE{구리 자궁내피임기구 시술 동의서 }\par \normalsize http://www.edenhospital.co.kr}
%\includegraphics[scale=.5]{edenlogo}\Huge{구리 자궁내피임기구 시술 동의서및 설명서}\normalsize\\
\begin{tabular} {c m{40cm}}
\hspace{0.5cm}\raisebox{-.45\totalheight}{\includegraphics[scale=.75]{edenlogo}} & \HUGE{\textsf{구리 자궁내피임기구 시술 동의서}} 
\end{tabular}
\rule{\linewidth}{5pt}
\tabulinesep =_2mm^2mm
\begin {tabu} to\linewidth {X[3,l]X[3,l]X[3,l]X[3,l]X[3,l]X[3,l]X[3,l]X[3,l]} %\tabucline[.5pt]{-}
%\rowcolor{Blue!40} & & & & & & & \\  
\multicolumn3{l}{\Large{\textsf{지속기간 }}} & \multicolumn4{l}{ \hspace{1.5cm} \Large{5 년} } \\
\multicolumn3{l}{\Large{\textsf{시술일/제거일 }}} & \multicolumn2{l}{\Large{200}\hspace{1.0cm}  \Large{년}\hspace{1.0cm}  \Large{월}\hspace{1.0cm}  \Large{일} /} & \multicolumn2{l}{\Large{200}\hspace{1.0cm}  \Large{년}\hspace{1.0cm}  \Large{월}\hspace{1.0cm}  \Large{일}} \normalsize\\
\end{tabu}

\subsection*{시술방법}
\begin{center}
\includegraphics[scale=.4]{LOOPinsertion} %p702
\end{center}
\subsection*{부 작 용}
현재까지 나온 피임방법중에 \textcolor{red}{100\%완벽한 피임방법은 없습니다.} 모든 피임방법에는 부작용과 피임실패의 확률이 있습니다. 특히 부작용이 생겼을때 피임실패 확률이 올라가게 됩니다. 구리 자궁내피임기구의 \textcolor{red}{피임실패 확률은 0.8입니다.}
\begin{enumerate}\tightlist
\item 구리 자궁내피임기구는\emph{ 월경 출혈이나 월경 곤란을 증가}시킬 수 있습니다.
구리 자궁내피임기구를 사용하는 동안 위의 증상이 나타난 경우 기기를 제거하는 것을 고려
\item \emph{골반염의 가능성이 있습니다.}  구리 자궁내피임기구 사용자에서 삽입하고 1달 동안 골반염 비율이 가장 높았으며 이후에는 감소합니다. 골반염은 수정능력을 손상시킬 수 있으며 자궁외 임신의 위험성을 증가시킬 수 있다. 매우 드물지만, 다른 부인과 진찰이나 수술과 마찬가지로 자궁내피임기구 삽입 이후에 중증의 감염이나 폐혈증(연쇄구균 A군에 의한 폐혈증 포함)이 발생할 수 있습니다.
\item \emph{방출} 자궁내 피임기구가 부분적이거나 전체적으로 방출될 때는 출혈 또는 통증의 증상이 나타날 수 있습니다. 그러나 인지하지 못한 상태로 기구가 자궁강에서 방출될 수도 있다. 부분적으로 방출되면 구리 자궁내피임기구의 유효성이 감소할 수 있으며 이러한 기구는 제거하고 새 기구를 삽입해야합니다.
\item \emph{천공} 자궁내피임기구로 인하여 자궁 또는 자궁경부에 구멍이 나거나 관통할 수 있으며 대부분 삽입시에 발생합니다. 이러한 경우, 자궁내피임기구는 가능한 빨리 제거되어야 합니다. 천공의 위험은 자궁 후굴이 고정된 여성에서 증가할 수 있다.
\item \emph{자궁외임신} 자궁내피임기구를 사용하는 동안 자궁외임신이 발생할 수 있으나 구리 자궁내피임기구를 사용하는 여성이 피임을 하지 않는 여성보다 자궁외 임신의 전반적 위험성이 더 높지는 않습니다.
\end{enumerate}

*발생가능한 이상반응은 아래와 같습니다.
\begin{itemize}\tightlist
\item 월경출혈 증가
\item 월경곤란증가
\item 하복부 또는 등통증
\item 빈혈
\item 피임에 실패하여 임신할수 있고 특히 자궁외임신일수 있음
\item 피부 알레르기
\item 자궁내피임기구를 사용하는 동안 골반염이 발생할 수 있습니다.
\item 자궁내피임기구 또는 그 일부가 자궁벽의 구멍을 내거나 관통할 수 있다.
\item 폐혈증(연쇄구균 A군에 의한 폐혈증 포함)의 사례가 자궁내장치의 삽입 이후에 보고 되었다. 
\end{itemize}

\subsection*{부작용 예방을 위한 환자 수칙}
\begin{enumerate}\tightlist
\item 첫 한 달 동안에 부작용의 빈도가 많이 증가하므로(골반염, 방출, 천공등) 꼭 루프한달후에는 외래에 방문하여 루프가 제자리를 잘 잡았는지 골반염이 없는지, 출혈양과 생리통이 심해져 있는지를 확인해 보아야 합니다.
\item 한달안에는 다른 보조적인 피임방법을 사용하여서 부작용과 피임실패확률을 최소화 할 수 있습니다.
\item 자궁내피임기구 시술후 3-4시간후에 거즈를 빼시고, 안정하십시요.
\item 위와 같은 환자수칙을 잘 지키시면 피임실패확률은 0.6입니다.
\end{enumerate}
\vspace{3cm}
\tabulinesep =_2mm^2mm
\begin{center}
\begin {tabu} to.75\linewidth {X[3,l]X[3,l]X[3,l]X[3,l]X[3,l]X[3,l]X[3,l]X[7,l]} %\tabucline[.5pt]{-}
\rowcolor{Blue!40} & & & & & & & \\  
\multicolumn8{l}{\textsf{위의 설명을 잘 듣고 시술에 동의합니다.} } \\
\multicolumn3{l}{\textsf{본인 또는 보호자 :}} & \multicolumn4{l}{ } & [사인] \\
\multicolumn3{l}{\textsf{설 명 자 :  }} & \multicolumn4{l}{ } & [사인]/날짜:\\
\end{tabu}
\end{center}
%\end{document}