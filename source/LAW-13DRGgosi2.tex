\subsection{보건복지부 고시 제2014 - 240호(2014.12.30)}
\leftrod{「기타 자궁 수술」및 「자궁부속기 수술」질병군의 가산점수 인정기준}
「복강경을 이용한 기타 자궁 수술(악성종양제외)」, 「기타 자궁 수술(악성종양제외)」,「복강경을 이용한 자궁부속기 수술(악성종양제외)」,「자궁부속기 수술(악성종양제외)」질병군의 가산점수는 진료담당의사의 의학적 판단 하에 임신ㆍ출산능력을 보존하는 수술을 시행한 경우 산정함을 원칙으로 하며 인정기준은 다음과 같이함\par
\begin{center}\emph{- 다 음 -}\end{center}
\begin{enumerate}[가.]\tightlist
\item 임신ㆍ출산을 담당하는 장기의 병변 부위만을 제거ㆍ교정하는 수술을 하여 임신ㆍ출산능력을 보존한 경우\par
다만, 자궁내막증이 있거나 불임(또는 난임) 등으로 임신가능성을 높이기 위해 난소 또는 난관 전절제술을 실시한 경우는 사례별로 인정
\item 임신ㆍ출산을 담당하는 장기의 수술을 동시에 실시하여 그 수술결과로 임신ㆍ출산능력이 보존된 경우
\item 아래의 경우는 가산점수를 산정하지 아니함
	\begin{enumerate}[(1)]\tightlist
	\item 폐경 또는 55세 이상 여성(55세 이상이나 폐경이 아닌 경우 관련자료 첨부시 이를 참조하여 인정)
	\item 기존에 시행한 수술로 임신ㆍ출산 능력을 상실한 경우
	\end{enumerate}
\end{enumerate}