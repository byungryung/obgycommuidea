\section{TOT}
\subsection{TOT수술후  흔한 합병증}
\begin{enumerate}\tightlist
\item 허벅지 통증 (thigh pain) / 멍 (bruise)
\item 출혈 (bleeding) / 혈종 (hematoma)
\item 질벽 천공 (vaginal  wall perforation)
\item 질 미란 (vaginal  erosion) / 테이프 노출 (mesh protrusion) 
\item 배뇨장애 (voiding dysfunction) 
\item 실패 (failure) : 지속성 or 재발성 (persistent  or  recurrent)
\item 방광천공 (bladder perforation) : rare
\end{enumerate}

\subsection{mesh protrusion시  management}
\begin{enumerate}\tightlist
\item 일단 테이프 노출이 확인되면 노출된 메쉬는 반드시 제거해야 한다. 
\item 노출된 메쉬의 상태와 주변조직을 면밀히 확인한다.
   \begin{itemize}\tightlist
   \item clean wound  노출된 메쉬 부분절제(resection)후 봉합   
   \item severe inflammatory wound  최대 박리가능한 메쉬 전절제(total remove)후 봉합
   \end{itemize} 
\item 봉합방법 : chromic catgut 보다는 \textcolor{blue}{vicryl 3-0 (or 4-0)을 권장하며, interrupted, vertical mattress suture with ‘sufficient’ debridement}
\item 봉합할 조직이 약하거나 괴사된 조직의 범위가 넓어 wide debridement후에 
   봉합시에는 vertical이 아닌 horizontal 로 질벽봉합을 권장 (훨씬 firm 하게 됨) 
\item 환자에게 메쉬 제거수술후 요실금이 재발(recurrent SUI)될 수 있음을 인지시킨다. 
\end{enumerate}

\begin{commentbox}{TOT수술후 Surgical intervention}
\begin{itemize}\tightlist
\item MESH resetion(partial, subtotal) : Mesh protursion(moderate이상), thigh pain(severe), urinary retention(severe), postop. aggrevating OAB
\item MESH shortening : early failure(persistent SUI)
\item Redo TOT : late failure(recurrent SUI)
\end{itemize}
\end{commentbox}

\subsection{simple mesh-cut}
\begin{itemize}\tightlist
\item 적응증
  \begin{enumerate}[①]\tightlist
  \item 질벽내로 mesh fiber가 1-2개 노출되어 있을때
  \item MUI 환자가 수술후 OAB증상의 지속 or 악화되어 비약물치료로 호전없고, recurrent SUI를 우려할때 
  \end{enumerate}
\item 처치 
  \begin{enumerate}[①]\tightlist
  \item 질벽절개없이 노출된 mesh fiber만 cut
  \item 질벽절개후  mesh확인하여 요도아래부위에서 scissor로 cut
  \end{enumerate}
\item 단점 : mesh fiber가 다시 노출될 수 있고, OAB증상이 호전되지 않을수 있음
\item 주의점 : 가능한 제한적으로 시행하며, 수술전 환자와 충분한 상담하여 동의후 결정     
\end{itemize}

\subsection{정확한 TOT수술을 위한 surgical tips}
\begin{commentbox}{꿀팁}
\begin{enumerate}\tightlist
\item 환자를 가능한 수술테이블 끝까지 내려서 위치한다  (hyperflexion)
\item 질벽 박리시 너무 얇게 되지 않도록 적절한 두께로 박리한다
\item 폐쇄공 내측(safe entry zone)을  충분히 눌러서  만진 후 절개창을  넣는다  (절개창 부위가 clitoris와 thigh fold가 만나는 지점이  아닐 수 있다)
\item 사타구니 절개창 넣을때  충분히 깊게 (메스 블레이드가 끝까지 다 들어가도록) 넣어 준다 
\item Needle 이 질쪽으로  나올때는 과감하게 시도하되 가볍게 통과되는 느낌이 중요하다
\end{enumerate}
\end{commentbox}
