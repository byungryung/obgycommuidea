\section*{부인과 초음파의 급여}
\begin{myshadowbox}
4대 중증질환이 다음과 같이 의심되는 경우에 초음파 검사는 급여대상임.
\begin{center}\emph{- 다   음 -}\end{center}
\begin{enumerate}[가.]\tightlist 
\item 증상\cntrdot{}징후 또는 타검사상 이상 소견이 있어 해당 질환을 의심하는 경우
\item 4대 중증질환이 의심되고, 특이적인 과거력이 있어 실시하는 경우
\item 무증상환자이나 의심되는 질환이 고위험군으로 분류할 수 있는 근거가 있는 경우(단, 검진 목적으로 무증상 환자에게 주기적으로 실시하는 초음파검사는 비급여대상임)
\item 중증질환 산정특례가 만료된 환자가 증상, 징후 또는 타검사상 이상 소견이 있어 질환의 재발을 의심하여 실시하는 경우 
\end{enumerate}
\end{myshadowbox}
\prezi{\clearpage}
\par
\medskip
\Que{초음파검사의 급여기준 중 질환별 급여대상에 해당되는 환자는 질환에 관계없이 초음파검사의 급여가 가능한가?}
\Ans{초음파검사는 산정특례 대상 중 \textcolor{red}{암, 뇌혈관질환, 심장질환, 희귀난치성질환에 급여함을 원칙으로 하되, 해당 질환으로 인한 합병증의 경우에도 산정특례 적용이 되는 점 등을 감안하여 급여대상 관련 합병증으로 초음파검사를 실시하는 경우}에도 급여함.
\begin{description}\tightlist
\item[Ex>] CIS환자의 경우 CIS와 관계없이 PID로 오면 초음파는 \highlight{비보험} 
\item[Ex>] Breast Ca환자가 Tamoxifen 사용시 부작용으로 인해 자궁내막암의심시 자궁초음파 \highlightR{보험}
\item[Ex>] Tyroid Ca환자가 여성골반초음파를 원하는 경우 비보험
\end{description}
}
\prezi{\clearpage}
\par
\medskip
\Que{중증질환자 산정특례 대상자의 타 상병에 대한 진료분의 본인부담률은 ?}
\Ans{
\begin{itemize}\tightlist 
\item 산정특례 대상 상병 및 관련 합병증에 대한 진료는 특례대상(10\%) 임. 
\item 산정특례 대상과 전혀 관련 없는 타 상병(기왕증 포함)의 진료는 해당되지 않음. 다만, 동일 진료과목(입원)\cntrdot{}동일의사(외래)에게 해당상병과 동시에 진료를 받은 경우에는 특례대상임. (동일의사 \cntrdot{} 동일처방전에 의거 조제하는 약국 약제비도 10\% 적용)
(2005-08-26)
\end{itemize}}
\prezi{\clearpage}
\begin{mdframed}[linecolor=red,middlelinewidth=2]  
\begin{enumerate}\tightlist
\item 유도초음파: 결합행위(시술, 치료, 검사 등)와 함께 시행하는 초음파 (예: 양막내양수주입술시 초음파, 자궁내막생검 시 초음파)\par
* 원래 임신부 산전 초음파와 함께 유도초음파도 급여화 예정이었으나 4대 중증질환자 (암, 뇌혈관, 심장, 희귀난치성 질환) 및 의심자만 유도초음파 급여 적용 하기로하고, 그 외에는 비급여 적용
\item 부인과 초음파: 4대 중증질환자 및 의심자만 급여대상 \par
* 부인과 초음파 급여화 시점에 대해선 아직 구체적 계획 없음
\item 부인과 초음파로는 \highlight[cyan!70]{endometrial Bx시 R/O endometrial malignancy하고 정밀/도플러와 유도초음파II 따로 따로 청구 가능}합니다. \highlight{AUB:POSTMENOPAUSE시에도 R/O endometrial cancer함으로써 정밀/도플러 초음파 산정 가능합니다.} \textcolor{red}{난소암의증과 HSIL이상의 자궁경부암의증으로 conization Bx를 실시}할때 초음파가 보험될것 같습니다.
\end{enumerate}
\end{mdframed}
\prezi{\clearpage}
\medskip
\textsf{여성생식기 초음파 검사 시 접근방법에 따른 수가 산정여부}
\begin{commentbox}{}
나944라 여성생식기 초음파는 여러 접근방법(경직장, 경질, 경회음부)으로 실시되는 검사를 반영한 수가로 접근방법을 불문하고 해당 검사 소정점수를 산정함
\end{commentbox}
\prezi{\clearpage}
\textsf{여성생식기 초음파 수가 산정방법}
\begin{commentbox}{}
일반은 해부학적 이상이 없이 기능적 문제가 있는 경우(단순 질 출혈)에 산정하고, 정밀은 해부학적 이상소견이 있는 경우(여성생식기 종괴, 여성생식기 기형, 종양)에 산정하며, 자궁내 생리식염수와 같이 약제를 검사 목적으로 주입한 경우 나944라(1)‘주’의 소정점수를 산정함.
\end{commentbox}
\prezi{\clearpage}
\textsf{유도초음파 산정방법}
\begin{commentbox}{}
「요양급여의 적용기준 및 방법에 관한 세부사항」초음파검사의 급여기준에 따라 급여목록 중 제2장(검사료) 또는 제9장(처치 및 수술료 등) 에 해당될 경우 산정함.
\end{commentbox}

\textsf{단순\bullet 유도초음파 산정 시 특정내역 기재방법}
\begin{commentbox}{}
단순초음파 또는 유도초음파를 시행한 경우 세부내역을 “JS013"에 기재함.(기재형식) 해부학적 구분코드/수가코드(5단코드)/구체적 사유
	\begin{description}\tightlist
	\item[(예시1)] 중심정맥관 삽입시 확인: 단순초음파(Ⅱ) 청구 ⇒ “L/O1650/"
	\item[(예시2)] 초음파 유도하 갑상선 생검: 유도초음파(Ⅱ) 청구 ⇒ “D/C8591/”
	\item[(예시3)] 초음파 유도하 유방 수술전 tattooing: 유도초음파(Ⅰ) 청구 ⇒ “E/ /수술전 tattooing”
	\item[(예시4)] \highlight{초음파 유도하 자궁내막소파검사} : 유도초음파(II) 청구 ⇒ "I/R4521/자궁내막암의심으로 초음파 유도하 조직검사함."
	\end{description} 
\end{commentbox}
\prezi{\clearpage}
\subsection{산부인과학회 세부급여기준 Q\&A}
1. 어떤 환자들에게 부인과 초음파를 보험으로 해드려야 하나요?
\begin{quotebox}
부인암 환자 또는 부인암 의심자가 초음파 보험대상입니다.
\end{quotebox}
2. 어떤 환자가 부인암 의심자인가요?
\begin{quotebox}
기본적으로 의사의 판단에 따릅니다. 따라서 절대적인 기준은 없지만 아래의 경우에 부인암 의심자로 볼 수 있습니다.\par
	\begin{enumerate}[1)]\tightlist
	\item 타 병원에서 골반종괴가 발견이 되어 치료를 위해 전원 된 경우
	\item 골반종괴, 외음부종괴, 질 출혈, 복부 팽만, 복수, 심한 체중 감소, 서혜부 림프절 종대 등 부인암이 의심되는 증상이 있는 경우
	\end{enumerate}
\end{quotebox}
3. 같은 환자가 초음파를 여러 번 하면 계속 보험 적용이 되나요?
\begin{quotebox}
부인암 중증환자로 등록된 분들의 경우 회수 제한이 없습니다. 그러나 임상 증상이 있어 부인암 의심자로 보험적용을 받은 경우엔 같은 이유로는 1회만 인정됩니다.
\end{quotebox}