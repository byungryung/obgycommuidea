\section{행위/치료대}
\subsection{급여 인정 범위가 있는 행위/치료대}
\tabulinesep =_2mm^2mm
\begin{tabu} to \linewidth {|X[4,l]|X[4,l]|X[4,l]|X[4,l]|} \tabucline[.5pt]{-}
\rowcolor{Gray!25}  보험/비보험 진료& 보험인정 범위  & 급여로 산정 & 비급여로 산정 \\ \tabucline[.5pt]{-}
\rowcolor{Yellow!5} 보험 진료 & 인정범위내 & 합법 & 불법(임의비급여)  \\ \tabucline[.5pt]{-}
\rowcolor{Yellow!5} 보험 진료 & 인정범위외 & 삭감 & 불법(임의비급여) \\ \tabucline[.5pt]{-}
\rowcolor{Yellow!5} 비보험 진료 & 인정범위내 & X  & 합법 \\ \tabucline[.5pt]{-}
\rowcolor{Yellow!5} 비보험 진료 & 인정범위외 & X  & 합법 \\ \tabucline[.5pt]{-}
\end{tabu}
\par
\medskip
\begin{enumerate}[①]\tightlist
\item 급여 진료시에는 보험 인정 범위 내에서 급여로 청구한 경우 합법이다.
\item 급여 진료시에는 보험 인정 범위 외에서 급여로 청구한 경우 삭감된다. 예를 들면 질강처치의 경우는 고시에 없는 상병에서나 한달에 두번 하게 되면 삭감된다.
\item 급여 진료시에는 보험인정범위가 있는 행위/치료대등을 임의로 비급여로 받으면 불법이다. 예를 들면 루프부작용으로  N989 상세불명의 질출혈로 루프제거R4275 시행시에 관습대로 루프제거를 비보험으로 받게되면 임의비급여로 불법이다.
\item 비급여 진료시에는 급여 행위/약제/재료는 모두 비급여 진료 행위로 간주하기 때문에 급여 항목은 비급여로 전환해서 받을 수 있게 되어 있습니다. 그래서 타원 IVF 중인 환자가 그 병원에서 받은 약을 가지고 본원에 온 경우에는 주사료를 비급여로 전환해서 받을 수 있습니다. 이렇게 생각한다면 보험 인정 범위 내의 행위/치료대의 경우 비급여 진료로 해서 진찰료를 받지 않고 접수해서 비급여로 산정하는 것은 괜찮지 않습니까?(필요하다면 급여 / 비급여로 분리 접수를 해서 말입니다.) 일단 진료가 급여 항목인경우인지, 비급여 항목인지 분리해야합니다 IVF-ET, IUI등의 시술과 관계없는 급여 진료인 경우 주사료와 진찰료가 급여 산정가능하지만 IVF-ET, IUI 등의 시슬과 관계있는 경우 진찰료와 주사수기료 모두 비급여입니다
\end{enumerate}

\subsection{비급여 인정 범위가 있는 행위/치료대(인정 비급여)}
\tabulinesep =_2mm^2mm
\begin{tabu} to \linewidth {|X[4,l]|X[4,l]|} \tabucline[.5pt]{-}
\rowcolor{Gray!25}  보험인정 범위  & 비급여로 산정 \\ \tabucline[.5pt]{-}
\rowcolor{Yellow!5} 인정범위내 & 합법   \\ \tabucline[.5pt]{-}
\rowcolor{Yellow!5} 인정범위외 & 합법  \\ \tabucline[.5pt]{-}
\end{tabu}

\par
\medskip
\begin{enumerate}[①]\tightlist
\item 급여 진료시 보험인정법위내의 경우엔 비급여 산정이 가능하다.
\item 급여 진료시 보험 인정범위외의 경우중 비급여 산정이 가능하다는 고시가 있는경우는 합볍적으로 비급여 산정이 가능합니다(별도 산정이 불가한 경우 비급여 산정도 임의 비급여) 예) 요실금 수술의 경우 보험기준외의 경우는 비급여  산정 가능
\end{enumerate}

\begin{commentbox}{급여진료시 보험인정외 비급여로 청구가능 한 경우}
인조테이프를 이용한 요실금수술은 요류역학검사(방광내압측정 및 요누출압검사)로 복압성 요실금 또는 복압성 요실금이 주된 혼합성 요실금이 확인되고 요누출압이 120cmH2O 미만인 경우에 .인정하며, 동 인정기준 이외에는 비용효과성이 떨어지고 치료보다 예방적 목적이 크다고 간주하여 시술료 및 치료재료 비용 전액은 환자가 부담토록 함(비급여).\par

인조테이프를 이용한 요실금수술의 구체적 적용기준에 대하여 붙임과 같이 통보하오니 업무에 참고하시기 바랍니다.\par
 
붙임)
\begin{enumerate}[1.]\tightlist
\item 요실금수술 인정기준에 해당되지 않는 경우 수술료 및 치료재료 외의 진료비용에 대하여 ‘인조테이프를 이용한 요실금수술 인정기준’ 에 해당되지 않는 경우에는 수술료 및 치료재료 비용 뿐 아니라 입원료, 마취료 등 제반 진료비용 전액은 환자가 부담토록 함. (비급여).
\item 요류역학검사(방광내압측정 및 요누출압검사)를 실시하지 않고 요실금 수술을 시행시 급여여부 ; 현행 인정기준에 해당되지 않으므로 비급여 대상임.
\end{enumerate}
\end{commentbox}

\subsection{비급여 인정 범위가 없는 행위/치료대(비인정 비급여)}
예를 들면 일회용질경등.. 환수대상임.\par
- 비급여 진료의 경우 진찰료 산정을 못하므로 이를 보전하기 위한 비용들
\begin{enumerate}[①]\tightlist
\item 피임 목적으로 방문 시의 ‘비급여 피임 처방료’ 
\item IUI CYCLE 중 내원하여 클로미펜이나 FSH 처방 받을 시의 ‘불임 환자 상담료’
\item 성의학 진료 시의 ‘성의학 상담료’
\end{enumerate}

피임은 비급여 항목이므로 비급여 진찰료, 상담료 산정 가능합니다 \par
불임은 VF-ET, IUI 등의 시슬과 관계있는 경우 진찰료와 주사수기료 상담료 모두 비급여입니다.\par
“성상담“은 합법 비급여 항목입니다. 

