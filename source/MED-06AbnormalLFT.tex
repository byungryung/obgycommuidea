\section{간기능 이상}
\subsection{SGOT/SGPT 가 높을때}
K7290 혼수를 동반하지 않은 상세불명의 간부전 \par
K769 상세불명의 간질환..\par
B188기타 만성 바이러스 간염\par
B19.9간성 혼수가 없는 상세불명의 바이러스간염(Unspecified viral hepatitis without hepatic coma)\par 
\medskip

\tabulinesep =_2mm^2mm
\begin{tabu} to\linewidth {|X[1,l]|X[7,l]|} \tabucline[.5pt]{-}
\rowcolor{ForestGreen!40} 코드 & 검사명  \\ \tabucline[.5pt]{-}
\rowcolor{Yellow!40} D7001003 & B형간염표면항원(일반) 또는 C4802 B형간염표면항원(정밀)  \\ \tabucline[.5pt]{-}
\rowcolor{Yellow!40} D7002003 & B형간염표면항체(일반) 또는 C4812 B형간염표면항체(정밀)  \\ \tabucline[.5pt]{-}
\rowcolor{Yellow!40} D7005003 & C형간염항체(일반) \\ \tabucline[.5pt]{-}
\rowcolor{Yellow!40} D7011013 & A형간염항체(IgG) \\ \tabucline[.5pt]{-}
\rowcolor{Yellow!40} D2510013 & 나-263 CPK \\ \tabucline[.5pt]{-}
\rowcolor{Yellow!40} D242002 & 나-421 Alpha-fetoprotein(AFP) \\ \tabucline[.5pt]{-}
\rowcolor{Yellow!40} D1870003 & 알카리포스파타제 \\ \tabucline[.5pt]{-}
\rowcolor{Yellow!40} D1830003 & 총빌리루빈 \\ \tabucline[.5pt]{-}
\rowcolor{Yellow!40} D1890 & r-GTP \\ \tabucline[.5pt]{-}
\rowcolor{Yellow!40} D1860 & SGOT \\ \tabucline[.5pt]{-}
\rowcolor{Yellow!40} D1850 & SGPT \\ \tabucline[.5pt]{-}
\rowcolor{Yellow!40} D0002050 & 혈색소(광전비색법) \\ \tabucline[.5pt]{-}
\rowcolor{Yellow!40} D0002040 & 헤마토크리트 \\ \tabucline[.5pt]{-}
\rowcolor{Yellow!40} D0002030 & 적혈구수 \\ \tabucline[.5pt]{-}
\rowcolor{Yellow!40} D0002010 & 백혈구수 \\ \tabucline[.5pt]{-} 
\rowcolor{Yellow!40} D0002070 & 혈소판수 \\ \tabucline[.5pt]{-}
\rowcolor{Yellow!40} D0013 & 백혈구백분율 \\ \tabucline[.5pt]{-}
\end{tabu}

\subsection{HBs Ag(+)일때}
K7290 혼수를 동반하지 않은 상세불명의 간부전 \par
K 769 상세불명의 간질환.. \par
B188기타 만성 바이러스 간염 \par

\tabulinesep =_2mm^2mm
\begin{tabu} to\linewidth {|X[1,l]|X[7,l]|} \tabucline[.5pt]{-}
\rowcolor{ForestGreen!40} 코드 & 검사명  \\ \tabucline[.5pt]{-}
\rowcolor{Yellow!40} C4821 & B형간염e항원(일반) 또는 C4822 B형간염e항원(정밀)  \\ \tabucline[.5pt]{-}
\rowcolor{Yellow!40} C4831 & B형간염e항체(일반) 또는 C4832 B형간염e항체(정밀) \\ \tabucline[.5pt]{-}
\rowcolor{Yellow!40} C4871 & C형간염항체(일반) \\ \tabucline[.5pt]{-}
\rowcolor{Yellow!40} C4212 & 나-421 Alpha-fetoprotein(AFP) \\ \tabucline[.5pt]{-}
\rowcolor{Yellow!40} C4821 & B형간염e항원(일반) \\ \tabucline[.5pt]{-}
\rowcolor{Yellow!40} C4831 & B형간염e항체(일반) \\ \tabucline[.5pt]{-}
\rowcolor{Yellow!40} C4854006 & B형간염바이러스DNA정량검사-실시간중합효 \\ \tabucline[.5pt]{-}
\end{tabu}

\begin{commentbox}{나485 B형 간염 DNA 정량검사의 인정기준}\index{B형 간염 DNA 정량검사의 인정기준}
나485 B형 간염 DNA 정량(HBV-DNA)검사 인정기준은 다음과 같이 함 \par
- 다 음 -
\begin{enumerate}[가.]\tightlist
\item HBsAg 양성인 만성 간질환 환자
\item 만성 B형간염 산모
\item 만성B형 간염환자, 간경변환자, 간암환자 중 항바이러스치료를 받고 있는 환자의 치료반응을 평가하기 위해 실시하는 경우
\item B형 간염 바이러스 보유자의 항암화학요법 또는 면역억제제 치료 시작시와 치료 후 경과 관찰 위해 실시하는 경우
\item (2010.11.1 시행)
\end{enumerate}
\end{commentbox}

\subsection{나-485 B형간염 DNA 정량검사 HBV-DNA Quantification}
C4851 가. DNA Probe법 333.66 \myexplfn{333.66 } 원 \par
C7485 주:핵의학적 방법으로 검사한 경우에는 102.74점을 산정한다.\par
C4852 나. 교잡포획검사 Hybridization [CMHA] 491.90 \myexplfn{491.9 } 원 \par
C4853 다. bDNA 유전자신호증폭측정법 496.26 \myexplfn{496.26 } 원\par
CX425 라. 중합효소연쇄반응 교잡반응법 PCR-Hybridization 893.12 \myexplfn{893.12 } 원\par
D704403C 마. 실시간 중합효소연쇄반응 Real-time PCR 813.53 \myexplfn{813.53 } 원\par
C4804* (2) 전기화학발광 면역측정법Electrochemiluminescence Immunoassay 327.82 \myexplfn{327.82 } 원 \par

\begin{commentbox}{B형간염바이러스 표면항원 정량검사 인정기준}
『B형간염바이러스 표면항원 정량검사』는 다음과 같은 경우에 요양급여를 인정함. \par
- 다 음 -
\begin{enumerate}[가.]\tightlist
\item 적용대상 : pegylated interferon-α를 투여하는 만성 B형 간염환자에서 치료반응 평가를 위해 시행하는 경우
\item 인정횟수
	\begin{enumerate}[1)]\tightlist
	\item 치료 전: 1회
	\item 치료 12주째와 24주째: 각 1회
	\item 치료 종결 시: 1회
	\item (2014.8.1 시행)
  	\end{enumerate}
\end{enumerate}  	
☞ 신의료기술 요양급여결정신청에 의거 비급여에서 급여전환됨에 따라 급여기준 신설
\end{commentbox}

나-480 B형간염 표면항원 Hepatitis B surface Antigen의 종류\par

\tabulinesep =_2mm^2mm
\begin{tabu} to\linewidth {|X[1,l]|X[1,l]|X[6,l]|X[1,l]|X[1,l]|} \tabucline[.5pt]{-}
\rowcolor{ForestGreen!40}  & 코드 &	\centering 분 류 & 점수 & 금액 \\ \tabucline[.5pt]{-}
\rowcolor{Yellow!40}  누700가 & & B형간염 표면항원 Hepatitis B surface Antigen의 종류 & & \\ \tabucline[.5pt]{-}
\rowcolor{Yellow!40} &  & 가. 정성검사 Qualitative &  &  \\ \tabucline[.5pt]{-}
\rowcolor{Yellow!40} &  D7001 & (1) 일반 General & 35.73 & \myexplfn{35.73 } \\ \tabucline[.5pt]{-}
\rowcolor{Yellow!40} &  D7015 & (2) 정밀 High Quality & 138.37 & \myexplfn{138.37 } \\ \tabucline[.5pt]{-}
\rowcolor{Yellow!40} &  D7016 & 주:핵의학적 방법으로 검사한 경우에는 & 136.62 &  \\ \tabucline[.5pt]{-}
\rowcolor{Yellow!40} & & 나. 정량검사 Quantitation & &  \\ \tabucline[.5pt]{-}
\rowcolor{Yellow!40} & C4803* & (1) 화학발광 미세입자 면역측정법Chemiluminescent Microparticle Immunoassay & 327.82 & \myexplfn{327.82 } \\ \tabucline[.5pt]{-}
\end{tabu}

  
\begin{commentbox}{C형간염항체검사의 인정기준}
C형간염항체(HCV Ab) 가.일반 또는 나.정밀 검사는 다음과 같은 경우에 요양급여를 인정함 \par
- 다 음 -
\begin{enumerate}[가.]\tightlist
\item \uline{간기능검사상 이상소견이 있는 경우}
\item \uline{급성 및 만성 간질환 환자에서 C형간염이 의심되거나 또는 C형간염의 배제가 필요한 경우}
\item 혈액종양 환자와 혈액투석을 받는 만성 신부전증 환자 등 잦은 수혈로 인해 C형 간염 감염의 위험이 있다고 판단되는 경우
\item 혈액, 골수, 조직, 장기 등의 공여자
\item \uline{C형간염 고위험군에서 감염원에 노출되었거나 노출될 위험이 높은 경우}
\item \uline{수술(관혈적 시술 포함)이 필요하거나 예측되는 경우}
\item 상기 가.-바. 이외 임상적으로 필요하여 실시하는 경우 사례별로 인정함.
\item (2015.7.1. 시행)
\end{enumerate}
\end{commentbox}

나-487 C형간염항체 HCV Antibody \par
D7005003 가. 일반 General 50.50 \myexplfn{50.50} 원 \par
D7026003 나. 정밀 High Quality 207.31 \myexplfn{207.31} 원 \par
D7027007 주:핵의학적 방법으로 검사한 경우에는 208.17점을 산정한다. \par
D703100C 다. Immunoblot법 508.18 \myexplfn{508.18} 원 \par
