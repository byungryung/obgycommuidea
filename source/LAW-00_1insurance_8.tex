\subsection{\newindex{제8조(요양급여의 범위 등)}}
\begin{enumerate}[①]\tightlist
\item 법 제41조제2항에 따른 \large{요양급여의 범위(이하 ``요양급여대상"이라 한다)는 다음 각 호와 같다.} <개정 2006\cntrdot{}12\cntrdot{}29, 2012ㆍ8ㆍ31>
	\begin{enumerate}[1.]\tightlist
	\item \uline{법 제41조제1항 각 호의 요양급여(약제를 제외한다) : 제9조에 따른 비급여대상을 제외한 일체의 것}
	\item 법 제41조제1항제2호의 요양급여(약제에 한한다) : 제11조의2, 제12조 및 제13조에 따라 요양급여대상으로 결정 또는 조정되어 고시된 것
	\end{enumerate}
\item 보건복지부장관은 제1항의 규정에 의한 요양급여대상을 급여목록표로 정하여 고시하되, 법 제41조제1항의 각호에 규정된 요양급여행위(이하 ``행위"라 한다), 약제 및 치료재료(법 제41조제1항제2호의 규정에 의하여 지급되는 약제 및 치료재료를 말한다. 이하 같다)로 구분하여 고시한다. 다만, 보건복지부장관이 정하여 고시하는 요양기관의 진료에 대하여는 행위\cntrdot{}약제 및 치료재료를 묶어 1회 방문에 따른 행위로 정하여 고시할 수 있다. <개정 2001\cntrdot{}12\cntrdot{}31, 2008\cntrdot{}3\cntrdot{}3, 2010\cntrdot{}3\cntrdot{}19, 2012ㆍ8ㆍ31>
\item 보건복지부장관은 제2항에도 불구하고 영 제21조제3항제2호에 따라 보건복지가족부장관이 정하여 고시하는 질병군에 대한 입원진료의 경우에는 해당 질병군별로 별표 2 제6호에 따른 비급여대상, 규칙 별표 6 제1호사목에 따른 이송처치료 및 같은 호 아목1)에 따른 요양급여비용의 본인부담 항목을 제외한 모든 행위\cntrdot{}약제 및 치료재료를 묶어 하나의 포괄적인 행위로 정하여 고시할 수 있다. 이 경우 하나의 포괄적인 행위에서 제외되는 항목은 보건복지부장관이 정하여 고시할 수 있다. <개정 2001\cntrdot{}12\cntrdot{}31,2005\cntrdot{}10\cntrdot{}11, 2008\cntrdot{}3\cntrdot{}3 , 2010\cntrdot{}3\cntrdot{}19, 2012ㆍ8ㆍ31, 2014ㆍ9ㆍ1, 2015ㆍ5ㆍ29, 2015ㆍ6ㆍ30>
\item 보건복지부장관은 제2항에도 불구하고 영 제21조제3항제1호에 따른 요양병원의 입원진료나 같은 항 제3호에 따른 완화의료의 입원진료의 경우에는 제2항의 행위\cntrdot{}약제 및 치료재료를 묶어 1일당 행위로 정하여 고시할 수 있다. 이 경우 1일당 행위에서 제외되는 항목은 보건복지부장관이 정하여 고시할 수 있다. <신설 2007\cntrdot{}12\cntrdot{}28 보건복지부령428, 2008\cntrdot{}3\cntrdot{}3 , 2010\cntrdot{}3\cntrdot{}19, 2012ㆍ8ㆍ31, 2015ㆍ6ㆍ30>
\item 보건복지부장관은 제2항부터 제4항까지의 규정에 따라 요양급여대상을 고시함에 있어 행위 또는 하나의 포괄적인 행위의 경우에는 영 제21조제2항에 따른 요양급여의 상대가치점수를 함께 정하여 고시하여야 한다. <2007\cntrdot{}12\cntrdot{}28 보건복지부령428, 2008\cntrdot{}3\cntrdot{}3, 2010\cntrdot{}3\cntrdot{}19, 2012ㆍ8ㆍ31>
\end{enumerate}  
\subsection{\newindex{제9조(비급여대상)}}
\begin{enumerate}[①]\tightlist
\item 법 제41조제3항에 따라 요양급여의대상에서 제외되는 사항(이하 \uline{``비급여대상"이라 한다)은 별표 2와 같다.} <개정 2012ㆍ8ㆍ31>
\end{enumerate}