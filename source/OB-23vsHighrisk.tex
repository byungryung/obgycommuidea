\section{HIGH RISK 비교}
\tabulinesep =_2mm^2mm
\begin {longtabu} to\linewidth {|X[6,l]|X[1,c]|X[1,c]|X[1,c]|} \tabucline[.5pt]{-}
\rowcolor{ForestGreen!40}  상 병 명 & 질식분만 & 제왕절개 & 초음파 \\ \tabucline[.5pt]{-}
\rowcolor{Yellow!40} Z355, Z358 출산 당시 나이가 만 35세 이상인 산모 & O & O\footnote{Z355 만35세 이상인 초산모, Z358 경산으로 만40세이상인 경우와 만 35세 이상인 경산으로 전 출산과 만 5년이상 interval이 있는 경우} & X \\ \tabucline[.5pt]{-}
\rowcolor{Yellow!40} E669, O260 임신 제1 삼분기 당시 BMI가 27.5 kg/㎡ 이상인 산모 & O & X & X  \\ \tabucline[.5pt]{-}
\rowcolor{Yellow!40} O3419 임신 중 5㎝ 이상의 자궁근종 또는 [O3409] 자궁기형을 가진 산모 & O & X & O \\ \tabucline[.5pt]{-}
\rowcolor{Yellow!40} O3439 자궁경관부전에 대한 산모관리, 상세불명의 임신기간 & X & X & O   \\ \tabucline[.5pt]{-}
\rowcolor{Yellow!40} O6011 임신 34주 미만의 조산 & O & ?O\footnote{O47 가진통} & X \\ \tabucline[.5pt]{-}
\rowcolor{Yellow!40} O12 전자간증, 자간증 또는 가중합병전자간증 & O & O & O \\ \tabucline[.5pt]{-}
\rowcolor{Yellow!40} O440 전치태반 또는 [O459] 태반 조기 박리 & O & O & O \\ \tabucline[.5pt]{-}
\rowcolor{Yellow!40} O40 양수과다증 또는 [O410] 양수과소증 & O & X & O \\ \tabucline[.5pt]{-}
\rowcolor{Yellow!40} 심혈관계 질환, 신장 질환, 당뇨병 [O249], 혈액응고장애, 백혈병, 매독 또는 HIV 양성 중 어느 하나 이상에 속하면서 분만에 직접적인 위험을 줄 수 있는 질환을, 임신 전 또는 임신 기간 중 진단 받고 지속 치료중인 산모 & O & O & O  \\ \tabucline[.5pt]{-}
\rowcolor{Yellow!40} O359 출산과정에 영향을 미치거나, 분만 중 태아 또는 신생아의 생존 능력에 영향을 미치는 태아 기형 & O & X & O \\ \tabucline[.5pt]{-}
\rowcolor{Yellow!40} O366 출생당시 체중이 4kg 이상 또는 [O365] 2.5Kg미만의 신생아 & O & X & O(IUGR)\newline X(LGA)
 \\ \tabucline[.5pt]{-}
\rowcolor{Yellow!40} O430 쌍태간 수혈 증후군 & O & X & X \\ \tabucline[.5pt]{-}
\end{longtabu}
