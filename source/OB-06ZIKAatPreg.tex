\section{임산부 지카바이러스 검사 Zika RT PCR}
\myde{}{%
\begin{itemize}\tightlist
\item[\dsjuridical] A929 상세불명의 모기매개 바이러스 열
\item[\dsjuridical] O985 임신, 출산 및 산후기에 합병된 기타 바이러스질환
\item[\dsjuridical] Z115 기타 바이러스질환에 대한 특수선별검사
\item[\dschemical]  \sout{나596-6 C606600C} 누658다 D658302C \myexplfn{651.81}  urine(3ml이상) serum(2ml이상)
\end{itemize}
}
{
\emph{지카바이러스 검사 보험}\par
위험노출 임신부의 경우, 임상증상 발생여부와 관계없이 본인 희망 및 의사가 필요하다고 판단할 경우 ※ 위험노출 임신부란?
\begin{enumerate}[①]\tightlist
\item 지카바이러스 감염증 발생국가 방문 또는 거주
\item 감염남성 또는 발생국가 방문남성과 성접촉
\item 산전 진찰을 통해 태아의 소두증 또는 뇌 석회화증 의심
\item 1 or 2가 있으면서 증상이 있으면 보건소 신고된 증상이 없으면 Zika RT PCR 검사의뢰(급여적용)
\end{enumerate}
}
\prezi{\clearpage}

\subsection{임신부의 지카바이러스 감염증에 대한 안내}
\begin{enumerate}[1.]\tightlist
\item 신고를 위한 진단기준
	\begin{itemize}[→]\tightlist
	\item 환자 : 지카바이러스 감염증에 부합되는 임상증상을 나타내면서 진단을 위한 검사기준에 따라 감염병병원체 감염이 확인된 사람
	\item 의사환자
		\begin{itemize}[-]\tightlist
		\item 의심환자 : 임상증상 및 역학적 위험요인을 감안하여 지카바이러스  감염증이 의심되나 검사기준에 부합하는 검사결과가 없는 사람
		\item 추정환자 : 임상증상 및 역학적 위험요인을 감안하여 지카바이러스 감염증이 의심되어 시행한 진단 검사에서 혈청 IgM 항체가 검출된 사람
		\item 역학적 위험요인 : (1) 증상 시작 전 2주이내  지카바이러스 감염증 발생국가 여행력 (최신 발생국 현황은 질병관리본부 홈페이지 참조)(2) 지카바이러스 감염 남성과 성접촉 (3) 지카바이러스 감염증 발생지역에 최근 2개월 이내 방문 이력이 있는 남성과 성접촉
		\end{itemize}
	\end{itemize}	
\item 지카바이러스 감염증 임상증상
	\begin{itemize}[→]\tightlist
	\item 발진과 함께 다음 증상 중 하나 이상이 동반된 경우
	\item 발열, 관절통/관절염, 근육통, 비화농성 결막염/결막충혈
	\end{itemize}
\item 지카바이러스 유행지역을 여행하고 온 임신부가 2주 내 증상이 있는 경우, 어떤 검사가 시행되나요?
	\begin{itemize}[→]\tightlist
	\item 임신부 혈청과 소변으로 바이러스 검사(RT-PCR) 
	\item 바이러스검사 양성인 경우: 태아초음파(소두증 또는 뇌내 석회화 여부확인), 양수천자
	\item 바이러스검사 음성인 경우: 태아초음파 (이상소견 발견되면 양수천자)
	\end{itemize}
\item 양수천자의 태아감염 확진율은 얼마나 되나요?
	\begin{itemize}[→]\tightlist
	\item 현재 태아감염에 대한 양수천자의 민감도와 특이도에 대해 정확히 알려진 바는 없습니다. 또한 양수에서 RT-PCR 결과가 양성으로 나온다고 하여 이것이 태아기형과 어느 정도의 상관관계가 있는가에 대해서 잘 알려져 있지 않습니다. 
	\end{itemize}
\item 임신 몇 주부터 양수천자가 가능한가요?
	\begin{itemize}[→]\tightlist
	\item 양수천자로 인한 합병증을 최소화하고 위음성(false negative)결과를 방지하기 위해 적어도 임신 21주 이상에서 시행하는 것이 권고됩니다. 감염 가능성이 있는 임신부들에게는 검사 전에 양수천자의 이익과 위험성에 대한 충분한 설명이 제공되어야 합니다. 
	\end{itemize}
\item 임신 14주 이하의 양수천자가 제안 또는 고려되는 임신부에서 융모막검사(CVS)를 통한 검사는 시행될 수 없나요?
	\begin{itemize}[→]\tightlist
	\item 현재 질병관리본부 가이드라인에서 지카바이러스 감염을 진단하기 위한 검사에 융모막검사는 포함되어 있지 않습니다.
	\end{itemize}
\item 출산 후에 신생아에서도 지카바이러스 검사가 가능한가요?
	\begin{itemize}[→]\tightlist
	\item 임신 중 모체 또는 태아 내 지카바이러스 감염의 증거가 있는 경우, 신생아에서 다음과 같은 검사를 시행할 수 있습니다.
	\item 태반과 탯줄의 병리학적 검사
	\item 동결 태반과 탯줄(frozen placental tissue and cord tissue) 바이러스(RT-PCR) 검사
	\end{itemize}
\item 사산된 태아에서는 어떤 검사가 가능한가요?
	\begin{itemize}[→]\tightlist
	\item 지카바이러스 발생국가 여행 후 2주 내 증상이 있었거나 태아소두증이 확인된 경우, 다음과 같은 검사들이 시행될 수 있습니다.
	\item 사산아 조직(탯줄과 태반 포함) 검사 : 바이러스(RT-PCR) 검사 및 면역학적 염색  
	\end{itemize}
\item 임신 기간 중 지카바이러스 감염시 생길 수 있는 선천성 기형은 무엇인가요?
	\begin{itemize}[→]\tightlist
	\item 임신 주수에 관계없이 감염이 되면 태아의 중추신경계 이상을 유발할 수 있고, 감염 확진을 받은 임신부에서 자궁내 성장지연, 소두증, 뇌 위축, 뇌실비대, 두 개내 석회화, 시각 상 결함, 두피 주름, 관절 구축, 간질, 청각\cntrdot{}시각장애, 정신\cntrdot{}운동발달(psycomotor development)과 같은 중추신경계(CNS) 이상 및 뼈\cntrdot{}관절 이상 장애등의 이상 소견이 현재까지 알려졌습니다.
	\end{itemize}
\item 소두증과 관련되어 발생할 수 있는 문제들은 어떤 것들이 있나요?
	\begin{itemize}[→]\tightlist
	\item 소두증이 아기에게 미치는 영향은 소두증의 정도와 비례하며, 다음과 같은 증상, 장애가 경증부터 생존을 위협하는 중증까지 다양하게 나타납니다.
    \item 경련, 발달장애, 지체장애, 보행장애, 수유장애, 연하곤란, 청각장애, 시각장애 등
	\end{itemize}	
\end{enumerate}