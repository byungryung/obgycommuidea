\section{의사의 진료거부권}
의료법 제15조 제1항은 “의료인은 진료나 조산 요청을 받으면 정당한 사유 없이 거부하지 못한다.”고 규정하고 있습니다.\par
□ 다만, 정당한 사유가 있으면 그러하지 아니하며, 보건복지부는 정당한 사유로 다음과 같은 사항을 예시하고 있습니다.

\begin{itemize}\tightlist
\item 일단 진료한 환자의 상태를 보아 의사가 의학적인 판단에 따라 퇴원 또는 타 의료기관 진료를 권유하는 행위를 진료거부로 보기는 어려움 (2000. 6. 2. 의정 65507-704)
\item 의사가 부재중이거나 신병으로 인하여 진료를 행할 수 없는 상황인 경우
\item 병상, 의료인력, 의약품, 치료재료 등 시설 및 인력 등이 부족하여 새로운 환자를 받아들일 수 없는 경우
\item 의원 또는 외래진료실에서 예약환자 진료 일정 때문에 당일 방문 환자에게 타 의료기관 이용을 권유할 수밖에 없는 경우
\item 의사가 타 전문과목 영역 또는 고난이도의 진료를 수행할 전문지식 또는 경험이 부족한 경우
\item 타 의료인이 환자에게 기 시행한 치료(투약, 시술, 수술 등) 사항을 명확히 알 수 없는 등 의학적 특수성 등으로 인하여 새로운 치료가 어려운 경우
\item 환자가 의료인의 치료방침에 따를 수 없음을 천명하여 특정 치료의 수행이 불가하거나, 환자가 의료인으로서의 양심과 전문지식에 반하는 치료방법을 의료인에게 요구하는 경우
\item 환자 또는 보호자 등이 해당 의료인에 대하여 모욕죄, 명예훼손죄, 폭행죄, 업무방해죄에 해당될 수 있는 상황을 형성하여 의료인이 정상적인 의료행위를 행할 수 없도록 하는 경우
\item 더 이상의 입원치료가 불필요함 또는 대학병원급 의료기관에서의 입원 치료는 필요치 아니함을 의학적으로 명백히 판단할 수 있는 상황에서, 환자에게 가정요양 또는 요양병원ㆍ1차의료기관ㆍ요양시설 등의 이용을 충분한 설명과 함께 권유하고 퇴원을 지시하는 경우
\end{itemize}
□ 만약 정당한 사유 없이 진료 또는 조산(助産)의 요청을 거부하거나 응급환자에 대한 응급조치를 하지 아니한 경우 1년 이하의 징역이나 500만원 이하의 벌금(법 제89조)에 처해질 수 있습니다.
□ 의료관계행정처분규칙 별표 행정처분기준(제4조 관련)
\begin{description}\tightlist
\item[위반사항] 3) 법 제15조를 위반하여 정당한 사유 없이 진료 또는 조산(助産)의 요청을 거부하거나 응급환자에 대한 응급조치
를 하지 아니한 경우
\item[근거법령] 제66조제1항 제10호
\item[행정처분기준] 자격정지 1개월
\end{description}


