\section{실사가 나왔을떄 전략}
\Large{그들의 특성}\normalsize
\par
\medskip
\leftrod{그들은 공무원이다.}
\begin{itemize}\tightlist
\item 나온 이상 반드시 기본 이상은 해야 한다.(기본이 안 될 것 같으면 억지 조항을 들이대서라도 일정 성과는 가지고 가야 함)
\item 사명감을 가지고 모든 문제를 낱낱이 밝힐 의지가 있는 것은 아니다.
\item 하지만 조사 과정에서 감정이 너무 틀어져 버리면 없던 열의가 생길 수 있다.
\end{itemize}

\leftrod{3사(복지부 / 심평원 / 공단) 연합 팀이고 급조된 팀이다.}
\begin{itemize}\tightlist
\item 초기부터 손발이 잘 맞는 것이 아니고 서서히 손발을 맞춰가면서 일을 한다.
\item 서로 감시가 되므로 눈에 보이는 것을 넘어갈 수는 없다. 
\end{itemize}

\Large{그들이 찾으려고 하는 내용들}\normalsize
\par
\medskip
\leftrod{1) 조사 }
\begin{itemize}\tightlist
\item 원칙적으로 현지 조사는 전수 조사이다.(불똥이 어디로든 튈 수 있으며 일단 번지기 시작하면 끝까지 갈 수도 있다.)
\item 우선 현지 조사가 나오게 된 부분에 대한 조사가 먼저 이루어진다.
\item 그들의 입장에서는 현지 조사 나온 부분에 대해서는 반드시 성과를 내야 한다.
\item 그 이후 다른 부분에 대해서 조사한다.(시간이 없으면 더 조사 안 한다.)
\end{itemize}

\leftrod{2) 진료 관련}

\begin{enumerate}[①]\tightlist
\item 기본 항목(가장 먼저 확인하는 내용)
	\begin{itemize}\tightlist
	\item 현지 조사를 나오게 된 직접적인 원인 
	\item 실사 나오기 전, 그 병원의 청구 내역을 분석해서 타기관에 비해서 특이해 보이는 부분들 \newline
-> 이 부분들은 그들이 생각하는 ‘기본’이다. 이 부분에서는 반드시 성과를 내야 하기 때문에 그들도 많은 시간을 할애한다. 나중에 밝혀지더라도 이 부분에서 최대한 시간을 끌어서 더 이상 진도를 못 나가게 막아야 한다.
	\end{itemize}
\item 추가 항목
	\begin{itemize}\tightlist
	\item DB를 이용해 뽑아낸 자료들
	\item 다빈도 청구 항목 
	\item 비급여 리스트 중 임의비급여 항목
	\item 비급여 리스트 중 다빈도 처방 항목
	\item 비급여 리스트 중 문제가 있을 만한 항목(유착방지제/영양제 등) \newline
-> DB를 분석하면 이 항목들과 그 수진자 리스트를 뽑을 수 있다. 이에 대해 서류 / 수진자 통화 / 직원 면담 등을 통해 사실 확인을 한다.
	\end{itemize}
\end{enumerate} 
\leftrod{3) 시설, 인력, 장비 등이 신고 된 내역과 실제가 일치하는 지 확인}

\leftrod{4) 함정 항목}
\begin{itemize}\tightlist
\item 2), 3)을 통해서 만족한 만한 성과가 나오지 않을 경우
\item 흔히 착오 청구를 할 만한 항목
\item 현실과 동떨어진 법 때문에 문제가 될 수 밖에 없는 항목 \newline
 - > 이들을 통해 어떻게든 성과를 맞추려 한다. 
\end{itemize}

\Large{실제적인 대응전략}\normalsize
\par
\medskip 
\leftrod{1) 눈에 보이면 잡는다.(-> 눈에 보이지 않으면 넘어간다.)}
\begin{itemize}\tightlist
\item 실사 나오면 일상으로 하던 병원 내 모든 업무를 멈춘다.
\item 모든 답변은 원장을 통해서만 한다.\newline
 원장 : ‘모든 내용은 내가 답변할테니 직원들에게는 묻지 말라.’ \newline
 직원 :  ‘저는 직원이라 드릴 말씀이 없으니 모든 내용은 원장님께 여쭈어보시라’
\item 간단한 자료라도 반드시 원장이 직접 확인 후 제출한다.
\item 자료 검토 후 필요하면 수정해서 준다.(그들은 수사관이 아니다. 자료의 진위를 밝힐 정도의 능력도, 의지도 있지 않다. 그냥 준 자료를 가지고 검토한다.)
\end{itemize}

\leftrod{2) 시간을 끌어야 한다.}
\begin{enumerate}[①]\tightlist
\item 자료 제출은 최대한 천천히 한다. 
	\begin{itemize}\tightlist
	\item 자료를 안 주면 안 된다.(업무 정지 1년)
	\item 그러나 천천히 주는 것은 문제되지 않는다.  ‘요청 자료 찾고 있고, 정리하고 있다.’
	\end{itemize}
\item 시간이 되면 가야 한다.(그들도 다음 스케쥴 있음)
	\begin{itemize}\tightlist
	\item ‘기본 항목’에서 싸우다가 가게 해야 한다.
	\item 자료를 빨리 줘서 ‘기본 항목’이 끝나고 나면 다음 항목으로 진도가 넘어간다.
	\item ‘기본 항목’에 대해서 잘못을 인정하는 순간, 이에 대해서는 상황이 종결되어 버리고 다음 진도로 넘어가게 된다.(그들은 ‘기본 항목’은 확실히 건져야 하므로, 인정 안 하면 다음으로 넘어갈 수 없다.)
	\end{itemize}
\end{enumerate} 
\leftrod{3) 험악한 분위기 조성하면 안 된다.}
\begin{itemize}\tightlist
\item 그들도 사람이다.
\item 감정 상하면 얼마든지 열의를 가지고 조사할 수 있다.
\item 태업은 하되, 읍소도 하고 능글거리며 둘러대기도 해서 상황을 모면해야 한다. 
\end{itemize}

\leftrod{4) 패턴을 인정하면 안 된다.(사실 확인서 포함)}
\begin{enumerate}[①]\tightlist
\item 행정 소송의 특성
	\begin{itemize}\tightlist
	\item 조사가 끝나면 보건복지부 사무관이 정리해서 사전 처분
    -> 이의 신청 -> 본 처분 -> 행정소송으로 진행됨
	\item 행정 소송 시 부당 청구의 입증 책임은 복지부에 있음
	\item  원칙적으로 복지부에서는 부당 청구자 명단 전체에 대해 그 사실을 입증해야 함(복지부 사무관 혼자서 감당 못함)
	\item 만일 병원 측에서 그 중 일부가 정당한 청구였음을 입증한다면 일부만 무혐의가 되는 것이 아니라 전체가 무혐의로 처리됨
	\item  잘못된 패턴을 인정하고 사실확인서를 쓰면 그 자체가 입증 자료가 되므로 상황이 종결되어 버림
	\end{itemize}
\item 대처법
	\begin{itemize}\tightlist
	\item 평소에 문제의 소지가 있는 진료를 루틴으로 하면 안 된다.
	\item 조사자가 곤란한 질문을 하면 그 자리에서 잘못을 인정해버리지 말고, 일단 내용 확인 후 답변 하겠다고 둘러댄다.(일단은 넘어가고 관계 자료를 검토 후 답변을 찾는다.)
	\item 문제 case에 대해서만 인정하고 패턴은 인정하지 않는다.
     ‘ 이런 처방이 나간 환자가 1000명이 있는데 그 중 무작위로 30명에게 전화를 걸었더니 사실과 다르더라. 1000명에 대해 잘못을 인정해라’ -> ‘그 30명에게 왜 그런 잘못된 처방이 나갔는지 정확히는 모르겠지만, 나는 항상 제대로 했고, 그 30명에게만 잘못된 처방이 나간 것 같다
	 \end{itemize}
\end{enumerate}
	 
\leftrod{5) 실사가 나온 순간 바로 도움을 청해야 한다.}
\begin{itemize}\tightlist
\item 가장 중요한 내용임
\item 모르는 상태에서는 그들의 함정에 빠질 수밖에 없고 일단 잘못 진행된 부분은 돌이킬 수 없다.
\item 실사가 나온 순간 바로 ‘산의회 법제 이사팀’에게 연락해서 실시간으로 조언을 받도록 한다.
\end{itemize}	 