\section{감염병 신고}
감염병 예방 및 관리에 관한 법률에 따른 감염병 신고의무의 주요내용은 다음과 같습니다.
\begin{enumerate}[1)]\tightlist
\item 법정 감염병 신고는 의사의 법정 의무 사항으로 미신고 시 처벌이 따르게 되며  200만원 이하의 벌금이 부과됨.  
\item 법정감염병은 제1군에서 제5군까지 있으며 그 외 지정 감염병이 있음. \uline{제5군과 지정감염병은 표본감시기관만 신고하는 항목}이므로 \highlight{일반 병의원은 제1군부터 제4군까지 해당하는 질병을 진단하거나 의심하게 되면 진료를 하신 당일, 지체 없이 신고해야 함.} 특히 \uline{결핵, 수두, 볼거리, 홍역} 등이 주요 신고대상임. 단, 인플루엔자는 표본감시기관만 신고함. 일반 병의원은 신고하지 않음.
\item 일반적으로 법정감염병은 ‘환자’뿐만이 아니라 ‘의사(의심되는) 환자’, ‘병원체 보유자’까지 신고해야 하지만, 구체적인 사항은 질환마다 상이함. 예를 들면 장티푸스나 세균성 이질은 환자, 의사환자, 병원체 보유자를 모두 신고하지만 A형 간염은 확진 환자만 신고하고, 홍역이나 볼거리는 환자와 의사환자를 신고하고, B형 간염은 급성 B형 간염 확진 환자와 만성 B형 간염 환자 중 산모 또는 주산기 감염자만 신고함. 따라서 각각의 질환들에 대한 신고 범위를 숙지하여야 하지만, 일반적으로는 환자, 의사환자, 병원체 보유자를 모두 신고하는 것으로 알고 계시면 됨.
\item 신고방법은 팩스를 이용한 방법과 인터넷을 이용한 방법이 있으며 대개 숫자가 적으면 팩스가 편리하고 숫자가 많으면 인터넷을 이용하시는 것이 더 편리함.
\item 일반적으로 전자챠트 프로그램을 사용해서 진료하시는 분들이 많으므로, 진료프로그램의 환경설정을 법정 감염병인 경우 자동으로 팝업창이 뜨도록 해놓으시면 잊어버리지 않게 됨.
\item 특히, 학교나 기타 외부 단체에 법정 감염병에 해당되는 질환명으로 진단서나 소견서를 발급하는 경우 반드시 신고를 하여야 하며, 신고가 누락된 경우 소견서 등으로 관리 당국이 알게 되고, 추후 확인 절차에서 병의원 신고 자료와 차이가 있으면 벌금이 부과될 수 있음. 기본적으로 상병명에 법정 감염병을 올리게 되면 신고를 하는 것으로 알고 계시면 됨. 
\item 표본감시기관은 보건소나 질병관리본부와 따로 계약이 되어 있는 기관을 말하며 일반 병의원과는 관련이 없음.
\end{enumerate}
이번 감염병 신고의무 위반 건에 대한 법적 판단은 다음과 같습니다.
\begin{enumerate}[첫째,]\tightlist
\item 이번 감염병 신고의무 위반은 행정법 위반 사안인바, 고의·과실을 불문하고 위반 시 그 위반에 따른 법적제재가 이루어지게 됩니다.
\item 위반 시 감염병 예방 및 관리에 관한 법률 제81조 제1호에 의거 ‘200만 원 이하의 벌금’에 처해지게 됩니다.
\item 관할 검찰청에서 신고의무위반 정도가 경미한 경우 ‘약식기소(구약식)’, 경미하고 정상이 참작되는 경우 ‘기소유예’가 예상이 됩니다.
\item 조사과정에서 신고의무를 하지 못한 부득이한 사유를 밝히고, 지역보건발전에 이바지한 경력, 기타 지역사회에 기여한 공로 등 정상참작사유를 적극적으로 제시함으로써 가급적 기소유예가 결정될 수 있도록 노력이 필요합니다.
\end{enumerate}
\flushright{2014. 10.  . 경기도 의사회 알림}
