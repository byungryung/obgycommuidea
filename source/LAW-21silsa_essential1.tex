\subsection{관리항목별 선정기준 예시}
\tabulinesep =_2mm^2mm
\begin{tabu} to \linewidth {|X[4,l]|X[6,l]} \tabucline[.5pt]{-}
\rowcolor{Gray!25}  관리항목  & 선정기준 \\ \tabucline[.5pt]{-}
\rowcolor{Yellow!5} 내원일수 & 내원일수지표내원일수지표(VI) 1.1 이상 \& 건강진료비고가도지표(CI) 1.0 이상 \& 개설기관 상위 15\%  \\ \tabucline[.5pt]{-}
\rowcolor{Yellow!5} 급성상기도감염 항생제 처방률 & 항생체 처방률 80\% 이상 기관 \\ \tabucline[.5pt]{-}
\rowcolor{Yellow!5} 주사제 처방률 & 주사제 처방률 60\% 이상 기관 \\ \tabucline[.5pt]{-}
\rowcolor{Yellow!5} 약품목수 & 6품목 이상 처방비율 40\% 이상 기관 \\ \tabucline[.5pt]{-}
\rowcolor{Yellow!5} 외래처방약품비 & 외래처방약품비고가도지표(OPCI) 1.3 이상 기관 \\ \tabucline[.5pt]{-}
\end{tabu}
\par
\medskip

\tabulinesep =_2mm^2mm
\begin{tabu} to \linewidth {|X[1,l]|X[1,c]|X[1,c]|X[1,c]|X[1,c]|X[1,c]|} \tabucline[.5pt]{-}
\rowcolor{Gray!25}  종별  & 내원일수 & 급성상기도감염 항생제처방률 & 주사제 처방률 & 약품목수 & 외래처방 약품비 \\ \tabucline[.5pt]{-}
\rowcolor{Yellow!5} 의원 & \bullet & \bullet & \bullet & \bullet & \bullet  \\ \tabucline[.5pt]{-}
\rowcolor{Yellow!5} 치과의원 & \bullet  &  &  &  &   \\ \tabucline[.5pt]{-}
\rowcolor{Yellow!5} 한의원 & \bullet &  &  &  &   \\ \tabucline[.5pt]{-}
\rowcolor{Yellow!5} 병원 & \bullet &  & \bullet & \bullet &   \\ \tabucline[.5pt]{-}
\rowcolor{Yellow!5} 요양병원 & \bullet & \bullet & \bullet & \bullet &   \\ \tabucline[.5pt]{-}
\rowcolor{Yellow!5} 치과병원 & \bullet &  &  &  &   \\ \tabucline[.5pt]{-}
\end{tabu}
\par
\medskip
\begin{itemize}\tightlist
\item 치과병원, 한방병원 : 종별가산율 20% 해당(이외는 2012년부터) 
\item 종합병원 이상 : 2012년부터 통보 
\end{itemize}
	
\subsection{실사 가능하게 하는 관련법규}
\leftrod{요양급여 대상 여부의 확인(국민건강보험법 제 48조)}
\begin{enumerate}[①]\tightlist
\item 가입자나 피부양자는 본인일부부담금 외에 자신이 부담한 비용이 제41조제3항에 따라 요양급여 대상에서 제외되는 비용인지 여부에 대하여 심사평가원에 확인을 요청할 수 있다. → 병원에서 부당하게 받았는지 확인을 할수 있다. → \highlightR{민원 발생의 요인}
	\begin{itemize}\tightlist
	\item 환자가 자기의 진료비를 확인할 수 있는 제도 → 이런 민원으로 인해 실사가 증가합니다.
	\item 해결책은 진료 현장에서 납득을 시켜야 한다.
	\item 똑같은 사안에서 민원인에게는 YES, 병원에는 No 하는 경우가 종종 있어 이중잣대 논란
	\end{itemize}
\item 제1항에 따른 확인 요청을 받은 심사평가원은 그 결과를 요청한 사람에게 알려야 한다. 이 경우 확인을 요청한 비용이 요양급여 대상에 해당되는 비용으로 확인되면 그 내용을 공단 및 관련 요양기관에 알려야 한다. → \highlightR{급여대상인지 비급여 대상인지 확인후 알려준다}
\item 제2항 후단에 따라 통보받은 요양기관은 받아야 할 금액보다 더 많이 징수한 금액(이하 "과다본인부담금"이라 한다)을 지체 없이 확인을 요청한 사람에게 지급하여야 한다. 다만, 공단은 해당 요양기관이 과다본인부담금을 지급하지 아니하면 해당 요양기관에 지급할 요양급여비용에서 과다본인부담금을 공제하여 확인을 요청한 사람에게 지급할 수 있다. → \highlightR{급여를 비급여로 받는등 과다 본인 부담금을 받으면 비용을 뱉어내거나 공단이 병의원에 뺏어서 지불할 수 있다.}
\end{enumerate}

\leftrod{현지조사 법적 근거(건강보험법)}
\begin{description}\tightlist
\item[현지조사] 법 제97조 제2항 : 요양기관에 대한 보건복지부장관의 현지조사 권한 및 자료제출 명
\item[형정처분] 법 제98조, 법 제99조 : 요양기관 업무정지 처분 또는 과징금 처분
	\begin{itemize}\tightlist
	\item 요양급여비용 거짓ㆍ부당청구 한 때
	\item 서류미제출, 조사거부ㆍ방해 또는 기피 한 때
	\end{itemize}
\item[형사고발] 법 제115조 제3항 4호 : 업무정지기간 중 요양급여를 한 경우 1년이하 징역 또는 1천만원 이하 벌금
\item[형사고발] 법 제116조 : 서류 미제출, 거짓보고, 거짓서류제출 및 조사 거부, 방해, 기피한 경우 1천만원 이하 벌금
\end{description}

\subsection{실사의 유형}
\begin{enumerate}[가.]\tightlist
\item 정기조사
	\begin{itemize}\tightlist
	\item (지표점검기관) 자율시정통보 미시정 기관, 부당청구상시감지시스템(데이터 마이닝), 본인부담금과다징수 다발생 기관 등에 의해 부당청구 개연성이 높다고 판단되는 기관에 대해 실시하는 통상적 조사
	\item (외부의뢰기관) 공단 및 심평원의 급여사후관리 혹은 민원제보 및 타 행정기관의 수사 등의 과정에서 요양급여 비용의 부당청구가 확인 혹은 인지되어 보험급여내역전반에 대해 행정조사를 실시할 필요가 있다고 판단되는 기관에 대해 실시하는 조사
	\end{itemize}
\item 기획조사
	\begin{itemize}\tightlist
	\item 건강보험제도 운영상 또는 사회적 문제가 제기된 분야에 대해 제도개선 및 올바른 청구문화 정착을 도모하기 위해 실시하는 조사 \par
- 조사 공정성$\cdot$객관성 제고를 위해 민간전문가가 포함된 ‘기획조사항목선정협의회’를 통해 조사대상 분야 및 기준 등 심의\par
※ 기획조사 실시 전 조사 분야 및 조사 시기 사전 예고
	\item 2015년 기획실사 조사항목 :  
		\begin{enumerate}[①]\tightlist
		\item 건강보험 기획 현지조사 항목은 : 진료비 이중청구 의심기관
		\end{enumerate}
		\begin{enumerate}[①]\tightlist
		\item 의료급여 기획 현지조사 항목은 : 의료급여 사회복지시설 수급권자 청구기관,상반기 
		\item 의료급여 장기입원 청구기관 조사, 하반기
		\end{enumerate}
	\end{itemize}
\item 긴급 조사
	\begin{itemize}\tightlist
	\item 허위$\cdot$부당청구 개연성이 높은 요양기관이 증거인멸$\cdot$폐업 등 우려가 있거나, 사회적 문제가 야기된 분야 등으로 긴급한 조사가 필요한 경우에 실시하는 조사
	\end{itemize}
\item 이행실태 조사
	\begin{itemize}\tightlist
	\item 건강보험 업무정지처분 기간 중 당해 처분을 편법적으로 회피할 우려가 높은 기관 혹은 불이행이 의심되는 양기관 등에 대해 처분의 사후 이행 여부를 점검하기 위해 실시하는 조사
	\end{itemize}
\end{enumerate}

\subsection{실사[현지조사]의 현황}
\begin{commentbox}{복지부 실사담당 사무관의 증언 中에서}
우리나라에 의료기관(병원.의원.한의원.약국 포함) 숫자가 8만 5000개정도이며, 1년에 실사가 900건~1000건 정도 나간다는데 그중 700-850군데가 소위 말하는 일반적인 실사이고 50-150군데 정도가 "기획실사"랍니다. \par 					
\end{commentbox}
2001년도(250개소) \par
2008년도(1018개소) \par
2009년도(954개소) \par
2010년도(920개소) \par
2011년도(1003개소) \par
2012년도(684개소) \par
2013년도(931개소)\par

\subsection{실사의 원인}
\leftrod{실사조사유형의 분류}%\par
\tabulinesep =_2mm^2mm
\begin{tabu} to .75\linewidth {|X[1,l]|X[6,l]} \tabucline[.5pt]{-}
\rowcolor{Gray!25}  순위  & 유형 \\ \tabucline[.5pt]{-}
\rowcolor{Yellow!5} 1위 & 잦은 민원 \\ \tabucline[.5pt]{-}
\rowcolor{Yellow!5} 2위 & 내부 고발자 \\ \tabucline[.5pt]{-}
\rowcolor{Yellow!5} 3위 & 기타(기획조사, 테이터 마이닝, 자율시정 불응기관) \\ \tabucline[.5pt]{-}
\end{tabu}

\begin{itemize}\tightlist
\item 공단 의뢰기기관 
\item 심평원 의뢰기관 
\item 민원제보기관 (검.경찰 권익위, 국민신고마당)
\item 자율시정 불응기관 :일년에 3-4000개 통보 기관중 40-50개 현지조사
\item 데이터 마이닝(부당감지지표)에 의한 기관 
\end{itemize}
\leftrod{데이터 마이닝}
\begin{enumerate}[①]\tightlist
\item 야간․공휴일 진찰료 청구율 
\item 진찰료 단독 청구빈도
\item 원외처방전 미발행률 
\item 초진료 청구빈도
\item 수진자당 보유상병 개수 
\item 처방투약 일수별 처방건수
\item 진찰료 및 검사료 단독 청구 빈도
\item 정신과 환자당 내원일수 및 정신요법료 청구 횟수
\item 이학요법료 항목별 청구빈도 
\item 처방전 2개소 이상 중복 청구건수
\item 약국 동일처방전 중복청구내역 
\item 처방전 건수 불일치건 발생빈도
\item 처방전 집중률과 고가약 처방빈도 
\item 진료지표 급등기관 
\item 요양급여비용 지연청구기관
\end{enumerate}

\subsection{현지조사 대상기관 선정}
\leftrod{\textbf{심평원 의뢰기관}}
\begin{itemize}\tightlist
\item 요양급여비용에 대한 심사 및 평가, 건강보험 재정지킴이 신고 등
 (위에서 부당청구의 개연성이 높게 나타난 기관)
\item 부당청구감지시스템을 통하여 부당청구 개연성이 높은 요양기관
\item 정당한 사유 없이 2회 이상 자료제출 거부하여 부당사실관계 확인 곤란한 기관
\end{itemize}
\textbf{업무 절차}\par
\menu{현지조사의뢰(본ㆍ지원) > 실익검토(조사운영부) > 보건복지부로 현지조사 의뢰}\par
\menu{ > ‘현지조사 선정 심의위원회’심의 > 현지조사 대상선정}\par
☞ ‘선정심의위원회’ 구성(12명) : 공공위원(3명), 의약단체(5명), 시민단체(1명), 법조계 등(3명)\par \medskip

\leftrod{\textbf{공단 의뢰기관 선정기준}}
\begin{itemize}\tightlist
\item 진료내역 통보 및 수진자조회 등 과정에서 부당청구 개연성이 높게 나타난 기관
\item 요양기관 내부종사자 등에 의해 신고된 요양기관
\item 정당한 사유 없이 2회 이상 자료제출 거부하여 부당사실관계 확인 곤란한 기관
\end{itemize}
\textbf{업무 절차}\par
\menu{현지조사의뢰 공단→복지부 > 실익검토요청 복지부→심평원 > 실익검토보고 심평원→복지부 }\par 
\menu{>‘현지조사 선정심의위원회’심의 > 현지조사 대상선정}\par 


※ 공단 지사 → 공단지역본부 → 공단 본부(급여관리실) → 보건복지부\par \medskip

\leftrod{\textbf{내부종사자 고발}}%\par
부당청구 요양기관 신고포상금제 시행 10년\cntrdots{}지급된 포상금 40억
\begin{itemize}\tightlist
\item 포상금제 시행 10년
\item 지급된 포상금 40억
\item 최고 10억 원 지급
\end{itemize}

\begin{commentbox}{내부종사자 공익신고기관 포상금제도}
법 제104조 (포상금지급)\par
공단은 속임수나 그 밖의 부당한 방법으로 보험급여 비용을 지급받은 요양기관을 신고한 사람에 대하여 대통령령에 따라 포상금을 지급
\end{commentbox}

\leftrod{\textbf{대외 의뢰 기관}}%\par
검ㆍ경찰ㆍ감사원ㆍ관련행정부처ㆍ지방자치단체ㆍ국민권익위원회 등 으로부터 건강보험 부당청구 등 혐의로 현지조사 의뢰된 기관으로 부당청구 개연성이 높은 요양기관 \par

\leftrod{\textbf{민원제보 기관}}%\par
요양기관 거짓ㆍ부당청구에 대해 제보된 기관 중 실명의 제보자가 구체적인 사례와 증거를 제시하는 등 부당청구 개연성이 높은 기관 \par

\leftrod{\textbf{본인부담과다징수 다발생 기관}}%\par
요양급여대상여부 확인 등에 의한 본인부담과다 징수 다발생 기관 중 \par
\begin{itemize}\tightlist
\item 매 분기별 발생건수, 부당건수, 부당금액 등 자율시정 정도 등을 검토하여 현지조사가 필요하다고 판단되는 기관
\item ※ 예: 2010년, 2012년 상급종합병원 전수 기획조사 실시
\end{itemize}

\subsection{2017년 기획조사 항목}
건강보험 분야[’16. 12. 21. 사전예고]
\begin{itemize}\tightlist
\item 본인부담금 과다징수 의심기관 [상반기, 하반기] - 상급종합병원 43개소 전수조사
\end{itemize}
의료급여 분야[’16. 12. 21. 사전예고]
\begin{itemize}\tightlist
\item 의료급여 장기입원 청구기관[상반기]
\item 선택의료급여기관에서 의뢰된 진료 다발생 청구기관[하반기]
\end{itemize}

\subsection{실사 피하기 – 진료지표 관리하기}
\begin{enumerate}[해결책1.]\tightlist
\item 복합 상병시 건당진료비가 높을거 같은 상병을 맨 위에 기재
:질염(N761~N768)보다는 골반염(N730~N739)또는 자궁근종(D250~D259) 
\item 다빈도 상병을 보조상병으로 입력: K30, K297, M5496, R1039등등
\item  진료비가 높은 경우 타과 상병을 입력 
 :동일 과에 대한 다빈도 상병을 기준으로 통계를 냄: 질염(N761~N768)보다는 고혈압(I10), 당뇨(E149)을 위의 상병으로 넣을것.
\item 이왕이면 엄청난 상병을 입력
: 정확한 코딩을 위하여서는 환자의 상태에 대한 상병(코드)을 적어야 하지만 실상 
	심평원에서는 환자의 상태는 별개로 상병(코드)과  처치 내역을 보고 심사를 
	하고 있으므로 광역의  상병 또는 중환 상병이 필요(난소의 양성 신생물(D279)
	보다는 난소의 악성신생물(C569)소신진료후 진료기록포함 증빙자료 작성하여
	보존
\end{enumerate}
\begin{commentbox}{}
\begin{itemize}\tightlist 
\item 한마디로 비쌀거 같은 상병을 주상병으로 
\item 상병은 겁나 많이 넣고 
\item 내과 상병  주상병으로 
\item 엄청난 상병 주상병으로..
\end{itemize}
\end{commentbox}

\subsection{실사 피하기 – 환자 납득시키기}
납득이 되게 영수증 발행하기
\subsection{실사 피하기 – 직원 만족}
퇴사시 따듯한 밥 한그릇이라도….
