\section{빈혈}
\myde{}
{
\begin{itemize}\tightlist
\item[\dsjuridical] D500,D508,D509  철결핍 빈혈
\item[\dsjuridical] D510,D511,D513,D519  비타민 B12결핍성 빈혈
\item[\dsjuridical] D520,D521,D528,D529  엽산 결핍성 빈혈(Folate deficiency anaemias, unspecified)
\item[\dsjuridical] D530,D531,D532,D538,D539 단백질, 거대적모구,괴혈병, 영양성 빈혈
\item[\dsjuridical] D550,D551,D552,D553,D558,D559  대사장애 ,효소장애에 의한 빈혈
\item[\dsjuridical] D560,D561,D562,D563,D568,D569  지중해 빈혈
\item[\dsjuridical] D570,D571  낫적혈구 빈혈
\item[\dsjuridical] D588,D589  유전성 용혈성 빈혈
\item[\dschemical] D0002050 혈색소(광전비색) \textcolor{red}{적합상병 D50}
\item[\dschemical] D0002040 헤마토크리트
\item[\dschemical] D0002030 적혈구수
\item[\dschemical] D0002010 백혈구수
\item[\dschemical] D0002070 혈소판수
\item[\dschemical] D0013 백혈구 백분율(혈액)
\item[\dschemical] D0002023 적혈구분포계수
\item[\dschemical] D0002063 혈소판분포계수
\item[\dschemical] D0502013 망상적혈구수(유세포분석법)
\item[\dschemical] D05100 혈구형태(말초혈액도말)
\item[\dschemical] D0521033 철
\item[\dschemical] D0521043 철결합능
\item[\dschemical] D0522013 훼리틴
\item[\dschemical] D4902053 비타민 B12 \textcolor{red}{적합상병 D51}
\item[\dschemical] D4902143 엽산 \textcolor{red}{적합상병 D52}
\end{itemize}
}
{
\tabulinesep =_2mm^2mm
\begin {tabu} to \linewidth {|X[3,l]|X[2,l]|} \tabucline[.5pt]{-}
\rowcolor{ForestGreen!40} \centering Age or gender group & \centering Hb threshold (g/dl)\\ \tabucline[.5pt]{-}
\rowcolor{Yellow!40} Children (0.5–5.0 yrs) & 11.0\\ \tabucline[.5pt]{-}
\rowcolor{Yellow!40} Children (5–12 yrs) & 11.5\\ \tabucline[.5pt]{-}
\rowcolor{Yellow!40} Women, non-pregnant (>15yrs) & 12.0\\ \tabucline[.5pt]{-}
\rowcolor{Yellow!40} Women, pregnant & 11.0\\ \tabucline[.5pt]{-}
\rowcolor{Yellow!40} Men (>15yrs) & 13.0\\ \tabucline[.5pt]{-}
\end{tabu}
\par
\medskip
}

\subsection{빈혈 분류 및 정리}

\leftrod{Common Causes for VariousTypes of Anemia}
\begin{enumerate}[1.]\tightlist
\item Hypochromic, microcytic
	\begin{itemize}\tightlist
	\item \textcolor{red}{Iron Deficiency}
	\item Thalassemia syndromes
	\item Sideroblastic anemia
	\item Transferrin deficiency
	\end{itemize}	
\item Macrocytic:
	\begin{itemize}\tightlist
	\item \textcolor{red}{Megaloblastic Anemias(Folic acid/ B12 deficiencies)}
	\item Liver Disease
	\item Reticulocytosis
	\item Normal newborn
	\item Bone marrow failure syndromes
	\item Drugs (AZT, Trimethoprin sulfate)
	\end{itemize}	
\item Normocytic, normal morphology:
	\begin{itemize}\tightlist
	\item \textcolor{red}{Hemorrhage or blood loss}
	\item Unstable hemoglobins
	\item Infections
	\item Chronic disease
	\end{itemize}
\item Normocytic, abnormal morphology:
	\begin{itemize}\tightlist
	\item Hemoglobinopathies, (SS, SC, CC)
	\item Hereditary Spherocytosis
	\item Autoimmune hemolytic anemia
	\item Some enzymatic deficiencies
	\end{itemize}
\end{enumerate}
\par
\medskip
\leftrod{빈혈의 정리}
The Complete Blood Count:
\begin{enumerate}[1.]\tightlist
\item Hematocrit (Hct) or packed cell volume (PCV):\newline
\textcolor{red}{Volume of packed red blood cells per unit of blood,expressed as a percentage.}\newline
Example: 44 ml packed red blood cells/ 100 ml of blood = 44\%
\item Hemoglobin = \textcolor{red}{grams of hemoglobin/ dL of blood}
\item Reticulocyte = \textcolor{red}{Young RBC}
Anemia due to hemolysis or bleeding is characterized by
the presence of a reticulocytosis. The reticulocyte count
is used to assess the appropriateness of the bone
marrow response to anemia. The normal reticulocyte
count in a patient with a normal Hb and Hct is about 1\%.
Approximately 1\% of circulating RBCs are removed daily
and replaced by marrow young RBCs or reticulocytes
(approximately 20 cc of RBCs /day)
\item \textcolor{red}{Mean Cell (or Corpuscular) Volume (MCV):}
The MCV \textcolor{red}{reflects the average size or volume of the RBC} 
expressed in fl. MCV will \textcolor{red}{tell you if the patient is micro, macro, or normocytic}. MCV is calculated as follows:\newline
MCV = Hct/ RBC Count = Volume of packed red cells (\% X 10)/Red cell count (x 1012/l)
\item Mean Cell Hemoglobin (MCH):
The MCH indicates the weight of Hb in the average red cell.
MCH is calculated as follows:\newline
MCH = Hb/RBC count 
= Hb (gm/dl X 10)/Red cell count (x 1012/l)
\item Mean Cell Hemoglobin Concentration (MCHC):
The MCHC indicates the concentration of Hb in the average
red cell or the ratio of the weight of the Hb to the volume in
which it is contained (chromicity), expressed in percent as
follows:\newline
MCHC = Hb/Hct 
= Hb (gm/dl X 100)/Volume of packed red cells (ml per 100 ml
\item Normal red blood cell morphology: is characterized
by a donut shape with the center 1/3 of the red cell being pale or
without hemoglobin. \textcolor{red}{This is assessed on peripheral smear. }
\end{enumerate}

\subsection{철 (Iron : Fe)이란?} 
\begin{enumerate}[1)]\tightlist
\item 정상 성인의 혈청 총철함량은 4~5g이다 
\item 체내의 철 존재형태
	\begin{enumerate}[⑴]\tightlist
	\item 활성형 70~75\% : Hemoglobin , Myoglobin , Transferrin , Enzyme의 형태 
		\begin{enumerate}[①]\tightlist
		\item Transferrin 
			\begin{itemize}\tightlist
			\item β-Globulin 분획에 위치하는 단백이다 
			\item Iron (Fe)과 특이적으로 결합하는 능력이 있다 
			\item Transferrin 100㎎은 Fe 120㎍과 결합한다 
			\end{itemize}
		\end{enumerate}
	\item 저장형 25~30\% : Ferritin , Hemosiderin의 형태 
		\begin{enumerate}[①]\tightlist
		\item Ferritin 
			\begin{itemize}\tightlist
			\item 철 및 저장철이다. 인체 철의 약 25\%를 차지한다 
			\item Apoferritin Protein과 복합체를 이루고 있으며 용해성이다 
			\item 세망내피계 세포 (골수 , 간 , 비장 , ... ... ... 등)에 있다 
			\item 혈청 Ferritin의 농도는 세망내피계 세포에 저장된 철의 양을 반영한다 
			\end{itemize}
		\item Hemosiderin 
			\begin{itemize}\tightlist
			\item 불용성이다 
			\item 세망내피계 세포에만 존재한다 
			\end{itemize}
		\end{enumerate}
	\end{enumerate}
\item 총 철 결합능 (Total Iron Binding Capacity : TIBC)
	\begin{enumerate}[1)]\tightlist
	\item Transferrin의 총량을 의미한다 
	\item Iron (Fe)과 결합하지 않은 Tansferrin을 불포화 철 결합능 (Unsaturated Iron Binding Capacity : UIBC)이라 한다 
	\item TIBC = UIBC + Serum Fe 
	\end{enumerate}
\item 정상치 
	\begin{enumerate}[1)]\tightlist
	\item ♂ 80~180㎍/㎗ 
	\item ♀ 70~160㎍/㎗ 
	\item TIBC 317~395㎍/㎗ 
	\end{enumerate}
\item 측정시 주의사항 
	\begin{enumerate}[1)]\tightlist
	\item 조기공복시 채혈이 가장 좋다 (혈청철은 아침에 높고, 야간에 낮다. 그 변동은 50㎍/㎗ 이상) 
	\item 용혈의 영향이 크므로 용혈이 있으면 안된다 
	\item 측정시의 모든 용기는 산처리 한다 (1N HCl , 1N HNO3를 사용한다) 
	\end{enumerate}
\end{enumerate}

\Que{현재 철분제와 관련된 보험 기준은 액제와 주사제만 있습니다.정제의 경우 고가의 철분제(헤모큐추어블정 등)(295원/1정)는 액제와 동일한 기준으로 보험을 적용하고, \textcolor{blue}{저가의 철분제(훼로바유서방정 등)(95원/1정)는 별도의 세부 기준이 없습니다.}또한 저가의 철분제의 경우 세부인정기준이 없으므로 진료담당의사가 환자진료에 반드시 필요하다고 판단하여 시행한 경우 허가사항범위내에서 필요, 적절하게 처방 투여된 의약품은 요양 급여가 가능하다고 알고 있습니다.
\begin{enumerate}\tightlist
\item 빈혈 환자에게 저가의 철분제를 처방할 경우, 별도의 수치 없이 보험이 적용 가능하다는 말로 해석해도 되는건지요? 혈액검사를 실시하지 않고, 의사의 판단만으로 처방을 내리면 급여가 가능한건지 궁금합니다.
\item 또한 빈혈로 판정되지 않더라도 빈혈 예방 차원에서 철분제가 필요하다고 판단되면 별도의 수치 없이 보험이 적용 가능한지 궁금합니다. 예를 들어 어지러움을 호소하는 환자에게 저가의 철분제를 처방하거나, 또는 임산부에게 빈혈 예방 차원에서 저가의 철분제를 처방할 경우 모두 보험 적용 가능한지요?
\item KIMS(kimsonline) 확인시 저가의 철분제 보험 기준으로 혈액검사결과 Hb 10g/dl 이하일 경우 가능하다는 내용이 나오는데 이는 심평원의 심사기준과 다른데 어느 것이 맞는 것인지요?
\end{enumerate} }\index{빈혈처치!저가 철분제 보험가능}
\Ans{저가의 철분제의 경우 세부인정기준이 없으므로 진료담당의사가 환자진료에 반드시 필요하다고 판단하여 시행한 경우 허가사항범위내에서 필요, 적절하게 처방 투여된 의약품은 요양 급여가 가능하므로 
\textcolor{red}{어지러움을 호소하는 환자나 임산부에게 빈혈 예방 차원에서 저가의 철분제를 처방할 경우 모두 보험 적용 가능}합니다.}

\subsection{저가의 철분제}\index{빈혈!저가 철분제 종류}
\begin{itemize}\tightlist
\item 훼마톤 에이 정 (78원/1정) : 성분:ferric hydroxide-polymaltose complex 357㎎ (100㎎ as iron) + folic acid 350㎍ \\
\item 훼럼포라 정 (78원/1정) : 성분:ferric hydroxide-polymaltose complex 357㎎ (100㎎ as iron) + folic acid 350㎍ \\
\item 훼로바-유 서방정 (95원/1정) : 성분:dried ferrous sulfate 256㎎ (80㎎ as iron) \\
\item 헤모골드-에프 정 45mg (97원/1정) : 성분:carbonyl iron 45㎎\\
\item 헤모니아 캅셀 150mg (102원/1캅셀) : 성분:polysaccharide-iron comlplex 326.1㎎ (150㎎ as iron) \\
\end{itemize}

\subsection{기타의 철분제}
\begin{itemize}\tightlist
\item Chondroitin sulfate-iron complex 경구제 (품명: 리코베론과립, 페리콘캡슐 등) \\
\item Iron acetyl-transferrin 200mg 경구제 (품명: 알부맥스캅셀 등) \\
\item Iron proteinsuccinylate 400mg 경구제 (품명: 헤모큐츄어블정) \\
\item 액제형 철분제제 (품명 : 헤모큐액 등) \\
\end{itemize}

%subsection{Iron proteinsuccinylate 400mg 경구제(품명: 헤모큐츄어블정)의 급여기준}
\begin{commentbox}{}
\emph{Iron proteinsuccinylate 400mg 경구제(품명: 헤모큐츄어블정)의 급여기준}\par
허가사항 범위 내에서 아래와 같은 기준으로 투여 시 요양급여를 인정하며, 동 인정기준 이외에는 약값 전액을 환자가 부담토록 함.\par
\begin{center}\emph{-아 래-}\end{center}
\begin{enumerate}[가)]\tightlist
%\begin{enumerate}\tightlist
\item 일반적인 철결핍성 빈혈에는 혈액검사결과 다음에 해당되고 타 경구 철분제제 투여 시 위장장애가 있는 경우에 급여하며, 투여기간은 통상 4~6개월 급여함.
	\begin{enumerate}[1)]\tightlist
	\item 일반 환자 혈청페리틴(Serum ferritin) 12ng/㎖ 미만 또는 트란스페린산호포화도(Transferrin saturation rate) 15\% 미만인 경우 
	\item 만성신부전증 환자Serum ferritin 100ng/㎖ 미만 또는 Transferrin saturation rate 20\% 미만인 경우 
	\end{enumerate}
\item 임신으로 인한 철결핍성 빈혈혈액검사결과 Hb 10g/㎗ 이하이고 타 경구 철분제제 투여 시 위장장애가 있는 경우에 급여하며, 투여기간은 4~6개월로 함. 
\item 급성출혈 등으로 인한 산후 빈혈혈액검사결과 Hb 10g/㎗ 이하인 경우에 급여하며, 투여기간은 4주로 함.    
	\begin{itemize}[*]\tightlist
	\item 시행일: 2013.9.1.
	\item 종전고시: 고시 제2011-163호(2012.1.1.)
	\item 변경사유: 용어정비
	\end{itemize}
\end{enumerate}
\end{commentbox}

%\subsection{액제형 철분제제(품명 : 헤모큐액 등) 급여기준}
