\section{직장질루교정술 Operation for Recto-Vaginal Fistula}
\myde{}
{\begin{itemize}\tightlist
\item[\dsjuridical] N823 대장으로 열린 질루 (직장질루)
\item[\dsmedical] R4085 가. 질부 조작 Vaginal Approach \myexplfn{2025.83} 원    
\item[\dsmedical] R4086 나. 경항문 혹은 경회음부 조작Transanal or Transperineal Approach \myexplfn{2353.41} 원
\item[\dsmedical] R4087 다. 복부 조작 Abdominal Approach \myexplfn{3015.16}
\end{itemize}
}
{}
\prezi{\clearpage}
\Que{환자가 분만이후 질쪽에서 회음부쪽으로 누공이 발생하여 생리혈등이 샌다고 합니다. 질누공절제술을 시행해야 하는데 수가를 어떤것으로 해야할지..수가에는 직장질루교정술 밖에 없는데 그거에 준해서 산정해야 하는지요?
산정한다면( R4086- 경항문 혹은 경회음부조작)에 준하여 산정해도 되는지요?}

\Ans{질의된 내용만으로는 수술방법에 대한 정보가 부족하여 정확한 답변드리지 못하는 점 양해바라며, \textcolor{red}{건강보험요양급여비용 행위 급여 일반원칙 제3호에 의해 제2부 각 장에 분류되지 아니한 처치를 실시한 경우에는 우선적으로 행위의 내용, 성격과 상대가치점수가 가장 유사한 분류항목에 준용}하여 산정토록 하고 있습니다. \par 참고로 건강보험요양급여비용 목록에 등재되어 있는 행위에 대한 자세한 설명은 우리원 홈페이지(www.hira.or.kr) → 기준법령 → 상대가치점수조회 → 행위정의에서 조회가 가능하며, 차후 진료비 청구시에는 진료기록지와 수술기록지 등 관련자료를 첨부하여 원활한 심사가 이루어질 수 있도록 하여 주시기 바랍니다.}
