\section{직원 해고} 
\subsection{해고의 제한}
\begin{mdframed}[linecolor=blue,middlelinewidth=2]
해고란 사용자 일방적 의사로 근로계약을 종료시키는 것으로 해고의 자유는 경제적,사회적  약자의 지위에 있는 근로자에게 직장상실의 위험을 의미하므로 근로기준법은 해고 사유와  절차를 제한하고 있음.
\end{mdframed}
\emph{해고의 정당한 사유}
\begin{itemize}[□]\tightlist 
\item 근로자를 해고하기 위해서는 정당한 사유가 있어야 함.
\item 정당한 사유란 사회통념상 근로계약을 유지시킬 수 없을 정도로 근로자에게 책임 있는 사유가 있거나 부득이한 경영상의 필요가  있는 것을 말함.
\item 정당한 사유(예시)
	\begin{itemize}[-]\tightlist 
	\item  노동능력의 상실로 근로자가 경미한 업무조차 감당할 수 없는  경우
	\item  무단 결근․지각․조퇴의 반복 등 근로제공 의무 위반
	\item  불성실한 업무태도로 인한 실적부진
	\item  업무와 관련된 절도․횡령 등
	\end{itemize}
\end{itemize}
\begin{mdframed}[linecolor=red,middlelinewidth=2]
\emph{해고예고(근기법제26조,제27조)}\\
근로자를  해고할   경우에는  해고일을   명시하여  개별 근로자에게  해고일의  30일 전에  해고사유와 해고시기를 첨부하여 서면으로 예고를 하고, 해고예고를  하지 아니하였을 경우 30일분의 이상의 통상임금을 해고예고 수당으로 지급하여야 함.
\end{mdframed}

\subsection*{해고의 절차}
\begin{itemize}[□]\tightlist 
\item 해고의 서면통지 규정을 제외하고는 근기법에서는 해고 절차에 대해 별다른 규정을 두고 있지 않음.
\item 구체적 해고 절차는 취업규칙으로 정하되, 해고는 근로계약을 소멸시키는 사용자의 의사표시이므로 사용자의 의사표시가 반드시 선행 되어야 함. 
\item 취업규칙에서 정하고 있는 해고절차를 위반한 해고는 절차상의 흠으로 위법한 해고가 됨. 
\end{itemize}

\subsection{근로계약관련 판례}
\begin{itemize}[□]\tightlist 
\item 근로계약 만료에 따른 재계약 거부(대법원 2011.04.14. 선고 2007두1729 판결 )
	\begin{itemize}[-]\tightlist
	\item 기간을 정하여 체결한 근로계약에서 근로자에게 근로계약이 갱신될 수 있으리라는 정당한 기대권이 인정되는 경우, 그 기대권에 반하는 사용자의
    부당한 근로계약 갱신 거절의 효력은 무효임. 이 경우 기간만료 후의 근로관계는 종전의 근로계약이 갱신된 것과 동일함.
	\end{itemize}
 
\item 정년과 근로계약 기간(대법원 2003. 12. 12 선고 2002두12809판결)
	\begin{itemize}[-]\tightlist
	\item 근로자가 정년이 지난 후에도 사용자의 동의 하에 기간의 정함이 없이 사용자와의 근로관계를 계속 유지하여 온 경우에 단순히 당해 근로자가 
    정년이 지났다거나 고령이라는 이유만으로 근로관계를 해지할 수 없음. 
	\end{itemize}
\end{itemize}