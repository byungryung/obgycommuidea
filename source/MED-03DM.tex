\section{합병증이 없는 당뇨병}
\myde{}{%
\begin{itemize}\tightlist
\item[\dsjuridical] E149 합병증이 없는 당뇨병 
\end{itemize}
\tabulinesep =_2mm^2mm
\begin{tabu} to \linewidth {|X[2,l]|X[2,l]|} \tabucline[.5pt]{-}
\rowcolor{ForestGreen!40} \centering 검사코드 & \centering 검사명 \\ \tabucline[.5pt]{-}
\rowcolor{Yellow!40}  D3021 & 당검사(반정량) \\ \tabucline[.5pt]{-}
\rowcolor{Yellow!40}  D3022003 &  당검사(정량) \\ \tabucline[.5pt]{-}
\rowcolor{Yellow!40}  D1890  &  GTP \\ \tabucline[.5pt]{-}
\rowcolor{Yellow!40}  D2800023 &  전해질(소디움) Na \\ \tabucline[.5pt]{-}
\rowcolor{Yellow!40}  D2800063 &  전해질(포타슘) K \\ \tabucline[.5pt]{-}
\rowcolor{Yellow!40}  D2800033 &  전해질(염소) Cl \\ \tabucline[.5pt]{-}
\rowcolor{Yellow!40}  D3050023 &  인슐린 \\ \tabucline[.5pt]{-}
\rowcolor{Yellow!40}  D3061003 &  헤모글로빈 A1C \\ \tabucline[.5pt]{-}
\rowcolor{Yellow!40}  D3002003 &  미량알부민검사(정량) \\ \tabucline[.5pt]{-}
\rowcolor{Yellow!40}  D3050013 & C-peptide \\ \tabucline[.5pt]{-}
\rowcolor{Yellow!40} 고혈압 & + 10종더 가능함.\\ \tabucline[.5pt]{-}
\end{tabu}
}{
폐경기 환자의 상당수는 고혈압, 당뇨등이 같이 있습니다.
}
\subsection{Hgb A1C와 Fructosamine검사인정기준}
나382마 Hemoglobin A1C 검사와 나397 Fructosamine검사는 당뇨병 환자에게 시행하는 혈당조절 지표검사로 Hemoglobin A1C 검사는 3-4개월 간격으로 인정하며, 나397 Fructosamine검사는 Hgb A1C검사가 부정확할 때(용혈성빈혈, 혈색소병증 등) 실시 시 인정함. (2009.8.1 시행)

\subsection{Insulin \& C-peptide 연속검사 인정기준}
췌장의 인슐린 분비능력을 평가하는 나341 Insulin검사와 나343 C-Peptide 연속검사의 인정기준은 다음과 같이 함.\par
-다 음 -
\begin{enumerate}[가.]\tightlist
\item 실시 횟수 - 특별한 문제가 없는 당뇨병 환자에서는 자극물질 투여후 시간별로 수회의 검사를 할 필요가 없으므로 \uline{자극물질의 종류와 관계없이 자극물질 투여전 1회(기초1회)와 투여후 1회 실시를 원칙으로 하여, Insulin검사와 C-Peptide검사를 각 2회씩 인정함. }
다만, 저혈당, Insulin Resistance 등 특수한 경우는 연속검사 1회 실시시 기초 1회와 자극후 수회를 실시할 수 있음.
\item 실시 간격 : \uline{6개월에 1회 인정함을 원칙으로 하되}, 갑작스런 환자상태의 변화가 있거나 치료방법의 변경(예 : 경구혈당강하제 → Insulin주사제) 등이 있는 경우는 6개월내에 실시하더라도 인정함.
\item (2007.11.1 시행)
\end{enumerate}  

나-230 미량알부민검사 Microalbumin 종류\par

\tabulinesep =_2mm^2mm
\begin{tabu} to\linewidth {|X[1,l]|X[1,l]|X[6,l]|X[1,l]|X[1,l]|} \tabucline[.5pt]{-}
\rowcolor{ForestGreen!40}  & 코드 &	\centering 분 류 & 점수 & 금액 \\ \tabucline[.5pt]{-}
\rowcolor{Yellow!40} 나-230  & & 미량알부민검사 Microalbumin & & \\ \tabucline[.5pt]{-}
\rowcolor{Yellow!40} & D3001003 & 가. 정성 Qualitative & 10.67 & \myexplfn{10.67 } \\ \tabucline[.5pt]{-}
\rowcolor{Yellow!40} & D3002003 & 나. 정량 Quantitative & 203.50 & \myexplfn{203.50 } \\ \tabucline[.5pt]{-}
\rowcolor{Yellow!40} & C7230 & 주:핵의학적 방법으로 검사한 경우에는 & 189.97 &  \\ \tabucline[.5pt]{-}
\end{tabu}

\medskip
\begin{commentbox}{나230 미량알부민검사 인정기준}
나230 미량알부민검사는 다음에 해당되는 환자로서 요일반검사 (나1 또는 나3)에서 요단백이 검출되지 아니하여 실시한 경우에 인정함. \par
- 다  음 -
\begin{enumerate}[가.]\tightlist
\item \uline{당뇨병성 신증이 의심되는 당뇨병 환자 }
\item \uline{심혈관계 합병 위험인자(비만, 당뇨, 고지혈증, 뇌졸중 등)가 있는 고혈압환자(2009. 4.1 시행)}
\end{enumerate}
\end{commentbox}

제2형 당뇨병에 수회 실시한 미량알부민검사 인정여부 \par
\begin{itemize}[■]\tightlist
\item 청구내역 (남/55세)
	\begin{itemize}[-]\tightlist
	\item 상 병 명: 당뇨병성 다발성 신경병증을 동반한 인슐린-비의존 당뇨병 
	\item 주요 청구내역 : [검사료] 나230나 미량알부민검사-정량(C2302) 1*1*2 
	\end{itemize}
\item 심의내용 
	\begin{itemize}[○]\tightlist
	\item 보건복지부 고시 제2009-55호(‘09.4.1 시행)에 의하면, 나230 미량알부민검사는 당뇨병성 신증이 의심되는 당뇨병 환자 ? 심혈관계 합병 위험인자(비만, 당뇨, 고지혈증, 뇌졸중 등)가 있는 고혈압 환자로서, 요일반검사(나1 또는 나3)에서 요단백이 검출되지 아니하여 실시한 경우에 인정하고,
관련 교과서 등에 의하면, 제2형 당뇨병의 경우 당뇨병의 발현 시기가 확실치 않은 경우가 많으므로 당뇨병 진단 시점부터 미세알부민뇨 검사를 시행해야 하며, 정상(임의소변<30μg/mg(mg/g) creatinine, 24시간 소변<30mg/day)인 경우에는 1년에 한 번씩 미량알부민뇨 검사 실시를 권유하고 있음. 
	\item 동 건(남/55세)은 제2형 당뇨병 상병('02년 진단)에‘13.6.24 실시한 요일반검사(나3) 결과 요단백 ‘Negative’, 미량알부민검사 결과‘1.3mg/dl(ACR 12mg/g creatinine)’이었으며, 이후‘13.7.16-7.29 입원기간 중 ‘13.7.17(2.6mg/dl(ACR 14mg/g creatinine)), ‘13.7.18(2.2mg/24hr)에 미량알부민검사를 실시하고 나230나 미량알부민검사-정량 2회를 청구한 사례임. 
	\item 전반적인 진료내역을 검토한 결과,‘13.6.24 미량알부민검사 결과가 정상인 상태에서 23-24일 경과한 ‘13.7.17, 7.18에 다시 동 검사를 실시할만한 타당한 사유가 확인되지 않고, 미량알부민뇨가 정상인 경우 미량알부민뇨 검사를 1회/년 실시하는 것이 타당하다는 관련 교과서 등을 참조하여,‘13.7.17, 7.18에 실시한 ‘나230나 미량알부민검사-정량 2회’는 인정하지 아니함
	\end{itemize}
\end{itemize}