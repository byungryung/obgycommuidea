\section{무월경}
\myde{}{
\begin{itemize}\tightlist
\item[\dsjuridical] N979 상세불명의 여성불임
\item[\dsjuridical] N915(희발월경), N912(상세불폄의 무월경)
\item[\dsjuridical] E282 다낭성 난소증후군
%\item[\dsmedical] 
\end{itemize}
}
{}
\includegraphics[scale=.75]{labamenorrhea}\\
\begin{enumerate}\tightlist
\item Step 1: Rule out pregnancy 
	\begin{itemize}\tightlist
	\item Serum b-HCG: most sensitive 
	\end{itemize}
\item Step 2: Past medical history 
	\begin{itemize}\tightlist
	\item Stress, weight change, diet, illness, medication, acne, hirsutism, deepening of voice, headaches, visual field defects, polyuria, polydipsia, fatigue, hot flashes, vaginal dryness, poor sleep, decreased libido, galactorrhea, irregular menses, severe bleeding, dilatation and curettage, endometritis, etc. 
	\end{itemize}
\item Step 3: Physical examination 
	\begin{itemize}\tightlist
	\item Height/weight, hirsutism, acne, stria, acanthosis nigricans, vitiligo, breast exam, vulvovaginal exam, parotid gland swelling, erosion of dental enamel
	\end{itemize}
\item Step 4: basic lab test
	\begin{itemize}\tightlist
	\item TFT, serum prolactin
	\end{itemize}
\item Step 5: follow up testing 
	\begin{itemize}\tightlist
	\item Progesterone challenge test -> detects endogenous estrogen
		\begin{itemize}\tightlist
		\item Withdrawal bleed (+): adequate estrogen
		\item Withdrawal bleed (-): outflow tract 이상, inadequate estrogen
		\end{itemize}
	\item Estrogen/Progesterone challenge test
		\begin{itemize}\tightlist
		\item Withdrawal bleed (+): H-P-O axis or ovaries 이상 
		\item Withdrawal bleed (-): outflow tract 이상 
		\end{itemize}
	\item FSH/LH 
		\begin{itemize}\tightlist
		\item Elevated FSH/LH: ovarian abnormality (hypergonadotropic hypogonadism)
		\item Normal/Low FSH/LH: pituitary or hypothalamic abnormality (hypogonadotropic hypogonadism) 
		\end{itemize}
	\item MRI 
	\end{itemize}
\end{enumerate}
\hspace{-1cm}
\includegraphics[scale=.75]{progesteronCT}\\
\begin{mdframed}[linecolor=blue,middlelinewidth=2]
Progesterone challenge test: withdrawal bleeding occurs within 2-7 days 
\end{mdframed}
\subsection{SUMMARY}
\begin{itemize}\tightlist
\item Hx. Taking
	\begin{itemize}\tightlist
	\item Dietary and exercise histories
	\item Attitudes toward eating and body image
	\item Psychosocial stressors
	\end{itemize}
\item Physical examination 
	\begin{itemize}\tightlist
	\item Physical stigmata of a chronic disease or self induced vomiting
	\item Pelvic examination should assess estrogen status and rule out abnormalities.
	\end{itemize}
\item LAB
	\begin{itemize}\tightlist
	\item TFT, prolactin, and FSH. 
	\end{itemize}
\item Imaging
	\begin{itemize}\tightlist
	\item Plain radiographs to look for possible stress fracture
	\item Bone density
	\item Brain MRI would not be indicated: neurologic symptoms (-), other evidence to suggest hypothalamic or pituitary dysfunction (-)
	\end{itemize}
\item If no other cause of amenorrhea is identified, the patient should be educated regarding \emph{the effect of excessive exercise and weight loss on menstrual cycles and the risks of associated bone loss}
\item Documentation of a stress fracture would warrant temporary cessation of or a marked reduction in exercise
Some reduction should be recommended in any case, since such a cutback in exercise and adequate caloric intake are likely to result in a resumption of menses
\item Consultation with a nutritionist and mental health provider
\item F/u: Nutritional intake, exercise levels, menstrual periods
\item An oral contraceptive pill should not be provided for the purpose of improving bone density.
\end{itemize}
