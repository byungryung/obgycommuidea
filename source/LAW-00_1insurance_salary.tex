\subsection{\newindex{건강보험요양급여비용}}
\begin{mdframed}[linecolor=blue,middlelinewidth=2]
제1편 행위 급여 \cntrdot{}  비급여 목록 및 급여 상대가치점수 >> 제1부 행위 급여 일반원칙 >> Ⅰ. 일반기준
\end{mdframed}

\paragraph{I.일반기준}
\begin{enumerate}[1.]\tightlist
\item 요양기관이 국민건강보험법령의 규정에 의한 \uline{요양급여를 실시하고 행위에 대한 비용을 산정할 때에는} 제2부 각 장에 분류된 분류항목의 \mycoloredbox{상대가치 점수}(이하 “점수”라 한다)에 국민건강보험법 제45조제3항과 같은 법 시행령 제21조제1항에 따라 \uline{정하여진 점수당 단가}(제16장에 분류된 항목은 「지역보건법」에 따른 보건소\cntrdot{} 보건의료원 및 보건지소와 「농어촌등 보건의료를 위한 특별조치법」에 따라 설치된 보건진료소의 점수당 단가)\uline{를 곱하여 10원 미만은 4사5입한 금액으로 산정한다.} 다만, 요양기관 종별가산율에 의하여 산출된 금액에 대하여는 원미만을 4사5입한다.
\item 각종 가감률에 의하여 산출된 금액에 대하여는 1호와 동일한 방법으로 산정하되 상대가치점수에 가감률을 곱하여 총 점수(소수점 이하 셋째 자리에서 4사5입)를 산출하고, 각종 가감률이 복합 적용될 경우에는 가감률을 모두 합한 총 가감률을 상대가치점수에 곱하여 총 점수(소수점 이하 셋째 자리에서 4사5입)를 산출한다. 이 경우 가감률이 중복 적용될 경우에는 중복 가산하지 아니한다.
\item 제2부 각 장에 \uline{분류되지 아니한 진찰\cntrdot{} 처치\cntrdot{} 수술 및 기타의 치료에 대한 요양급여를 실시한 경우에 우선적으로 행위의 내용\cntrdot{} 성격과 상대 가치점수가 가장 유사한 분류항목에 준용하여 산정하여야 한다.(준용산정)}
\item 상급종합병원, 종합병원, 병원, 요양병원(의과), 의원, 보건의료원(의과), 의과 진료과목이 있는 한방병원\cntrdot{}치과병원인 요양기관은 제2부 제1장 내지 제9장 및 제16장, 제17장에 분류된 분류항목과 제10장에 분류된 분류항목 중 고정장치의 제거, 악간고정술, 치간고정술, 순열수술후
보호장치, 상고정장치술, 구강내소염수술, 구강외소염수술, 구강내열상 봉합술, 구강외열상봉합술, 협순소대성형술, 악골수염수술, 악골내고정용 금속제거술에 한하여 산정한다
\item 치과병원, 치과의원, 보건의료원(치과), 치과 진료과목이 있는 상급 종합병원 ㆍ종합병원ㆍ 병원
ㆍ한방병원인 요양기관은 제2부 제1장 내지 제10장과 제16장 내지 제18장에 분류된 항목에 한하여 산정한다.
\item 국립병원 한방진료부, 한방병원, 한의원, 보건의료원(한방과), 한방 진료 과목이 있는 상급종합병원\cntrdot{} 종합병원\cntrdot{} 병원\cntrdot{} 요양병원\cntrdot{}치과병원인 요양기관은 제2부 제1장, 제4장, 제13장, 제14장 및 제17장에 분류된 분류 항목에 한하여 산정한다.
\item 약국 및 한국희귀의약품센터인 요양기관은 제2부 제15장에 분류된 분류항목에 한하여 산정한다.
\item 조산원인 요양기관은 다음 분류항목에 한하여 산정한다.
	\begin{enumerate}[가.]\tightlist
	\item 제2부 제11장 및 제17장에 분류된 분류항목
	\item 제2부 제9장에 분류된 분류항목 중 자궁내장치삽입술 및 자궁내장치 제거료
	\item 기타 보건복지부장관(이하 “장관”이라 한다)이 불가피하다고 인정하는 경우
	\end{enumerate}
\item 보건소, 보건지소, 보건진료소인 요양기관은 다음 분류항목에 한하여 산정한다.
	\begin{enumerate}[가.]\tightlist
	\item 제2부 제12장에 분류된 분류항목
	\item 제2부 제9장에 분류된 정관절제술 또는 결찰술, 난관결찰술, 자궁내장 치삽입술,자궁내장치제거료
	\item 기타 장관이 불가피하다고 인정하는 경우
	\end{enumerate}
\item 의료법 제35조에 의한 부속 의료기관은 다음 분류항목에 한하여 산정한다.
	\begin{enumerate}[가.]\tightlist
	\item 제2부 제1장 재진진찰료, 의약품관리료, 혈액관리료
	\item 제2부 제4장 퇴장방지의약품 사용장려비
	\item 제2부 제5장, 제9장, 제10장, 제13장, 제14장 및 제16장에 분류된 분류항목
	\end{enumerate}	
\end{enumerate}

\paragraph{II. \newindex{요양기관 종별가산율}}
\begin{enumerate}[1.]\tightlist
\item 제2부 제2장 내지 제10장, 제13장 및 제14장에 분류된 분류항목에 \uline{대하여는 소정점수에 점수당 단가를 곱한 금액을 모두 합산한 금액에 요양기관의 종별에 따라 다음 각 호의 비율을 가산한다.}
	\begin{enumerate}[가.]\tightlist
	\item \uline{다음 각 항의 요양기관은 30\%}
		\begin{enumerate}[(1)]\tightlist
		\item 상급종합병원으로 인정받은 종합병원
		\item 상급종합병원에 설치된 치과대학 부속 치과병원
		\item 상급종합병원에 설치된 특수전문병원
		\item 의료법 제35조에 의한 부속 의료기관
		\end{enumerate}
	\item \uline{다음 각 항의 요양기관은 25\%}
		\begin{enumerate}[(1)]\tightlist
		\item 상급종합병원을 제외한 종합병원
		\item 상급종합병원에 설치된 경우를 제외한 치과대학 부속 치과병원
		\item 허가 병상 수가 30병상 이상이고, 한방 6개 과가 설치되어 있는 한의과대학 부속 한방병원
		\item 국립병원 한방진료부
		\item 의료법 제35조에 의한 부속 의료기관
		\end{enumerate}
	\item \uline{다음 각 항의 요양기관은 20\%}
		\begin{enumerate}[(1)]\tightlist
		\item 병원
		\item 위 “가-⑵” 또는 “나-⑵”에 해당되지 아니하는 치과병원
		\item 위 “나-⑶"에 해당되지 아니하는 한방병원
		\item 요양병원
		\item 의료법 제35조에 의한 부속 의료기관
		\end{enumerate}
	\item \uline{다음 각 항의 요양기관은 15\%}
		\begin{enumerate}[(1)]\tightlist
		\item \uline{의원}
		\item 치과의원
		\item 한의원
		\item 보건의료원
		\item 의료법 제35조에 의한 부속 의료기관
		\end{enumerate}		
	\item 다음 각 항의 요양기관은 종별가산율을 적용하지 아니한다.
		\begin{enumerate}[(1)]\tightlist
		\item 약국 및 한국희귀의약품센터 
		\item 조산원, 보건소, 보건지소, 보건진료소
		\item 의료법 제35조에 의한 부속 의료기관
		\end{enumerate}
	\end{enumerate}
\item 위 “1"의 규정에도 불구하고 \uline{아래 항목에 대해서는 요양기관 종별 가산율을 적용하지 아니한다.}
	\begin{enumerate}[가.]\tightlist
	\item 바이러스 혈청검사(나-476, C4760)
	\item 각 장의 산정지침 또는 분류항목의 “주"에서 별도로 산정할 수 있도록 규정한 약제비, 치료재료대 등
	\item 영상저장 및 전송시스템 (Full PACS)(GB011-GB045, HB011-HB041, HG011-HG045, HG111 -HG141)을 이용한 처리비용, C-Arm형 영상 증폭장치 이용료(다-101, G0400)
	\item 생혈(마-103, X3010), 교환(마-104, X4000), 조혈모세포의 이식 준비 -냉동 처리 및 보관(마-105-다-(1), X5020), 기증제대혈제제 비용 (마-105-라-(3)-가, X5137), 자가수혈채혈료(마-106-가, X6001 내지
X6008), 연성신요관경하 요관협착확장술 “주"(자-319-3 “주", R3196),연성신요관경하 결석제거술 “주3"(자-321-3 “주3", R3429)
	\item 퇴장방지의약품 사용장려비
	\item 검체검사 위탁에 관한 기준에서 정한 수탁기관으로 위탁하는 경우의
검사료 및 위탁검사관리료
	\item \uline{Infusion Pump 사용료(KK058, KK158)}
	\item \uline{마취통증의학과 전문의 초빙료(L7990)}
	\end{enumerate}
\item 위 “1-나" 항의 종별가산율을 적용받은 종합병원이 의료법 제3조의3 기준에 부적합한 경우에는 3월 이내의 범위 내에서 기간을 정하여 시정 하도록 하고 동 시정기간 내에 시정하지 아니한 때에는 시정기간 종료 익일부터는 위 “1-다"항의 종별가산율을 적용한다
\end{enumerate}

\paragraph{Ⅲ. \newindex{차등수가}}
의과의원, 치과의원, 한의원, 보건의료원, 약국 및 한국희귀의약품센터의 경우에는 \uline{의사, 치과의사, 한의사, 약사 1인당 1일 진찰횟수, 약국 및 한국 희귀의약품센터의 경우에는 조제건수(처방전 매수를 말한다.} 이하 같다)에 따라서 요양기관에 진찰료와 조제료 등(조제료, 약국관리료, 조제기본료, 복약지도료를 말한다. 이하 같다)을 아래와 같이 차등지급한다. 다만, 의료 급여 환자, 장관이 별도로 정한 평일 18시(토요일은 13시)-익일 09시의 진찰료와 조제료 등, 기타 장관이 별도로 정하는 경우에는 차등수가 적용 대상에서 제외할 수 있다.\par
\begin{enumerate}[가.]\tightlist
\item \uline{의과의원, 치과의원, 한의원, 보건의료원의 의사, 치과의사, 한의사 1인당 1일 진찰횟수를 기준으로 진찰료에 대하여 다음과 같이 차등지급한다.}
	\begin{enumerate}[(1)]
	\item \uline{75건 이하:100\%}
	\item \uline{75건을 초과하여 100건까지:90\%}
	\item \uline{100건을 초과하여 150건까지:75\%}
	\item \uline{150건을 초과한 건:50\%}
	\end{enumerate}
\item 약국 및 한국희귀의약품센터의 약사 1인당 1일 조제건수(의약분업 예외 지역에서는 직접조제건수 포함)를 기준으로 조제료 등에 대하여 다음과 같이 차등지급한다.
	\begin{enumerate}[(1)]
	\item 75건 이하:100\%
	\item 75건을 초과하여 100건까지:90\%
	\item 100건을 초과하여 150건까지:75\%
	\item 150건을 초과한 건:50\%
	\end{enumerate}
\begin{mdframed}[linecolor=blue,middlelinewidth=2]
\Large{2015년 12월 차등수가제 폐지(의원급만)}\par
\bigskip

\Large{2016년 전문병원 의료 질 지원금(입원일당 1820원, 29억원 규모)과 전문병원 관리료(3개 분야 차등지원, 70억원 규모) 신설}\par
전문병원 관리료의 경우, ▲척추(한방 포함)와 관절, 대장항문은 790원(입원 1일당) ▲화상과 수지접합, 심장, 알코올, 유방, 주산기, 뇌혈관, 산부인과, 신경과, 안과, 외과, 이비인후과,재활의학과, 한방 중풍은 1980원 등이다.\par
\uline{암환자 교육상담료 신설과} 위험분담제 `피레스파정'(특발성 폐섬유증 치료제) 급여 적용, 당뇨병 환자 소모품 확대(인슐린 투여 환자, 채혈침, 인슐린 주사기, 펜인슐린바늘)
\end{mdframed}

\medskip
\tabulinesep =_2mm^2mm
\begin {tabu} to\linewidth {|X[2,l]|X[1,l]|X[2,l]|} \tabucline[.5pt]{-}
\rowcolor{ForestGreen!40}  \centering 전문과목\cntrdot{}질환 & \centering 금액(입원1일당) & \centering 비고 \\ \tabucline[.5pt]{-}
\rowcolor{Yellow!40} 척추(한방포함), 관절, 대장 항문 분야 & 790원 & 비급여 모니터링 결과 연계 조정 \\ \tabucline[.5pt]{-}
\rowcolor{Yellow!40} 화상, 수지접합, 심장, 알코올, 유방, 주산기, 뇌혈관, 산부인과, 신경과 안과, 외과, 이비인후과, 재활의학과, 한방중풍 분야 & 1,980원 & 평균 재원일수가 짧은 분야(안과, 이비인후과)는 외래환자수가(390원)신설 \\ \tabucline[.5pt]{-}
\rowcolor{Yellow!40} 수지접합, 알코올, 화상, 재활의학, 뇌혈관, 주산기, 유방, 심장 & 390원 가산(20\%) & 사회적 필요 서비스 분 \\ \tabucline[.5pt]{-}
\end{tabu}

\medskip
 
\item 차등지급되는 진찰료(약국 및 한국희귀의약품센터의 경우에는 조제료 등을 말한다)는 차등지수에 1개월(또는 1주일)간 총 진찰료를 승하여 산출하되 10원 미만은 4사5입한 금액으로 산출하며 차등지수는 의사, 치과의사, 한의사, 약사 1인당 1일평균 진찰횟수(약사의 경우에는 조제 건수)를 n으로 할 때에 다음과 같이 산정하되 소수점 여덟째 자리에서 4사5입 한다.
	\begin{enumerate}[(1)]
	\item n이 75 이하일 경우에는 차등지수를 1로 한다.
	\item n이 75를 초과하여 100 이하일 경우에는 \{75×1.00 + (n-75)×0.90\}/n
	\item n이 100을 초과하여 150 이하일 경우에는 \{75×1.00 + 25×0.90 + (n-100)×0.75 \}/n
	\item n이 150을 초과하는 경우에는 \{75×1.00 + 25×0.90 + 50×0.75 + (n-150)×0.50 \}/n
	\end{enumerate}
\item 의사, 치과의사, 한의사 1인당 1일 평균 진찰횟수, 약사 \uline{1인당 1일 평균 조제건수는 내원환자의 순서 및 초\cntrdot{} 재진을 구분하지 아니하고 1개월 (또는 1주일)간 총 진찰(조제)횟수의 합을 구하고} 이를 해당 요양기관이
국민건강보험법 시행규칙 제12조제1항 및 제2항의 규정에 의하여통보한 의사, 치과의사, 한의사가 진료한 총일수, 약국 및 한국희귀 의약품센터의 약사가 조제한 \uline{총일수로 나누어서 계산하되} 소수점 첫째 자리에서 절사하여 산정한다.
\item \uline{진료(조제)일수는 1개월(또는 1주일) 동안 의사(약사)가 실제 진료(조제)한 날수를 말한다.}
\end{enumerate}

\paragraph{Ⅳ. 예외규정}
1. 의료법 제35조에 의한 부속 의료기관은 해당 산정항목에 대하여 공휴\cntrdot{} 야간 가산 등 각종 가산을 산정하지 아니한다.
2. 공무원 및 교직원의 공무상 질병 또는 부상에 대한 요양급여에 소요된 비용의 산정은 산업재해보상보험법 제40조제5항의 규정에 의한 기준에 의한다.
  
\begin{mdframed}[linecolor=blue,middlelinewidth=2]
건강보험요양급여비용제1편 행위 급여 \cntrdot{}  비급여 목록 및 급여 상대가치점수 >> 제2부 행위 급여 목록 상대가치점수및  산정지침 >>  제1장 기본진료료
\end{mdframed}  