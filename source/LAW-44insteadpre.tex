\section{대리처방}
\leftrod{보건복지부 고시(제2013-192호)}
\begin{itemize}\tightlist
\item 대리처방사유 : 재진으로 아래의 사유로 받는 경우 인정
	\begin{itemize}[▲]\tightlist
	\item 동일 상병
	\item 장기간 동일 처방
	\item 환자 거동 불능 
	\item 주치의가 인정하는 경우에만 처방전 대리 수령과 방문당 수가 산정(재진진찰료 소정점수의 50\%)을 인정”
	\end{itemize}
\end{itemize}	

\subsection{대리처방에 대한 안내문}
\begin{figure}
\centering
\includegraphics[scale=.5]{medication}
\end{figure}
\Que{대리처방코드는?}
\Ans{AA254090}
\Que{야즈등 대리처방가능한가요?}
\Ans{만성질환에서 진찰이 크게 필요없고, 같은 약을 repeat하는 경우에서 환자가 거동이 불편하여 내방이 불편하여  힘든경우는 가족중에서 되는걸로 알고 있습니다. 고로 초진은 안되고 재진시에만 가능함으로 야즈등의 처방시에는 안됩니다.}
%\begin{commentbox}{}
\Que{아버지가 구치소수감되어서 아버지 약처방을 대리처방받겠다 하는데 구치소 수감되어 있는지... 확인을 어찌하나요? 확인되면  처방되는지요?}
\Ans{안됩니다. \par
구치소안에도 의사들이 있고, 함부로 약물 반입이 되지 않는것으로 알고 있습니다.}
%\end{commentbox}
\Que{부득이한 대리처방시에 청구는 어떻게하나요?  구치소계신분이고 증명서해오셨길래 급여해드렸는데 혹시 청구가 안되거나 하진않겠지요?  따로 코드를 넣거나해야하나요?}
\Ans{위의 3가지 조건이 맞으면 4항의 주치의 인정으로 가능할 것 같습니다, 접수때 대리처방으로 접수하면 됩니다\par
교정시설에 도서, 의약품의 반입도 가능합니다. 다만, 반입 절차 등 구체적인 사항은 사전에 수용기관 민원실로 문의하시기 바랍니다.라는 문구로 봐서는 구치소에서 어떠한 요구가 있을떄만 가능 할것 같습니다.}