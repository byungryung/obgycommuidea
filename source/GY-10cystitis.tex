\section{방광염}
\myde{}{
\begin{itemize}\tightlist
\item[\dsjuridical] N300 급성 방광염
\item[\dsjuridical] R300 배뇨통
\item[\dschemical] B0030 요일반검사10종까지
\item[\dschemical] B0041 요침사현미경검사
\item[\dschemical] B0043 요침사검사[유세포분석법]
%2017 년 9 월부터는
% 
%균자라든 안자라든
%나 -407 미생물 ( 항산균제외 ) 배양 , 동정 , 약제감수성검사
%Microrganism(except AFB) Culture, Identification and
%Antibiotics Sensitivity Test
\item[\dschemical] B413 (3) 비뇨기, 생식기검체 Urogenital Specimens \myexplfn{161.20} 원( Microrganism(except AFB) Culture, Identification and
Antibiotics Sensitivity Test)
\item[\dschemical] B4143 (3) 비뇨기, 생식기검체 Urogenital Specimens \myexplfn{184.12} 원( Microrganism Culture, Identification and
Antibiotics Sensitivity Test)
%\item[\dschemical] B4051 미생물배양 및 동정검사
%\item[\dschemical] B4061 미생물약제감수성 검사(디스크확산법) 
\item[\dsmedical] 비마약성 진통소염제 코리돌 주등의 전문비보험약물 \footnote{\pageref{analgesia} 쪽 참조}
\end{itemize}
}{
\begin{itemize}\tightlist
\item 일반적인 UA with Micro인 경우는 B0030+B0041
\item 충분한 항생제치료에도 재발되는 방광염시에 B4051+B4061
\item 2017 년 9 월부터는 균자라든 안자라든 나-407 미생물 ( 항산균제외 ) 배양 , 동정 , 약제감수성검사 Microrganism(except AFB) Culture, Identification and
Antibiotics Sensitivity Test 가능 
%B4061은 B4051에서 균이 검출되었을때 검사하세요. 즉 약제 감수성 검사는 같은날 하시면 삭감되시고 검사결과 나올때 청구하십시요. 진찰료 표시를 하지 않고 검사만 청구해야 합니다. [환자가 방문하지 않았으므로]
\item 가끔씩 심평원에서 B0030으로 청구하면 B0020 요일반검사7종으로 하라면서 삭감하는 경우도 있음. : 현재까진 방법이 없음. 왜 이렇게 삭감하는지 알수없음. 워낙 적은 금액으로 대개 별다른 대처를 하지 않음.
\end{itemize}
}
\subsection{방광염시 플루오로퀴놀론계 약물 사용}
Ciprofloxacin :이약을 포함한 플루오로퀴놀론계 약물은 중대한 이상반응과 관련이 있으므로 급성세균성부비동염(축농증), 만성기관지염의 급성세균악화 및 \emph{단순요로감염은 다른 치료방법이 없는 환자에게 사용}한다.\par
핵심은 다른 치료방법이 없는 환자입니다. 1차로 타항생제 처방하고 안들어서 2차로 사용하는게 아닙니다.\par
금일부(2017년 1월 6일)로 시행을 하나 심평원에서 유예를 하겠다고 합니다.\par
다음은 비뇨기과내 공지사항입니다.
\begin{enumerate}\tightlist
\item 앞으로 플루오로퀴놀론은 단순요로감염에는 다른약 못쓰는 상황에만 쓰는게 인정됩니다. 약물 부작용 때문에 전세계적으로 적응증 제한이 생기는 추세여서 학술적 반박이 무리인 상황입니다. 일단 고시는 1월 6일부터 적용입니다.
다음은 비뇨기과밴드에 올린 내용입니다.
\item 일단 다빈도질환별로 대체할 수 있는 항생제를 질환별로 공지드리며 추가로 대체 가능한 항생제 단독 및 복합요법에 대한 제보를 부탁드립니다.
\item 방광염을 포함한 단순 요로 감염   - 오구멘틴 / 세파계 항생제
\item STD   - 독시사이클린 + 세파계 항생제(1세대 or 2세대) / 아지스로마이신
\end{enumerate}

%\begin{enumerate}\tightlist
%\item 좀더 중질환 상병이(배제상병, 의증) 필요합니다.(N730 PID -N739).
%\item 다른 의원에서 1차약 사용하고 치료가 안되서 바꾼다는 청구메모
%\item 시프로사신으로 처방한다. 약도 더 잘듣고,삭감도 안된다고 합니다.
%\end{enumerate}
\subsection{IV or IM 약물사용}
\begin{enumerate}\tightlist
\item Ribostamycin,겐타마이신 - 보험으로 사용가능
\item Urine culture, MIC test 해보면 GM, tobra, amikacin주사 사용 가능하다고 함.
\end{enumerate}

\subsection{모누롤산}
방광염, 요도염에 하루한번 물에 타서 먹는약. 대개 2팩 내구요.  오늘 집에가서  바로 한번 물반컵에 타마시고 모레 저녁에 자기전 배뇨후 타 마시라고 해요.

\subsection{유로박솜}
\myde{}{
\begin{itemize}\tightlist
\item[\dsjuridical] N302 기타 만성 방광염(Other chronic Cystitis)
\end{itemize}
}{\begin{itemize}\tightlist
\item (경구 : 캅셀제)
\item 성인 : 동결 건조균체 용해물로서 1일 1회 60mg을 아침 공복시 소량의 물과 함께 3        개월간 경구 투여한다.
\item 급성 증상 발현 시, 항생물질과 병용하여 적어도 10일 이상 투여한다.
\item 연령, 증상에 따라 적절히 증감한다.
\end{itemize}}

\begin{commentbox}{유로박솜 보험 몇 달 정도 되는가요?}
1일 1회 60mg을 아침 공복시 소량의 물과 함께 3 개월간입니다
\end{commentbox}