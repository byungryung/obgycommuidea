\section{VIN }
\myde{}{%
\begin{itemize}\tightlist
\item[\dsjuridical] C51외음의 악성 신생물Malignant neoplasm of vulva
\item[\dsjuridical] C51.0 대음순Labium majus
\item[\dsjuridical] C51.8 외음의 중복병변 Overlapping lesion of vulva
\item[\dsjuridical] C51.9 상세불명의 외음Vulva, unspecified 
\item[\dsmedical] EB455010 일반/도플러 병원 \myexplfn{953.44} 원
\item[\dsmedical] EB457010 정밀/도플러 병원 \myexplfn{1397.03} 원
\item[\dsmedical] C8506 나-850 침생검 Nedle Aspiration Biopsy
\item[\dsmedical] C5602 조직병리검사 [1장기당] -(Level B)
\item[\dsmedical] E7721  질확대경검사(단순) or E7722 질확대경검사(자궁내구경사용)
%\item[\dsmedical] 포로포폴 사용한 경우 : 30분 초과 2시간 이내의 경우 보험청구 그이외는 약값 전액을 환자가 부담(100/100) 
\item[\dsmedical] R4300 자궁경부(질)약물소작술 Chemical Cauterization of Cervix(Vagina) [\myexplfn{228.47}원] (LEEP당일0.5배 청구.추적조사중 정상청구가능) 알보칠 비보험청구
%\item[\dsmedical] 포폴(Propofol) (100대100)
%\item[\dsmedical] L0101	정맥마취(전신마취) or LA271 pudendal block* 
%\item[\dsmedical] 영양제 (비급여)
\item[\dsmedical] KK059	정맥내유치침
\item[\dschemical] 소수술시 인정되는 Lab.
\item[\dsmedical] 리도카인 액(사용한 만큼)
\end{itemize}
%* 참고에 paracervical block 하였는데 이와 유사한 pudendal block 으로 청구한다고 쓰면 대부분 지급해 줍니다
}
{%\begin{enumerate}\tightlist
%\item CIN III = severe dysplasia(N872) = CIS (D06) 입니다. 
%\item 제가 알기로는 소송결과로 지급해 줘야 되는데 진단서에 n872라거 적으면 원장님 때문에 지급안된다고 하면서 지급 안해줍니다. 이제는 n872 코드는 쓰지 마시고 D06코드로 쓰세요.
%\end{enumerate}
}
%\subsection{산정특례 등록제도는 무엇이며, 등록에 대한 혜택은 무엇인가? 또 등록일로부터 5년이 경과한 이후에도 산정특례 적용이 가능한가?}
%진료비 본인 부담이 높은 암 등 중증질환자와 희귀난치성질환자, 중증화상환자에 대해 건강보험의 보장성을 높여 질병으로 인한 빈곤층 전락을 방지함으로써 사회안전망으로서의 건강보험 역할을 강화하고 한정된 재원을 효율적으로 관리하기 %위해 시행하고 있다. 암등 중증질환\bullet희귀난치성질환\bullet중증화상질환으로 확진된 경우 의사가 발행한 '건강보험 산정특례 등록신청서'를 공단에 직접 제출하거나 전자문서시스템을 활용하는 요양기관에 제출해 등록할 수 있다. %암\bullet희귀난치성질환의 등록은 5년, 중증화상질환 등록은 1년간 산정특례 혜택을 받을 수 있다. 
%\highlight{대상 상병의 외래 및 입원 진료 때} 암과 중증화상의 경우는 요양급여비용의 100분의 5, 희귀난치성질환의 경우는 100분의 10을 본인이 일부 부담하면 된다. 아울러 심장 및 뇌혈관질환의 경우에는 별도의 등록 절차는 필요하지 %않으며 입원해 수술을 받는 경우 1회 수술당 최대 30일까지 혜택을 받을 수 있다.\par
%산정특례 대상 상병 및 관련 합병증에 대한 진료는 특례대상으로 경감적용을 받을 수 있다. 그러나 \uline{이와 상관없는 다른 상병이나 기왕증에 의한 진료분은 해당되지 않는다.} 다만, 동일 진료과목(입원)의 동일 의사(외래)에게 해당 %상병과 동시에 진료를 받는 경우 산정특례 적용을 받는다.\par
%한편 등록일로부터 5년이 경과하면 원칙적으로 산정특례 적용은 종료된다. 다만, 5년 종료시점에 암 조직이 있거나(잔존암, 전이암이 있거나 추가로 재발이 확인된 경우), 방사선ㆍ항암치료 호르몬을 받고 있는 자는 신규 등록 신청을 함으로써 %산정특례를 다시 적용받을 수 있다. 그러나 중증화상질환 등록자의 경우 산정특례 적용종료기간(1년) 경과전에 담당의사의 소견서를 제출해 연장을 요청할 경우 최초 산정특례 등록했을 때 종료일로부터 6개월 연장이 가능하다. 
