\section{양성신생물}
\tabulinesep =_2mm^2mm
\begin {tabu} to\linewidth {|X[1,l]|X[1,l]|} \tabucline[.5pt]{-}
\rowcolor{ForestGreen!40} 적합상병 &	청구항목   \\ \tabucline[.5pt]{-}
\rowcolor{Yellow!40} 외음부일경우: Soft Fibroma(=Achrocodon,Skin tag(쥐젖)) \newline D289 상세불명의 여성생식기의 양성신생물 & 외음부종양적출술-양성:R4066(\myexplfn{1460.86} 원) \newline 리도카인 \newline 병리조직검사[1장기당]-생검(1-3개):C5911(\myexplfn{282.35} 원)   \\ \tabucline[.5pt]{-}
\rowcolor{Yellow!40} 외이도일 경우 \newline D232 귀 및 외이도의 양성 신생물 & 외이도종양적출술-양성:S5591(\myexplfn{2612.89} 원) \newline 리도카인 \newline 병리조직검사[1장기당]-생검(1-3개):C5911(\myexplfn{282.35} 원)   \\ \tabucline[.5pt]{-}
\rowcolor{Yellow!40} 기타 부위일때 \newline D179 상세불명부위의 지방종 \newline D1809 상세불명 부위의 혈관종 \newline D280 상세 불명 피부 양성 신생물 & 피부양성종양적출술(간단한표재성):N0141(\myexplfn{484.99} 원) \newline 리도카인 \newline 병리조직검사[1장기당]-생검(1-3개):C5911(\myexplfn{282.35} 원)   \\ \tabucline[.5pt]{-}
\rowcolor{Yellow!40} L84 티눈 및 굳은살인 경우 &	사마귀제거술(자14-1 티눈제거술 준용)은 근접하고 있는 2개 이상을 동시에 제거하는 경우에는 제1의 것은 100\%, 제2의 것부터는 50\%를 산정하되 최대 200\%를 산정하고, 손(발)가락(손등 포함)과 손(발)바닥은 타 부위로 간주하여 소정금액을 각각 산정하고 있음(수부 또는 족부 기준 편측당 최대 400\% 산정). (2001.1.1 시행) \newline 티눈제거술(전기소작,냉동응고술 또는 약물:N0143(\myexplfn{347.85} 원)(편측당 최대 400\% 산정)\newline
리도카인   \\ \tabucline[.5pt]{-}
\rowcolor{Yellow!40} 전염성연속종(molluscum contagiosum )인 경우는 \newline B081전염성 물렁종 & 	전염성연속종제거술(기타의 것):N0148(\myexplfn{189.23} 원) \newline 전염성연속종제거술(전신성인 것):N0147(\myexplfn{254.86} 원)   \\ \tabucline[.5pt]{-}
\end{tabu}
\par
\medskip
%\clearpage
\section{술후 granulation}
\myde{}{%
\begin{itemize}\tightlist
\item[\dsjuridical] L923 피부 및 피하조직의 이물육아종	Foreign body granuloma of skin and subcutaneous tissue
\item[\dsmedical] R4300 자궁경부(질)약물소작술 Chemical Cauterization of Cervix(Vagina) [\myexplfn{228.47}원]
\item[\dsmedical] 알보칠 비보험 
%\item[\dsmedical] D289 상세불명의 여성생식기의 양성신생물 
%\item[\dsmedical] R4066 외음부 양성종양 적출술[\myexplfn{1460.86} 원]
%\item[\dsmedical] D281 질의 양성신생물
%\item[\dsmedical] R4070 질종양적출술 [\myexplfn{1382.33}원]
\end{itemize}
청구메모>> ``수술후 질전정부에 granulation tissue생겨 알보칠 이용하여 cauterization함"
} {
vaginal vault granulation
	\begin{enumerate}\tightlist
	\item D289 상세불명의 여성생식기의 양성신생물 - R4066 외음부 양성종양 적출술[\myexplfn{1460.86} 원]
	\item D281 질의 양성신생물 - \uline{R4070 질종양적출술} [\myexplfn{1382.33}원] 를 사용하자는 의견이 나왔습니다.
	\end{enumerate}
}