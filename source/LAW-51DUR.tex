\section{DUR 예외사항}
\Que{환자가 별다른 사유없이 조제약을 먹고 구토 증상을 나타내 다시 약을 처방·조제해야 할 경우, 심사평가원에는 어떻게 청구해야 할까?}
\Ans{의원\cntrdot{}약국 등 요양기관 현장에서 벌어지는 돌발적인 상황에 맞춰 처방·조제지원서비스(DUR)도 불가피한 상황마다 대처할 수 있도록 예외 인정 유형과 코드가 있다. \par
22일 심사평가원에서 예외를 인정하는 13가지 유형과 코드를 살펴보면, 먼저 중복처방의 경우 총 3가지(A-C) 유형이 있다. 환자가 \emph{장기출장이나 여행} 등으로 기존에 처방\cntrdot{}조제받았던 약을 다 복용하지 않았지만 추가로 처방\cntrdot{}조제 받아야 하는 경우가 이에 해당된다. 또 파우더 형태 조제 등으로 기존 약 중 특정 성분만 구분해 따로 처방할 수 없는 경우, 환자가 실수나 별다른 이유 없이 \emph{약을 먹다가 구토} 등으로 약이 소실\cntrdot{}변질돼 다시 처방\cntrdot{}조제 받아야 하는 경우도 포함된다.\par
이와 함께 중복\cntrdot{}병용\cntrdot{}임부 등 DUR 점검을 받는 약제들 중 공통적으로 예외가 인정되는 경우는 총 8가지(F-L, P)다. 대표적으로는 ▲처방일과 투약일이 다른 경우 ▲주\cntrdot{}월 단위 투약하는 약제, 환자 임의로 기존 처방약을 먹지 않은 경우 ▲의사가 기존 약을 먹지 못하게 한 경우, 처방한 의사 또는 조제한 약사와 전화통화나 연락이 되지 않은 경우 등 해당된다. \par
이 밖에 DUR 창에 텍스트로 보고하는 유형도 2가지 있다. 병용\cntrdot{}연령\cntrdot{}임부금기에 해당되지만 불가피하게 임상적 치료 목적으로 환자 동의 하에 약제를 투약하는 경우, 중복처방 유형에 해당되진 않지만 중복 처방해야 할 경우는 청구할 때 각각의 사유를 적으면 된다. }

\par
\medskip
\begin{center}
\includegraphics{DUR.jpg}   
\end{center}