\begin{myshadowbox}
\begin{enumerate}[4.]\tightlist
\item 복강경을 이용한 수술 중 부득이한 사유로 중도에 개복술로 전환하여 수술을 종결한 경우에는 복강경을 이용하지 아니한 질병군에 해당하는 소정점수를 적용하고 복강경 등 내시경하 수술시 보상하는 239,000원(100분의 20에 해당하는 47,800원은 본인부담)의 금액을 추가 산정한다.
\end{enumerate}
\end{myshadowbox}
\prezi{\clearpage}
\Que{복강경을 이용한 수술 중 부득이한 사유로 \textcolor{red}{중도}에 개복술로 전환하여 수술을 종결한 경우 “복강경 등 내시경하 수술시 보상하는 239,000원의 금액을 추가 산정함. 이 경우 본인일부부담금 산정특례 대상자의 본인부담률 적용 방법은?}
\Ans{본인일부부담금 산정특례 대상자와 차상위 본인부담 경감대상자 등은 각각의 해당 본인부담률을 적용하며, 본인부담률이 5\%인 경우에는 239,000의 5\%에 해당하는 금액인 11,950원을 본인이 부담함\par
☞ 「국민건강보험법시행령」 별표2 제3호에 해당하는 대상자인 경우에는 그 각목에서 정한 본인부담률을 적용}
\prezi{\clearpage}
\par
\medskip
\Que{복강경을 이용한 LAVH시도 할려고 봤더니 골반장기의 유착이 심해서 개복하여 TAH한 경우에 추가 산정 가능여부.}
\Ans{단지 봤다는 소견으로만은 추가산정이 불가하다고 합니다.  \textcolor{red}{“도중”} 이란 말의 뜻대로 뭔가를 하다가 안되었다는 차트의 기록이 필요하다고 합니다. \par 
"질병군 요양급여비용 모니터링"이란 방법을 통해서 위의 과정을 통제한다고 합니다.}