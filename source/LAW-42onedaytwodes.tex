\section{당일 2회 진료시 진찰료 산정기준은 ?}
다른 상병시 진찰료 인정
같은 상병으로 연속된 처치시 진찰료 인정 안됨

\Que{내과 입니다. 오전에 고혈압으로 진료후 같은날 오후에 복통으로 내원하여 진료하고 처방 했습니다. 보험 청구및 처방전  발행이 가능 한지요? 1일 2회 진료후 처방한 경우인데, 이에 대한 보험 급여 기준도 알려 주세요.}

\Ans{동일날에 다른 상병으로 요양기관에 2회 내원하여 진료를 받는 경우 요양급여비용 청구 가능여부와 처방전 발행이 가능한지 여부에 대한 문의관련하여 진찰료는 외래에서 환자를 진찰한 경우에 처방전 발행과는 관계없이 산정(건강보험요양 급여비용 제2부 행위 급여 목록ㆍ상대가치점수 및 산정지침 제1장 기본진료료 [산정지침]) 하도록 되어 있으며, 하나의 상병 치료 중 전혀 다른 상병이 발생하여 진찰을 한 경우에는 재진진찰료를 산정(보건복지부고시 제2001-40호, 2001.7.1시행)토록 하고 있습니다.  따라서 동일날 오전에 고혈압으로 진료 후 오후에 복통으로 내원하여 진료를 한 경우에는 재진진찰료를 산정할 수 있으며, 처방전도 발행 가능함을 알려드립니다. \par

다만 수진자가 내원하여 진찰한 결과, 진료상 같은 날에 일정시간 경과 후 연속된 처치 또는 치료(주사 또는 검사 등)를 필요로 하여 다시 내원하게 하였을 경우에는 초진 또는 재진진찰료는 1일 1회만 청구가능함을 알려드리오니 진료에 참고하시기 바랍니다.
}