\section{제왕절개 DRG관련 QA}
응급이나 야간에 수술을 하게 되면 응급.야간이 적용 된다고 들었습니다.그러면 재제왕절개술이나 둔위나 쌍태아일 경우는 선택적 제왕절개이지만 조기진통으로 입원을 할 수도 있고 조기진통으로 외래 진료 후 응급이나 야간에 수술을 할 수 있습니다.
이러한 경우에도 동일하게 응급이나 야간이 적용 되는지 궁금합니다.\par
또 \highlight{난관결찰은 비급여로 적용} 된다고 들었는데요.그러면 충수절제술도 비급여로 가능한지 궁금합니다.\par
마지막으로 흉터연고나 영양제, 질세정세,쾌변이나 자궁내 피임장치(미레나)등도 비급여로 적용되는지 궁금합니다.\par
유선으로 문의를 했을 때는 정확한 답변이 어렵다고 말씀 하셨는데요 . 곧 시행해야 하는 병원은 정확하게 알아야 환자나 환자보호자 분들께 명확한 답변을 드릴 수가 있습니다. 
\Ans{18시-09시 또는 공휴일에 응급진료가 불가피하여 수술을 행한 경우에는 야간\bullet공휴 소정점수를 추가 산정할 수 있습니다. 이 경우 수술 또는
마취를 시작한 시간을 기준으로 산정한다.\par
따라서 \uline{예정된 제왕절개수술이더라도 응급진료가 불가피하여 수술을 행한 경우에는 야간\bullet공휴 소정점수를 추가 산정}할 수 있습니다.\\
개복 수술시 \uline{질환이 없는 상태에서 시행한 충수절제술은 별도 산정할 수 없으며, 해당 비용은 질병군 수가산출시 제출한 자료에 근거하여 포괄수가에 포함되었습니다.}\\
제왕절개분만 질병군에서 환자가 원하여 \highlight{피임시술(난관결찰술(자-434, R4341-R4345) 및 자궁내장치삽입술(자427, R4271))을 실시한 경우에는 비급여로 수진자에게 별도로 받을 수 있으나}, 피임시술의 요양급여 인정기준에 의거 아래 요양급여대상에 해당되는 경우에는 질병군 상대가치점수에 포함되어 별도 산정할 수 없습니다. \\
영양제는 일상생활에 지장이 없는 단순한 피로 또는 권태에 투여시 비급여 대상이나 질병군 진료를 위해 입원하여 투여하는 경우에는 DRG 수가에 포함되어 있으므로 별도로 산정 할 수 없습니다. 또한 미용목적으로 처방하는 흉터연고는 비급여로 받을 수 있음을 알려드립니다}

2번째 반복 제왕절개 수술후 2일째 되는 날 CSF leakage로 blood patch 시행했습니다. 혹 의료보험이나 환자 부담으로 시술료 청구 가능한지요? 
\begin{quotebox}
시술료 청구는 안되고 O89.4 산후기중 척추및 경막외 마취 유발 두통 상병명 (중증도 점수2)
넣으시면  DRG 등급 한단계 올라갑니다.
\end{quotebox}

임신 중 가진통으로 확인되어 입원 치료 중(치료를 위한 재원일수에 무관) 갑자기 분만이 진행되어 제왕절개술을 시행하게 된 경우 가진통 기간중의 재원일 동안은 행위별로 청구, 분만이 진행되어진 시점부터 제왕절개술 분만 후 퇴원일까지의 재원일 동안은 포괄수가로 청구해야하는지, 아니면 모든 재원일을 합산하여 포괄수가로 청구해야하는지 문의드립니다.
\begin{quotebox}
2012년 7월 1일부터 질병군 진료 이외의 목적으로 입원하여 입원일수가 6일을 초과한 시점에 질병군 수술이 이루어진 경우 입원일로부터 수술시행일 전일까지의 진료분은 행위별로 청구하고 이후 수술일부터는 DRG로 청구하도록 행위\bullet질병군 분리청구 내용이 신설되었습니다.\\
따라서 임신 중 조기진통으로 입원하여 조산방지를 위한 치료 중 6일을 초과한 시점에 제왕절개술을 실시한 경우 입원일부터 제왕절개술 전일까지 행위별로 청구하고 제왕절개술 실시일부터 퇴원일까지 포괄수가제로 청구하시길 바랍니다. 
\end{quotebox}

\clearpage
\section{약품분실로 동일한 처방전 재발급}
\Que{약품 분실로 동일한 처방전 재발급을 위해 내원하였습니다. 이 경우 발생 한 진료비는 어떻게 청구해야 하나요?}
\Ans{환자가 약을 분실하여 다시 처방해야 하는 경우, 이미 수령한 약제를 분실한 것은 환자에게 귀책사유가 있으므로 진찰료 및 약국에서의 약제료, 조제료는 모두 전액 본인이 부담하도록 하고 요양급여비용은 청구할 수 없습니다. 
이 경우 처방전양식 중 “기타”란에 “전액 본인부담”으로, “조제시 참고사항”란에 재처방 사유(예시 : 처방약 분실에 따른 재처방)를 표시하여야 합니다.  
\par 행정해석, 보험약제과-1070호 (2008.05.27.)
}