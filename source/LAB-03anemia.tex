\begin{commentbox}{}
\emph{액제형 철분제제(품명 : 헤모큐액 등) 급여기준}\par
허가사항 범위 내에서 아래와 같은 기준으로 투여 시 요양급여를 인정하며, 동 인정기준 이외에는 약값 전액을 환자가 부담토록 함.\\
\begin{center}\emph{-아 래-}\end{center}
\begin{enumerate}[가.]\tightlist
\item 일반적인 철결핍성 빈혈에는 혈액검사결과 다음에 해당되고 타 경구 철분제제 투여 시 위장장애가 있는 경우에 급여하며, 투여기간은 통상 4~6개월 급여함.
	\begin{enumerate}[1)]\tightlist
	\item 일반 환자 혈청페리틴(Serum ferritin) 12ng/㎖ 미만 또는 트란스페린산호포화도(Transferrin saturation rate) 15\% 미만인 경우 
	\item 만성신부전증 환자Serum ferritin 100ng/㎖ 미만 또는 Transferrin saturation rate 20\% 미만인 경우 
	\end{enumerate}
\item 임신으로 인한 철결핍성 빈혈혈액검사결과 Hb 10g/㎗ 이하이고 타 경구 철분제제 투여 시 위장장애가 있는 경우에 급여하며, 투여기간은 4~6개월로 함. 
\item 급성출혈 등으로 인한 산후 빈혈혈액검사결과 Hb 10g/㎗ 이하인 경우에 급여하며, 투여기간은 4주로 함.    
	\begin{itemize}[*]\tightlist
	\item 시행일: 2013.9.1.
	\item 종전고시: 고시 제2005-57호(2005.9.1.)
	\item 변경사유: 용어정비
	\end{itemize}
\item 8세 미만의 소아는 철결핍성 빈혈이 확인된 경우 1차로 투여 시에도 요양급여하며, 미숙아의 경우는 예방 투여 시에도 인정함.
\end{enumerate}
\end{commentbox}

%\subsection{철분주사제(품명: 부루탈주 등)의 급여기준}
\begin{commentbox}{}
\emph{철분주사제(품명: 부루탈주 등)의 급여기준}\par
허가사항 범위 내에서 아래와 같은 기준으로 투여한 경우로서 요양급여비용 청구 시 매월 혈액검사 결과지, 철결핍을 확인할 수 있는 검사결과지, 투여소견서가 첨부된 경우에 요양급여를 인정하며, 동 인정기준 이외에는 약값 전액을 환자가 부담토록 함.\\
\begin{center}\emph{-아 래-}\end{center}
\begin{enumerate}[가)]\tightlist
\item 일반 환자헤모글로빈(Hb) 8g/dl이하이고 경구투여가 곤란한 경우로서 출혈 등이 있어 철분을 반드시 신속하게 투여할 필요성이 있는 철결핍성 빈혈환자로 혈청 페리틴(Serum ferritin) 12ng/㎖ 미만 또는 트란스페린 포화도(Transferrin saturation) 15\%미만인 경우(인페드주의 경우 투여용량은 8㎖ 이내) 
\item 투석중이 아닌 만성신부전증 환자Hb 10g/dl 이하인 경우에 투여하고, 목표(유지) 수치는 Hb 11g/dL까지 요양 급여를 인정하며, Serum ferritin 100ng/㎖ 미만 또는 Transferrin saturation 20\% 미만인 경우(다만, 경구투여가 곤란한 경우만 인정) 
\item 투석중인 만성신부전증 환자Hb 11g/dl 이하인 경우에 투여 시 인정하며,
	\begin{enumerate}[1)]\tightlist
	\item Serum ferritin 100ng/㎖ 미만 또는 Transferrin saturation 20\% 미만인 경우(다만, 복막투석환자는 경구투여가 곤란한 경우만 인정)
	\item 충분한 양의 Erythropoietin주사제를 투여함에도 빈혈이 개선되지 않는 Erythropoietin주사제 저항인 경우에는 Serum ferritin 300ng/㎖ 미만 또는 Transferrin saturation30\%미만인 경우
	\end{enumerate}
\item 항암화학요법을 받고 있는 비골수성 악성종양을 가진 환자 Hb 10g/dl 이하인 경우로서 
	\begin{enumerate}\tightlist
	\item 경구투여가 곤란한 환자로 Serum ferritin 100ng/㎖ 미만 또는 Transferrin saturation 20\%미만인 경우
	\item 충분한 양의 Erythropoietin주사제를 투여함에도 빈혈이 개선되지 않는 Erythropoietin주사제 저항인 경우에는 Serum ferritin 300ng/㎖ 미만 또는 Transferrin saturation 30\%미만인 경우
  	\end{enumerate}
	\begin{itemize}[*]\tightlist
	\item 시행일: 2013.9.1.
	\item 종전고시: 고시 제2013-34호(2013.3.1.)
	\item 변경사유: 용어정비
	\end{itemize}
\end{enumerate}

\begin{shaded}
\emph{심사참고(메모)}
\begin{itemize}\tightlist
\item 경구제, 액상제 인 경우 Ferritin,Transferrin saturation rate(=Fe/TIBC x 100) 기입
\item 액상제인 경우 위장장애 (구역, 구토, 변비, 설사 등의 위장증상, 경구제 복용 불가능이유 기입)
\item 산모인경우, 산후빈혈인경우  Hb 기입(또는 Ferritin,Transferrin saturation rate기입)
\item 주사제인경우 Hb 또는 Ferritin,Transferrin saturation rate 기입
\end{itemize}
\end{shaded}
\end{commentbox}
\leftrod{주사제 청구}
\begin{itemize}\tightlist
\item 베노훼럼(베노스틴,페로빈등) 2앰플+ Nomal saline100ml + 수액제주입로를통한주사(KK054)(13.98점)
\item TIBC=UIBC+serum Fetransferrin saturation rate=Fe/TIBC x 100 \newline
체중 60 kg 기준 Hb:8.0은 이틀간격 2개씩 5번 투여 합니다
\end{itemize}

%\subsection{Erythropoietin 주사제 (품명: 에포카인주 등)}
\begin{commentbox}{}
\emph{Erythropoietin 주사제 (품명: 에포카인주 등)}\par
허가사항 범위 내에서 아래와 같은 기준으로 투여 시 요양급여를 인정하며, 동 인정기준 이외에는 약값 전액을 환자가 부담토록 함.\\
\begin{center}\emph{-아 래-}\end{center}
\begin{enumerate}[가.]\tightlist
\item 만성신부전 환자의 빈혈치료(철결핍성 빈혈로 진단된 환자는 제외함)
	\begin{enumerate}[1)]\tightlist
	\item 헤모글로불린(Hb) 10g/dL 이하 또는 헤마토크리트(Hct) 30\% 이하인 경우(투석을 받고 있지 않는 환자는 사구체여과율(GFR) 30mL/min/1.73㎡ 미만인 환자에 한함)
	\item Hb 11g/dL 또는 Hct 33\%까지 요양급여를 인정하며, 진료비 청구 시 매월별 혈액검사 결과치를 첨부하여야 함.
	\end{enumerate}
\item 항암화학요법(Chemotherapy)을 받고 있는 성인 비골수성 악성종양 환자의 빈혈치료
	\begin{enumerate}[1)]\tightlist
	\item Hb 10g/dL 이하 또는 Hct 30\% 이하인 경우
	\item Hb 12g/dL 또는 Hct 36\%까지 요양급여를 인정하며, 진료비 청구 시 매월별 혈액검사 결과치를 첨부하여야 함.
	\end{enumerate}
\end{enumerate}
허가사항 범위를 초과하여 골수이형성증후군 환자에게 아래와 같은 기준으로 투여 시 요양급여를 인정하며, 동 인정기준 이외에는 약값 전액을 환자가 부담토록 함. \\
-아 래-
\begin{enumerate}[가.]\tightlist
\item 헤모글로불린(Hb) 10g/dL 이하 또는 헤마토크리트(Hct) 30\% 이하이면서 Erythropoietin 500mU/ml 이하인 IPSS risk category low 또는 Intermediate-1 환자
\item Hb 12g/dL 또는Hct 36\%까지 요양급여를 인정하며, 진료비 청구 시 매월별 혈액검사 결과치를 첨부하여야 함. 8주 투여 후 매 2주마다 반응평가를 하여 감량하는 등 용량을 조절해야 함
\item ※ IPSS: NCCN practice guidelines에 의한 International Prognostic Scoring System risk category 분류
\end{enumerate} 

\begin{itemize}[*]
\item 시행일: 2013.9.1.
\item 종전고시: 고시 제2010-80호(2010.10.1.)
\item 변경사유: 용어정비
\end{itemize}
\end{commentbox}

\Que{환자가 빈혈증상이 있어서 검사할때...
\begin{enumerate}\tightlist
\item 빈혈증상 의심될때, CBC시행하고 빈혈소견이 보여 Fe, Ferritin, TIBC,Reticulocyste, PB morpholgy 검사 시행할때....다 급여인가요...? 아니면 이중에 비급여 항목이 있나요...?
\item 빈혈증상 의심될때, 환자가 피 여러번 뽑기 싫다고 한번에 해달라고 할때(즉 CBC 결과가 없을때) CBC, Fe, Ferritin, TIBC,Reticulocyste, PB morpholgy 검사 시행할때....다 비급여인가요...?
\end{enumerate} }\index{빈혈!검사의 보험여부}

\Ans{「국민건강보험법」제41조에 의하면, 요양급여는 가입자와 피부양자의 질병, 부상, 출산 등에 대하여 진찰·검사, 약제·치료재료의 지급, 처치·수술 및 그 밖의 치료 등을 실시하는 것이며 요양급여의 방법·절차·범위·상한 등의 기준은 보건복지부령으로 정한다고 명시되어 있습니다. 또한,「국민건강보험 요양급여의 기준에 관한 규칙」[별표1] 요양급여의 적용기준 및 방법(제5조 제1항 관련) 2조 가. 진찰·검사, 처치·수술 기타의 치료에 의하면, 각종 검사를 포함한 진단 및 치료행위는 진료상 필요하다고 인정되는 경우에 한하여야 하며 연구의 목적으로 하여서는 아니된다고 명시되어 있습니다. 따라서, 고객님께서 말씀하신 내역만으로는 급여·비급여 여부를 확인하기 곤란함을 알려드리니 이점 이해있으시기 바라며, 진료담당의의 의학적 판단에 따라 진료상 필요하여 실시한 검사는 요양급여 할 수 있으며 이에 대한 타당성 여부는 사례별로 심사됨을 알려드립니다.
다만, 현행 건강보험 급여체계는‘행위별 수가’체계로서 모든 의료행위, 치료재료는 급여/비급여로 구분하여 법령 및 복지부장관 고시로 정하고 있으며, 비급여대상을 제외한 일체의 사항을 요양급여의 범위로 규정하고 있는 바, 규정된 비급여대상이 아님에도 상기 급여기준에 해당되지 않는다고 하여 임의로 비급여할 수 없음을 알려드립니다. 
}

\Que{일반환자 경구용 철분제 처방의 보험기준이 어떻게 되나요?(임산부.산후츨혈이 아닌경우)
\begin{enumerate}[1)]\tightlist
\item Hb:10 미만인경우
\item HB :10 미만이고 혈청 페리틴(Serum ferritin) 12ng/㎖ 미만 또는 트란스페린 포화도(Transferrin saturation) 15\%미만인 경우
\item HB :10 미만이거나 혈청 페리틴(Serum ferritin) 12ng/㎖ 미만 또는 트란스페린 포화도(Transferrin saturation) 15\%미만인 경우
\item 혈청 페리틴(Serum ferritin) 12ng/㎖ 미만 또는 트란스페린 포화도(Transferrin saturation) 15\%미만인 경우
\end{enumerate} }
\Ans{ 4 }
