\section{고지혈증}
\myde{}{%
\begin{itemize}\tightlist
\item[\dsjuridical] E785 상세불명의 고지질혈증
\end{itemize}
\tabulinesep =_2mm^2mm
\begin{tabu} to \linewidth {|X[2,l]|X[2,l]|} \tabucline[.5pt]{-}
\rowcolor{ForestGreen!40} \centering 검사코드 & \centering 검사명 \\ \tabucline[.5pt]{-}
\rowcolor{Yellow!40}  D3021 & 당검사(반정량) \\ \tabucline[.5pt]{-}
\rowcolor{Yellow!40}  D3022003 &  당검사(정량) \\ \tabucline[.5pt]{-}
\rowcolor{Yellow!40}  D1890  &  GTP \\ \tabucline[.5pt]{-}
\rowcolor{Yellow!40}  D2800023 &  전해질(소디움) Na \\ \tabucline[.5pt]{-}
\rowcolor{Yellow!40}  D2800063 &  전해질(포타슘) K \\ \tabucline[.5pt]{-}
\rowcolor{Yellow!40}  D2800033 &  전해질(염소) Cl \\ \tabucline[.5pt]{-}
\rowcolor{Yellow!40}  D3050023 &  인슐린 \\ \tabucline[.5pt]{-}
\rowcolor{Yellow!40}  D3061003 &  헤모글로빈 A1C \\ \tabucline[.5pt]{-}
\rowcolor{Yellow!40}  D3002003 &  미량알부민검사(정량) \\ \tabucline[.5pt]{-}
\rowcolor{Yellow!40}  D3050013 & C-peptide \\ \tabucline[.5pt]{-}
\rowcolor{Yellow!40} 고혈압 & + 10종더 가능함.\\ \tabucline[.5pt]{-}
\end{tabu}
}{
}
\begin{commentbox}{고지혈증치료제}
허가사항 범위 내에서 아래와 같은 기준으로 투여시 요양급여를 인정하며, 동 인정기준 이외에 투여한 경우에는 약값 전액을 환자가 부담토록 함.\par
- 아 래 -
\begin{enumerate}[가.]\tightlist
\item 순수 고저밀도지단백콜레스테롤(LDL-C)혈증
	\begin{enumerate}[1)]\tightlist
	\item 투여대상
		\begin{enumerate}\tightlist
		\item 위험요인\footnote{① 흡연, ② 고혈압(BP≥140/90 mmHg 또는 항고혈압제 복용), ③ 낮은 고밀도지단백콜레스테롤(HDL-C)(<40 mg/dL), ④ 관상동맥질환 조기 발병의 가족력(부모, 형제자매 중 남자<55세, 여자<65세에서 관상동맥질환이 발병한 경우), ⑤ 연령(남자≥45세, 여자≥55세), ※ HDL-C≥60 mg/dL은 보호인자로 간주하여 총 위험요인 수에서 하나를 감한다.}
이 0-1개인 경우: 혈중 LDL-C≥160 mg/dL일 때
		\item 위험요인이 2개 이상인 경우: 혈중 LDL-C≥130 mg/dL일 때
		\item 관상동맥질환 또는 이에 준하는 위험(말초동맥질환, 복부대동맥류, 증상이 동반된 경동맥질환, 당뇨병)인 경우: 혈중 LDL-C≥100 mg/dL일 때
		\item 급성 관동맥 증후군인 경우: 혈중 LDL-C≥70 mg/dL일 때
		\end{enumerate}
	\item 해당 약제: HMG-CoA 환원효소억제제, 담즙산제거제, Fibrate계열 약제 중 1종
	\end{enumerate}
\item 순수 고트리글리세라이드(TG)혈증
	\begin{enumerate}[1)]
	\item 투여대상
		\begin{enumerate}\tightlist
		\item 혈중 TG≥500 mg/dL일 때
		\item 위험요인*또는 당뇨병이 있는 경우: 혈중 TG≥200 mg/dL일 때
		\end{enumerate}
	\item 해당 약제: Fibrate 계열, Niacin 계열 중 1종
	\end{enumerate}
\item 고LDL-C 및 고TG혈증 복합형
	\begin{enumerate}[1)]
	\item 투여대상 : “가. 순수 고LDL-C혈증”과 “나. 순수 고TG혈증”에 해당하는 경우
	\item 해당 약제 : LDL-C 및 TG에 작용하는 약제별로 각각 1종씩 인정
	\end{enumerate}
\item 약제투여는 치료적 생활습관 변화(therapeutic lifestyle changes)를 병행하여 실시토록 권장함
\end{enumerate}
\end{commentbox} 
