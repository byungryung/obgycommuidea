\section{공익근무요원인 의사의 요양병원 인력가산 적용 여부}
\Que{공익근무요원으로 복무중인 의사를 요양병원에서 야간전담의사로 지정하려고 하는 경우에 의사인력확보에 따른 입원료 차등제에 대한 인력으로 적용 가능한지요?}
\Ans{「병역법」 제33조 및 「공익근무요원 복무관리규정」 제28조의 규정에서 공익근무요원은 복무기관의 장의 허가를 받아 다른 직무를 겸할 수 있습니다.요양병원의 의사인력확보수준에 따른 입원료 차등제에서 의사 수는 요양병원입원료차등제 산정현황통보서상의 상근자를 의미하므로 당직의사가 상근자인 경우에 포함되며, 의사확보수준에 따라 수가를 차등적용하는 취지 및 상대가치점수 산출 배경등을 감안하여 주5일이상 주40시간이상 근무의사는 1인으로 주3일이상 주20시간이상 근무의사는 0.5인으로 인정합니다.
 보험급여과-4556호 (2009.12.24.)\par
따라서 해당 복무기관의 장으로부터 겸직근무허가를 받은 공익근무요원인 의사가 요양병원의 당직의사로서 주5일 이상이면서 주40시간 이상 근무하거나 주3일 이상이면서 주20시간이상 근무하는 경우 요양병원 의사인력확보수준에 따른 입원료차등제 적용이 가능합니다.
 보험급여과-341호 (2012.02.03.)}
