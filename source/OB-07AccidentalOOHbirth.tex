\section{Accidental out-of-hospital birth}
\begin{paracol}{2}
\setlength{\columnseprule}{0.4pt}
\setlength{\columnsep}{2em}
\begin{leftcolumn}
\begin{commentbox}{}
\begin{itemize}\tightlist
\item[\dsjuridical] R4021 회음절개 및 봉합술 [분만시]
	\begin{itemize}\tightlist
	\item R4023 회음열창봉합술-항문에 달하는 것 자402-1가 
	\item R4024 회음열창봉합술-질원개에 달하는 것 자402-1나 
	\item R4025 회음열창봉합술-직장열창을 동반하는 것 자402-1다 
	\end{itemize}
\item[\dsmedical] R4026 자궁경관열상봉합술 자402-2 
\item[\dsmedical] R4526 태반용수박리술 자452-1 
\item[\dsmedical] R4376 분만후 처치 자437-1 
\item[\dsmedical] R4379 산모용1회용PAD 
\item[\dsmedical] M0141 좌욕(1일당):(1490원)(분만후 익일부터 적용)
\end{itemize}
\end{commentbox}
%\medskip
%\centering

%\includegraphics[width=0.75\linewidth]{labial-fusion}
\end{leftcolumn}

\begin{rightcolumn}
\textsf{분만후 처치란?} \par
\noindent Uterine Massage, Breast Care, Heat Lamp,Dressing 등을 포함\par
\noindent\textsf{산모용1회용PAD}는 입원 기간중 사용한 1회용 Pad는 1일 1통(10개입)이상 사용한 경우에 한하여 16.25점을 산정\par
\noindent 산모용패드 청구착오 : 산모에게 사용한 1회용 패드는 분만, 둔위분만, 제왕절개 기왕력이 있는 질식분만 또는 태반용수박리술 시술에 한하여 산정할 수 있느나 제왕절개만출을 실시한 경우에 산모용 패드가 청구되어 심사조정됨.\par
\noindent\textsf{신생아실이 있는경우}, 요양기관 이외의 장소(요양기관으로 이동중 자동차내 등)에서 분만후 산모와 함께 입원한 신생아의 산정기준은 다른 신생아와 동일합니다.
\end{rightcolumn}
\end{paracol} 
