\section{회음열창봉합술에서 봉합사 산정방법} 
\myde{}{\begin{itemize}\tightlist
\item[\dsjuridical] O702 분만중 제3도 회음열상, K5900 변비
\item[\dsmedical] 처치 R4023 회음열창봉합술 – 항문에 달하는 것
\item[\dsmedical] 재료 B0531009. Vicryl 1
\item[\dsmedical] 기타 항생제및 지사제
\end{itemize}
청구참고사항 : \par
항문에 달하는 회음부열창있고, 항문괄약근의 분리있어서 vicryl 1-0로 
3-4번의 단면을 맞춰서 interrupted suture  해줌. 회음부통증과 변비로
인한 상처감염과 tearing우려로 항생제와 설사제, 진통제 처방함.
}{ 처치 및 수술시 사용된 봉합사는 다음의 경우를 제외하고는 실사용량으로 산정할 수 있으며, "치료재료급여목록및상한금액표" 범위내에서 실 구입가로 산정함.\par
봉합사 제품명(Catalog No.),굵기(Gauge), 사용량 등을 진료기록부(수술기록지) 에 반드시 기재하여야 함. (적용일: 2008.1.1)
}
