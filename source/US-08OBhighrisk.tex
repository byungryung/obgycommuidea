\section{고위험 임신의 범위} %\keystroke{Win}+\keystroke{H}\footnote{\url{https://dl.dropboxusercontent.com/u/6869091/2016obgysono/obgySONO.exe} \par\noindent\url{https://dl.dropboxusercontent.com/u/6869091/2016obgysono/obgy.sqlite}}
\begin{commentbox}{}
\large \textcolor{red}{임신 기간중 딱 3번만 청구} 가능합니다. 또한 선행제왕절개분만의 경우는 high risk에 속하지 않는다고 합니다.\par
\large 고위험 범위중 \textcolor{red}{아기에게 문제 있는경우는 성장지연뿐}입니다. 태아 기형이나 다태아등은 없습니다.
\normalsize\par
%\begin{mdframed}[linecolor=red,middlelinewidth=2]  
\noindent 나951나(1) ‘주’항을 산정할 수 있는 경우는 아래와 같음. \par
\begin{center}\emph{- 아   래 -}\end{center}
\end{commentbox}%mdframed} 
\par
\includegraphics[scale=.75]{UShighrisk}
\par
\medskip
\prezi{\clearpage}
\begin{enumerate}[가)]\tightlist
\item 태아에게 문제를 초래하는 산모의 질환상태(임신성 당뇨병, 임신성 고혈압 등)
	\begin{itemize}\tightlist
	\item O244 임신성 당뇨병 
	\item O140 임신성 고혈압 
	\item O992 Endocrine, nutritional and metabolic diseases complicating pregnancy, childbirth and the puerperium 갑상선기능항진증(E05) 및 저하증(E03)
	\end{itemize}
\item 태아에게 문제를 초래하는 산모 자궁의 이상(여성생식기종양, 자궁경관무력증, 자궁기형 등) 
	\begin{itemize}\tightlist
	\item O34 골반기관의 알려진 또는 의심되는 이상에 대한 산모관리	 
	\item O340	자궁의 선천기형에 대한 산모관리	 
%	\item O34.00	자궁의 선천기형에 대한 산모관리, 임신 22주 미만	 
%	\item O34.01	자궁의 선천기형에 대한 산모관리, 임신 22주 이상, 34주 미만	 
%	\item O34.02	자궁의 선천기형에 대한 산모관리, 임신 34주 이상	 
	\item O3409	자궁의 선천기형에 대한 산모관리, 상세불명의 임신기간	 
	\item O341	자궁체부종양에 대한 산모관리
%	\item O3419 자궁근종에 대한 산모관리, 상세불명의 임신기간	
%	\item O34.10	자궁체부종양에 대한 산모관리, 임신 22주 미만	 
%	\item O34.11	자궁체부종양에 대한 산모관리, 임신 22주 이상, 34주 미만	 
%	\item O34.12	자궁체부종양에 대한 산모관리, 임신 34주 이상	 
%	\item O3419	자궁체부종양에 대한 산모관리, 상세불명의 임신기간	 
%	\item O34.2	이전의 외과수술로 인한 자궁흉터에 대한 산모관리	 
%	\item O34.21	이전의 자궁하부횡절개로 인한 자궁흉터에 대한 산모관리	 
%	\item O34.22	이전의 기타 및 상세불명의 제왕절개로 인한 자궁흉터에 대한 산모관리	 
%	\item O34.28	이전의 기타 외과수술로 인한 자궁흉터에 대한 산모관리	 
	\item O34.3	자궁경관부전에 대한 산모관리	 
%	\item O34.30	자궁경관부전에 대한 산모관리, 임신 22주 미만	 
%	\item O34.31	자궁경관부전에 대한 산모관리, 임신 22주 이상, 34주 미만	 
%	\item O34.32	자궁경관부전에 대한 산모관리, 임신 34주 이상	 
%	\item O3439	자궁경관부전에 대한 산모관리, 상세불명의 임신기간	 
	\item O34.4	자궁경부의 기타 이상에 대한 산모관리	 
%	\item O34.40	자궁경부의 기타 이상에 대한 산모관리, 임신 22주 미만	 
%	\item O34.41	자궁경부의 기타 이상에 대한 산모관리, 임신 22주 이상, 34주 미만	 
%	\item O34.42	자궁경부의 기타 이상에 대한 산모관리, 임신 34주 이상	 
%	\item O34.49	자궁경부의 기타 이상에 대한 산모관리, 상세불명의 임신기간	 
	\item O34.5	임신자궁의 기타 이상에 대한 산모관리	 
%	\item O34.50	임신자궁의 기타 이상에 대한 산모관리, 임신 22주 미만	 
%	\item O34.51	임신자궁의 기타 이상에 대한 산모관리, 임신 22주 이상, 34주 미만	 
%	\item O34.52	임신자궁의 기타 이상에 대한 산모관리, 임신 34주 이상	 
%	\item O34.59	임신자궁의 기타 이상에 대한 산모관리, 상세불명의 임신기간	 
%	\item O34.6	질의 이상에 대한 산모관리	 
%	\item O34.60	질의 이상에 대한 산모관리, 임신 22주 미만	 
%	\item O34.61	질의 이상에 대한 산모관리, 임신 22주 이상, 34주 미만	 
%	\item O34.62	질의 이상에 대한 산모관리, 임신 34주 이상	 
%	\item O34.69	질의 이상에 대한 산모관리, 상세불명의 임신기간	 
%	\item O34.7	외음 및 회음의 이상에 대한 산모관리	 
%	\item O34.70	외음 및 회음의 이상에 대한 산모관리, 임신 22주 미만	 
%	\item O34.71	외음 및 회음의 이상에 대한 산모관리, 임신 22주 이상, 34주 미만	 
%	\item O34.72	외음 및 회음의 이상에 대한 산모관리, 임신 34주 이상	 
%	\item O34.79	외음 및 회음의 이상에 대한 산모관리, 상세불명의 임신기간	 
%	\item O34.8	골반기관의 기타 이상에 대한 산모관리	 
%	\item O34.80	골반기관의 기타 이상에 대한 산모관리, 임신 22주 미만	 
%%	\item O34.82	골반기관의 기타 이상에 대한 산모관리, 임신 34주 이상	 
%	\item O34.89	골반기관의 기타 이상에 대한 산모관리, 상세불명의 임신기간	 
%	\item O34.9	골반기관의 상세불명 이상에 대한 산모관리	 
%%%	\item O34.92	골반기관의 상세불명 이상에 대한 산모관리, 임신 34주 이상	 
%	\item O34.99	골반기관의 상세불명 이상에 대한 산모관리, 상세불명의 임신기간	 
	\end{itemize}
\item 정상 분만이 불가능한 태반의 이상(전치태반, 태반조기박리 등)
	\begin{itemize}\tightlist
	\item O440 전치태반 
	\item O459 태반조기박리 
	\end{itemize}
\item 양수과다증 또는 양수과소증
	\begin{itemize}\tightlist
	\item O40 양수과다증 Polyhydramnios 
	\item O410 양수과소증 Oligohydramnios 
	\end{itemize}
\item 자궁내 태아 성장지연
	\begin{itemize}\tightlist
	\item O365 태아성장불량에 대한 산모관리 Maternal care for poor fetal growth
	\end{itemize}
\end{enumerate}
\prezi{\clearpage}
\subsection{산부인과학회 세부급여기준 Q\&A}
\prezi{\clearpage}
1. 태아에게 문제를 초래하는 산모의 질환상태 중 임신성 당뇨병과 임신성 고혈압에는 어떤 것이 포함 되나요?
\begin{quotebox}
임신성 당뇨병에는 임신 전 진단된 당뇨병(pregestational diabetes), 임신 중 진단 또는 발견된 당뇨병 (gestational diabetes)을 모두 포함합니다.
임신성 고혈압에는 만성고혈압(chronic hypertension), 임신성 고혈압(gestational hypertension), 자간전증 (preeclampsia), 자간증(eclampsia), 가중합병자간전증(superimposed preeclampsia)이 모두 포함됩니다.
\end{quotebox}
\prezi{\clearpage}
2. 양수과다증 및 양수과소증의 정의는 어느 기준으로 하나요?
\begin{quotebox}
양수과다증은 양수지수(AFI)가 24 cm 이상인 경우 또는 단일최대깊이(single deepest pocket)가 8 cm 이상인 경우로, 양수과소증은 양수지수(AFI)가 5 cm 이하인 경우 또는 단일최대깊이(single deepest pocket)
2 cm 이하인 경우로 정의합니다.
\end{quotebox}
\prezi{\clearpage}
3. 자궁경관무력증에서 cerclage 를 하지 않고, 추적관찰만 하는 경우에도 고위험 가산에 포함 되나요?
\begin{quotebox}
Cerclage 여부와 상관 없이 임상적으로 자궁경관무력증 진단에 부합하면 가능합니다.
\end{quotebox}
\prezi{\clearpage}
4. 전치태반 중 low lying placenta는 어떻게 정의하나요?
\begin{quotebox}
Cervical internal os에서 placenta tip까지의 거리가 2 cm 미만인 경우로 정의합니다.
\end{quotebox}
\prezi{\clearpage}
5. 임신 20주에 전치태반 소견이 보이는 경우에도 고위험 가산에 포함되나요?
\begin{quotebox}
초음파 상 전치태반 소견이 명확한 경우에는 고위험 가산에 포함이 됩니다.
\end{quotebox}
\prezi{\clearpage}
6. Single umbilical artery도 고위험으로 들어가나요?
\begin{quotebox}
Isolated single umbilical artery는 고위험에 해당하지 않습니다.
\end{quotebox}
\prezi{\clearpage}
7. 기왕 제왕절개수술이나 기왕 자궁근종수술도 고위험인가요?
\begin{quotebox}
고위험 가산에 포함되지 않습니다.
\end{quotebox}