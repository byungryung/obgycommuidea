\section{Reactive Cellular Change RCC}
\myde{}{%
\begin{itemize}\tightlist
\item[\dsjuridical] N86 자궁목의 미란 및 외반증, 
\item[\dsjuridical] N72 자궁목의 염증성 질환등
\item[\dschemical] \sout{C6014006} 하부요로생식기 및 성매개 감염원인균(다중종합효소연쇄반응법) STD 6종, STD 12종
\item[\dschemical] \sout{C5896006} 하부요로생식기 및 성매개 감염원인균(다중실시간 종합효소연쇄반응법) STD 6종(RT-PCR), STD 7종(RT-PCR)
\item[\dschemical] D6802026 STD12종multiplex 76,640원
\item[\dschemical] D6802016 STD7종Real-Time 76,640원
\item[\dschemical] 적극적인 Cx cancer screening : Cxgram, HPV test, Endocervical sonogram
\item[\dsmedical] R4300 자궁경부(질)약물소작술
\item[\dsmedical] R4310 자궁경부(질)전기소작술
\item[\dsmedical] R4320 자궁경부(질)냉동 또는 열응고술
\item[\dsmedical] 암보칠 비보험 청구.
\item[\dsmedical] R4106 질강처치 [\myexplfn{58.04} 원]
\end{itemize}
청구메모>> 1.클라미디어 감염 가능성 있어 검사 시행. 2. 질분비물이 많고 악취가 심해 부인과적 감염이 의심되어 검사시행등 상황에 맞게 작성.
}{
The Scientific World Journal\par
Volume 2014 (2014), Article ID 756713, 5 pages\par
\url{http://dx.doi.org/10.1155/2014/756713}
Research Article\par
Cervical Cytopathological Findings in Korean Women with Chlamydia trachomatis, Mycoplasma hominis, and Ureaplasma urealyticum Infections\par
Yuri Choi and Jaesook Roh
Laboratory of Reproductive Endocrinology, Department of Anatomy \& Cell Biology, College of Medicine, Hanyang University, San 17 Haengdang-dong, Seongdong-gu, Seoul 133-792, Republic of Korea\par

Received 21 August 2013; Accepted 22 October 2013; Published 8 January 2014\par

Academic Editors: R. Medeiros and K. Savik\par

%Copyright © 2014 Yuri Choi and Jaesook Roh. This is an open access article distributed under the Creative Commons Attribution License, which permits unrestricted use, distribution, and reproduction in any medium, provided the original work is properly cited.\par

\emph{Abstract}\par

This is to investigate the cervical cytological abnormalities associated with Chlamydia trachomatis, Mycoplasma hominis, Mycoplasma genitalium, and Ureaplasma urealyticum infections on routine screen. A total of 714 subjects who had undergone cervical Pap smears and concomitant analyses for cervical infections were included by a retrospective search. The frequencies of reactive cellular change (RCC) and squamous epithelial abnormalities were significantly higher in Chlamydia positive subjects than in uninfected subjects . Of the 124 subjects tested for M. hominis, M. genitalium, and U. urealyticum, 14 (11\%) were positive for M. hominis and 29 (23\%) were positive for U. urealyticum. Squamous abnormalities were more frequent in subjects with Ureaplasma infections than in uninfected subjects (24\% versus 8\%). Taking together these findings, C. trachomatis and U. urealyticum may have a causal role in the development of cervical epithelial changes, including RCC. Thus, extra awareness is warranted in cervical screening of women with Chlamydia or Ureaplasma infections.
}

%\hspace{-1cm}
\includegraphics{RCC}

자궁경부(질)약물 소작술은 환자가 병원에 올 때마다 해도 되나요?
\begin{quotebox}
YES
\end{quotebox}

보비를 쓰는 경우는 R4310인가요? R4320인가요?
\begin{quotebox}
R4310
\end{quotebox} 
고주파 기계를 사용하는 경우는 R4310인가요? R4320인가요?
\begin{quotebox}
R4320
\end{quotebox}
레이저 사용 후 비 급여로 받아도 되나요?
\begin{quotebox}
NO
\end{quotebox}
급성질염에 자궁경부(질)약물소작술 인정여부
\begin{quotebox}
자430 자궁경부(질)약물소작술은 자궁경부에 Eversion(외번)이 있는경우는 인정하나, 급성질여염에 실시하는 경우에는 인정하지 아니 한다.
\end{quotebox}
