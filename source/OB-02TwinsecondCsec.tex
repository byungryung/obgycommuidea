\section{쌍둥이 첫애 분만후 둘째 제왕절개시 숫가코드}
\myde{}{
\begin{itemize}\tightlist
\item[\dsjuridical] O325 하나 이상의 태아의 태위장애를 동반한 다태임신의 산모관리
\item[\dsjuridical] O629 분만힘의 상세불명 이상
\item[\dsjuridical] O300 쌍둥이임신
\item[\dsjuridical] Z372 쌍둥이, 둘 다 생존 출생
\item[\dsjuridical] O321 Bx 산모관리 
\item[\dsmedical] R4351 정상분만 Normal 초산 Primiparous 제1태아 First Fetus \myexplfn{3942.97} 원
\item[\dsmedical] R4519 다태아임신의 경우 Multiple Pregnancy (1) 초회 Initial (가) 초산 Primiparous \myexplfn{4680.86}
\end{itemize}
}
{
쌍둥이 분만시 첫째는 자연분만, 둘째는 제왕절개로 했을 경우 포괄수가제 적용되는지 궁금합니다.
쌍둥이였고 첫째는 자연분만으로 낳았고, 둘째도 자연분만 시도하는 과정중에 자세가 바뀌어 불가피하게 제왕절개로 낳았습니다.이와 같은 경우에도 포괄수가제가 적용되나요?\par
\Ans{자연분만과 제왕절개분만으로 분만방법을 달리하여 쌍둥이를 출산한 경우 전체 진료내역은 \highlight{행위별수가제가 적용}되오니 참고하시기 바랍니다}
}